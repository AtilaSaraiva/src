%opening
\title{Introduction to stability condition and dispersion analysis for wave equation}

\renewcommand{\thefootnote}{\fnsymbol{footnote}}
\author{Pengliang Yang\footnotemark[1]}

\lefthead{P.L. Yang}
\righthead{Stability and dispersion}

\address{\footnotemark[1]Xi'an Jiaotong University\\
National Engineering Laboratory for Offshore Oil Exploration\\
Xi'an, China, 710049}

\maketitle

\begin{abstract}
This tutorial provides a simple yet useful guide to do stability and dispersion analysis for wave equation.
\end{abstract}

\section{1D Warm up\citep{johnsonpset18336}}
In numerical analysis, von Neumann stability analysis is a procedure used to check the stability of finite difference schemes as applied to linear partial differential equations. The analysis is based on the Fourier decomposition of numerical error and was developed at Los Alamos National Laboratory after having been briefly described in a 1947 article by British researchers Crank and Nicolson. This method is an example of explicit time integration where the function that defines governing equation is evaluated at the current time. Later, the method was given a more rigorous treatment in an article co-authored by John von Neumann.

\subsection{Stability example: Crank-Nicolson scheme}
Question: When Crank-Nicolson scheme for $u_t+a u_x=0$ is unconditionally stable for the discretized solution $u(m\Delta x, n\Delta t):= u_m^n$
\begin{equation}
u_m^{n+1}=u_m^n-a\Delta t \left[
\alpha \frac{u_{m+1}^{n+1}-u_{m-1}^{n+1}}{2\Delta x}+ (1-\alpha)\frac{u_{m+1}^{n}-u_{m-1}^{n}}{2\Delta x}
\right]?
\end{equation}

Let $u_m^n=g^n e^{im\theta}$ and solve for the amplification factor $g(\theta,\Delta x, \Delta t)$. Then,
\begin{equation}
 g=1-a\Delta t \left[
 \alpha g \frac{i\sin\theta}{\Delta x}+(1-\alpha)\frac{i\sin\theta}{\Delta x}
 \right]
\end{equation}
Thus,
\begin{equation}
 g=\frac{1-ia\lambda(1-\alpha)\sin\theta}{1+ia\lambda\alpha\sin\theta}
\end{equation}
where $\lambda=\Delta t/\Delta x$. To guarantee the stability, it requires $|g|\leq 1$. That is,
\begin{equation}
 1+a^2\lambda^2(1-\alpha^2)\sin^2\theta \leq 1+a^2 \lambda^2\alpha^2\sin^2\theta
\end{equation}
leading to
\begin{equation}
 \alpha\geq 0.5.
\end{equation}


\subsection{Staggered-grid Leap-frog}
Consider 2-component wave equation:
\begin{gather}
u_t=b v_x-\sigma u\\
v_t=c u_x-\sigma v 
\end{gather}
including PML dissipation coefficient $\sigma$, where $b>0$, $c>0$, and $\sigma\geq 0$. The staggerred-grid leap-frog scheme is
\begin{equation}\label{eq:pml1}
 \frac{u_m^{n+1}-u_m^n}{\Delta t}=b\frac{v_{m+1/2}^{n+1/2}-v_{m-1/2}^{n+1/2}}{\Delta x}-\sigma \frac{u_m^{n+1}+u_m^n}{2}
\end{equation}
\begin{equation}\label{eq:pml2}
 \frac{v_{m+1/2}^{n+3/2}-v_{m+1/2}^{n+1/2}}{\Delta t}=c \frac{u_{m+1}^{n+1}-u_m^{n+1}}{\Delta x}-\sigma \frac{v_{m+1/2}^{n+3/2}+v_{m+1/2}^{n+1/2}}{2}
\end{equation}

\subsubsection{$\sigma=0$, PML excluded:}

Define $\lambda=\Delta t/\Delta x$ and Fourier transform $\hat{u}^n=F{u^n}$. The above 2 equations are
\begin{equation}
 \hat{u}^{n+1}=\hat{u}^n+b\lambda(2i\sin\frac{\theta}{2})\hat{v}^{n+1/2}
\end{equation}
\begin{equation}
 \hat{v}^{n+3/2}=\hat{v}^{n+1/2}+c\lambda(2i\sin\frac{\theta}{2})\hat{u}^{n+1}
\end{equation}

This can be expressed as a produce of $2\times2$ matrices:
\begin{gather}
\begin{bmatrix}
        \hat{u}^{n+1}\\
        \hat{v}^{n+3/2}\\
\end{bmatrix}
=\begin{bmatrix}
  1 & 0\\
  2ic\lambda\sin\frac{\theta}{2} & 1\\
 \end{bmatrix}
 \begin{bmatrix}
  \hat{u}^{n+1}\\
  \hat{v}^{n+1/2}
 \end{bmatrix}
=\begin{bmatrix}
  1 & 0\\
  2ic\lambda\sin\frac{\theta}{2} & 1\\
 \end{bmatrix}
 \begin{bmatrix}
  1 &2ib\lambda\sin\frac{\theta}{2}\\
  0 & 1\\
 \end{bmatrix}
 \begin{bmatrix}
  \hat{u}^n\\
  \hat{v}^{n+1/2}
 \end{bmatrix}
\end{gather}
The amplification matrix $G$ is
\begin{equation}
\begin{bmatrix}
 1 & 2ib\lambda\sin\frac{\theta}{2}\\
 2ic\lambda\sin\frac{\theta}{2} &1-4bc\lambda^2\sin^2\frac{\theta}{2}
\end{bmatrix}
\end{equation}
with eigenvalues
\begin{equation}
 g_{\pm}=1-2bc\lambda^2\sin^2\frac{\theta}{2}\pm\sqrt{\left(1-2bc\lambda^2\sin^2\frac{\theta}{2} \right)^2-1}
\end{equation}
If $\sqrt{\cdots}$ is imaginary, then $|g_{\pm}|=1$ and it is stable. It leads to $bc\lambda^2\leq 1$.

\subsubsection{$\sigma\neq0$, PML incorporated:}
Equations \eqref{eq:pml1} and \eqref{eq:pml2} result in
\begin{equation}
 \hat{u}^{n+1}=\frac{1-\frac{\sigma\Delta t}{2}}{1+\frac{\sigma\Delta t}{2}}\hat{u}^n+\frac{b\lambda(2i\sin\frac{\theta}{2})}{1+\frac{\sigma\Delta t}{2}}\hat{v}^{n+1/2}
\end{equation}
\begin{equation}
 \hat{v}^{n+3/2}=\frac{1-\frac{\sigma\Delta t}{2}}{1+\frac{\sigma\Delta t}{2}}\hat{v}^{n+1/2}+
 \frac{c\lambda(2i\sin\frac{\theta}{2})}{1+\frac{\sigma\Delta t}{2}}\hat{u}^{n+1}
\end{equation}
Following the similar manipulation, the amplification matrix $G$ is
\begin{equation}
 \frac{1}{\sigma_{+}^2}\begin{bmatrix}
                        \sigma_{+} & 0\\
                        2ic\lambda \sin\frac{\theta}{2} & \sigma_{-}
                       \end{bmatrix}
                       \begin{bmatrix}
                        \sigma_{-} & 2ib\lambda \sin\frac{\theta}{2}\\
                        0 & \sigma_{+}
                       \end{bmatrix}
=\frac{1}{\sigma_{+}^2}\begin{bmatrix}
                        \sigma_{+}\sigma_{-} & 2ib\lambda \sigma_{+}\sin\frac{\theta}{2}\\
                        2ic\lambda \sigma_{-}\sin\frac{\theta}{2} & \sigma_{+}\sigma_{-}-4bc\lambda^2\sin^2\frac{\theta}{2}  
			\end{bmatrix}                       
\end{equation}
The eigenvalues can be found as
\begin{equation}
 g_{\pm}=\frac{1}{\sigma_{+}^2}\left(
 d\pm \sqrt{ d^2-\sigma_{+}^2\sigma_{-}^2}
 \right)
\end{equation}
where $d=\sigma_{+}\sigma_{-}-bc\lambda^2\sin^2\frac{\theta}{2}$.
 As long as $\sigma>0$,  $bc\lambda^2\leq 1$, it is easy to verify that $|d|^2<\sigma_{+}^2\sigma_{-}^2$ and $\sqrt{\cdots}$ is imaginary, then $|g_{\pm}|=1$ and it is stable. It implies that PML doesn't cause the system unstable, at least from von Neumann analysis.

\subsection{Phase velocity vs. group velocity}

Assume $\sigma=0$ and consider individual Fourier components $u=e^{-i(\phi n-\theta m)}$, $v=Ae^{-i(\phi n-\theta m)}$ in which $\phi=\omega\Delta t$, $\theta=k\Delta x$, $A$ is a scaling factor. Plugging $u$ and $v$ into the update equations leads to
\begin{equation}
 e^{i\phi}\begin{bmatrix}
           1\\
           Ae^{i\phi/2}
          \end{bmatrix}
=G\begin{bmatrix}
  1\\
  Ae^{i\phi/2}
 \end{bmatrix}
\end{equation}
Clearly, the eigenvalue satisfies
\begin{equation} 
e^{i\phi}=\cos\phi+i\sin\phi=1-2bc\lambda^2\sin^2\frac{\theta}{2}\pm i\sqrt{1-\left(1-2bc\lambda^2\sin^2\frac{\theta}{2} \right)^2}
\end{equation}
Therefore,
\begin{equation}
 \phi(\theta)=\pm 2\sin^{-1}\left(\sqrt{bc}\lambda\sin\frac{\theta}{2} \right)
\end{equation}
\begin{equation}
 \frac{d\phi}{d\theta}=\pm \frac{\sqrt{bc}\lambda\cos\frac{\theta}{2}}{\sqrt{1-bc\lambda^2\sin^2\frac{\theta}{2}}}.
\end{equation}


 Note that phase velocity and the group velocity are defined as
\begin{equation}
 v_{phase}=\frac{\omega}{k}=\frac{\phi/\Delta t}{\theta/\Delta x}=(\phi/\theta)/\lambda
\end{equation}
and
\begin{equation}
 v_{group}=\frac{d \omega}{d k}=(d\phi/d\theta)/\lambda
\end{equation}
respectively. Thus, for $\theta\rightarrow 0$, both $v_{phase}$ and $v_{group}$ are approximating the exact velocity. For $\theta\rightarrow\phi$, $v_{group}\rightarrow 0$.



\section{2D Acoustic wave equation}

\subsection{Notation}
With the notation $\frac{\partial^2 u}{\partial s^2}=u_{ss}, s=x,y,t$, in the case of constant density the basic acoustic wave equation is given by
\begin{equation}
 u_{tt}=v^2(u_{xx}+u_{yy}).
\end{equation}
Using the Fourier transform
\begin{equation}
u_t\leftrightarrow i\omega \hat{u}, u_x\leftrightarrow i k_x \hat{u}, u_y\leftrightarrow i k_y \hat{u};
u_{tt}\leftrightarrow -\omega^2 \hat{u},
u_{xx}\leftrightarrow -k_x^2 \hat{u},
u_{yy}\leftrightarrow-k_y^2 \hat{u},
\end{equation}
the relationship is then obtained
\begin{equation}
 \omega^2=v^2 (k_x^2+k_y^2)
\end{equation}
Let $k_x=k\cos\theta$, $k_y=k\sin\theta$. It yields  $k=\omega/v$.

The discretized wavefield with finite difference (FD) method $u$ is denoted as
\begin{equation}
 u(x,y,t)=u(j\Delta x, m\Delta y, n\Delta t):=u_{j,m}^n.
\end{equation}
Therefore, the plane wave expression
\begin{equation}
u(x,y,t)=e^{-i(\omega t- k_x x- k_y y)}=e^{-i(\omega t -k\cos\theta\cdot x-k\sin\theta\cdot y)}
\end{equation}
leads to
\begin{equation}
 u_{j+p,m+q}^{n+l}=e^{-i(\omega t -k\cos\theta\cdot x-k\sin\theta\cdot y)} \cdot
 e^{-i(\omega \cdot l\Delta t -k\cos\theta \cdot p\Delta x-k\sin\theta \cdot q\Delta y)}
\end{equation}

\subsection{Dispersion analysis with regular grid FD \citep{liu2009new}}
With 2nd order FD in time and $2N$-th order FD in space, the wave equation becomes
\begin{equation}\label{eq:disp1}
\frac{u_{j,m}^{n+1}-2u_{j,m}^n+u_{j,m}^{n-1}}{\Delta t^2}
=v^2\left(\frac{1}{\Delta x^2}\sum_{l=-N}^{N}C_l u_{j+l,m}^n+
 \frac{1}{\Delta y^2}\sum_{l=-N}^{N}C_l u_{j,m+l}^n\right)
\end{equation}
According to the plane wave expression, equation \eqref{eq:disp1} is
\begin{equation}
\begin{split}
&e^{-i(\omega t- k\cos\theta \cdot x- k\sin\theta\cdot y)}\frac{e^{i\omega\Delta t}-2+e^{-i\omega\Delta t}}{\Delta t^2}\\
=&v^2 e^{-i(\omega t- k\cos\theta\cdot x- k\sin\theta\cdot y)}\left(
\frac{1}{\Delta x^2}\sum_{l=-N}^{N}C_l e^{i k\cos\theta \cdot l\Delta x}
+\frac{1}{\Delta y^2}\sum_{l=-N}^{N}C_l e^{i k\sin\theta \cdot l\Delta y}
\right)
\end{split}
\end{equation}
Note that $C_l=C_{-l}$, $k=\omega/v$, it follows that
\begin{equation}\label{eq:disp2}
\begin{split}
 &2\cos( v k\Delta t)-2\\
 =&v^2\Delta t^2 (
\frac{1}{\Delta x^2}(C_0+\sum_{l=1}^{N}C_l (e^{i k\cos\theta \cdot l\Delta x}+e^{-i k\cos\theta \cdot l\Delta x}))\\
&+\frac{1}{\Delta y^2}(C_0+\sum_{l=1}^{N}C_l (e^{i k\cos\theta \cdot l\Delta y}+e^{-i k\cos\theta \cdot l\Delta y}))
)\\
 =&v^2\Delta t^2 (
\frac{1}{\Delta x^2}(C_0+2\sum_{l=1}^{N}C_l \cos(k\cos\theta \cdot l\Delta x))\\
&+\frac{1}{\Delta y^2}(C_0+2\sum_{l=1}^{N}C_l \cos(k\cos\theta \cdot l\Delta y))
)\\
\end{split}
\end{equation}

Assume $\Delta x=\Delta y=h$, $r=\frac{v\Delta t}{h}$. Equation \eqref{eq:disp2} becomes
\begin{equation}\label{eq:disp3}
 r^{-2}( \cos (r kh)-1)
=\left(C_0+\sum_{l=1}^{N}C_l (\cos(klh\cos\theta)+\cos(klh\sin\theta))\right)
\end{equation}
Using the Taylor series expansion for cosine functions, we have
\begin{equation}
\begin{split}
 &2r^{-2}\sum_{j=1}^{\infty}\frac{(rkh)^{2j}}{(2j)!}\\
 =&C_0+\sum_{l=1}^{N}C_l \left(2+\sum_{j=1}^{\infty}\frac{(klh\cos\theta)^{2j}+(klh\sin\theta)^{2j}}{(2j)!}\right)\\
 =&\left(C_0+2\sum_{l=1}^{N}C_l\right)+
 \sum_{j=1}^{\infty} \left(\sum_{l=1}^{N}C_l\frac{(klh)^{2j}}{(2j)!}(\cos^{2j}\theta+\sin^{2j}\theta)\right)\\
\end{split}
\end{equation}
It implies that:
\begin{equation}\label{eq:c0}
 C_0+2\sum_{l=1}^{N}C_l=0
\end{equation}
\begin{equation}
 \sum_{l=1}^{N}l^{2j}C_l(\cos^{2j}\theta+\sin^{2j}\theta)=r^{2j-2}, j=1, \ldots, N.
\end{equation}
It forms a Vandermonde-like system.

Based on $C_l=C_{-l}$ as well as equations \eqref{eq:disp3} and \eqref{eq:c0}, we have:
\begin{equation}
 r^{-2}\sin^2\frac{rkh}{2}=\sum_{l=1}^N C_l \left(
 \sin^2\frac{rkh\cos\theta}{2}+\sin^2\frac{rkh\sin\theta}{2} \right)
\end{equation}
The dispersion (ratio of FD velocity and true velocity) is then defined by
\begin{equation}
 \delta=\frac{v_{FD}}{v}=\frac{2}{rkh}\sin^{-1}\sqrt{\sum_{l=1}^N C_l \left(
 \sin^2\frac{rkh\cos\theta}{2}+\sin^2\frac{rkh\sin\theta}{2} \right)}
\end{equation}
$kh$ ranges from 0 to $\pi$ (Nyquist frequency). If $\delta$=1, there is no dispersion. If $\delta$ is far from 1, a large dispersion will occur. 


\subsection{Stability analysis}
Allowing for the amplification factor, the wavefield is expressed as
\begin{equation}\label{eq:stab1}
 u_{j,m}^{n}=g^n e^{-i(n\phi-j\theta_1-m\theta_2)}
\end{equation}
where $\phi=\omega\Delta t$, $\theta_1=k_x\Delta x$, $\theta_2=k_y\Delta y$.
Substituting equation \eqref{eq:stab1} into equation \eqref{eq:disp1} results in
\begin{equation}
 g-2+g^{-1}=v^2\Delta t^2\left(
 \frac{1}{\Delta x^2}\sum_{l=-N}^N C_l e^{i l \theta_1}+
 \frac{1}{\Delta y^2}\sum_{l=-N}^N C_l e^{i l \theta_2}
 \right)
\end{equation}
Define
\begin{equation}
 c=v^2\Delta t^2\left(
 \frac{1}{\Delta x^2}\sum_{l=-N}^N C_l e^{i l \theta_1}+
 \frac{1}{\Delta y^2}\sum_{l=-N}^N C_l e^{i l \theta_2}
 \right)
\end{equation}
That is,
\begin{equation}
 g^2-\left(2+c\right)g+1=0\Rightarrow
 g_{\pm}=\frac{c+2\pm\sqrt{(c+2)^2-4}}{2}
\end{equation}
\begin{equation}
|g_{\pm}|\leq 1\Leftrightarrow
|c+2\pm\sqrt{(c+2)^2-4}|\leq 2
\end{equation}
If $(c+2)^2-4\geq 0$, we have:
\begin{equation}
 0\leq\sqrt{(c+2)^2-4}\leq \mp(c+2)
\end{equation}
That leads to the following contraditions:
\begin{equation}
 \begin{cases}
  c+2=0\\
  (c+2)^2-4=0
 \end{cases}
\end{equation}
Therefore, the stability requires
\begin{equation}
 (c+2)^2-4\leq 0 \Leftrightarrow -4\leq c\leq 0
\end{equation}
Note that  $C_l=C_{-l}$,  $C_0=-2\sum_{l=1}^N C_l$,
\begin{equation}
 c=-2r^2\sum_{l=1}^N C_l(2-\cos\theta_1-\cos\theta_2)
 =-4r^2\sum_{l=1}^N C_l(\sin^2\frac{\theta_1}{2}+\sin^2\frac{\theta_2}{2})\geq -4
\end{equation}
The stability condition is
\begin{equation}
 r\leq \left(2\sum_{l=1}^N |C_l|
 \right)^{-1}
\end{equation}


\bibliographystyle{plainnat}
\bibliography{dispstabbib}
