\chapter*{Themes}

The main theme of this book is to take a good quality reflection
seismic data set from the Gulf of Mexico and guide you through the
basic geophysical data processing steps from raw data to the
best-quality final image. Secondary themes are to introduce you (1) to
cleaned up but real working % Fortran 
code that does the job, (2) to the
concept of “adjoint operator”, and (3) to the notion of electronic
document.

\subsection*{What it does, what it means, and how it works}
A central theme of this book is to merge the abstract with the
concrete by linking mathematics to runnable computer codes. The codes
are in a consistent style using nomenclature that resembles the
accompanying mathematics so the two illuminate each other. The code
shown is exactly that used to generate the illustrations. There is
little or no mathematics or code that is not carried through with
examples using both synthetic and real data.

%The code itself is in
%a dialect of Fortran more suitable for exposition than standard Fortran. (This "ratfor" dialect
%easily translates to standard Fortran). 

Some codes have been heavily tested while others have
only been tested by the preparation of the illustrations.

\subsection*{Imaging with adjoint (conjugate-transpose) operators}

A secondary theme of this book is to develop in the reader an
understanding of a universal linkage beween forward modeling and data
processing. Thus the codes here that incarnate linear operators are
written in a style that also incarnates the adjoint
(conjugate-transpose) operator thus enabling both modeling and data
processing with the same code. This style of coding, besides being
concise and avoiding redundancy, ensures the consistency required for
estimation by conjugate-gradient optimization as described in my other
books.  Adjoint operators link the modeling activity to the model
estimation activity. While this linkage is less sophisticated than
formal estimation theory (“inversion”), it is robust, easily
available, and does not put unrealistic demands on the data or
imponderable demands on the interpreter.


\subsection*{Electronic document}

A goal that we met with the 1992 CD-ROM version of this book was to
give the user a full copy, not only of the book, but of all the
software that built the book including not only the seismic data
processing codes but also the word processing, the data, and the whole
superstructure.  Although we succeeded for a while having a book that
ran on machines of all the major manufacturers, eventually we were
beaten down by a host of incompatibilities. This struggle
continues. With my colleagues, we are now working towards having books
on the World Wide Web where you can grab parts of a book that
generates illustrations and modify them to create your own
illustrations.

\subsection*{Acknowledgements}

I had the good fortune to be able to establish a summer 1992
collaboration with Jim Black of IBM in Dallas who, besides bringing
fresh eyes to the whole undertaking, wrote the first version of
chapter 8 on dip moveout, made significant contributions to the other
chapters, and organized the raw data.  In this book, as in my previous
(and later) books, I owe a great deal to the many students at the
Stanford Exploration Project. The local computing environment from my
previous book is still a benefit, and for this I thank Stew Levin,
Dave Hale, and Richard Ottolini. In preparing this book I am specially
indebted to Joe Dellinger for his development of the intermediate
graphics language vplot that I used for all the figures. I am grateful
to Kamal Al-Yahya for converting my thinking from the troff
typesetting language to LATEX. Bill Harlan offered helpful
suggestions. Steve Cole adapted vplot to Postscript and X. Dave
Nichols introduced our multivendor environment. Joel M. Schroeder and
Matthias Schwab converted from cake to gmake. Bob Clapp expanded
Ratfor for Fortran 90. Martin Karrenbach got us started with
CD-ROMs. Sergey Fomel upgraded the Latex version to “2e” and he
implemented the basic changes taking us from CD-ROM to the WWW, a
process which continues to this day in year 2000.

\noindent Jon Claerbout \\
Stanford University \\
December 26, 2000