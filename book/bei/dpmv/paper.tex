%\def\SEPCLASSLIB{../../../../sepclasslib}
\def\CAKEDIR{.}

\title{Dip and offset together}
\author{Jon Claerbout}
\maketitle
\label{paper:dpmv}

\def\vhalf{v_{\rm half}}

\par
\footnote{
%	Jim Black prepared many figures
%	and the geometrical derivations at the end of the chapter.
	Matt Schwab prepared a draft of the Gardner DMO derivation.
	Shuki Ronen gave me the ``law of cosines'' proof.
	}When
dip and offset are combined,
some serious complications arise.
For many years it was common industry practice
to ignore these complications and to handle dip and offset separately.  
Thus offset was handled by velocity analysis, 
normal moveout and stack (chapter~\ref{paper:vela}).
And dip was handled by zero-offset 
migration after stack (chapters~\ref{paper:krch} and \ref{paper:dwnc}).
This practice is a good approximation only when
the dips on the section are small.
We need to handle large offset angles at the same
time we handle large dip angles
at the same time we are estimating rock velocity.
It is confusing!
Here we see the important steps
of bootstrapping yourself towards both the velocity and the image.

%In fact, using this {\it separable} style
%of seismic processing generally leads to seismic sections on which
%the steeply-dipping reflectors are artificially attenuated relative 
%to the horizontal reflectors.
%The easiest way to understand how things fail 
%is to consider the case of a planar dipping reflector at finite offset.

\section{PRESTACK MIGRATION}
\inputdir{XFig}
%One solution to the problems pointed out in the previous section
%is to forget about NMO, stack, and zero-offset migration
%and replace them by prestack migration.
Prestack migration creates an image of the earth's reflectivity
directly from prestack data.
It is an alternative to
the ``\bx{exploding reflector}'' concept
that proved so useful in zero-offset migration.
In \bx{prestack migration},
\sx{migration!prestack}
we consider both downgoing and upcoming waves.
\par
A good starting point for discussing prestack migration is a reflecting 
point within the earth.
A wave incident on the point from any direction
reflects waves in all directions.
This geometry is particularly important because
any model is a superposition of such point scatterers.
The point-scatterer geometry for a point
located at $(x,z)$ is shown in Figure~\ref{fig:pgeometry}. %
\sideplot{pgeometry}{width=3.5in}{
  Geometry of a point scatterer.
}The equation
for travel time  $t$  is the sum of the two travel paths is
\begin{equation}
t\,v\  \eq \  \sqrt { z^2\ +\ {( s \ -\  x ) }^2}  
\ +\  \sqrt { z^2 \ +\ {( g \ -\  x )}^2} 
\label{eqn:dsrsg}
\end{equation}
We could model field data
with equation~(\ref{eqn:dsrsg}) by copying reflections from any point
in $(x,z)$-space into $(s,g,t)$-space.
The adjoint program would form an image stacked over
all offsets.
This process would be called prestack migration.
The problem here is that the real problem
is estimating velocity.
In this chapter we will see that it is not satisfactory
to use a horizontal layer approximation to estimate velocity,
and then use equation~(\ref{eqn:dsrsg}) to do migration.
Migration becomes sensitive to velocity when wide angles are involved.
Errors in the velocity would spoil whatever benefit could
accrue from prestack (instead of poststack) migration.

\subsection{Cheops' pyramid}
\sx{Cheops' pyramid}
\inputdir{Math}
\par
Because of the importance of the point-scatterer model,
we will go to considerable lengths to visualize the functional dependence
among $t$, $z$, $x$, $s$, and $g$ in equation (\ref{eqn:dsrsg}).
This picture is more difficult---by one dimension---than is
the conic section of the exploding-reflector geometry.
\par
To begin with,
suppose that the first square root in (\ref{eqn:dsrsg}) is constant
because everything in it is held constant.
This leaves the familiar hyperbola in  $(g,t)$-space,
except that a constant has been added to the time.
Suppose instead that the other square root is constant.
This likewise leaves a hyperbola in  $(s,t)$-space.
In  $(s,g)$-space, travel time is a function of  $s$  plus a function of  $g$.
I think of this as one coat hanger, which is 
parallel to the $s$-axis, being hung from another coat hanger,
which is parallel to the $g$-axis.
\par
A view of the traveltime pyramid on the  $(s,g)$-plane
or the  $(y,h)$-plane is shown in Figure~\ref{fig:cheop}a.
\plot{cheop}{height=3.4in}{
	Left is a picture of the traveltime pyramid
	of equation (\protect(\ref{eqn:dsrsg}))
	for fixed  $x$  and  $z$.
	The darkened lines are constant-offset sections.
	Right is a cross section through the pyramid
	for large  $t$  (or small  $z$).  (Ottolini)
	}
%\newslide
Notice that a cut through the pyramid at
large  $t$  is a square, the corners of which have been smoothed.
At very large  $t$,
a constant value of  $t$  is the square contoured in  $(s,g)$-space,
as in Figure~\ref{fig:cheop}b.
Algebraically, the squareness becomes evident for a point reflector
near the surface, say,  $z  \to  0$.
Then (\ref{eqn:dsrsg}) becomes
\begin{equation}
v\,t\  \eq \ | s \ -\  x |\ \ +\ \ | g \ -\  x |
\label{eqn:2.4}
\end{equation}
The center of the square is located at  $(s,g) = (x,x)$.
Taking travel time  $t$  to increase downward
from the horizontal plane of  $(s,g)$-space,
the square contour is like a horizontal slice through the Egyptian pyramid
of Cheops.
To walk around the pyramid at a constant altitude 
is to walk around a square.
Alternately,
the altitude change of a traverse over
$g$  (or $s$) at constant
$s$  (or $g$) is simply a constant plus an absolute-value function.
\par
More interesting and less obvious are the curves
on common-midpoint gathers and constant-offset sections.
Recall the definition that the midpoint between the shot and geophone is  $y$.
Also recall that  $h$  is half the horizontal offset
from the shot to the geophone.
\begin{eqnarray}
y\ \ \ \ &=&\ \ \ \ {g \ +\  s  \over 2 }
\label{eqn:2.5a}
\\
h\ \ \ \ &=&\ \ \ \ {g \ -\  s  \over 2 }
\label{eqn:2.5b}
\end{eqnarray}
A traverse
of $y$ at constant $h$ is shown in Figure~\ref{fig:cheop}.
At the highest elevation on the traverse,
you are walking along a flat horizontal
step like the flat-topped hyperboloids of Figure~\ref{fig:twopoint}.
Some erosion to smooth the top and edges of the pyramid
gives a model for nonzero reflector depth.
\par
For rays that are near the vertical,
the traveltime curves are far from the hyperbola asymptotes.
Then the square roots in (\ref{eqn:dsrsg}) may be expanded in Taylor series,
giving a parabola of revolution.
This describes the eroded peak of the pyramid.

\subsection{Prestack migration ellipse}
\inputdir{dmovplot}
Denoting the horizontal coordinate $x$ of the scattering point by $y_0$
Equation (\ref{eqn:dsrsg}) in $(y,h)$-space is
\begin{equation}
t\,v\  \eq \  \sqrt { z^2\ +\ {( y \ -\  y_0  \  - \  h) }^2}  
     \ +\  \sqrt { z^2\ +\ {( y \ -\  y_0   \ + \  h) }^2} 
\label{eqn:dsryh}
\end{equation}
A basic insight into equation (\ref{eqn:dsrsg}) is to notice
that at constant-offset  $h$  and constant travel time  $t$
the locus of possible reflectors is
an ellipse in the $(y ,z)$-plane centered at $y_0$.
The reason it is an \bx{ellipse}
follows from the geometric definition of an ellipse.
To draw an ellipse,
place a nail or tack into $s$ on Figure~\ref{fig:pgeometry}
and another into $g$.
Connect the tacks by a string
that is exactly long enough to go through  $(y_0 ,z)$.
An ellipse going through  $(y_0 ,z)$  may be constructed
by sliding a pencil along the string,
keeping the string tight.
The string keeps the total distance  $tv$  constant as is shown in
Figure~\ref{fig:ellipse1}
\sideplot{ellipse1}{width=3.5in}{
	Prestack migration ellipse, the locus of all scatterers with
	constant traveltime for	source S and receiver G.
	}
%\newslide
\par

Replacing depth $z$ in equation~(\ref{eqn:dsryh})
by the vertical traveltime depth
$\tau = 2z/v=z/\vhalf$ we get
\begin{equation}
t  \eq   {1 \over 2}\
       \left(
	\sqrt { \tau^2\ +\ [( y-y_0)-h]^2 / \vhalf^2 } 
	\ +\  
	\sqrt { \tau^2\ +\ [( y-y_0)+h]^2 / \vhalf^2 } 
	\ 
	\right)
\label{eqn:dsryhtau}
\end{equation}

\subsection{Constant offset migration}
\inputdir{matt}
\sx{constant-offset migration}
\sx{migration!constant-offset}
Considering $h$ in equation~(\ref{eqn:dsryhtau})
to be a constant,
enables us to write a subroutine for migrating constant-offset sections.
%Subroutine \texttt{flathyp()} \vpageref{lst:flathyp} is easily prepared
%from subroutine \texttt{kirchfast()} \vpageref{lst:kirchfast} by
%replacing its hyperbola equation with equation~(\ref{eqn:dsryhtau}).
%\progdex{flathyp}{const offset migration}
%The amplitude in subroutine {\tt flathyp()}
%should be improved when we have time to do so.
Forward and backward responses to impulses
%of subroutine {\tt flathyp()}
are found in Figures~\ref{fig:Cos1} and \ref{fig:Cos0}.
\sideplot{Cos1}{width=3.in,height=3.in}{
	Migrating impulses on a constant-offset section.
%	with subroutine {\tt flathyp()}.
	Notice that shallow impulses
	(shallow compared to $h$)
	appear ellipsoidal
	while deep ones appear circular.
	}
%\newslide
\sideplot{Cos0}{width=3.in,height=3.in}{
	Forward modeling
	from an earth impulse. % with subroutine {\tt flathyp()}.
	}
%\newslide

\par
It is not easy to show that equation (\ref{eqn:dsryh}) can be
cast in the standard mathematical form of an ellipse, namely, 
a stretched circle.
But the result is a simple one,
and an important one for later analysis.
Feel free to skip forward over the following verification
of this ancient wisdom.
To help reduce algebraic verbosity,
define a new  $y$  equal to the old one shifted by  $y_0 $.
Also make the definitions
\begin{eqnarray}
t\,v \ \ \ \ &=&\ \ \ \ 2\ A\  			\label{eqn:mymajor}
\\
\alpha\ \ \ \ &=&\ \ \ \ z^2 \ \ +\ \  (y\ +\ h)^2   \nonumber
\\
\beta\ \ \ \ &=&\ \ \ \ z^2 \ \ +\ \  (y\ -\ h)^2   \nonumber
\\
\alpha\ \ -\ \ \beta\ \ \ \ &=&\ \ \ \ 4\ y\ h  \nonumber
\end{eqnarray}
With these definitions, (\ref{eqn:dsryh}) becomes
\begin{displaymath}
2\ A\  \eq \  \sqrt \alpha \ \ +\ \   \sqrt \beta  
\end{displaymath}
Square to get a new equation with only one square root.
\begin{displaymath}
4\ A^2 \ \ -\ \ (\alpha\ +\ \beta) \  \eq \ 2\  \sqrt{ \alpha \beta }
\end{displaymath}
Square again to eliminate the square root.
\begin{eqnarray*}
16\ A^4 \ \ -\ \ 8\ A^2 \, (\alpha\ +\ \beta) \ \ +\ \ (\alpha\ +\ \beta)^2 \ \ \ \ &=&
\ \ \ \ 4\ \alpha\ \beta
\\
16\ A^4 \ \ -\ \ 8\ A^2 \, (\alpha\ +\ \beta) \ \ +\ \ (\alpha\ -\ \beta)^2 \ \ \ \ &=&
\ \ \ \ 0
\end{eqnarray*}
Introduce definitions of  $\alpha$  and  $\beta$.
\begin{displaymath}
16\ A^4 \ \ -\ \  8\ A^2 \ [\,2\,z^2 \ +\ 2\,y^2 \ +\  2\,h^2 ] \ \ +\ \ 
16\ y^2 \, h^2 \  \eq \ 0 
\end{displaymath}
Bring  $y$  and  $z$  to the right.
\begin{eqnarray}
A^4 \ \ -\ \  A^2 \, h^2 \ \ \ \ &=&\ \ \ \ 		\nonumber
A^2 \, ( z^2 \ +\  y^2 ) \ \ -\ \  y^2 \, h^2
\\
A^2 \, ( A^2 \ -\  h^2 ) \ \ \ \ &=&\ \ \ \ 
A^2 \,  z^2 \ +\   ( A^2 \ -\   h^2 ) \, y^2		\nonumber
\\
A^2 \ \ \ \ &=&\ \ \ \  {z^2  \over 1 \ -\  {h^2  \over A^2}}\ \ +\ \ y^2
							\label{eqn:torick}
\end{eqnarray}
Finally, recalling all earlier definitions and replacing $y$ by $y-y_0$, we
obtain the canonical form of an ellipse with semi-major axis $A$ and 
semi-minor axis $B$:
\begin{equation}
{(y\ -\ y_0)^2 \over A^2} \ +\ {z^2 \over B^2} \eq 1 \ \ \ ,
\label{eqn:canellipse}
\end{equation}
where
\begin{eqnarray}
 A &\eq& {v\ t \over 2}    \\
 B &\eq& \sqrt{A^2\ -\ h^2}
\end{eqnarray}
\par
Fixing $t$, equation (\ref{eqn:canellipse}) is the equation for a circle with
a stretched $z$-axis.
The above algebra confirms that the
``string and tack'' definition of an \bx{ellipse}
matches the ``stretched circle'' definition.
An \bx{ellipse} in earth model space corresponds
to an impulse on a constant-offset section.

\section{INTRODUCTION TO DIP}
\inputdir{XFig}
We can consider a data space to be a superposition of points
and then analyze the point response,
or we can consider data space
or model space to be a superposition of planes
and then do an analysis of planes.
Analysis of points is often easier than planes,
but planes, particularly local planes,
are more like observational data and earth models.
\par
The simplest environment for reflection data
is a single horizontal reflection interface,
which is shown in Figure~\ref{fig:simple}.
\plot{simple}{height=2.5in}{
	Simplest earth model. 
	}
%\newslide
As expected, the zero-offset section mimics the earth model.
The common-midpoint gather is a hyperbola
whose asymptotes are straight lines
with slopes of the inverse of the velocity  $v_1$.
The most basic data processing is called
{\em 
\bx{common-depth-point stack}
}
or \bx{CDP stack}.
In it, all the traces on the common-midpoint (CMP) gather
are time shifted into alignment
and then added together.
The result mimics a zero-offset trace.
The collection of all such traces is called the
{\em 
CDP-stacked section.
}
In practice the CDP-stacked section is often interpreted and migrated
as though it were a zero-offset section.
In this chapter we will learn to avoid this popular,
oversimplified assumption.
\par
The next simplest environment is to have a planar reflector
that is oriented vertically rather than horizontally.
This might not seem typical,
but the essential feature (that the rays run horizontally)
really is common in practice
(see for example Figure~\ref{fig:shell}.)
Also, the effect of dip, while generally complicated,
becomes particularly simple in the extreme case.
If you wish to avoid thinking of 
wave propagation along the air-earth interface
you can take the reflector to be inclined a slight angle from the vertical,
as in Figure~\ref{fig:vertlay}.
\plot{vertlay}{height=2.5in}{
	Near-vertical reflector, a gather, and a section.
	}
%\newslide
\par
Figure~\ref{fig:vertlay} shows that the travel time
does not change as the offset changes.
It may seem paradoxical that the travel time does not increase as
the shot and geophone get further apart.
The key to the paradox is that midpoint is held constant, not shotpoint.
As offset increases,
the shot gets further from the reflector
while the geophone gets closer.
Time lost on one path is gained on the other.
\par
A planar reflector may have any dip between horizontal and vertical.
Then the common-midpoint gather lies between
the common-midpoint gather of Figure~\ref{fig:simple}
and that of Figure~\ref{fig:vertlay}.
The zero-offset section in Figure~\ref{fig:vertlay} is a straight line,
which turns out to be the asymptote of a family of hyperbolas.
The slope of the asymptote is the inverse of the velocity  $v_1$.

\par
It is interesting to notice that at small dips,
information about the earth velocity is essentially carried
on the offset axis whereas at large dips,
the velocity information is essentially on the midpoint axis.

\subsection{The response of two points}
\par
Another simple geometry is a reflecting point within the earth.
A wave incident on the point from any direction
reflects waves in all directions.
This geometry is particularly important because
any model is a superposition of such point scatterers.
Figure~\ref{fig:twopoint} shows an example.
\plot{twopoint}{height=2.4in}{
	Response of two point scatterers.
	Note the flat spots.
	}
%\newslide
The curves in Figure~\ref{fig:twopoint}
include flat spots for the same reasons
that some of the curves in Figures~\ref{fig:simple} and \ref{fig:vertlay}
were straight lines.

\inputdir{.}
\plot{shell}{height=6.5in}{
%	TO SEE THIS FIGURE, RENAME IT shell. (instead of unshell)
	Undocumented data from a recruitment brochure.
	This data may be assumed to be of textbook quality.
	The speed of sound in water is about 1500 m/sec.
	Identify the events at A, B, and C.
	Is this a common-shotpoint gather or a
	common-midpoint gather?  (Shell Oil Company)
	}
%\newslide

\subsection{The dipping bed}
\sx{dipping bed}
\inputdir{XFig}
\par
While the traveltime curves resulting from a dipping bed are simple,
they are not simple to derive.
Before the derivation, the result will be stated:
for a bed dipping at angle  $\alpha$  from the horizontal,
the traveltime curve is
\begin{equation}
t^2 \, v^2  \eq 
4 \, (y - y_0 )^2 \, \sin^2  \alpha
\ +\ 
4 \, h^2 \, \cos^2  \alpha
\label{eqn:shuki}
\end{equation}
For $\alpha \,=\,$ 45$^\circ$,
equation (\ref{eqn:shuki}) is the familiar Pythagoras cone---it
is just like  $t^2 = z^2 \,+\, x^2 $.
For other values of  $\alpha$,  the equation is still a cone,
but a less familiar one because of the stretched axes.
\par
For a common-midpoint gather at  $y\,=\, y_1$  in $(h,t)$-space,
equation (\ref{eqn:shuki}) looks
like  $t^2 \,=\, t_0^2 \,+$
$4h^2 / v_{\rm apparent}^2$.
Thus the common-midpoint gather contains an
{\em  exact}
hyperbola, regardless of the earth dip angle  $\alpha$.
The effect of dip is to change the asymptote of the hyperbola,
thus changing the apparent velocity.
The result has great significance in applied work and is
known as Levin's dip correction [1971]:
\begin{equation}
v_{\rm apparent}  \eq  {v_{\rm earth}  \over  \cos ( \alpha ) }
\label{eqn:2.2}
\end{equation}
(See also Slotnick [1959]).
In summary, dip increases the stacking velocity.
\sx{velocity!dip dependent}
\par
Figure~\ref{fig:dipray} depicts some rays from a common-midpoint gather.
%\activesideplot{dipray}{width=6in}{NR}{  I think this printed OK
%\activesideplot{dipray}{width=3.8in}{NR}{	 SPARC printer error.
\sideplot{dipray}{width=3in}{
	Rays from a common-midpoint gather.
	}
Notice that each ray strikes the dipping bed at a different place.
So a common-%
{\em  midpoint %
} gather is not a common-%
{\em  depth-point %
} gather.
To realize why the reflection point moves on the reflector,
recall the basic geometrical fact that an
angle bisector in a triangle generally doesn't bisect the opposite side.
The reflection point moves
{\em  up}
dip with increasing offset.
\par
Finally, equation (\ref{eqn:shuki}) will be proved.
Figure~\ref{fig:lawcos}
shows the basic geometry along with an ``image'' source
on another reflector of twice the dip.
%\activesideplot{lawcos}{width=3.8in}{NR}{ SPARC printer error
%\activesideplot{lawcos}{width=3in}{NR}{	I think this was OK.
\sideplot{lawcos}{width=3in}{
	Travel time from image source at  $s'$  to  $g$  may be
	expressed by the law of cosines.
	}
%\newslide
For convenience, the bed intercepts the surface at  $y_0 \,=\, 0$.
The length of the line $s'  g$ in Figure~\ref{fig:lawcos} is determined by
the trigonometric Law of Cosines to be
\begin{eqnarray*}
t^2 \, v^2 \ \ \ &=&\ \ \ 
s^2 \ +\  g^2 \ -\  2\,s\,g\,\cos\,2 \alpha
\\
t^2 \, v^2 \ \ \ &=&\ \ \ 
(y\,-\,h)^2 \ +\  (y\,+\,h)^2 \ -\ 2\,(y\,-\,h)(y\,+\,h)\, \cos \, 2 \alpha
\\
t^2 \, v^2 \ \ \ &=&\ \ \ 
2\,( y^2 \,+\,h^2 ) \ -\  2\,(y^{2\,-\,} h^2 ) \,
( \cos^2 \alpha \,-\, \sin^2 \alpha )
\\
t^2 \, v^2 \ \ \ &=&\ \ \ 
4\, y^2 \, \sin^2 \alpha \ \ +\ \  4\, h^2 \, \cos^2 \alpha
\end{eqnarray*}
which is equation (\ref{eqn:shuki}).
\par
Another facet of equation (\ref{eqn:shuki}) is that it describes
the constant-offset section.
Surprisingly, the travel time of a dipping planar bed becomes curved
at nonzero offset---it too becomes hyperbolic.




\subsection{Randomly dipping layers}
\inputdir{randip}
On a horizontally layered earth,
a common shotpoint gather looks like
a common midpoint gather.
For an earth model of random dipping planes
the two kinds of gathers have quite different
traveltime curves as we see in
Figure \ref{fig:randip}.
\plot{randip}{width=6in,height=3in}{
	Seismic arrival times on an earth of random dipping planes.
	Left is for CSP.  Right is for CMP.
	}
%\newslide

\par
The common-shot gather is more easily understood.
Although a reflector is dipping,
a spherical wave incident remains a spherical wave after reflection.
The center of the reflected wave sphere is called the image point.
The traveltime equation is again a cone centered
at the image point.
The traveltime curves are simply hyperbolas
topped above the image point having the usual asymptotic slope.
The new feature introduced by dip
is that the hyperbola is laterally shifted
which implies arrivals before the fastest possible straight-line arrivals
at $vt = |g|$.
Such arrivals cannot happen.
These hyperbolas must be truncated where $vt = |g|$.
This discontinuity has the physical meaning
of a dipping bed hitting the surface
at geophone location
$|g|=vt$.
Beyond the truncation, either the shot or the receiver
has gone beyond the intersection.
Eventually both are beyond.
When either is beyond the intersection, there are no reflections.

\par
On the common-midpoint gather we see 
hyperbolas all topping at zero offset,
but with asymptotic velocities
higher (by the Levin cosine of dip)
than the earth velocity.
Hyperbolas truncate, now at $|h| = tv/2 $,
again where a dipping bed hits the surface at a geophone.


\par
On a CMP gather, some hyperbolas may seem high velocity,
but it is the dip, not the earth velocity itself that causes it.
Imagine Figure \ref{fig:randip} with all layers at $90^\circ$ dip
(abandon curves and keep straight lines).
Such dip is like the backscattered groundroll
seen on the common-shot gather of Figure \ref{fig:shell}.
The backscattered groundroll
becomes a ``flat top'' on
the CMP gather in Figure \ref{fig:randip}.

\par
\inputdir{.}
Such strong horizontal events near zero offset
will match any velocity,
particularly higher velocities such as primaries.
Unfortunately such noise events
thus make a strong contribution to a CMP stack.
Let us see how these flat-tops in offset create
the diagonal streaks you see in midpoint in Figure \ref{fig:shelikof}.
\plot{shelikof}{height=3.3in}{
        CDP stack with water noise from the Shelikof Strait, Alaska.
        (by permission from
        {\em  Geophysics,}
        Larner et al.{[1983]})   % Don't remove the braces!  Latex bug!
        }
%\newslide

\par
Consider 360 rocks of random sizes scattered in an exact
circle of 2 km diameter on the ocean floor.
The rocks are distributed along one degree intervals.
Our survey ship sails from south to north
towing a streamer across the exact center of the circle,
coincidentally crossing directly over rock number 180 and number 0.
Let us consider the common midpoint gather corresponding
to the midpoint in the center of the circle.
Rocks 0 and 180 produce flat-top hyperbolas.
The top is flat for $0<|h|<1$ km.
Rocks 90 and 270 are $90^\circ$ out of the plane of the survey.
Rays to those rocks propagate entirely within the water layer.
Since this is a homogeneous media,
the travel time expression of these rocks
is a simple hyperbola of water velocity.
Now our CMP gather at the circle center has
a ``flat top'' and a simple hyperbola
both going through zero offset at time $t=2/v$ (diameter 2 km, water velocity).
Both curves have the same water velocity asymptote
and of course the curves are tangent at zero offset.

\par
Now consider all the other rocks.
They give curves inbetween the simple water hyperbola and the flat top.
Near zero offset, these curves range in apparent velocity
between water velocity and infinity.
One of these curves will have an apparent velocity that matches
that of sediment velocity.
This rock (and all those near the same azimuth)
will have velocities that are near the sediment velocity.
This noise will stack very well.
The CDP stack at sediment velocity will stack
in a lot of water borne noise.
This noise is propagating somewhat off the survey line
but not very far off it.

\par
Now let us think about the appearance of the CDP stack.
We turn attention from offset to midpoint.
The easiest way to imagine the CDP stack
is to imagine the zero-offset section.
Every rock has a water velocity asymptote.
These asymptotes are evident on
the CDP stack in Figure \ref{fig:shelikof}.
This result was first recognized by Ken Larner.

\par
Thus, backscattered low-velocity noises have
a way of showing up on higher-velocity stacked data.
There are two approaches
to suppressing this noise:
(1) mute the inner traces,
and as we will see,
(2) dip moveout processing.


\section{TROUBLE WITH DIPPING REFLECTORS}
\par
The ``standard process'' is NMO, stack, and zero-offset migration.
Its major shortcoming is the failure of NMO and stack
to produce a section that resembles the true zero-offset section.
In chapter~\ref{paper:vela} we derived
the NMO equations for a stratified earth,
but then applied them to seismic field data 
that was not really stratified.
That this works at all is a little surprising,
but it turns out that NMO hyperbolas
apply to dipping reflectors as well as horizontal ones.
%One of the first results of this section is the establishment of this result.
When people try to put this result into practice,
however,
they run into a nasty conflict:
reflectors generally require a {\em dip-dependent}
NMO velocity in order to produce a ``good'' stack.
Which NMO velocity are we to apply
when a dipping event is near (or even crosses) a horizontal event?  
Using conventional NMO/stack techniques
generally forces velocity analysts to choose
which events they wish to preserve on the stack.
This inability to simultaneously produce a good stack
for events with {\em all} dips is a serious shortcoming,
which we now wish to understand more quantitatively.

\subsection{Gulf of Mexico example}
\inputdir{krchdmo}
\par
Recall the Gulf of Mexico dataset presented in chapter~\ref{paper:vela}.
We did a reasonably careful job of NMO velocity analysis
in order to produce the stack shown in Figure~\ref{vela/fig:agcstack}.
But is this the best possible stack?
To begin to answer this question, Figure~\ref{fig:cvstacks} shows
some constant-velocity stacks of this dataset done with subroutine
\texttt{velsimp()} \vpageref{lst:velsimp}.
This figure clearly shows that
there are some very steeply-dipping reflections
that are missing in Figure~\ref{vela/fig:agcstack}.
These steep reflections appear only when the NMO velocity
is quite high compared with the velocity
that does a good job on the horizontal reflectors.
This phenomenon is consistent with 
the predictions of equation~(\ref{eqn:shuki}),
which says that dipping events 
require a {\em higher} NMO velocity than nearby horizontal events.
\plot{cvstacks}{width=6.0in}{
	Stacks of Gulf of Mexico data
	with two different constant NMO velocities.
	Press button to see a \bx{movie} in which each frame
	is a stack with a different constant velocity.
	}
%\newslide
%
\par
Another way of seeing the same conflict in the data
is to look at a velocity-analysis panel
at a single common-midpoint location
such as the panel shown in Figure~\ref{fig:velscan}
made by subroutine~\texttt{velsimp()} \vpageref{lst:velsimp}.
In this figure it is easy to see that the velocity
which is good for the dipping event at 1.5 sec is too high
for the horizontal events in its vicinity.
%
\sideplot{velscan}{width=4.0in,height=3.3in}{
	Velocity analysis panel of one of the panels
	in Figure~\protect\ref{fig:cvstacks}
	before (left) and after (right) DMO.
	Before DMO, at 2.2 sec you can notice two values of slowness,
	the main branch at .5 sec/km, and another at .4 sec/km.
	The faster velocity $s=.4$ is a fault-plane reflection.
	}
%\newslide

\section{SHERWOOD'S DEVILISH}
\inputdir{.}
The migration process should be thought of as being interwoven with the
velocity estimation process.
J.W.C. \bx{Sherwood} [1976] indicated how the two processes,
migration and velocity estimation, should be interwoven.
The moveout correction should be considered in two parts,
one depending on offset, the NMO, and the other depending on dip.
This latter process was conceptually new.
Sherwood described the process as a kind of filtering,
but he did not provide implementation details.
He called his process
{\em  Devilish,}
an acronym for ``dipping-event velocity inequalities licked.''
The process was later described more functionally by \bx{Yilmaz} as
{\em 
\bx{prestack partial migration},
}
\sx{migration!prestack partial}
and now
the process is usually called
{\em 
dip moveout
}
(\bx{DMO})
although some call it MZO, migration to zero offset.
We will first see Sherwood's results,
then Rocca's conceptual model of the \bx{DMO} process,
and finally two conceptually distinct, quantitative specifications
of the process.
\par
Figure~\ref{fig:digicon} contains a panel from a stacked section.
\plot{digicon}{height=3.8in}{
	Conventional stacks with varying velocity.
	(distributed by Digicon, Inc.)
	}
%\newslide
The panel is shown several times;
each time the stacking velocity is different.  
It should be noted that at the low velocities,
the horizontal events dominate,
whereas at the high velocities,
the steeply dipping events dominate.
After the
{\em  Devilish}
correction was applied, the data was restacked as before.
Figure~\ref{fig:devlish} shows that the stacking velocity
no longer depends on the dip.
\plot{devlish}{height=3.8in}{
	{\em  Devilish} 
	stacks with varying velocity.
	(distributed by Digicon, Inc.)
	}
%\newslide
This means that after 
{\em  Devilish,}
the velocity may be determined without regard to dip.
In other words,
events with all dips contribute to the same consistent velocity
rather than each dipping event predicting a different velocity.
So the 
{\em  Devilish} 
process should provide better velocities for data with conflicting dips.
And we can expect a better final stack as well.

\section{ROCCA'S SMEAR OPERATOR}
\inputdir{matt}
Fabio \bx{Rocca} developed
a clear conceptual model for Sherwood's dip corrections.
Start with an impulse on a common offset section,
and migrate it getting ellipses like in Figure~\ref{fig:Cos1}.
%We did this with subroutine~\texttt{flathyp()} \vpageref{lst:flathyp}
%using some constant-offset  {\tt h}.
Although the result is an ellipsoidal curve,
think of it as a row of many points along an ellipsoidal curve.
Then diffract the image
thus turning each of the many points into a hyperbola.
%We do this with the return path of the same subroutine {\tt flathyp()},
%however the path back is taken with {\tt h=0}.
The result is shown in Figure~\ref{fig:frocca}.
To enhance the appearance of the figure,
I injected an intermediate step of converting the ellipsoid
curve into a trajectory of dots on the ellipse.
Notice that the hyperbola tops
are not on the strong smear function that results
from the superposition.

\par
The strong smear function that you see in Figure~\ref{fig:frocca}
is Rocca's \bx{DMO}+NMO operator,
the operator that converts a point on a constant-offset section
to a zero-offset section.
The important feature of this operator is that
the bulk of the energy is in a much narrower region
than the big ellipse of migration.
The narrowness of the Rocca operator is important
since it means that energies will not move far,
so the operator will not have a drastic effect
and be unduly affected by remote data.
(Being a small operator also makes it cheaper to apply).
The little signals you see away from the central burst
in Figure~\ref{fig:frocca} result mainly from
my modulating the ellipse curve into a sequence of dots.
However, noises from sampling and nearest-neighbor interpolation
also yield a figure much like Figure~\ref{fig:frocca}.
This warrants a more careful theoretical study
to see how to represent the Rocca operator
directly (rather than as a sequence of two nearly opposite operators).
\sideplot{frocca}{width=3in,height=2in}{
	Rocca's prestack partial-migration operator is
	a superposition of hyperbolas, each with its top on an ellipse.
	}
%\newslide
\par
\inputdir{Math}
To get a sharper, more theoretical view of the Rocca operator,
Figure~\ref{fig:rocca2} shows line drawings of the curves in
a Rocca construction.
It happens, and we will later show,
that the Rocca operator lies along an ellipse
that passes through $\pm h$
(and hence is independent of velocity!)
Curiously,
we see something we could not see on Figure~\ref{fig:frocca},
that the Rocca curve ends part way up the ellipse
and it does not reach the surface.
The place where the Rocca operator ends
and the velocity independent ellipse continues is, however,
velocity dependent as we will see.
The Rocca operator is along the curve of osculation in Figure~\ref{fig:rocca2},
i.e.,~the smile-shaped curve where the hyperbolas reinforce one another.
\plot{rocca2}{width=5in}{
	Rocca's smile. (Ronen)
	}
%\newslide

\subsection{Push and pull}
\inputdir{matt}
%A program is a ``pull'' program if the loop creating the output
%covers each location in the output and gathers the input from wherever
%it may be.
%A program is a ``push'' program if it takes each input and
%pushes it to wherever it belongs.
%The moveout and stack and Kirchhoff migration programs
%we have been writing are
%``pull'' programs
%for doing the model building (data processing),
%and they are ``push'' programs for the data building.
%We could have written programs that worked the other way around,
%namely, a loop over $t$ with $z$ found
%by the semicircle calculation $z=\sqrt{t^2/v^2-x^2}$.

\par
Migration and diffraction operators can be conceived
and programmed in two different ways.
Let $\vec {\bold t}$ denote data and $\vec {\bold z}$ denote the depth image.
We have
\begin{eqnarray}
\vec {\bold z} &=& \bold C_h\ \vec {\bold t}
        \quad\quad {\rm spray\ or\ push\ an\ ellipse\ into\ the\ output}
 \\
\vec {\bold t} &=& \bold H_h\ \vec {\bold z}
\quad\quad {\rm spray\ or\ push\ a\ flattened\ hyperbola\ into\ the\ output}
\end{eqnarray}
where $h$ is half the shot-geophone offset.
The adjoints are
\begin{eqnarray}
\vec {\bold t} &=& \bold C_h' \; \vec {\bold z} \quad\quad
	{\rm sum\ or\ pull\ a\ semiCircle\ from\ the\ input}
	\\
\vec {\bold z} &=& \bold H_h' \; \vec {\bold t} \quad\quad
	{\rm sum\ or\ pull\ a\ flattened\ Hyperbola\ from\ the\ input}
\end{eqnarray}
In practice we can choose either of $\bold C \approx \bold H'$.
A natural question is which is more correct or better.
The question of ``more correct''
applies to modeling and is best answered by theoreticians
(who will find more than simply a hyperbola;
they will find its waveform including its amplitude and phase
as a function of frequency).
The question of ``better'' is something else.
An important practical issue is that the transformation
should not leave miscellaneous holes in the output.
It is typically desirable to write programs that
loop over all positions in the {\em  output} space,
``pulling'' in whatever inputs are required.
It is usually less desirable
to loop over all positions in the {\em  input} space,
``pushing'' or ``spraying'' each input value to the appropriate
location in the output space.
Programs that push the input data to the output space
might leave the output
too sparsely distributed.
Also, because of gridding, the output data might be irregularly positioned.
Thus, to produce smooth outputs, we usually
{\em  prefer the summation operators}
$\bold H'$ for migration and $\bold C'$ for diffraction modeling.
Since one could always force smooth outputs by lowpass filtering,
what we really seek is the highest possible resolution.

\par
Given a nonzero-offset section,
we seek to convert it to a zero-offset section.
Rocca's concept is to first migrate the constant offset data
with an ellipsoid push operator $\bold C_h$
and then take each point on the ellipsoid
and diffract it out to a zero-offset hyperbola
with a push operator $\bold H_0$.
The product of push operators
$\bold R = \bold H_0 \bold C_h$
is known as Rocca's smile.
This smile operator includes both normal moveout and dip moveout.
(We could say that dip moveout is defined by Rocca's smile
after restoring the normal moveout.)

\par
Because of the approximation $\bold H \approx \bold C'$,
we have four different ways to express the Rocca smile:
\begin{equation}
\bold R \eq
		 \bold H_0 \bold C_h
\quad\approx\quad
		 \bold H_0 \bold H'_h
\quad\approx\quad
		 \bold C'_0 \bold H'_h
\quad\approx\quad
		 \bold C'_0 \bold C_h
\label{eqn:fourR}
\end{equation}
$\bold H_0 \bold H'_h$ says sum over a flat-top and then spray
a regular hyperbola.
\par
The operator $\bold C'_0 \bold H'_h$,
having two pull operators should have smoothest output.
Sergey Fomel suggests an interesting illustration of it:
Its adjoint is two push operators,
$ \bold R' = \bold H_h \bold C_0$.
$ \bold R'$ takes us from zero offset to nonzero offset
first by pushing a data point to a semicircle
and then by pushing points on the semicircle to flat-topped hyperbolas.
As before, to make the hyperbolas more distinct,
I broke the circle into dots along the circle
and show the result in Figure \ref{fig:sergey}.
The whole truth is a little more complicated.
Subroutine %\texttt{flathyp()} \vpageref{lst:flathyp} 
implements $\bold H$ and $\bold H'$.
Since I had no subroutine for $\bold C$,
figures \ref{fig:frocca} and \ref{fig:sergey}
were actually made with only $\bold H$ and $\bold H'$.
\sideplot{sergey}{width=3in,height=2in}{
	The adjoint of Rocca's smile
	is a superposition of flattened hyperbolas,
	each with its top on a circle.
        }
%\newslide
We discuss the
$\bold C'_0 \bold C_h$
representation of $\bold R$ in the next section.

\subsection{Dip moveout with $v(z)$}
\inputdir{yalei}
It is worth noticing that the concepts in this section
are not limited to constant velocity
but apply as well to $v(z)$.
However,
the circle operator $\bold C$ presents some difficulties.
Let us see why.
Starting from the Dix moveout approximation,
$t^2 = \tau^2 + x^2/v(\tau )^2$,
we can directly solve for $t(\tau ,x)$
but finding $\tau (t,x)$ is an iterative process at best.
Even worse, at wide offsets, hyperbolas cross one another
which means that $\tau (t,x)$ is multivalued.
The spray (push)
operators $\bold C$ and $\bold H$ loop over inputs
and compute the location of their outputs.
Thus
$ \vec {\bold z} = \bold C_h\ \vec {\bold t}$
requires we compute $\tau$ from $t$ so it is
one of the troublesome cases.
Likewise, the sum (pull)
operators $\bold C'$ and $\bold H'$ loop over outputs.
Thus
$ \vec {\bold t} = {\bold C'}_h\ \vec {\bold z}$
causes us the same trouble.
In both cases, the circle operator
turns out to be the troublesome one.
As a consequence, most practical work is done with
the hyperbola operator.
\par
A summary of the meaning of the Rocca smile
and its adjoint is found in Figures
\ref{fig:yalei2} and \ref{fig:yalei1}.
%which were computed using subroutine
%\texttt{flathyp()} \vpageref{lst:flathyp}.

\plot{yalei2}{width=6.in,height=1.5in}{ 
	Impulses on a zero-offset section migrate
	to semicircles.
	The corresponding constant-offset section
	contains the adjoint of the Rocca smile.
	}
%\newslide

\plot{yalei1}{width=6.in,height=1.5in}{ 
	Impulses on a constant-offset section
	become ellipses in depth and
	Rocca smiles on the zero-offset section.
	}
%\newslide


\section{GARDNER'S SMEAR OPERATOR}
\inputdir{XFig}
A task, even in constant velocity media,
is to find analytic expressions for the travel time
in the Rocca operator.
This we do now.

\par
The Rocca operator $ \bold R = \bold C'_0 \bold C_h$
says to spray out an ellipse and then sum over a circle.
This approach,
associated with Gerry \bx{Gardner},
says that we are interested in all circles
that are inside and tangent to an ellipse,
since only the ones that are tangent
will have a constructive interference.

\par
The Gardner formulation answers this question:
Given a single nonzero offset impulse,
which events on the zero-offset section
will result in the same migrated subsurface picture?
Since we know the migration response
of a zero and nonzero offset impulses (circle and ellipse)
we can rephrase this question:
Given an ellipse corresponding
to a nonzero offset impulse,
what are the circles tangent to it
that have their centers at the earth's surface?
These circles if superposed will yield the ellipse.
Furthermore,
each of these circles corresponds
to an impulse on the zero-offset section.
The set of these impulses in the zero offset section
is the \bx{DMO}+NMO impulse response for a given nonzero offset event.

\plot{ell}{width=4.in,height=3.in}{
	The nonzero offset migration impulse response is an ellipse.
	This ellipse can be mapped as a superposition
	of tangential circles with centers along the survey line.
	These circles correspond to zero offset migration impulse responses.
	}
%\newslide

\subsection{Restatement of ellipse equations}
Recall equation~(\ref{eqn:canellipse})
for an ellipse centered at the origin.
\begin{equation}
        \label{eqn:ellipse}
	0 \eq {y^{2}\over{A^{2}}} + {z^{2}\over{B^{2}}} -1 .
\end{equation}
where
\begin{equation}
        \label{eqn:ellipseA}
	A \eq  \vhalf\, t_h ,
\end{equation}
\begin{equation}
        \label{eqn:ellipseB}
	B^2 \eq A^2 - h^2 .
\end{equation}

The ray goes from the shot at one focus of the ellipse
to anywhere on the ellipse,
and then to the receiver in traveltime $t_h$.
The equation for a circle of radius $R=t_0 \vhalf$
with center on the surface
at the source-receiver pair coordinate $x=b$ is
\begin{equation}
        \label{eqn:circle}
	R^2 \eq (y - b)^2 + z^{2} ,
\end{equation}
\noindent
where
\begin{equation}
	\label{eqn:radius}
        R \eq t_0 \, \vhalf .
\end{equation}

To get the circle and ellipse tangent to each other,
their slopes must match.
Implicit differentiation of equation (\ref{eqn:ellipse}) and (\ref{eqn:circle})
with respect to $y$ yields:
\begin{equation}
        \label{eqn:slope.ell}
	0 \eq {y \over{A^2}} + {z \over{B^2}}
	\ {dz \over dy}
\end{equation}
\begin{equation}
        \label{eqn:slope.cir}
	0 \eq (y-b) + z
	\ {dz \over dy}
\end{equation}
\noindent
Eliminating $dz/dy$ from equations~(\ref{eqn:slope.ell}) and (\ref{eqn:slope.cir}) yields:
\begin{equation}
        \label{eqn:b}
	y \eq {b\over 1 - {B^{2}\over{A^{2}}}}   .
\end{equation}
At the point of tangency the circle and the ellipse should coincide.
Thus we need to combine equations to eliminate $x$ and $z$.
We eliminate $z$ from equation (\ref{eqn:ellipse}) and (\ref{eqn:circle})
to get an equation only dependent on the $y$ variable. 
This $y$ variable can be eliminated by inserting equation~(\ref{eqn:b}).
\begin{equation}
        \label{eqn:geodmo}
	R^2 \eq B^2 \left( {A^2 - B^2 - b^2 \over{A^2 - B^2}} \right).
\end{equation}
\par
Substituting the definitions (\ref{eqn:ellipseA}), (\ref{eqn:ellipseB}), 
(\ref{eqn:radius}) of various parameter gives the
relation between zero-offset traveltime $t_0$ and nonzero traveltime
$t_h$:
\begin{equation}
        \label{eqn:opdmo}
	t_0^2\eq
	\left(t_h^2-{h^{2}\over\vhalf^2}\right)
	\left(1-{b^2\over h^2}\right).
\end{equation}

As with the Rocca operator, equation (\ref{eqn:opdmo})
includes both dip moveout \bx{DMO} and NMO.
%For a more careful derivation,
%please see Forel and Gardner's paper \shortcite{GEO53.05.06040610}.

\section{DMO IN THE PROCESSING FLOW}
\inputdir{matt}
\par
Instead of implementing equation (\ref{eqn:opdmo})
in one step we can split it into two steps. 
The first step converts raw data at time $t_h$
to NMOed data at time $t_n$.

\begin{equation}
	\label{eqn:nmoeq}
	t_n^2 \eq t_h^2 - {h^2 \over{\vhalf^2}}
\end{equation}

The second step is the \bx{DMO} step which
like Kirchhoff migration itself
is a convolution over the $x$-axis (or $b$-axis) with
\begin{equation}
	\label{eqn:dmoeq}
	t_0^2 \eq t_n^2 \left( 1 - {b^2 \over h^2} \right)
\end{equation}
and it converts time $t_n$ to time $t_0$.
Substituting (\ref{eqn:nmoeq}) into (\ref{eqn:dmoeq}) leads back to (\ref{eqn:opdmo}).
As equation (\ref{eqn:dmoeq}) clearly states,
the \bx{DMO} step itself is essentially velocity independent, 
but the NMO step naturally is not. 

Now the program.
Backsolving equation~(\ref{eqn:dmoeq}) for $t_n$ gives

\begin{equation}
        \label{eqn:dmoprog}
	t_n^2 \eq
	{t_0^2 \over 1-b^2/h^2 } .
\end{equation}

%Like subroutine~\texttt{flathyp()} \vpageref{lst:flathyp},
%our \bx{DMO} subroutine~\texttt{dmokirch()} \vpageref{lst:dmokirch} is based on
%subroutine~\texttt{kirchfast()} \vpageref{lst:kirchfast}.
%It is just the same,
%except where {\tt kirchfast()} has a hyperbola
%we put equation~(\ref{eqn:dmoprog}).
%In the program,
%    the variable $t_0$ is called {\tt z}
%and the variable $t_n$ is called {\tt t}.
%Note, that the velocity {\tt velhalf} does exclusively
%occur in the break condition
%(which we have failed to derive,
%but which is where the circle and ellipse touch at $z=0$).
%\progdex{dmokirch}{fast Kirchhoff dip-moveout}

In figures ~\ref{fig:dmatt} and ~\ref{fig:coffs},
%were made with subroutine~\texttt{dmokirch()} \vpageref{lst:dmokirch}. 
notice the big noise reduction over Figure~\ref{fig:frocca}.
\sideplot{dmatt}{width=3.in}{
	Impulse response of DMO and NMO
	}
%\newslide
\sideplot{coffs}{width=3.in}{
	Synthetic Cheop's pyramid
	}
%\newslide

\subsection{Residual NMO}
\sx{residual NMO}
Unfortunately, the theory above shows
that \bx{DMO} should be performed {\em  after} NMO.
\bx{DMO} is a convolutional operator,
and significantly more costly than NMO.
This is an annoyance because it would
be much nicer if it could be done
once and for all, and not need to be redone for each new NMO velocity.

%A deeper problem is that it is not obvious how
%we should process data with a depth variable velocity $v(z)$
%since the theory here
%seems so strongly dependent on mathematics of constant velocity media.
%As has often been demonstrated in practice,
%it is easy to do better with \bx{DMO} than without.
%We can imagine two approaches.
%\begin{enumerate}
%
%\item
%	\begin{enumerate}
%	\item
%		Do NMO with a regional constant velocity $\bar v$.
%	\item
%		Do DMO  with a regional constant velocity $\bar v$.
%		(Alternately, you could do NMO and DMO together).
%	\item
%		Do adjoint NMO with velocity $\bar v$.
%	\item
%		Treat the result as input to ``standard processing,''
%		i.e.~pick the velocity $v(\tau)$,
%		do moveout, stack, and do zero-offset migration.
%	\end{enumerate}
%
%\item
%	\begin{enumerate}
%	\item
%	Do NMO as always for the best velocity model $v(\tau)$.
%	\item
%	Then do DMO at some constant regional velocity  $\bar v$.
%	\item
%	Then do zero-offset migration using $v(\tau)$.
%	\end{enumerate}
%
%\end{enumerate}

\par
Much practical work is done with using constant velocity for the DMO process.
This is roughly valid since DMO, unlike NMO, does little
to the data so the error of using the wrong velocity is much less.

\par
It is not easy to find a theoretical impulse response
for the DMO operator in $v(z)$ media,
but you can easily compute the impulse response in $v(z)$
by using $\bold R=\bold H_0\bold H_h'$ from
equation (\ref{eqn:fourR}).

\subsection{Results of our DMO program}
\inputdir{krchdmo}
\par
We now return to the field data from the Gulf of Mexico,
which we have processed earlier
in this chapter and in chapter~\ref{paper:vela}.  

%Figure~\ref{fig:cvdmostks} is a repeat of Figure~\ref{fig:cvstacks} except that
%the data has now been processed
%with \bx{DMO} prior to the final NMO and stack operation.
%Notice that the amount of conflict in velocity choice is greatly diminished,
%although not completely absent.
%\activeplot{cvdmostks}{width=6.0in}{ER}{
%	Stacks of Gulf of Mexico data
%	with two different constant NMO velocities.
%	This figure is the same as Figure~\protect\ref{fig:cvstacks}
%	except that dip-moveout has been applied to the data.
%	}\newslide

\plot{wgdmostk}{width=6.20in,height=7.8in}{
	Stack after the dip-moveout correction.
	Compare this result
	with Figure~\protect\ref{vela/fig:agcstack}.
	This one has fault plane reflections to the right of the faults.
	}
%\newslide

\plot{wgdmomig}{width=6.20in,height=7.8in}{
	Kirchhoff migration of the previous figure.
	Now the fault plane reflections jump to the fault.
	}
%\newslide

%HEY, MAYBE THEY WOULD LIKE TO SEE SOME CONSTANT OFFSET MIGRATIONS
%FOLLOWED BY STACK?
%I think I will postpone that until I
%can verify quality of existing results and
%get some timings of individual figures.

%\todo{	 % comment out Black DMO geometry and miscellaneous IEI stuff
%
%\section{THE END}
%The Figure~\FIG{dmospike1} demonstrates the processing 
%sequence. The left frame shows the input, a column of
%spikes in a constant-offset section. The middle frame
%shows the same constant-offset gather after Normal moveout with the 
%corresponding velocity $vhalf$. All the events are moved up in time.
%The rightmost frame displays the gather after DMO was applied. The
%Events are smeared into neighboring midpoints along the DMO smile.
%\activeplot{dmospike1}{height=3.0in}{}{
%	Illustration of the DMO impulse response.
%	The left panel is the common-offset input data
%	at an offset of $1.0$ km.
%	The center panel show the result of applying normal moveout
%	with a velocity of $2.0$ km/s.
%	The right panel is after the application of dip-moveout
%	with the program {\tt kdmoslow()}
%	}
%\subsection{Ottolini's radial trace dip moveout}
%\par
%Ordinarily we regard a common-midpoint gather as a collection of seismic traces,
%that is, a collection of time functions,
%each one for some particular offset  $h$.
%But this  $(h,t)$  data space could be represented
%in a different coordinate system.
%A system with some nice attributes is the radial-trace
%system introduced by Turhan Taner.
%In this system the traces are not taken at constant  $h$,
%but at constant angle.
%The idea is illustrated in Figure~\FIG{otto}.
%\activesideplot{otto}{height=2.5in}{}{
%	Inside the data volume of a reflection seismic traverse
%	are planes called {\it radial-trace sections.}
%	A point scatterer inside the earth puts a hyperbola on a
%	radial-trace section.
%	}
%\par
%Besides having some theoretical advantages, which will become apparent,
%this system also has some practical advantages, notably:
%(1) the traces neatly fill the space where data is nonzero;
%(2) the traces are close together at early times where wavelengths are short,
%and wider apart where wavelengths are long; and
%(3) the energy on a given trace tends to represent
%wave propagation at a fixed angle.
%The last characteristic is especially important
%with multiple reflections.
%But for our purposes the best attribute of radial traces is still 
%another one.
%\par
%Richard Ottolini noticed that a point scatterer in the earth
%appears on a radial-trace section as an
%{\it exact}
%hyperbola, not a flat-topped hyperboloid.
%The traveltime curve for a point scatterer, Cheops' pyramid,
%can be written as a ``string length'' equation,
%or a stretched-circle equation.
%Making the definition
%\begin{equation}
%\sin \, \psi  \eq  { 2\,h  \over  v\,t}
%\EQNLABEL{rickdef}
%\end{equation}
%and substituting
%it and equation~\EQN{mymajor} into
%equation~\EQN{torick} yields
%\begin{equation}
%v\,t  \eq 
%2\  \sqrt{ { z^2    \over  \cos^2 \psi } \ +\  (y\ -\ y_0 )^2 }
%\EQNLABEL{6.8}
%\end{equation}
%Scaling the $z$-axis by  $ \cos \, \psi $  gives the circle and
%hyperbola case all over again!
%Figure~\FIG{ottohyp} shows a three-dimensional sketch
%of the hidden hyperbola.
%\activeplot{ottohyp}{height=2.8in}{}{
%	An unexpected hyperbola in Cheops' pyramid is the
%	diffraction hyperbola on a radial-trace section.
%	(Harlan)
%	}
%\subsection{Anti-alias characteristic of dip moveout}
%\par
%You might think that if $(y,h,t)$-space is sampled
%along the $y$-axis at a sample interval  $\Delta y$,
%then any final migrated section  $P(y,z)$  would
%have a spatial resolution no better than  $\Delta y$.
%This is not the case.
%\par
%The basic principle 
%at work here has been known since 
%the time of Shannon.
%If a time function and its derivative are sampled
%at a time interval $2 \Delta t$,
%they can both be fully reconstructed provided that
%the original bandwidth of the signal is lower than  $1/(2 \Delta t)$.
%More generally, if a signal is filtered with  $m$  independent filters,
%and these  $m$  signals are sampled at an interval  $m \Delta t$,
%then the signal can be recovered.
%\par
%Here is how this concept applies to seismic data.
%The basic signal is the earth model.
%The various filtered versions of it are the constant-offset sections.
%Recall that the CDP reflection point moves up dip as the offset is increased.
%Further details can be found in a paper by
%Bolondi, Loinger, and Rocca [1982],
%who first pointed out the anti-alias properties of dip moveout.
%At a time of increasing interest in 3-D seismic data,
%special attention should be paid to the anti-alias character of dip moveout.
%\section{DIPPING-REFLECTOR RAY GEOMETRY (Black)}
%\par
%While the finite-offset traveltime curves
%and raypath positions for a reflector dipping at angle $\theta$
%from the horizontal are simple,
%they are not simple to derive.
%Before presenting the derivations,
%we will state the two principal results.  
%The first result is that 
%the traveltime curve seen at midpoint location $y$ is 
%\begin{equation}
%t^2  \eq 
%t_m^2 
%\ +\ 
%{4 \, h^2 \, \cos^2\theta \over v^2 }
%\EQNLABEL{levin1}
%\end{equation}
%where $t_m$ is the traveltime at $h=0$ at the same midpoint location.
%\par
%For a common-midpoint gather at midpoint $y$, 
%equation \EQN{levin1} looks
%like  $t^2 \,=\, t_m^2 \,+$
%$4h^2 / v_{\rm apparent}^2$.
%Thus the common-midpoint gather contains an
%{\it exact}
%hyperbola, regardless of the earth dip angle  $\theta$.
%The effect of dip is to change the asymptote of the hyperbola,
%thus changing the apparent velocity.
%The result has great significance in practical seismic processing and is
%known as Levin's dip correction [1971]:
%\begin{equation}
%v_{\rm apparent}  \eq  {v_{\rm earth}  \over  \cos\theta }
%\EQNLABEL{levin2}
%\end{equation}
%(See also Slotnick [1959]).
%This increase of stacking velocity with dip is the first complication of
%the combination of offset and dip.
%\par
%The second complication is more subtle, but equally important.
%Figure~\FIG{conflict} depicts some rays from a common-midpoint gather.
%\activesideplot{conflict}{width=3.5in}{}{
%	Finite-offset rays for dipping and flat reflectors
%	at a common midpoint M.
%	As the offset increases,
%	the shot S and receiver G move symmetrically away from M.
%	Note how the reflection point ``creeps up''
%	the dipping reflector.
%	Press button for \bx{movie}.
%	}
%Notice that each ray strikes the dipping bed at a different reflection point.
%In fact, there is a simple formula for the position of the reflection point
%along the dipping reflector:
%\begin{equation}
%\Delta L \eq {h^2 \over d} \cos\theta \ \sin\theta \ \ ,
%\EQNLABEL{cdpsmear}
%\end{equation}
%where $d$ is the perpendicular distance from the midpoint to the reflector.
%So a common-%
%\it midpoint %
%\rm gather %
%is not a 
%\it common-reflection-point
%\rm gather.
%The reflection point moves
%{\it up}
%dip with increasing offset.
%This effect is usually referred to as 
%{\it CDP smear} or {\it reflection-point smear}.
%It means that stacking the traces in a common-midpoint 
%gather does not generally create an accurate estimate of the zero-offset 
%section when there are dipping reflectors.
%\par
%\subsection{Moveout derivation}
%\par
%We now derive the two equations we have just presented, 
%starting  with a geometrical derivation of equation~\EQN{levin1}.
%Figure~\FIG{clayer} shows the raypath geometry for that go from S to R to G.
%Inspection of the figure shows
%that the traveltime along SRG is the same as the traveltime along S'RG.  
%The point S' is the ``image source,''
%the mirror-image of S in the reflector.
%\activesideplot{clayer}{height=2.7in}{}{
%	Traveltime from source  $S$ to reflection point  $R$
%	to receiver $G$ is the same as the straight-line traveltime 
%	from image
%	source  $S'$ to  $R$ to  $G$.
%	}
%\par
%Triangle S'XG is a right triangle,
%whose hypotenuse length $S'G$ gives the desired traveltime.
%Once the lengths $XG$ and $S'X$ are known,
%simple application of the Pythagorean theorem
%will yield equation~\EQN{levin1}.
%The expression for $XG$ is very easy to derive.
%Since $SG = 2h$, 
%the trigonometry of the right triangle XSG yields
%\begin{equation}
%XG \eq 2\, h\, \cos\theta \ \ \ . 
%\EQNLABEL{XG}
%\end{equation}
%The length $XS'$ is only slightly more complicated,
%being the sum
%\begin{displaymath}
%XS' \eq XS \ + \ SA \ + \ AS' \ \ . 
%\end{displaymath}
%The first term in this equation, $XS$, is again related to $SG$
%by simple trigonometry on triangle XSG:
%\begin{displaymath}
%XS \eq 2\, h\, \sin\theta   \ \ \ . 
%\end{displaymath}
%Getting the lengths $SA$ and $AS'$ uses 
%$SA = BA - BS$, where $BA$ is simply $d$, 
%the perpendicular distance from reflector to midpoint.
%The trigonometry of the right triangle BSM shows that
%$BS = h \sin\theta$ so that 
%%
%\begin{equation}
%SA \eq AS'  \eq d \ - \ h \ \sin\theta \ \ \ . 
%\EQNLABEL{ASp}
%\end{equation}
%%
%where the first equality comes from the fact that S' is the mirror image of
%S through point A.
%Thus the final expression for $XS'$ is
%\begin{equation}
%XS' \eq 2\,h\,\sin\theta \,+\, 2\,(d-h\sin\theta) \eq 2\,d
%\EQNLABEL{XSp}
%\end{equation}
%Finally, applying the Pythagorean theorem yields the desired traveltime:
%\begin{displaymath}
%v^2t^2 \eq 4\,d^2 + 4\,h^2\,\cos^2\theta \ \ , 
%\end{displaymath}
%which is equation~\EQN{levin1}
%with the definition of two-way zero-offset traveltime
%at the midpoint as
%\begin{equation}
%t_m \eq {2\,d \over v}
%\EQNLABEL{TM}
%\end{equation}
%\subsection{Smear derivation}
%\par
%Now turning to \EQN{cdpsmear},
%we want to compute the distance $RR'$ in Figure~\FIG{clayer},
%which  is the quantity $\Delta L$.
%We begin by noting that $RR'=AR'-AR$ and that 
%\begin{equation}
%AR' \eq BM \eq h \, \cos\theta \ \ \ .
%\EQNLABEL{ARp}
%\end{equation}
%Thus all we need to find out is the distance $AR$.
%From the figure it is apparent that triangle $AS'R$
%is similar to triangle $XS'G$.
%Thus it follows that
%\begin{displaymath}
%{AR \over AS'} \eq {XG \over XS'}  \ \ . 
%\end{displaymath}
%But we already know $AS'$, $XG$, and $XS'$ from equations~\EQN{ASp},
%\EQN{XG}, and \EQN{XSp} above.
%Putting these all together yields
%\begin{displaymath}
%AR \eq (d\,-\,h\,\sin\theta)\ {h\,\cos\theta \over d} \ \ \ . 
%\end{displaymath}
%Combining this result with equation~\EQN{ARp} and $RR'=AR'-AR$ yields
%\begin{displaymath}
%\Delta L \eq h\,\cos\theta - (d\,-\,h\,\sin\theta) %
%	{h\,\cos\theta \over d} \ , 
%\end{displaymath}
%which simplifies to equation~\EQN{cdpsmear}.
%\subsection{Transforming to zero offset correctly}
%\par
%The above derivations show that applying NMO and stack
%to the traces in a common-midpoint gather
%causes two problems when the reflector is dipping:
%\begin{itemize}
%\item The NMO velocity depends on the dip angle $\theta$ according to
%equation~\EQN{levin2}.
%\item  Energy from different points along the reflector are being
%stacked together as though they were from the same point.
%This happens because the finite-offset reflection point $R$ creeps 
%up the reflector away from the zero-offset reflection point $R'$ as 
%offset increases in Figure~\FIG{clayer}.
%\end{itemize}
%These facts often prevent the resulting stacked trace from being
%a good estimate of a zero-offset trace 
%recorded at midpoint location $M$ in Figure~\FIG{clayer}.
%We will now show {\em what should have happened}
%in converting the finite-offset reflections
%in Figures~\FIG{conflict} and \FIG{clayer} to zero-offset reflections.  
%%In a later section in this chapter, we will show {\em how}
%%to make this happen with a process called {\em dip-moveout}.
%\par
%Figure~\FIG{slayer} is a simplified version of Figure~\FIG{clayer},
%showing how a finite-offset dipping event
%should be converted to zero-offset.
%Recall that the finite-offset event
%in this figure corresponds to raypaths $SR$ and $RG$.
%We have drawn an additional line segment $ZR$ in this figure.
%This segment is called a {\em normal ray} because it strikes the reflector 
%at normal incidence.
%Thus the normal ray from reflection point $R$
%corresponds to what a zero-offset seismic survey
%above this reflector would record.
%Note that the zero-offset survey would ``see'' the reflection energy 
%from $R$ not at point $M$ but rather at point $Z$,
%where the normal ray hits the earth's surface.
%Thus the proper conversion from finite offset to zero offset
%must somehow ``move'' this reflection energy
%to surface location $Z$ and give it a two-way traveltime
%corresponding to the zero-offset raypath $ZR$.
%This movement to the correct zero-offset surface location
%and traveltime is a type of prestack migration called ``dip-moveout'' or
%``prestack partial migration.''
%\activesideplot{slayer}{height=2.4in}{}{
%	Reflection event from source {\bf S} to reflection point {\bf R}
%	to receiver {\bf G} needs to be ``migrated'' to zero-offset, 
%	corresponding to {\bf ZR}.
%	}
%\par
%To see what kind of movement is required,
%we first must clarify where we are moving {\em from}.
%In midpoint-offset coordinates,
%the reflection energy in Figure~\FIG{slayer} is initially
%located at midpoint $M$ with offset $2\ h$ and traveltime given by 
%equation~\EQN{levin1}.
%The distance along the surface from initial finite-offset position
%$M$ to final zero-offset position $Z$ is easily deduced from the geometry of
%the small triangle $ZPM$ in the figure.
%\begin{equation}
%\Delta y \eq  -\ ZM \eq -\ {\Delta L \over \cos\theta} \eq %
%			-\ {h^2 \over d} \sin\theta  \ \ ,
%\EQNLABEL{dmody}
%\end{equation}
%where we have used equation~\EQN{cdpsmear} and the fact that $ZP = \Delta L$.
%Similarly, the expression for two-way traveltime $t_0$ along the 
%zero-offset raypath
%$ZR$ is obtained by noting that $ZR = d - PM$, which yields
%\begin{equation}
%v\,t_0\, /\,2 \eq d            \ -\ \Delta L\  \tan\theta \eq %
%                  v\,t_m\, /\,2\ -\ {h^2 \over d} \sin^2\theta \ \ ,
%\EQNLABEL{dmot0}
%\end{equation}
%where we have used equations~\EQN{TM} and \EQN{cdpsmear}
%to get the second equality.
%Notice that both the spatial shift in equation~\EQN{dmody} and the
%temporal shift in equation~\EQN{dmot0} vanish when either dip is zero
%or offset is zero,
%so that they are definitely consequences
%of the occurrence of dip and offset together.
%	\subsection{
%	OTHER THINGS TO POSSIBLY PUT IN THIS SECTION:}
%	\begin{itemize}
%	\item Illustration of CDP smear by comparing two common-offset section
%	\begin{itemize}
%	\item A near-offset section with NMO appropriate for the steeply-dipping
%	reflector applied.
%	\item A far-offset section with NMO appropriate for the steeply-dipping
%	reflector applied.
%	\end{itemize}
%	\end{itemize}
%\section{DMO VIA DIPPING-BED ANALYSIS (BLACK)}
%\par
%In this section we investigate a two-step process for carrying out the 
%prestack migration described in the previous section.
%The first step is to convert the finite-offset data to 
%equivalent zero-offset data by a procedure that is closely related to normal
%moveout.  This procedure is most commonly called ``dip moveout''.  It is also
%known by the names ``prestack partial migration'' and ``offset continuation''
%and ``migration to zero offset.''  The second step is to perform 
%zero-offset migration on the data that has been converted to zero offset.  This
%two-step approach has considerable practical advantages over full prestack 
%migration, especially in processing 3--D seismic data.  
%%
%\subsection{From prestack migration to DMO}
%In earlier sections, we showed that the difficulty with 
%NMO and stack on common-midpoint gathers is that they do
%not produce the correct zero-offset sections when there are dipping reflectors.
%We also derived  expressions for what kind of spatial and temporal shifts
%would be required to fix things up.
%In this section, we are finally ready to derive the migration operator that
%carries out the correct transformation from finite offset to zero offset.  
%This transformation is called ``{\em dip moveout}'' or ``{\em DMO}'' for short.
%Instead of
%relying on the dipping reflector for this derivation
%we begin by asking the question ``What
%is the impulse response of the DMO operator?''  
%This is very similar to the
%question that we asked about zero-offset migration in chapter~\CHAP{krch}
%and about prestack migration in the previous section of the current chapter.
%In fact we can directly employ the prestack migration ellipse to answer our
%question.  The logic, first invented by Deregowski and Rocca[???] goes 
%like this:
%\activesideplot{ellipse2}{width=3.5in}{}{
%	The zero-offset ray, which is normal to  the prestack migration 
%	ellipse at the reflection point {\bf R}.
%	}
%\begin{enumerate}
%\item A spike in the finite-offset input data implies the
%elliptical reflector shown in Figure~\FIG{ellipse2}.
%\item Imagine collecting a zero-offset seismic section by moving
%a source and receiver along the earth's 
%surface above this elliptical reflector.  (The source and receiver in this
%thought experiment are, of course, on top of each other at point $Z$.)
%\item The resulting section is thus the zero-offset response to a spike in
%the finite-offset input data.  Thus this is the desired DMO impulse response.
%\end{enumerate}
%To make progress, we need to compute the zero-offset raypaths for the
%elliptical reflector of Figure~\FIG{ellipse2}.  To do this it is necessary
%to find the normal ray at each point along the ellipse.  Such a normal
%ray is shown as line segment $RZ$, 
%which is perpendicular to the ellipse at point $R$ in Figure~\FIG{ellipse2}.  
%The traveltime
%along $RZ$
%and the point $Z$ where the normal ray intersects the earth's surface 
%determine
%the desired DMO impulse response.
%\par
%The most critical step in deriving DMO's impulse response is choosing a 
%parametrization of 
%the prestack migration ellipse in equation~\EQN{canellipse} so that we
%can conveniently find the normal ray $RZ$.  The best choice
%is shown in Figure~\FIG{ellipse3}, in which the ellipse is parametrized 
%by the parameter $\theta$, which turns out to be the dip angle of the line
%segment that is tangent to the ellipse at point $R$.  
%We also introduce an auxiliary $\theta$-dependent distance $d$, which is
%the perpendicular distance between the midpoint $M$ and the tangent line
%segment.
%\activesideplot{ellipse3}{width=3.5in}{}{
%	Parametrizing the prestack-migration ellipse by 
%	$\theta$, the dip angle of the line segment that is tangent to the 
%	ellipse at each point $R$.
%	}
%Our parametrization expresses $y-y_0$ and $z$ as the following functions
%of $\theta$:
%\begin{eqnarray}
%\EQNLABEL{ytheta}
%y\ -\ y_0 &\eq& -\ {a^2 \over d}\ \sin\theta		\\
%\EQNLABEL{ztheta}
%z         &\eq&\ \ \  {b^2 \over d}\ \cos\theta		\ \ \ ,
%\end{eqnarray}
%where $d$ is given by 
%\begin{equation}
%\EQNLABEL{dtheta}
%d^2	   \eq  a^2\ \sin^2\theta \ +\ b^2\ \cos^2\theta     \ \ \  .
%\end{equation}
%To confirm that this parametrization works, substitute these equations 
%into equation~\EQN{canellipse}:
%\begin{displaymath}
%{(y\ -\ y_0)^2 \over a^2} \ +\ {z^2 \over b^2} \eq %
%{a^2\ \sin^2\theta \over d^2} \ +\ {b^2\ \cos^2\theta \over d^2} %
% 				\eq 1 \ \ \ , 		 
%\end{displaymath}
%where we have used the definition of $d$ in the final step.  Thus this is
%a valid parametrization of the ellipse in equation~\EQN{canellipse}.  
%\par
%The next step is to show that the parameter $\theta$ is really the dip angle
%of the tangent line segment.  To do this, compute 
%the slope of the tangent line segment by making a small
%change $\Delta \theta$ in $\theta$ and forming the ratio of the
%resulting changes in $y$ and $z$.  
%\begin{eqnarray*}
%\Delta y &\eq& -\ {a^2 \over d  }\ \cos\theta\ \Delta\theta  %
%	       +\ {a^2 \over d^2}\ \sin\theta\ \Delta d %
%	  \eq\ -\ {a^2 b^2 \over d^3}\ \cos\theta 		\\
%\Delta z &\eq& -\ {b^2 \over d  }\ \sin\theta\ \Delta\theta  %
%	       -\ {b^2 \over d^2}\ \cos\theta\ \Delta d %
%	  \eq\ -\ {a^2 b^2 \over d^3}\ \sin\theta 	
%\end{eqnarray*}
%where we have used equation~\EQN{dtheta} to find that
%\begin{displaymath}
%\Delta d \eq {a^2\ -\ b^2 \over d}\ \sin\theta\ \cos\theta\ \Delta d\ \ .
%\end{displaymath}
%Taking the ratio of $\Delta z$ to $\Delta y$ in the limit that $\Delta\theta$
%vanishes gives the simple result
%\begin{equation}
%{dz \over dy} \eq \tan\theta \ \ ,
%\end{equation}
%which confirms that $\theta$ is indeed the dip angle of the 
%tangent line segment in Figure~\FIG{ellipse3}.
%\par
%The final feature of Figure~\FIG{ellipse3} to be verified is that $d$ is
%really the normal distance $MR'$ between the midpoint $M$ and the tangent line
%segment.  This distance is given by the dot-product 
%between the vector $MR$
%and the unit
%vector normal to the line segment, ${\bf n} = (\sin\theta,\cos\theta)$:
%\begin{equation}
%MR' \eq {\bf n}\ \cdot\ (\,y-y_0\,,\ z\,) \eq % 
%  {a^2\,\sin^2\theta\ +\ b^2\,\cos^2\theta \over d} \eq d \ \ .
%\end{equation}
%In confirming this result, we have used equations~\EQN{ytheta}-\EQN{dtheta}.
%\par
%Armed with the parametrization of Figure~\FIG{ellipse3}, 
%it is now a simple matter
%to derive the equations for the zero-offset raypath $ZR$ in 
%figure~\FIG{ellipse2}.  The zero-offset two-way traveltime $t_0$ is given by
%\begin{equation}
%{v\,t_0 \over 2} \eq ZR \eq {z \over \cos\theta} \eq {b^2 \over d} \ \ ,
%\EQNLABEL{t0ell1}
%\end{equation}
%where we have used the trigonometry of triangle $CRZ$ and the fact that
%$CR = z$.  It is convenient to rewrite this equation
%in slightly different form as
%\begin{equation}
%{t_0 \over t_h} \eq {t_h \over t_m} \eq {b \over d}
%\EQNLABEL{t0ell2}
%\end{equation}
%where
%\begin{equation}
%\left({v\,t_h \over 2}\right)^2 \eq b^2 %
%	\eq \left({v\,t \over 2}\right)^2\ -\ h^2 \ \ ,
%\EQNLABEL{tnell}
%\end{equation}
%and $t_m = 2d/v$ is the two-way zero-offset time at 
%the midpoint $M$, as defined 
%earlier in equation~\EQN{TM}.  Similarly the spatial shift $\Delta y$ between
%the midpoint $M$ and the endpoint $Z$ of the zero-offset raypath is given by
%\begin{eqnarray}
%\Delta y \eq  CZ\ -\ CM &\eq& z\,\tan\theta\ -\ {b^2 \over d}\sin\theta %
%					\nonumber \\
%   &\eq& \left({b^2 \over d}\ -\ {a^2 \over d}\right)\,\sin\theta \nonumber \\
%	&\eq& -\ {h^2 \over d}\,\sin\theta  \ \ \ ,
%\EQNLABEL{dyell1}
%\end{eqnarray}
%where we have used the trigonometry of triangle $CRZ$ to find $CZ$ from $CR=z$
%and have used equation~\EQN{ytheta} to find $CM$.  
%\par
%Now it is a matter of simple algebra to find the DMO impulse response, which is
%the relationship between $t_0$ and $\Delta y$:
%\begin{eqnarray}
%\left({t_0 \over t_h}\right)^2 &\eq& {b^2 \over d^2} %
%   \eq 1\ -\ {{d^2\ -\ b^2}  \over d^2}  \nonumber \\
%% &\eq& 1\ -\ {a^2\ \sin^2\theta\ +\ b^2\ %
%%	\cos^2\theta -\ b^2 \over d^2} \nonumber \\
%        &\eq&  1\ -\ {h^2\ \sin^2\theta \over d^2} \nonumber     \\
%	&\eq&  1\ -\ \left({\Delta y \over h}\right)^2 \ \ \ ,
%\EQNLABEL{dmoimpl}
%\end{eqnarray}
%where we have used the definition of $d$ in equation~\EQN{dtheta} to reach 
%the second line and have used equation~\EQN{dyell1} to attain the final
%expression.
%Note that the above equation has the form of a simple 
%ellipse (different from the prestack-migration ellipse) in 
%$(t_0,\Delta y)$-space!
%This is the famous DMO impulse response (Rocca's smile) first derived by
%Deregowski and Rocca [19XX].
%It takes a spike in $(y,h)$-space and converts it into a zero-offset image.
%%Subroutine \GPROG{kdmoslow} follows from
%%equation~\EQN{dmoimpl} and subroutine \GPROG{kirchslow}.
%%\progdex{kdmoslow}{Slow Kirchhoff dip-moveout}
%%A program is \GPROG{kdmoslow}.
%\subsection{The impulse response}
%\subsection{Back to the dipping reflector}
%\par
%Before concluding this section,
%we pause to make sure that the DMO impulse response
%we have just derived is consistent with the dipping-reflector analysis
%that we carried out earlier in this chapter.
%Clearly the impulse response's 
%spatial shift formula,
%equation~\EQN{dyell1}, is {\em identical}
%to the required spatial shift for a dipping reflector,
%given in equation~\EQN{dmody}.
%To verify that the zero-offset times
%are the same takes a little additional work.
%Rewrite the impulse-response's equation~\EQN{t0ell1} as
%\begin{equation}
%{v\,t_0 \over 2} \eq d\ -\ {d^2\ -\ b^2 \over d} \eq %
%{v\,t_m \over 2}\ - {h^2\,\sin^2\theta \over d} \ \ ,
%\end{equation}
%which is {\em precisely} the same as equation~\EQN{dmot0}!
%
%}	% end of commented out Black geometry of DMO
%
%
%
%

%\section{PUTBIB destroys following information.  keep it last!}

%\putbib[GEOTLE]	








































































