\title{Imaging in shot-geophone space}
\author{Jon Claerbout}
\maketitle

Till now, we have limited our data processing to midpoint-offset space.
We have not analyzed reflection data directly in shot-geophone space.
In practice this is often satisfactory.
Sometimes it is not.
The principal factor that drives us away from $(y,h)$-space into $(s,g)$-space
is lateral velocity variation $v(x,z) \ne v(z)$.
In this chapter,
we will see how migration can be performed
in the presence of $v(x,z)$
by going to $(s,g)$-space.
\par
Unfortunately, this chapter has no prescription for finding $v(x,z)$,
although we will see how the problem manifests itself
even in apparently stratified regions.
We will also see why, in practice, amplitudes are dangerous.



\section{TOMOGRAPY OF REFLECTION DATA}
\sx{tomography}
\par
Sometimes the earth strata lie horizontally with little irregularity.
There we may hope to ignore the effects of migration.
Seismic rays should fit a simple model with large reflection
angles occurring at wide offsets.
Such data should be ideal for the measurement of reflection
coefficient as a function of angle,
or for the measurement of the earth acoustic absorptivity  $1/Q$.
In his doctoral dissertation, Einar
\bx{Kjartansson}
reported such a study.
The results were so instructive that the study will be thoroughly reviewed here.
I don't know to what extent the Grand Isle gas field typifies
the rest of the earth,
but it is an excellent place to begin learning about the
meaning of shot-geophone offset.
\subsection{The grand isle gas field: a classic bright spot}
\par
The dataset \bx{Kjartansson} studied was a seismic line across the Grand Isle
gas field, off the shore of Louisiana.
The data contain several classic ``bright spots'' (strong reflections)
on some rather flat undisturbed bedding.
Of interest are the lateral variations in amplitude
on reflections at a time depth of about 2.3 seconds on Figure~\ref{fig:kjcos}.
It is widely believed that such bright spots
arise from gas-bearing sands.

\par
Theory predicts that reflection coefficient should
be a function of angle.
For an anomalous physical situation
like gas-saturated sands, the function should be distinctive.
Evidence should be found on common-midpoint gathers
like those shown in Figure~\ref{fig:kjcmg}.
\inputdir{.}
\plot{kjcmg}{height=7.5in}{
	Top left is shot point 210; top right is shot point 220.
	No processing has been applied to the data except
	for a display gain proportional to time.
	Bottom shows shot points 305 and 315.  (Kjartansson)
	}
Looking at any one of these gathers you will note that the reflection
strength versus offset seems to be a smooth,
sensibly behaved function, apparently quite measurable.
Using layered media theory, however, it was determined that only
the most improbably bizarre medium could exhibit such strong
variation of reflection coefficient with angle,
particularly at small angles of incidence.
(The reflection angle of the energy arriving at wide offset at time 2.5 seconds
is not a large angle.
Assuming constant velocity, ${\rm arccos} (2.3/2.6)\,=\,28$$^\circ$).
Compounding the puzzle, each common-midpoint gather shows a
{\em  different}
smooth, sensibly behaved, measurable function. 
Furthermore, these midpoints are near one another,
ten shot points 
spanning a horizontal distance of 820 feet.
\subsection{Kjartansson's model for lateral variation in amplitude}
\inputdir{XFig}
\par
The Grand Isle data is incomprehensible in terms of the
model based on layered media theory.
Kjartansson proposed an alternative model.
Figure~\ref{fig:kjidea} illustrates a geometry in which rays travel
in straight lines from any source to a flat horizontal reflector,
and thence to the receivers.
%\activeplot{kjidea}{height=5.5in}{NR}{
\plot{kjidea}{height=4.5in}{
	Kjartansson's model.  The model on the top 
	produces the disturbed data space sketched below it. 
	Anomalous material in pods A, B, and C may be detected
	by its effect on reflections from a deeper layer.
	}
The only complications are ``pods'' of some material
that is presumed to disturb seismic rays in some anomalous way.
Initially you may imagine that the pods absorb wave energy.
(In the end it will be unclear whether the disturbance results from
energy focusing or absorbing).

\inputdir{.}
\par
Pod A is near the surface.
The seismic survey is affected by it twice---once
when the pod is traversed by the shot and once when it is
traversed by the geophone.
Pod C is near the reflector and encompasses a small area of it.
Pod C is seen at all offsets  $h$  but only at one midpoint,  $y_0$.
The raypath depicted on the top of Figure~\ref{fig:kjidea} is one that is
affected by all pods.
It is at midpoint  $y_0$  and at the widest offset  $h_{\rm max}$.
Find the raypath on the lower diagram in Figure~\ref{fig:kjidea}.
\par
Pod B is part way between A and C.
The slope of affected points in the $(y,h)$-plane is part way between
the slope of A and the slope of C.
\par
Figure~\ref{fig:kjcos} shows a common-offset section across the gas field.
The offset shown is the fifth trace from the near offset,
1070 feet from the shot point.
Don't be tricked into thinking the water was deep.
The first break at about .33 seconds is wide-angle propagation.
\plot{kjcos}{height=8.5in}{
	A constant-offset section across the Grand Isle gas field.
	The offset shown is the fifth from the near trace.  (Kjartansson, Gulf)
	}
\plot{kja}{height=8.5in}{
	(a) amplitude (h,y), (b) timing (h,y)
	(c) amplitude (z,y), (d) timing (d,y)
	}
\par
The power in each seismogram was computed in the interval from
1.5 to 3 seconds.
The logarithm of the power is plotted in Figure~\ref{fig:kja}a as a function
of midpoint and offset.
Notice streaks of energy slicing across the  $(y,h)$-plane
at about a 45$^\circ$ angle.
The strongest streak crosses at exactly 45$^\circ$ 
through the near offset at shot point 170.
This is a missing shot, as is clearly visible in Figure~\ref{fig:kjcos}.
Next, think about the gas sand described as pod C in the model.
Any gas-sand effect in the data should show up as a streak
across all offsets at the midpoint of the gas sand---that is,
horizontally across the page.
I don't see such streaks in Figure~\ref{fig:kja}a.
Careful study of the figure shows that
the rest of the many clearly visible streaks cut the plane at
an angle noticeably
{\em  less}
than $\pm$45$^\circ$.
The explanation for the angle of the streaks in the figure 
is that they are like pod B.
They are part way between the surface and the reflector.
The angle determines the depth.
Being closer to 45$^\circ$ than to 0$^\circ$, the pods are
closer to the surface than to the reflector.
\par
Figure \ref{fig:kja}b shows timing information in the same form that
Figure~\ref{fig:kja}a shows amplitude.
A CDP stack was computed, and each field
seismogram was compared to it.
A residual time shift for each trace was determined
and plotted in Figure~\ref{fig:kja}b.
The timing residuals on one of the common-midpoint
gathers is shown in Figure~\ref{fig:kjmid}.
\sideplot{kjmid}{width=3.0in}{
	Midpoint gather 220 (same as timing of (h,y) in
	Figure~\protect\ref{fig:kja}b) after moveout.
	Shown is a one-second window centered at 2.3 seconds, time shifted
	according to an NMO for an event at 2.3 seconds, using a velocity
	of 7000 feet/sec.  (Kjartansson)
	}
\par
The results resemble the amplitudes, except that the
results become noisy when the amplitude is low or
where a ``leg jump'' has confounded the measurement.
Figure \ref{fig:kja}b clearly shows that the disturbing influence on timing
occurs at the same depth as that which disturbs amplitudes.

\par
The process of 
{\em  inverse \bx{slant stack}}
(not described in this book)
enables one to determine the depth distribution of the pods.
This distribution is displayed
in figures \ref{fig:kja}c and \ref{fig:kja}d.
\subsection{Rotten alligators}
\inputdir{meander}
\par
The sediments carried by the Mississippi River are dropped at the delta.
There are sand bars, point bars, old river bows now silted in,
a crow's foot of sandy distributary channels,
and between channels, swampy flood plains 
are filled with decaying organic material.
The landscape is clearly laterally variable,
and eventually it will all sink of its own weight,
aided by growth faults and the weight of later sedimentation.
After it is buried and out of sight the lateral variations will remain
as pods that will be observable by the seismologists of the future.
These seismologists may see something like Figure~\ref{fig:meander}.
Figure~\ref{fig:meander} shows a %
{\em  three-dimensional %
} seismic survey,
that is, the ship sails many parallel lines about 70 meters apart.
The top plane, a slice at constant time, shows buried river meanders.
%The data shown in Figure~\ref{fig:meander} is described more fully by its donors, Dahm and Graebner [1982].
\plot{meander}{width=4.5in}{
	Three-dimensional seismic data from the Gulf of Thailand.
	Data planes from within the cube are displayed on the faces of the cube.
	The top plane shows ancient river meanders now submerged.
	(Dahm and Graebner)
	}

\subsection{Focusing or absorption?}
\par
Highly absorptive rocks usually have low velocity.
Behind a low velocity pod, waves should be weakened by absorption.
They should also be strengthened by focusing.
Which effect dominates?
How does the phenomenon depend on spatial wavelength?
Maybe you can figure it out
knowing that black on Figure~\ref{fig:kja}c denotes
low amplitude or high absorption, and
black on Figure~\ref{fig:kja}d denotes low velocities.
\par
I'm inclined to believe the issue is focusing, not absorption.
Even with that assumption, however, a reconstruction of the velocity
$v(x,z)$
for this data has never been done.
This falls within the realm of ``reflection \bx{tomography}'',
a topic too difficult to cover here.
Tomography generally reconstructs a velocity model $v(x,z)$
from travel time anomalies.
It is worth noticing that with this data, however,
the amplitude anomalies seem to give more reliable information.


\begin{exer}

\item
Consider waves converted from pressure $P$ waves
to shear $S$ waves.
Assume an $S$-wave speed of about half the $P$-wave speed.
What would Figure~\ref{fig:kjidea} look like for these waves?
\end{exer}



\section{SEISMIC RECIPROCITY IN PRINCIPLE AND IN PRACTICE}
\inputdir{toldi}
\par
The principle of \bx{reciprocity} says
that the same seismogram should be recorded if the 
locations of the source and geophone are exchanged.
A physical reason for the validity of reciprocity is that no matter how
complicated a geometrical arrangement,
the speed of sound along a ray is the same in either direction.

\par
Mathematically, the reciprocity principle arises because
symmetric matrices arise.
The final result is that very complicated electromechanical systems mixing
elastic and electromagnetic waves generally fulfill the reciprocal principle.
To break the reciprocal principle,
you need something like a windy atmosphere so that sound
going upwind has a different velocity than sound going downwind.

\par
Anyway, since the
impulse-response matrix is symmetric,
elements across the matrix diagonal are equal to one another.
Each element of any pair is a response to an impulsive source.
The opposite element of the pair refers to
an experiment where the source and receiver
have had their locations interchanged.

\par
A tricky thing about the reciprocity principle
is the way antenna patterns must be handled.
For example, a single vertical geophone has a natural antenna pattern.
It cannot see horizontally propagating pressure waves nor vertically
propagating shear waves.
For reciprocity to be applicable,
antenna patterns must not be interchanged
when source and receiver are interchanged.
The antenna pattern must be regarded as attached to the medium.

\par
I searched our data library for split-spread land data that
would illustrate reciprocity under field conditions.
The constant-offset section in Figure~\ref{fig:toldi} was recorded
by vertical vibrators into vertical geophones.
\plot{toldi}{height=4.0in}{
	Constant-offset section from the Central Valley of California. 
	(Chevron)
	}

The survey was not intended to be a test of reciprocity,
so there likely was a slight lateral offset of the source line
from the receiver line.
Likewise the sender and receiver arrays (clusters)
may have a slightly different geometry.
The earth dips in Figure~\ref{fig:toldi} happen to be quite small
although lateral velocity variation
is known to be a problem in this area.
\par
In Figure~\ref{fig:reciptrace},
three seismograms were plotted on top of their reciprocals.
\plot{reciptrace}{width=6in,height=2.5in}{
	Overlain reciprocal seismograms.
	}
Pairs were chosen at near offset, at mid range,
and at far offset.
You can see that reciprocal seismograms usually have the same polarity,
and often have nearly equal amplitudes.
(The figure shown is the best of three such figures I prepared).
\par
Each constant time slice in Figure~\ref{fig:recipslice}
shows the reciprocity of many seismogram pairs.
\plot{recipslice}{width=6in,height=2.5in}{
	Constant time slices after NMO at 1 second and 2.5 seconds.
	}
Midpoint runs horizontally over the same range as in Figure~\ref{fig:toldi}.
Offset is vertical.
Data is not recorded near the vibrators
leaving a gap in the middle.
To minimize irrelevant variations,
moveout correction was done before making the time slices.
(There is a missing source that shows up on the left side of the figure).
A movie of panels like Figure~\ref{fig:recipslice} shows that
the bilateral symmetry you see in the individual panels
is characteristic of all times.
On these slices, you notice that the long wavelengths
have the expected bilateral symmetry
whereas the short wavelengths do not.

\par
In the laboratory,
reciprocity can be established
to within the accuracy of measurement.
This can be excellent.
(See White's example in FGDP).
In the field,
the validity of reciprocity will be dependent on the degree
that the required conditions are fulfilled.
A marine air gun should be reciprocal to a hydrophone.
A land-surface weight-drop source should be reciprocal to a vertical geophone.
But a buried explosive shot need not be reciprocal to
a surface vertical geophone
because the radiation patterns are different
and the positions are slightly different.
Under varying field conditions Fenati and Rocca found
that small positioning errors
in the placement of sources and receivers
can easily create discrepancies
much larger than the apparent reciprocity discrepancy.

\par
Geometrical complexity within the earth
does not diminish the applicability of the principle of linearity.
Likewise,
geometrical complexity does not reduce the applicability of reciprocity.
Reciprocity does not apply to sound waves in the presence of \bx{wind}.
Sound goes slower upwind than downwind.
But this effect of wind is much less than
the mundane irregularities of field work.
Just the weakening of echoes with time leaves noises that are not reciprocal.
Henceforth we will presume that reciprocity is generally applicable
to the analysis of reflection seismic data.

\section{SURVEY SINKING WITH THE DSR EQUATION}
\par
Exploding-reflector imaging will be replaced
by a broader imaging concept, 
{\em \bx{survey sinking}.}
A new equation called the
double-square-root (DSR) equation will be developed
to implement survey-sinking imaging. 
The function of the \bx{DSR equation}
is to downward continue an entire seismic survey,
not just the geophones but also the shots.
Peek ahead at equation (\ref{eqn:3p9})
and you will see an equation with two square roots.
One represents the cosine of the wave
{\em  arrival}
angle.
The other represents the
{\em  takeoff}
angle at the shot.
One cosine is expressed in terms of  $k_g$, the Fourier component
along the geophone axis of the data volume in $(s,g,t)$-space.
The other cosine, with  $k_s$, is the Fourier component
along the shot axis.
%\par
%Our field seismograms lie in the $(s,g)$-plane.
%To move onto
%the $(y,h)$-plane inhabited by seismic interpreters
%requires only a simple rotation.
%The data could be Fourier transformed with 
%respect to  $y$  and  $h$, for example.
%Then downward continuation would proceed
%with equation (\ref{eqn:3p17}) instead of equation (\ref{eqn:3p9}).
\subsection{The survey-sinking concept}
\par
The exploding-reflector concept has great utility because it
enables us to associate the seismic waves
observed at zero offset in many experiments
(say 1000 shot points) with the wave of a single thought experiment,
the exploding-reflector experiment.
The exploding-reflector analogy has a
few tolerable limitations connected with lateral velocity
variations and multiple reflections,
and one major limitation:
it gives us no clue as to
how to migrate data recorded at nonzero offset.
A broader imaging concept is needed.
\par
Start from field data where a 
survey line has been run along the  $x$-axis.
Assume there has been an infinite number of experiments,
a single experiment consisting of placing  a point
source or shot at  $s$  on the $x$-axis and 
recording echoes with geophones
at each possible location  $g$  on the $x$-axis.
So the observed data is an upcoming wave that is a two-dimensional
function of  $s$  and  $g$,  say  $P(s,g,t)$.
\par
Previous chapters have shown how to downward continue the
{\em  upcoming}
wave.
Downward continuation of the upcoming wave is really the
same thing as downward continuation of the geophones.
It is irrelevant for the continuation procedures where the
wave originates.
It could begin from an exploding reflector,
or it could begin at the surface, go down, and then be reflected back upward.
\par
To apply the imaging concept of survey sinking,
it is necessary to downward continue the sources as well as the geophones.
We already know how to downward continue geophones.
Since reciprocity permits interchanging geophones with shots,
we really know how to downward continue shots too.
\par
Shots and geophones may be downward continued to different levels,
and they may be at different levels during the process,
but for the final result they are only required to be at the same level.
That is, taking  $ z_s $  to be the depth of the shots
and  $ z_g $  to be the depth of the
geophones, the downward-continued survey will be required at all
levels  $z = z_s = z_g $.  
\par
The image of a reflector at  $(x,z)$  is
defined to be the strength and
polarity of the echo seen by the
closest possible source-geophone pair.
Taking the mathematical limit, this 
closest pair is a source and geophone located
together on the reflector.
The travel time for the echo is zero.
This survey-sinking concept of imaging is summarized by
\begin{equation}
\hbox{Image} (x,z)   \ \ \  = \ \ \ \hbox{Wave} (s=x,g=x,z,t=0)
\label{eqn:3p1}
\end{equation}
For good quality data, i.e. data that fits the
assumptions of the downward-continuation method,
energy should migrate to zero offset at zero travel time.
Study of the energy that doesn't do so should enable improvement of the model.
Model improvement usually amounts to improving the
spatial distribution of velocity.

\subsection{Survey sinking with the double-square-root equation}
\par
An equation was derived for paraxial waves.
The assumption of a
{\em  single}
plane wave means that the arrival time
of the wave is given by a single-valued  $t(x,z)$.
On a plane of constant  $z$, such as the earth's surface,
Snell's parameter  $p$  is measurable.
It is
\begin{equation}
{ \partial t   \over  \partial x } \  \  \eq \ 
{ \sin \, \theta   \over v }\  \eq \  p
\label{eqn:3p2a}
\end{equation}
In a borehole there is the constraint that measurements 
must be made
at a constant  $x$,  where the relevant measurement from an
{\em  upcoming}
wave would be
\begin{equation}
{ \partial t   \over  \partial z } \   \  \eq \ 
-\ { \cos \, \theta   \over v }\ \ \ \ = \quad
- \  \sqrt{ {1 \over  v^2 } \ -\ 
\left( {\partial t  \over \partial x} \  \right)^2 \ } \ 
\label{eqn:3p2b}
\end{equation}
Recall the time-shifting partial-differential equation and its
solution  $U$  as some arbitrary functional form  $f$:
\begin{eqnarray}
{ \partial U   \over  \partial z } \ \  \ \ &=&\ \ \ \  - \ 
{ \partial t   \over  \partial z } \ 
{ \partial U   \over  \partial t }
\label{eqn:3p3a}
\\
U \ \ \ \ &=&\ \ \ \ 
f \left( \ t \ -\  \int_0^z \  {\partial t  \over \partial z} \  dz \right)
\label{eqn:3p3b}
\end{eqnarray}
The partial derivatives
in equation (\ref{eqn:3p3a}) are taken to be at constant $x$,
just as is equation (\ref{eqn:3p2b}).
After inserting (\ref{eqn:3p2b}) into (\ref{eqn:3p3a}) we have
\begin{equation}
{ \partial U   \over  \partial z } \quad = \quad \sqrt{ {1 \over  v^2 } \ -\ 
\left( {\partial t  \over \partial x} \  \right)^2
 \ }\  { \partial U  
\over  \partial t }
\label{eqn:3p4a}
\end{equation}
Fourier transforming the wavefield over  $(x,t)$,   we
replace  $ \partial / \partial t $  by  $ -\,i \omega $.
Likewise, for the traveling wave
of the Fourier kernel  $ \exp (-\,i \omega t \ +\  ik_x x )$,
constant phase means that  ${\partial t}/{\partial x} \,=\, k_x / \omega $.
With this, (\ref{eqn:3p4a}) becomes
\begin{equation}
{ \partial U   \over  \partial z } \  \eq \  - \, i \omega  \  
\sqrt{
{1 \over  v^2 } \ -\  { k_x^2   \over \omega^2} \  } \  U
\label{eqn:3p4b}
\end{equation}
The solutions to (\ref{eqn:3p4b}) agree with those to the scalar wave equation
unless  $v$  is a function of  $z$,  in which case
the scalar wave equation has both upcoming and downgoing solutions,
whereas (\ref{eqn:3p4b}) has only upcoming solutions.
We
go into the lateral space
domain by replacing  $ i k_x $  by  $ \partial / \partial x $.
The resulting equation is useful for superpositions of many local plane waves
and for lateral velocity variations  $v(x)$.
\subsection{The DSR equation in shot-geophone space}
\par
Let the geophones descend a distance  $ dz_g $  into the earth.
The change of the travel time of the observed upcoming wave will be  
\begin{equation}
{{\partial t}\   \over {\partial z}_g} \  \eq \ 
- \   \sqrt{\  {1 \over  v^2 } \ -\ 
\left( {\partial t  \over \partial g} \, \right)^2 \  }
\label{eqn:3p5a}
\end{equation}
Suppose the shots had been let off at depth  $ dz_s $  instead of at  $z = 0$.
Likewise then,
\begin{equation}
{{\partial t}\   \over {\partial z}_s} \  \eq \ 
- \  \sqrt{\  {1 \over  v^2 } \ -\ 
\left( {\partial t  \over \partial s} \, \right)^2 \  }
\label{eqn:3p5b}
\end{equation}
Both (\ref{eqn:3p5a}) and (\ref{eqn:3p5b})
require minus signs because the travel time
decreases as either geophones or shots move down.
\par
Simultaneously downward project both the shots and
geophones by an identical vertical amount  $dz=dz_g = dz_s$.
The travel-time change is the sum
of (\ref{eqn:3p5a}) and (\ref{eqn:3p5b}), namely,
\begin{equation}
dt \  \eq \ 
{{\partial t}\ \over {\partial z}_g } \  dz_g \ +\  
{ {\partial t} \   \over {\partial z}_s } \  dz_s
\  \eq \ 
\left( {{\partial t}\   \over  {\partial z}_g } \ +\  { {\partial t} \ 
\over  {\partial z}_s} \right) \  dz \ \ \ \ \ \ \ 
\label{eqn:3p6}
\end{equation}
or
\begin{equation}
{\partial t  \over \partial z}  \  \eq \ 
- \  \left( \  \sqrt{ {1 \over  v^2 } \ -\ 
\left( {\partial t  \over \partial g} \, \right)^2 \  } \ +\ 
\sqrt{ {1 \over  v^2 } \ -\ 
\  {\left( {\partial t  \over \partial s} \, \right)^2} 
\  } \  \right)
\label{eqn:3p7}
\end{equation}
This expression for  ${\partial t}/{\partial z} $  may be substituted
into equation (\ref{eqn:3p3a}):
\begin{equation}
{ \partial U   \over  \partial z } \ \  =\ \ \ \  + \  \left(\   \sqrt{
{1 \over  v^2 } \ -\  \left( {\partial t  \over \partial g} \, \right)^2
\  } \ +\  \sqrt{ {1
\over  v^2 } \ -\  \left( {\partial t  \over \partial s} \, \right)^2 \ 
} \  \right) \  { \partial U   \over  \partial t }
\label{eqn:3p8}
\end{equation}
\par
Three-dimensional Fourier transformation converts
upcoming wave data  $u(t,s,g)$  to  $U( \omega , k_s , k_g )$.
Expressing equation (\ref{eqn:3p8}) in Fourier space gives
\begin{equation}
{ \partial U   \over  \partial z } \eq -\,i\,\omega \  \left[ \  
\sqrt{ {1 \over  v^2 } \ -\  \left( { k_g 
\over \omega }\, \right)^2 \  } \ +\  \sqrt{{1 \over  v^2 } \ -\  \left(
{ k_s   \over \omega }\, \right)^2 \  } \ \right] \  U
\label{eqn:3p9}
\end{equation}
Recall the origin of the two square roots in equation (\ref{eqn:3p9}).
One is the cosine of the arrival angle at the geophones
divided by the velocity at the geophones.
The other is the cosine of the takeoff angle at the shots
divided by the velocity at the shots.
With the wisdom of previous chapters
we know how to go into the lateral space domain by
replacing  $i k_g$  by  $\partial / \partial g$  and
$i k_s$  by  $\partial / \partial s$.
To incorporate lateral velocity variation  $v(x)$,
the velocity at the shot location must be distinguished
from the velocity at the geophone location.
Thus,
\par
\boxit{
	\begin{equation}
	{ \partial U   \over  \partial z }  \eq  
	\left[ \   \sqrt{\   \left( \ 
	{-\,i \omega\   \over v (g)} \, \right)^2 -\ 
	{\partial^2 \   \over \partial g^2} \, } \ +\  
	 \sqrt{\  \left( {-\,i \omega\   \over v (s)} \, \right)^2 -\ 
	{\partial^2 \   \over \partial s^2} \, } \  \right] \  U
	\label{eqn:3p10}
	\end{equation}
	}

\par
Equation (\ref{eqn:3p10}) is known as
the double-square-root (DSR) equation in shot-geophone space.
It might be more descriptive to call it the survey-sinking equation
since it pushes geophones and shots downward together.
Recalling the section on splitting and full separation
we realize that the two square-root operators are commutative
($v(s)$  commutes with  $ \partial / \partial g $),
so it is completely equivalent
to downward continue shots and geophones separately or together.
This equation will produce waves for the rays
that are found on zero-offset sections
but are absent
from the exploding-reflector model.
\subsection{The DSR equation in midpoint-offset space}
\par
By converting the DSR equation to midpoint-offset space
we will be able to identify the familiar zero-offset migration part
along with corrections for offset.
The transformation between  $(g,s)$  recording parameters
and  $(y,h)$  interpretation parameters is
\begin{eqnarray}
y \ \ \ \ &=&\ \ \ \  { g \ +\  s   \over 2 }
\label{eqn:3p11a}
\\
h \ \ \ \ &=&\ \ \ \  { g \ -\  s   \over 2 }
\label{eqn:3p11b}
\end{eqnarray}
Travel time  $t$  may be parameterized in $(g,s)$-space or $(y,h)$-space.
Differential relations for this 
conversion are given by the chain rule for derivatives:
\begin{eqnarray}
{\partial t  \over \partial g} \ \ \ \ &=&\ \ \ \ 
{\partial t  \over \partial y} \  {\partial y  \over \partial g} \ +\ 
{\partial t  \over \partial h} \  {\partial h   \over \partial g} \  \eq \ 
{1 \over 2 }\   \left( {\partial t  \over \partial y} \ +\ 
{\partial t  \over \partial h } \, \right)
\label{eqn:3p12a}
\\
{\partial t  \over \partial s} \ \ \ \ &=&\ \ \ \ 
{\partial t  \over \partial y} \  {\partial y  \over \partial s} \ +\ 
{\partial t  \over \partial h} \  {\partial h   \over \partial s} \  \eq \ 
{1 \over 2 }\   \left( {\partial t  \over \partial y} \ -\ 
{\partial t  \over \partial h } \, \right)
\label{eqn:3p12b}
\end{eqnarray}
\par
Having seen how stepouts transform from shot-geophone space
to midpoint-offset space,
let us next see that spatial frequencies transform in much the same way.
Clearly, data could be transformed from $(s,g)$-space
to $(y,h)$-space with (\ref{eqn:3p11a}) and (\ref{eqn:3p11b})
and then Fourier transformed to $ ( k_y , k_h ) $-space.
The question is then,
what form would the double-square-root equation (\ref{eqn:3p9})
take in terms of the spatial frequencies  $ ( k_y , k_h ) $?
Define the seismic data field in either coordinate system as
\begin{equation}
U ( s, g )\  \eq \ U'  ( y , h )
\label{eqn:3p13}
\end{equation}
This introduces a new mathematical function  $ U'  $  with the same
physical meaning as  $U$  but,
like a computer subroutine or function call,
with a different subscript look-up procedure
for  $(y,h)$  than for  $(s,g)$.
Applying the chain rule for partial differentiation to (\ref{eqn:3p13}) gives
\begin{eqnarray}
{ \partial U \over \partial s} \ \ \ \ &=&\ \ \ \ 
{ \partial y \over \partial s} \  { \partial U'    \over  \partial y \ } \ +\ 
{ \partial h \over \partial s} \  { \partial U'    \over  \partial h \ }
\label{eqn:3p14a}
\\
{ \partial U \over \partial g}\ \ \ \ &=&\ \ \ \ 
{ \partial y \over \partial g} \  { \partial U'    \over  \partial y \ }\ +\ 
{ \partial h \over \partial g}\  { \partial U'    \over  \partial h \ }
\label{eqn:3p14b}
\end{eqnarray}
and utilizing (\ref{eqn:3p11a}) and (\ref{eqn:3p11b}) gives
\begin{eqnarray}
{ \partial U   \over  \partial s }\ \ \ \ &=&\ \ \ \ 
{1 \over 2 }\ \left( { \partial U'    \over  \partial y \, }\ -\ 
{ \partial U'    \over  \partial h \, } \, \right)
\label{eqn:3p15a}
\\
{ \partial U   \over  \partial g }\ \ \ \ &=&\ \ \ \ 
{1 \over 2 }\  \left( { \partial U'    \over  \partial y \, }\ +\ 
{ \partial U'    \over  \partial h \, } \, \right)
\label{eqn:3p15b}
\end{eqnarray}
In Fourier transform space
where  $ \partial / \partial x $  transforms to  $ i k_x $,
equations (\ref{eqn:3p15a}) and (\ref{eqn:3p15b}),
when  $i$  and $U =  U'  $  are cancelled, become
\begin{eqnarray}
k_s\ \ \ \ &=&\ \ \ \ {1 \over 2 }\ ( k_y\ -\ k_h )
\label{eqn:3p16a}
\\
k_g\ \ \ \ &=&\ \ \ \ {1 \over 2 }\  ( k_y\ +\ k_h )
\label{eqn:3p16b}
\end{eqnarray}
Equations~(\ref{eqn:3p16a})
and (\ref{eqn:3p16b})
are Fourier representations of (\ref{eqn:3p15a}) and (\ref{eqn:3p15b}).
Substituting (\ref{eqn:3p16a}) and (\ref{eqn:3p16b})
into (\ref{eqn:3p9}) achieves the main purpose of this section,
which is to get the double-square-root migration equation
into midpoint-offset coordinates:
\begin{equation}
{\partial\   \over \partial z} \ U\ \ =\ \  -\,i \, 
{\omega \over v }\  \left[ \  \sqrt{\ 
1\,-\,\left( { v k_y \,+\, v k_h   \over  2\,\omega } \, \right)^2
\  } + 
\sqrt{\ 
1\,-\,\left( { v k_y \,-\, v k_h   \over  2\,\omega } \, \right)^2
\  } \  \right] \ U
\label{eqn:3p17}
\end{equation}
\par
Equation (\ref{eqn:3p17}) is the takeoff point
for many kinds of common-midpoint seismogram analyses.
Some convenient definitions that simplify its appearance are
\begin{eqnarray}
G\ \ \ \ &=&\ \ \ \  { v\ k_g   \over \omega }
\label{eqn:3p18a}
\\
S\ \ \ \ &=&\ \ \ \ { v\ k_s   \over \omega }
\label{eqn:3p18b}
\\
Y\ \ \ \ &=&\ \ \ \ { v\ k_y   \over  2\ \omega }
\label{eqn:3p18c}
\\
H\ \ \ \ &=&\ \ \ \ { v\ k_h   \over  2\ \omega }
\label{eqn:3p18d}
\end{eqnarray}
%Chapter \ref{paper:omk} showed that the quantity  $v \, k_x / \omega $  can
%be interpreted as the angle of a wave.
The new definitions  $S$  and  $G$  are the sines
of the takeoff angle and of the arrival angle of a ray.
When these sines are at their limits of  $ \pm 1 $  they refer
to the steepest possible slopes in $(s,t)$- or $(g,t)$-space.
Likewise, $Y$ may be interpreted as the dip of the data as seen
on a seismic section.
The quantity  $H$  refers to stepout observed on a common-midpoint gather.
With these definitions (\ref{eqn:3p17}) becomes slightly less cluttered:
\begin{equation}
\begin{tabular}{|c|}  \hline
 \\  $\displaystyle {\strut\partial\over\partial z}U
      \ =\ -\displaystyle {\strut i\omega\over v} 
                    \left(\sqrt{1-(Y+H)^2} +
 \sqrt{1-(Y-H)^2} \ \right) U$  \\
 \\    \hline
\end{tabular}
\label{eqn:3p19}
\end{equation}

%\par
%Most present-day before-stack migration procedures
%can be interpreted through
%equation (\ref{eqn:3p19}).
%Further analysis of it should explain
%the limitations of conventional processing procedures
%as well as suggest improvements in the procedures.

\begin{exer}

\item
Adapt equation (\ref{eqn:3p17}) to allow for a difference in velocity
between the shot and the geophone.
\item
Adapt equation (\ref{eqn:3p17}) to allow for downgoing pressure waves
and upcoming shear waves.
\end{exer}

\section{THE MEANING OF THE DSR EQUATION}
\par
The double-square-root equation
%contains most nonstatistical aspects of seismic data
%processing for petroleum prospecting.
%This equation, which was derived in the previous section,
is not easy to understand because it is an operator in a
four-dimensional space, namely,  $(z,s,g,t)$.
We will approach it through various applications, each of which is like a
picture in a space of lower dimension.
In this section lateral velocity variation will be neglected
(things are bad enough already!).
%Begin with
%\begin{eqnarray}
%{dU \over dz }\ \  &=&\ \  { -\,i \omega  \over v }\  \left( \  \sqrt
%{ 1\ -\ G^2 } \ \ +\ \  \sqrt { 1\ -\ S^2  } \  \right) \ U
%\label{eqn:4p1a}
%\\
%{dU \over dz }\ \  &=&\ \  { -\,i \omega   \over v }\   \left( \   \sqrt
%{1\ -\ (Y+H)^2 } \ \ +\ \  \sqrt { 1\ -\ (Y-H)^2 }\  \right) \ U
%\label{eqn:4p1b}
%\end{eqnarray}
%\subsection{Zero-offset migration (H = 0)}
\par
One way to reduce the dimensionality of (\ref{eqn:3p10})
is simply to set  $H \,=\, 0$.
Then the two square roots become the same, so that they can be
combined to give the familiar paraxial equation:
\begin{equation}
{dU \over dz }\  \eq 
{ -\,i  \omega } \ {2 \over v }\ \sqrt  { 1 \ -\ 
\  { v^2 \, k_y^2   \over  4 \, \omega^2 } } \ \  U
\label{eqn:4p2}
\end{equation}
In both places in equation (\ref{eqn:4p2}) where the rock velocity occurs,
the rock velocity is divided by 2.
Recall that the rock velocity needed to be halved in order for field
data to correspond to the exploding-reflector model.
So whatever we did by setting  $H \,=\, 0$,
gave us the same migration equation we used in chapter \ref{paper:dwnc}.
Setting  $H\,=\,0$  had the effect of making the survey-sinking concept
functionally equivalent to the exploding-reflector concept.

\subsection{Zero-dip stacking (Y = 0)}
\inputdir{.}
\par
When dealing with the offset  $h$  it is common to assume that
the earth is horizontally layered so that experimental results will be
independent of the midpoint  $y$.
With such an earth the Fourier transform of all data over  $y$  will
vanish except for  $ k_y = 0 $,
or, in other words, for  $Y  =  0$.
The two square roots in (\ref{eqn:3p10})
again become identical,
and the resulting equation is once more the paraxial equation:
\begin{equation}
{dU \over dz }\  \eq 
{ - \, i \omega } \ {2 \over v }\ \sqrt  { 1 \ -\ 
\  { v^2 \, k_h^2   \over  4\,\omega^2 } } \  \  U
\label{eqn:4p3}
\end{equation}
Using this equation to downward continue hyperboloids from the
earth's surface, we find the hyperboloids shrinking with depth, until the
correct depth where best focus occurs is reached.
This is shown in Figure~\ref{fig:dc2}.
\plot{dc2}{width=4in}{
	With an earth model of three layers,
	the common-midpoint gathers are three hyperboloids.
	Successive frames show downward continuation
	to successive depths where best focus occurs.
	}
\par
The waves focus best at zero offset.
The focus represents a downward-continued experiment,
in which the downward continuation has gone just to a reflector.
The reflection is strongest at zero travel time for
a coincident source-receiver pair just above the reflector.
Extracting the zero-offset value at  $t = 0$  and
abandoning the other offsets amounts
to the conventional procedure of summation along
a hyperbolic trajectory on the original data.
Naturally the summation can be expected to be best
when the velocity used for downward continuation
comes closest to the velocity of the earth.

\par
Actually, the seismic energy will not all go precisely to zero offset;
it goes to a focal region near zero offset.
A further analysis (not begun here) can analyze the focal region
to upgrade the velocity estimation.   Dissection of this focal region
can also provide information about reflection strength versus angle.


\subsection{Giving up on the DSR}
\par
The DSR operator
defined by (\ref{eqn:3p19})
is fun to think about,
but it doesn't really go
to any very popular place very easily.
There is a serious problem with it.
It is {\em not separable }
into a sum of an offset operator and a midpoint operator.
{\em  Nonseparable}
means that a Taylor series for (\ref{eqn:3p10})
contains terms like $ Y^2 \, H^2 $.
Such terms cannot be expressed as a function of  $Y$  plus a function of  $H$.
Nonseparability is a data-processing disaster.
It implies that migration and stacking must be done simultaneously,
not sequentially.
The only way to recover pure separability would be
to return to the space of  $S$  and  $G$.

\par
This chapter tells us that lateral velocity variation
is very important.
Where the velocity is known,
we have the DSR equation in shot-geophone
space to use for migration.
A popular test data set is called the Marmousi data set.
The DSR equation is particularly popular with it
because with synthetic data, the velocity really is known.
Estimating velocity $v(x,z)$ with
real data is a more difficult task,
one that is only crudely handled by
by methods in this book.
In fact, it is not easily done by the even best
of current industrial practice.

