\title{Field recording geometry}
\author{Jon Claerbout} 
\label{paper:fld}

\maketitle

\pagenumbering{arabic}  % hopefully, this causes page 1 to be on chapter 1 page.

%{\em \today .  This chapter is owned by JFC.}

\par
The basic equipment for reflection seismic prospecting
is a source for impulsive sound waves,
a geophone (something like a microphone),
and a multichannel waveform display system.
A survey line is defined along the earth's surface.
It could be the path for a ship,
in which case the receiver is called a hydrophone.
About every 25 meters the source is activated, and
the echoes are recorded nearby.
The sound source and receiver have almost no directional tuning capability
because the frequencies that penetrate the earth
have wavelengths longer than the ship.
Consequently, echoes can arrive from several directions at the same time.
It is the joint task of geophysicists and geologists
to interpret the results.
Geophysicists assume the quantitative, physical, and statistical tasks.
Their main goals, and the goal to which this
book is mainly directed, is to make good 
pictures of the earth's interior from the echoes.

\section{RECORDING GEOMETRY}

\par
Along the horizontal $x$-axis we define two points,
$s$, where the source (or shot or sender) is located,
and  $g$, where the geophone (or hydrophone or microphone) is located.
Then, define the \bx{midpoint}  $y$  between the
shot and geophone, and
define  $h$  to be half the horizontal \bx{offset}
between the shot and geophone:
%.ne 1.2i
\begin{eqnarray}
y\ \ &=&\ \ {g \ +\  s  \over 2 }
\label{eqn:0.1a}
\\
h\ \ &=&\ \ {g \ -\  s  \over 2 }
\label{eqn:0.1b}
\end{eqnarray}
The reason for using
{\em half}
the offset in the equations
is to simplify and symmetrize many later equations.
Offset is defined
with  $g - s$  rather than with  $s - g$  so that
positive offset means waves
moving in the positive  $x$ direction.
In the marine case, this means the ship is presumed to sail
negatively along the  $x$-axis.
In reality the ship may go either way,
and shot points may either increase or decrease as the survey proceeds.
In some situations you can clarify matters by setting the
field observer's shot-point numbers to negative values.
\par
Data is defined experimentally in the space of  $(s,\,g)$.
Equations~(\ref{eqn:0.1a}) and (\ref{eqn:0.1b})
represent a change of coordinates to the space of $(y,\,h)$.
Midpoint-offset coordinates are especially useful
for interpretation and data processing.
Since the data is also a function of the travel time  $t$,
the full dataset lies in a volume.
Because it is so difficult to make a satisfactory display of such a volume,
what is customarily done is to display slices.
The names of slices vary slightly from one company to the next.
The following names seem to be well known and clearly understood:
\begin{center}
\begin{tabular}{lp{2.75in}}
$(y,\ h = 0,\ t)$       & zero-offset section \\
$(y,\ h = h_{\rm min\,,}\  t)$ & near-trace section \\
$(y,\ h = \hbox{const} ,\  t)$ &        constant-offset section \\
$(y,\ h = h_{\rm max\,,}\  t)$ & far-trace section \\
$(y = \hbox{const} ,\  h,\  t)$ & common-midpoint gather \\
$(s = \hbox{const} ,\  g,\  t)$ & field profile (or common-shot gather) \\
$(s,\ g = \hbox{const} ,\  t)$ & common-geophone gather \\
$(s,\ g,\ t = \hbox{const} )$ & time slice \\
$(h,\ y,\ t = \hbox{const} )$ & time slice \\
%\end{tabbing}
\end{tabular}
\end{center}
\sx{time slice}
\sx{zero-offset section}
\sx{near-trace section}
\sx{section!near-trace}
\sx{section!zero-offset}
\par
\inputdir{XFig}
A diagram of slice names is in Figure~\ref{fig:sg}.
Figure~\ref{fig:cube} shows three slices from the data volume.
The first mode of display is ``engineering drawing mode.''
The second mode of display is on the faces of a cube.
But notice that although the data is displayed on the surface
of a cube, the slices themselves are taken from the interior of the cube.
The intersections of slices across one another are shown by dark lines.
\plot{sg}{width=6.0in}{
        Top shows field recording of marine seismograms from
        a shot at location  $s$  to geophones at locations labeled  $g$.
        There is a horizontal reflecting layer to aid interpretation.
        The lower diagram is called a \bx{stacking diagram}.
        (It is
        {\em not}
        a perspective drawing).
        Each dot in this plane depicts a possible seismogram.
        Think of time running out from the plane.
        The center geophone above (circled)
        records the seismogram (circled dot) that
        may be found in various geophysical displays.
        Lines in this $(s,g)$-plane are planes in the $(t,s,g)$-volume.
	Planes of various orientations have the names given in the text.
        }
%\activeplot{rick}{height=4.5in}{}{
%       Slices from a cube of data from the Grand Banks.
%       Left is ``engineering drawing'' mode.
%       At the right slices from within the cube are shown as faces on the cube.
%       (Data from Amoco.
%       Display via Rick Ottolini's movie program).
%       }
\inputdir{cube}
\plot{cube}{width=6.00in,height=8.0in}{
        Slices from within a cube of data.
        Top: Slices displayed as a mechanical drawing.
        Bottom: Same slices shown on perspective of cube faces.
%       (Push button to interact with Steve Cole's Cubeplot program.)
        }

\par
A common-depth-point (CDP) gather is defined
by the industry and by common usage
to be the same thing as a common-midpoint (CMP) gather.
But in this book
a distinction will be made.
A \bx{CDP gather} is a \bx{CMP gather} with its time
axis stretched according to some velocity model, say,
$$
(y=\hbox{const},\ h,\ \sqrt{t^2 - 4h^2 /v^2 }) \quad \quad
\hbox{common-depth-point gather}
$$
This offset-dependent stretching makes the time axis of the gather 
become more like a
{\em depth}
axis, thus providing the
{\em D} 
in CDP.
The stretching is called 
{\em 
\bx{normal moveout} correction
}
(NMO).
Notice that as the velocity goes to infinity, the amount of stretching
goes to zero.

\par
There are basically two ways to get two-dimensional information
from three-dimensional information.
The most obvious is to cut out the slices defined above.
A second possibility is to remove a dimension by summing over it.
In practice, the offset axis is the best candidate for summation.
Each CDP gather is summed over offset.
The resulting sum is a single trace.
Such a trace can be constructed at each midpoint.
The collection of such traces, a function of midpoint and time,
is called a CDP stack.
Roughly speaking, a CDP stack is like a zero-offset section,
but it has a less noisy appearance.
\par
The construction of a CDP stack requires that a numerical choice
be made for the moveout-correction velocity.
This choice is called the {\em stacking velocity.}
The stacking velocity may be simply someone's guess of the earth's velocity.
Or the guess may be improved by stacking with some trial velocities
to see which gives the strongest and least noisy CDP stack.
\par
Figures~\ref{fig:yc02} and~\ref{fig:yc20} show
typical marine and land \bx{profile}s
(common-shot gathers).

\inputdir{yc}
\sideplot{yc02}{height=4.0in,width=3.5in}{
        A seismic land profile.
        There is a gap where there are no receivers near the shot.
        You can see events of three different velocities.
%        Press button to change plot parameters.
        (Western Geophysical).
        }

\sideplot{yc20}{height=4.0in,width=3.5in}{
        A marine profile off the Aleutian Islands.
%        Press button to change plot parameters.
        (Western Geophysical).
        }

The land data has geophones on both sides of the source.
The arrangement shown is called an
{\em uneven \bx{split spread}.}
The energy source was a vibrator.
The marine data happens to nicely illustrate
two or three head waves.
\todo{define head waves?}
The marine energy source was an air gun.
These field profiles were each recorded with about 120 geophones.

\subsection{Fast ship versus slow ship}
\inputdir{sg}
For marine seismic data,
the spacing between shots $\Delta s$ is a function of the speed
of the ship and the time interval between shots.
Naturally we like $\Delta s$ small
(which means more shots)
but that means either the boat slows down,
or one shot follows the next so soon
that it covers up late arriving echos.
The geophone spacing $\Delta g$ is fixed
when the marine \bx{streamer} is designed.
Modern streamers are designed for more powerful
computers and they usually have smaller $\Delta g$.
Much marine seismic data is recorded with
$\Delta s = \Delta g$
and much is recorded with
$\Delta s = \Delta g/2$.
There are unexpected differences in what happens in the processing.
Figure~\ref{fig:geqs} shows
$\Delta s = \Delta g$,
and
Figure~\ref{fig:geq2s} shows
$\Delta s = \Delta g/2$. %
\sideplot{geqs}{width=2.5in}{
        $\Delta g = \Delta s$.
        The zero-offset section lies under the zeros.
        Observe the common midpoint gathers.
        Notice that even numbered receivers
        have a different geometry than odd numbers.
        Thus there are two kinds of CMP gathers
        with different values of the \bx{lead-in} $x_0$ = {\tt x0}
        }%
When $\Delta s = \Delta g$ there are some irritating complications
that we do not have for $\Delta s = \Delta g/2$.
When $\Delta s = \Delta g$, even-numbered traces
have a different midpoint than odd-numbered traces.
For a common-midpoint analysis,
the evens and odds require different processing.
The words ``\bx{lead-in}'' describe the distance ($x_0$ = {\tt x0})
from the ship to the nearest trace.
When $\Delta s = \Delta g$ the lead-in of a CMP gather
depends on whether it is made from the even or the odd traces.
In practice the lead-in is about $3\Delta s$.
Theoretically we would prefer no lead in,
but it is noisy near the ship,
the tension on the cable pulls it out of the water near the ship,
and the practical gains
of a smaller lead-in are evidently not convincing.

\sideplot{geq2s}{width=2.5in}{
        $\Delta g = 2\Delta s$.
        This is like Figure~\protect\ref{fig:geqs}
        with odd valued receivers omitted.
        Notice that each common-midpoint gather has the same geometry.
        }

\section{TEXTURE}

\par
Gravity is a strong force for the stratification of rocks,
and many places in the world rocks are laid down
in horizontal beds.
Yet even in the most ideal environment the bedding is
not mirror smooth; it has some
{\em \bx{texture}.}
We begin with synthetic data that mimics the most ideal environment.
Such an environment is almost certainly marine,
where sedimentary deposition can be slow and uniform.
The wave velocity will be taken to be constant,
and all rays will reflect as from horizontally lying mirrors.
Mathematically,
{\em texture}
is introduced by allowing the reflection coefficients
of the beds to be laterally variable.
The lateral variation is presumed to be a random function,
though not necessarily with a white spectrum.
Let us examine the appearance of the resulting field data.

\subsection{Texture of horizontal bedding, marine data}
\inputdir{synmarine}
Randomness is introduced into the earth with
a random function of midpoint  $y$  and depth  $z$.
This randomness is impressed on
some geological ``layer cake'' function of depth $z$.
This is done in the first half of program \texttt{Msynmarine} \vpageref{lst:Msynmarine}. %
\moddex{Msynmarine}{synthetic marine}{52}{74}{user/gee}
The second half of program \texttt{Msynmarine} \vpageref{lst:Msynmarine}
scans all shot and geophone locations and depths
and finds the midpoint,
and the reflection coefficient for that midpoint,
and adds it into the data at the proper travel time.

\par
There are two confusing aspects of subroutine \texttt{synmarine()} \vpageref{lst:synmarine}.
First, refer to figure~\ref{fig:sg} and notice that since the ship
drags the long cable containing the receivers,
the ship must be moving to the left, so data is recorded
for sequentially {\em decreasing} values of $s$.
Second, to make a continuous \bx{movie}
from a small number of frames,
it is necessary only to make the midpoint axis periodic,
i.e.~when a value of {\tt iy} is computed beyond the end of the axis
{\tt ny}, then it must be moved back an integer multiple of {\tt ny}.

\par
What does the final data space look like?
This question has little meaning until we decide how the three-dimensional
data volume will be presented to the eye.
Let us view the data much as it is recorded in the field.
For each shot point we see a frame 
in which the vertical axis is the travel time
and the horizontal axis is the distance from the ship down the towed
hydrophone cable.
The next shot point gives us another frame.
Repetition gives us the accompanying program
that produces a cube of data, hence a \bx{movie}.
This cube is synthetic data for the ideal marine environment.
And what does the \bx{movie} show?

\sideplot{synmarine}{width=3in}{
        Output from {\tt synmarine()} subroutine
        (with temporal filtering on the $t$-axis).
%        Press button for \bx{movie}.
        }

\inputdir{shotmovie}

\par
A single frame shows hyperbolas with imposed texture.
The \bx{movie} shows the texture
moving along each hyperbola to increasing offsets.
(I find that no sequence of still pictures can
give the impression that the \bx{movie} gives).
Really the ship is moving; the texture of
the earth is remaining stationary under it.
This is truly what most marine data looks like,
and the computer program simulates it.
Comparing the simulated data to real marine-data \bx{movie}s,
I am impressed by the large amount of random lateral variation required
in the simulated data to achieve resemblance to field data.
The randomness seems too great to represent lithologic variation.
Apparently it is the result of something not modeled.
Perhaps it results from our incomplete understanding of the
mechanism of reflection from the quasi-random earth.
Or perhaps it is an effect of the 
partial focusing of waves sometime after they
reflect from minor topographic irregularities.
A full explanation awaits more research.

\sideplot{shotmovie}{width=3in}{
        Press button for field data \bx{movie}.
        }

\subsection{Texture of land data: near-surface problems}
\par
Reflection seismic data recorded on land frequently displays randomness
because of the irregularity of the soil layer.
Often it is so disruptive that the seismic energy sources are
deeply buried (at much cost).
The geophones are too many for burial.
For most land reflection data, the texture caused by these near-surface
irregularities exceeds the texture resulting from the reflecting layers.
\par
To clarify our thinking, an ideal mathematical model will be proposed.
Let the reflecting layers be flat with no texture.
Let the geophones suffer random time delays of several time points.
Time delays of this type are called
{\em statics.}
Let the shots have random strengths.
For this \bx{movie}, let the data frames be
{\em 
common-midpoint gathers,
}
that is, let each frame show data in  $(h,t)$
-space at a fixed midpoint  $y$.
Successive frames will show successive midpoints.
The study of Figure~\ref{fig:sg} should convince you that the
traveltime irregularities associated with the geophones should
move leftward, while the amplitude irregularities associated with
the shots  should move rightward (or vice versa).
In real life, both amplitude and time anomalies are associated
with both shots and geophones.

\begin{exer}
\inputdir{XFig}

\item
Modify the program of Figure~\ref{fig:cube} to produce a movie
of synthetic %
{\em midpoint %
} gathers
with random shot amplitudes and random geophone time delays.
\sideplot{wirecube}{height=1.0in}{
        }
Observing this \bx{movie} you will note the perceptual problem of being
able to see the leftward motion along with the rightward motion.
Try to adjust anomaly strengths so that both left-moving
and right-moving patterns are visible.
Your mind will often see only one,
blocking out the other,
similar to the way you perceive a 3-D cube,
from a 2-D projection of its edges.
\item
Define recursive dip filters to pass and reject the
various textures of shot, geophone, and midpoint.
\end{exer}
%3.1.textemp
















