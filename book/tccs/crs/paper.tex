\hyphenation{Gel-chin-sky}
\hyphenation{Lan-da}

\published{Geophysical Prospecting, 61, 21-27 (2013)}

\renewcommand{\thefootnote}{\fnsymbol{footnote}}

\title{Non-hyperbolic common reflection surface}
\author{Sergey Fomel\footnotemark[1] and Roman Kazinnik\footnotemark[1]\footnotemark[2]}
\lefthead{Fomel \& Kazinnik}
\righthead{Nonhyperbolic CRS}

\address{
\footnotemark[1]Bureau of Economic Geology \\
John A. and Katherine G. Jackson School of Geosciences \\
The University of Texas at Austin \\
University Station, Box X \\
Austin, TX 78713-8924 \\
USA \\
sergey.fomel@beg.utexas.edu \\
\footnotemark[2]Currently ConocoPhillips \\
600 N. Dairy Ashford \\
Houston, TX 77079-1175 \\
USA \\
roman.only@gmail.com
}

\maketitle

\newpage

\begin{abstract}
  The method of common reflection surface (CRS) extend conventional
  stacking of seismic traces over offset to multidimensional stacking
  over offset-midpoint surfaces. We propose a new form of the stacking
  surface, derived from the analytical solution for reflection
  traveltime from a hyperbolic reflector. Both analytical comparisons
  and numerical tests show that the new approximation can be
  significantly more accurate than the conventional CRS approximation
  at large offsets or at large midpoint separations while using
  essentially the same parameters.
\end{abstract}

\section{Introduction}

Seismic data stacking is (together with deconvolution and migration)
one of the fundamental operations in seismic data analysis
\cite[]{yilmaz}. Conventional stacking operates on com\-mon-midpoint
(CMP) gathers and stacks traces after a hyperbolic moveout. The method
of multifocusing (MF), originally developed by
\cite{gelchinsky,gelchinsky2} and modified to the
common-reflection-surface (CRS) method by \cite{jager},
stacks data from multiple CMP locations. As a result, the
signal-to-noise ratio is improved considerably. Both MF and CRS
require estimation of multiple parameters in addition to the
conventional stacking velocity. These parameters correspond to the
slope and curvature of seismic events in the midpoint direction and
have physical interpretation in terms of wavefront slopes and
curvatures. Many successful applications of MF and CRS have been
reported in the literature
\cite[]{landa,GEO67-02-06390643,basin,topography,saudi,interp}.

The CRS method employs a multiparameter hyperbolic approximation of
the reflection traveltime surface \cite[]{tygel}. The hyperbolic
approximation can be justified from a truncated Taylor series
expansion of the squared traveltime around a reference ray. As such,
it is always accurate at small deviations from the central ray.
However, it loses its accuracy at large offsets or large midpoint
separations.

In this paper, we propose a new nonhyperbolic approximation. The form
of this approximation follows from an analytical equation for
reflection traveltime from a hyperbolic reflector.  The idea of
approximation reflection traveltimes by approximating reflector
surfaces was first proposed by \cite{revisited} and
\cite{sigma}. However, these publications did not
provide a closed-form representation of the stacking surface. By
analyzing the accuracy of the proposed nonhyperbolic approximation on
a number of examples, we show that the proposed approximation can
significantly extend the accuracy range of CRS.

\section{Hyperbolic and nonhyperbolic CRS}

If $P(t,m,h)$ represents the prestack seismic data as a function of
time $t$, midpoint $m$ and half-offset $h$, then conventional stacking
can be described as
\begin{equation}
\label{eq:stack}
S(t_0,m_0) = \int P\left(\theta(h;t_0),m_0,h\right)\,d h\;,
\end{equation}
where $S(t,m)$ is the stack section, and $\theta(h;t_0)$ is the
moveout approximation, which may take a form of a hyperbola
\begin{equation}
\label{eq:hyper}
\theta(h;t_0) = \sqrt{t_0^2 + \frac{4\,h^2}{v^2}}
\end{equation}
with $v$ as an effective velocity parameter or, alternatively, a more
complicated non-hyperbolic functional form, which involves other
parameters \cite[]{hyper}.

The MF or CRS stacking takes a different form,
\begin{equation}
\label{eq:stack2}
\widehat{S}(t_0,m_0) = \iint P\left(\widehat{\theta}(m-m_0,h;t_0),m,h\right)\,d m\,d h\;,
\end{equation}
where the integral over midpoint $m$ is typically carried out only over a limited
neighborhood of $m_0$. The multifocusing approximation of
\cite{gelchinsky} takes the form 
\begin{equation}
\label{eq:mf}
\widehat{\theta}_{MF}(d,h;t_0) = t_0 + T_{(+)}(d,h) + T_{(-)}(d,h)\;,
\end{equation}
where, in the notation of \cite{tygel2},
\begin{equation}
\label{eq:tpm}
T_{(\pm)} = \frac{\sqrt{1+2\,K_{(\pm)}\,(d \pm h)\,\sin{\beta} + K_{(\pm)}^2\,(d \pm h)^2}-1}{V_0\,K_{(\pm)}}\;, 
\end{equation}
\begin{equation}
\label{eq:kpm}
K_{(\pm)} = \frac{K_N \pm \sigma\,K_{NIP}}{1 \pm \sigma}\;, 
\end{equation}
and 
\begin{equation}
\label{eq:sigma}
\sigma(d,h) = \frac{h}{d + K_{NIP}\,\sin{\beta}(d^2-h^2)}\;.
\end{equation}
The four parameters $\{K_N,K_{NIP},\beta,V_0\}$ have clear physical
interpretations in terms of the wavefront and ray geometries
\cite[]{gelchinsky}. $V_0$ represents the velocity at the surface and is typically 
assumed known and constant around the central ray. One important property of the MF 
approximation is that, in a constant velocity medium with velocity $V_0$, it can 
accurately describe both reflections from a plane dipping interfaces and diffractions from point diffractors.

%Different choices of $\sigma$ can lead to more accurate approximations \cite[]{sigma}.

The CRS approximation \cite[]{jager} is
\begin{equation}
\label{eq:crs}
\widehat{\theta}_{CRS}(d,h;t_0) = \sqrt{F(d) + b_2\,h^2}\;,
\end{equation}
where $F(d) = (t_0 + a_1\,d)^2 + a_2\,d^2$, and the three parameters
$\{a_1,a_2,b_2\}$ are related to the multifocusing parameters as
follows:
\begin{eqnarray}
\label{eq:a1}
a_1 & = & \frac{2\,\sin{\beta}}{V_0}\;, \\
\label{eq:a2}
a_2 & = & \frac{2\,\cos^2{\beta}\,K_N\,t_0}{V_0}\;, \\
\label{eq:b2}
b_2 & = & \frac{2\,\cos^2{\beta}\,K_{NIP}\,t_0}{V_0}\;.
\end{eqnarray}
Equation~(\ref{eq:crs}) is equivalent to a truncated Taylor expansion
of the squared traveltime in equation~(\ref{eq:mf}) around $d=0$ and
$h=0$. In comparison with MF, CRS possesses a fundamental
simplicity, which makes it easy to extend the method to 3-D. However,
it looses the property of accurately describing diffractions in a
constant-velocity medium.

We propose the following modification of approximation~(\ref{eq:crs}):
\begin{equation}
\widehat{\theta}(d,h;t_0) = \sqrt{\frac{F(d) + c\,h^2
 + \sqrt{F(d-h)\,F(d+h)}}{2}}\;,
\label{eq:nhcrs}
\end{equation}
where $c=2\,b_2+a_1^2-a_2$.  Equation~(\ref{eq:nhcrs}), which we call
\emph{non-hyperbolic common reflection surface}, is derived in
Appendix A. A truncated Taylor expansion of the squared traveltime
from equation~(\ref{eq:nhcrs}) around $d=0$ and $h=0$ is equivalent to
equation~(\ref{eq:crs}).

There are two important special cases:
\begin{enumerate}
\item If $a_2=0$ or $K_N=0$, equation~(\ref{eq:nhcrs}) becomes equivalent to equation~(\ref{eq:crs}), with $F(d) = (t_0 + a_1\,d)^2$. 
In a constant-velocity medium, this case corresponds to reflection from a planar reflector.
\item If $a_2=b_2$ or $K_{NIP}=0$, equation~(\ref{eq:nhcrs}) becomes equivalent to 
\begin{equation}
\widehat{\theta}(d,h;t_0) = \frac{\sqrt{F(d-h)}+\sqrt{F(d+h)}}{2}\;.
\label{eq:dsr}
\end{equation}
In a constant-velocity medium, this case corresponds to a point diffractor.
\end{enumerate}

\subsection{3-D extension}

In the case of 3-D multi-azimuth acquisition, both $d$ and $h$ become
two-dimensional vectors. A natural way to extend
approximation~(\ref{eq:crs}) is to replace it with
\begin{equation}
\label{eq:crs3}
\widehat{\theta}_{CRS}(\mathbf{d},\mathbf{h};t_0) = 
\sqrt{F(\mathbf{d}) + \mathbf{h}^T\,\mathbf{B}_2\,\mathbf{h}}\;,
\end{equation}
where $F(\mathbf{d}) = (t_0 + \mathbf{d}^T\,\mathbf{a}_1)^2 +
\mathbf{d}^T\,\mathbf{A}_2\,\mathbf{d}$, $\mathbf{a}_1$ is a
two-dimensional vector, and $\mathbf{A}_2$ and $\mathbf{B}_2$ are
two-by-two symmetric matrices \cite[]{tygel}. A similar approach 
works for extending approximation~(\ref{eq:nhcrs}) to
\begin{equation}
\widehat{\theta}(\mathbf{d},\mathbf{h};t_0) = 
\sqrt{\frac{F(\mathbf{d}) + \mathbf{h}^T\,\mathbf{C}\,\mathbf{h} + 
\sqrt{F(\mathbf{d}-\mathbf{h})\,F(\mathbf{d}+\mathbf{h})}}{2}}\;,
\label{eq:nhcrs3}
\end{equation}
where
$\mathbf{C}=2\,\mathbf{B}_2+\mathbf{a}_1\,\mathbf{a}_1^T-\mathbf{A}_2$.
In the 3-D case, we have not found a simple connection between
approximation~(\ref{eq:nhcrs3}) and the analytical reflection
traveltime for a 3-D hyperbolic reflector.
 
\section{Accuracy comparisons}

\subsection{Analytical Example}
\inputdir{.}

\multiplot{2}{crs,ncrs}{height=0.4\textheight}{Relative error in
  prediction of reflection traveltime in the case of a circular
  reflector in a homogeneous medium using different approximations:
  (a) CRS, (b) non-hyperbolic CRS. The model parameters are $m=R=D$
  (see Figure~\ref{fig:crefl}.)}

The first example we use to compare the accuracy of different
approximations is that of a circular reflector under a homogeneous
overburden. As shown in Appendix B, the exact traveltime in this case
can be derived analytically in a parametric form. Obtaining a
non-parametric closed-form expression in this case would require a
solution of a high-order algebraic equation \cite[]{sigma}.
Figure~\ref{fig:crs,ncrs} compares the accuracy of CRS and nonhyperbolic
CRS approximations for a range of offsets and midpoints. We display
the relative absolute error as a function of offset to depth ratio and
midpoint separation to depth ratio for a range of offsets and
midpoints. The central midpoint is taken at the same horizontal
distance from the center of the circle as the depth.  The CRS
approximation~(\ref{eq:crs}) develops an error both at large offsets
and at large midpoint separations. The proposed non-hyperbolic CRS
approximation~(\ref{eq:nhcrs}) shows a significantly smaller error in
the full range of offsets and midpoints. In our experiments, the
multifocusing approximation~(\ref{eq:mf}) was even more accurate in
this example. However, because of its different functional form, we
focus our analysis on comparing CRS and non-hyperbolic CRS.

\subsection{Numerical Example}

\newcommand{\parvec}[1]{\textbf{#1}}

%%%%%%%%%%%% romankazinnik CIRCLE-1 SMALL : Rcre >> Rcee
%\multiplot*{4}{circle1-x0-3500-F2,circle1-x0-3500-F6,circle1-x0-3500-F4,circle1-x0-3500-F11}{width=\columnwidth}
%{Small-radius circular reflector: Kirchoff modeling time correction surface, and three difference surfaces: 
%Non-hyperbolic CRS, CRS, MF at the circular reflector parameters $R_{CRS}=\frac{1}{4}~km$, $R_{CEE}=TBD$ }
%%%%%%%%%%%% SIN
\inputdir{dome2}
%\plot{ref_sin_dip1}{width=\columnwidth}{Reflector-1: $z(x)=2+\frac{x}{2}+sin(\frac{3 (x - 4) }{2}$ }
\multiplot{2}{dome,pick}{width=0.6\textwidth}{(a) Synthetic velocity model with a Gaussian-shape reflector. (b) Modeled reflection traveltime.}
%
%\multiplot{3}{cmps_sin_grad_dip4-F10,cmps_sin_grad_dip4-F4,cmps_sin_grad_dip4-F8}{height=0.9\columnwidth}
%\multiplot{3}{cmpsfinal1sq-F10,cmpsfinal1sq-F4,cmpsfinal1sq-F8}{height=0.9\columnwidth}
%\multiplot{3}{cmpsfinal1sq-F10-1,cmpsfinal1sq-F4-1,cmpsfinal1sq-F8-1}{height=0.7\columnwidth}
%{Relative error in prediction of the  reflection traveltime in the case of Reflector-1 $V(x,z)=1000~ms$
%in a  homogeneous medium using different approximations: (a)  multifocusing $\sigma=0.12~\%$, 
%(b) CRS $\sigma=1.15~\%$, (c) non-hyperbolic CRS $\sigma=0.11~\%$.}
%
%\plot{crssrfsergey11}{width=\columnwidth}
%{Reflection traveltime in the case of Reflector-1 in a  homogeneous medium with CRS approximation (overlaid) }
%
%
%%%%%% SIN V_z=1/2 V_x=0
%\multiplot{3}{cmps_sin_grad_dip1-F10,cmps_sin_grad_dip1-F4,cmps_sin_grad_dip1-F8}{height=0.9\columnwidth}
%\multiplot{3}{cmpsfinal2sq-F10-1,cmpsfinal2sq-F4-1,cmpsfinal2sq-F8-1}{height=0.7\columnwidth}
%{Relative error in prediction of the  reflection traveltime in the case of Reflector-1
%in a  homogeneous medium $V(x,z)=1000+\frac{z}{2}~(ms)$ using different approximations: 
%(a)  multifocusing $\sigma= 2.12~\%$, 
%(b) CRS $\sigma=1.62~\%$, (c) non-hyperbolic CRS $\sigma= 0.78~\%$.}
%
%\plot{cmps_sin_grad_dip_sq1}{width=\columnwidth}{Model-1: Reflector-1 and linear velocity $V(x,z)=z$}
%\plot{cmps_sin_grad_dip1_hyper}{width=\columnwidth}{Model-1: Non-hyperbolic CRS}
%\plot{cmps_combined_kir_crs94}{width=\columnwidth}{Reflection traveltime in the case of Reflector-1
%in a  non-homogeneous medium $V(x,z)=1+z~(kmh)$ (left) vs. CRS approximation (right)}

\multiplot{2}{crs-err4,ncrs-err4}{width=0.6\textwidth}{Absolute error of
  (a) CRS approximation, (b) nonhyperbolic CRS approximation.  The
  reference midpoint is at 4~km.}

\multiplot{2}{crs-data4,ncrs-data4}{width=0.6\textwidth}
{Modeled synthetic data with overlaid traveltime surfaces from
least-squares fitting (colored curves). (a) CRS approximation. (b)
Nonhyperbolic CRS approximation. The
  reference midpoint is at 4~km.}

%
%\multiplot*{2}{circle1-x0-3500-F10,F10}{width=0.45\textwidth}{rrr}

%\multiplot{4}{marm,vel0,vel100,vel1000}{width=0.47\textwidth}{Velocity
%estimation for the Marmousi model: true velocity (a), initial velocity (b),
%velocity after 100 iterations (c), velocity after 1000 iterations (d).}
%
%Figure~\ref{fig:ref_dip}

In our next test, we generate a reflection traveltime surface by
modeling reflection seismic data from a Gaussian-shape reflector,
shown in Figure~\ref{fig:dome} by Kirchhoff modeling. The velocity
changes linearly with depth. We extract the traveltime surface, shown
in Figure~\ref{fig:pick}, and fit it with different approximations by
non-linear least-squares optimization. 

Our method for fitting the multivariate time-correction function
$\hat{\theta}(d,h;t_0,\parvec{a})$ (either CRS or nonhyperbolic
CRS) to a given experimental time data $t(d,h)$ by finding optimal
parameters set $\parvec{a}$ is a minimization approach, defined as
follows:
\begin{equation}
\label{eq:minimize}
\min_{ \parvec{a} } f( \parvec{a}  )\;,
\end{equation}
where $f$ is the squared-error sum:
\begin{equation}
\label{eq:minimize-function}
f( \parvec{a}) = \frac{1}{2}\,\sum\limits_d\,\sum\limits_h | \hat{\theta}(d,h; t_0, \parvec{a}) - t(d,h)|^2 \;.
\end{equation}
The gradient of the objective function is defined by
\begin{equation}
\label{eq:minimize-grad}
\frac{\partial f}{\partial {a_i}}
=
\sum\limits_d\,\sum\limits_h \left[ \hat{\theta}(d,h; t_0, \parvec{a}) - t(d,h) \right]\, 
%
\frac{\partial \hat{\theta}}{\partial a_i}\;.
\end{equation}
The partial derivatives of $\hat{\theta}$ are continuous
functions. Therefore, classical minimization methods, such as the
Gauss-Newton iteration, can be employed for finding a local minimum of
$f(\parvec{a})$ \cite[]{lsq}. 
In the case of CRS, the solution is
unique because of the linear dependence of $\hat{\theta}^2$ on
parameters.

The absolute error of CRS and nonhyperbolic CRS approximations, for a
range of offsets and midpoints, is plotted in
Figure~\ref{fig:crs-err4,ncrs-err4}. The nonhyperbolic CRS error is
significantly smaller for a wide range of offsets and midpoints, which
extends the applicability of the approximation.  To effect of
different approximations is shown additionally in
Figure~\ref{fig:crs-data4,ncrs-data4}, which displays the modeled
synthetic data with overlaid CRS and nonhyperbolic CRS approximations.
The average relative error in different approximations, for the
selected range of offsets and midpoints, is 1.24~\% for the case of
CRS and 0.41~\% for the case of nonhyperbolic CRS. The average
  absolute error is 25~ms for the case of nonhyperbolic CRS and 8~ms
  for the case of nonhyperbolic CRS.

%
%In the CRS fitting one obtains the solution $\parvec{x}$ analytically, since the CRS squared travel time is hyperbolic.
%Let 
%$T(\parvec{x},m,h)=T_{CRS}^2 = t_0^2 + a\cdot m+b \cdot m^2+c \cdot h^2$ and $T_e(m,h)=T^2_{Kir}(m,h)$.
%The corresponding Hessian matrix is constant and this leads to a unique solution:
%\[
%\parvec {grad} T
%= 
%\left(
%\frac{\partial{T}}{\partial a},
%\frac{\partial{T}}{\partial b},
%\frac{\partial{T}}{\partial c}
%\right)
%= (m, m^2, h^2) \; \Rightarrow
%\frac{\partial^2 T}{\partial x_i \partial x_j} = 0 \; ,\Rightarrow
%\]
%%
%\[
%\frac{\partial^2 f (\parvec{x},m, h) }{\partial x_i \partial x_j} 
%
%= 
%
%-2 \int \int \left(
%(T_e - T)\frac{\partial^2 T}{\partial x_i \partial x_j} 
%- 
%\frac{\partial T}{\partial x_j} \cdot \frac{\partial T}{\partial x_i} 
%\right) dm dh =
%\]
%\[
%= 2 \int \int \frac{\partial T}{\partial x_j} \frac{\partial T}{\partial x_i}
%dm dh \;.
%\]
%The CRS parameters vector $\parvec{x}$ is obtained by differentiating the second-order Taylor series equation:
%\[
%f( \parvec{x} ) = f( \parvec{x}_0 ) + 
%\parvec{grad} f \cdot ( \parvec{x} - \parvec{x}_0 ) + 
%\frac{1}{2} ( \parvec{x} - \parvec{x}_0 )^{t} H ( \parvec{x} - \parvec{x}_0 ) \Rightarrow
%\]
%\[
%\parvec{x}
%=
%\parvec{x}_0
%- H^{-1} \parvec{grad} f \; ,
%\]
%where $H$ is the Hessian matrix and $\parvec{x}_0$ is an initial parameters set.
%We employ classical minimization methods for fitting both in the Non-hyperbolic CRS and Multifocusing, 
%where the initial parameters set $\parvec{x}_0$ 
%is obtained with the  conversion formulas from the CRS fitting solution.
%In  our numerical experiments, gradient methods found sufficient minimum 
%($\|\nabla f\|_{\infty} < 10^{-5}$).


\section{Conclusions}

We have presented the non-hyperbolic common reflection surface, a new
approximation for prestack reflection traveltimes. Non-hyperbolic CRS
uses the same set of parameters as the hyperbolic CRS but in a
different functional form, which can make the approximation
significantly more accurate in a large range of offsets and
midpoints. The proposed approximation is derived from the analytical
expression of the reflection traveltime in the case of a hyperbolic
reflector in a constant velocity medium.

Why use a hyperbolic reflector? A special property of this reflector
is that it reduces to a plane reflector or a point diffractor with a
special choice of parameters. ($z_0=0$ or $\alpha=\pi/2$
respectively). Thus, it encompasses two particularly important special
cases. 

Numerical experiments show that the new approximation can be
significantly more accurate than the conventional hyperbolic CRS while
using essentially the same set of parameters. The multifocusing
approximation can be even more accurate but uses a different set of
parameters, which makes it more difficult to extend it to 3-D.

\section{Acknowledgments}

The first author is grateful to Emil Blias, Evgeny Landa, Tijmen
Jan Moser, Alexey Stovas, and Martin Tygel for inspiring discussions.

This publication is authorized by the Director, Bureau of Economic
Geology, The University of Texas at Austin.

\appendix
\section{Apendix A: Hyperbolic reflector}

In this appendix, we reproduce the derivation of an analytical
expression for reflection traveltime from a hyperbolic reflector in a
homogeneous velocity model \cite[]{hyper}. Similar derivations apply
to an elliptic reflector and were used previously in the theory of 
offset continuation \cite[]{stovas,GEO68-02-07180732}.

Consider the source point $s$ and the receiver point $r$ at the
surface $z=0$ above a 2-D constant-velocity medium and a hyperbolic
reflector defined by the equation 
\begin{equation}
  \label{eq:hypref}
  z(x) = \sqrt{z_0^2 +
    x^2\,\tan^2{\alpha}}\;.
\end{equation}
The reflection traveltime as a function of the reflection point
location $y$ is
\begin{equation}
  \label{eq:tofy}
  t = \frac{\sqrt{(s-y)^2 + z^2(y)} + \sqrt{(r-y)^2+z^2(y)}}{V}\;.
\end{equation}
According to Fermat's principle, the traveltime should be stationary
with respect to the reflection point~$y$:
\begin{eqnarray}
\nonumber
0 = \frac{\partial T}{\partial y} & = &
\frac{y-s + y\,\tan^2{\alpha}}{V\,\sqrt{(s-y)^2 + z_0^2 + y^2\,\tan^2{\alpha}}} \\
& + &
\frac{y-r + y\,\tan^2{\alpha}}{V\,\sqrt{(r-y)^2 + z_0^2 + y^2\,\tan^2{\alpha}}}\;.
\label{eq:fermat}
\end{eqnarray}
Putting two terms in equation~(\ref{eq:fermat}) on the different sides
of the equation, squaring them, and reducing their difference to a
common denominator, we arrive at the following quadratic equation with
respect to $y$:
\begin{eqnarray}
\nonumber
  y^2\,(s+r)\,\tan^2{\alpha} & - & 2\,y\,\left(s\,r\,\sin^2{\alpha} - z_0^2\right) \\
  & - & z_0^2\,(s+r)\,\cos^2{\alpha} = 0\;.
  \label{eq:y2}
\end{eqnarray}
Only one of the two branches of the solution
\[
  y =
  \frac{z_0^2\,(s+r)\,\cos^2{\alpha}}{z_0^2 - s\,r\,\sin^2{\alpha} +
    \sqrt{(z_0^2+s^2\,\sin^2{\alpha})\,(z_0^2+r^2\,\sin^2{\alpha})}}
  \label{eq:ys}
\]
has physical meaning. Substituting this solution into
equation~(\ref{eq:tofy}), we obtain, after a number of algebraic
simplifications,
\begin{eqnarray}
\nonumber
  t^2 & = & \frac{2 z_0^2 + s^2 + r^2 - 2\,s\,r\,\cos^2{\alpha}}{V^2} \\
  & + &	\frac{2\,\sqrt{(z_0^2+s^2\,\sin^2{\alpha})\,(z_0^2+r^2\,\sin^2{\alpha})}}{V^2}\;.
	\label{eq:t}
\end{eqnarray}
Making the variable change in equation~(\ref{eq:t}) from $s$ and $r$
to midpoint and half-offset coordinates $m$ and $h$ according to
$s=m-h=m_0+d-h$, $r=m+h=m_0+d+h $, we transform this equation to
form~(\ref{eq:nhcrs}), where the following correspondence between
parameters is applied:
\begin{eqnarray}
\label{eq:z0} 
z_0^2 & = & \frac{t_0^2\,a_2}{(a_1^2+a_2)\,(a_1^2+b_2)}\;, \\
 \label{eq:m} 
m_0 & = & \frac{t_0\,a_1}{a_1^2+a_2}\;, \\
\label{eq:alpha} 
\sin^2{\alpha} & = & \frac{a_1^2 + a_2}{a_1^2 + b_2}\;, \\
\label{eq:v} 
V^2 & = & \frac{4}{a_1^2 + b_2}\;.
\end{eqnarray}
The inverse relationships are given by
\begin{eqnarray}
\label{eq:t0i} 
t_0 & = & \frac{2\,\sqrt{m_0^2\,\sin^2{\alpha} + z_0^2}}{V}\;, \\
 \label{eq:a1i} 
a_1 & = & \frac{2\,m_0\,\sin^2{\alpha}}{V\,\sqrt{m_0^2\,\sin^2{\alpha} + z_0^2}}\;, \\
\label{eq:a2i} 
a_2 & = & \frac{4\,z_0^2\,\sin^2{\alpha}}{V^2\,\left(m_0^2\,\sin^2{\alpha} + z_0^2\right)}\;, \\
\label{eq:b2i} 
b_2 & = & \frac{4\,\left(m_0^2\,\sin^2{\alpha}\,\cos^2{\alpha}+z_0^2\right)}{V^2\,\left(m_0^2\,\sin^2{\alpha} + z_0^2\right)}\;.
\end{eqnarray}

The connection with the multifocusing parameters is summarized in
  Table~\ref{tbl:hyper} for the general case and three special cases
  (a plane dipping reflector, a flat reflector, and a point
  diffractor). The first two special cases turn the nonhyperbolic CRS
  equation into the hyperbolic form~(\ref{eq:crs}). The last case
  turns it into the double-square-root form~(\ref{eq:dsr}).

\tabl{hyper}{Multifocusing parameters for a hyperbolic reflector in a homogeneous medium ($V_0=V$). The general case and three special cases.}{
    \begin{center}
      \begin{tabular}{|c||c|c|c|c|} \hline
         & $t_0$ & $K_{NIP}$ & $K_N$ & $\sin{\beta}$ \\
        \hline\hline
        Hyperbolic reflector &
        \rule[-10mm]{0mm}{21mm} $\frac{2\,\sqrt{m_0^2\,\sin^2{\alpha} + z_0^2}}{V}$ &
        $\frac{1}{\sqrt{m_0^2\,\sin^2{\alpha} + z_0^2}}$ &
        $K_{NIP}\,\frac{z_0^2\,\sin^2{\alpha}}{m_0^2\,\sin^2{\alpha}\,\cos^2{\alpha}+z_0^2}$ &
        $\frac{m_0\,\sin^2{\alpha}}{\sqrt{m_0^2\,\sin^2{\alpha} + z_0^2}}$ \\
        \hline
        Plane dipping reflector & & & & \\
        $z_0=0$ &
        \raisebox{3ex}[0pt]{$\frac{2\,m_0\,\sin{\alpha}}{V}$} &
        \raisebox{3ex}[0pt]{$\frac{1}{m_0\,\sin{\alpha}}$} &
        \raisebox{3ex}[0pt]{$0$} &
        \raisebox{3ex}[0pt]{$\sin{\alpha}$} \\
        \hline
        Flat reflector & & & & \\
        $\alpha=0$ &
        \raisebox{3ex}[0pt]{$\frac{2\,z_0}{V}$} &
        \raisebox{3ex}[0pt]{$\frac{1}{z_0}$} &
        \raisebox{3ex}[0pt]{$0$} &
        \raisebox{3ex}[0pt]{$0$} \\
        \hline
        Point diffractor & & & & \\
        $\alpha=\pi/2$ &
        \raisebox{3ex}[0pt]{$\frac{2\,\sqrt{m_0^2 + z_0^2}}{V}$} &
        \raisebox{3ex}[0pt]{$\frac{1}{\sqrt{m_0^2 + z_0^2}}$} &
        \raisebox{3ex}[0pt]{$K_{NIP}$} &
        \raisebox{3ex}[0pt]{$\frac{m_0}{\sqrt{m_0^2 + z_0^2}}$} \\
        \hline
      \end{tabular}
    \end{center}
  }

\appendix
\section{Appendix B: Circular reflector}
\inputdir{XFig}

In the case of a circular (cylindrical or spherical) reflector in a
homogeneous velocity model, the closed-form
analytical solution is complicated, because it involves a
solution of a high-order polynomial equation \cite[]{sigma}. However,
the traveltime surface can be easily described analytically by
parametric relationships \cite[]{mirror}.

\plot{crefl}{width=0.7\textwidth}{Reflection from a circular reflector
  in a homogeneous velocity model (a scheme).}

Consider the reflection geometry shown in
Figure~\ref{fig:crefl}. According to the trigonometry of the
reflection triangles, the source and receiver positions can be
expressed as
\begin{eqnarray}
\label{eq:xs}
s & = & R\,\sin{\gamma} + (L - R\,\cos{\gamma})\,\tan{(\gamma-\theta)}\;, \\
\label{eq:xr}
r & = & R\,\sin{\gamma} + (L - R\,\cos{\gamma})\,\tan{(\gamma+\theta)}\;,
\end{eqnarray}
where $R$ is the reflector radius, $D$ is the minimum reflector depth, $L=D+R$,
$\gamma$ is the reflector dip angle at the reflection point, and
$\theta$ is the reflection angle. Correspondingly, the midpoint and
half-offset coordinates are expressed as
\begin{eqnarray}
\nonumber
m & = & R\,\sin{\gamma} + (L - R\,\cos{\gamma})\,\frac{\cos{\gamma}\,\sin{\gamma}}{\cos^2{\theta} - \sin^2{\gamma}}\;, \\
\label{eq:mh}
h & = & (L - R\,\cos{\gamma})\,\frac{\cos{\theta}\,\sin{\theta}}{\cos^2{\theta} - \sin^2{\gamma}}\;,
\end{eqnarray}
and the reflection traveltime is expressed as
\begin{eqnarray}
\nonumber
t & = & \frac{L - R\,\cos{\gamma}}{V}\,\left[\frac{1}{\cos{(\gamma-\theta)}} + \frac{1}{\cos{(\gamma+\theta)}}\right] \\
& = & 2\,\frac{L - R\,\cos{\gamma}}{V}\,\frac{\cos{\gamma}\,\cos{\theta}}{\cos^2{\theta} - \sin^2{\gamma}}\;,
\label{eq:tcirc}
\end{eqnarray}
where $V$ is the medium
velocity. Equations~(\ref{eq:mh}-\ref{eq:tcirc}) define the reflection
traveltime surface $t(m,h)$ parametrically via the dependence
$\{m(\gamma,\theta),h(\gamma,\theta),t(\gamma,\theta)\}$. The
connection with the multifocusing parameters is given by
\begin{eqnarray}
\label{eq:ct0} 
t_0 & = & \frac{2\,\left(\sqrt{m_0^2 + L^2}-R\right)}{V} \\
\label{eq:cknip}
K_{NIP} & = & \frac{1}{\sqrt{m_0^2 + L^2}-R}\;,   \\
\label{eq:ckn}
K_N & = & \frac{1}{\sqrt{m_0^2 + L^2}}\;,   \\
\label{eq:cbeta}
\sin{\beta} & = & \frac{m_0}{\sqrt{m_0^2 + L^2}}\;,
\end{eqnarray}
and $V_0=V$.

\bibliographystyle{seg}
\bibliography{SEG,crs}
