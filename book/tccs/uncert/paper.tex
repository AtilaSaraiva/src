\published{Journal of Applied Geophysics, 101, 27–30, (2014)}

\title{Structural uncertainty of time-migrated seismic images}

\renewcommand{\thefootnote}{\fnsymbol{footnote}}

\author{Sergey Fomel\footnotemark[1] and Evgeny Landa\footnotemark[2]}

\address{
\footnotemark[1]Bureau of Economic Geology, \\
Jackson School of Geosciences \\
The University of Texas at Austin \\
University Station, Box X \\
Austin, TX 78713-8924 \\
USA \\
\footnotemark[2]OPERA \\
Bat. IFR, Rue Jules Ferry \\
64000 Pau \\
France}

\maketitle

\begin{abstract}
  Structural information in seismic images is uncertain. The main
  cause of this uncertainty is uncertainty in velocity
  estimation. We adopt the technique of velocity continuation for
  estimating velocity uncertainties and corresponding structural
  uncertainties in time-migrated images. Data experiments indicate
  that structural uncertainties can be significant even when both
  structure and velocity variations are mild.
\end{abstract}


\section{Introduction}

The usual outcome of seismic data processing is an image of the
subsurface \cite[]{yilmaz}. In the conventional data analysis
workflow, the image is passed to the seismic interpreter, who makes
geological interpretation, often by extracting structural information,
such as positions of horizons and faults in the image. Hidden in this
process is the fact that structural information is fundamentally
uncertain, mainly because of uncertainties in estimating seismic
velocity parameters, which are required for imaging. Apart from the
trivial case of perfectly flat seismic reflectors, which are positioned
correctly in time even when incorrect stacking or migration
velocities are used, seismic images can be and usually are structurally
distorted because of inevitable errors in velocity estimation
\cite[]{hamburg}.

Understanding and quantifying uncertainty in geophysical information
can be crucially important for resource exploration \cite[]{caers}. The
issue of structural uncertainty in seismic images was analyzed
previously by \cite{GEO67-03-08400852} and
\cite{GEO70-02-S39S42}. \cite{GEO66-04-12081216} studied the impact of
velocity uncertainties on migrated images and AVO
attributes. \cite{SEG-2004-21882191,SEG-2004-21922195} studied the
influence of velocity and anisotropy uncertainties on structural
uncertainties.

In this paper, we propose a constructive procedure for estimating the
degree of structural uncertainty in seismic images obtained by
prestack time migration. The basis for our approach is the method of
velocity continuation \cite[]{me,hubral,GEO68-05-16621672,GEO68-05-16501661,will}, which
constructs seismic images by an explicit continuation in migration
velocity. Velocity continuation generalizes the earlier ideas of
residual and cascaded migrations
\cite[]{SEG-1982-S1.4,GEO50-01-01100126,GEO52-05-06180643}. In addition to
generating accurate time-migration images, it provides a direct access
to measuring the structural dependence (sensitivity) of these images
on migration velocities. We define structural uncertainty as a 
product of velocity picking uncertainty and structural sensitivity.

We use a simple data example to illustrate our approach and to show
that structural uncertainty can be significant even when both
structure and velocity variations are mild. Although the proposed
approach is directly applicable only to prestack time migration, it
can be extended in principle to prestack depth migration using
velocity-ray approaches for extending the velocity continuation
concept \cite[]{GEO67-01-01260134,iversen,duchkov}.

\section{Velocity continuation and structural sensitivity}
\inputdir{beivc}

\plot{vlf}{width=\textwidth}{Velocity continuation cube for prestack time migration of the Gulf of Mexico dataset.}

\plot{npk}{width=\textwidth}{Migration velocity picked from velocity continuation.} 

\plot{bei-agc}{width=\textwidth}{Seismic prestack time-migration image generated by velocity continuation.}

Velocity continuation is defined as the process of image
transformation with changes in migration velocity
\cite[]{me,GEO68-05-16501661}. Its output is equivalent to the output of repeated migrations with different migration velocities \cite[]{GEO66-06-16991713} but
produced more efficiently by using propagation of images in
velocity \cite[]{hubral}. If we denote the output of velocity
continuation as $C(t,x,v)$, where $t$ and $x$ are time-migration
coordinates and $v$ is migration velocity, the time-migrated image is
simply
\begin{equation}
\label{eq:image}
I(t,x) = C(t,x,v_M(t,x))\;,
\end{equation}
where $v_M(t,x)$ is the picked migration
velocity. Figure~\ref{fig:vlf} shows the velocity continuation cube
$C(t,x,v)$ generated from a benchmark 2-D dataset from the Gulf of
Mexico \cite[]{bei}. Migration velocity $v_M(t,x)$ picked from the semblance
analysis is shown in Figure~\ref{fig:npk}. The velocity variations
reflect a dominantly vertical gradient typical for the Gulf of Mexico and only
mild lateral variations, which justifies the use of prestack time
migration. The corresponding migration image $I(t,x)$ is shown in
Figure~\ref{fig:bei-agc} and exhibits mild, nearly-horizontal
reflectors and sedimentary structures.

\multiplot{2}{slice,tslice}{height=0.35\textheight}{Common-image gather (a) and time slice (b) from velocity continuation with overlaid time-migration velocity.}

\multiplot{2}{bei-dtdv,bei-dxdv}{height=0.35\textheight}{Estimated structural sensitivity in time (a) and lateral position (b) with respect to velocity.}


The structural sensitivity of an image can be described through
derivatives $\partial t/\partial v$ and $\partial x/\partial v$, which
correspond to slopes of events in the $C(t,x,v)$ volume evaluated at
$v=v_M(t,x)$. These slopes are easy to measure experimentally from the
$C(t,x,v_M)$ volume, using, for example, the plane-wave destruction
algorithm \cite[]{GEO67-06-19461960,fpwd,opwd}. Figure~\ref{fig:slice,tslice}
shows one common-image gather $G(t,v)=C(t,x_0,v)$ for
$x_0=10\,\mbox{km}$ and the time slice $S(x,v)=C(t_0,x,v)$ for
$t_0=2\,\mbox{s}$. Measuring the slope of events $\partial t/\partial
v$ in this gather and evaluating it at the picked migration velocity
produces the slope
\begin{equation}
\label{eq:pt}
p_t(t,x) = \left.\frac{\partial t}{\partial v}\right|_{v=v_M(t,x)}\;.
\end{equation}
We measure the slope $p_x(t,x)$ analogously by evaluating local slopes
in time slices of constant $t$:
\begin{equation}
\label{eq:px}
p_x(t,x) = \left.\frac{\partial x}{\partial v}\right|_{v=v_M(t,x)}\;.
\end{equation}
Figure~\ref{fig:bei-dtdv,bei-dxdv} shows the estimated $p_t$ and $p_x$, which comprise the structural sensitivity of our image.

Theoretically, structural sensitivity can be inferred from the zero-offset velocity ray equations \cite[]{GEO46-05-07170733,GEO68-05-16501661}
\begin{eqnarray}
\label{eq:dtray}        
\displaystyle \frac{d t}{d v} & = & v_M\,t\,t_x^2 = \frac{t}{v_M}\,\tan^2{\theta}\;, \\
 \label{eq:dxray}        
\displaystyle \frac{d x}{d v} & = & -2\,v_M\,t\,t_x = -2\,t\frac{t}{v_M}\,\tan^2{\theta}\;, 
\end{eqnarray}
where $t_x$ corresponds to the slope of the reflector, and $\theta$ is
the reflector dip angle. According to
equations~\ref{eq:dtray}-\ref{eq:dxray}, the reflector dip is the
dominant factor in structural sensitivity.

\section{Uncertainty in velocity picking}

\multiplot{2}{scan,scan2}{height=0.35\textheight}{Velocity scan at
  10~km image gather. The curve in (a) corresponds to the
  automatically picked velocity trend.  The curves in (b) identify an
  approximate range of velocity uncertainty around the picked trend.}

Figure~\ref{fig:scan} shows a semblance scan produced in the process
of velocity continuation. A common procedure in migration velocity
analysis is picking a velocity trend from the semblance, either
manually or automatically. In this example, we use automatic picking
with the algorithm described by \cite{avo}.

While picking may select the most probable velocity function, its
probability is less than 100\%. If we view normalized semblance as a
probability distribution and determine a confidence interval 
corresponding roughly to one standard deviation, it provides an
approximate range of uncertainty in velocity determination. This range
is shown in Figure~\ref{fig:scan2} and computed according to
\begin{equation}
\label{eq:std}
\delta v(t,x) = \sqrt{\displaystyle \frac{\int\limits_{v_{min}}^{v_{max}} \left[v-v_M(t,x)\right]^2\,S(t,x,v)\,dv}{\int\limits_{v_{min}}^{v_{max}} S(t,x,v)\,dv}}\;,
\end{equation}
where $S(t,x,v)$ is the semblance volume that corresponds to
$C(t,x,v)$, {and $[v_{min},v_{max}]$ is the full range of
  velociies.  The interpretation of semblance picks as
  probability distributions is heuristic but helps in quantifying
  uncertainties in velocity picking.

\section{Structure uncertainty}

\plot{arr}{width=\textwidth}{Estimated structural uncertainty in the
  seismic image from Figure~\ref{fig:bei-agc}, displayed as
  displacements.}

\plot{hors}{width=\textwidth}{Estimated structural uncertainty in the
  seismic image from Figure~\ref{fig:bei-agc}, displayed as horizon
  uncertainties.}

Putting structural sensitivity and velocity uncertainty together, we
can define \emph{structural uncertainty} simply as their product: 
\begin{eqnarray}
\label{eq:dt}
\delta t & = & \displaystyle \frac{\partial t}{\partial v}\,\delta v\;, \\
\label{eq:dx}
\delta x & = & \displaystyle \frac{\partial x}{\partial v}\,\delta v\;. 
\end{eqnarray}
The uncertainty $\{\delta t,\delta x\}$ is the main output of our
study. It is shown as small line segments in Figure~\ref{fig:arr}
and as uncertainty in horizons in Figure~\ref{fig:hors}.  The
estimated uncertainty varies inside the image space and generally
increases with depth. It is surprisingly large, given the mild
variations in structure and velocity. We believe that, when making
quantitative estimates related to structural interpretation, it is
important to take this kind of uncertainty into account.

When converting seismic images from time to depth, it is also
  important to realize that the time-to-depth conversion itself is a
  mathematically ill-posed problem \cite[]{cameron} and has
  its own significant uncertainties.

%\plot{slice2}{width=\textwidth}{Velocity continuation.}

%\section{Structural uncertainty and velocity-independent imaging}

\section{Conclusions}

We have estimated structural uncertainty in seismic time-domain images
simultaneously with performing prestack time migration. To accomplish
this task, we projected the uncertainty in migration velocity picking
into the structural uncertainty by measuring the structural
sensitivity of seismic images to velocity. The latter measure is
provided by velocity continuation, which serves both as an imaging
tool and as a tool for sensitivity analysis. Field data examples show
that structural uncertainties can be significant even in the case of
mild structures and slow velocity variations. Taking these
uncertainties into account should improve the practice of seismic
structural interpretation by making it more compliant with
risk-management assessment in reservoir characterization.

\bibliographystyle{seg}
\bibliography{SEG,uncert}

