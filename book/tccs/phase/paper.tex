\published{Interpretation, 2, no. 1, SA49-SA56, (2014)}

\hyphenation{Van-der-Baan}

\title{Local skewness attribute as a seismic phase detector}

\renewcommand{\thefootnote}{\fnsymbol{footnote}}
\newcommand{\arccosh}{\mbox{arccosh}}

\lefthead{Fomel \& van der Baan}
\righthead{Local skewness}
\ms{INT-2013}

\author{Sergey Fomel\footnotemark[1]
and Mirko van der Baan\footnotemark[2]}

\address{
\footnotemark[1]Bureau of Economic Geology, \\
John A. and Katherine G. Jackson School of Geosciences \\
The University of Texas at Austin \\
University Station, Box X \\
Austin, TX 78713-8924 \\
USA \\
sergey.fomel@beg.utexas.edu \\
\footnotemark[2]Department of Physics \\
University of Alberta \\
CEB 11322 - 89 Avenue \\
Edmonton, Alberta T6G 2G7 \\
Canada \\
Mirko.VanderBaan@ualberta.ca
}

\maketitle

\begin{abstract}
  We propose a novel seismic attribute, local skewness, as an indicator
  of localized phase of seismic signals. The proposed attribute
  appears to have a higher dynamical range and a better stability than
  the previously used local \old{varimax norm} \new{kurtosis}. Synthetic and real data
  examples demonstrate the effectiveness of local skewness in
  detecting and correcting time-varying, locally-observed phase of
  seismic signals.
\end{abstract}

\section{INTRODUCTION}

Wavelet phase is an important characteristic of seismic
signals. Physical causal systems exhibit minimum-phase behavior
\cite[]{robinson}. For purposes of seismic interpretation, it is
convenient to deal with zero-phase wavelets with minimum or maximum
amplitudes centered at the horizons of interest \new{because it leads
to the highest resolution as well as more accurate estimates of both
reflection times and spacings \cite[]{Schoenberger74}}. Zero-phase
correction is \new{therefore} a routine procedure applied to seismic
images before they are passed to the interpreter \cite[]{brown}.

It is important to make a distinction between phases of
\emph{propagating} and \emph{locally observed} wavelets
\cite[]{mirko3,mirko4}. The former is the physical wavelet that propagates
through the Earth, thereby sampling the local geology. It is subject
to geometric spreading,
% apparent and intrinsic 
attenuation, and concomitant dispersion. The
latter is the wavelet as observed at a certain point in space and
time. Its immediate shape results from its interaction (convolution)
with the reflectivity of the Earth and the current shape of the
propagating wavelet. For instance, a thin layer with
\old{an outstanding} \new{a positive change in} seismic impedance has opposite polarities of seismic
reflectivities at the top and the bottom interface, which make it act
like a derivative filter and generate a wavelet with the locally
observed phase subjected to a $90^{\circ}$ rotation
\cite[]{GEO70-03-C7C15}. In the absence of well log information, it is
usually difficult to separate unambiguously the locally observed phase
from the phase of the propagating wavelet. Nevertheless, measuring the
local phase can provide a useful attribute for analyzing seismic data
\cite[]{mirko2,envelope,mirko4,pierre}.

\cite{GEO52-01-00510059} proposed a method of phase detection based
on maximization of the varimax norm or kurtosis as an objective
measure of zero-phaseness. By rotating the phase and measuring the
kurtosis of seismic signals, one can detect the phase rotations
necessary for zero-phase correction \cite[]{mirko1}. \cite{mirko2}
applied local kurtosis, a smoothly nonstationary measure
\cite[]{diffr}, and demonstrated its advantages in measuring phase
variations as compared with kurtosis measurements in sliding
windows. Local kurtosis is an example of a local attribute
\cite[]{attr} defined by utilizing regularized least-squares
inversion.

In this paper, we revisit the problem of phase estimation and propose
a novel attribute, \emph{local skewness}, as a phase
detector. Analogous to local kurtosis, local skewness is defined using
local similarity measurements via regularized least squares. This
attribute is maximized when the locally observed phase is close to
zero. Advantages of the new attribute are a higher dynamical range and
a better stability, which make it suitable for \old{implementing
automatic} \new{picking} phase corrections. Using synthetic and
field-data examples, we demonstrate properties and applications of the
proposed attribute.

\section{Localized phase estimation}

%Figure~\ref{fig:rv} illustrates the fundamental difference between the
%propagating wavelet and the locally observed one. In this synthetic example,
%borrowed from \cite{GEO26-02-01380157}, a zero-phase seismic wavelet
%is convolved with a reflectivity series to form a synthetic seismic
%trace. While isolated jumps in acoustic impedance produce isolated
%spikes in reflectivity, thin layers create a reflectivity series with
%its own phase and can rotate the apparent phase of the seismic trace
%by as much as $90^{\circ}$.

%\inputdir{imped} 

%\plot{rv}{width=\columnwidth}{Synthetic impedance log (top) and
%  corresponding seismic trace (bottom). The seismic trace, formed by
%  convolving reflectivity with a zero-phase Ricker wavelet, exhibits a
%  variable phase: $0^{\circ}$ for an isolated interface and
%  $90^{\circ}$ for thin layers, possibly with a tuning effect on the
%  amplitudes. Reproduced from \cite{GEO26-02-01380157}.}

Our goal is to estimate the time-variant, localized phase from seismic
data. What objective measure can indicate that a certain signal has a
zero phase? One classic measure is the varimax norm or kurtosis
\cite[]{Wiggins78,GEO52-01-00510059,Whit88}. Varimax is defined as
\begin{equation}
  \label{eq:focus} \phi[\mathbf{s}] = \frac{\displaystyle N\,\sum_{n=1}^N
  s_n^4}{\displaystyle \left(\sum_{n=1}^{N} s_n^2\right)^2}\;,
\end{equation}
where $\mathbf{s}=\{s_1,s_2,\ldots,s_N\}$ represents a vector of seismic amplitudes 
inside a window of size $N$. Varimax is simply related to kurtosis of
zero-mean signals.

The statistical rationale behind the Wiggins algorithm and its
variants is that convolution of any filter with a time series that is
white with respect to all statistical orders makes the outcome more
Gaussian.  The optimum deconvolution filter is therefore one that
ensures the deconvolution output is maximally
non-Gaussian \cite[]{Dono81}. The constant-phase assumption made
by \cite{GEO52-01-00510059} and \cite{Whit88} reduces the number of
free parameters to one, thereby stabilizing performance compared with
the Wiggins method. Wavelets derived in seismic-to-well ties often
have a near-constant phase, thus justifying this assumption.

Noticing that the correlation
coefficient of two sequences $a_n$ and $b_n$ is defined as
\begin{equation}
  \gamma[\mathbf{a},\mathbf{b}] = {\frac{\displaystyle \sum_{n=1}^N a_n\,b_n}{\displaystyle \sqrt{\sum_{n=1}^N a_n^2\,\sum_{n=1}^N b_n^2}}}
  \label{eq:c}
\end{equation}
and the correlation of $a_n$ with a constant is
\begin{equation}
  \gamma[\mathbf{a},\mathbf{1}] = {\frac{\displaystyle \sum_{n=1}^N a_n}{\displaystyle \sqrt{N\,\sum_{n=1}^N a_n^2}}}\;,
  \label{eq:c2}
\end{equation}
\cite{diffr} interpreted the \old{varimax} \new{kurtosis} measure in
equation~\ref{eq:focus} as the inverse of the squared correlation
coefficient between $s_n^2$ and a constant, $\phi[\mathbf{s}] =
1/\gamma^2[\mathbf{s}^2,\mathbf{1}]$. Well-focused or zero-phase
signals exhibit low correlation with a constant and correspondingly
higher \old{varimax} \new{kurtosis} (Figure~\ref{fig:kur}). This provides an alternative
interpretation to the goal of making the deconvolution outcome
maximally non-Gaussian for desired phase estimation. Note that
equation~\ref{eq:c} is usually applied to zero-mean sequences ${\bf
a}$ and ${\bf b}$. This is neglected in the derivation of
expression~\ref{eq:c2}.

\inputdir{signal}

\plot{kur}{width=\columnwidth}{(a) Squared $0^{\circ}$-phase Ricker wavelet
  compared with a constant. (b) Squared $90^{\circ}$-phase Ricker
  wavelet compared with a constant. The $90^{\circ}$-phase signal has
  a higher correlation with a constant and correspondingly a lower
  kurtosis.}

\plot{sqr}{width=\columnwidth}{(a) $0^{\circ}$-phase Ricker wavelet
  compared with its square. (b) $90^{\circ}$-phase Ricker wavelet
  compared with its square. The $0^{\circ}$-phase has a stronger
  correlation with its square and correspondingly a higher skewness.}

In this paper, we suggest a different measure, skewness, for
measuring the apparent phase of seismic signals. Skewness of a
sequence $s_n$ is defined
as \cite[]{bulmer}
\begin{equation}
\label{eq:skew}
\kappa[\mathbf{s}] = \frac{\displaystyle \frac{1}{N}\,\sum\limits_{n=1}^N
  s_n^3}{\displaystyle \left(\frac{1}{N}\,\sum\limits_{n=1}^{N} s_n^2\right)^{3/2}}\;.
\end{equation}
In statistics, skewness is used for measuring asymmetry of probability distributions.
Simple algebraic manipulations show that skewness squared can be
represented as
\begin{equation}
\label{eq:kappa2}
\kappa^2[\mathbf{s}] = \frac{\displaystyle \left(\sum\limits_{n=1}^N
  s_n^2\cdot s_n\right)^2}{\displaystyle \sum\limits_{n=1}^N s_n^4\,\sum\limits_{n=1}^N s_n^2}\,\frac{\displaystyle \sum\limits_{n=1}^N s_n^4\,\sum\limits_{n=1}^N 1^2}{\displaystyle \left(\sum\limits_{n=1}^N s_n^2\right)^2} =
\frac{\gamma^2[\mathbf{s}^2,\mathbf{s}]}{\gamma^2[\mathbf{s}^2,\mathbf{1}]}
= \phi[\mathbf{s}]\,\gamma^2[\mathbf{s}^2,\mathbf{s}]\;.
\end{equation}
In other words, squared skewness is equal to the \old{varimax} \new{kurtosis} measure
modulated by the squared correlation coefficient between the signal
and its square. Zero-phase signals
tend to exhibit higher correlation with the square and correspondingly
higher skewness (Figure~\ref{fig:sqr}). \new{Following experiments with synthetic and field data, we find it advantageous to use sometimes 
the inverse skewness
\begin{equation}
\label{eq:iskew}
\displaystyle \frac{1}{\kappa^2[\mathbf{s}]} = \frac{\gamma^2[\mathbf{s}^2,\mathbf{1}]}{\gamma^2[\mathbf{s}^2,\mathbf{s}]}\;.
\end{equation}}

%\plot{env}{width=\columnwidth}{(a) $0^{\circ}$-phase Ricker wavelet
%  compared with its envelope. (b) $90^{\circ}$-phase Ricker wavelet
%  compared with its envelope. The $0^{\circ}$-phase has a higher
%  correlation with the envelope.}

\old{Contrary to the varimax norm} \new{Unlike kurtosis} which measures non-Gaussianity, skewness
is related to asymmetry. Whereas convolution of two non-Gaussian
sequences makes the outcome more Gaussian, convolution of two
asymmetric series becomes more symmetric. Both phenomena are a
consequence of the central limit theorem. A zero-phase wavelet is more
compact than a nonzero phase one \cite[]{Schoenberger74}, and
therefore also more asymmetric. Skewness-based criteria can thus
detect the appropriate wavelet phase by applying a series of constant
phase rotations to the data and then evaluating the angle that
produces the most skewed distribution.

The two measures do not necessarily agree with one another, which is
illustrated in
Figures~\ref{fig:ricker-all,ricker-sq-corr,ricker-sq-corr-inv}
and~\ref{fig:ricker2-all,ricker2-sq-corr,ricker2-sq-corr-inv}. For an
isolated positive spike convolved with a compact zero-phase wavelet,
the two measures agree in the picking of the zero-phase result as
having both a high kurtosis and a high skewness
(Figure~\ref{fig:ricker-all,ricker-sq-corr,ricker-sq-corr-inv}). For a
slightly more complex case of a double positive spike convolved with
the same wavelet
(Figure~\ref{fig:ricker2-all,ricker2-sq-corr,ricker2-sq-corr-inv}),
the two measures disagree: kurtosis picks a signal rotated by
$90^{\circ}$ whereas skewness picks the original signal. \old{This
example shows that, although kurtosis might be better at picking the
true signal phase of the propagating wavelet, skewness tends to pick
better focused signals.} Note that, in both examples, skewness
exhibits a significantly higher dynamical range, which makes it more
suitable for picking optimal phase rotations.

\inputdir{zero}
\multiplot{3}{ricker-all,ricker-sq-corr,ricker-sq-corr-inv}{width=0.3\textwidth}{(a)
  Ricker wavelet rotated through different phases. (b) Skewness (solid
  line) and \old{varimax} \new{kurtosis} (dashed line) as functions of the phase rotation
  angle. (c) Inverse skewness (solid line) and inverse \old{varimax} \new{kurtosis} (dashed
  line) as functions of the phase rotation angle. The two measures
  agree in picking the signal at $0^{\circ}$ and $180^{\circ}$. Note
  the higher dynamical range of skewness.}

\multiplot{3}{ricker2-all,ricker2-sq-corr,ricker2-sq-corr-inv}{width=0.3\textwidth}{(a)
  Ricker wavelet convolved with a double impulse and rotated through
  different phases. (b) Skewness (solid line) and \old{varimax} \new{kurtosis} (dashed
  line) as functions of the phase rotation angle. (c) Inverse skewness
  (solid line) and inverse \old{varimax} \new{kurtosis} (dashed line) as functions of the
  phase rotation angle. The two measures disagree by $90^{\circ}$ in
  picking the optimal \old{signal} \new{phase}. The skewness attribute picks a better
  focused signal.}

\section{Defining skewness as a local attribute}

The method of local attributes \cite[]{attr} is a technique for
extending stationary or instantaneous attributes to smoothly varying
or nonstationary attributes by employing a regularized least-squares
formulation. In particular, \old{one number for the} \new{the scalar}
correlation coefficient $\gamma$ in equation~\ref{eq:c} is replaced
with a \old{set of numbers, $c_n$,} \new{vector, $\mathbf{c}$,} defined
as a componentwise product of vectors $\mathbf{c}_1$ and
$\mathbf{c}_2$, where
\begin{eqnarray}
  \label{eq:c1s}
  \mathbf{c}_1 & = & 
  \left[\lambda^2\,\mathbf{I} + 
    \mathbf{S}\,\left(\mathbf{A}^T\,\mathbf{A} - \lambda^2\,\mathbf{I}\right)\right]^{-1}\,
  \mathbf{S}\,\mathbf{A}^T\,\mathbf{b}\;, \\
  \label{eq:c2s}
  \mathbf{c}_2 & = & 
    \left[\lambda^2\,\mathbf{I} + 
      \mathbf{S}\,\left(\mathbf{B}^T\,\mathbf{B} - \lambda^2\,\mathbf{I}\right)\right]^{-1}\,
    \mathbf{S}\,\mathbf{B}^T\,\mathbf{a}\;.
\end{eqnarray}
In equations~\ref{eq:c1s}-\ref{eq:c2s}, $\mathbf{a}$ and $\mathbf{b}$
are vectors composed of $a_n$ and $b_n$, respectively; $\mathbf{A}$
and $\mathbf{B}$ are diagonal matrices composed of the same elements;
and $\mathbf{S}$ is a smoothing operator. \new{We use triangle
smoothing \cite[]{pvi} controlled by specifying the smoothing radius,
which can be different in vertical and horizontal directions.}

Regularized inversion
appearing in equations~\ref{eq:c1s} \new{and} \ref{eq:c2s} is justified in the
method of shaping regularization \cite[]{shape}. The corresponding
local similarity attribute has been used previously to align
multicomponent and time-lapse images \cite[]{attr,timelapse,hesam,rui}, to
detect focusing of diffractions \cite[]{diffr}, to enhance
stacking \cite[]{illum,stack}, to create time-frequency
distributions \cite[]{timefreq}, and to perform zero-phase correction
with local kurtosis \cite[]{mirko2}. In this paper, we apply it to
zero-phasing \new{seismic data} using local skewness.

\inputdir{nonstat}
\multiplot{3}{trace,tskew,trace0}{width=0.3\textwidth}{(a) Input
  synthetic trace with variable-phase events. (b) Inverse local
 skewness as a function of the phase rotation angle. Red colors
 correspond to high inverse similarity. (c) Synthetic trace after
 non-stationary rotation to zero phase \new{using picked phase}.}

We illustrate the proposed zero-phase correction procedure in
Figure~\ref{fig:trace,tskew,trace0}. The input synthetic trace
contains a set of Ricker wavelets with a gradually variable phase
(Figure~\ref{fig:trace}). We start with a number of phase rotations
with different angles, each time computing the local skewness. The
result of this step is displayed in Figure~\ref{fig:tskew} and shows a
clear high-similarity trend. After picking the trend, adding
$90^{\circ}$ to it, and performing the corresponding nonstationary
trace rotation, we end up with the phase-corrected trace, shown in
Figure~\ref{fig:trace0}. All the original phase rotations are clearly
detected and removed. The \old{length} \new{radius} of the
regularization smoothing in this example was 100 samples or 0.4~s.

\section{Application example}

\inputdir{.}
\plot{barnett-strat}{height=0.6\textheight}{Stratigraphic column of the Fort Worth Basin where the Boonsville
dataset is located, after~\cite{Pollastro07}. Karstification in the
Ellenburger Carbonates has caused local sags in the overlying Barnett
and Bend Conglomerate formations, creating reservoir
compartmentalization.}

\inputdir{boon}
\plot{boon}{width=\columnwidth}{Input data: a section of the
  Boonsville dataset}.
\plot{bfoc}{width=\columnwidth}{Inverse local
  skewness as a function of the phase rotation angle, with application
  to the section from Figure~\ref{fig:boon}. Red colors correspond to
  high inverse similarity.}
%\plot{bpik}{width=\columnwidth}{Nonstationary phase rotation picked from the similarity analysis shown in Figure~\ref{fig:bfoc}} 

The input dataset for our field-data example is the Boonsville dataset
from the Fort\old{h} Worth Basin in North-Central Texas,
USA \cite[]{Hardage1,Hardage2}. The formations of interest are the
Ellenburger Carbonates, the Barnett Shale and the Bend Conglomerates
(see Figure~\ref{fig:barnett-strat} for a stratigraphic column). The
Ellenburger Carbonates are of Ordovician age. Their karstification due
to post-Ellenburger carbonate dissolution and subsequent cavern
collapses has created sags in the overlying formations, affecting
sedimentation patterns and structures in the overlying Barnett Shales
and Bend Conglomerates. The collapse features look like vertical
chimneys with roughly circular cross-sections, extending up to
600-760~m above the Ellenburger Carbonates \cite[]{Hardage1}, sometimes
even reaching into the Strawn Group above the Bend Conglomerates.


The Barnett Shales are of Mississippian age. They are the target of
much current exploitation in Texas as these are tight-shale
reservoirs \cite[]{Pollastro07}. Zones with karst-induced cavern
collapses form a drilling and completion hazard for the mainly
horizontal drilling programs in these tight-shale reservoirs
and \old{most} \new{must} be mapped. They may \old{also} affect local
fracture densities and thus permeabilities and reservoir drainage
positively but can also lead to fluid barriers due to reservoir
compartmentalization.

The shallower clastic Bend Conglomerates are of Middle Pennsylvanian
(Atokan) age. The formation has a thickness of 300-360~m in this area
with depths between 1370 to 1830~m. It was targeted throughout the
1980s and 1990s as it contains several gas and oil-bearing reservoirs
in a stacked fashion. \cite{Hardage1,Hardage2} describe how the
karstification has greatly impacted the system tracts and
sedimentation patterns in the Bend formation which were characterized
by low accommodation space. Resulting reservoir compartmentalization
is a significant challenge for this formation and has also affected
the reflector character. Reflection near the base of this formation
display both reflector weakening and sometimes even polarity reversals
in areas depressed due to local sagging. Acquisition and processing of
this dataset \old{is} \new{are} described by \cite{Hardage1}. A
stacked section is shown in Figure~\ref{fig:boon}.  A zero-phase
correction has been applied to the data but has left regions with
variable localized phase.

Our processing sequence is similar to the one used in
Figure~\ref{fig:trace,tskew,trace0}. First, we apply a number of phase
rotations with different angles and compute local skewness for each
rotation. The regularization lengths in this examples were 500 samples
or 0.5~s in time and 50 \old{samples} \new{traces} in space. The
result is displayed in Figure~\ref{fig:bfoc}. Next, we apply automatic
picking with the algorithm described by \cite{avo} to extract the
nonstationary phase rotation that maximizes the local skewness.
%The result is displayed in Figure~\ref{fig:bpik}.
Finally, the phase correction is applied to the data, with the result
displayed in Figure~\ref{fig:boon-win}. A zoomed-in comparison shows
the effects of non-stationary phase correction: rotating major seismic
events to zero degrees and improving their continuity. These effects
can be useful both for improving structural interpretation and for
improved matching of seismic data and well logs.

\plot{boon-win}{width=\columnwidth}{Zoomed-in comparison of the
  data before phase correction (a) and after phase correction
  (b). Nonstationary phase correction helps in identifying significant
  horizons and increasing their resolution in time.}

We applied the local phase detection to the 3-D volume in a window
centered on the target horizon (Figure~\ref{fig:boonm}). The estimated
local phase variation along the target horizon is shown in
Figure~\ref{fig:mpik}. Comparing the amplitude before and after phase
correction (Figure~\ref{fig:ampl,ampr}), we observe a noticeable
improvement in event continuity. Once the processing and
interpretation are done on the zero-phase-corrected volume, it is easy
to restore the original phase by applying the inverse phase rotation.

\inputdir{boon3}

\plot{boonm}{width=\textwidth}{Boonsville dataset windowed around the target horizon.}

\plot{mpik}{width=\textwidth}{Local phase variation along the target horizon estimated from the data shown in Figure~\ref{fig:boonm}.}

\multiplot{2}{ampl,ampr}{width=0.45\textwidth}{Amplitude along the target horizon before and after phase rotation.}

\section{Discussion}

The data example as well as the previous case studies by \cite{mirko3}
underline how analysis of the local phase can be used as a
complementary attribute to spectral decomposition to highlight
variations in wavelet character. There are two complimentary
applications, namely analysis of the propagating and locally observed
wavelets. 
\begin{comment}
The propagating wavelet is the physical wavelet traversing
the Earth, which is affected by attenuation and dispersion; the
locally observed one is generated by interference between neighboring
reflections and thus most affected by local reflector
character \cite[]{mirko3}. 
\end{comment}
The targeted wavelet type is determined by the chosen regularization
length: the propagating wavelet is estimated by using long temporal
regularization lengths, and the locally observed one from shorter
lengths. The underlying assumption is that, for long regularization
lengths, variations in the local geology are averaged out, revealing
only the propagating wavelet. In this paper, we used relatively short
regularization lengths as the aim is to highlight changes in the local
reflection character.

\new{Well-log analyses have
demonstrated that the Earth's reflectivity series is
non-Gaussian \cite[]{WaHo86} and, to first order,
white \cite[]{WaHo85}. In addition, impedances tend to increase with
depth, hence positive reflection coefficients are slightly more likely
than negative one, producing an asymmetric reflectivity distribution.}
Statistically, the skewness-based criterion assumes that the Earth's
reflectivity series are white and asymmetric. This is in contrast to
\old{the varimax norm} \new{kurtosis} used previously
by \cite{mirko2}, which assumes a non-Gaussian reflectivity
series. Both the non-Gaussianity and asymmetry assumptions seem
\old{therefore} warranted \old{Both criteria} \new{but} may 
fail if the local reflectivity
series becomes respectively purely Gaussian or symmetric.
\begin{comment}
This may also explain why both methods do not always agree
(Figures~\ref{fig:ricker-all,ricker-sq-corr,ricker-sq-corr-inv}
and~\ref{fig:ricker2-all,ricker2-sq-corr,ricker2-sq-corr-inv}).
\end{comment}

The local skewness attribute has the advantage over \old{the varimax norm} \new{kurtosis} 
because of its higher dynamic range, which facilitates
picking. Variance is the second statistical order, skewness is the
third one, and \old{varimax} \new{kurtosis} is related to the fourth
order. Estimation variances increases with the order of a
moment \cite[]{Mend91}. In other words, less samples are needed to
estimate skewness with the same accuracy as \old{the} kurtosis\old{or varimax norm}.
We hypothesize that this contributes to the higher dynamic range
of the skewness criterion.

\section{Conclusions}
We have presented a novel approach to nonstationary identification of
apparent (locally observed) phase. Our approach is based on a new
attribute, local skewness. In synthetic and field-data examples, local
skewness exhibits a tendency to pick focused signals and a higher
dynamical range than the previously used local kurtosis. Its
computation involves a local similarity between the input signal and
its square. Practical applications of using local skewness for
zero-phase correction of seismic signals should be combined with
well-log analysis in order to better separate the locally-observed
phase from the propagating-wavelet phase.

\section{Acknowledgments}
We thank Milo Backus, Guochang Liu, Mike Perz, and Hongliu Zeng for
stimulating discussions.

\clearpage
\bibliographystyle{seg}
\bibliography{SEG,phase}

