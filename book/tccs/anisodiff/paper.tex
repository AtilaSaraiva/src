\published{Geophysics, v. 85, S313-S325, (2020)}

\renewcommand{\thefootnote}{\fnsymbol{footnote}}

\title{Least-squares diffraction imaging using shaping regularization by anisotropic smoothing}

\author{Dmitrii Merzlikin\footnotemark[1], dmitrii.merzlikin@utexas.edu, \\
	Sergey Fomel\footnotemark[2], sergey.fomel@beg.utexas.edu, \\
        Xinming Wu\footnotemark[3], xinmwu@ustc.edu.cn}

\address{
\footnotemark[1]Formerly Bureau of Economic Geology \\
John A. and Katherine G. Jackson School of Geosciences \\
The University of Texas at Austin \\
University Station, Box X \\
Austin, Texas 78713-8924, USA; \\
presently WesternGeco \\
Schlumberger \\
3750 Briarpark Drive \\
Houston, Texas 77042, USA\\
E-mail: dmitrii.merzlikin@utexas.edu (corresponding author)
\footnotemark[2]Bureau of Economic Geology \\
John A. and Katherine G. Jackson School of Geosciences \\
The University of Texas at Austin \\
University Station, Box X \\
Austin, Texas 78713-8924, USA \\
\footnotemark[3]Formerly Bureau of Economic Geology \\
John A. and Katherine G. Jackson School of Geosciences \\
The University of Texas at Austin \\
University Station, Box X \\
Austin, Texas 78713-8924, USA; \\
presently School of Earth and Space Sciences \\
University of Science and Technology of China \\
Hefei, China \\
}

\lefthead{Merzlikin et al.}
\righthead{Anisotropic Smoothing Diffraction Imaging}

\maketitle

\begin{abstract}
We use least-squares migration to emphasize edge diffractions. The inverted forward modeling operator is the chain of three operators: Kirchhoff modeling, azimuthal plane-wave destruction and
path-summation integral filter. Azimuthal plane-wave destruction removes reflected energy without damaging edge diffraction signatures. Path-summation integral guides the inversion towards probable diffraction locations. We combine sparsity constraints and anisotropic smoothing in the form of shaping regularization to highlight edge diffractions. Anisotropic smoothing enforces continuity
along edges. Sparsity constraints emphasize diffractions perpendicular to edges and have a denoising effect. Synthetic and field data examples
illustrate the effectiveness of the proposed approach in denoising and highlighting edge diffractions, such as channel edges and faults.   
\end{abstract}

\section{Introduction}

Diffraction imaging is able to highlight subsurface discontinuities associated with channel edges, fracture swarms and faults.
Since diffractions are usually weaker than reflections \cite[]{klem-musatov94} and have lower signal-to-noise ratio, robust
diffraction extraction is of utmost importance for the imaging of subtle discontinuities.
\new{A number of diffraction imaging methods have been developed and can be
classified based on the separation technique being employed.
Methods based on optimal stacking of diffracted energy and suppression
of reflections are described by
\cite{kanasewich88},
\cite{landa98},
\cite{berkovitch09},
\cite{dell11},
\cite{tsingas11} and
\cite{rad14}.
Wavefield separation methods aim to decompose conventional full-wavefield seismic records into different
components representing reflections and diffractions \cite[]{papziner98,taner06,fomel07,schwarz2017accessing,schwarz2019coherent,dell2019ausing}.
Decomposition can be carried out in the common-image gather domain \cite[]{reshef09,klokov12,landa15}.
Other authors \cite[]{kozlov04,moser08,koren2011full,klokov2013selecting,mihai15} modify migration
kernel to eliminate specular energy coming from the first Fresnel zone and image diffractions only. For the methods involving migration,
diffraction extraction quality becomes dependent on the velocity model accuracy.}
Numerous case studies show that diffraction images carry valuable additional information for seismic interpretation
\cite[]{schoepp15,burnett15,sturzu15,tyiasning2016comparison,klokov2017integrated,klokov2017diffraction,merzlikin2017diffraction,
de2017high,pelissier2017interpretation,koltanovsky2017enhancing,zelewski2017diffraction,foss2018examples,glockner2019imaging,moser2020diffraction,montazeri2020improved}.
\old{new papers added.}

Consideration of diffraction phenomena in 3D \cite[]{keller1962geometrical,klem2008edge,hoeber2010diffractions} requires taking into account edge diffractions.
\old{Due to lateral symmetry they have both reflective (observed along the edge) and diffractive (observed perpendicular to the edge) components.}
\new{Due to lateral symmetry they kinematically behave as reflections when observed along the edge and as diffractions when observed perpendicular to the edge \cite[]{moser2011edge}.}
%Imaging of edge diffractions requires their signature to be properly processed.
Thus, edge diffraction signature is neither a ``pure'' reflection nor a ``pure'' diffraction but rather a combination of both and therefore requires a special processing procedure to be emphasized.
\cite{serfaty2017wavefield} separate reflections, tip and edge diffractions and noise using principal component analysis and deep learning.
\cite{klokov2011point} \new{and }\old{;}\cite{bona2015using} investigate 3D signatures of different types of diffractors. 
\cite{alonaizi13}\old{;}\new{ and }\cite{merzlikin2017unconventional} propose workflows to properly process energy diffracted on the edge. 
\cite{keydar2019wave} propose a method for edge diffraction imaging based on time-reversal principle and the stacking operator 
directly targeting edge diffractions. \cite{dell2019azimuthal} extract edge diffraction responses from full wavefield data by analyzing amplitude distribution
along different azimuths on a 3D prestack time Kirchhoff migration stacking surface.
\new{\cite{znak2019identification} develop a common-reflection-surface-based framework for distinguishing between point and edge diffractions and separating them from reflections.}

Separation of reflections and diffractions can be done as a part of least-squares migration \cite[]{Nemeth99,ronen2000least}. \cite{harlan84}
pioneer in separating diffractions from noise by comparing observed data and data modeled from a migration image in a least-squares sense.
\cite{merzlikin2016least} perform least-squares migration chained with plane-wave destruction and path-summation integral filtering and enforce sparsity
in a diffraction model.
\cite{merzlikin2019least} extend the approach and simultaneously decompose the input wavefield into reflections, diffractions and noise.
\new{\cite{decker2017enhancing} denoise diffractions by applying semblance-based weights estimated
in dip-angle gather (DAG) domain.}
\cite{yu2016sparse} utilize common-offset Kirchhoff least-squares migration with a sparse model regularization to emphasize diffractions.
\cite{yu2017seismic} extract diffractions based on plane wave destruction and dictionary learning for sparse representation.
\cite{yu2016separation} use two separate modeling operators for diffractions and reflections and impose a sparsity constraint on diffractions.
Sparse inversion is an efficient tool to perform extraction and denoising of diffractions since scatterers have spiky and intermittent distribution. However, a simple sparsity
constraint does not account for the signature of the energy scattered on the edge\new{, which is kinematically similar to a reflection when observed along the edge, and thus can distort it.}
\old{, which can lead to a distortion of a reflective component of an edge diffraction.}

We combine sparsity constraints and structure-oriented smoothing in the form of shaping regularization \cite[]{fomel2007shaping} to highlight edge diffractions and
account for their signature. Structure-oriented smoothing performs smoothing along the edges emphasizing their continuity \cite[]{hale2009structure}. 
Sparsity constraints imposed by thresholding in the model space force the model to describe the data with the fewest parameters and 
therefore \new{denoise and emphasize edge diffraction signatures observed perpendicular to the edge. Thus, we properly account for edge diffraction kinematic behavior for
both parallel and perpendicular to the edge directions.}
\old{denoise and emphasize the diffracted component of an edge diffraction. Thus, we are able to emphasize both reflective and diffractive components of edge diffractions.}

For forward modeling we use a chain of operators introduced by \cite{merzlikin2016least}. We extend this workflow to three-dimensions and modify reflection destruction operator
to account for an edge diffraction signature by suppressing reflected energy perpendicular to edges. Edge orientations are determined through a plane-wave destruction based structure tensor
\cite[]{merzlikin2017unconventional}. We start with a method introduction, then validate its performance on a synthetic, on a noisy marine field dataset and
on a land field dataset by separating edge diffractions from reflections and noise.

\section{Method}

\subsection{Objective function}
  
To solve for a seismic diffraction image $\mathbf{m}_d$, we extend the approach developed by \cite{merzlikin2016least} to three dimensions: 
\begin{equation}
\label{eq:chain}
J(\mathbf{m}_d) = \|\mathbf{d}_{PI} - \mathbf{PDL}\mathbf{m}_d\|_{2}^{2},
\end{equation}
\new{where $J(\mathbf{m}_d)$ is the objective function, $\mathbf{d}_{PI} = \mathbf{PDd}$ and $\mathbf{d}$ is ``observed'' data.}
Here, forward modeling corresponds to the chain of operators: three-dimensional path-summation integral filter $\mathbf{P}$ \cite[]{merzlikin15,merzlikin17}, 
azimuthal plane wave destruction (AzPWD) filter $\mathbf{D}$ \cite[]{merzlikin2016diffraction,merzlikin2017diffraction} and three-dimensional Kirchhoff modeling $\mathbf{L}$.
The path-summation integral filter $\mathbf{P}$ can be treated as the probability
of a diffraction at a certain location. Azimuthal plane wave destruction filter $\mathbf{D}$
removes reflected energy perpendicular to the edges.
Therefore, AzPWD
\new{emphasizes edge diffraction signature, which when measured in the direction perpendicular to the edge
exhibits a hyperbolic moveout and kinematically behaves as a reflection when observed along the edge.}
\old{emphasizes diffracted component and preserves reflected component of an edge diffraction.}
AzPWD application is the key distinction from the 2D version of the workflow \cite[]{merzlikin2016least}, in which
plane-wave destruction filter (PWD) \cite[]{fomel02} is applied along the time-distance
plane \cite[]{fomel07,merzlikin2018ad}.
After weighting the data $\mathbf{d}_{PI} = \mathbf{PDd}$
by path-summation integral $\mathbf{P}$ and AzPWD filter $\mathbf{D}$, model fitting is constrained to most probable diffraction locations.

\subsection{Determination of edge diffraction orientation\old{.}}
For AzPWD workflow \cite{merzlikin2016diffraction,merzlikin2017diffraction} show that a volume with plane-wave destruction filter
applied in arbitrary direction $x'$ corresponding to the azimuth $\theta$ can be generated as a linear combination of
PWDs applied in inline ($\mathbf{D}_x$) and crossline ($\mathbf{D}_y$) directions: $\mathbf{D}_{x'}\ = \mathbf{D}_{x}\,\mathrm{cos\theta}\ +\ \mathbf{D}_{y}\,\mathrm{sin\theta}$. Due to migration procedure linearity the same
relationship holds for the images of the corresponding PWD volumes.
Azimuth $\theta$ should be perpendicular to the edge at each location to remove reflections but\old{keep reflective components of edge diffractions.}
\new{preserve edge diffraction signatures, which are kinematically similar to reflections when observed along the edge.}

This azimuth can be determined from the structure tensor
\cite[]{van1995estimators,weickert1997review,fehmers2003fast,hale2009structure,wu2017directional,wu2017directional2}, which is defined as
an outer product of migrated plane-wave destruction filter volumes in inline and crossline directions \cite[]{merzlikin2016diffraction,merzlikin2017diffraction}:
\begin{equation}
\label{eq:structuretensor}
\mathbf S=
\begin{bmatrix}
\langle p_x p_x \rangle & \langle p_x p_y \rangle \\ \langle p_x p_y \rangle & \langle p_y p_y \rangle
\end{bmatrix}
,
\end{equation}
\new{where $\langle\rangle$ denotes smoothing of structure-tensor components, which is done in the edge-preserving fashion \cite[]{liu2010nonlinear}.
Smoothing stabilizes structure-tensor orientation determination in the presence of noise \cite[]{weickert1997review,fehmers2003fast},
while edge-preservation keeps information related to geologic discontinuities, which otherwise would be lost due to smearing.}
Here, $p_x$ and $p_y$ are the samples of inline
\old{($\mathbf{D}_x$)}and crossline\old{($\mathbf{D}_y$)}
migrated PWD volumes \new{($\mathbf{P}_x$ and $\mathbf{P}_y$)}
at each location. PWD filter can be treated
as a derivative along the dominant local slope \cite[]{fomel02,fomel07}. Thus, a 2D PWD-based structure tensor (equation~\ref{eq:structuretensor}) effectively represents 3D structures without
the need for the third dimension because the orientations are determined along the horizons.

Edge orientation can be determined by an eigendecomposition of a structure tensor \cite[]{fehmers2003fast,hale2009structure}:
$\mathbf{S}\ =\ \lambda_{u} \mathbf{u}\mathbf{u}^T +\ \lambda_{v} \mathbf{v}\mathbf{v}^T$.
If a linear feature (edge) is encountered eigenvector $\mathbf{u}$ corresponding to a larger eigenvalue $\lambda_{u}$ points in the direction perpendicular to the edge. Eigenvector $\mathbf{v}$
of a smaller eigenvalue $\lambda_{v}$ points along the edge. Thus, azimuth $\theta$ of a direction $x'$ perpendicular to the edge can be computed from either $\mathbf{u}$ or $\mathbf{v}$.
If no linear features are observed, there is no preferred PWD direction.

The PWD-based tensor (equation~\ref{eq:structuretensor}) describes 3D structures. Its components ($p_x$ and $p_y$) are computed along the ``structural frame"
defined by the reflecting horizons. Thus, vectors $\mathbf{u}$ and $\mathbf{v}$ ``span" the surfaces, which at each point are determined by dominant local slopes.
Eigenvectors of a PWD-based structure tensor (equation~\ref{eq:structuretensor}) are parallel to a reflection surface at each point.  

\subsection{Regularization}

We use shaping regularization to constrain the model \cite[]{fomel2007shaping}. We penalize
edge diffractions by two shaping operators: thresholding and smoothing by anisotropic diffusion.
\old{First emphasizes
diffracted component of an edge diffraction perpendicular to the edge and denoises diffractions by enforcing sparsity. The second one emphasizes the
reflected component from an edge and enforces its continuity.}
\new{Iterative application of the first operator forces the model to describe the data with the
fewest parameters possible \cite[]{daubechies2004iterative} and therefore denoises and emphasizes edge diffraction signatures in the direction perpendicular to the edge.
The second operator emphasizes the signatures of edge diffractions along the edge by enforcing their continuity.}  

Smoothing based on anisotropic diffusion enhances flow-like structures and
completes interrupted lines \cite[]{weickert1998anisotropic}. The input is an unfiltered
image, smoothness of which is increasing while diffusion is proceeding in time \cite[]{fehmers2003fast}. In a discrete
form anisotropic diffusion can be formulated as follows:
\begin{equation}
\label{eq:anisodiff1}
\frac{\mathbf{U}_k-\mathbf{U}_{k-1}}{\Delta t} = - \,\mathbf{D}^T_{im}\,\mathbf{V}\,\,\mathbf{V}^T\,\mathbf{D}_{im}\,\mathbf{U}_k\;,
\end{equation}
where $\mathbf{U}_k$ is an image at a diffusion time step $k$ \new{with $\Delta t$ sampling interval}, $\mathbf{D}_{im}$ is a matrix operator combining PWDs in inline and crossline
directions in the image domain and $\mathbf{V}$ is a matrix of eigenvectors $\mathbf{v}$, which are parallel
to linear features' orientations along the dominant local slopes. Classic anisotropic diffusion partial differential equation \cite[]{weickert1998anisotropic} requires
three-dimensional structure tensor and gradient. In equation~\ref{eq:anisodiff1}, we utilize PWD-based structure tensor instead of its
three-dimensional counterpart based on derivatives along the coordinate axes and a combination of PWDs in inline and
crossline directions in the image domain instead of a three-dimensional gradient.
After terms rearrangement, equation~\ref{eq:anisodiff1} takes the form of a linear least-squares solution
with forward modeling operator corresponding to the identity matrix:
\begin{equation}
\label{eq:diffuse}
\mathbf{U}_k = \left(\mathbf{I} + \epsilon^2\,\mathbf{D}^T_{im}\,\mathbf{V}\,\,\mathbf{V}^T\,\mathbf{D}_{im}\right)^{-1}\,\mathbf{U}_{k-1}\, ,
\end{equation}
where $\epsilon$ corresponds to the time step $\Delta t$. At diffusion time step $k$ we update the previous image $\mathbf{U}_{k-1}$ with $N$ conjugate gradients iterations
until the desired smoothness is achieved in the output image $\mathbf{U}_{k}$.
Total number of diffusion time steps $K$, number of conjugate gradients iterations $N$ and parameter $\epsilon$ control the smoothness of the final result.

Figure~\ref{fig:zigzag0SEG,zigzagSEG,diffuseSEG,anisodiffuseSEG,zigzag-thrSEG,anisodiffuse-thrSEG} shows anisotropic smoothing operator
action on a zigzag pattern contaminated with Gaussian noise. Isotropic smoothing operator
(see, e.g., \cite{weickert1998anisotropic} for details) action is equal for all the directions and thus results in sharpness loss incurred by smoothing
across the edges (Figure~\ref{fig:diffuseSEG}). Figure~\ref{fig:anisodiffuseSEG} shows that anisotropic
smoothing operator (equation~\ref{eq:diffuse}) preserves edges by smoothing along them
and thus results in both S/N ratio enhancement and sharpness of the image.

\new{Flow-like coherent noise patterns
in Figure~\ref{fig:anisodiffuseSEG} are induced by using ``true''
azimuths of edges across the whole image including both signal and noise regions.
The result of combining both regularization operators - thresholding and anisotropic smoothing - is shown in Figure~\ref{fig:anisodiffuse-thrSEG}:
noise and flow-like artifacts are suppressed in regions with no signal. Thus, the combination of thresholding and anisotropic smoothing
is required for proper edge diffraction regularization.}

\new{In the following examples, edge orientation estimation is done for each location individually based on a structure tensor.}

\inputdir{simple}

\multiplot{6}{zigzag0SEG,zigzagSEG,diffuseSEG,anisodiffuseSEG,zigzag-thrSEG,anisodiffuse-thrSEG}{width=0.3\columnwidth}{Zigzag pattern: (a) with no noise;
(b) with Gaussian noise added; (c) corresponds to (b) processed by isotropic smoothing operator (notice
smoothing across the edges); (d) corresponds to (b) processed by anisotropic smoothing operator (edges are highlighted and preserved).
\new{Flow-like coherent noise patterns in (d) are induced by using ``true'' azimuths of edges
across the whole image including both signal and noise regions. Thresholding operator action on (b) is shown in (e).
The result of combining both regularization operators - anisotropic smoothing and thresholding - is shown in (f). 
Noise including flow-like artifacts is attenuated.}}   

\subsection{Optimization}

For inversion we adopt a conjugate gradients scheme \cite[]{fomel07}:
\begin{equation}
\label{eq:itthr}
\mathbf{m}_d^i\ \leftarrow\ \mathbf{H}_{\epsilon,N,K} \mathbf{T_{\lambda}} \big[ \mathbf{m}_{d}^{j}\ +\ \alpha_{j}\mathbf{s}_{j} \big],\ \mathbf{s}_{j}=-\nabla J(\mathbf{m}_{d}^{j})\ +\ \beta_{j}\mathbf{s}_{j-1}
\end{equation}
where $\mathbf{H}_{\epsilon,N,K}$ and $\mathbf{T_{\lambda}}$ are anisotropic-smoothing and thresholding operators,
$-\nabla J(\mathbf{m}_{d}^{j})$ is the gradient at iteration $j$, $\mathbf{s}_{j}$ is a conjugate direction,
$\alpha_{j}$ is an update step length , and $\beta_j$ is designed
to guarantee that $\mathbf{s}_j$ and $\mathbf{s}_{j-1}$ are conjugate.
After several internal iterations $j$ of the conjugate
gradient algorithm we generate $\mathbf{m}_d^j$, to which
we apply thresholding to drop samples corresponding to noise with values lower than the threshold $\lambda$,
and which we then smooth along edges by applying anisotropic smoothing operator $\mathbf{H}_{\epsilon,N,K}$. Outer model shaping iterations
are denoted by $i$. 

Inversion results also depend on the numbers of inner and outer iterations:
their tradeoff determines how often shaping regularization is applied
and therefore controls its strength. Regularization by early stopping can also be conducted.
The optimization strategy with 
$\mathbf{H}_{\epsilon,N,K}$ removed corresponds to the iterative thresholding approach \cite[]{daubechies2004iterative}. 

\subsection{Workflow}

\new{The workflow takes stacked data as the input.} 
To generate all the inputs necessary for the inversion we propose the following sequence of procedures:
\begin{enumerate}
\item estimate inline and crossline dips describing dominant local slopes associated with reflections in the stack;
\item perform PWD filtering on the stack in inline and crossline directions;
\item migrate the corresponding volumes;
\item combine the migrated volumes in a structure tensor (equation~\ref{eq:structuretensor});
\item smooth structure tensor components along structures with edge preservation;
\item perform eigendecomposition of a structure tensor and determine orientations of edges;
\item apply AzPWD and path-summation integral to the stacked data;
\new{\item apply conventional full-wavefield migration to the dataset stack (same as the input to step 1);
estimate dips and generate PWD volumes in the inline and in the crossline directions
in the image domain ($\mathbf{D}_{\text{im}}$ in equation~\ref{eq:anisodiff1})
for anisotropic smoothing regularization.}
\end{enumerate} 

\new{The sequence of procedures with their corresponding inputs and outputs is shown in Figure~\ref{fig:schematic}.
In the first step, dips are estimated using PWD \cite[]{fomel07}, and two volumes - one for the inline dips and
one for the crossline dips - are produced and further used for reflection removal in step 2.
In the second step, based on the two input dip volumes, reflections are predicted and suppressed:
two outputs are generated, which correspond to PWD filter application
in the inline ($\mathbf{D}_x$)
and in the crossline ($\mathbf{D}_y$) directions
using the corresponding dip distributions. These two volumes with
reflections removed are then migrated (step 3) using conventional full-wavefield migration, e.g. 3D post-stack Kirchhoff migration.
For step 4, instead of explicitly computing structure tensor for each data sample according to equation~\ref{eq:structuretensor},
volumes for each of its components can be pre-computed. The term $p_x p_x$ structure-tensor component volume can be
generated by the Hadamard product between the migrated inline PWD volume ($\mathbf{P}_x$) and itself, the
$p_y p_y$-component volume -
by the Hadamard product between the migrated crossline PWD volume ($\mathbf{P}_y$) and itself,
and the $p_x p_y$-component volume -
by the Hadamard product between the migrated inline PWD volume ($\mathbf{P}_x$) and the migrated crossline PWD volume ($\mathbf{P}_y$).
Then, in step 5, the $p_x p_x$, $p_y p_y$, and $p_x p_y$ structure-tensor component volumes
are input to edge preserving smoothing, which outputs the $\langle p_x p_x\rangle$, $\langle p_y p_y\rangle$,
and $\langle p_y p_y\rangle$ volumes. In step 6, structure tensor (equation~\ref{eq:structuretensor})
eigendecomposition is performed ``on the fly'' by combining structure tensor component values
from the $\langle p_x p_x\rangle$, $\langle p_y p_y\rangle$,
and $\langle p_y p_y\rangle$ volumes for each data sample. The result is the smaller eigenvalue eigenvector volume,
which is then converted to edge diffraction orientations $\mathbf{\Theta}$.
Step 7 gives the data to be fit by the inversion (equation~\ref{eq:chain}).}
Orientations of structures for AzPWD and for anisotropic smoothing regularization are estimated in step \old{5}\new{6}.
In anisotropic diffusion \new{instead of derivatives in Cartesian coordinates} we use PWDs in inline and crossline directions in the image domain
(\new{$\mathbf{D}_{\text{im}}$ in }equation~\ref{eq:anisodiff1}), dip estimation for which should be performed on a ``conventional" image
of a full wavefield stack \new{(step 8)}.   
\old{Step 6 gives the data to be fit by the inversion (equation~\ref{eq:chain}).} Then, we invert the data \new{for edge diffractions.} 

\inputdir{.}
\plot{schematic}{width=0.9\columnwidth}{\new{Workflow chart illustrating the sequence of procedures and the relation of their corresponding inputs and outputs. 
Here, $\mathbf{D}_x$ and $\mathbf{D}_y$ correspond to inline and crossline PWD volumes of the input stack, $\mathbf{P}_x$
and $\mathbf{P}_y$ correspond to inline and crossline PWD volumes after migration, $\odot$ corresponds to the Hadamard (element-wise) matrix
product, $\langle\rangle$ corresponds to the edge-preserved smoothing, $\mathbf{\Theta}$ corresponds to the volume of edge diffraction azimuths,
$\mathbf{P}\mathbf{D}\mathbf{L}$ corresponds to the forward modeling operator corresponding to the chain of path-summation
integral, AzPWD and Kirchhoff modeling operators respectively; $\mathbf{d}_{PI} = \mathbf{PDd}$, where $\mathbf{d}$ corresponds to the input stack; and
$\mathbf{H}_{\epsilon}$ and $\mathbf{T_{\lambda}}$ are anisotropic-smoothing and thresholding operators.
The term $\mathbf{m}_d$ is the edge diffractivity we invert for: $\mathbf{m}_d^{j}$ describes model updates from internal iterations minimizing the misfit, and
$\mathbf{m}_d^{i}$ is the result of regularization applied during external iterations. 
Notice in step 8, inline and crossline PWDs are computed in the image domain ($\mathbf{D}_{\text{im}}$ in equation~\ref{eq:anisodiff1}), and then are used instead of Cartesian derivatives
in anisotropic smoothing operator $\mathbf{H}_{\epsilon}$ (equation~\ref{eq:anisodiff1}). The edge diffraction azimuths $\mathbf{\Theta}$ are used in the AzPWD
operator $\mathbf{D}$ (step 6) and in the anisotropic smoothing operator $\mathbf{H}_{\epsilon}$ to prevent smearing across the edges.
}}

\section{Synthetic Data Example}
\inputdir{simple3d}

We test the approach on a synthetic data example\old{.}
\new{with a reflectivity model is shown in Figure~\ref{fig:burden}.
The synthetic seismic data are generated by Kirchhoff modeling method with
$15\ \text{Hz}$ peak-frequency Ricker wavelet, reflectivity distribution shown in Figure~\ref{fig:burden} and $2.0\ \text{km/s}$
constant velocity model, which is further utilized in both
migration and inversion. Zero-offset geometry with $1\ \text{m}$ sampling in both inline and crossline directions is used.}   
\old{Zero-offset data are}\new{The synthetic data are} shown in Figure~\ref{fig:zo3}. Random noise
\new{with a maximum amplitude of 30\% of that of the signal,}
filtered to the signal frequency band is added to the synthetic.
After 3D Kirchhoff migration diffractions become focused and a channel-like structure with a zigzag pattern becomes prominent (Figure~\ref{fig:mig3}). However, channel edges
appear to be somewhat blurred and masked by the reflections. We follow the workflow described above. Data to be fit by the inversion is shown in Figure~\ref{fig:linpi-data}.
We use the following parameters for the inversion: $\lambda=100$, $\epsilon=10$, $N=5$ and $K=1$. In total we use $20$ iterations - 5 inner by 4 outer.
The inversion result is shown in Figure~\ref{fig:modl}. Edges are highlighted and denoised.   

%\plot{zo3}{width=0.8\columnwidth}{Zigzag zero-offset synthetic.}
%\plot{mig3}{width=0.8\columnwidth}{3D Kirchhoff migration of the Zigzag zero-offset synthetic (Figure~\ref{fig:zo3}).} 
%\plot{linpi-data}{width=0.8\columnwidth}{AzPWD and 
%path-summation integral migration applied to the Zigzag zero-offset synthetic (Figure~\ref{fig:zo3}).}
%\plot{modl}{width=0.8\columnwidth}{Inversion result: edge diffractions are highlighted and denoised.}
%\multiplot{4}{zo3,mig3,linpi-data,modl}{width=0.8\columnwidth}{(a) Zigzag zero-offset synthetic; (b) 3D Kirchhoff migration of the Zigzag zero-offset synthetic (Figure~\ref{fig:zo3});
%(c) AzPWD and path-summation integral migration applied to the Zigzag zero-offset synthetic (Figure~\ref{fig:zo3}); (d) Inversion result: edge diffractions are highlighted and denoised.}

\plot{burden}{width=0.9\columnwidth}{Reflectivity model for the synthetic data example. Reflectivity corresponds to density
contrasts while the velocity is kept constant at $2.0\ \text{km/s}$ throughout the volume.}

\multiplot{2}{zo3,mig3}{width=0.8\columnwidth}{(a) Zigzag zero-offset synthetic; (b) 3D Kirchhoff time migration of the zigzag zero-offset synthetic (Figure~\ref{fig:zo3}).
\new{While focusing of both reflections and edge diffractions can be observed in (b), channel edges are masked by specular energy
and appear to be blurred.}}

\multiplot{2}{linpi-data,modl}{width=0.8\columnwidth}{(a) AzPWD and path-summation integral migration applied to the zigzag zero-offset synthetic (Figure~\ref{fig:zo3})
\new{: A high diffraction probability along the channel edges can be observed}; (b) Inversion result: edge diffractions are highlighted and denoised.}

\section{Field Data Example I}

We test the capabilities of the proposed approach on the field dataset acquired with a high-resolution 3D (HR3D)
marine seismic acquisition system (P-cable) in the Gulf of Mexico to characterize structure and stratigraphy of
the shallow subsurface \cite[]{meckel2016use,klokov2017diffraction,merzlikin2017diffraction,sgreer18}.
\new{The acquisition geometry is defined by a `cross cable' of a catenary shape with paravans (diverters) at the ends oriented perpendicular
to the inline (transit) direction that tows twelve $25\ \text{m}$ long streamers with separation of $10-15\ \text{m}$. Source-receiver
offsets are on order of $100-150\ \text{m}$, while the source sampling is $6.25\ \text{m}$.
The target interval of interest in this paper is associated with the $0.222\ \text{s}$ time slice, which
approximately corresponds to $160\ \text{m}$ depth. 
Diffraction imaging is applied to the dataset stack preprocessed with the following procedures: 
$40\ \text{Hz}$ low-pass filter, wavelet deconvolution, surface-consistent amplitude corrections,
predictive deconvolution, missing trace interpolation and acquisition footprint elimination
using F-K filtering in the image domain.} More detailed description of the acquisition geometry, geology, and interpretation of diffraction images can be found in
\cite{klokov2017diffraction} and \cite{merzlikin2017diffraction}.
Here we focus on the interval with a channel of high wavelength sinuosity \cite[]{merzlikin2017diffraction}.
\new{For simplicity, constant velocity model of $1.5\ \text{km/s}$ is used for both migration and inversion.
Higher accuracy diffraction-based velocity estimation for the same dataset is described in \cite{merzlikin2017diffraction}.}
%Provided narrow-offset P-cable acquisition geometry and shallow unconsolidated sediments being imaged, constant velocity model
%serves the purpose of proving method's validity.}

The stacked volume is shown in Figure~\ref{fig:stack}.
The post-stack 3D Kirchhoff time migration image of the target slice is shown in Figure~\ref{fig:mig3-slice}.
Channel delineation is hindered by stronger reflections and acquisition footprint - horizontal lines. 
We eliminate the footprint in inline and crossline PWD diffraction images using F-K filtering before
combining them in a structure tensor (equation~\ref{eq:structuretensor}) and forming AzPWD linear combination \cite[]{merzlikin2017diffraction}.
Result of AzPWD volume migration is shown in Figure~\ref{fig:xspr-prec-slice}.
Channel and other small-scale features (e.g. fault at in-lines $2.5-3.0\ \text{km}$ and cross-lines $6.0-6.4\ \text{km}$) are highlighted but the image is still noisy.

\inputdir{pcable}

\plot{stack}{width=0.9\columnwidth}{\new{High-resolution 3D marine seismic dataset from the Gulf of Mexico: stacked volume.}
\old{P-cable dataset stack.}}

We follow the proposed workflow and first generate the data to be fit by the inversion (Figure~\ref{fig:linpi-data-slice}) by applying AzPWD and path-summation integral migration to the stack
(Figure~\ref{fig:stack}). We run five outer and two inner iterations and use $\lambda=0.008$, $\epsilon=20$, $N=30$ and $K=1$. Due to
low signal-to-noise ratio of the dataset small number of inner iterations is used to prevent leakage of noise to the diffraction image domain.
The result of the proposed approach is shown in Figure~\ref{fig:modl-wk-100}. The channel appears to be highlighted and denoised. Most of the
low-amplitude events are also preserved and highlighted including the fault feature (in-lines $2.5-3.0\ \text{km}$ and cross-lines $6.0-6.4\ \text{km}$).
Discontinuities of edges intersecting each other at inlines $1.0-1.5\ \text{km}$ and crosslines $4.0-5.0\ \text{km}$ are caused by their rapidly varying orientations prohibiting smoothing
from emphasizing different strikes simultaneously. 

Figures~\ref{fig:hs-modl-wk-100-thr} and~\ref{fig:hs-modl-wk-100} show
the diffractivity model after thresholding and after thresholding followed by anisotropic smoothing applied correspondingly. The figures
illustrate that anisotropic smoothing merges neighbor samples and thus is necessary for edge diffraction regularization.
\new{The result is consistent with the experiments shown in Figure~\ref{fig:zigzag0SEG,zigzagSEG,diffuseSEG,anisodiffuseSEG,zigzag-thrSEG,anisodiffuse-thrSEG}.} 
 
Figure~\ref{fig:xspr-notmig,diffractions-ortho,noise-ortho} shows the stack after reflection elimination, modeled diffractions from denoised diffractivity
shown in Figure~\ref{fig:modl-wk-100} and their difference. Denoised diffractions in Figure~\ref{fig:diffractions-ortho} show clear hyperbolic signatures. Reflection
energy remainders after AzPWD application prominent in Figure~\ref{fig:xspr-notmig}
(e.g.,
$0.222\ \text{s}$ TWTT (two-way traveltime), inline $1.625\ \text{km}$ and crosslines $5.5-6.25\ \text{km}$ and
$0.222\ \text{s}$ TWTT, inlines $2.4-2.7\ \text{km}$ and crosslines $6.25-6.5\ \text{km}$)
are also removed.
Difference in Figure~\ref{fig:noise-ortho} is predominated by noise,
reflection remainders (e.g., regions mentioned above)
and acquisition footprint and proves the effectiveness of the approach.

To account for amplitude differences and for higher continuity of denoised diffractions (Figure~\ref{fig:diffractions-ortho}) in comparison to the stack with AzPWD (Figure~\ref{fig:xspr-notmig})
signal and noise orthogonolization \cite[]{chen2015random} has been applied. Similarity between restored signal (Figure~\ref{fig:diffractions-ortho})
and restored noise (Figure~\ref{fig:noise-ortho}) is measured. Then, signal energy, which leaked
to the difference volume (Figure~\ref{fig:noise-ortho}) and which has high similarity with the signal events actually predicted (Figure~\ref{fig:diffractions-ortho}),
is withdrawn from the noise volume and added to the signal estimate.   

\multiplot{2}{mig3-slice,xspr-prec-slice}{width=0.9\columnwidth}{$0.222$ s time slice: (a) 3D post-stack Kirchhoff migration of the stack shown in Figure~\ref{fig:stack};
(b) 3D post-stack Kirchhoff migration of the stack after AzPWD.
\new{Subsurface discontinuities appear to be highlighted in (b) in comparison to (a), in which they are masked by specular energy. At the same time,
diffraction image (b) still appears to be noisy.}}

\multiplot{2}{linpi-data-slice,modl-wk-100}{width=0.9\columnwidth}{$0.222$ s time slice: (a) observed data (Figure~\ref{fig:stack}) preconditioned
by AzPWD and path-summation integral migration; (b) inversion result (strong smoothing is used to 
highlight the continuity of edge diffractions).
\new{Significant improvement in signal-to-noise ratio of edge diffractions has been achieved (compare with Figure~\ref{fig:xspr-prec-slice}).}}

\multiplot{2}{hs-modl-wk-100-thr,hs-modl-wk-100}{width=0.47\columnwidth}{Last iteration diffractivity model: (a)
after thresholding; (b) after thresholding followed by anisotropic smoothing. \new{While thresholding allows for denoising, anisotropic smoothing emphasizes edge diffraction continuity and accounts for its
kinematic behavior when observed along the edge.}}

\multiplot{3}{xspr-notmig,diffractions-ortho,noise-ortho}{width=0.75\columnwidth}{(a) Stack with reflections removed; (b) Kirchhoff modeling
of diffractivity from Figure~\ref{fig:hs-modl-wk-100}; (c) difference between (a) and (b) is predominated by noise
\new{and, in particular, by reflection remainders
(notice coherent events with slowly varying amplitudes, e.g.,
$0.222\ \text{s}$ TWTT, inline $1.625\ \text{km}$ and crosslines $5.5-6.25\ \text{km}$ and
$0.222\ \text{s}$ TWTT, inlines $2.4-2.7\ \text{km}$ and crosslines $6.25-6.5\ \text{km}$)
and acquisition footprint}.}

\section{Field Data Example II}

\new{The} second field data example comes from the Cooper Basin \new{onshore}\old{in the} Western Australia.
\new{The dataset corresponds to stacked (with $25\times25\ \text{m}$ bin size) preprocessed data acquired as a 3D land seismic survey
with fully azimuthal distribution of offsets and a far offset of $4000\ \text{m}$.
Preprocessing sequence includes noise attenuation, near-surface static corrections, despike, surface-consistent deconvolution,
Q-compensation whitening and time-variant filter applied after stacking.} 
\new{The} target horizon slice picked by the interpreter and used
for the performance evaluation of the method corresponds to the interface between tight-gas sand and coals \cite[]{tyiasning2016comparison}.
\new{The depth of the interface is approximately $2500\ \text{m}$. Structural complexity of the overburden can be characterized as high.}
Detailed geological description of the area as well as the comparison of diffraction imaging results to discontinuity-type attributes is given by
\new{\cite{tyiasning2016comparison}}\old{(Tyiasning et al., 2016)}.
Here, we apply the developed approach to the dataset.
\new{Prestack time migration velocity is used for both full-wavefield migration and the proposed inversion approach for edge diffraction imaging.}

Figure~\ref{fig:stack-barrolka} shows a stacked volume of the dataset. We focus on the window around the target horizon, which on average corresponds to $1.72\ \text{s}$ TWTT.
Figure~\ref{fig:mig3-ts} shows conventional image of the target horizon slice generated with 3D post-stack time Kirchhoff migration.
We then apply reflection removal
procedure to the stack (Figure~\ref{fig:stack-barrolka}), migrate it, and generate the image shown in Figure~\ref{fig:xspr-mig3-ts}. Smaller scale features become highlighted, e.g., 
faults between $8-10\ \text{km}$ inlines and $0-4\ \text{km}$ crosslines. Acquisition footprint is noticeable on both of the images (Figure~\ref{fig:mig3-ts,xspr-mig3-ts}) and corresponds
to the low-amplitude events aligned in a grid-like fashion (easy to notice between inlines $0-2\ \text{km}$ and crosslines $6-8\ \text{km}$).
We apply the developed workflow to further highlight and denoise diffractions.

\inputdir{barrolka}
\plot{stack-barrolka}{width=0.99\columnwidth}{\new{3D sesimic dataset from the Cooper Basin onshore Western Australia: stacked volume.}
\old{Barrolka dataset stack.}}

\multiplot{2}{mig3-ts,xspr-mig3-ts}{width=0.8\columnwidth}{Target horizon slice
(the interface between tight-gas sand and coals picked by the interpreter ($\sim1.72\ \text{s}$ TWTT)):
(a) 3D post-stack Kirchhoff migration of the stack shown in Figure~\ref{fig:stack-barrolka};
(b) 3D post-stack Kirchhoff migration of the stack after AzPWD.
\new{Subsurface discontinuities appear to be highlighted in (b) in comparison to (a), in which they are masked by specular energy. At the same time,
diffraction image (b) still has some noise present.}}

First, we generate the data to be fit by the inversion - the stack weighted by a path-summation integral after reflection elimination (Figure~\ref{fig:linpi-data-ts}).
%Image shown in Figure~\ref{fig:linpi-data-ts} represents diffraction probability.
We run twenty outer and two inner iterations and use $\lambda=4$, $\epsilon=10$, $N=20$ and $K=1$.
As in the previous data example we use a low number of internal iterations to avoid noise fitting in general and footprint in particular. The inversion result along the target horizon is shown 
in Figure~\ref{fig:modl-test-ts}.
3D cubes of the migrated stack, the
migrated stack after reflection elimination and the inversion result are shown in Figures~\ref{fig:mig3-barrolka},~\ref{fig:xspr-mig3-barrolka} and~\ref{fig:modl-test}.
\new{Edges masked by the specular energy on the full-wavefield image (Figure~\ref{fig:mig3-barrolka}), are highlighted on the image after
reflection elimination (Figure~\ref{fig:xspr-mig3-barrolka}). At the same time,
edge diffractions produced by the inversion (Figure~\ref{fig:modl-test}) appear to be clearer and have higher signal-to-noise ratio.} 

Further, we apply Kirchhoff modeling to the inversion result (Figure\new{s}~\ref{fig:modl-test-ts}\new{ and~\ref{fig:modl-test}}), restore diffractions
(\new{Figure~\ref{fig:diffractions-ortho-barrolka}}\old{Figure~\ref{fig:diffractions-ortho-barrolka}}), and subtract the result from the stack after reflection elimination
(\new{Figure~\ref{fig:xspr-notmig-barrolka}}\old{Figure~\ref{fig:noise-ortho-barrolka}}). The result
of subtraction shown in Figure~\ref{fig:noise-ortho-barrolka} can be treated as noise eliminated during the inversion. Signal and noise orthogonalization \cite[]{chen2015random}
is applied to account for aperture difference between observed and restored diffractions (for instance, often only a single diffraction flank can be observed in the data
leading to spuriously high difference or the ``noise estimate") and to bring back some of the energy accidentally leaked to the noise domain.

The acquisition footprint evident in Figure~\ref{fig:noise-ortho-barrolka} suggests the
high performance of the method. Improvement can be noticed between $6-10\ \text{km}$ inlines and $2-4\ \text{km}$ crosslines associated
with the extraction of small-scale discontinuities. Compare the inversion result Figure~\ref{fig:modl-test-ts}
and the result of migration shown in Figure~\ref{fig:xspr-mig3-ts}. Same subtle events are located at crossline $3.25\ \text{km}$ between inlines $7-9\ \text{km}$
(Figures~\ref{fig:diffractions-ortho-barrolka} and~\ref{fig:xspr-notmig-barrolka}) and have a clear hyperbolic shape
\new{(Figure~\ref{fig:diffractions-ortho-barrolka}), which appears when edge diffractions are observed perpendicular to edges}.
Good restoration quality can also be inferred from the ``circular" structure
between $6-8\ \text{km}$ inlines and $0-2\ \text{km}$ crosslines. Both of these areas
have high similarity between the initial \new{diffraction} stack (Figure~\ref{fig:xspr-notmig-barrolka}) and the stack with diffractions ``restored"\new{ and denoised by inversion} (Figure~\ref{fig:diffractions-ortho-barrolka})
leading to low amplitudes in the difference section (Figure~\ref{fig:noise-ortho-barrolka}) primarily associated with noise. These features also follow ``frowning"-focusing-``smiling"
behaviour under migration velocity perturbation supporting their diffraction nature in the plane perpendicular to the edge.

\multiplot{2}{linpi-data-ts,modl-test-ts}{width=0.8\columnwidth}{Target horizon slice
(the interface between tight-gas sand and coals picked by the interpreter ($\sim1.72\ \text{s}$ TWTT)): (a) observed data (Figure~\ref{fig:stack-barrolka}) preconditioned
by AzPWD and path-summation integral migration; (b) inversion result:
\new{edge diffractions have been denoised (compare with Figure~\ref{fig:xspr-mig3-ts}).}}

%\multiplot{3}{mig3-barrolka,xspr-mig3-barrolka,modl-test}{width=0.65\columnwidth}{Migration result of (a) the stack (Figure~\ref{fig:stack-barrolka}); (b) the stack after reflection elimination
%(Figure~\ref{fig:xspr-notmig-barrolka}) and (c) the inversion result.}

%\multiplot{3}{xspr-notmig-barrolka,diffractions-ortho-barrolka,noise-ortho-barrolka}{width=0.65\columnwidth}{(a) Stack with reflections removed; (b) Kirchhoff modeling
%of diffractivity from Figure~\ref{fig:hs-modl-wk-100}; (c) difference between (a) and (b) is predominated by noise.}

\plot{mig3-barrolka}{width=0.99\columnwidth}{\old{Image of the stack (Figure~\ref{fig:stack-barrolka})}
\new{3D post-stack Kirchhoff migration of the stack shown in Figure~\ref{fig:stack-barrolka}. }}

\plot{xspr-mig3-barrolka}{width=0.99\columnwidth}{\old{Image of the stack after reflection elimination
(Figure~\ref{fig:xspr-notmig-barrolka})}
\new{3D post-stack Kirchhoff migration of the stack (Figure~\ref{fig:stack-barrolka}) after reflection elimination and edge diffraction extraction with AzPWD.
Subsurface discontinuities appear to be highlighted in comparison to Figure~\ref{fig:mig3-barrolka}, in which they are masked by specular energy. At the same time,
diffraction image still has some noise present}.}

\plot{modl-test}{width=0.99\columnwidth}{Inversion result.
\new{Edge diffractions are both highlighted and denoised (compare with Figure~\ref{fig:xspr-mig3-barrolka}).}}

%\multiplot{3}{xspr-notmig-barrolka,diffractions-ortho-barrolka,noise-ortho-barrolka}{width=0.65\columnwidth}{(a) Stack with reflections removed; (b) Kirchhoff modeling
%of diffractivity from Figure~\ref{fig:hs-modl-wk-100}; (c) difference between (a) and (b) is predominated by noise.}

\plot{diffractions-ortho-barrolka}{width=0.99\columnwidth}{Kirchhoff modeling of diffractivity from \old{Figure~\ref{fig:hs-modl-wk-100}}
\new{Figure~\ref{fig:modl-test}}\old{fixed the order of figures}.
\new{``Clean'' diffraction signatures are recovered: notice hyperbolic shapes when edge diffractions are observed perpendicular to edges.}} 

\plot{xspr-notmig-barrolka}{width=0.99\columnwidth}{Stack with reflections removed\old{fixed the order of figures}.
\new{The majority of reflections is removed, some hyperbolic signatures can be seen (compare with Figures~\ref{fig:stack-barrolka} and~\ref{fig:diffractions-ortho-barrolka}).
Reflection remainders and noise including acquisition footprint are apparent (compare with Figure~\ref{fig:diffractions-ortho-barrolka})}.}  

\plot{noise-ortho-barrolka}{width=0.99\columnwidth}{Difference between
\old{Figures~\ref{fig:xspr-notmig-barrolka} and~\ref{fig:diffractions-ortho-barrolka}}
\new{Figures~\ref{fig:diffractions-ortho-barrolka} and~\ref{fig:xspr-notmig-barrolka}} is predominated by noise
\new{and thus supports the validity of denoised edge diffractions (Figure~\ref{fig:diffractions-ortho-barrolka})}.}

Some features are lost in the area between $10-11\ \text{km}$ inlines and $6-8\ \text{km}$ crosslines. Event between inlines $12-14\ \text{km}$
and crosslines $6-8\ \text{km}$ is not predicted (Figure~\ref{fig:noise-ortho-barrolka}).
High magnitude events observed between inlines $1-3\ \text{km}$ at crossline $3.25\ \text{km}$ and between crosslines $4-5\ \text{km}$ at inline $5.35\ \text{km}$
(Figure~\ref{fig:modl-test} and~\ref{fig:diffractions-ortho-barrolka})
can be associated with reflections leaked to the diffraction image domain
or can actually be edge diffractions observed in the plane not perpendicular to the edge and thus having elongated signature.
The event located at crossline $3.25\ \text{km}$ between inlines $5-7\ \text{km}$ inversion interprets as a reflection and removes it
from the diffraction imaging result (Figure~\ref{fig:modl-test} and~\ref{fig:diffractions-ortho-barrolka}).
High difference values following the channel in the difference cube (Figure~\ref{fig:noise-ortho-barrolka})
can also be associated with the reflection remainders removed from the diffraction image domain.

It should be mentioned that the target horizon has a highly oscillating pattern including a high magnitude drop
in the left part of the cube (inlines $1-3\ \text{km}$) making dip estimation for ``ideal" reflection-diffraction separation simultaneously for the whole area challenging. More careful
dip estimation possibly with different parameters in different regions of the dataset should further improve the result. Inversion parameters can definitely be tweaked
to improve the results but even with these trial values the majority of the edges including some subtle features
(e.g., events at crossline $3.25\ \text{km}$ between inlines $7-9\ \text{km}$ (Figure~\ref{fig:modl-test})) hardly noticeable on
the migrated stack after reflection elimination (Figure~\ref{fig:xspr-mig3-barrolka})
has been extracted and highlighted. 

\section{Discussion}

\new{The dependency of workflow's ability to produce sharp images of edge diffractions upon the migration velocity accuracy requires further investigation.
On one hand, the approach incorporates least-squares migration framework known to be quite sensitive towards velocity model \cite[]{Nemeth99}.
On the other hand, path-summation integral provides a velocity-model-independent weighting of the misfit,
which is expected to increase the method's tolerance towards velocity model errors.}

% which could allow for velocity error alleviation as demonstrated by the first field data example

The second field example illustrates the efficiency of the proposed approach in a complex geological environment.
Most edge diffractions, and especially those associated with major discontinuities, are extracted and denoised. 
Some reflection energy remainders are present but are limited to locations characterized by a peculiar reflection pattern,
which was not picked up by the dip estimation tuned to perform reflection-diffraction separation over the whole area.
This is the dip estimation problem, results of which can be improved, for instance, by subdividing the area into smaller fragments, and, thus,
does not question the validity of the approach presented. The latter statement is also supported by the high performance of the developed approach when applied to \old{a synthetic
and}the first field data example. First field data example is characterized by a low signal-to-noise ratio and is highly contaminated by the acquisition footprint. The approach
allows \old{to}successfully \old{untangle}\new{untangling} edge diffractions from reflections and noise and provides high resolution images of subsurface discontinuities. 
\new{We expect that in production-like environment, geological knowledge can be used to further adjust the parameter values. 
For instance, in the first field data example
expected channel sinuosity could guide the smoothing strength for the edges.  
In this paper, we focus on highlighting the advantages of the method rather than on delivering final results for drilling decisions.}

%\old{In general, diffraction imaging can provide additional information about subsurface discontinuities. Combination of different methods providing different information
%about the subsurface structure is the key to successfull reservoir development. In the method proposed, key features to be extracted are associated with diffracting edges. The inversion
%at the same time has a set of parameters - starting from dip estimation and ending with the regularization weights - which affect the results. We want to highlight
%that utilization of the geological knowledge during the processing stage is of utmost importance for the delivery of the robust reservoir predictions. In this paper, we focus more
%on highlighting the advantages of the method than on delivering final results for drilling decisions. We suggest treating the parameters used as trial parameters.}

The natural extension of the approach is to include reflection modeling into the inversion. As demonstrated by \cite{merzlikin2019least}, both reflections and diffractions can be
inverted for by the same forward modeling operator whereas the separation into the components can be done on the regularization level.
Regularization of diffractions can stay the same whereas reflections, for instance, can be penalized by a strong isotropic smoothing operator along the dominant local slopes:
specular events locally do not exhibit lateral symmetry as opposed to edges. Extension
of the model space to include reflections can help to eliminate the reflection remainders in the diffraction image domain. 

\new{High complexity of the overburden often leads to the interference between seismic events, which results in the presence of
multiple dominant local slopes at a single data sample. While, in this case, the effectiveness of the proposed inversion
scheme in general and of AzPWD in particular will be degraded, the performance could be improved by pre-applying 
migration with approximate velocity model to untangle interfering events, running AzPWD, and then going back to the original
data domain.}

Anisotropic smoothing is capable of emphasizing one edge direction at once. Edges with conflicting orientations can be a challenge. The ``brute force"
way to tackle the challenge can be scanning for various edge diffraction orientations and picking the desired ones \cite[]{merzlikin2016diffraction}. Then, inversion results
with alternative orientations can be compared.
\new{At the same time, if coherent noise with a consistent spatial orientation is present in the data, it can be emphasized by anisotropic smoothing.
Structure tensor orientation determination will treat this noise as signal.
Poor illumination and velocity model errors can also reduce the accuracy of structure-tensor based
edge diffraction orientation determination. The latter will degrade the performance of anisotropic-smoothing
regularization operator. We expect the problem can be alleviated by utilizing a priori information about
geologic discontinuities' orientation, e.g. by using predominant azimuths of the faults in the region extracted
from a geomechanical model.}

\new{The workflow described in this paper is a 3D extension of the method proposed by \cite{merzlikin2016least}. In two dimensions one cannot discriminate
between point and edge diffractions. Distribution of scatterers is spiky and intermittent, which leads to a natural choice of sparsity constraints
on the diffractivity model. The new inversion scheme is based on two regularization operators: thresholding and anisotropic smoothing.
While thresholding operator imposing sparsity constraints applicable to both point and edge diffractions remains to be the same as in the
2D counterpart of the workflow, anisotropic smoothing enforces continuity along the directions picked up by the structure tensor
and thus is only applicable to edge diffractions. Point diffractions, which are not elongated in space, will be smeared under anisotropic smoothing operator action.
Currently, our method is biased towards edge diffractions.}

Anisotropic smoothing can have spatially variable diffusion coefficient defining its strength \cite[]{weickert1998anisotropic,hale2009structure}.
For instance, the coefficient and the direction of smoothing can depend on the linearity, which can be computed
as a ratio of eigenvalues of the PWD-based structure tensor and which can distinguish between the edges and regions of continuous amplitude variation \cite[]{hale2009structure,wu2017directional}.
\old{For instance, strong isotropic smoothing can be applied in areas dominated by reflections and can become anisotropic when edges are encountered}
\new{Spatially variable diffusion coefficient could help to alleviate smearing of point diffractions.
Point diffractions in the diffractivity model will exhibit low linearity. For these samples smoothing power can be reduced
and thresholding will be a predominant regularization operator.}
\old{Anisotropic smoothing can be used to regularize full wavefield images in ``conventional" least squares migration and even in iterative velocity-model building methods.}  

\old{From this point of view}The proposed approach is equivalent to total variation (TV) regularization (e.g., \cite{chambolle2010introduction}),
in which minimizing $l_1$ norm of a second derivative penalizes the model. Hessian of TV is guided by a structure tensor, \old{smoothing the updates
to the model according to the corresponding anisotropic diffusion process}\new{which forces model smoothing to be applied along the edges with no smearing across them.
Thus, TV regularization is similar to the one proposed in this paper and}\old{It} can also be \old{a viable option}\new{used} to penalize edge diffractions.
TV implementation challenges are associated
with\old{$l_1$-norm}\new{regularization term} differentiation during optimization\old{and might lead to slower convergence}\new{. While this obstacle can be accommodated by using sophisticated optimization methods
and representing $l_1$-norm as a square root with a damper \cite[]{chan1999nonlinear,burstedde2009algorithmic,anagaw2012edge},
shaping regularization by anisotropic diffusion appears to be a viable, simple to implement and fast to converge alternative with no approximations required.}  
\old{The advantages of shaping regularization by anisotropic diffusion are its simplicity to implement and fast convergence.}
\new{Anisotropic smoothing can be used to regularize full wavefield images in ``conventional" least squares migration and even in iterative velocity-model building methods.}

% good TV references
% burstedde2009algorithmic
% anagaw2012edge
% maharramov2015simultaneous

\new{The workflow can be utilized to extract and denoise diffractions for their subsequent depth imaging. Alternatively,
a depth imaging operator can replace Kirchhoff time migration in forward modeling to allow for depth-domain model conditioning, while
misfit weighting by path-summation integral is still performed in time migration domain.}

\new{The inversion can be extended to pre-stack domain.
In this case, pre-stack counterparts of the chained forward modeling operators should be used:
pre-stack path-summation integral \cite[]{merzlikin17}, pre-stack migration engine and pre-stack AzPWD.
While expressions for the former two exist, AzPWD has not been 
applied in the pre-stack domain. The corresponding method can be derived based on the approach developed by \cite{taner06}.}

\old{In general, diffraction imaging can provide additional information about subsurface discontinuities. Combination of different methods providing different information
about the subsurface structure is the key to successfull reservoir development. In the method proposed, key features to be extracted are associated with diffracting edges. The inversion
at the same time has a set of parameters - starting from dip estimation and ending with the regularization weights - which affect the results. We want to highlight
that utilization of the geological knowledge during the processing stage is of utmost importance for the delivery of the robust reservoir predictions. In this paper, we focus more
on highlighting the advantages of the method than on delivering final results for drilling decisions. We suggest treating the parameters used as trial parameters.}     

\section{Conclusions}

We have developed an efficient approach to highlight and denoise edge diffractions based on least-squares migration.
\new{The inverted operator corresponds to the chain of path-summation integral filter, AzPWD and Kirchhoff modeling operators.
While the combination of path-summation integral filter and AzPWD emphasizes edge diffraction signatures in the data domain,
thresholding and anisotropic smoothing precondition them in the model domain by denoising and enhancing their continuity.}
Both \old{the}forward modeling \new{and shaping regularization operators}\old{operator, which incorporates AzPWD and path-summation integral filter,
and regularization by thresholding and anisotropic smoothing are necessary to} guide the inversion towards restoration of edge diffractions. Synthetic and field data examples show high fidelity of the approach.

The efficiency of the proposed inversion scheme comes from the workflow application in time post-stack domain and
shaping regularization framework leading to fast convergence.
\old{The workflow can be utilized to extract and denoise diffractions for depth imaging and can be easily extended to pre-stack domain.}
The inversion scheme we propose can be thought of as an effective operator
directly tailoring edge diffractions and extracting them from the full wavefield.

\section{Acknowledgements}
We are grateful to TCCS sponsors and in particular to Equinor (formerly Statoil) for financial support of this research. We thank Evgeny Landa, Andrej Bona, Timothy Meckel, Omar Ghattas
and Luke Decker for inspiring discussions. We thank developers of and contributors to the Madagascar open source software project.
We thank four anonymous reviewers, associate editors and Jeffrey Shragge for valuable comments and help in improving the manuscript.
Pre-stack time migration velocity and preprocessed stack are provided by GeoFrac research consortium at the Australian School of Petroleum at the University of Adelaide.

%\onecolumn
\bibliographystyle{seg}
\bibliography{SEG,SEP2,sources-global}
