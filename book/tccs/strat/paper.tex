\title{Stratigraphic coordinates, a coordinate system tailored to seismic interpretation}

\lefthead{Karimi \& Fomel}
\righthead{Stratigraphic Coordinates}
\footer{TCCS-9}

\renewcommand{\thefootnote}{\fnsymbol{footnote}} 


\author{Parvaneh Karimi and Sergey Fomel}



\address{
Email: parvaneh@utexas.edu, \\
 Bureau of Economic Geology \\
John A. and Katherine G. Jackson School of Geosciences \\
The University of Texas at Austin \\
University Station, Box X \\
Austin, TX 78713-8924}
\maketitle

\begin{abstract}

In certain seismic data processing  and interpretation tasks\new{,} such as spiking deconvolution,  tuning analysis, impedance inversion,  spectral decomposition, etc., it is \new{commonly} assumed that the vertical direction is normal to reflectors. This assumption is \old{not true}\new{false} in the case of dipping layers and may \new{therefore} lead to inaccurate results\old{\mbox{ \cite[]{guomarfurt}}}. To overcome this limitation, we propose a coordinate system\old{,} in which geometry follows the shape of each reflector and the vertical direction corresponds to normal reflectivity. \new{We call this coordinate system \emph{stratigraphic coordinates}.} We \old{propose}\new{develop} a constructive algorithm that transfers seismic images into \old{a new}\new{the stratigraphic} coordinate system\new{.} \old{and }\new{The algorithm} consists of two steps. First, local slopes of seismic events are estimated by plane-wave destruction\old{,}\new{;} then structural information is spread along the estimated local slopes, and horizons are picked \new{everywhere in the seismic volume} by the predictive-painting algorithm. These picked horizons \old{can be considered as}\new{represent} level sets of the first axis of the stratigraphic coordinate system. Next, an upwind finite-difference scheme is used to find the two other axes\new{,} which are perpendicular to the first axis\new{,} by solving the \old{relevant}\new{appropriate} gradient equations. After seismic data are transformed into stratigraphic coordinates, \old{layers}\new{seismic horizons} should \old{be flattened}\new{appear flat}, and seismic traces should represent the direction normal to the reflectors. Immediate applications of the stratigraphic coordinate system are in seismic image flattening and spectral decomposition. Synthetic and real data examples demonstrate the effectiveness of stratigraphic coordinates.
     
\end{abstract}

\section{Introduction}

In certain seismic data processing  and interpretation tasks\new{,} such as spiking deconvolution,  tuning analysis, impedance inversion,  spectral decomposition, etc., it is \new{commonly} assumed that the vertical direction is normal to reflectors. This assumption \old{doesn't}\new{does not} hold true in the case of dipping layers and may \new{therefore} lead to inaccurate results \new{\cite[]{guomarfurt}}. \new{\cite{Mallet2004} defined a mathematical framework, called GeoChron, for transforming the geologic\old{al} space into a new space in which all horizons \old{are}\new{appear} flat, and faults, if any,  disappear.} In this paper, \old{to address this problem,}we propose \old{a coordinate frame}\old{,}\new{the \emph{stratigraphic coordinate} system,} in which geometry follows the shape of each reflector and the vertical direction corresponds to normal reflectivity \old{to get over this problem}.

 Flattening \old{in application to }post-stack seismic data\old{,} is an immediate \old{application}\new{use} of the proposed coordinate system. Flattened seismic images \old{can be used for improving}\new{facilitate} the interpreter's ability to extract \old{structural}\new{detailed stratigraphic} information \new{from the seismic data}. \old{\mbox{\cite{Mallet2004}}defined a mathematical framework, called GeoChron, for transforming the geological space into a new space in which all horizons are flat, and faults, if any,  disappear. }In interpretational applications, several different algorithms for image flattening have been developed by different authors. The idea of seismic image flattening by extracting stratal slices was introduced by \cite{hongliuzeng1998}. Automatic picking of horizons using local shifts was studied by \cite{bienati2001} and \cite{stark2005}. \cite{lomask2006} and \cite{parks2010} present\new{ed} inversion methods in which horizons are calculated on the basis of local slopes\old{,}  and \new{are} then \old{computed horizons}\old{they}\old{are}used to flatten seismic events. \cite{sergeyfomel2010} proposed the method of predictive painting \old{for automatic spreading of information according to the local slopes of seismic events. In this method, each trace is predicted from its neighbors, and then information is spread inside the seismic image volume by using prediction operators}\new{that uses the prediction operators extracted by plane-wave destruction to spread information inside the seismic volume recursively}. 

  
Conventionally, seismic image flattening is performed by shifting samples in the original image up or down \old{or, }\new{-} in other words, \old{stretching in some areas of the original image and squeezing in others }\new{differentially stretching and squeezing the original image in order }to flatten the reflection events\old{ in the image}. \cite{luo2013} proposed a method for image flattening\old{, which} \new{that} uses the vector shift field\old{,} instead of the scalar field of vertical shifts to define deformations in the image. Flattening by vector shift uses either \old{a }vertical shear or rotation or \old{their combination}\new{a combination of the two}, depending on the type of geologic deformation. \old{In this paper, we introduce a new framework for seismic interpretation and processing: the stratigraphic coordinate system.}

\new{The stratigraphic coordinate system, introduced in this paper, represents a new framework for seismic interpretation and processing.} To construct stratigraphic coordinates, we combine predictive painting with an upwind finite-difference scheme \cite[]{franklin2001} for solving relevant gradient equations. The stratigraphic coordinate system is semi-orthogonal\new{;} \old{and}\new{i.e.,} picked horizons that are level sets of the first axis are orthogonal to the other two axes. In other words, stratigraphic coordinates are aligned with horizons\new{}, and \new{the} vertical direction \new{in stratigraphic coordinates} corresponds to \old{normal reflectivity}\new{the direction normal to the major reflection boundaries}. Application\old{s} of the stratigraphic coordinate\old{s} \new{system} \old{are}\new{is} not limited to seismic image flattening\old{,} and may \new{be} extend\new{ed} to many data processing and interpretation tasks\old{, where} \new{in which} \new{the} vertical direction is \old{often}\new{commonly} assumed to be normal to reflection boundaries\new{: a crude assumption in all structures but flat geology}. In the following sections, we start by describing \old{our}\new{a constructive} algorithm for generating stratigraphic coordinates. We then illustrate \old{its}applications \new{of the stratigraphic coordinate system} \old{in}\new{to} seismic image flattening and spectral decomposition \old{with}\new{using} synthetic and field data examples. 


\section{Theory}


In order to define the first step for transformation to stratigraphic coordinates, we follow the predictive-painting algorithm \cite[]{sergeyfomel2010}, which is reviewed in appendix A. \new{Predictive painting spreads the time values along a reference trace into the seismic volume to output the \emph{relative geologic age} attribute $\left(Z_0(x,y,z)\right)$. }\new{The painted horizons output\old{ted} by predictive painting are used as the first axis of our stratigraphic coordinate system.} \old{Predictive painting outputs horizons $\left(Z_0(x,y,z)\right)$ as the first axis of our stratigraphic coordinate system.}\old{There are}\new{Several} \old{other }alternative methods \new{exist} to track horizons automatically in a seismic volume and produce horizon cubes \cite[]{hoyes2011,wolak2013}\old{,}\new{.} \new{We choose predictive painting because of its simplicity and efficiency.}                                                                                

 
In the next step, we find the two other axes\new{,} \new{$X_0\left(x,y,z\right)$ and $Y_0\left(x,y,z\right)$}, orthogonal to the first axis\new{,} \new{$Z_0\left(x,y,z\right)$}, by numerically solving the following gradient equations:
\begin{equation}
\label{eq:gradients}
\mathbf{\nabla} Z_0 \cdot \mathbf{\nabla }X_0 = 0
\end{equation}
and
\begin{equation}
\label{eq:gradients1}
\mathbf{\nabla} Z_0 \cdot \mathbf{\nabla} Y_0 = 0.
\end{equation}
\old{If $Z_0\left(x,y,z\right)$ is the first axis obtained from predictive painting, $X_0\left(x,y,z\right)$ and \\$Y_0\left(x,y,z\right)$ are the two other axes. }Equations \ref{eq:gradients}  and \ref{eq:gradients1} \new{simply} state that \new{the} $X_0$ and $Y_0$ axes should be perpendicular to $Z_0$.  We can define the boundary condition for the first gradient equation (equation \ref{eq:gradients}) as
\begin{equation}
\label{boundary condition}
X_0\left(x,y,0\right)=x                                                                         
\end{equation}
and the boundary condition for equation \ref{eq:gradients1} as
\begin{equation}
\label{boundary condition1}
Y_0\left(x,y,0\right)=y.                                                                         
\end{equation} 
These two boundary conditions mean that the stratigraphic coordinate system and the regular coordinate system $\left(x,y,z\right)$ \old{are}\new{become} \old{equal}\new{equivalent} at the surface $\left(z = 0\right)$. 

The stratigraphic coordinates are \new{originally} designed for depth images. When applied to time-domain images, \new{the definition of the gradient operator becomes
\begin{equation}
\label{eq:timegradient}
\mathbf{\nabla} = \left(\frac{\partial}{\partial x},\frac{\partial}{\partial y},\frac{\partial}{\partial z}\frac{\partial z}{\partial t}\right),
\end{equation}
}so a scaling factor with dimensions of velocity-squared is needed in equations \ref{eq:gradients} and \ref{eq:gradients1}.

\subsection{Algorithm}

In summary, our \old{method}\new{algorithm} for transferring seismic images from the regular Cartesian coordinate system into the stratigraphic coordinate system \old{follows}\old{can be specified as}\old{the following algorithm}\new{consists of the following steps}:

1. Extract \new{the first axis of the stratigraphic coordinate system} $Z_0\left(x,y,z\right)$ from \new{seismic image} $P\left( x,y,z \right)$ \old{(seismic image)} by predictive painting; 

2. Start with $X_0$ at $\left(z = 0\right)$ as an initial value and solve equation \ref{eq:gradients} with boundary condition \ref{boundary condition} numerically for $X_0$;

3. Start with $Y_0$ at $\left(z = 0\right)$ as an initial value and solve equation \ref{eq:gradients1} with boundary condition \ref{boundary condition1} numerically for $Y_0$.

\new{We solve} equations \ref{eq:gradients} and \ref{eq:gradients1} \old{can be solved effectively}\new{numerically} with an explicit upwind finite-difference scheme \cite[]{franklin2001,li2013}.


\section{ Examples of Stratigraphic coordinates}

For a simple illustration of \old{the}stratigraphic coordinates\old{concept}, we use a 2D synthetic seismic image from \cite{claerbout2006}\new{,} which contains \old{dipping beds with constant slope in the top part, an uncomformity, a fault, and layers with sinusoidal variation in the dip in the bottom part}\new{layers with sinusoidal dip variations, faulted and truncated by an unconformity, and dipping beds with constant slope above the unconformity} (Figure \ref{fig:sigmoid}). Figure \ref{fig:sdip} shows local slopes measured by plane-wave destruction and delineates the slope field variation.  Figure \ref{fig:t0sig} shows automatically picked horizons obtained by the predictive-painting algorithm. After solving the gradient equation \ref{eq:gradients}, we \old{add}\new{acquire} the other axis of the stratigraphic coordinate system (Figure \ref{fig:x0sig}). Figure \ref{fig:sigmoid-t0sigx0sig} shows \old{a grid of stratigraphic coordinates of the image displayed in its}\new{the stratigraphic coordinates grid overlain on the} regular Cartesian coordinates of the seismic image. \old{Geologic deformations such as faults and folds are complicated and so seismic image flattening in their presence just by vertical shifts will distort them. However, in startigraphic coordinates seismic image gets fattened by shifting samples both vertically and laterally, which can be a better representative of the geologic deformations.}\new{Complex tectonic deformations expressed in faults and folds cannot in general be undone by a simple vertical stretch and squeeze operator. In contrast, the transformation to stratigraphic coordinates allows for complex displacements, which can better capture and thus undo non-trivial tectonic deformations.} Arrows in Figure \ref{fig:shift} show\old{s} \old{corresponding shifts for different samples for transforming to stratigraphic coordinates, through using our algorithm}\new{the amount and direction of shift that it takes for different samples to be transformed from their original position in the seismic image\old{,} to their corresponding positions in the flattened image through \old{using}\new{the} stratigraphic coordinates algorithm}. For comparison\new{,} arrows in Figure \ref{fig:shift0} represent how different samples shift \old{using}\new{under} conventional flattening methods.  The seismic image gets flattened \old{by transferring}\new{when} the data from the regular coordinate system \new{are transferred} to stratigraphic coordinates. The result is shown in Figure \ref{fig:sigmoid2}. Apart from the \old{fault and structural discontinuities}\new{structural (fault) and stratigraphic (erosional truncation) discontinuities}, the input image is successfully flattened. Figure \ref{fig:sigmoid1} shows that by returning from stratigraphic coordinates to regular coordinates, \new{one can recostruct} the features of the original image\old{ can be reconstructed} effectively. 

Figure \ref{fig:win} shows the input \new{image} for a field-data test reproduced from \cite{lomask2006} and \cite{sergeyfomel2010}. The input is a depth-migrated 3-D image with structural folding and angular unconformities.  The three axes of the stratigraphic coordinates are shown in Figure \ref{fig:t0real1,x0real1,y0real2}. Figure \ref{fig:coord} displays \new{the image in the regular Cartesian coordinates overlain by its stratigraphic coordinates grid}\old{these three axes of the stratigraphic coordinates relative to the regular coordinates of the image}. The flattened image in the stratigraphic coordinate system, shown in Figure \ref{fig:win1}, can be transferred back to the regular Cartesian coordinates to reconstruct the original image (Figure \ref{fig:win2}). 



\inputdir{sigmoid1}
\multiplot{2}{sigmoid,sdip}{width=0.46\textwidth}
{(a) Synthetic seismic image from \cite{claerbout2006}. (b) Local slopes measured by plane-wave destruction.}


\inputdir{sigmoid1}
\multiplot{2}{t0sig,x0sig}{width=0.46\textwidth}
{(a) First axis ($Z_0$) of stratigraphic coordinates in the regular coordinates obtained by predictive painting. (b) \old{and }Second axis ($X_0$) of the stratigraphic coordinate system acquired by solving the gradient equation (equation \ref{eq:gradients}).}


\inputdir{sigmoid1}
\multiplot{3}{sigmoid-t0sigx0sig,shift,shift0}{width=0.46\textwidth}
{(a) Two axes of the stratigraphic coordinate system relative to the regular coordinate system.  The synthetic image with corresponding shift represented by red arrows through using stratigraphic coordinates (b)\old{,} and \old{by using }conventional flattening methods (c).}


\inputdir{sigmoid1}
\multiplot{2}{sigmoid2,sigmoid1}{width=0.46\textwidth}
{(a) Synthetic image from Figure \ref{fig:sigmoid} flattened by transferring the image to the stratigraphic coordinate system. (b) \new{Unflattened} synthetic image reconstruction by returning from stratigraphic coordinates to regular coordinates.}
 

\inputdir{three-d}
\multiplot{3}{win,wdip1,wdip2}{width=0.46\textwidth}
{(a) North Sea image from \cite{lomask2006} and its inline (b) and cross-line (c) slopes estimated by plane-wave destruction.}



\inputdir{three-d}
\multiplot{3}{t0real1,x0real1,y0real2}{width=0.46\textwidth}
{(a) First axis ($Z_0$), (b) second axis ($X_0$), and (c) third axis ($Y_0$) of stratigraphic coordinates in the North Sea image.}


\inputdir{three-d}
\multiplot{3}{coord,win1,win2}{width=0.46\textwidth}
{(a) Three axes of stratigraphic coordinates of Figure \ref{fig:win} plotted as a grid in their Cartesian coordinates. (b) North Sea image after flattening (transferring image to stratigraphic coordinates). (c) North Sea image reconstruction by returning from stratigraphic coordinates to regular coordinates.}

%\inputdir{.}
%\multiplot{4}{slice-310,slice-110,slice-320,slice-120}{width=0.46\textwidth}
%{(a) Three axes of stratigraphic coordinates of Figure \ref{fig:win} plotted as a grid in their Cartesian coordinates. (b) North Sea image after flattening (transferring image to stratigraphic coordinates)%. (c) North Sea image reconstruction by returning from stratigraphic coordinates to regular coordinates.}

%\multiplot{4}{slice-330,slice-130,slice-340,slice-140}{width=0.46\textwidth}
%{(a) Three axes of stratigraphic coordinates of Figure \ref{fig:win} plotted as a grid in their Cartesian coordinates. (b) North sea image after flattening (transferring image to stratigraphic coordinates)%. (c) North see image reconstruction by returning from stratigraphic coordinates to regular coordinates.}

\section{Application of stratigraphic coordinates to spectral decomposition}

\old{In this section, we discuss}\new{Improving the accuracy of spectral decomposition is} one of the possible applications of the stratigraphic coordinate system. 
Spectral decomposition is a window-\new{based} analysis to characterize the reflecting wavelet of an interpretation target and refers to any method that produces a continuous time-frequency analysis of a seismic trace \cite[]{partyka1999}. According to the convolutional model, seismic traces are considered as normal-incidence 1D seismograms, which is true in the case of horizontal layers and allows for capturing \old{true}\new{the} signal wavelet while performing spectral decomposition along the seismic trace. However, when the subsurface \old{has}\new{exhibits} dipping layers, \new{the} convolutional model no longer \old{will hold}\new{holds} true, \old{and sampling through the analysis window reveals a biased signal wavelet and not the true one}\old{because of sampling the seismic waveform vertically instead of normal to the reflector, which introduces a bias in spectral decomposition.}\new{and \old{vertically }sampling the seismic waveform \new{vertically} instead of perpendicularly to reflectors\old{,} introduces a dip-dependent stretch that will carry over to any frequency estimation or spectral decomposition.} \cite{guomarfurt} proposed to solve this problem by sampling the signal wavelet along the ray-path on which the wavelet travels. \old{Since}\new{Because} this path is normal to reflectors, we \old{propose to}\old{frame this problem}\old{implement this idea}\new{implement the same idea} \old{using}\new{by employing} the stratigraphic coordinate system\new{,} \old{in }which \new{honors} \new{the} convolutional model\old{ assumption is more accurate} and \old{one }can \old{capture non-biased signal wavelet through the analysis window}\new{capture and analyze seismic waveforms perpendicularly to seismic reflectors}. Figure \ref{fig:chev} shows \old{data from the}Gulf of Mexico \new{data} reproduced from \cite{lomask2006} \new{and \cite{liu2011}} \new{that contain a salt dome and horizons that \old{are}\new{dip} steeply \old{dipping }on the flank of the dome \old{due to}\new{because of} the salt piercement}\old{in which, three axes of its stratigraphic coordinate system are plotted in its Cartesian coordinate system}. \new{Following \cite{liu2011}} we calculated the spectral decomposition of \new{the} data \old{\mbox{[]{liu2011}}}in the Cartesian coordinate system. Figure \ref{fig:slice-310,slice-320,slice-330,slice-340} shows horizon slices from spectral decomposition calculated in the Cartesian coordinate system at different frequencies \old{since different frequency slices reveal different stratigraphic features.}\new{because depositional elements of different thicknesses tune it at different frequencies}. Figure \ref{fig:slice-110,slice-120,slice-130,slice-140} also shows the same horizon slices as Figure \ref{fig:slice-310,slice-320,slice-330,slice-340}\new{,} but this time from spectral decomposition calculated in \new{the} stratigraphic coordinate system. Compar\old{ing}\new{ed} with \new{horizon slices in} Figure \ref{fig:slice-310,slice-320,slice-330,slice-340}, \old{horizon slices}\new{those} from spectral decomposition calculated in \new{the} stratigraphic coordinate system \old{can}better highlight \new{detailed} geologic features\old{,} such as sand channels.\new{ That is because\old{,} in the stratigraphic coordinates\old{,} seismic horizons get flatten\new{ed} and the vertical direction corresponds to the normal direction to reflectors\old{,}\new{.} \old{so w}\new{W}e can \new{therefore} analyze the unbiased seismic wavefrom and \old{the}\new{achieve a more accurate} spectral decomposition result\old{ \old{will be}\new{becomes} more accurate}. \old{However, the transformation between the Cartesian coordinates and the stratigraphic coordinates may introduce noise to our result.}}% This example proves that cautions should be taken in interpreting spectral decomposition results of dipping events.%can be other potential applications of the stratigraphic coordinate system, since these analyses 
\old{ Post-stack impedance inversion, spiking deconvolution, tuning analysis, and etc. are trace-based and have this implicit assumption that, layers are flat and the vertical direction coincide with the normal to reflectors, so seismic waveform can be sampled in the vertical direction which can cause error in the presence of dipping layers. These analyses can be other potential applications of the stratigraphic coordinate system, where the vertical direction and the normal to reflectors coincide with each other and we can sample unbiased waveform in the vertical direction.} \new{Indeed, since seismic reflectors appear flat in the stratigraphic coordinate system, (vertical) trace analysis methods such as spectral decomposition probe the unbiased seismic waveform and thus yield more accurate measurements and attributes. Conversely, methods such as post-stack seismic inversion, spiking deconvolution, tuning analysis, \old{and }etc., usually assume that layers are flat\old{, hence} \new{and} might \new{therefore lead interpreters to} incur errors in the presence of dips. The same methods, just like spectral decomposition, may benefit from being applied in the stratigraphic coordinate system and there\new{by} produce results unbiased by structural dip.}
%In order to solve this problem we use
%Spectral decomposition refers to any method that produces a continuous time-frequency analysis from a seismic trace \cite[]{castagna2003}. Generally spectral decomposition is applied on a window around the interpretation target along time axis. In practice, when the target layer is almost flat, the window analysis yields the best result. However, when layers have a strong dip, window analysis will not work because what is sampled through the time window is not the true thickness anymore, and, instead, apparent thickness is sampled. As the theory predicts, the tuning frequency extracted from spectral decomposition is inversely proportional to thickness, which means apparent thickness has a lower tuning frequency than true thickness does. \cite{guomarfurt} proposed to solve this problem by sampling the signal wavelet along the ray-path on which the wavelet travels. The ray starts from a point in the picked horizon and follows a path at the minimum travel-time, which is computed from the picked horizon as the exploding source. We solve this problem using the stratigraphic coordinate system. Figures \ref{fig:tftf22} and \ref{fig:tftf23} are time frequency analysis of the same layer with constant thickness but captured at different dips (yellow and red windows in Figure \ref{fig:sig2}). Figures \ref{fig:tftf1} and \ref{fig:tftf11} are time frequency analysis of the same layer in the flattened image at the corresponding locations (yellow and red windows in Figure \ref{fig:sig}).

%By comparison we can say that apparent thickness (yellow and red windows in Figure \ref{fig:sig2}) captured through the analysis window in the Cartesian coordinates, shows higher tuning frequency than true thickness (yellow and red windows in Figure \ref{fig:sig}) sampled in the stratigraphic coordinates. This example proves that cautions should be used in interpreting spectral decomposition results of high-dipping events.

\inputdir{spec-decom}
\plot{chev}{height=0.8\textheight}{Seismic image from Gulf of
  Mexico. (a) Time slice. (b) Inline section. (c) Cross-line section.}


\multiplot{4}{slice-310,slice-320,slice-330,slice-340}{width=0.47\textwidth}{Horizon slices from spectral decomposition at (a) 10 Hz,
  (b) 20 Hz, (c) 30 Hz, and (d) 40 Hz in Cartesian coordinate system. }

\multiplot{4}{slice-110,slice-120,slice-130,slice-140}{width=0.47\textwidth}{Horizon slices from spectral decomposition at (a) 10 Hz,
  (b) 20 Hz, (c) 30 Hz, and (d) 40 Hz in stratigraphic coordinate system. The \new{30 Hz} slice\old{of 30 Hz} \new{most clearly} displays \old{most clearly }visible channel features \new{(red arrows)}.}

%\inputdir{sigmoid1}
%\multiplot{3}{sig,tftf22,tftf23}{width=0.4\textwidth}
%{Synthetic seismic image (a) and time-frequency analysis of yellow window (b) and red one (c).}


%\inputdir{sigmoid1}
%\multiplot{3}{sig2,tftf1,tftf11}{width=0.4\textwidth}
%{Flattened seismic image (a) and time-frequency analysis of yellow window (b) and red one (c).}


\section{Discussion and Conclusions}

We have introduced the stratigraphic coordinate system, a novel framework for seismic interpretation\old{ and processing}. Our algorithm for constructing \new{the} stratigraphic coordinate system consists of two steps. In the first step, \new{we use} predictive painting \old{is used }to produce \old{a}\new{an implicit} horizon volume\old{, which leads to defining} \new{that defines} the first axis of the stratigraphic coordinates\new{, aligned with reflection boundaries (seismic horizons)}. \new{We obtain the remaining two axes of the stratigraphic coordinate system by solving the relevant gradient equations using an upwind finite-difference scheme.}\old{For finding the next axes, upwind finite difference is used to solve the relevant gradients equations.} Seismic image flattening is an immediate application of \old{the }stratigraphic coordinates. Other possible applications include post-stack impedance inversion, tuning analysis, spiking deconvolution, or any other \old{task based on assuming vertical direction as normal to reflection boundaries in the presence of dipping layers. Problems like these can be solved through using the stratigraphic coordinates because \new{the corresponding} vertical direction is normal to layers}\new{process that implicitly assumes that reflectors are flat or\old{, in other words,} that the seismic waveforms should be sampled vertically. In all \new{structures} but layercake geology, trace-based \old{analysis or }attributes can be biased in the presence of dipping layers. In contrast, the stratigraphic coordinate system offers a local reference frame naturally oriented to sample the unbiased seismic waveform\old{,} and\new{,} hence\new{,} promises to yield more accurate waveform analysis and trace attributes.}

\old{Stratigraphic coordinates are based on predictive painting algorithm, and predictive-painting obtains best results when seismic traces can be predicted by their neighbors. However, predictive painting algorithm fails to capture some of the events across structural discontinuities. Also, in the presence of crossing dips or no coherent event, predictive painting is susceptible to errors, and it requires further improvement}\new{Our implementation of stratigraphic coordinates is based on the predictive-painting algorithm, which produces \new{the} best results when traces can be predicted by their neighbors. Note that \new{the} predictive-painting algorithm may fail to capture some of the events across structural or stratigraphic discontinuities. Also, in the presence of crossing dips or \old{no }\new{in}coherent event\new{s}, predictive painting is susceptible to errors and warrants further improvement.}

\section{Acknowledgments}

We thank Stephane Gesbert and other reviewers for their valuable comments and suggestions.

\appendix
\section{Appendix A: Predictive painting}

In the most general case, \new{the} predictive-painting \new{method} \cite[]{sergeyfomel2010} can be \old{defined}\new{described} as follows:

\old{1. }Local spatially-variable inline and cross-line slopes of seismic events are estimated by \new{the} plane-wave destruction method \cite[]{sergeyfomel2002}. Plane-wave destruction originates from a local plane-wave model for characterizing seismic data, which is based on \new{the} plane-wave differential equation \cite[]{claerbout1992}\new{:}

\begin{equation}
\label{eq:localplane}
\frac{\partial P}{\partial x}
+\sigma\frac{\partial P}{\partial t}
 =0.
\end{equation}

Here $P\left( t,x \right)$ is the seismic wave-field at time $t$ and location $x$, and $\sigma$ is the local slope, which can be either constant or variable in both time and space. The local plane differential equation can \new{easily} be solved \old{easily }where the slope is constant, and \new{it} has a simple general solution

\begin{equation}
\label{eq:generalsolution}
P\left(t,x\right)=f\left(t-\sigma x\right),
\end{equation}


where $f\left(t\right)$ is an arbitrary waveform. Equation \ref{eq:generalsolution} is just a mathematical description of a plane wave. In the case of variable slopes, a local operator is designed to propagate each trace to its neighbors by shifting seismic events along their local slopes.

\old{2. A seed trace is inserted in the volume, and the information contained in that trace is spread inside the volume, thus automatically ``painting'' the data space. }By writing the plane-wave destruction operation in the linear operator notation, we have\old{:}

\begin{equation}
\label{eq:pw destruction operation}
\mathbf{r=Ds},
\end{equation}

where\old{,}  $\mathbf{s}$ is a seismic section \new{as a collection of traces} $\left(\mathbf{s}=\left[s_1 s_2 \dots s_N\right]^T\right)$, $\mathbf{r}$ is the destruction residual,  \new{and} $\tensor{D}$ is the destruction operator, defined as

\begin{equation}
\label{eq:destruction operator}
\mathbf{D}=\left[\begin{array}{ccccc}
\mathbf{I} & 0 & 0 & \dots & 0\\
\mathbf{P}_{1,2} & \mathbf{I} & 0 & \dots & 0\\
0 & -\mathbf{P}_{2,3} & \mathbf{I} & \dots & 0\\
\dots & \dots & \dots & \dots & \dots\\
0 & 0 & \dots & -\mathbf{P}_{N-1,N} & \mathbf{I}\end{array}\right].
\end{equation}

$\mathbf{I}$ is the identity operator\new{,} and $\mathbf{P}_{i,j}$ \old{is the prediction operator determined by plane-wave destruction}\new{is an operator \old{which}\new{that} predict\new{s} trace $j$ from trace $i$. By minimizing the prediction residual $\mathbf{r}$ using least-squares optimization and smooth regularization, the dominant slopes will be obtained. For $3D$ structure characterization, a pair of inline and crossline slopes, $\sigma_x(t,x,y)$ and $\sigma_y(t,x,y)$, and a pair of destruction operators, $\tensor{D}_x$ and  $\tensor{D}_y$, are required}. The prediction of trace $\mathbf{s}_k$ from reference trace $\mathbf{s}_r$ can be defined as $\mathbf{P}_{r,k}\mathbf{s}_r$, where
 
\begin{equation}
\label{eq:predictive}
\mathbf{P}_{r,k} = \mathbf{P}_{k-1,k}  \dots  \mathbf{P}_{r+1,r+2}  \mathbf{P}_{r,r+1}.
\end{equation}

 This is a simple recursion\new{,} and $\mathbf{P}_{r,k}$ is called the predictive-painting operator. \new{After obtaining elementary prediction operators in equation \ref{eq:destruction operator} by plane-wave destruction, predictive painting spreads the information contained in a seed trace to its neighbors by following the local slope of seismic events. In order to be able to paint all events in the seismic volume, one can use multiple references and average painting values extrapolated from different reference traces.}\old{Once we spread the time or depth coordinate along a reference trace to its neighbors by following the local slope of seismic events, the first step of our algorithm is done, and the horizons are picked.}    

%\newpage

\bibliographystyle{seg}
\bibliography{sources}

