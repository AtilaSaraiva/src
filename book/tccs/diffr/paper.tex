\published{Geophysics, 72, U89-U94, (2007)}
\title{Post-stack velocity analysis by 
separation and imaging of seismic diffractions}

\renewcommand{\thefootnote}{\fnsymbol{footnote}} 

%\ms{GEO-2007-0054.R1}

\lefthead{Fomel, Landa, \& Taner}
\righthead{Diffraction imaging}

\author{Sergey Fomel\footnotemark[1],
Evgeny Landa\footnotemark[2], and M. Turhan Taner\footnotemark[3]}

\address{
\footnotemark[1]Bureau of Economic Geology \\
John A. and Katherine G. Jackson School of Geosciences \\
The University of Texas at Austin \\
University Station, Box X \\
Austin, TX 78713-8972 \\
USA \\
\footnotemark[2]OPERA \\ 
Batiment IFR \\
rue Jules Ferry \\
64000 Pau \\
France \\
\footnotemark[3]Rock Solid Images \\
2600 South Gessner, Suite 650 \\
Houston, TX 77063 \\
USA}

\maketitle

\begin{abstract}
  Small geological features manifest themselves in seismic data in the
  form of diffracted waves, which are fundamentally different from
  seismic reflections. Using two field data examples and one synthetic
  example, we demonstrate the possibility of separating seismic
  diffractions in the data and imaging them with optimally chosen
  migration velocities. Our criterion for separating reflection and
  diffraction events is the smoothness and continuity of local event
  slopes that correspond to reflection events. \old{Our criterion} For
  optimal focusing, \old{is} \new{we develop} the local varimax
  measure. The objective\new{s} of this work \old{is} \new{are}
  \old{efficient} velocity analysis \new{implemented} in the
  post-stack domain and high-resolution imaging of small-scale
  heterogeneities. Our examples demonstrate the effectiveness of the
  proposed method for high-resolution imaging of such geological
  features as faults, channels, and salt boundaries.
\end{abstract}

\section{Introduction}

Diffracted and reflected seismic waves are
fundamentally different physical phenomena
\cite[]{TSD00-00-04090409}. Most \old{of} seismic data processing is tuned
to imaging and enhancing reflected waves, which carry
most of the information about subsurface. The value of
diffracted waves, however, should not be underestimated
\cite[]{GEO69-06-14781490}.  When seismic exploration focuses
on identifying small subsurface features (such as faults, fractures,
channels, and rough edges of salt bodies) or small changes in seismic
reflectivity (such as those caused by fluid presence or fluid flow
during \new{reservoir} production), it is diffracted waves that
contain the most valuable information.

In this paper, we develop an integrated approach for extracting and
imaging of diffracted events. We start with stacked or zero-offset data
as input and produce time-migrated images with separated and optimally
focused diffracted waves as output. The output of our processing flow
can be compared to coherence cubes
\cite[]{TLE14-10-10531058,GEO63-04-11501165}. 
While the coherence cube algorithm tries to enhance incoherent
features, such as faults, in the migrated image domain, we perform the
separation in unmigrated data, where these features appear in the form
of diffracted waves.

We also introduce diffraction-event focusing as a criterion for
migration velocity analysis, \old{different from} \new{as opposed to}
the usual ``flat-gather'' criterion used in seismic imaging. Focusing
analysis is applicable not only to multi-coverage prestack data but
also to post-stack or single-coverage data.

The idea of extracting information from seismic diffractions is not
new. \cite{GEO49-11-18691880} used forward modeling and local slant
stacks for estimating velocities from
diffractions; \cite{GEO63-03-10931100} used common-diffraction-point
sections for imaging of diffraction energy and detecting local
heterogeneities; \cite{SEG-2002-22932296} simulated diffraction
responses for enhancing velocity analysis. \cite{sava} incorporated
diffraction imaging in wave-equation migration velocity analysis.

The novelty of our approach is in integration of two essential steps:
\begin{enumerate}
\item Separating diffracted and reflected events in the data space,
\item Focusing analysis for automatic detection of migration velocities optimal for imaging diffractions. 
\end{enumerate}
We explain both steps \old{in detail} and illustrate their
application with field and synthetic datasets.

\section{Separating diffractions}

The underlying assumption that we employ for separating diffracted and
reflected events is that, in a stacked data volume, background
reflections correspond to strong coherent events with continuously
variable slopes. Removing those events reveals other coherent
information, often in the form of \new{seismic} diffractions. We
propose to identify and remove reflection events with the method of
plane-wave destruction
\cite[]{pvi,GEO67-06-19461960}. Plane-wave destruction estimates 
continuously variable local slopes of dominant seismic events by
forming a prediction of each data trace from its neighboring traces
with optimally compact non-stationary filters that 
follow seismic energy along the estimated slopes. Minimizing the
prediction residual while constraining the local slopes to vary
smoothly provides an optimization objective function analogous to
differential semblance \cite[]{GEO56-05-06540663}. 
Iterative optimization of the objective function generates a field of
local slopes. The prediction residual then contains all events,
including seismic diffractions, that do not follow the dominant
slope pattern. An analogous
idea, but with implementation based on prediction-error filters, was
previously discussed by \cite{SEG-1994-1572} and
\cite{SEG-1996-0302}. \new{Although} separation of reflection and
diffraction energy can never be exact, \old{Nevertheless} our method
serves the practical purpose of enhancing the wave response of small
subsurface discontinuities.

\section{Imaging diffractions}

How can one detect the spatially-variable velocity \new{necessary} for
\old{which the} focusing of different \new{diffraction} events\old{ occurs}? A good measure of
focusing is the \emph{varimax norm} used by \cite{wiggins} for
minimum-entropy deconvolution and by \cite{GEO52-01-00510059} for
zero-phase correction. The varimax norm is defined as
\begin{equation}
  \label{eq:focus} \phi = \frac{\displaystyle N\,\sum_{i=1}^N
  s_i^4}{\displaystyle \left(\sum_{i=1}^{N} s_i^2\right)^2}\;,
\end{equation}
where $s_i$ are seismic signal amplitudes inside a window of
size $N$. Varimax is simply related to kurtosis of zero-mean signals.

Rather than working with data windows, we turn focusing into a
continuously variable attribute using the technique of
\emph{local attributes} \cite[]{attr}. Noting that the correlation
coefficient of two sequences $a_i$ and $b_i$ is defined as
\begin{equation}
  c[a,b] = {\frac{\displaystyle \sum_{i=1}^N a_i\,b_i}{\displaystyle \sqrt{\sum_{i=1}^N a_i^2\,\sum_{i=1}^N b_i^2}}}
  \label{eq:c}
\end{equation}
and the correlation of $a_i$ with a constant is
\begin{equation}
  c[a,1] = {\frac{\displaystyle \sum_{i=1}^N a_i}{\displaystyle \sqrt{N\,\sum_{i=1}^N a_i^2}}}\;,
  \label{eq:c2}
\end{equation}
one can interpret the varimax measure in equation~\ref{eq:focus} as
the inverse of the squared correlation coefficient between $s_i^2$ and
a constant: $\phi = 1/c[s^2,1]^2$. Well-focused signals \old{will
clearly} have low correlation with a constant and
\new{correspondingly} high varimax.

\old{Taking it} \new{Going} further \new{toward a continuously variable focusing attribute}, notice that the squared correlation coefficient can
be represented as the product of two quantities $c[s^2,1]^2 = p\,q$,
where 
\begin{equation}
\label{eq:pq}
p=\frac{\displaystyle \sum_{i=1}^N s_i^2}{\displaystyle N}\;,\quad q=\frac{\displaystyle \sum_{i=1}^N s_i^2}{\displaystyle \sum_{i=1}^N s_i^4}\;.
\end{equation}
\new{Furthermore}, $p$ is the solution of the least-squares
minimization problem
\begin{equation}
  \label{eq:p}
  \min_p \sum_{i=1}^N \left(s_i^2 - p\right)^2\;,
\end{equation}
and $q$ is the
solution of the least-squares minimization
\begin{equation}
  \label{eq:q}
  \min_q \sum_{i=1}^N \left(1 - q\,s_i^2\right)^2\;.
\end{equation}
This allows us to define a continuously variable attribute $\phi_i$ by
using continuously variable quantities $p_i$ and $q_i$, which are
defined as solutions of regularized optimization problems
\begin{eqnarray}
  \label{eq:pt}
  \min_{p_i} 
  \left(\sum_{i=1}^N \left(s_i^2 - p_i\right)^2 + R\left[p_i\right]\right)\;, \\
   \label{eq:qt}
   \min_{q_i}
   \left(\sum_{i=1}^N \left(1 - q_i\,s_i^2\right)^2 + R\left[q_i\right]\right)\;,
\end{eqnarray}
where $R$ is a regularization operator designed to \new{avoid trivial
solutions by enforcing} \old{enforce} a desired behavior (such as
smoothness). Shaping regularization \cite[]{shape} provides a
particularly convenient method \old{of} \new{for} enforcing smoothing
in an iterative optimization scheme.

\old{To apply the local focusing measure, we scan different migration
velocities while evaluating local focusing of migrated events at each
image location. Example local focusing gathers are shown in
Figure~\ref{fig:panel}, which corresponds to the first example from
the next section. Such gathers are suitable for picking velocities
that provide best focusing of imaged diffractions.  They are analogous
to semblance panels used in prestack velocity analysis.}

\new{We apply the local focusing measure to obtain migration-velocity panels for every
point in the image. First, we follow the procedure outlined in the
previous section to replace a stacked or zero-offset section with a
section containing only separated diffractions. Next, we migrate the
data multiple times with different migration velocities. This is
accomplished by} \emph{velocity continuation}
\cite[]{GEO68-05-16621672}, a method that performs time-migration velocity analysis
by continuing seismic images in velocity with the process also called
``image waves'' \cite[]{hubral}. The velocity continuation theory
\cite[]{GEO68-05-16501661} shows that one can accomplish time migration with a set of 
different velocities by making differential steps in velocity
similarly to the method of cascaded migrations
\cite[]{GEO52-05-06180643} but described and implemented as a
continuous process. While comparable in theory to an ensemble of Stolt
migrations \cite[]{SEG-1984-S1.8,GEO57-01-00510059}, velocity
continuation has the advantage of working directly in the image
domain. It is implemented with an efficient and robust algorithm based
on the Fast Fourier Transform. 

\new{Finally, we compute $\phi_i$ for every sample point in each of the
migrated images. Thus, $N$ in equations~\ref{eq:pt} and~\ref{eq:qt}
refers to the total number of sample points in an image. The output is
\emph{focusing image gathers} (FIGs), exemplified in Figure~\ref{fig:panel}. 
A FIG is analogous to a conventional migration-velocity analysis panel
and suitable for picking migration velocities. The main difference is
that the velocity information is obtained from analysis of diffraction
focusing as opposed to semblance of flattened image gathers used in
prestack analysis.}

\section{Examples}

Three different examples illustrate applications of our method to
imaging of geological faults and irregular salt boundaries.

\subsection{Fault detection}
\inputdir{fault}

\plot{panel}{width=\columnwidth}{Focusing image gathers (\new{F}IG) for
  post-stack migration velocity analysis by diffraction focusing. Red
  colors indicate strong focusing. Superimposed black curves are
  slices of the picked migration velocity shown in
  Figure~\ref{fig:bei-pik}.}

\multiplot{2}{bei-stk2,bei-dip}{width=0.6\textwidth}{First test example. 
(a) Stacked section from a Gulf of Mexico dataset. (b) Local slopes
estimated by plane-wave destruction.}
\multiplot{2}{bei-pwd,bei-pik}{width=0.6\textwidth}{Diffraction separation. 
(a) Diffraction events separated from data in
Figure~\ref{fig:bei-stk2}. (b) Migration velocity picked from local
varimax scans after velocity continuation of diffractions.}
\multiplot{2}{bei-slc,bei-slc2}{width=0.6\textwidth}{Migrated images. 
(a) Migrated diffractions from Figure~\ref{fig:bei-pwd}. (b) Initial
data from Figure~\ref{fig:bei-stk2} migrated with velocity estimated
by diffraction imaging.}

The data for our first example are shown in Figure~\ref{fig:bei-stk2},
which displays a stacked section of a vintage Gulf of Mexico dataset
\cite[]{Claerbout.bei.95}. Diffractions caused by irregular fault
boundaries are preserved in the stack thanks to dip moveout processing
but are hardly visible underneath strong reflection
responses. Figure~\ref{fig:bei-dip} shows the dominant slope of
reflection events estimated by the plane-wave destruction
method. Numerous diffractions \old{got} \new{were} separated from
reflections by plane-wave destruction and are shown in
Figure~\ref{fig:bei-pwd}.

Figure~\ref{fig:bei-pik} shows the migration
velocity picked from focusing common-image gathers \new{(FIGs). Example FIGs} are
\old{partially} shown in Figure~\ref{fig:panel}. Figure~\ref{fig:bei-slc}
is the image of diffracted events, which collapse to collectively form
fault surfaces. Figure~\ref{fig:bei-slc2} is the image obtained by
migrating the original stack with velocities estimated from
\new{diffraction} focusing analysis. In this final image, fault
surfaces align with discontinuities in seismic reflectors. \old{It}
\new{The image} compares favorably with images of the same dataset
from the conventional processing shown by \cite{Claerbout.bei.95}.

\subsection{Salt detection}
\inputdir{gom}

\multiplot{2}{gom,gom-dip}{width=0.6\textwidth}{Second test example. (a) Near-offset section from a Gulf of Mexico dataset. (b) Local slopes estimated by plane-wave destruction.}
\multiplot{2}{gom-pwd,gom-pik}{width=0.6\textwidth}{Diffraction separation. (a) Diffraction events separated from data in Figure~\ref{fig:gom}. (b) Migration velocity picked from local varimax scans after velocity continuation of diffractions.}
\multiplot{2}{gom-slc,gom-slc2}{width=0.6\textwidth}{Migrated images. (a) Migrated diffractions from Figure~\ref{fig:gom-pwd}. (b) Initial data from Figure~\ref{fig:gom} migrated with velocity estimated by diffraction imaging.}

Figure~\ref{fig:gom} shows another example, also from the Gulf of
Mexico. We used the nearest-offset section for diffraction analysis.
Plane-wave destruction estimates dominant slopes of continuous
reflection events [Figure~\ref{fig:gom-dip}] and reveals numerous
diffractions generated by rough edges of a salt body
[Figure~\ref{fig:gom-pwd}]. We used shaping regularization
\cite[]{shape} with the smoothing radius of 40-by-10 samples to
constrain the slope-estimation \old{part} \new{process}.  Focusing
analysis \new{generates} \old{provides} a time migration velocity
[Figure~\ref{fig:gom-pik}] suitable for collapsing diffractions
[Figure~\ref{fig:gom-slc}]. \old{In the final image
[Figure~\ref{fig:gom-slc2}],} Both sharp edges of the salt body and
continuous specular reflections appear in the \new{final} image
\new{[Figure~\ref{fig:gom-slc2}]}. Inevitably, prestack depth
migration (as opposed to time migration) is required to properly
position the salt boundary in depth. \old{However,} Time migration,
\new{however,} \old{brings} \new{provides} a reasonable first-order approximation computed
at a small fraction of the cost.

\subsection{Channel detection}
\inputdir{chan3d}

\multiplot{2}{zovel,zovol}{width=0.6\textwidth}{3-D synthetic test. (a) Synthetic velocity model for a channelized reservoir. (b) Modeled zero-offset data.}
\multiplot{2}{xdip,xpwd}{width=0.6\textwidth}{Diffraction separation for the 3-D synthetic test from Figure~\ref{fig:zovel,zovol}. (a) Dominant in-line slope estimated by plane-wave destruction. (b) Diffractions separated from the data.}
\multiplot{2}{zomig,pwmig}{width=0.6\textwidth}{Depth migration of the 3-D synthetic test data. (a) Migrated data. (b) Migrated diffractions.}

The third example is a 3-D synthetic dataset. The velocity model was
designed to simulate a complex sand channel geometry in a deep-water
clastic reservoir (Figure~\ref{fig:zovel}).  Including an overburden
with stochastically generated velocity fluctuations on top of the
reservoir model, we generated zero-offset data shown in
Figure~\ref{fig:zovol}. The data contain reflections from continuous
parts of the model and numerous diffractions generated by the channel
edges. Separating diffractions using in-line plane-wave destruction
(Figure~\ref{fig:xdip,xpwd}), we compare depth-migrated images of the
original data and of the separated diffractions
(Figure~\ref{fig:zomig,pwmig}). The fine details of the stacked
channel geometry are revealed by diffraction imaging.

\section{Conclusions}

We have developed a method of efficient migration velocity analysis
based on separation and imaging of seismic diffractions. The
efficiency follows from the fact that the proposed analysis is applied
in the post-stack domain as opposed to the conventional prestack
velocity analysis.  We used continuity of dominant reflections in the
zero-offset or stacked sections as a criterion for separating
reflections from diffractions. We then imaged separated diffractions
using local focusing analysis for picking \old{best} \new{optimal}
migration velocities. A prestack extension of \new{our approach was}
\old{this work is} presented by \cite{tury}.

\begin{comment}
The examples included in this abstract are from 2-D datasets. However,
the method is fully applicable in 3-D, where it can be regarded as an
alternative to coherence cube attributes. We plan to include 3-D
examples in the presentation and describe, in a companion paper
\cite[]{tury}, an extension of diffraction separation to prestack data
analysis.
\end{comment}

\section{Acknowledgments}

We thank TOTAL for partially supporting this work. The field datasets
used in this study were released by Western Geophysical, and the
synthetic reservoir model was created by Jim Jennings at the Bureau of
Economic Geology in collaboration with Chevron. \new{We are grateful
to Ken Larner for a thorough and helpful review.}

\bibliographystyle{seg}
\bibliography{SEG2005,diffr}
