\title{Omnidirectional plane-wave destruction}


\address{
\footnotemark[1] Department of Automation,\\
State Key Laboratory of Intelligent Technology and Systems\\
Tsinghua National Laboratory for Information Science and Technology \\
Tsinghua University \\
Beijing, China. 100084 \\
Email: zhonghuanchen@gmail.com. \\
\footnotemark[2]Bureau of Economic Geology, \\
Jackson School of Geosciences \\
The University of Texas at Austin \\
University Station, Box X \\
Austin, TX 78713-8924\\
Email: sergey.fomel@beg.utexas.edu.
}

\author{Zhonghuan Chen\footnotemark[1],
 Sergey Fomel\footnotemark[2], and Wenkai Lu\footnotemark[1]}


\footer{TCCS}
\lefthead{Chen, Fomel \& Lu}
\righthead{OPWD}

\maketitle

\begin{abstract}
Steep structures in seismic data may bring directional aliasing,
thus plane-wave destruction (PWD) filter 
can not obtain an accurate dip estimation.
We propose to interpret 
plane-wave construction (PWC) filter as a line-interpolating operator and 
introduce a novel circle-interpolating model.
The circle-interpolating PWC can avoid phase-wrapping problems,
and the related circle-interpolating plane-wave destruction (PWD)
can avoid aliasing problems.
We design a 2D maxflat fractional delay filter
to implement the circle interpolation,
and prove that
the 2D maxflat filter is separable in each direction.
Using the maxflat fractional delay filter in the circle interpolation,
we propose the omnidirectional plane-wave destruction (OPWD).
The omnidirectional PWD
can handle both vertical and horizontal structures.
With a synthetic example, we show
how to obtain an omnidirectional dip estimation using the proposed 
omnidirectional PWD.
An application of the omnidirectional PWD to a field dataset 
improves the results of predictive painting and event picking,
as compared to conventional PWD.
\end{abstract}

\section{Introduction}

Seismic wavefields can be approximated by local plane waves,
and the local slope field of these plane waves
has been a significant parameter in seismic analysis.
Plane-wave destruction (PWD) is one of the most popular tools 
to estimate the slope field.
The linear PWD proposed by \cite{claerbout1992earth} 
uses explicit finite differences and
obtains the slope field by least-squares estimation.  
The nonlinear PWD proposed by \cite{fomel:1946} 
uses the maxflat fractional delay filter 
\cite[]{thiran1971recursive,zhang2009maxflat}
to approximate a linear phase operator (or phase shift operator)
and to provide a polynomial equation for the local slope \cite[]{chen:2012a}. 
When the nonlinear PWD is applied iteratively,
its first iteration corresponds to the linear PWD.



Local slope field
has been widely applied in seismic applications.
In applications such as 
model parameterization \cite[]{fomel:A43,fomel:U89},
trace interpolation \cite[]{bardan1987trace}
%seislet transform \cite[]{fomel:V25},
and wavefield separation and denoising \cite[]{harlan:1869}
in prestack datasets,
seismic events usually have moderate slopes.
However, in other applications,
there might be steep or even vertical structures in the data
and the slopes might be large or even infinite.
Some common examples are:
(1) migrated datasets, where
geological structures can be steeply dipping,
and attribute analysis may require the slope field
\cite[]{marfurt:104};
(2) time slices, where azimuths
can follow any directions
\cite[ Figure 2a is a good example]{marfurt:P29};
(3) profiles with insufficient horizontal sampling interval,
causing dip aliasing problems \cite[]{barnes:264}.
\cite{halelocal} has shown that neither linear nor nonlinear PWD 
can cope with these steep structures well.
In this case, the PWD-based slope estimation may be inaccurate.

In order to handle steep structures, 
several methods have 
been proposed to improve the linear PWD:
\cite{davis:2775} and \cite{Noye2000385} introduced other finite-difference 
methods to obtain a better phase-shift response.
\cite{halelocal} applied the linear PWD 
in both horizontal and vertical directions,
to construct the directional Laplacian operator.
\cite{schleicher:P25} proposed total least-squares estimation
to improve the least-squares estimation.

In this paper, we propose to interpret
the phase shift operator in the nonlinear PWD as 
a 1D wavefield interpolator 
(vertical interpolation in a seismic trace).
In order to handle omnidirectional structures,
we introduce a 2D interpolator,
which interpolates the wavefield along
a circle instead of a vertical line.
Circle interpolation can avoid aliasing problems and enable
efficient modeling of steep structures.

We design a 2D maxflat fractional delay filter 
to implement circle interpolation.
This 2D filter can be decoupled into
a cascade of two 1D filters applied
in different directions.
Using the polynomial design of the maxflat fractional delay filter
\cite[]{chen:2012a},
we propose an omnidirectional plane-wave destruction (OPWD).
We use a synthetic example to demonstrate 
the omnidirectional modeling ability
and apply OPWD to improve events picking 
for a field dataset.



\section{From line interpolation to circle interpolation}

\inputdir{Gnuplot}
\plot{opwd}{width=0.5\textwidth}
{Interpolation in plane-wave construction: 
line-interpolating PWC interpolates the wavefield at point $A$,
while circle-interpolating PWC interpolates at point $B$.}

Considering the wavefield $u(x_1,x_2)$ observed in a 2D sampling system
and following \cite{fomel:1946}, plane-wave 
destruction can be represented in the $Z$-transform domain as
\begin{equation}\label{eq:pwd}
(1-Z_2Z_1^p)U(Z_1,Z_2)=0\;,
\end{equation}
where
$Z_1, Z_2$ are the unit shift operators in the first and second dimensions.
$U(Z_1,Z_2)$ (or $U$ for convenience) 
denotes the $Z$ transform of $u(x_1,x_2)$.
$p$ is the local slope. We call
$Z_2Z_1^p$ and $1-Z_2Z_1^p$
plane-wave constructor and destructor, respectively.
The slope $p$ has the following relationship with the dip angle $\theta$:
%\begin{equation}\label{eq:slope1}
$p=\tan \theta$.
%\end{equation}

Applying 
$Z_2Z_1^p$ at one point, 
for example, point $O$ in Figure \ref{fig:opwd},
PWC obtains the wavefield at the point
with a unit shift in the second dimension and 
$p$ unit shifts in the first dimension,
denoted by $A(x_1+p\Delta x_1,x_2+\Delta x_2)$.
As $-\pi/2 \leq \theta \leq \pi/2$, 
$p$ can be any value from $-\infty$ to $+\infty$.
That is to say, the forward plane-wave constructor $Z_2Z_1^p$
interpolates 
the wavefield along the vertical line at $x_2+\Delta x_2$.
Similarly, the backward PWC interpolates the wavefield
along the vertical line at $x_2-\Delta x_2$.

In order to handle both vertical and horizontal structures, 
we propose to modify
the plane-wave destruction in equation \ref{eq:pwd}
into the following form:
\begin{equation}\label{eq:opwd}
(1-Z_2^{p_2}Z_1^{p_1})U=0,
\end{equation}
where
$p_1,p_2$ are parameters related to the trial dip angle, as follows:
%\begin{equation}\label{eq:slope2}
%\left\{\begin{array}{l}
$p_1=r\sin \theta$, %\\
$p_2=r\cos \theta$.
%\end{array}\right.
%\end{equation}


In other words, 
we consider a circle in polar coordinates,
parameterized by the radius $r$ and the dip angle $\theta$.
Applying the new PWC $Z_2^{p_2}Z_1^{p_1}$ at point $O$, 
it obtains the wavefield at the point with 
$p_1$ unit shifts in the first dimension and
$p_2$ unit shifts in the second dimension.
That is point $B(x_1+p_1\Delta x_1,x_2+p_2\Delta x_2)$.
As $\theta$ changes,
the new PWC $Z_1^{p_1}Z_2^{p_2}$ interpolates the wavefield
along a circle with radius $r$.
We draw the interpolating circle with $r=1$ in Figure \ref{fig:opwd}.
The circle-interpolating PWC $Z_1^{p_1}Z_2^{p_2}$ 
corresponds to a 2D interpolation.
Equation \ref{eq:pwd} can also be seen as 
a special case of equation \ref{eq:opwd} when $p_2=1$.
Compared with the 1D line-interpolating method,
the main benefit of circle interpolation 
is its antialiasing ability.

\subsection{Anti-aliasing ability}

\inputdir{freq}

\multiplot{3}{lidl20,lidl50,lidl80,oidl20,oidl50,oidl80}
{width=0.29\textwidth}{
Magnitude responses of the line-interpolating PWD $1-Z_1Z_2^p$ (top)
and the circle-interpolating PWD 
$1-Z_1^{p_1}Z_2^{p_2}$ with $r=1$ (bottom):
from left to right $\theta=20^\circ,50^\circ,80^\circ$.
Here, $w1$ and $w2$ are normalized frequencies
(0.5 denotes Nyquist frequency, half the sampling frequency)
in vertical and horizontal directions.
White color denotes one and dark denotes zero.
}

We compare the line-interpolating and circle-interpolating PWD operators
in the frequency domain.
At different dip angles,
the magnitude responses of $1-Z_2Z_1^p$ and $1-Z_1^{p_1}Z_2^{p_2}$
are shown in Figure \ref{fig:lidl20,lidl50,lidl80,oidl20,oidl50,oidl80}:
When dip angle $\theta=20^\circ$, the two operators have similar responses
(Figure \ref{fig:lidl20,lidl50,lidl80,oidl20,oidl50,oidl80}a and
\ref{fig:lidl20,lidl50,lidl80,oidl20,oidl50,oidl80}d);
when 
$\theta=50^\circ$, the line-interpolating PWD become slightly aliased
(Figure \ref{fig:lidl20,lidl50,lidl80,oidl20,oidl50,oidl80}b),
while the circle-interpolating PWD 
is not aliased
(Figure \ref{fig:lidl20,lidl50,lidl80,oidl20,oidl50,oidl80}e);
as $\theta$ increases to $80^\circ$, the former is badly aliased
(Figure \ref{fig:lidl20,lidl50,lidl80,oidl20,oidl50,oidl80}c),
and the latter is still not aliased
(Figure \ref{fig:lidl20,lidl50,lidl80,oidl20,oidl50,oidl80}f).

In summary, the line-interpolating PWD 
has different frequency responses for different dip angles.
It may become aliased when the slope is large.
The circle-interpolating PWD avoids aliasing
for both small and large dip angles.

%As shown in Figure \ref{fig:phase}, when the slope is large,
%$H_1(Z_1)$ can not approximate the object linear phase $Z_1^p$ because of aliases.
%In fact, even we use the ideal linear phase $Z_1^p$, 
%the plane-wave constructor $Z_2Z_1^p$ may still badly aliased for large slopes.

%The radius of interpolating circle can control the aliasing problem:
%the larger the radius we choose,
%the worse the aliasing would be.
%In the case with $r=0.5$, there is no alias for all the three dips.


%\subsection{Anti-phase-wrapping ability}

In line-interpolating PWD,
we must design a digital filter to approximate 
the linear phase operator (or phase shift operator) $Z_1^p$.
The slope has an infinite range $[-\infty,+\infty]$.
In circle-interpolating PWD,
there are two linear phase operators $Z_1^{p_1}$ and $Z_2^{p_2}$,
related to the respective directions.
Both the slopes $p_1, p_2$ have a finite range $[-r,r]$.

Following \cite{fomel:1946},
the phase shift operators can be approximated by
the following maxflat fractional delay filter \cite[]{thiran1971recursive}:
\begin{equation}\label{eq:fd1d}
H_1(Z_1)=\frac{B(1/Z_1)}{B(Z_1)}\approx Z_1^p,
\end{equation}
where
\begin{equation}
B(Z_1)=\sum_{{k_1}=-N}^N b_{{k_1}}Z_1^{-{k_1}},
\end{equation}
$N$ is the filter order
and coefficients $b_{{k_1}}$ are polynomial functions of local slopes $p$
\cite[]{chen:2012a}:
\begin{equation}\label{eq:coef}
b_{k_1}(p)=
\frac{(2N)!(2N)!}{(4N)!(N+k_1)!(N-k_1)!}
\prod_{m=0}^{N-1-k_1}(m-2N+p)
\prod_{m=0}^{N-1+k_1}(m-2N-p).
\end{equation}


\inputdir{wrap}
\plot{wrap}{width=0.5\textwidth}
{Phase approximating performances of the maxflat fractional delay 
filter $H_1(Z_1)$ when $p=0.2,1.2,5.2$:
dash lines denote first-order filter and dotted lines denote second-order filter.}


In Figure \ref{fig:wrap}, we show the phase approximating performances
of the maxflat fractional delay filters for different slopes.  For
small slope $p=0.2$, the approximations are good, but when the slopes
become large, the phases get wrapped.  It is obvious
that the phase wrapping comes when and only when $p>1$.  The larger
the slope $p$, the more narrow
the linear-phase frequency bands become. 
%\new{Increasing $N$ solves this problem but at an additional cost.}

As mentioned above, in line-interpolating PWC,
the slope $p$ is in the infinite interval $[-\infty,+\infty]$.
For steep structures, 
where the slope $p$ becomes larger than $1$, 
there may be phase wrapping in the linear phase approximator.
However, in circle-interpolating PWC,
the ranges of $p_1,p_2$ can be easily controlled by the radius $r$.
If we choose $r\leq 1$, 
the circle-interpolating can avoid phase wrapping completely for all dip angles.

\section{2D linear phase approximation}

Similar to the line-interpolating PWD,
the 2D phase shift operator $Z_1^{p_1}Z_2^{p_2}$
can be approximated by the following 2D allpass system:
\begin{equation}
H_2(Z_1,Z_2)=
\frac{F(1/Z_1,1/Z_2)}{F(Z_1,Z_2)},
\end{equation}
where 
\begin{equation}
F(Z_1,Z_2)=
\sum_{k_1=-N}^N\sum_{k_2=-N}^N
f_{k_1k_2}Z_1^{-k_1}Z_2^{-k_2}.
\end{equation}



\subsection{2D maxflat condition}

The frequency response of the objective phase shift operator is
$Z_1^{p_1}Z_2^{p_2}=e^{\txt j(w_1p_1+w_2p_2)}$,
where, $w_1,w_2$ are frequencies in radius 
in vertical and horizontal directions.
%It has a 2-D linear phase response.
We must design the coefficients $f_{k_1k_2}$ so that 
the allpass system $H_2(Z_1,Z_2)$ can obtain a similar linear phase response.
The frequency response of $H_2(Z_1,Z_2)$ is
\begin{equation}
H_2(e^{\txt jw_1},e^{\txt jw_2})
=\frac{
F^*(e^{\txt jw_1},e^{\txt jw_2})
}{
F(e^{\txt jw_1},e^{\txt jw_2})
}=e^{-\txt j2\theta_F(w_1,w_2)},
\end{equation}
where $\theta_F(w_1,w_2)$ is the phase of $F(Z_1,Z_2)$,
which takes the following form:
\begin{equation}
\theta_F(w_1,w_2)=-\tan^{-1}\frac{
\displaystyle{\sum_{k_1=-N}^N\sum_{k_2=-N}^N
f_{k_1k_2}\sin(k_1w_1+k_2w_2)}
}{
\displaystyle{\sum_{k_1=-N}^N\sum_{k_2=-N}^N
f_{k_1k_2}\cos(k_1w_1+k_2w_2)}
}.
\end{equation}

The phase approximating error is $w_1p_1+w_2p_2+2\theta_F(w_1,w_2)$.
In order to obtain an analytical $f_{k_1k_2}$,
we remove $\tan^{-1}$ and redefine the phase approximating error as
\begin{eqnarray}
\epsilon(w_1,w_2) &=&
\tan(\frac{w_1p_1+w_2p_2}{2})
-\frac{
\displaystyle{\sum_{k_1=-N}^N\sum_{k_2=-N}^N
f_{k_1k_2}\sin(k_1w_1+k_2w_2)}
}{
\displaystyle{\sum_{k_1=-N}^N\sum_{k_2=-N}^N
f_{k_1k_2}\cos(k_1w_1+k_2w_2)}
}
\nonumber \\ &=&
\frac{
\displaystyle{\sum_{k_1=-N}^N\sum_{k_2=-N}^N
f_{k_1k_2}\sin[w_1(\frac{p_1}{2}-k_1)+w_2(\frac{p_2}{2}-k_2)]}
}{
\cos(\frac{w_1p_1+w_2p_2}{2})
\displaystyle{\sum_{k_1=-N}^N\sum_{k_2=-N}^N
f_{k_1k_2}\cos(k_1w_1+k_2w_2)}
}.
\end{eqnarray}

The sine function in the numerator can be expressed by
2D Taylor's expansion as 
\begin{eqnarray*}
&&\sin(w_1(\frac{p_1}{2}-k_1)+w_2(\frac{p_2}{2}-k_2))
\\ &=&
\sum_{j_1=0}^\infty\sum_{j_2=0}^\infty
(-1)^{j_1+j_2}\frac{
(\frac{p_1}{2}-k_1)^{2j_1+1}(\frac{p_2}{2}-k_2)^{2j_2+1}
}{(2j_1+1)!(2j_2+1)!}
w_1^{2j_1+1}w_2^{2j_2+1}.
\end{eqnarray*}

We use the maxflat phase criterion 
\cite[]{thiran1971recursive},
which means that the filter has a phase response 
as flat as the desired linear response.
In the 2D case, the criterion 
is equivalent to the mathematical expression that
the partial derivatives of the error function should be as small as possible.
We set them to be zero:
\begin{equation}
\frac{\partial^{j_1+j_2}\epsilon(w_1,w_2)}
{\partial w_1^{j_1}\partial w_2^{j_2}}=0
\qquad j_1,j_2=0,1,\dots\;,
\end{equation}
which is equivalent to the following 2D maxflat condition:
\begin{equation}\label{eq:mf2d}
\sum_{k_1=-N}^N\sum_{k_2=-N}^N
(\frac{p_1}{2}-k_1)^{2j_1+1}(\frac{p_2}{2}-k_2)^{2j_2+1}f_{k_1k_2}=0.
\end{equation}

\subsection{Additional constraint}

The above maxflat condition is a linear system
with $(2N+1)^2$ unknown variables $f_{k_1k_2}$.
In order to obtain a non-zero solution, 
we use only the first $4N^2+4N$ equations,
which zeros the partial derivatives 
with order less than $(2N+1)^2$,
and impose an additional constraint on the filter coefficients,
\begin{equation}\label{eq:pcmf}
\sum_{k_1=-N}^N\sum_{k_2=-N}^Nf_{k_1k_2}=1.
\end{equation}

This additional constraint yields a
unit amplitude response at zero frequency,
similarly to the 1D maxflat approximation \cite[]{chen:2012a}.


\subsection{Directional separability}

The combined linear system can be solved analytically.  It
is based on the property that $f_{k_1k_2}$ can be
decoupled into the product of two terms, 
as shown in the appendix,
\begin{equation}\label{eq:decouple}
f_{k_1k_2}(p_1,p_2)=b_{k_1}(p_1)b_{k_2}(p_2),
\end{equation}
where $b_{k_1}(p_1)$ and $b_{k_2}(p_2)$ denote 
coefficients of the 1D maxflat fractional delay filter used 
to approximate $Z_1^{p_1}$ and $Z_2^{p_2}$ respectively.
Equation \ref{eq:decouple} implies 
that the 2D maxflat filter $H_2(Z_1,Z_2)$ is equivalent 
to a cascade of two 1D maxflat filters,
\begin{equation}
H_2(Z_1,Z_2)=
\frac{B_1(1/Z_1)}{B_1(Z_1)}
\frac{B_2(1/Z_2)}{B_2(Z_2)}.
\end{equation}

$B_1(Z_1)$ and $B_2(Z_2)$ can be designed
in the same way as for the line-interpolating PWD,
whose coefficients are given by
equation \ref{eq:coef}.
$H_2(Z_1,Z_2)$ can be implemented by applying the 1D maxflat filter 
on each direction independently.

This separability of the maxflat linear phase filter extends to
3D and higher dimensions.

\subsection{Frequency responses}

\inputdir{freq}

We show the magnitude responses of the 
infinite impulse response (IIR)
omnidirectional PWD (OPWD)
$1-H_2(Z_1,Z_2)$ when $N=2$ in 
Figure \ref{fig:liir20,liir50,liir80,oiir20,oiir50,oiir80}d-f.
Compared with the ideal responses in 
Figure \ref{fig:lidl20,lidl50,lidl80,oidl20,oidl50,oidl80},
it has good approximations for all the three dip angles 
in most of the frequency band.
There are distortions in the high frequency bands,
due to 
phase approximation errors in $H_2(Z_1,Z_2)$ for high frequencies.
They are not significant in practice, 
because the frequency band of seismic data is temporally limited.

\multiplot{3}{liir20,liir50,liir80,oiir20,oiir50,oiir80}
{width=0.29\textwidth}{
Magnitude responses of the IIR implementation of
LPWD $1-Z_2H_1(Z_1)$ (top) and
OPWD $1-H_2(Z_1,Z_2)$ (bottom): 
from left to right $\theta=20^\circ,50^\circ,80^\circ$.
}


When $\theta=20^\circ$ or $50^\circ$,
the omnidirectional PWD has a low-frequency response similar to that
of line-interpolating PWD (LPWD)
shown in Figure \ref{fig:liir20,liir50,liir80,oiir20,oiir50,oiir80}a-b.
When $\theta=80^\circ$, the LPWD has 
aliasing in low-frequency bands
(Figure \ref{fig:liir20,liir50,liir80,oiir20,oiir50,oiir80}c),
while the OPWD exhibits a more desirable low-frequency response
(Figure \ref{fig:liir20,liir50,liir80,oiir20,oiir50,oiir80}f).

Following \cite{fomel:1946},
the IIR LPWD $1-Z_2H_1(Z_1)$ can be approximated by the following 
finite impulse response (FIR) filter:
\begin{equation}
H_3(Z_1,Z_2)=B(Z_1)-Z_2B(1/Z_1).
\end{equation}

Similarly, the IIR OPWD filter $1-H_2(Z_1,Z_2)$
can also be approximated by an FIR implementation:
\begin{equation}\label{eq:pwd:h4}
H_4(Z_1,Z_2)=F(Z_1,Z_2)-F(1/Z_1,1/Z_2).
\end{equation}

\multiplot{3}{lfir20,lfir50,lfir80,ofir20,ofir50,ofir80}
{width=0.29\textwidth}{
Magnitude response of the FIR approximation of
LPWD $H_3(Z_1,Z_2)$ (top) and
OPWD $H_4(Z_1,Z_2)$ (bottom):
from left to right $\theta=20^\circ,50^\circ,80^\circ$.
}

The frequency responses of these two FIR approximations
are shown in Figure \ref{fig:lfir20,lfir50,lfir80,ofir20,ofir50,ofir80}.
Similar to the IIR implementation, 
the FIR OPWD (Figure \ref{fig:lfir20,lfir50,lfir80,ofir20,ofir50,ofir80}d-f)
can obtain an expected low-frequency response
for all three dip angles.
While the FIR LPWD cannot obtain a good low-frequency response 
when $\theta=80^\circ$
(Figure \ref{fig:lfir20,lfir50,lfir80,ofir20,ofir50,ofir80}c). 
Compared with the IIR implementations, 
the FIR approximations of both LPWD and OPWD have 
less desirable high-frequency responses.

\section{Applications}

\subsection{Omnidirectional dip estimation}

Dip estimation by OPWD is equivalent to the parameter estimation 
of the OPWD filter in equation \ref{eq:pwd:h4}.
The desired parameters $p_1,p_2$ minimize the predictive error:


\begin{equation}
H_4(Z_1,Z_2,p_1,p_2)U \approx 0,
\end{equation}
with
\begin{equation}
p_1^2+p_2^2=r^2.
\end{equation}

We solve the above equation set by Newton's search.
At each iteration,
the increments $\Delta p_1, \Delta p_2$ are calculated 
from the following linearization:
\begin{equation}
\left[ \begin{array}{cc}
\displaystyle{\frac{\partial H_4}{\partial p_1}U} & 
\displaystyle{\frac{\partial H_4}{\partial p_2}U} \\ 
2p_1 & 2p_2 \\
\end{array}\right]
\left[ \begin{array}{c} 
\Delta p_1 \\ \Delta p_2 
\end{array}\right] \approx
\left[ \begin{array}{c} 
-H_4U \\
r^2-p_1^2-p_2^2
\end{array}\right]\;.
\end{equation} 
Both $p_1$ and $p_2$ are updated until the residual of OPWD becomes 
less than a specified tolerance.
Similarly to the line-interpolating PWD,
when we are solving for $\Delta p_1$ and $\Delta p_2$, 
either Tikhonov's regularization \cite[]{fomel:1946} or
shaping regularization \cite[]{fomel:R29} 
can be applied to obtain a robust estimation.

In order to test the performance of OPWD,
we generate a 2D synthetic image 
in polar coordinates using the following equation:
\begin{equation}\label{eq:chirp}
u(r)=e^{-\alpha r}\sin(2\pi (f_0+\beta r)r).
\end{equation}

Structures in this image have constant dip angles along radial directions,
but the frequency and magnitude vary with radius.
Figure \ref{fig:circle,odip,oerr,lerr}a shows a test example 
with $f_0=0.02~,\alpha=0.01,\beta=0.0005$. 
We apply the 
first-order ($N=1$) OPWD filter to this test image.
With the starting dip $\theta_0=\frac{\pi}{4}$,
we obtain an estimation result in 10 iterations,
as shown in Figure \ref{fig:circle,odip,oerr,lerr}b.
Obviously, dip estimation by OPWD is effective 
for both horizontal and vertical structures.

\inputdir{circle}
\multiplot{2}{circle,odip,oerr,lerr}
{width=0.4\textwidth}{
An omnidirectional dip estimation example:
(a) test image;
(b) dip angles estimated by the OPWD filter;
(c) estimation error of OPWD;
(d) estimation error of LPWD.
}


We compare the proposed OPWD with LPWD by 
estimating errors in Figures
\ref{fig:circle,odip,oerr,lerr}c and \ref{fig:circle,odip,oerr,lerr}d.
As the test image is noise-free, 
we do not need to use any smoothing regularization 
in both of these methods.
In this case, 
the modeling errors are the dominate factor of the final estimating errors.
Obviously, the LPWD cannot detect 
the vertical structure accurately,
while the OPWD can. 


\subsection{Improved predictive  picking}

Event picking is a useful tool for interpretation.
However, when there are
salt bodies and buried hills in our images,
picking becomes difficult.
Using some additional information,
such as cross-correlation analysis \cite[]{yung:1947},
high-order statistics \cite[]{GPR:GPR231}
or local slope fields \cite[]{fomel:A25},
automatic picking tools can be improved.
A challenge for automatic picking is 
therefore the robust estimation of the statistics 
or slope fields.
This challenge 
is especially difficult near boundaries of salt bodies and buried hills 
because of steep structures. 
In this paper, we use the slope-based predictive painting method 
\cite[]{fomel:A25} to pick events.
We use the proposed OPWD to estimate the slope field,
and show how it improves event-picking results.


In Figure \ref{fig:data},
we show a profile of a post-stack dataset from an
oilfield in China.
We can see a buried hill in this area,
and there are several steep boundaries and faults 
near the buried hill.
The slope field estimated by OPWD is shown in Figure \ref{fig:odip,dip}a.
Compared with the slope by LPWD in Figure \ref{fig:odip,dip}b,
the OPWD yields a larger estimated slope
near the 360-th trace and the 475-th trace.
As shown in Figure \ref{fig:data}, 
there exist faults and steep buried-hill boundaries 
at these two positions, respectively.
In this application, we use the shaping regularization with 
25-points window size.
We choose the 400-th and 560-th traces
as two reference traces 
and use the two slope fields in 2D predictive painting.
Painting results are 
shown in Figure \ref{fig:pred-odip,pred-dip}a
and Figure \ref{fig:pred-odip,pred-dip}b.
Meanwhile, we pick some of the events,
as shown in Figure \ref{fig:pick-odip,pick-dip}.
Compared with the painting and picking results based on the LPWD,
the OPWD provides more reasonable results 
near the fault (360-th trace) and near 
the buried-hill boundaries (475-th trace).

\inputdir{pick}

\plot{data}
{width=0.4\textwidth}{
Field data example: 
steep structures and faults are rich in this selected area.
}

\multiplot{2}{odip,dip}
{width=0.4\textwidth}{
Slope field estimated by OPWD (a) and LPWD (b).
In both the two methods, we choose $N=2$ and 
use 25 points regularization window in each directions. 
}


\multiplot{2}{pred-odip,pred-dip}
{width=0.4\textwidth}{
Predictive painting results using the OPWD slope (a)
and LPWD slope (b):
the reference traces are the 400-th and 560-th traces.
}

\multiplot{2}{pick-odip,pick-dip}
{width=0.4\textwidth}{
Predictive picking results using the OPWD slope (a)
and LPWD slope (b).
}

\section{Conclusions}

We have addressed the problem of steep slope estimation
by introducing a novel circle-interpolating method in plane-wave construction, 
which leads to omnidirectional plane-wave destruction.
Frequency analysis demonstrates the ability 
of the proposed method to handle steep structures.
We presented both synthetic and field data examples to show 
the performance of the omnidirectional dip estimation.
The new estimation method can also be used in other applications 
that require slope fields for steep structures,
such as attribute analysis on
a migrated image or a time slice.

As a byproduct, 
we proved that the high-dimensional maxflat fractional delay filter 
is separable in all directions.
A typical $m-$dimensional $n-$th order 
($n=2N+1$ in this paper) linear-phase 
approximating system is supported by $n^m$ coefficients.
Using the separability property,
the maxflat linear-phase
approximating system has only $mn$ coefficients.
It is useful in applications that require 
high-dimensional linear-phase approximations.

\section*{Acknowledgment}
The joint research is sponsored by by China Scholarship Council and
China State Key Science and Technology Projects (2011ZX05023-005-007).
We thank the reviewers and the Associate Editor for
their helpful suggestions.

\appendix
\section{Appendix: Separability of the 2D maxflat filter}
\label{apd:decouple}

Following \cite{thiran1971recursive}, 
the maxflat condition of 1D fractional delay
filter $B_1(1/Z_1)/B_1(Z_1)$ is expressed as
\begin{equation}
\sum_{k_1=-N}^N(\frac{p_1}{2}-k_1)^{2j_1+1}b_{k_1}(p_1)=0\;,
\qquad j_1=0,\dots,2N-1.
\end{equation}

With the additional constraint \cite[]{chen:2012a}
\begin{equation}
\sum_{k_1=-N}^Nb_{k_1}(p_1)=1,
\end{equation}
we can obtain the unique $b_{k_1}(p_1)$.
Similarly, $b_{k_2}(p_2)$ satisfies 
the following linear system:
\begin{equation}
\left\{\begin{array}{l}
\displaystyle{\sum_{k_2=-N}^N(\frac{p_2}{2}-k_2)^{2j_2+1}b_{k_2}(p_2)=0}\;,
 \quad j_2=0,\dots,2N-1 \\
\displaystyle{\sum_{k_2=-N}^Nb_{k_2}(p_2)=1}
\end{array}\right..
\end{equation}

Substituting $b_{k_1}(p_1)b_{k_2}(p_2)$ into 
the 2D maxflat equation \ref{eq:mf2d}, for all
$j_1,j_2=0,\dots,2N-1$, we obtain
\begin{equation}
\sum_{k_1=-N}^N(\frac{p_1}{2}-k_1)^{2j_1+1}b_{k_1}(p_1)
\sum_{k_2=-N}^N(\frac{p_2}{2}-k_2)^{2j_2+1}b_{k_2}(p_2)=0.
\end{equation}
Also, for
$j_1=2N$, and $j_2=0,\dots,2N-1$ 
or $j_2=2N$, and $j_1=0,\dots,2N-1$,
equation \ref{eq:mf2d} still holds true.

Substituting $b_{k_1}(p_1)b_{k_2}(p_2)$ into 
the additional constraint in Equation \ref{eq:pcmf},
\begin{equation}
\sum_{k_1=-N}^Nb_{k_1}(p_1)\sum_{k_2=-N}^Nb_{k_2}(p_2)=1.
\end{equation}
In other words, 
$b_{k_1}(p_1)b_{k_2}(p_2)$ is a solution of 
the combined linear system of equation \ref{eq:mf2d} and \ref{eq:pcmf}.
This linear system must have a unique solution, therefore
\begin{equation}
f_{k_1k_2}(p_1,p_2)=b_{k_1}(p_1)b_{k_2}(p_2).
\end{equation}

\bibliographystyle{seg}
\bibliography{opwd}

