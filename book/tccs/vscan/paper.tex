
\title{Velocity analysis using similarity-weighted semblance}

%\renewcommand{\thefootnote}{\fnsymbol{footnote}}
\renewcommand{\thefootnote}{\fnsymbol{footnote}}

\author{Yangkang Chen\footnotemark[1], Tingting Liu\footnotemark[2] and Xiaohong Chen\footnotemark[2]}

\ms{GEO-2014-0618}

\address{
\footnotemark[1]Bureau of Economic Geology \\
John A. and Katherine G. Jackson School of Geosciences \\
The University of Texas at Austin \\
University Station, Box X \\
Austin, TX 78713-8924 \\
ykchen@utexas.edu\\

\footnotemark[2] State Key Laboratory of Petroleum Resources and Prospecting \\
China University of Petroleum \\
Fuxue Road 18th\\
Beijing, China, 102200 \\
liutingjiy@126.com \& chenxh@cup.edu.cn\\
%+86-18810415565 
}

\footer{TCCS-9}
\lefthead{Chen et al.}
\righthead{Similarity-weighted semblance}

\maketitle

\begin{abstract}
Weighted semblance can be used for improving the performance of the \old{conventional semblance}\new{traditional semblance} for specific datasets. We propose a novel approach for prestack velocity analysis using weighted semblance. The novelty comes from a different weighting criteria in which the local similarity between each trace and a reference trace is used. On one hand, low similarity corresponds to a noise point or a point indicating incorrect moveout, which should be given a small weight. On the other hand, high similarity corresponds to a point indicating correct moveout, which should be given a high weight. %Synthetic and field datasets show that the proposed weighting approach can obtain significantly higher resolution compared with the conventional semblance and AB semblance. 
The proposed approach can also be effectively used for analyzing AVO anomalies with increased resolution compared with AB semblance. Both synthetic and field CMP gathers demonstrate \old{obviously }higher resolution using the proposed approach. Applications of the proposed method on a prestack dataset further confirms that the stacked data using the similarity-weighted semblance can obtain better energy-focused events, which indicates a more precise velocity picking.
\end{abstract}


\section{Introduction}
Building \new{a} subsurface velocity model is one of the most important issues in exploration geophysics. There are generally four ways for building the velocity model. The first one is normal-moveout (NMO) based velocity analysis, which requires picking peaks in the velocity spectra \cite[]{taner2009,fomel20091,luosimon2012}. The velocity spectra is obtained by applying a number of NMO\old{s} \new{corrections} with different velocities and then calculating their corresponding semblances. The second type  is \new{Born-approximation-based} wave equation \old{based} migration velocity analysis (WEMVA), which is a non-linear optimization process \old{which}\new{that} aims at estimating the migration velocity using \new{the Born-approximated} wave equation \new{in the image domain} \new{and is less vulnerable to the local minimum and more effective for reflection signals} \cite[]{sava20041,sava20042,siwei2013}. %The flatness of the subsurface angle-domain commonimage gathers (ADCIGs) is currently a popular choice to forming the objective functions \cite[]{}. 
The third type is ray-based migration velocity analysis, also known as traveltime tomography. The \old{process is achieved}\new{velocity model is built} by updating \old{a}\new{the} velocity model \old{such}\new{so} that the misfit\old{s} between predicted and observed first\new{-}break\old{s} \new{traveltimes} \old{decrease}\new{is minimized} \cite[]{zhu1992,osypov2000,noble2010,siwei20131,jianxiong2013}. The fourth type is the recently popular full waveform inversion (FWI), which \old{uses the complete seismic record to invert the properties of the Earth}\new{minimizes the least-squares misfit between the measured data and the synthesized data predicted from the current velocity model in the data domain} \cite[]{virieux2009,antoine2012,honbo2012} and \old{thus} can improve the resolution of velocity structures. %There are close connection between these four approaches, however, we just 
In this paper, we focus on the NMO based velocity analysis \new{, which provides the important initial velocity model for the following more complicated velocity analysis procedures}.

NMO based velocity analysis using \old{the} semblance \old{measure} has been an indispensable step for building the initial subsurface velocity model since its introduction by \cite{taner2009}. There exist several \old{modified types from the}\new{modifications of} \old{conventional semblance}\new{traditional semblance} \old{for}\new{that} better meet\old{ing} the requirements \old{for}\new{of} specific datasets \cite[]{fomel20091,sarkar2001,luosimon2012}. \cite{sarkar2001} and \cite{fomel20091} modified the \old{conventional semblance}\new{traditional semblance} formulation and proposed amplitude-versus-offset (AVO) adaptive semblances in different ways. \cite{luosimon2012} increase the resolution of the semblance map in order to distinguish the peaks between primary and multiple reflections.


Increasing the resolution of the semblance spectra is beneficial for picking the true NMO velocity, especially important when multiples are existing in the CMP gathers \cite[]{luosimon2012}. We can increase the resolution by weighting terms in the \old{conventional semblance}\new{traditional semblance} calculation.  \cite{hale2009} define\old{s}\new{d} a weighted semblance \old{coefficient} that conform\old{s}\new{ed} to structural features. \cite{luosimon2012} \old{choose}\new{chose} an offset-dependent weighting function that \old{minimizes semblance}\new{effectively increased the semblance spectra resolution}. In this paper, we use a different weighting scheme, by utilizing the local similarity between each trace and a reference trace. The reference trace can be a zero-offset trace, or the stacked trace using a traditional stacking technique. Both synthetic and field data examples demonstrate a\old{n} \old{excellent}\new{successful} performance of the proposed approach. 

\section{Theory}
\subsection{\old{conventional semblance}\new{Traditional semblance}}
The \old{conventional semblance}\new{traditional semblance} is defined by \cite{neidell1971} as:
\begin{equation}
\label{eq:semb}
s(i) = \frac{\displaystyle\sum_{j=i-M}^{i+M}\left(\sum_{k=0}^{N-1}a(j,k)\right)^2}{\displaystyle N\sum_{j=i-M}^{i+M}\sum_{k=0}^{N-1}a^2(j,k)},
\end{equation}
where $i$ and $j$ are time sample indices, $s(i)$ denotes the semblance for time index $i$, $2M+1$ is the length of the smoothing window (in this formulation a boxcar filter is used), along time axis, \old{and} $a(j,k)$ is the trace amplitude at time index $j$ and trace number $k$ of the NMO-corrected common midpoint (CMP) gather\new{, and $N$ is the number of traces}.


\subsection{Weighted semblance}
The semblance can be thought as a squared correlation between the analyzed signal and a weighting function $w(j,k)$ in the following form:
\begin{equation}
\label{eq:semb-weight}
s_w(i) = \frac{\displaystyle\sum_{j=i-M}^{i+M}\left(\sum_{k=0}^{N-1}a(j,k)w(j,k)\right)^2}{\displaystyle \sum_{j=i-M}^{i+M}\left(\sum_{k=0}^{N-1}a^2(j,k)\sum_{k=0}^{N-1}w^2(j,k)\right)},
\end{equation}
where $w(j,k)$ denotes the weighting function for time index $j$ and trace number $k$.

With $w$ selected using different criterion, equation \ref{eq:semb-weight} can lead to different well-known semblance types:
\begin{enumerate}
\item When $w(j,k)\equiv C$, where $C$ is a constant, equation \ref{eq:semb-weight} turns to the \old{conventional semblance}\new{traditional semblance} \ref{eq:semb}. 
\item When \old{$w(j,k)= A(j,k) + B(j,k)\phi(j,k)$}\new{$w(j,k)= A(j) + B(j)\phi(j,k)$}, where $\phi(j,k)$ is a known function, $A(j)$ and $B(j)$ are two coefficients estimated from the least-squares fitting for the trend function $w(j,k)$, equation \ref{eq:semb-weight} turns to the AB semblance proposed by \cite{fomel20091}. The AB semblance is proposed to handle the velocity analysis for those CMP gathers with AVO phenomenon. \new{Appendix A provides a short review of calculating $A(j)$ and $B(j)$ in the AB semblance.}
\end{enumerate}
More recently, \cite{luosimon2012} proposed another weighting approach which honors the large-offset data and damping the small-offset data to increase the resolution for the resulted velocity scanning map.
In this abstract, we propose to use local similarity \cite[]{fomel2007localattr} to weight different traces:
\begin{equation}
\label{eq:simiweight}
w(j,k)=\mathcal{\mathbf{S}}[a(j,k),r(j)],
\end{equation}
where $\mathcal{\mathbf{S}}[\mathbf{x},\mathbf{y}]$ denotes the local similarity between traces $\mathbf{x}$ and $\mathbf{y}$, $r(j)$ denotes the $j$th time point for a selected reference trace $\mathbf{r}$. The reference can be easily selected as the zero-offset trace, or more plausible, as the stacked trace using the conventional stacking technique or a more sophisticated stacking technique \cite[]{guochang2009}. \new{In most cases, when the signal-to-noise ratio (SNR) of the CMP gather is not low, the choice of reference trace will not affect the performance too much. However, when the CMP gather is very noisy, the stacked trace using a better stacking technique can help obtain a better focused and smoother semblance map.}

Figure \ref{fig:syn,weights} gives a brief demonstration of the proposed similarity-weighted semblance. Figure \ref{fig:syn} is a simple NMO corrected CMP gather with some Gaussian white noise. There are two abnormal traces in the gather. For traditional semblance, the two abnormal traces are given equal weights, which is not plausible. Figure \ref{fig:weights} shows the weights applied to each trace for the semblance calculation. The weights equal to the local similarity between each trace and the stacked trace.
As we can see from the map of weights, the two abnormal traces are given much low weights. The more reasonable semblance calculation will result in a much higher resolution, which will be exclusively demonstrated in the \old{following }section \new{of examples}. \new{In the next section, we will give a brief review of the theory of local similarity.}

\new{\section{Review of local similarity}
\cite{fomel2007localattr} defined local similarity between vectors $\mathbf{a}$ and $\mathbf{b}$ as:
\begin{equation}
\label{eq:local}
\mathbf{c}=\sqrt{\mathbf{c}_1^T\mathbf{c}_2}
\end{equation}
where $\mathbf{c}_1$ and $\mathbf{c}_2$ come from two least-squares minimization problems:
\begin{align}
\label{eq:local1}
\mathbf{c}_1 &=\arg\min_{\mathbf{c}_1}\Arrowvert \mathbf{a}-\mathbf{B}\mathbf{c}_1 \Arrowvert_2^2, \\
\label{eq:local2}
\mathbf{c}_2 &=\arg\min_{\mathbf{c}_2}\Arrowvert \mathbf{b}-\mathbf{A}\mathbf{c}_2 \Arrowvert_2^2,
\end{align}
where $\mathbf{A}$ is a diagonal operator composed from the elements of $\mathbf{a}$: $\mathbf{A}=diag(\mathbf{a})$ and $\mathbf{B}$ is a diagonal operator composed from the elements of $\mathbf{b}$: $\mathbf{B}=diag(\mathbf{b})$.
Least-squares problems \ref{eq:local1} and \ref{eq:local2} can be solved with the help of shaping regularization with a smoothness constraint:
\begin{align}
\label{eq:local3}
\mathbf{c}_1 &= [\lambda_1^2\mathbf{I} + \mathcal{T}(\mathbf{B}^T\mathbf{B}-\lambda_1^2\mathbf{I})]^{-1}\mathcal{T}\mathbf{B}^T\mathbf{a},\\
\label{eq:local4}
\mathbf{c}_2 &= [\lambda_2^2\mathbf{I} + \mathcal{T}(\mathbf{A}^T\mathbf{A}-\lambda_2^2\mathbf{I})]^{-1}\mathcal{T}\mathbf{A}^T\mathbf{b},
\end{align}
where $\mathbf{\mathcal{T}}$ is a smoothing operator, and $\lambda_1$ and $\lambda_2$ are two parameters controlling the physical dimensionality and enabling fast convergence when inversion is implemented iteratively. These two parameters can be chosen as the least-squares norms of $\mathbf{B}$ and $\mathbf{A}$, respectively.}


\inputdir{weight}
\multiplot{2}{syn,weights}{width=0.46\textwidth}{A demonstration of similarity-weighted semblance. (a) NMO corrected gather. (b) Weights applied to each trace for semblance calculation based on the local similarity between each trace and a reference trace.}

\section{Examples}
\subsection{Synthetic examples}
The first synthetic example is a simple four-layer hyperbolic CMP gather. By using the \old{conventional semblance}\new{traditional semblance}, AB semblance and the proposed similarity-weighted semblance, the semblance spectra are shown in Figure \ref{fig:hw-comp1}. From the comparison, we can see that the resolution is \old{obviously}\new{clearly} different for three approaches. The AB semblance suffers from \new{the} low\new{-}resolution \new{problem} as pointed out by \cite{fomel20091}\new{,} compared with the other two approaches. For both \old{conventional}\new{traditional} and AB semblances, there exist\new{s} a fake peak between the third and fourth layer. The similarity-weighted semblance obtain\new{s} a very high resolution both vertically and horizontally. \new{In this example, the reference trace is chosen as the traditionally stacked trace.} 

By adding some Gaussian white noise onto the first synthetic example, we get the second example (Figure \ref{fig:hw-comp2}). The performance of \new{the} three approaches are improved a lot due to the added random noise. The fake peaks appearing in the first example disappear because of the random property of Gaussian white noise, which indicate\new{s} the fact that a small level of random noise can aid \old{to some extent in} the semblance \old{measuring}\new{calculation} process \new{to some extent}. Compared with the other two approaches, the similarity-weighted semblance obtained a very high resolution. \new{In this example, the reference trace is chosen as the traditionally stacked trace.} 

The third synthetic example is a CMP gather with class II AVO anomalies (Figure \ref{fig:synth-comp2}).  \new{The class II AVO anomalies cause\old{s} seismic amplitudes to go through a polarity reversal \cite[]{rutherford1989,fomel20091}.} In this test, the weighting function is a combination between the similarity based weight and trend based weight. We implement this by first weighting each trace using the proposed approach and then apply the trend based approach \cite[]{fomel20091}. While the \old{conventional semblance}\new{traditional semblance} can not capture the information of the AVO anomalies, the AB semblance and the proposed similarity-weighted semblance can handle the AVO phenomena correctly. \new{Because of the combination between the AB semblance and similarity-weighted semblance, the resolution of the proposed approach will decrease a little bit. This phenomenon results from the fact that taking AVO effect into consideration will decrease the resolution, as analyzed in \cite{fomel20091}. However, the resolution using the proposed approach is still higher than that of the AB semblance and comparable to the traditional semblance.} \old{It's still obvious that the proposed approach can get a higher resolution.} \new{In this example, the reference trace is chosen as the traditionally stacked trace.} 

It's worth to be mentioned that the black strings appearing on the top of velocity maps denote the automatically picked optimum velocities using the algorithm introduced in \cite{fomel20091}. Thus, the trends of black strings between different semblance approach do not differ too much.  With manual picking, the results for both traditional semblance and AB semblance will result much larger velocity uncertainties, which will result in larger migration uncertainties \cite[]{fomel20141}. There are also some unreasonable changes in the black strings, which result\old{s} from the fact that the synthetic events are sparse and the automatically picking algorithms can not handle such problem. \new{Figure \ref{fig:vcomp} shows a comparison between the true velocity and the picked velocities from different approaches for the third synthetic example. The black solid line denotes the true velocity. The blue dash line denotes the picked velocity using the similarity-weighted semblance. The green long dash line denotes the picked velocity from the traditional semblance and the red dot dash line denotes the picked velocity from the AB semblance. It is clear that the AB semblance and similarity-weighted semblance are much similar and the similarity-weighted semblance can make the automatic picking more accurate. The velocity picked from the traditional semblance, however, deviates the true velocity a lot and is not acceptable.}

\inputdir{hw}
\plot{hw-comp1}{width=\textwidth}{Comparison between semblance spectra for a clean synthetic  data. Left: CMP gather. Middle left: using \old{conventional semblance}\new{traditional semblance}. Middle right: using AB semblance. Right: using similarity-weighted semblance.}

\plot{hw-comp2}{width=\textwidth}{Comparison between semblance spectra for a noisy synthetic  data. Left: CMP gather. Middle left: using \old{conventional semblance}\new{traditional semblance}. Middle right: using AB semblance. Right: using similarity-weighted semblance.}


\inputdir{synth}
\plot{synth-comp2}{width=\textwidth}{Comparison between semblance spectra for a synthetic data with class II AVO anomalies. Left: CMP gather. Middle left: using \old{conventional semblance}\new{traditional semblance}. Middle right: using AB semblance. Right: using similarity-weighted semblance.}

\plot{vcomp}{width=\textwidth}{Comparison between the true velocity and picked velocities for the third synthetic example with class II AVO anomalies. Black solid line denotes the true velocity. Green long dash line corresponds to the picked velocity by traditional semblance. Red dot dash line corresponds to the picked velocity by AB semblance. Blue dash line corresponds to the picked velocity by similarity-weighted semblance.}


\subsection{Field data examples}
The first field data is a marine CMP gather (Figure \ref{fig:comp1}). In this example, the proposed similarity-weighted semblance as shown in the right panel in Figure \ref{fig:comp1} obtains a very high resolution. The peaks on the semblance panel is easy to be picked out and thus reduce the velocity-peaking uncertainties \cite[]{fomel20141}. \new{In this example, the reference trace is chosen as the stacked trace using the SNR-weighting strategy.} 
\inputdir{bei}
\plot{comp1}{width=\textwidth}{Comparison between semblance spectra for field data. Left: CMP gather. Middle left: using \old{conventional semblance}\new{traditional semblance}. Middle right: using AB semblance. Right: using similarity-weighted semblance.}

The second field data example is a 3D prestack dataset from a 2D survey of the Gulf of Mexico, as shown in Figure \ref{fig:gulf}. We stacked the prestack data using automatically picked NMO velocities from different semblance map. Figure \ref{fig:stack} shows the stacked data using traditional semblance and Figure \ref{fig:simistack} shows the stacked data using the similarity-weighted semblance. The stacked data using the proposed approach can obtain better stacked result with stronger stacked energy, which indicates a more precise picked velocity. The comparison between Figures \ref{fig:stack} and \ref{fig:simistack} is not very obvious. However, when we zoomed two parts from the original stacked sections, as shown in Figures \ref{fig:stack-A} and \ref{fig:simistack-A}, we can find significant improvement. The two zoomed parts are highlighted by the two frame boxes in Figures \ref{fig:stack} and \ref{fig:simistack}. \new{In this example, the reference trace is chosen as the traditionally stacked trace.} 

\inputdir{stack}
\plot{gulf}{width=\textwidth}{A prestack dataset from the Gulf of Mexico.}

\multiplot{4}{stack,simistack,stack-A,simistack-A}{width=0.46\textwidth}{(a) Stacked data using automatically picked NMO velocity from the traditional semblance. (b) Stacked data using automatically picked NMO velocity from the similarity-weighted semblance. (c) Zoomed section from (a). (b) Zoomed section from (b). The two zoomed parts are highlighted by the frame boxes in (a) and (b).}

\section{Discussion}
Velocity analysis is of exceptional importance to the whole seismic data processing tasks and the NMO based velocity picking can provide the very initial velocity model for all the other velocity analysis approaches. Most of the current velocity-picking procedures are based on manual endeavors and the resolution of the velocity spectrum greatly affects the finally picked velocity. The above two reasons make the subject of this paper significant. Even with automatic picking algorithms, the proposed similarity-weighted semblance can still be superior than the traditional semblance, which has been demonstrated by the \new{third synthetic} and \new{the} second field data example\new{s}.

When implementing the proposed algorithm, we also need to calculate the traditional semblance in order to obtain a stacked reference trace. The inappropriate traditional stacking result will result in inappropriate calculation of the similarity-weighted semblance. In practice, we need to iterate several times to obtain an acceptable reference trace. Because of the iteration and the calculation of the local similarity, the proposed semblance will have some extra computational cost. However, the computational efficiency is still acceptable, and considering that the semblance calculation can be parallelized CMP by CMP, the computational cost is not a serious issue.

The proposed similarity-weighted semblance can be conveniently implemented based on the traditional semblance calculation framework. The only change is to weight each trace using the local similarity. The local similarity is a robust local attribute that can be easily calculated and has found many applications in exploration geophysics field \cite[]{guochang2009,yangkang2015ortho}. 
Thus, the proposed approach has the potential to be widely used in the industry.


\section{Conclusions}
We have proposed a general formulation for the weighted semblance. The \old{conventional semblance}\new{traditional semblance} is no more than a weighted semblance with constant weight. The AB semblance corresponds to a weighted semblance with a trend function derived from the data. With a weighting function using the local similarity between each trace and a reference trace, the acquired semblance spectra can get an obvious higher resolution, which is picking friendly. The proposed approach can also be combined with the AB semblance with an increased resolution and ability for handling AVO anomalies. Three kinds of synthetic example and two field data examples demonstrate the performance of our proposed approach.

\section{Acknowledgments}
We would like to thank Jingye Li, Jiang Yuan, Zhaoyu Jin, Keling Chen, \old{and} Pan Deng, \new{James Rickett, Jonathan Blair Ajo-Franklin} and two anonymous reviewers for helpful comments and suggestions. Yangkang Chen would like to thank Sergey Fomel for inspiring discussions on the local similarity algorithm and the Texas Consortium for Computational Seismology (TCCS) for financial support. This paper and all the figures are prepared with the Madagascar software package, which is an open-source research environment for reproducible computational experiments \cite[]{mada2013}. % Thus, we are grateful for its developers for providing such convenience. 

%\newpage



\appendix
\section{Appendix: Review of AB semblance}
Suppose that the weight $w(j,k)$ in equation \ref{eq:semb-weight} has a trend of trace amplitude $a(j,k)$, 
\begin{equation}
\label{eq:trend}
w(j,k)=A(j)+B(j)\phi(j,k),
\end{equation}
where $\phi(j,k)$ is a known function, and $A(j)$ and $B(j)$ are two coefficients. In the simplest form, $\phi(j,k)$ can be chosen as the offset at trace $k$. In order to estimate $A(j)$ and $B(j)$, we can turn to minimize the following objection function of misfit between the trend and trace amplitude:
\begin{equation}
\label{eq:obj}
F_j=\sum_{k=0}^{N-1}\left(a(j,k)-A(j)-B(j)\phi(j,k)\right)^2.
\end{equation}
Taking derivatives with respect to $A(j)$ and $B(j)$ in equation \ref{eq:obj}, setting them to zero, and solving the two linear equations, we can obtain the the following two least-squares fitting coefficients:
\begin{equation}
\label{eq:A}
A(j)=\frac{\displaystyle\sum_{k=0}^{N-1}\phi(j,k)\sum_{k=0}^{N-1}a(j,k)\phi(j,k) - \sum_{k=0}^{N-1}\phi^2(j,k)\sum_{k=0}^{N-1}a(j,k)}{\displaystyle\left(\sum_{k=0}^{N-1}\phi(j,k)\right)^2-N\sum_{k=0}^{N-1}\phi^2(j,k)},
\end{equation}
\begin{equation}
\label{eq:B}
B(j)=\frac{\displaystyle\sum_{k=0}^{N-1}\phi(j,k)\sum_{k=0}^{N-1}a(j,k) - N\sum_{k=0}^{N-1}a(j,k)\phi(j,k)}{\displaystyle\left(\sum_{k=0}^{N-1}\phi(j,k)\right)^2-N\sum_{k=0}^{N-1}\phi^2(j,k)}.
\end{equation}
Substituting $w(j,k)=A(j)+B(j)\phi(j,k)$ into equation \ref{eq:semb-weight} leads to the AB semblance.

\bibliographystyle{seg}
\bibliography{vscan}




