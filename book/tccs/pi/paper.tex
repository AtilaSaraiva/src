\published{Geophysics, 82, S51-S59, (2017)}

\title{Analytical path-summation imaging of seismic diffractions}
\author{Dmitrii Merzlikin and Sergey Fomel}

\lefthead{Merzlikin \& Fomel}
\righthead{Analytical path-summation imaging}

\address{
Bureau of Economic Geology \\
John A. and Katherine G. Jackson School of Geosciences \\
The University of Texas at Austin \\
University Station, Box X \\
Austin, TX 78713-8924 \\
dmitrii.merzlikin@utexas.edu
}
\footer{TCCS-12}

\maketitle

\begin{abstract}
Diffraction imaging aims to emphasize small subsurface objects, such as faults, fracture swarms, channels, etc.
Similarly to classical reflection imaging, 
velocity analysis is crucially important for \new{accurate} diffraction imaging. \old{Path-integral
calculation}\new{Path-summation migration} 
provides an imaging method, which produces a\new{n}\old{ velocity-independent}
image of the subsurface \new{without picking a velocity model}. 
Previous methods of path-\old{integral}\new{summation} \old{evaluation}\new{imaging} involve\old{d} a discrete summation of the
images corresponding to all possible migration velocity distributions within a predefined
integration range \new{and thus involve}\old{. Approximation of the integral in such a way has} a significant computational cost. 
We propose a direct analytical formula \new{for path-summation imaging based on the continuous
integration of the images along the velocity dimension}, which reduces the cost\old{ of path-integral evaluation} 
to that of \old{the}\new{only} two fast Fourier transforms.
%We discuss the problem of the artifacts in the path-integral images and propose two
%techniques compatible with direct evaluation formula, which address this issue.
The analytic approach also enables automatic
migration velocity extraction from diffractions using double path-\old{integral}\new{summation migration} framework.
Synthetic and field data examples confirm the efficiency 
of the proposed techniques.
% you can call it three problems: efficiency, tails and velocity extraction. 
\end{abstract}  

\section{Introduction}

Path-integral formulation provides an efficient imaging method, which does not require subjective and 
time-consuming velocity picking and produces a\new{n}\old{ velocity-independent} image of the 
subsurface \new{without specifying a velocity model} \cite[]{landa_etal06, Burnett11}.
The concept of path-integral imaging is based on the
fact that an image can be obtained from summing (stacking) of wavefields along wavefronts: the image accumulates 
contributions from only those events that are nearly in phase and 
which correspond to \new{the} true media features \cite[]{keydar04, landa04}.
In other words, path-integral method constructs the final image by
summing a set of images obtained from the range of different velocity distributions
without searching for the "optimal" one.
%, and velocity acts as a parameter defining different paths.

\old{\mbox{\cite{landabook}} proposes to use a focusing measure to weight the contributions of
images with different velocities.}
\new{\cite{landa_etal06} and \cite{landabook} propose to measure
the flatness of common image gathers and use it to weight images before stacking them to perform
path-integral imaging.}
These weights are used to enhance coherent summation of images migrated with
velocities in the vicinity of the true velocity and to cancel contributions from incorrect velocities.
This approach brings path-integral application for seismic imaging purposes in correspondence
with path-integral formulation in quantum mechanics \old{given by }\cite[]{feynman65}.
In this paper,
we follow \cite{Burnett11} and employ the inherent property of diffractions' behavior under time-migration
transformation - apex stationarity - for time path-summation imaging implementation \new{for diffraction imaging}.
\new{We use a term "path summation"
in order to highlight that no weighting is applied before stacking the images as opposed to path-integral
scheme, where weights represent assumed probability of each individual image.} 


%The wavefield, in this method, is constructed by superposition of 
%elementary signals propagated along different possible paths between source and receiver. 
%The contribution of each path is defined by the Lagrangian of the system: trajectories close to the 
%Fermat’s path sum constructively whereas others cancel each other. 

%In application to seismic migration,  \cite{landabook} proposes
%to use a focusing measure as a Lagrangian to weight the contributions of images with different velocities.
%Two choices of weights are available: Feynman’s and Einstein-Smoluchovsky's. Images, whose velocity
%is close to the true one, contribute to the final image more than other images. Depth migration
%path-integral formulation is provided by \cite{landa05}.

A path-\new{summation}\old{integral} image can be calculated through a summation of a set of constant velocity images, 
generated by velocity continuation (VC), which accounts for both lateral and vertical shifts of the event
under velocity perturbation. The \new{VC} transformation is continuous and analytical:
a velocity step between constant velocity images can be in theory 
infinitely small \cite[]{claerbout86,fomel94,Fomel_03_VC_,Fomel_03_VC,Burnett_aniso,decker}. Diffractions are sensitive to velocity
perturbation \cite[]{novais_etal1}: hyperbola flanks change their shape under VC transformation whereas their apexes remain stationary. 
Therefore,\old{ by} stacking\old{ of} constant velocity images\old{ one} can superimpose diffraction apexes constructively,
cancel out hyperbola flanks 
and generate a\new{n} \old{path-integral }image \cite[]{Burnett11}. If the constant velocity images are weighted before stacking
by their corresponding velocities, they produce
a double path-\new{summation}\old{integral} image, which can be used for velocity estimation \cite[]{schleicher_costa09,santos2016robust}.
%Division of double path integral by path integral allows us to build velocity distribution of a medium.      

The forementioned approach for path-\new{summation}\old{integral} \old{calculation} \new{migration} requires a sequence of VC steps to generate a set
of constant velocity images and therefore may consume significant computational resources. Moreover, produced images appear
to be contaminated by artifacts 
compared to the images produced by picking velocities \cite[]{Burnett11}. The artifacts come from incomplete 
cancellation of under- and overmigrated flanks of hyperbolas. We refer to these flanks as tails. Tails might 
overlap with useful signal and generate spurious features in the images. \new{Note that tails
are an intrinsic feature of unweighted path-summation
images, as opposed to path-integral images, where the weighting describing the probability of each image and
corresponding velocity is applied before summation.}  
 
In this paper, we propose a direct analytical \old{way of}\new{approach to} performing path\old{-integral}\new{-summation} and double
path\old{-integral}\new{-summation} \old{summation}\new{migration} as opposed to 
stacking of \new{multiple} constant\new{-}velocity images. Computational efficiency is improved significantly because
we do not require multiple velocity continuation steps for calculation of constant velocity images but instead compute
the path integral in one step at the cost of \new{only} two \new{Fast Fourier Transforms} \new{(}FFTs\new{)}.
To improve the image quality, we propose to apply\old{ adaptive 
filtering or, alternatively,} a Gaussian weighting scheme to eliminate tails \old{on}\new{in} path-summation migration images.
%Current path-integral weighting schemes \cite[]{landabook,schleicher_costa09} can 
%reduce the artifacts but at an additional expense of determining the weights, and do not
%allow for analytical integral evaluation in one step as opposed to adaptive filtering and
%Gaussian weighing schemes.
We test the effectiveness and the robustness of the proposed workflow 
on synthetic and field data examples.  

\section{Method}
We deduce analytical formulas for path-\new{summation}\old{integral evaluation} \new{migration} using the velocity continuation
concept for migration \cite[]{Fomel_03_VC_}.
Velocity continuation (VC) describes vertical and lateral shifts of time-migrated events under the change of migration velocity.
It is a continuous process, which, in the zero-offset isotropic case, can be described by the following partial differential equation 
\cite[]{claerbout86,fomel94,Fomel_03_VC_}: 
\begin{equation}
\label{eq:im_prop}
\frac{\partial^2 P}{\partial t\,\partial v} + \frac{v\,t}{2}\,\frac{\partial^2 P}{\partial x^2}=0\;.
\end{equation}
After a substitution $\sigma=t^2$ and a 2D Fourier transform, the analytical solution for equation~(\ref{eq:im_prop})
takes the following form \cite[]{Fomel_03_VC}:
\begin{equation}
\label{eq:vc}
\tilde{P}(\Omega,k,v) = \hat{P}_0 (\Omega,k)\,e^{-\frac{i k^2 v^2 }{16\Omega}}\;,
\end{equation}
where $v$ is VC velocity, $\Omega$ is the Fourier dual of $\sigma$, $k$ is the wavenumber, $\hat{P}_0 (\Omega,k)$ 
is input stacked and not migrated data, which corresponds to $v=0$, and
$\tilde{P}(\Omega,k,v)$ is a constant velocity image.  
Thus, an input image gets transformed to a specified velocity by a \new{simple} phase shift in the Fourier domain.
\cite{Burnett_aniso}
extend the formulation to the 3D azimuthally anisotropic case.
A sequence of phase shifts corresponding
to a predefined velocity range with a certain step can be calculated to generate a set of constant velocity images
\cite[]{larner, mikulich,yilmaz2001unified}. 
The final image is built by picking the parts of constant velocity images with the highest focusing measure and sewing them together
\cite[]{fomel07}.
In other words, for each $(t,x)$ location we browse through constant velocity images and pick the velocity corresponding 
to the highest focus of diffraction events. However, \new{generating} focusing \old{measure}\new{attributes}
can be computationally expensive, and picking is subjective.  
 
Path-integral formulation aims to provide velocity\new{-model-}independent images of the subsurface \cite[]{landa_etal06}. \old{Unweighted path integral
can be computed}\new{Path-summation migration can be performed} by simply stacking 
constant velocity images generated by velocity continuation \cite[]{Burnett11}.\old{ Unweighted path-integral evaluation corresponds to the formula} \new{Theoretically, the velocity step between images generated by velocity continuation
can be infinitely small, and path-summation migration in this case will correspond to
the following integral over the velocity dimension:}
\begin{equation}
\label{eq:pi}
I_{PI}(t,x,v_a,v_b) = \int_{v_a}^{v_b} P(t,x,v)\,dv\;,
\end{equation}
where $P(t,x,v)$ is a constant velocity image.
The \old{acquired}\new{resultant} image is \new{not associated with a particular velocity model}
\old{velocity independent} since it stacks all the constant velocity images \old{acquired}\new{generated} by VC.
%Equation~(\ref{eq:pi}) corresponds to unweighted path-integral formulation.

Note that stacking of images can be performed both in $(t,x)$ and $(\Omega,k)$ domains. For the latter case it takes the form:
\begin{equation}
\label{eq:pi_fft}
\hat{I_{PI}}(\Omega,k,v_a,v_b) = \int_{v_a}^{v_b} \hat{P}_0(\Omega,k)\ e^{-\frac{i k^2 v^2 }{16\Omega}}dv\;,
\end{equation}
where $\hat{P}_0(\Omega,k)$ corresponds to zero-offset or stacked data and does not depend
on migration velocity. Therefore, it can be taken outside of the integral:
\begin{equation}
\label{eq:pi_fft_filter}
\hat{I_{PI}}(\Omega,k,v_a,v_b) = \hat{P}_0(\Omega,k)\ \int_{v_a}^{v_b} e^{-\frac{ i k^2 v^2 }{16\Omega}}dv \;.
\end{equation}
We can treat the remaining integral as a filter $F_{PI}(\Omega,k,v_a,v_b) = \int_{v_a}^{v_b} e^{-\frac{ i k^2 v^2 }{16\Omega}}dv $,
which depends on frequency $\Omega$,
wavenumber $k$ and velocity integration range limits $v_a$ and $v_b$. Moreover, the filter has an analytical
expression: the integral of
the complex exponential function is proportional to the imaginary error function (erfi): 
\begin{equation}
\label{eq:erfi_pi_fft}
F_{PI}(\Omega,k,v_a,v_b) = e^{i\frac{5\pi}{4}}\ \frac{2\sqrt{\Omega \pi}}{k}\ \mbox{erfi}\big(e^{i\frac{3\pi}{4}}\ \frac{k\ v}{4\sqrt{\Omega}}\big) \bigg|_{v_a}^{v_b}.
\end{equation}
%To acquire unweighted path-integral image
%we simply need to convolve the input data with the filter expression.
%The mathematical form of the filter~(\ref{eq:pi_fft_filter}) allows us to analytically evaluate the path-integral image in the Fourier domain:
%where $\alpha(\Omega,k)$ and $\beta(v_b,\Omega,k)$ are rational functions of $\Omega$, $k$, $v_b$ or $v_a$:
%\begin{eqnarray}
%\label{eq:alpha}
%\alpha(\Omega,k) = e^{i\frac{5\pi}{4}}\ \frac{2\sqrt{\Omega \pi}}{k}\ ,\\
%\label{eq:beta}
%\beta(\Omega,k,v) = e^{i\frac{3\pi}{4}}\ \frac{k\ v}{4\sqrt{\Omega}} .
%\end{eqnarray}
Thus,\old{ path-integral imaging} \new{ path-summation migration} according to the equation~(\ref{eq:pi_fft_filter}) 
amounts to filtering the data with the filter given by \old{the formula}\new{equation}~(\ref{eq:erfi_pi_fft}).
%According to equation~(\ref{eq:pi_fft}),
%the analytical expression in formula~(\ref{eq:erfi_pi_fft}) is multiplied by the data values in $(\Omega,k)$ domain. 
Inverse Fourier transform and time-stretch removal produce the final \old{path-integral} image in the $(t,x)$ domain.
The cost of the computation corresponds to the cost of two fast Fourier transforms needed to transform the data to \new{the}
frequency-wavenumber domain and back \old{from it}\new{to time-space domain}.
The \new{filtered} image can be treated as if it
was calculated by stacking of constant velocity images with an infinitely small velocity discretization step.

%\inputdir{pi-pr-bei}
%\plot{a-bei-erfi-fft}{width=0.5\columnwidth}{Absolute values of equation~(\ref{eq:erfi_pi_fft}) on a predefined $(\Omega,k)$ grid.}

As an illustration, a path-\new{summation}\old{integral} image corresponding to a point scatterer with 1.5 km/s velocity
(Figure~\ref{fig:a-data}), which has been calculated
by application of equations~(\ref{eq:pi_fft}) and~(\ref{eq:erfi_pi_fft}), is shown in Figure~\ref{fig:a-path-integral}.
The path-\new{summation}\old{integral} migration image contains artifacts because of incomplete cancellation of two hyperbolas, which are still prominent \old{on}\new{in} the image: 
one corresponding to the lowest constant velocity ($v_a$) image (\old{we call it }an undermigrated tail), and the other corresponding
to the highest velocity ($v_b$) image (overmigrated tail).
Those two tails are not cancelled out during stacking of constant velocity images whereas hyperbolas' 
flanks within the VC velocity range sum destructively. \old{When path integral is calculated for}
\new{When path-summation migration is applied to}
real diffraction images the tails may 
interfere with useful signal and create artifacts.

\old{To extend path-integral analytical evaluation formulae}
\new{To extend analytical path-summation migration formulae} to 3D isotropic
post-stack case one needs to substitute \new{the absolute value $\left| \mathbf{k} \right|$ of the}\old{a} wavenumber vector $\mathbf{k} = (k_x,k_y)$\old{absolute value $\left| \mathbf{k} \right|$} instead of a 
one-dimensional wavenumber $k$ in the expressions given above. Pre-stack path-\new{summation}\old{integral evaluation}
\new{migration} is discussed in Appendix A. \new{Double path-summation migration \cite[]{schleicher_costa09,santos2016robust} allows for automatic migration velocity extraction. Corresponding analytical evaluation
formulae are provided in Appendix B.} 

\subsection{Attenuating tail artifacts}

Artifacts \old{on}\new{in} the \new{path-summation}\old{unweighted path-integral} images appear due to the absence of weights, which 
\new{describe probability of the image
migrated with a certain velocity and} taper out "unlikely" solutions \cite[]{landa_etal06,landabook,schleicher_costa09}.%On the other hand, estimation
%of weights appears to be an expensive procedure, which, besides that, hinders analytical
%evaluation of path-integral in one step. Due to those reasons
%path-integral evaluation becomes prohibitively expensive.
%To overcome this obstacle,
\old{ We propose two alternative techniques to attenuate tail artifacts. Both techniques
are compatible with direct evaluation formulas derived perviously.}
 \new{We propose to use
analytical Gaussian weighting of images being stacked in path-summation migration, in which
velocities in the vicinity of the most likely one are assigned a higher weight. An alternative
approach based on the adaptive subtraction method is described
in our previous work \cite[]{merzlikin15}.}  

%We employ the intrinsic property of diffractions' - apex stationarity - to
%formulate path-integral imaging.
  
\old{Our first approach is an adaptive subtraction technique. 
We follow the approach described by \mbox{\cite{Fomel_09}} for adaptive subtraction
with a smoothly-variable filter constructed by regularized non-stationary regression.
For undermigrated tail elimination we use the lowest constant velocity image in the VC set
(corresponding to $v_a$ in our notation) as a starting model 
for adaptive filtering, whereas for overmigrated tail we use the highest constant velocity ($v_b$) image.
Over- and undermigrated
tail models are updated independently and, when a reasonable model has been acquired, are
combined together.  
Then the model of the tails (Figure~\ref{fig:a-tails}) is subtracted from the path-integral image.
As a result, we get the apexes of the stacked hyperbolas without the tails (Figure~\ref{fig:a-path-int-te}).
Tail elimination via adaptive subtraction might not be accurate for a dense distribution of diffractions, when
tails overlap and form a complex pattern, which is hard to estimate and subtract.}

%For adaptive filtering tails' elimination approach after analytical evaluation of
%unweighted path-integral image (equation~\ref{eq:erfi_pi_fft},\ref{eq:pathint3d})
%we first need to compute tails' initial models through velocity
%continuation of diffraction image to $v_a$ and $v_b$. Then we update an overmigrated tail model
%by adaptive filtering and subtract it from the path-integral image. Same procedure should be applied
%for an undermigrated tail

\old{As an alternative, we propose the following Gaussian weighting scheme for tails' elimination.} 
The following \new{analytical} weighting term can be added to the velocity continuation phase-shift:
\begin{equation}
\label{eq:weighted_vc}
\tilde{P}(\Omega,k,v,\beta,v_{bias}) = \hat{P}_0 (\Omega,k)\,e^{-\frac{i k^2 v^2 }{16\Omega} - \beta(v_{bias} - v)^2}\;,
\end{equation}
and corresponding \old{path-}integral expression \new{describing path-summation migration} takes the following form:
\begin{align}
\label{eq:gpi_int}
\hat{I_{GPI}}(\Omega,k,v_a,v_b,\beta,v_{bias}) & = \int^{v_b}_{v_a} \hat{P}_0(k,\Omega)\ e^{-\frac{i k^2 v^2}{16\Omega} - \beta(v_{bias} - v)^2} dv\ \nonumber \\
& = \hat{P}_0(\Omega,k)\ \int^{v_b}_{v_a} e^{-\frac{i k^2 v^2}{16\Omega} - \beta(v_{bias} - v)^2} dv  \nonumber \\
& =  \hat{P}_0(\Omega,k) \cdot F_{GPI}(\Omega,k,v_a,v_b,\beta,v_{bias}),
\end{align}
%where $\beta$ controls the decay of the amplitude of the image when migration velocity is
%not equal to the bias velocity $v_{bias}$. Therefore, for the following path-int
where filter $F_{GPI}(\Omega,k,v_a,v_b,\beta,v_{bias})$ is evaluated analytically as follows:

%\begin{equation}
%\label{eq:gaussian}
%\begin{split}
\begin{multline}
\label{eq:gaussian}
F_{GPI}(\Omega,k,v_a,v_b,\beta,v_{bias}) = \\
%=\frac{e^{-(\beta v_0^2 + \frac{\beta^2 v_0}{i\alpha(k,\Omega) - \beta}) }}{\sqrt{i\alpha(k,\Omega) - \beta}}\ erfi\big( \sqrt{i\alpha(k,\Omega)-\beta}\ v + \frac{\beta v_0}{ \sqrt{i \alpha(k,\Omega)-\beta}} \big) \bigg|_{v_a}^{v_b}
\frac{\sqrt{\pi}\ e^{-(\beta v_{bias}^2 + \frac{\beta^2 v_{bias}^2}{\gamma(\Omega,k,\beta)^2})}}{ 2\gamma(\Omega,k,\beta) } \mbox{erfi}\big( \gamma(\Omega,k,\beta) v + \frac{\beta v_{bias}}{\gamma(\Omega,k,\beta)} \big) \bigg|_{v_a}^{v_b}.
\end{multline}
%\end{split}
%\end{equation} 
Here, $\gamma(\Omega,k,\beta)=i\sqrt{ik^2/16\Omega + \beta}$, $\beta$ is a parameter controlling contribution of each velocity to the
final\old{ path-integral} image and $v_{bias}$ is a velocity bias. 
By choosing a constant value of $v_{bias}$ for the whole section between $v_a$ and $v_b$ we can enhance the apexes of summed 
hyperbolas and decrease tails’ contribution to the final image. 
Since in a section, velocity varies and $v_{bias}$ is constant, over- or undermigrated diffraction curves may get emphasized at 
locations where the true velocity is higher or lower than $v_{bias}$ respectively. This is a possible drawback of the Gaussian weighting
approach. 

%\new{Weights used in Gaussian path-summation migration are not associated with the probability
%of each image being stacked - weights are associated with the choice of the most likely velocity and
%the tapering rate at the integration limits. All these parameters are defined by the user.}

\new{We analyze a response of a single diffraction hyperbola modelled with 1.5 [km/s] velocity (Figure~\ref{fig:a-data}) to
both path-summation and Gaussian weighted path-summation migrations. Synthetic model spectrum in $\Omega-k$ domain
is shown in Figure~\ref{fig:a-fft-mag}. Path-summation migration filter magnitude as well as its phase spectrum are shown
in Figures~\ref{fig:a-erfi-fft-mag} and~\ref{fig:a-erfi-fft-phase}. It can be interpreted as
a particular form of dip filtering.
Wave-numbers in the vicinity of zero appear to be preserved and either remain stationary or exhibit small phaze-shift values. This
area corresponds to the diffraction hyperbola apex remaining stationary under time-migration velocity perturbation. Non-zero
wave-number events have lower corresponding filter coefficients, which taper out to the spectrum boundaries.
Figure~\ref{fig:a-path-integral} shows the result of path-summation migration applied to diffraction hyperbola. Several lobes with
non-zero wavenumber values appear to be noticeable. Corresponding phase-shift values (Figure~\ref{fig:a-erfi-fft-phase})
are not zero and lead to displacement of the corresponding events on the path-summation migration image (Figure~\ref{fig:a-path-integral}).
These events are associated with tail artifacts.}

\new{Gaussian weighting scheme spectrum magnitude and phase are shown in Figures~\ref{fig:a-gaussian-erfi-mag} and~\ref{fig:a-gaussian-erfi-phase}.
The result of corresponding path-summation migration in $\Omega-k$ domain is shown in Figure~\ref{fig:a-gpi-fft-mag}. Gaussian weighting
scheme tapers non-stationary lobes and, therefore, suppresses the tails.}
Figure~\ref{fig:a-pi-gaussian} corresponds to the Gaussian weighting scheme for path\old{-integral}\new{-summation migration}\old{ evaluation}.
The tail artifacts are successfully eliminated. 

Proposed expressions for direct and analytical path-integral evaluation in one step allow us to
drastically reduce computational costs.\old{ Tail elimination techniques compatible with
direct evaluation formulae are available as well.} Following sections illustrate the \old{viability}\new{applicability}
and the effectiveness of the \new{proposed} techniques\old{described}. 

\inputdir{simple-model}

%\multiplot{4}{a-path-integral,a-tails,a-path-int-te,a-pi-gaussian}{width=0.45\textwidth}{(a) Unweighted path-integral image of a diffraction hyperbola (v = 1.5 km/s) acquired by formulae~(\ref{eq:pi_fft}) and~(\ref{eq:erfi_pi_fft}); (b) predicted tail's model;
%(c) path-integral image after adaptive tails' subtraction; (d) path-integral image with Gaussian weighting acquired by formulae~\ref{eq:gpi_int} and~\ref{eq:gaussian}.}

\multiplot{2}{a-data,a-fft-mag}{width=0.45\textwidth}{(a) Diffraction hyperbola modelled with 1.5 km/s velocity; (b) amplitude spectrum of (a)
in $\Omega-k$ domain.}

\multiplot{4}{a-erfi-fft-mag,a-gaussian-erfi-mag,a-erfi-fft-phase,a-gaussian-erfi-phase}{width=0.45\textwidth}{
Path-summation migration filter (formula~\ref{eq:erfi_pi_fft}): (a) amplitude spectrum; (c) phase-spectrum; Gaussian weighted path-summation migration filter (formula~\ref{eq:gpi_int}): (b) amplitude spectrum; (d) phase spectrum.}

\multiplot{4}{a-pi-fft-mag,a-gpi-fft-mag,a-path-integral,a-pi-gaussian}{width=0.45\textwidth}{Unweighted path\old{-integral}\new{-summation} \old{image}\new{migration} of a diffraction hyperbola (v = 1.5 km/s) (formulae~(\ref{eq:pi_fft}) and~(\ref{eq:erfi_pi_fft})):
(a) $\Omega-k$ amplitude spectrum; (c) image;
path\old{-integral}\new{-summation} \old{image}\new{migration} with Gaussian weighting (formulae~\ref{eq:gpi_int} and~\ref{eq:gaussian}):
(b) $\Omega-k$ amplitude spectrum; (d) image.}   

%For double path-integral framework both unweighted path-integral (equations~\ref{eq:erfi_pi_fft},\ref{eq:pathint3d})
%and double path-integral (equations~\ref{eq:dpi_pr2},\ref{eq:dpi_pr3d}) filters should be evaluated at the step number 4.
%Then we smoothly divide the double path integral by the path integral

%\begin{equation}
%\label{eq:dpi_vel_extr}
%\hat{P}_0(\Omega,\mathbf{k}) \cdot F_{DPI}(\Omega,\mathbf{k},v_a,v_b)/F_{PI}(\Omega,\mathbf{k},v_a,v_b)
%\end{equation}
%In the step 4, instead of formula~(\ref{eq:erfi_pi_fft}), we can apply the Gaussian weighting equation~(\ref{eq:gaussian})
%to suppress artifacts or equation~(\ref{eq:dpi_pr2})
%to compute a double path integral.
%In the former case, steps 7 and 8 are discarded. In the double path-integral approach, we need to analytically
%evaluate double  we smoothly divide the double path integral by the
%path integral to acquire a velocity field, which
%is then used to build an image of the subsurface. Picking is not needed for all above-mentioned variants of the workflow.  

\section{\new{Field data examples}}

\subsection{\new{Reflection/diffraction separation}}

%%%
\new{
Diffraction amplitudes can be several orders of magnitude lower than 
those of reflections \cite[]{klem-musatov94}. In order to highlight
subtle diffractivity features a reflection/diffraction separation procedure has to be applied to the data.  
There is a variety of methods based on different separation techniques. \cite{kanasewich88,landa98,berkovitch09,dell11,tsingas11,rad14,bauer15} employed optimal 
stacking of diffracted energy along diffraction travel-time curves. \cite{harlan84,papziner98,taner06,fomel07,reshef09,klokov12}
decompose full-waveform seismic records into diffracted and reflected components, \cite{kozlov04,moser08,koren2011full,sturzu13,klokov2013selecting,mihai15} modify
a Kirchhoff migration
kernel to eliminate specular energy coming from the first Fresnel zone and image diffractions only.}

%%%
\new{To perform reflection/diffraction separation 
we used plane-wave destruction filter
\cite[]{Fomel_02,fomel07} for both field data examples. For both examples
reflections appear to be continuous and laterally consistent. This behavior aids
PWD reflection's prediction along estimated local slopes followed by their subtraction from the stack.}

%\hl{It should be noticed that all the separation techniques
%mentioned above do not allow for exact reflection and diffraction
%separation. Therefore, there exists a reflection energy
%leakage to diffraction image domain. However, diffraction energy associated
%with small scale subsurface heterogeneities is significantly emphasized in comparison with the initial data. }

\subsection{2D field data example}

We tested the proposed workflow on a vintage Gulf of Mexico dataset \cite[]{claerbout05,fomel07}. 
%Figure~\ref{fig:a-pi-nw-nwi,a_pi_nw,a-pi-te-nwi,a_pi_te,a-pi-gaussian-nwi,a-pi-gaussian}.
A stacked section of diffractions separated using plane-wave destruction filter is shown in Figure~\ref{fig:a-bei-pwd}.  
The path-\old{integral}\new{summation migration} image, which was calculated directly according to equation~(\ref{eq:erfi_pi_fft}),
is shown in Figure~\ref{fig:a-pi-nw-nwi}. The integration limits corresponded to $1.4\ [km/s]$-$2.6\ [km/s]$.
Diffractions collapse to form an image of faults.

\old{The result of adaptive subtraction of tails is shown in Figure~\ref{fig:a-pi-te-nwi}. The model of predicted tails
is shown in 
Figure~\ref{fig:a-tails-nwi}. After elimination of 
tails, path-integral image becomes cleaner: in the lower part of the image events become more prominent since they are not
affected by the tails' superposition. Additionally, it becomes easier to distinguish fault planes in the upper part of the image.
On the other hand, we notice that some of the diffraction apexes in the upper part of the section
and between the fault planes, got suppressed
during adaptive subtraction of the tails.}

Path-\new{summation}\old{integral} \new{migration}
image with Gaussian weighting evaluated with $1.4\ [km/s]$-$2.6\ [km/s]$ integration limits,
$v_{bias}=2.0\ [km/s]$ and $\beta=10$
is shown in Figure~\ref{fig:a-pi-gaussian-nwi}. The
image in comparison with\old{ the result of adaptive subtraction approach}\new{ the regular
path-summation migration image} appears to have higher resolution\new{. }\old{ and to preserve
smaller features}\new{Tail artifacts
appear to be suppressed as well}. However, \new{some} velocity bias is noticeable in the upper part of the section.

The migration velocity field 
acquired by division of double path\new{-summation} integral by path\new{-summation} integral and the corresponding image of diffractions are shown in Figures~\ref{fig:a-dpi-slice} and~\ref{fig:a-dpi-vel}.
\new{Velocity values in the regions without diffraction energy are associated with noise.
Strong velocity anomalies at the boundaries of the section are associated with edge effects. A possible extension of 
the workflow would include velocity model weighting by diffraction probability at selected locations.}

The image derived from double-path-\new{summation}\old{integral} velocity has the highest resolution. 
%Full stacked image, containing both reflections and diffractions, migrated with 
A full image with both reflections and diffractions imaged using
the same velocity field is shown in Figure~\ref{fig:a-refl-image}.
Discontinuities in seismic reflections align with faults imaged
by path-\old{integral}\new{summation} diffraction imaging.

%In application to Gulf of Mexico dataset, the latter approach appears to be the optimal one
%in terms of resolution and the absence of tail artifacts.   
%contaminated by high frequency artifacts

%\clearpage\pagestyle{empty}
%\addtolength{\topmargin}{-.7in}

\inputdir{bei}

%\multiplot{4}{a-bei-pwd,a-dpi-vel,a-dpi-slice,a-refl-image}{width=0.47\columnwidth}
%{Gulf of Mexico dataset: (a) Separated diffractions; (b) velocity model from division of double path integral by path integral;
%(c) diffraction's image generated with the velocity field (b); (d) reflections and diffractions migrated with the same 
%velocity field (b).}

%\multiplot{4}{a-pi-nw-nwi,a-pi-te-nwi,a-pi-gaussian-nwi,a-tails-nwi}{width=0.47\columnwidth}
%{Gulf of Mexico dataset: (a) path-integral image; (b) path-integral image with eliminated tails;
%(c) path-integral image with Gaussian weighting; (d) predicted tail's model.}

\multiplot{6}{a-bei-pwd,a-dpi-vel,a-dpi-slice,a-refl-image,a-pi-nw-nwi,a-pi-gaussian-nwi}{width=0.47\columnwidth}
{Gulf of Mexico dataset: (a) separated diffractions; (b) velocity model from division of double path\new{-summation}
integral by path\new{-summation} integral;
(c) diffraction's image generated with the velocity field (b); (d) reflections and diffractions migrated with the same 
velocity field (b); (e) path\new{-summation migration}\old{integral} image; (f) path\new{-summation}\old{-integral}
\new{migration} image with Gaussian weighting.}

\section{3D field data example}

We applied the proposed path\new{-summation}\old{-integral} diffraction imaging framework to a 3D land seismic dataset
acquired to characterize a tight-gas sand reservoir in the Cooper Basin in Western Australia.
The approach has been applied to a 3D stacked volume. Detailed geological interpretation of \old{the acquired}\new{diffraction}
images as well as their comparison to discontinuity-type attributes are provided in \cite{tyiasning2016comparison}.

\old{For reflection/diffraction separation 
we applied plane-wave destruction filter \mbox{\cite[]{Fomel_02,fomel07}}: 
corresponding full-waveform and diffraction-and-noise 
volumes are shown in Figure~\ref{fig:stack-cube,dif-cube}.
As it is easy to notice from}
\old{Figure~\ref{fig:dif-cube}, there is some reflection energy remainder in the diffraction-and-noise volume. Reflection energy
leakage to diffraction image happens because separation of reflected and diffracted energy
by PWD filter is not exact \mbox{\cite[]{fomel07}}. However, diffraction energy associated
with small scale subsurface heterogeneities is significantly emphasized in comparison with the initial data.} 

\new{The input stack is shown in Figure~\ref{fig:stack-cube}. Diffraction-and-noise volume, which corresponds
to the result of PWD applied to the stacked volume, is shown in Figure~\ref{fig:dif-cube}.
Diffraction energy associated
with small scale subsurface heterogeneities is significantly emphasized in comparison with the initial
stack (Figure~\ref{fig:stack-cube}).}
   
We have applied automatic migration velocity extraction by double path\old{-integral} \new{summation} approach to the dataset.
Extracted velocity
distribution is shown in Figure~\ref{fig:vel-hts-sing}. The velocity limits used for path-\new{summation}\old{-integral}
\old{evaluation}\new{integration} correspond to $2.9\ [km/s]$ and $3.1\ [km/s]$. They were chosen based on an
a priori migration velocity distribution.
% - pre-stack time migration velocity distribution
%provided by Santos Inc. 

We used automatically extracted velocity to migrate initial (Figure~\ref{fig:conv-image-ts})
and diffraction data (Figure~\ref{fig:dpi-ts}). Images shown in the
Figures~\ref{fig:conv-image-ts}\old{,} \new{and} ~\ref{fig:dpi-ts}\old{,~\ref{fig:refl-image-ts}} are extracted from the whole migrated volume
along the target horizon - interface between coal layer and tight gas sand.
Diffraction image (Figure~\ref{fig:dpi-ts})
appears to have \new{a} high spatial resolution. It should be noticed that some of the \new{most visible} features 
\old{on}\new{in} the diffraction image can be identified in the conventional image as well.
For instance, the elliptical structure limited by inlines $6-8.5\ [km]$ and crosslines $0.5-2.5\ [km]$
is prominent in both images, but is more emphasized in the diffraction image. This observation
correlates with the fact that diffractions and reflections coexist on conventional images, but reflections
exhibit higher energy and may mask smaller-scale scatterers.\old{Reflection-only data corresponding
to the difference between diffraction-and-noise and full-waveform stacks migrated with 
the same velocity model is shown on the Figure~\ref{fig:refl-image-ts}. Discontinuities are significantly
less prominent on reflection-only image and can be identified only in the case of noticeable
reflection misalignment.}  

We also \old{computed an unweighted path-integral image}\new{performed unweighted
path-summation migration}.
To illustrate the differences between the images
we focus on the area of the image corresponding to $6-12\ [km]$ inlines and $0.5-4.5\ [km]$ crosslines.
Unweighted path-\old{integral}\new{summation} image (Figure~\ref{fig:w-pathint-ts}) has lower spatial resolution in comparison 
to \old{DPI}\new{double path-summation} velocity image (Figure~\ref{fig:w-dpi-ts}). However, the latter one appears to be noisier. 

%To validate
%the DPI image (Figure~\ref{fig:w-dpi-ts}) we have migrated diffraction-and-noise volume using
%above-mentioned a priori PSTM velocity and acquired the image shown in (Figure~\ref{fig:w-snt-dpi-ts}).
%These images have subtle differences illustrating the validity of the DPI image.

A number of features such
as small faults corresponding to $9.5-10.5\ [km]$ inlines and $1-3\ [km]$ crosslines and the elliptical structure
in the lower left corner of the figures look more focused on the \old{DPI}\new{double-path-summation} image (Figure~\ref{fig:w-dpi-ts}) than
on the Figure~\ref{fig:w-snt-dpi-ts}. This observation can be attributed 
to \old{DPI}\new{double-path-summation} migration velocity extraction directly from diffractions as opposed to PSTM velocity, which 
has been estimated on
conventional full-waveform data containing significantly stronger reflections masking diffractions.
%to the differences
%between the ways DPI and a priori velocities have been evaluated. 
%PSTM velocity has been estimated on
%conventional full-waveform data containing both reflections and diffractions. Since reflections
%are much stronger than diffractions PSTM migration velocity is primarily controlled by continuous
%interfaces whereas path-integral velocity estimation is based on direct small-scale discontinuities'
%seismic responses - diffractions. Differences between the images can be also related to pre-stack and
%post-stack migration velocity estimation peculiarities. 

Figure~\ref{fig:w-dpi-wr-ts} shows an image acquired with \old{DPI}\new{double-path-summation} migration velocity and a wider integration
range - $1.5 - 4.5\ [km/s]$. \new{The} \old{I}\new{i}mage appears to be much less focused, however the pattern of the elliptical
structure can still be identified. \old{Unweighted path-integral}\new{Unweighted
path-summation migration} image (Figure~\ref{fig:w-pi-wr-ts})
appears to be significantly blurred due to overlap between tails and useful signal.
%and no insights about small-scale subsurface heterogeneities can be acquired.
%Subsurface
%structure loss is believed to be connected with tail artifacts, which overlap with useful signal. Besides overlap,
%tails might stack coherently and give rise to spurious features on the images.
Gaussian weighting path-\old{integral}\new{summation}
scheme allows to significantly suppress tails and highlight actual apexes of hyperbolas (Figure~\ref{fig:w-gpi-wr-ts}).
We used $3.1\ [km/s]$ velocity bias with $\beta=10$. In comparison to Figure~\ref{fig:w-pathint-ts} \old{GPI}
\new{Gaussian weighting path-summation} image
has lower spatial resolution. This observation is connected with wider integration range used for \old{GPI}\new{Gaussian
weighting path-summation migration}\old{evaluation}
in comparison to the image~\ref{fig:w-pathint-ts} acquired with $2.9\ - 3.1\ [km/s]$ velocity limits. When a
wider integration range is used, area around the apex, within which diffraction hyperbolas are stacked coherently,
becomes larger leading to the resolution loss.

%When we use wider integration range, or, in other words, higher number of diffraction hyperbolae being stacked,
%stationary part, which is coherently summed between constant migration velocity images, becomes broader: diffraction
%hyperbola tends to a horizontal line shape when over- or under-migration velocity is being increased or decreased
%respectively. Therefore, apex area becomes flatter what leads to a larger zone where scattered energy is stacked coherently.
%This behavior is an intrinsic property of the path-integral imaging approach.
%can be treated as an uncertainty principle: . 

%w-dpi-ts,w-dpi-wr-ts,w-pathint-ts,w-pi-wr-ts,w-snt-dpi-ts,w-gpi-wr-ts           

%This velocity range can be treated as an uncertainty in velocity estimation on the depth
%level corresponding to the target horizon.

%may be you want to include wrong velocity estimates... 

%Figure~\ref{fig:w-pathint-ts} illustrates post-stack 3D unweighted path-integral filter given
%by the formula~(\ref{eq:pathint3d}) and convolved with diffraction-and-noise wavefield.
%The result is shown in figure.

%%%
\inputdir{barrolka}

\multiplot{2}{stack-cube,dif-cube}{width=0.7\textwidth}{(a) Stacked volume; (b) volume after plane-wave destruction: diffractions are emphasized.}

\plot{conv-image-ts}{width=\textwidth}{Conventional reflection image acquired \old{with DPI}\new{using double path-summation} velocity shown in the Figure~\ref{fig:vel-hts-sing}.}

\plot{dpi-ts}{width=\textwidth}{Diffraction image acquired \old{with DPI}\new{using double path-summation} velocity shown in the Figure~\ref{fig:vel-hts-sing}.}

\old{\plot{refl-image-ts}{width=\textwidth}{Reflection-only image acquired \old{with DPI}\new{using double path-summation} velocity shown in the Figure~\ref{fig:vel-hts-sing}.}}

\plot{vel-hts-sing}{width=\textwidth}{Migration velocity distribution acquired by smooth division of double path\new{-summation} integral by path\new{-summation} integral \new{according to} \old{(}equation~\ref{eq:dpi_vel}\old{)}.}

%wide velocity range
\multiplot{6}{w-dpi-ts,w-dpi-wr-ts,w-pathint-ts,w-pi-wr-ts,w-snt-dpi-ts,w-gpi-wr-ts}{width=0.47\textwidth}
{ Integration range $2.9-3.3$ $km/s$: (a) diffraction image with \old{DPI}\new{double path-summation} velocity, (c) \old{PI}\new{path-summation migration} image; integration range $1.5-4.5$ $km/s$: (b) diffraction image with \old{DPI}\new{double path-summation} velocity, (d) \old{PI}\new{path-summation migration} image, (f) \old{GPI}\new{path-summation migration} image \new{with Gaussian weighting} ($\beta=10$, $v_{bias}=3.1$ $km/s$); (e) image with velocity from prestack reflection imaging.}    

\section{Discussion \old{and conclusions}}
Path-\old{integral}\new{summation} images can be contaminated by tails, which come
from an incomplete cancellation of hyperbolas'
flanks corresponding to the 
lowest and highest velocities. \old{We propose to 
remove artifacts using an adaptive subtraction technique.
In some cases, tail patterns become more complex and hard to eliminate with adaptive subtraction.
Alternatively,}\new{We propose to apply}
Gaussian weighting scheme\old{can be applied} for tails' elimination. In that case,
no additional computational cost is required besides the direct integral
evaluation itself. However, the weighting approach may introduce some velocity bias into path\old{-integral}\new{-summation} images.

Resolution of path\old{-integral}\new{-summation} images might \old{be}\new{appear} inferior to that of conventional workflows
based on \old{finding}\new{picking} "one optimal" \old{solution}\new{velocity}. The loss of resolution comes from the summation over a set
of diffraction images: coherent summation is not limited to the very apex of diffraction hyperbola but rather
to the flat area around it.
"Blurred" path\old{-integral}\new{-summation} images naturally incorporate the uncertainty of velocity estimation. Therefore,
they can be treated as diffraction probability volumes. Analytical evaluation allows path\old{-integral}\new{-summation}
\new{imaging} framework
to be used to constrain the model space
for diffraction imaging and inversion problems.

%, which might be an extremely useful
%constrain to validate the images.

On the other hand, images built with velocity models from division of double path\new{-summation} integral
by \new{not weighted} path\new{-summation} integral have \new{a relatively}
high resolution and
are not blurred. In the examples provided in this paper those images appear to be optimally focused.
In the analytical double-path\old{-integral}\new{-summation} technique, no picking is required. 

%Analytical evaluation makes path-integral diffraction imaging an attractive tool to validate the images or extract
%migration velocity distribution. The workflow is easily extended to 3{D} elliptically-anisotropic case where it should
%bring significant computational advantages.   

%When the integration limits tend
%to the true velocity distribution resolution of path-integral images improves.      

%To conclude, it should be mentioned that there is no "perfect" general solution in terms of the best accuracy and the lowest computational cost for
%path integral images calculation. The "best" variant of the workflow should be chosen for the particular task one tries to solve.          

\section{\new{Conclusions}}
We have developed an approach for direct and analytical path-\old{integral evaluation}\new{summation 
imaging}, which improves \new{the}
accuracy and significantly reduces \new{the} computation cost \new{of numerical summation}\old{of path-integral imaging
as opposed to discrete summation schemes approximating the integral}.
The proposed approach is based on velocity continuation, which
describes continuous image transformation with perturbation of migration velocity, and in theory
can have an infinitely small velocity discretization step.\old{Integral evaluation} \new{ As a result,
path-summation migration}
amounts to
filtering in the frequency-wavenumber domain and has a total cost equivalent to the cost of \new{only} two FFTs.

\new{Tail artifacts inherent to path-summation migration images can be efficiently removed
by a Gaussian weighting scheme. In this case, path-summation migration can still be performed analytically.
Analytical evaluation formulas for double-path summation allow for efficient migration
velocity extraction based on diffractions.}

%\new{Corresponding images are comparable to the conventional ones. In some cases, better focusing of discontinuities has
%been achieved. A priori velocity information is necessary for path summation velocity range determination.}    

\section{Acknowledgements}
We thank TCCS sponsors and Geofrac consortium for financial support and William Burnett,
Dennis Cooke and Evgeny Landa for inspiring discussions. \new{We thank anonymous reviewers for constructive suggestions}. 

%\appendix
%\section{Appendix A: 3{D} path-integral evaluation}

%\cite{Burnett_aniso} describe a velocity continuation workflow for imaging of elliptically-anisotropic media using
%3{D} post-stack data. Here, we derive formulas for direct path-integral evaluation in the 3{D}
%post-stack isotropic case, for which image propagation in velocity equation takes the following form
%\cite[]{claerbout86,fomel94,Fomel_03_VC_}: 
%\begin{equation}
%\label{eq:im_prop_3d}
%\frac{\partial^2 P}{\partial t\,\partial v} + v\,t\,\big( \frac{\partial^2 P}{\partial x^2} + \frac{\partial^2 P}{\partial y^2} \big)=0\;.
%\end{equation}
%Application of $\sigma=t^2$ stretch and 3{D} Fourier transform yields an ordinary differential
%equation, the solution of which corresponds to the following phase-shift: 
%\begin{equation}
%\label{eq:vc3d}
%\tilde{P}(\Omega,\mathbf{k},v) = \hat{P}_0 (\Omega,\mathbf{k})\,e^{\frac{-i (k_x^2 + k_y^2) v^2 }{16\Omega}}\;,
%\end{equation}
%where $k_x$ and $k_y$ are the wavenumbers along in-line and
%cross-line directions respectively and $\Omega$ is the Fourier dual of $\sigma$. Equation~\ref{eq:vc3d}
%corresponds to equation~\ref{eq:vc} extended to the 3{D} case. 

%Thus, extension of the post-stack path-integral analytical evaluation formulae to 3{D} case
%is acquired by substitution
%of a vector $\mathbf{k} = (k_x,k_y)$ into corresponding filter expressions
%(formulas~\ref{eq:erfi_pi_fft}(check if it is correct),~\ref{eq:dpi_pr2} and~\ref{eq:gaussian}). Following equations

%\begin{gather}
%\label{eq:pathint3d}
%F_{PI}(\Omega,\mathbf{k},v_a,v_b) = e^{i\frac{5\pi}{4}}\ \frac{2\sqrt{\Omega \pi}}{\sqrt{k_x^2 + k_y^2}}\ erfi\big(e^{i\frac{3\pi}{4}}\ \frac{\sqrt{k_x^2 + k_y^2}\ v}{4\sqrt{\Omega}}\big) \bigg|_{v_a}^{v_b},
%\\
%\label{eq:dpi_pr3d}
%F_{DPI}(\Omega,\mathbf{k},v_a,v_b) = \frac {8 i \Omega}{(k_x^2 + k_y^2)} e^{\frac{-i(k_x^2 + k_y^2)v^2}{16\Omega}}\ \bigg|_{v_a}^{v_b},
%\\
%\label{eq:gaussian3d}
%F_{GPI}(\Omega,\mathbf{k},v_a,v_b,v_{bias},\beta) =\frac{\sqrt{\pi}e^{-(\beta v_{bias}^2 + \frac{\beta^2 v_{bias}^2}{\gamma(\Omega,\mathbf{k},\beta)^2})}}{ 2 \gamma(\Omega,\mathbf{k},\beta) } erfi\big( \gamma(\Omega,\mathbf{k},\beta) v + \frac{\beta v_{bias}}{\gamma(\Omega,\mathbf{k},\beta)} \big) \bigg|_{v_a}^{v_b}
%\end{gather}
%are 3{D} versions of path-integral, double path-integral and Gaussian path-integral filters correspondingly. 
%In formula~(\ref{eq:gaussian3d}) $\gamma(\Omega,\mathbf{k},\beta) = i \sqrt{i(k_x^2+k_y^2)/16\Omega + \beta}$.

%Velocity extraction for 3D isotropic case will take the following form:
%\begin{equation}
%\label{eq:dpi_vel_3d}
%I_{DPI}(t,\mathbf{x},v_a,v_b)/I_{PI}(t,\mathbf{x},v_a,v_b) = v(t,\mathbf{x})\ .
%\end{equation}

%Extension to elliptically-anisotropic case would be a natural further step to take. It should be noticed that parameters $\beta$
%and $v_{bias}$ in Gaussian path integral can take different values for different migration directions along the dataset, what 
%would allow us to perform finer tail elimination in the presence of anisotropy.  

\appendix
\subsection{Appendix A: pre-stack path\new{-summation migration}\old{pre-stack path-integral evaluation}}

\cite{keydar09} describe pre-stack data imaging using path-integral formulation where the
curvature of the diffraction hyperbola acts as a parameter describing different images
being stacked. In this appendix, we derive a direct and  analytical \new{formula for}
unweighted pre-stack \old{path-integral formula}\new{path-summation imaging} based on the extension of velocity continuation
technique to the pre-stack domain.   
\cite{Fomel_03_VC_,Fomel_03_VC} describes velocity continuation phase-shift
for a pre-stack case:

\begin{equation}
\label{eq:prestack_vc}
\tilde{P}(\Omega,k,v) = \sum_{h}\tilde{P_0}(\Omega,k,h,v_0) e^{ i \frac{k^2 (v_0^2 - v^2)}{16\Omega}\ +\ 4i\Omega h^2 ( \frac{1}{v^2} - \frac{1}{v_0^2} ) },
\end{equation}
where $h$ is a constant half-offset value, and $v_0$ is a velocity for the first migration run
%needed to avert division by zero in the degree of exponent and
applied before cascade of VC transformations. 
%Therefore, $v_0$ can be treated as a reference velocity, in respect to which migration velocity perturbation is
%considered.
\new{Integral describing pre-stack path-summation migration}\old{Path-integral formulation} will take the following form:

\begin{multline}
\label{eq:prestack_pi1}
I(\Omega,k,v_a,v_b) = \int_{v_a}^{v_b} \sum_{h}\tilde{P_0}(\Omega,k,h,v_a) e^{ i \frac{k^2 (v_a^2 - v^2)}{16\Omega}\ +\ 4i\Omega h^2 ( \frac{1}{v^2} - \frac{1}{v_a^2} ) } dv\\
= \sum_{h}\tilde{P_0}(\Omega,k,h,v_a) \int_{v_a}^{v_b} e^{ i \frac{k^2 (v_a^2 - v^2)}{16\Omega}\ +\ 4i\Omega h^2 ( \frac{1}{v^2} - \frac{1}{v_a^2} ) } dv\\
= \sum_{h}\tilde{P_0}(\Omega,k,h,v_a) F(\Omega,k,h,v_a,v_b).
\end{multline}
$F(\Omega,k,h,v_a,v_b)$ is a pre-stack path\old{-integral}\new{-summation migration} filter, which can be evaluated analytically as: 

% CORRECT - without 2 pi
\begin{multline}
\label{eq:prestack_pi2}
F(\Omega,k,h,v_a,v_b) = \frac{e^{-i\frac{\pi}{4}}\sqrt{\Omega \pi}}{k} e^{ \frac{iv_a^2k^2}{16\Omega} - \frac{4i\Omega h^2}{v_a^2} } \bigg[ e^{-2\lambda(\Omega,k) \mu(\Omega,h)}\mbox{erf}\big(\lambda(\Omega,k)v + \frac{\mu(\Omega,h)}{v}\big)
\\+ e^{2\lambda(\Omega,k) \mu(\Omega,h)}\mbox{erf}\big(\lambda(\Omega,k)v - \frac{\mu(\Omega,h)}{v}\big) \bigg] \bigg|_{v_a}^{v_b},
\end{multline}
where $\lambda(\Omega,k) = e^{i\frac{\pi}{4}} \frac{k}{4\sqrt{\Omega}}$ and
$\mu(\Omega,h) = 2ie^{i\frac{\pi}{4}} h\sqrt{\Omega}$. 

\section{Appendix B: Double \new{path-summation migration}\old{path integral}}

Double path-integral formulation \new{was} proposed by \cite{schleicher_costa09}\new{, \cite{santos2016robust}}.
\new{In unweighted path-summation migration framework it can be described by}\old{corresponds to} the following equation:
\begin{equation}
\label{eq:dpi}
I_{DPI}(t,x,v_a,v_b) = \int^{v_b}_{v_a} v\ P(t,x,v)\,dv\ .
\end{equation}
Velocity distribution can be acquired through a smooth division of the double path\new{-summation} integral
by the path\new{-summation} integral:
\begin{equation}
\label{eq:dpi_vel}
v(t,x) = I_{DPI}(t,x,v_a,v_b)/I_{PI}(t,x,v_a,v_b)\ .
\end{equation}

We propose to evaluate the double path\new{-summation} integral analytically as well:
%\begin{equation}
%\begin{eqnarray}
%\begin{multline}
\begin{align}
\label{eq:dpi_pr}
I_{DPI}(\Omega,k,v_a,v_b) & = \int^{v_b}_{v_a} v\ \hat{P}_0(k,\Omega)\ e^{-\frac{i k^2 v^2}{16\Omega}}dv\ \nonumber \\
& = \hat{P}_0(\Omega,k)\ \int^{v_b}_{v_a} v\ e^{-\frac{i k^2 v^2}{16\Omega}}dv \nonumber \\
& =  \hat{P}_0(\Omega,k) \cdot F_{DPI}(\Omega,k,v_a,v_b),
\end{align}
%\end{multline}
%\end{eqnarray}
%\end{equation}
where the filter $F_{DPI}(\Omega,k,v_a,v_b)$ has the following analytical expression:
\begin{equation}
\label{eq:dpi_pr2}
F_{DPI}(\Omega,k,v_a,v_b) = \frac {8 i \Omega}{k^2} e^{-\frac{ik^2v^2}{16\Omega}}\ \bigg|_{v_a}^{v_b}.
\end{equation}

%\plot{a_diffr_data}{width=0.8\columnwidth}{Modelled diffraction hyperbola: velocity = 1.5 km/s.}
%\plot{a-path-integral}{width=0.8\columnwidth}{Unweighted path integral image acquired by formulae~(\ref{eq:pi_fft}) and~(\ref{eq:erfi_pi_fft}). }
%\plot{a-path-int-te}{width=0.8\columnwidth}{Unweighted path integral image after application of tails elimination workflow.}
%\multiplot*{8}{a_path1,a_path2,a_pre_s,a_pre_h,a_sig_s,a_sig_h,a-tails,a-path-int-te}{width=0.35\textwidth}{}

%We propose the following workflow to apply path-integral formulation for
%diffraction imaging purposes:
%\begin{enumerate}
% \item Separate reflections and diffractions, for instance, using a plane-wave destruction filter \cite[]{fomel_landa_taner07};
% \item Apply $\sigma=t^2$ transform to the diffraction image;
% \item FFT transform time-stretched diffraction data;
% \item Evaluate desired filter values (equations~\ref{eq:erfi_pi_fft},\ref{eq:dpi_pr2},\ref{eq:gpi_int},\ref{eq:pathint3d},\ref{eq:dpi_pr3d},\ref{eq:gaussian3d},\ref{eq:prestack_pi2})
% on the same $(\Omega,\mathbf{k})$ grid, which has been created by FFT of diffraction data;
% \item Multiply the result of step 4 by FFT of time-stretched data;
% \item Inverse FFT and inverse time-stretch to generate path-integral image;
 %\item Compute tails' initial models through velocity continuation of diffraction image to $v_a$ and $v_b$;
 %\item Update an overmigrated tail model by adaptive filtering and subtract it from the path-integral image;
%same procedure for an undermigrated tail.
%\end{enumerate}

To extract migration velocity as described by the equation~(\ref{eq:dpi_vel}) we need to
evaluate both double path\new{-summation} integral and \old{unweighted}path\new{-summation} integral. These evaluations can be done 
analytically as described by equations~\ref{eq:pi_fft_filter} and~\ref{eq:dpi_pr}. A smooth division
procedure \new{\cite[]{fomel2007shaping}} is applied in equation~\ref{eq:dpi_vel} to acquire a regularized time migration velocity estimate, which
then can be applied to image the subsurface using any available time imaging algorithm.

% WITH TWO PI
%\begin{multline}
%\label{eq:prestack_pi2}
%F(\Omega,k,h,v_a,v_b) = \frac{e^{-i\frac{\pi}{4}}\sqrt{\Omega}}{\sqrt{2}k} \bigg[ e^{2\lambda(\Omega,k) \mu(\Omega,h)}\mbox{erf}\big(\lambda(\Omega,k)v + \frac{\mu(\Omega,h)}{v}\big)
%\\+ e^{-2\lambda(\Omega,k) \mu(\Omega,h)}\mbox{erf}\big(\lambda(\Omega,k)v - \frac{\mu(\Omega,h)}{v}\big) \bigg] \bigg|_{v_a}^{v_b},
%\end{multline}
%where $\lambda(\Omega,k) = e^{i\frac{\pi}{4}}\ \frac{k\sqrt{2\pi}}{4\sqrt{\Omega}}$ and
%$\mu(\Omega,h) = ie^{i\frac{\pi}{4}}\ h\sqrt{8\pi\Omega}$. 
%transform the formula to the case without 2\pi
%gaussian weighting scheme
%dpi - asymptotic evaluation

%\onecolumn
%\addtolength{\topmargin}{+.7in}
\bibliographystyle{seg}
\bibliography{SEG,sources,proposal}
