\published{Geophysical Journal International, 193, 960-969, (2013)}
\renewcommand{\thefootnote}{\fnsymbol{footnote}}

\lefthead{Song, Fomel, \& Ying}
\righthead{LFD and LFFD}

\title{Lowrank finite-differences and lowrank Fourier finite-differences for seismic wave extrapolation 
in the acoustic approximation}

\author{Xiaolei Song\footnotemark[1], Sergey Fomel\footnotemark[1], and Lexing Ying\footnotemark[2]}

\address{\footnotemark[1]Bureau of Economic Geology, 
John A. and Katherine G. Jackson School of Geosciences, \\
The University of Texas at Austin \\
University Station, Box X \\
Austin, TX 78713-8924 \\
USA \\
sergey.fomel@beg.utexas.edu \\ 
\footnotemark[2]Department of Mathematics \\
The University of Texas at Austin \\
1 University Station \\
Austin, TX 78712 \\
USA \\
lexing@math.utexas.edu
}
%\pagerange{\pageref{firstpage}--\pageref{lastpage}}
%\pubyear{2012}
%\let\leqslant=\leq

%\newtheorem{theorem}{Theorem}[section]


\label{firstpage}


\maketitle

\begin{abstract}
  We introduce a novel finite-difference (FD) approach for seismic wave extrapolation in time.  
  We derive the coefficients of the finite-difference operator from 
  a lowrank approximation of the space-wavenumber, wave-propagator matrix. 
  Applying the technique of lowrank finite-differences, 
  we also improve the finite difference scheme of the two-way Fourier finite differences (FFD).
  We call the new operator lowrank Fourier finite differences (LFFD). 
   Both the lowrank FD and lowrank FFD methods can be applied to enhance accuracy in seismic imaging by reverse-time migration.
  Numerical examples confirm the validity of the proposed technique.
\end{abstract}

%\begin{keywords}
%finite differences, lowrank, Fourier finite differences, wave extrapolation.
%\end{keywords}

\section{Introduction}

Wave extrapolation in time is crucial in seismic modeling, imaging (reverse-time migration), and full-waveform inversion. 
The most popular and straightforward way to implement wave extrapolation in time is 
the method of explicit finite differences (FDs), 
which is only conditionally stable and 
suffers from numerical dispersion \cite[]{Wu,kastner}. 
In practice, a second-order FD for temporal derivatives and a high-order FD for spatial derivatives are often employed to reduce dispersion and improve accuracy.
FD coefficients are conventionally determined using a Taylor-series expansion around zero wavenumber \cite[]{dablain,kindelan}.
Therefore, traditional FD methods are accurate primarily for long-wavelength components.
 
More advanced methods have been applied previously to FD schemes in the case of one-way wave extrapolation (downward continuation).
\cite{holberg,holberg1} designed the derivative operator by matching the spectral response in the wavenumber domain.
\cite{soubaras} adopted the Remez exchange algorithm to obtain the $L_{\infty}$-norm-optimized coefficients for second-derivative filters.  
\cite{mousa} designed stable explicit depth extrapolators using projections onto convex sets (POCS). 
These approaches have advantages over conventional FD methods in their ability to propagate shorter-wavelength seismic waves correctly. 
To satisfy the general criterion for optimal accuracy \cite[]{gellerg},
\cite{geller1} derived an optimally accurate time-domain finite difference method for computing
synthetic seismograms for 1-D problems extended later to 2-D and 3-D \cite[]{geller2}.
\cite{liu} proposed FD schemes for two-way scalar waves on the basis of time-space dispersion relations and plane-wave theory.
Later on, they suggested adaptive variable-length spatial operators in order to decrease computing costs significantly without reducing accuracy \cite[]{adapt}.
The Liu-Sen scheme satifies the exact dispersion relation and has
greater accuracy and better stability than a conventional one. 
However, it still uses an expansion around the zero wavenumber.

In sedimentary rocks, anisotropic phenomena are often observed as a result 
of layering lithification, which is described as transversely isotropic (TI). 
Tectonic movement of the crust may rotate the
rocks and tilt the natural vertical orientation
of the symmetry axis
(VTI), causing a tilted TI (TTI) anisotropy. 
Wavefields in anisotropic media are well described by the anisotropic
elastic-wave equation.  
However, in practice, seismologists often have little
information about shear waves and
prefer to deal with scalar wavefields.
Conventional P-wave modeling may contain shear-wave numerical artifacts in the simulated wavefield \cite[]{grechkat,zhang2,duveneckt}.
Those artifacts as well as sharp changes in symmetry-axis tilting may introduce severe numerical
dispersion and instability in modeling.
\cite{yoon} proposed to reduce the instability by introducing elliptical anisotropy in regions
with rapid tilt changes.
\cite{fletcher} suggested to include a finite S-wave velocity in order to enhance stability
when solving coupled equations.
These methods can alleviate the instability problem;
however, they may alter the wave propagation kinematics or leave residual S-wave components in the P-wave simulation.
A number of spectral methods are proposed to provide solutions
which can completely avoid the shear-wave artifacts 
\cite[]{etgen,chu,songx,lr,fowlerg,ge,cheng} at the cost of several Fourier
transforms per time step.
These methods differ from conventional pseudo-spectral methods \cite[]{gazdag,fornberg},
because they approximate the space-wavenumber mixed-domain propagation matrix instead of a Laplacian operator. 

Our goal is to design an FD scheme that matches the spectral response in the mixed space-wavenumber domain for a wide range of spatial wavenumbers.
The scheme is derived from the lowrank approximation of the mixed-domain operator \cite[]{ying,lr} and its representation by FD with adapted coefficients. 
We derive this kind of FD schemes which we call lowrank FD or LFD for both isotropic and TTI media.
Using this approach, we only need to compute the FD coefficients once
and save them for the whole process of wave extrapolation or reverse-time migration.
 The method is flexible enough to control accuracy by the rank of approximation and by FD order selection. 

The paper is organized as follows. 
We first give a brief review of the lowrank approximation method.
As a spectral method, it provides an accurate wave extrapolation,
but it is not optimally efficient.
Next, we present the derivation of LFD. 
LFD as an FD method can reduce the cost and 
is also more adaptable for parallel computing on distributed computer systems.
We also propose lowrank Fourier FD (LFFD), by replacing the original FD operator in the two-way Fourier FD (FFD) \cite[]{songx} with the corresponding LFD. 
LFFD improves the accuracy of FFD, in particular in tilted transversely isotropic (TTI) media.
A number of synthetic examples of increasing complexity validate the proposed methods.
In this paper, we solve the acoustic wave equation in constant-density media,
aiming at incorporating wave extrapolation with LFD and LFFD into seismic imaging by reverse-time migration. 
It is possible to extend LFD and LFFD to variable-density media by factoring the second-order k-space operator into first-order parts \cite[]{tabei,denffd}.


\section{Theory}

\subsection{Lowrank Approximation}

The following acoustic-wave equation is widely used in 
seismic modeling and reverse-time migration \cite[]{etgen1}:
\begin{equation}
\label{eq:acoustic} 
\frac{\partial^2p}{\partial t^2} = v(\mathbf{x})^2\,\nabla^2p\;,
\end{equation}
where $\mathbf{x}\,=\,(x_1,x_2,x_3)$, $p(\mathbf{x},t)$ is the seismic pressure wavefield 
and $v(\mathbf{x})$ is the propagation velocity.
Assuming a constant velocity, $v$, after Fourier transform in space,
 we could obtain the following explicit expression,
\begin{equation}
\label{eq:ode} 
\frac{d^2\hat{p}}{dt^2} = -v^2|\mathbf{k}|^2\hat{p}\;,
\end{equation}
where
\begin{equation}
\label{eq:p} 
\hat{p}(\mathbf{k},t)=\int^{+\infty}_{-\infty}{p(\mathbf{x},t)e^{i\mathbf{k}\cdot\mathbf{x}}d\mathbf{x}}\;,
\end{equation}
and $\mathbf{k}\,=\,(k_1,k_2,k_3)$.

Equation~\ref{eq:ode} has an explicit solution:
\begin{equation}
\label{eq:fourier} 
\hat{p}(\mathbf{k},t+\Delta t) = e^{\pm i|\mathbf{k}|v\Delta t}\hat{p}(\mathbf{k},t)\;.
\end{equation}
A second-order time-marching scheme and the inverse Fourier transform lead to 
the well-known expression \cite[]{etgen3,yu}\hyphenation{Sou-ba-ras}:
%\begin{eqnarray}
%\label{eq:exact} 
%\lefteqn{p(\mathbf{x},t+\Delta t)+p(\mathbf{x},t-\Delta t)  = }\nonumber \\
%& & 2\int^{+\infty}_{-\infty}{\hat{p}(\mathbf{k},t)\cos(|\mathbf{k}|v\Delta t)e^{-i\mathbf{k}\cdot\mathbf{x}}d\mathbf{k}}\;.
%\end{eqnarray} 

\begin{equation}
\label{eq:exact} 
p(\mathbf{x},t+\Delta t)+p(\mathbf{x},t-\Delta t)  = 2\int^{+\infty}_{-\infty}{\hat{p}(\mathbf{k},t)\cos(|\mathbf{k}|v\Delta t)e^{-i\mathbf{k}\cdot\mathbf{x}}d\mathbf{k}}\;.
\end{equation} 

Equation~\ref{eq:exact} provides an efficient solution 
in the case of a constant-velocity medium with the aid of the fast Fourier transform (FFT). 
When velocity varies in space, 
equation~\ref{eq:exact} can provide an approximation by replacing $v$ with $v(\mathbf{x})$. 
In such a case, a mixed-domain term, $\cos(|\mathbf{k}|v(x)\Delta t)$, appears in the expression.
As a result, the computational cost of a straightforward application of equation~\ref{eq:exact} is $O(N_x^2)$, where $N_x$ is the total size of the three-dimensional space grid.

\cite{ying,lr} showed that the mixed-domain matrix,
\begin{equation}
\label{eq:mat}
W(\mathbf{x},\mathbf{k}) = \cos(|\mathbf{k}|v\Delta t),
\end{equation} 
can be efficiently decomposed into a separate representation of the following form:
\begin{equation}
  \label{eq:lra}
  W(\mathbf{x},\mathbf{k}) \approx \sum\limits_{m=1}^M \sum\limits_{n=1}^N W(\mathbf{x},\mathbf{k}_m) a_{mn} W(\mathbf{x}_n,\mathbf{k}),
\end{equation}
where $W(\mathbf{x},\mathbf{k}_m)$ is a submatrix of $W(\mathbf{x},\mathbf{k})$ that consists of 
selected columns associated with ${\mathbf{k}_m}$,
$W(\mathbf{x}_n,\mathbf{k})$ is another submatrix that contains 
selected rows associated with ${\mathbf{x}_n}$,
and $a_{mn}$ stands for the middle matrix coefficients.
The construction of the separated form~\ref{eq:lra} follows the method of \cite{eng2009}.
The main observation is that the columns of $W(\mathbf{x},\mathbf{k}_m)$ need to span the column space of the original matrix and that the rows of $W(\mathbf{x}_n,\mathbf{k})$ need to span the row space as well as possible.

Representation~(\ref{eq:lra}) speeds up the computation of
$p(\mathbf{x},t+\Delta t)$ because 
\begin{eqnarray}
\nonumber
  p(\mathbf{x},t+\Delta t) + p(\mathbf{x},t-\Delta t) = 2 \int e^{-i
    \mathbf{x} \mathbf{k}} W(\mathbf{x},\mathbf{k})
  \hat{p}(\mathbf{k},t) d \mathbf{k} \\
 \approx 2 \sum\limits_{m=1}^M W(\mathbf{x},\mathbf{k}_m) \left(
   \sum\limits_{n=1}^N a_{mn} \left(\int e^{-i \mathbf{x}\mathbf{k}}
     W(\mathbf{x}_n,\mathbf{k}) \hat{p}(\mathbf{k},t) d\mathbf{k}
   \right) \right).
  \label{eq:lrapp}
\end{eqnarray}
%\begin{equation}
%\begin{array}{*{20}c}
%  p(\mathbf{x},t+\Delta t) + p(\mathbf{x},t-\Delta t) = 2 \int e^{-i \mathbf{x} \mathbf{k}} W(\mathbf{x},\mathbf{k}) \hat{p}(\mathbf{k},t) d \mathbf{k}\\
% \approx 2 \sum\limits_{m=1}^M W(\mathbf{x},\mathbf{k}_m) \left( \sum\limits_{n=1}^N a_{mn} \left(\int e^{i \mathbf{x}\mathbf{k}} W(\mathbf{x}_n,\mathbf{k}) \hat{p}(\mathbf{k},t) d\mathbf{k} \right) \right).\\
%\end{array}
%\end{equation}
Evaluation of equation~\ref{eq:lrapp} requires N inverse FFTs. 
Correspondingly, the lowrank approximation reduces the cost to $O(NN_x\log N_x)$, where $N$ is a small integer,
which is related to the rank of the above decomposition
and can be automatically calculated at some given error level with a pre-determined $\Delta t$.
Increasing the time step size $\Delta t$ may increase the rank of the
approximation ($M$ and $N$) and correspondingly the number of the required Fourier
transforms.


As a spectral method, the lowrank approxmation is highly accurate. 
However, its cost is several FFTs per time step.
Our goal is to reduce the cost further by deriving an FD scheme that matches 
the spectral response of the output from the lowrank decomposition.



\subsection{Lowrank Finite Differences}

In a matrix notation, the lowrank decomposition problem takes the following form:
\begin{equation}
  \label{eq:lramatrix}
  \mathbf{W} \approx \mathbf{W}_1 \cdot \mathbf{A} \cdot \mathbf{W}_2,
\end{equation} 
where $\mathbf{W}$ is the $N_x\times N_x$ matrix with entries
$W(\mathbf{x},\mathbf{k})$, $\mathbf{W}_1$ is the submatrix of $\mathbf{W}$ that
consists of the columns associated with $\{\mathbf{k}_m\}$, $\mathbf{W}_2$ is
the submatrix that consists of the rows associated with $\{\mathbf{x}_n\}$,
and $\mathbf{A} = \{a_{mn}\}$.

Note that $\mathbf{W}_2$ is a matrix related only to wavenumber $\mathbf{k}$. 
We propose to further decompose it as follows:
\begin{equation}
  \label{eq:w2}
  \mathbf{W_2} \approx \mathbf{C} \cdot \mathbf{B}, 
\end{equation}
where we determine $\mathbf{B}$ to be an $L\times N_x$ matrix, and the entry, $\mathbf{B}(\mathbf{\xi},\mathbf{k})$, has the form of $\cos(\sum\limits_{j=1}^3{\xi^j k_j\Delta x_j})$,
in which $\mathbf{\xi}$ is a 3-D integer vector, $\mathbf{\xi}=(\xi^1,\xi^2,\xi^3)$, $k_j$ is the jth component of wavenumber $\mathbf{k}$, $\Delta x_j$ is the space grid size in the jth direction, $j=1,2,3$, 
and $\mathbf{C}$ is the matrix product of $\mathbf{W}_2$ and the pseudo-inverse of $\mathbf{B}$.
In practice, we apply a weighted-inversion to achieve the pseudo-inverse:
putting a larger weight on the low-wavenumber part and a smaller weight on the high-wavenumber part to enhance the stability.
Now we have a new decompostion for the mixed-domain matrix: 
\begin{equation}
  \label{eq:dematrix}
  \mathbf{W} \approx \mathbf{G} \cdot \mathbf{B},
\end{equation} 
where $\mathbf{G}$ is an $N_x\times L$ matrix, 
\begin{equation}
  \label{eq:Gmatrix}
  \mathbf{G} = \mathbf{W}_1 \cdot \mathbf{A} \cdot \mathbf{C},
\end{equation} 
and
\begin{eqnarray}
\nonumber
  p(\mathbf{x},t+\Delta t) + p(\mathbf{x},t-\Delta t) = \\
\nonumber
2 \int e^{-i \mathbf{x} \mathbf{k}} W(\mathbf{x},\mathbf{k})
\hat{p}(\mathbf{k},t) d \mathbf{k} \approx 2 \sum\limits_{m=1}^L
G(\mathbf{x},m)  \left(\int e^{-i \mathbf{x}\mathbf{k}}
  B(\mathbf{\xi}_m,\mathbf{k}) \hat{p}(\mathbf{k},t) d\mathbf{k}
\right)\\
\nonumber
 \approx  \sum\limits_{m=1}^L G(\mathbf{x},m)  \left(\int e^{-i
     \mathbf{x}\mathbf{k}} 2\cos(\sum\limits_{j=1}^3{\xi_m^j k_j\Delta
     x_j}) \hat{p}(\mathbf{k},t) d\mathbf{k} \right) \\
\approx  \sum\limits_{m=1}^L G(\mathbf{x},m)  \left[\int e^{-i \mathbf{x} \cdot \mathbf{k}} (e^{i \sum\limits_{j=1}^3{\xi_m^j k_j\Delta x_j}} + e^{-i \sum\limits_{j=1}^3{\xi_m^j k_j\Delta x_j}})\hat{p}(\mathbf{k},t) d\mathbf{k} \right].
\label{eq:la}
\end{eqnarray}
According to the shift property of FFTs, we finally obtain an expression in the space-domain 
\begin{equation}
  \label{eq:stencil}
  p(\mathbf{x},t+\Delta t) + p(\mathbf{x},t-\Delta t) = \sum\limits_{m=1}^L G(\mathbf{x},m) [p(\mathbf{x}_L,t)+p(\mathbf{x}_R,t)], 
\end{equation}
where $\mathbf{x}_L\,=\,(x_1-\mathbf{\xi}_m^1\Delta x_1,x_2-\mathbf{\xi}_m^2\Delta x_2,x_3-\mathbf{\xi}_m^3\Delta x_3)$, and 
 $\mathbf{x}_R\,=\,(x_1+\mathbf{\xi}_m^1\Delta x_1,x_2+\mathbf{\xi}_m^2\Delta x_2,x_3+\mathbf{\xi}_m^3\Delta x_3)$.\\

Equation~\ref{eq:stencil} indicates a procedure of finite differences for wave extrapolation: the integer vector, $\mathbf{\mathbf{\xi}_m}\,=\,(\xi_m^1,\xi_m^2,\xi_m^3)$ provides the stencil information, and $G(\mathbf{x},m)$ stores the corresponding coefficients.
We call this method \emph{lowrank finite differences} (LFD) 
because the finite-difference coefficients are derived from a lowrank approximation of the mixed-domain propagator matrix.
We expect the derived LFD scheme to accurately propagate seismic-wave components within a wide range of wavenumbers, 
which has advantages over conventional finite differences that focus mainly on small wavenumbers. 
In comparison with the Fourier-domain approach, the cost is reduced to $O(L\,N_x)$, 
where $L$, as the row size of matrix $\mathbf{B}$, is related to the order of the scheme. 
$L$ can be used to characterize the number of FD coefficients in the LFD scheme, shown in equation~\ref{eq:stencil}. 
Take the 1-D 10th order LFD as an example, there are 1 center point, 5 left points ($\mathbf{x}_L$) and 5 right ones ($\mathbf{x}_R$).
So $\xi_m^1={0,1,2,3,4,5}$, and $\mathbf{\xi}_m=(\xi_m^1,0,0)$.
Thanks to the symmetry of the scheme, 
coefficients of $\mathbf{x}_L$ and $\mathbf{x}_R$ are the same, as indicated by equation~\ref{eq:stencil}.
As a result, one only needs 6 coefficients: $L=6$. 

\inputdir{oned}
\multiplot{4}{Mexact,Mlrerr,Mapperr,Mfd10err}{width=0.4\textwidth}{(a) Wavefield extrapolation matrix for 1-D linearly increasing velocity model. Error of wavefield extrapolation matrix by:(b) lowrank approximation, (c) the 10th-order lowrank FD (d) the 10th-order FD.}

We use a one-dimentional example shown in Figure 1 to demonstrate the accuracy of the proposed LFD method. 
The velocity linearly increases from 1000 to 2275 m/s. 
The rank is 3 ($M=N=3$) for lowrank decomposition for this model with 1 ms time step. 
The propagator matrix is shown in Figure~\ref{fig:Mexact}.
Figure~\ref{fig:Mlrerr}-Figure~\ref{fig:Mfd10err} display the errors corresponding to different approximations.
The error by the 10th-order lowrank finite differences (Figure~\ref{fig:Mapperr}) appears significantly smaller than that of the 10th-order finite difference (Figure~\ref{fig:Mfd10err}).
Figure~\ref{fig:slicel} displays the middle column of the error matrix. Note that the error of the LFD is significantly closer to zero than that of the FD method. 

\sideplot{slicel}{width=0.8\textwidth}{Middle column of the error matrix. Solid line: the 10th-order LFD. Dash line: the 10th-order FD.}


To analyze the accuracy, we let
\begin{equation}
  \label{eq:plane}
  p(\mathbf{x},t) = e^{i(\mathbf{k}\cdot\mathbf{x}-\omega t)}, 
\end{equation}
by using the plane wave theory.
Inserting \ref{eq:plane} into equation~\ref{eq:stencil} and also adopting the dispersion relation $\omega\,=\,|k|\,v$, 
defines the phase velocity of LFD ($v_{LFD}$) as follows:
\begin{equation}
  \label{eq:phasev}
  v_{LFD}=\frac{1}{|k|\Delta t} \arccos(\sum\limits_{m=1}^L G(\mathbf{x},m) (\cos(\xi_m^1\,k_1\Delta x_1)+\cos(\xi_m^2\,k_2\Delta x_2)+\cos(\xi_m^3\,k_3\Delta x_3))) , 
\end{equation}
For 1-D 10th order LFD, $L=6$, $\mathbf{\xi}_m=(\xi_m^1,0,0)$ and $\xi_m^1={0,1,2,3,4,5}$.
With equation~\ref{eq:phasev}, 
we can calculate phase-velocities ($v_{LFD}$) by 1-D 10th order LFD with different velocities ($v=2500,3000,3500,4000$),
and we use the ratio $\delta=v_{LFD}/v$ to describe the dispersion of FD methods.
Figure~\ref{fig:app} displays 1D dispersion curves by 1-D 10th order LFD,     
and Figure~\ref{fig:fd10} shows those by conventional FD method. 
Note that compared with the conventional FD method, LFD is accurate in
a wider range of wavenumbers (up to 70\% of the Nyquist frequency). 

\inputdir{dispersion1}

\multiplot{2}{app,fd10}{width=0.4\textwidth}{Plot of 1-D dispersion curves for different velocities, $v=2500$ (red), $3000$ (pink), $3500$ (green), $4000$ (blue) $m/s$, $\Delta t=1$ ms, $\Delta x=10$ m by: (a) the 10th-order  LFD (b) the 10th-order conventional FD.}

%As Figure~\ref{fig:Mlr_err} and Figure~\ref{fig:slice} show, lowrank FD has relatively larger error in big wavenumber part. 
%We apply weighted inversion for matrix $\mathbf{C}$ in equation~\ref{eq:w2}: putting high weight to small wavenumbers but low weight to big ones.
%Figure~\ref{fig:curves} shows error plots for lowrank FD of different orders with different cut. From this figure, we find $\frac{2}{3}$ of the whole range of wavenumbers is a proper threshold.   %\plot{curves}{width=0.45\textwidth}{Error plots for lowrank FD of different orders with different cut. Solid line: 4th order. Short dash line: 6th order. Dot line: 8th order. Long dash line: 10th order.}
\subsection{TTI Lowrank Finite Differences}
The LFD approach is not limited to the isotropic case.
In the case of transversely isotropic (TTI) media, the term $v(\mathbf{x})\,|\mathbf{k}|$ on the right-hand side of equation~\ref{eq:mat},
can be replaced with the acoustic approximation \cite[]{alkhalifah1,alkhalifah2,anelliptic},
\begin{equation}
\label{eq:ttiexact} 
\begin{array}{*{20}l}
f(\mathbf{v},\mathbf{\hat{k}},\eta)=\;
\displaystyle \sqrt{\frac{1}{2}(v_1^2\,\hat{k}_1^2+v_2^2\,\hat{k}_2^2)+\frac{1}{2}\sqrt{(v_1^2\,\hat{k}_1^2+v_2^2\,\hat{k}_2^2)^2-\frac{8\eta}{1+2\eta}v_1^2v_2^2\,\hat{k}_1^2\,\hat{k_2^2}}}\;, 
\end{array}
\end{equation}
where $v_1$ is the P-wave phase velocity in the symmetry plane,
$v_2$ is the P-wave phase velocity in the direction normal to the symmetry plane,
$\eta$ is the anisotropic elastic parameter \cite[]{alkhalifah} related to Thomsen's elastic parameters $\epsilon$ and $\delta$ \cite[]{thomsen} by
\begin{equation}
\label{eq:tom}
\frac{1+2\delta}{1+2\epsilon}=\frac{1}{1+2\eta}\,;
\end{equation}
and $\hat{k}_1$ and $\hat{k}_2$ stand for the wavenumbers evaluated in a rotated coordinate system aligned with the symmetry axis:

\begin{equation}
\label{eq:wavenumber}
\begin{array}{*{20}c}
\hat{k}_1=k_1\cos{\theta}+k_2\sin{\theta}\; \\
\hat{k}_2=-k_1\sin{\theta}+k_2\cos{\theta}\; 
 \end{array}
\end{equation}
where $\theta$ is the tilt angle measured with respect to vertical. 
Using these definitions, we develop a version of the lowrank finite-difference scheme for 2D TTI media. 

\subsection{Lowrank Fourier Finite Differences}
\cite{songx} proposed FFD approach to solve the two-way wave equation. 
The FFD operator is a chain operator that combines FFT and FD, analogous to the concept introduced previously for one-way wave extrapolation by \cite{ffd}.
The FFD method adopts the pseudo-analytical solution of the acoustic wave equation, shown in equation~\ref{eq:exact}.
It first extrapolates the current wavefield with some constant reference velocity 
and then applies FD to correct the wavefield according to local model parameter variations.  
In the TTI case, the FD scheme in FFD is typically a 4th-order operator, derived from Taylor's expansion around $k=0$. 
However, it may exhibit some dispersion caused by the inaccuracy of the FD part. 
We propose to replace the original FD operator with lowrank FD in order to increase the accuracy of FFD in isotropic and anisotropic media.
We call the new operator lowrank Fourier Finite Differences (LFFD). 

\section{Numerical Examples}


Our first example is wave extrapolation in a 2-D, smoothly variable velocity model. The velocity ranges between 500 and 1300 m/s, and is formulated as 
\begin{equation}
v(x,z)\,=\,500+1.2\times10^{-4}(x-800)^2+10^{-4}(z-500)^2;
\end{equation}
$0\le x\le2560,\;0\le z\le2560.$
A Ricker-wavelet source with a 20 Hz dominant frequency ($f_{d}$) is located at the center of the model. 
The maximum frequency ($f_{max}$) is around 60 Hz.  
The amplitude corresponding to $f_{max}$ is about $10^{-5}$ of that of $f_{d}$.
For numerical simulations based on this model, we use the same grid size: $\Delta x\,=\,5$ m and $\Delta t\,=\,2$ ms.
We use $\alpha=v_{max}\,\Delta t/\Delta x$ to specify the stability condition and $\beta=v_{min}/(f_{max}\Delta x)$ as the dispersion factor, where $v_{max}$ and $v_{min}$ are the maximum and minimum velocities of the model.
The dispersion factor $\beta$ indicates the number of sampling points 
for the minimum wavelength. 
For simulations with the above parameters, $\alpha\,\approx\,0.52$ and
$\beta \,\approx\, 1.67$.

\inputdir{twod}

\multiplot{2}{wavfd7sn,wavel}{width=0.4\textwidth}{Wavefield snapshot in a varable velocity field by: (a) conventional 4th-order FD method (b) Lowrank method.}

It is easy to observe obvious numerical dispersions on the snapshot computed by the 4th-order FD method (Figure~\ref{fig:wavfd7sn}). 
The lowrank FD method with the same order exhibits higher accuracy and fewer dispersion artifacts (Figure~\ref{fig:wavapp7sn}). 
The approximation
rank decomposition in this case is $N=3\, M=4$, with the expected error of
less than $10^{-4}$.
Figure~\ref{fig:wavapp10sn} displays the snapshot by the 10th-order LFD method with a larger time step: $\Delta t\,=\,2.5$ ms, $\alpha\,\approx\,0.65$. Note that the result is still accurate. However, the regular FD method becomes unstable in this case.   
For comparison, Figure~\ref{fig:wavel} displays the snapshot by the lowrank method with the same time step. 
The approximation
rank decomposition in this case is $N=M=3$, with the expected error of
less than $10^{-4}$.
Thanks to the spectral nature of the algorithm, the result appears accurate and free of dispersion artifacts.   

\multiplot{2}{wavapp7sn,wavapp10sn}{width=0.4\textwidth}{Wavefield snapshot in a varable velocity field by: (a) the 4th-order lowrank FD method (b) the 10th-order lowrank FD method. Note that the time step is 2.5 ms and the LFD result is still accurate. However, the FD method is unstable in this case.}

Next, we test the lowrank FD method in a complex velocity model.
Figure~\ref{fig:sub1} shows a part of the BP velocity model \cite[]{bp}, which is a complicated model containing a salt body and sharp velocity contrasts on the flanks of the salt body. 
We use a Ricker-wavelet at a point source. The dominant frequency is 17 Hz ($f_{max}\,\approx\,54$). 
The horizontal grid size $\Delta x$ is 12.5 m, the vertical grid size $\Delta z$ is 12.5 m, and the time step is 1.5 ms.
Thus $\alpha\,\approx\,0.57$ and $\beta\,\approx\,2.2$. 
The approximation
rank decomposition in this case is $N=4\, M=5$, with the expected error of
less than $10^{-4}$.
In this case, we adopt a disk-shaped compact scheme (8th-order) for LFD with a 4-point radius ($|\mathbf{\xi}|\,\le\,4$, $L=25$). 
Figure~\ref{fig:wavsnapabc} displays a wavefield snapshot in the above velocity model.
The snapshot is almost free of dispersions. 
This experiment confirms that the lowrank FD method is able to handle sharp velocity variations.

\inputdir{bp}
\plot{sub1}{width=0.6\textwidth}{Portion of BP 2004 synthetic velocity  model.}
\plot{wavsnapabc}{width=0.6\textwidth}{Wavefield snapshot by the 8th-order lowrank FD (compact scheme) in the BP Model shown in Figure~\ref{fig:sub1}.}

Our next example is wave propagation in a TTI model with a tilt of $45\,^{\circ}$ and smooth velocity variation ($v_x$: 800-1225.41 m/s, $v_z$: 700-883.6 m/s). 
Figure~\ref{fig:snapshotlfd} shows wavefield snapshots at different time steps by a 16th-order LFD operator 
in the TTI model. 
The space grid size is 5 m and the time step size is 2 ms.
So $\alpha\,\approx\,0.49$ and $\beta\,\approx\,2.3$.
The approximation
rank decomposition in this case is $N=6\, M=6$, with the expected error of
less than $10^{-4}$.
For TTI model, we adopt a high-order (16th order) LFD operator in order to reduce dispersions. 
The scheme is compact and shaped as a disk with a radius of 8 points ($L=99$).
%\plot{snapshot_lffd}{width=0.9\textwidth}{Wavefield snapshots by lowrank FFD in a TTI medium with tilt of 45 degrees.
%$v_x(x,z)=800+10^{-4}(x-1000)^2+10^{-4}(z-1200)^2$; $v_z(x,z)=700+10^{-4}(z-1200)^2$; $\eta=0.3$; $\theta=45\,^{\circ}$.}

\inputdir{aniso}
\multiplot{2}{snapshotlfd,snapshotlffd}{width=0.4\textwidth}{Wavefield snapshots in a TTI medium with a tilt of $45\,^{\circ}$ by: (a) Lowrank FD method; (b) Lowrank FFD method.$v_x(x,z)=800+10^{-4}(x-1000)^2+10^{-4}(z-1200)^2$; $v_z(x,z)=700+10^{-4}(z-1200)^2$; $\eta=0.3$; $\theta=45\,^{\circ}$.}

\cite{songx} showed an application of FFD method for TTI media. 
However, the example wavefield snapshot by FFD method still had some dispersion
caused by the fact that the FD scheme in the FFD operator is derived from Taylor's expansion around zero wavenumber. 
It was apparent that 4th-order FD scheme is not accurate enough for TTI case
and requires denser sampling per wavelength ($\beta\,\approx\,4.6$).
We first apply lowrank approximation
to the mixed-domain velocity correction term in FFD.
The rank  is $N=9\, M=9$, with the expected error of
less than $10^{-6}$.
Then we propose to replace that 4th-order FD operator with an 8th-order LFD compact scheme. 
The scheme has the shape of a disk with a radius of 4 points ($L=25$), the same as the one for LFD in the above BP model.
Figure~\ref{fig:snapshotlffd} shows wavefield snapshots by the proposed LFFD operator. 
The time step size is 1.5 ms ($\alpha\,\approx\,0.37$).
Note that the wavefront is clean and almost free of dispersion with $\beta\,\approx\,2.3$.
Because we use the exact dispersion relation, Equation~\ref{eq:ttiexact} for TTI computation,
there is no coupling of q-SV wave and q-P wave \cite[]{grechkat,zhang2,duveneckt} in our snapshots by either LFD or LFFD methods. 

\inputdir{bptti}
\multiplot{4}{vp0,vx0,yita0,theta0}{width=0.4\textwidth}{Partial region of the 2D BP TTI model. a: $v_z$. b: $v_x$. c: $\eta$. d:$\theta$.}

Next we test the LFD and LFFD methods in a complex TTI model.
Figure~\ref{fig:vp0}-\ref{fig:theta0} shows parameters for part of the BP 2D TTI model \cite[]{bptti}.
The dominant frequency is 15 Hz ($f_{max}\,\approx\,50$). 
The space grid size is 12.5 m and the time step is 1 ms.
Thus $\alpha\,\approx\,0.42$ and $\beta\,\approx\,2.4$. 
The approximation
rank decomposition for LFD method is $N=22\, M=22$, with the expected error of
less than $10^{-6}$. 
For FFD, $N=24\, M=30$, with the expected error of
less than $10^{-6}$.
Both methods are able to simulate an accurate qP-wave field in this model as shown in Figure~\ref{fig:snapshotslfd} and \ref{fig:snapshotslffd}.

\multiplot{2}{snapshotslfd,snapshotslffd}{height=0.4\textwidth}{Scalar wavefield snapshots by LFD and LFFD methods in the 2D BP TTI model.}

It is difficult to provide analytical stability analysis for LFD and LFFD operators. 
In our experience, the values of $\alpha$ are around 0.5 for 2D LFD and LFD methods appear to allow for a larger time step size than that of the LFFD method.
In TTI case, the conventional FD method for acoustic TTI has known issues of instability caused by shear-wave numerical artifacts or sharp changes in the symmetry-axis tilting \cite[]{grechkat,zhang2,duveneckt}.
Conventional methods may place limits on anisotropic parameters, smooth parameter models or include a finite shear-wave velocity to alleviate the instability problem\cite[]{yoon,zhang3,fletcher}.
Both LFD and LFFD methods are free of shear-wave artifacts. 
They require no particular bounds for anisotropic parameters and can also handle sharp tilt changes.
 
\section{Conclusions}

Explicit finite difference (FD) methods are the most popular and straightforward methods for seismic modeling and seismic imaging,
particularly for reverse-time migration.
Traditionally the coefficients of FD schemes are derived from a Taylor series expansion around the zero wavenumber.
We present a novel FD scheme: Lowrank Finite Differences (LFD), 
which is based on the lowrank approximation of the mixed-domain space-wavenumber propagator.
LFD uses compact FD schemes, which are more suitable for parallelization on multi-core computers than spectral methods that require FFT operations.
This technique promises higher accuracy and better stability than those of the conventional, explicit FD method. 
We also propose to replace the 4th-order FD operator based on Taylor's expansion in Fourier Finite Differences (FFD) 
with an 8th-order LFD operator to reduce dispersion, particularly in the TTI case.
Results from synthetic experiments illustrate the stability of the proposed methods in complicated velocity models. 
In TTI media, there is no coupling of qP-waves and qSv-waves by either method.
Both methods can be incorporated in seismic imaging by reverse-time migration to enhance its accuracy and stability.

%\clearpage

\bibliographystyle{seg}
\bibliography{ofd,SEP2}

%\bsp % ``This paper has been produced using the Blackwell
     %   Publishing GJI \LaTeXe\ class file.''

%\label{lastpage}



