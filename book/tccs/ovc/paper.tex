\published{Geophysics, 82, U25-U35, (2017)}

\title{Diffraction imaging and time-migration velocity analysis using oriented velocity continuation}

\author{Luke Decker, Dmitrii Merzlikin, and Sergey Fomel}

\lefthead{Decker et al.}
\righthead{Oriented velocity continuation}
\maketitle

\address{
Bureau of Economic Geology, \\
Jackson School of Geosciences \\
The University of Texas at Austin \\
University Station, Box X \\
Austin, TX 78713-8924}
\footer{TCCS-12}

\begin{abstract}
We perform seismic diffraction imaging and time-migration velocity analysis by
separating diffractions from specular reflections and decomposing them
into slope components. We image slope components using migration velocity extrapolation in time-space-slope coordinates. The
extrapolation is described by a convection-type partial differential
equation and implemented \old{efficiently}\new{in a highly parallel manner} in the Fourier domain. Synthetic
and field data experiments show that the proposed algorithms are able to
detect accurate time-migration velocities by measuring
the flatness of\new{ diffraction} events in \old{dip-angle}\new{slope} gathers for both single and multiple offset data.
\end{abstract}

\section{Introduction}

Seismic diffraction occurs when a seismic wave encounters \new{a heterogeneity without a clearly defined tangent plane, such as an edge or tip, and the reflection part of the ray theory breaks down \cite[]{klem}.} \old{a small heterogeneity and formerly parallel rays diverge }\old{\cite[]{klem}.} These divergent diffraction
rays have similar behavior to a subsurface secondary source located at the 
heterogeneity \cite[]{keller}\old{, therefore a}\new{.  A}nalyzing diffraction moveout behavior
in different domains can provide subsurface velocity information analogous 
to analyzing reflection moveout behavior from a surface source.

The fact that diffractions migrated with correct velocity collapse to points motivated
 \cite{GEO49-11-18691880} to propose the idea of separating diffractions from specular reflections and using diffraction focusing as a tool for velocity analysis.
Separation of diffraction events from seismic data is a necessary step
for velocity analysis because diffraction signals \new{are typically significantly weaker than those of reflections} \old{can be an order of magnitude weaker than those of reflections} \cite[]{klem}.\old{
\cite{sava05} adopted velocity analysis through diffraction focusing for wave-equation depth migration.} 
\cite{diffr} developed a constructive procedure for
diffraction separation based on plane-wave destruction and diffraction
focusing analysis based on velocity continuation and local
kurtosis. The procedure was extended to 3-D azimuthally-anisotropic
velocity analysis by \cite{will}. However, local kurtosis may not be
an optimal measure for diffraction focusing because it requires
smoothing or windowing in space, which reduces spatial \new{velocity }resolution \new{through the smoothing window parameters}.

A particularly convenient domain for separating diffractions and
reflections and for analyzing migration velocities is dip-angle
gathers (\new{\citealp{dahl2003,biondi2004}; }\citealp{landa08,reshef09,klokov12}). In the dip-angle domain,
specular reflections appear as hyperbolic events centered at the
reflector dip\old{, and diffractions appear flat when imaged at the
location of the diffractor with the correct velocity.} \new{ and bending upwards, even when over or under-migrated, and diffractions appear flat when imaged at the
location of the diffractor with the correct velocity \cite[]{reshef07}.} Measuring
flatness of diffraction events in dip-angle gathers, as opposed to
flatness of reflection and diffraction events in reflection-angle
gathers, provides an alternative constraint on seismic
velocity\new{ \cite[]{reshef09}}. Traditionally, dip-angle gathers are constructed with
Kirchhoff migration
\cite[]{Fomel.sep.100.sergey1,GEO66-06-18771894,GEO68-01-02320254,cheng11,koren11,bash,klokov3}.

In this paper, we adopt \new{an analogous method to }the dip-angle approach \new{used by \cite{reshef09} }to devise a
constructive \old{and efficient }\new{and highly parallel }procedure for estimating velocities in
time-domain processing using data decomposition in \new{slope}\old{dip} \cite[]{ghosh}
and velocity continuation in the mid\-point-time-slope domain. By
analogy with the ``oriented wave
equation'' \cite[]{SEG-2003-08930898}, we call this
approach \emph{oriented velocity continuation} (\emph{OVC}) and develop a fast 
spectral method for its implementation on common-offset data.\new{ This differs
from the methods devised by \cite{reshef09}, which utilize a separate
Kirchhoff-based angle prestack time or depth migration and calculation of travel time tables
for each tested migration velocity.  OVC uses a continuation
approach where a single migration is used to determine an initial
image in the mid\-point-time-slope domain to which a velocity dependent phase
shift is applied over the range of plausible migration velocities,
enabling OVC to test a greater number of velocities at a lower computational cost.}

Using a field-data experiment\new{,} we demonstrate the effectiveness of oriented
velocity continuation in zero-offset diffraction imaging and velocity analysis\new{,}
and using a synthetic model we \old{illustrate the theoretically higher
velocity resolution that can be achieved when including multiple offsets in the process}\new{observe that higher velocity resolution can be achieved when multiple offsets are included in the process}.




\section{Oriented velocity continuation}

Velocity continuation \cite[]{fomel2003velocity} is the imaginary
process of a continuous transformation of seismic time-migrated
images \old{in}\new{as they are propagated through different} migration \new{velocities}\old{velocity}. In the most general terms, the
kinematics of velocity continuation can be described by an equation of
the Hamilton-Jacobi type
\begin{equation}
\label{eq:jacobi}
\frac{\partial \tau}{\partial v} = F(v,\tau,\mathbf{x},\nabla \tau)\;,        
\end{equation}
where $\tau(\mathbf{x},v)$ is the location of a time-migrated
reflector\old{ imaged} with \new{time-domain coordinates $\mathbf{x}=(x_1,x_2,t)$ imaged with spatially constant }time-migration velocity $v$. The particular form
of function $F$ in equation~(\ref{eq:jacobi}) depends on the acquisition
geometry of the input data. For the case of common-offset 2D
velocity continuation for data with half-offset $h$, 
\begin{equation}
\label{eq:f2d}
F(v,t,x,p\old{,h})=v\,t\,p^2\ + \frac{h^2}{v^3t},
\end{equation}
and equation~(\ref{eq:jacobi}) corresponds to the characteristic
equation of the image propagation process which describes a propagation of the time-migrated image $I(t,x,v)$ in
velocity $v$ \cite[]{fomel2003velocity}. Time-domain imaging can be performed effectively by extrapolating
images in velocity and estimating velocity $v_m(t,x)$ of the best
image \cite[]{GEO52-05-06180643,GEO68-05-16621672,fomel14}.

As shown by \cite{SEG-2003-08930898}, it is possible to extend the
formulation of a wave propagation process from the usual
time-and-space coordinates to the phase space consisting of time,
space, and slope. Applying a similar approach to
equation~(\ref{eq:jacobi}), we\old{ can} first employ the Hamilton-Jacobi
theory \cite[]{courant,evans} to write the corresponding system of
ordinary differential equations for the characteristics (velocity
rays), as follows:
\begin{eqnarray}
\label{eq:dxdv}
\frac{d\mathbf{x}}{dv} & = & - \nabla_p F\;, \\
\label{eq:dpdv}
\frac{d\mathbf{p}}{dv} & = & \nabla_x F + \frac{\partial F}{\partial t}\,\mathbf{p}\;, \\
\label{eq:dtdv}
\frac{dt}{dv} & = & F - \nabla_p F \cdot \mathbf{p}\;, 
\end{eqnarray}
where $\mathbf{p}$ stands for $\nabla \tau$\new{, the gradient or slope of time-migrated wavefield energy}.

If the image $I(t,\mathbf{x},v)$ is decomposed in slope components $\widehat{I}(t,\mathbf{x},\mathbf{p},v)$ so that
\begin{equation}
\label{eq:txp}  
I(t,\mathbf{x},v) = \int \widehat{I}(t,\mathbf{x},\mathbf{p},v)\,d\mathbf{p}\;,
\end{equation}
we can then look for an equation that would adequately describe a continuous
transformation of $\widehat{I}$. To preserve the geometry of the
transformation, it is sufficient to require that $\widehat{I}$ transports
along the characteristics described by
equations~(\ref{eq:dxdv}-\ref{eq:dtdv}). Applying partial derivatives
and the chain rule, we arrive at the \old{Liouville }
equation\new{ analogous to the Liouville equation \cite[]{engquist}:} 
\begin{equation}
\label{eq:ovc}
\frac{\partial \widehat{I}}{\partial v} = \left(F - \nabla_p F \cdot \mathbf{p}\right)\,\frac{\partial \widehat{I}}{\partial t} - \nabla_p F \cdot \nabla_x  \widehat{I} + \left(\nabla_x F + \frac{\partial F}{\partial \tau}\,\mathbf{p}\right) \cdot \nabla_p \widehat{I}\;\old{.}\new{,}
\end{equation} 
Equation~(\ref{eq:ovc}) describes, in the most general form, the
process of \emph{oriented velocity continuation}, image propagation
in velocity in the coordinates of time-space-slope. It is a linear
first-order partial differential equation of convection type \old{-}\new{which} operat\new{es}\old{ing} in the phase space.

\subsection{Common-offset oriented velocity continuation}
To adopt the general theory described above to the case of
common-offset 2D oriented velocity continuation, we can\old{ simply} substitute
equation~(\ref{eq:f2d}) into~(\ref{eq:ovc}), arriving at the
equation
\begin{equation}
\label{eq:ovc2d}
\frac{\partial \widehat{I}}{\partial v} = \left(\frac{h^2}{v^3t} -\ v\,t\,p^2\ \right)\frac{\partial \widehat{I}}{\partial t} - 2\,v\,t\,p\,\frac{\partial \widehat{I}}{\partial x} + v\,p^3\,\frac{\partial \widehat{I}}{\partial p}\;.
\end{equation} 
which \old{is analogous to equation~(\ref{eq:vczo}) but }describes image
propagation in the time-space-slope coordinates rather than the usual
time-space coordinates. After this kind of extrapolation, regular
images can be reconstructed by stacking over offset and slope.

\old{Dip-angle}\new{Slope} gathers\new{, analogous to dip-angle gathers,} can be
extracted before stacking over slope by analyzing $\{t,p\}$ panels for different
image locations $x$ and velocities $v$. Measuring flatness of
diffraction events in \old{dip-angle}\new{these} gathers provides a means for
estimating migration velocity \cite[]{landa08,reshef09}.

For practical implementation, the formulation of oriented velocity
continuation can be simplified by employing a stretch from the regular
time coordinate to squared time
$\sigma=t^2$ \cite[]{GEO68-05-16621672}. According to this
transformation, the Hamilton-Jacobi equation~(\ref{eq:jacobi}) becomes
\begin{equation}
\label{eq:jacobi2}      
\frac{\partial \sigma}{\partial v} = \frac{v}{2}\,\left(\frac{\partial \sigma}{\partial x}\right)^2\ +\ 2\frac{h^2}{v^3}\ ,
\end{equation}
which leads to the simpler form of the oriented equation
\begin{equation}
\label{eq:ovcs}
\frac{\partial \widehat{I}}{\partial v} = \left(2\frac{h^2}{v^3}\ - \frac{v}{2}\,q^2\right)\,\frac{\partial \widehat{I}}{\partial \sigma} - v\,q\,\frac{\partial \widehat{I}}{\partial x}\;. 
\end{equation} 
where $q$ corresponds to $\frac{\partial \sigma}{\partial x}$, and the image is
constructed in $\{\sigma,x,q,h\}$ coordinates instead of $\{t,x,p,h\}$
coordinates. Applying the Fourier transform, we
can further transform equation~(\ref{eq:ovcs}) to 
\begin{equation}
\label{eq:ovcsf}
\frac{\partial \tilde{I}}{\partial v} = i\,\omega\,\left(\frac{v}{2}\,q^2\ - 2 \frac{h^2}{v^3}\right)\tilde{I}- i
\,v\,q\,k\,\tilde{I}\;, 
\end{equation} 
where $\tilde{I}(\omega,k,q,v,h)$ is the double Fourier transform of
$\widehat{I}(\sigma,x,q,v,h)$ in $\sigma$ and $x$.
Equation~(\ref{eq:ovcsf}) has the\old{ simple} analytical solution:
\begin{equation}
\label{eq:pshift}
\tilde{I}(\omega,k,q,v,h) = \tilde{I}(\omega,k,q,v_0,h)\,e^{i\,(\omega\,q^2/4+q\,k/2)\,(v^2-v_0^2)\ +\ i\omega h^2 (\frac{1}{v^2} - \frac{1}{v_0^2})}\;\new{,}
\end{equation} 
\new{w}\old{W}here $v_0$ is a constant non-zero initial migration velocity.

Stacking over offset \old{according to equation~(\ref{eq:offsetsum}) }provides a slope-decomposed formulation for oriented velocity continuation:

\begin{equation}
\label{eq:pshiftstps}
\tilde{I}(\omega,k,q,v) = \sum_{h}\tilde{I}(\omega,k,q,v_0,h)\,e^{i\,(\omega\,q^2/4+q\,k/2)\,(v^2-v_0^2)\ +\ i\omega h^2 (\frac{1}{v^2} - \frac{1}{v_0^2})}\;.
\end{equation}

This derivation suggests the following algorithm for time-domain imaging using common-offset 2D oriented velocity continuation: 
\begin{enumerate}
\item Start with initial time migration with a constant velocity $v_0$ to generate $I(t,x,v_0,h)$.
\item Apply vertical time stretch to transform from $t$ to $\sigma$.
\item Apply Fourier transform from $\sigma$ to $\omega$.
\item Perform slope decomposition (described in the next section) to generate $\hat{I}(\omega,x,q,v_0,h)$. Note that this operation is parallel in $\omega$ and $h$.
\item Apply Fourier transform from $x$ to $k$ to generate $\tilde{I}(\omega,k,q,v_0,h)$. Note that this operation is parallel in $q$ and $h$.
\item Apply the phase-shift filter from equation~(\ref{eq:pshift}) to generate $\tilde{I}(\omega,k,q,v,h)$ for multiple values of $v$. Note that this operation is data-intensive but parallel in $q$, $k$, and $h$.
\item Stack over offset to generate $\tilde{I}(\omega,k,q,v)$.
\item Apply inverse double Fourier transform to generate $\hat{I}(\sigma,x,q,v)$.
\item Apply inverse time stretch from $\sigma$ to $t$.
\item Stack over $q$ and extract the slice at time-migration velocity $v_m(t,x)$ to generate the final time-migrated image $I(t,x,v_m(t,x))$.
\end{enumerate}

In order to estimate the velocity~$v_m(t,x)$, we apply the workflow
described above to diffraction imaging and modify it as follows:
\begin{itemize}
\item Before Step 1, we separate reflections and diffractions in the common-offset data using local plane-wave destruction \cite[]{fomel1,diffr,decker}.
\item After Step 9, we analyze \old{dip-angle}\new{slope} gathers $\hat{I}(t,x,q,v)$ and\new{ automatically} pick the velocity $v_m(t,x)$ that corresponds to the maximum flatness (semblance) over $q$\new{ using the picking algorithm described by \cite{fomelvelanal}}. This approach follows the principle of flatness of diffraction events in \old{dip-angle}\new{slope} gathers \cite[]{landa08,reshef09,klokov12}.
\end{itemize}

\new{The computational cost associated with determining velocity using oriented velocity continuation is linear with the number of time samples, spatial samples, offsets, velocities, and slopes considered.  It is parallel in spatial samples, offsets, velocity, and slope.  The cost may then be considered as $O \left( \frac{N_t N_x N_h N_v N_p}{N_c} \right)$ where $N_c$ is the number of cores available.}

\section{Slope decomposition}
In order to perform the initial slope decomposition (Step 4 in the
algorithm above), we adopt the method of \cite{ghosh}. The idea of slope
decomposition was discussed previously by \cite {Ottolini.sep.37.143} and
implemented using the local-slant stack
transform \cite[]{ventosa}. The slope-decompo\-sition algorithm
suggested by \cite {ghosh} is based on the time-frequency
decomposition of \cite{liu2013}. Namely, at each frequency $\omega$,
we apply regularized non-stationary regression \cite[]{lpf} to
transform from space $x$ to space-slope $x$-$q$ domain. The
non-stationary regression amounts to finding complex coefficients
$A_n(\omega,x)$ in the decomposition
\begin{equation}
D(\omega,x) = \sum\limits_{n=1}^{N_p} D_n(\omega,x)\;,
\label{eq:d}
\end{equation}
where $D(\omega,x)$ is the image slice, and $D_n(\omega,x)$ is its slope component corresponding to slope $q_n$:
\begin{equation}
\label{eq:dn}
D_n(\omega,x) =  A_n(\omega,x)\,e^{i\,\omega\,x\,q_n}\;.
\end{equation}
Equation~(\ref{eq:d}) is the discrete analog of
equation~(\ref{eq:txp}). Similarly to the time-frequency
decomposition proposed by \cite{liu2013}, shaping
regularization is used to control the variability of
$A_n$ coefficients and to accelerate the algorithm. 

\section{Examples}
\subsection{Toy Model Example}
\inputdir{toy}

We first illustrate the concept of oriented velocity continuation using a simple toy model \cite[]{landa08,klokov3} with constant $1.0\ $\old{$[km/s]$}\new{$\frac{km}{s}$} velocity containing one dipping and one flat reflector and a single diffractor centered at 0.5 km (Figure~\ref{fig:data}).  Figure~\ref{fig:data2} shows data warped to squared time.  
\par
Warped data are decomposed into their constituent slope components and initially under-migrated
using $v_0=0.5\ \frac{km}{s}$. The initial slope decomposed image is shown in Figure~\ref{fig:ovc0}, which illustrates a slope gather
centered above the diffractor on the right panel, and a partial image containing energy with the slope of the top dipping reflector on the front panel.
The partial image contains energy of the top dipping reflector, which has the selected slope, diffraction energy with that slope, and
a small portion of the \old{flat bottom reflector's }energy\new{ from the flat bottom reflector}. Stacking over all constituent slopes provides an image
(top left panel of Figure~\ref{fig:vc}).  
\par
The slope decomposed initial migration from Figure~\ref{fig:ovc0} is propagated through a suite of plausible migration velocities.
We illustrate the initial migration and example velocities of $0.75\ $\old{$[km/s]$}\new{$\frac{km}{s}$}, $1.0\ $\old{$[km/s]$}\new{$\frac{km}{s}$} (the correct velocity), and $1.25\ $\old{$[km/s]$}\new{$\frac{km}{s}$}.
Slope gathers showing this process are shown in Figure~\ref{fig:cig0,cig1,cig2,cig3}. Stacking these propagated images over slope produces
the image in Figure~\ref{fig:vc}.
\par



\multiplot{2}{data,data2}{width=0.7\columnwidth}{Toy model data: (a) zero-offset synthetic data featuring two sloping reflectors and a diffractor centered at $0.5\ km$; (b) zero-offset data warped to squared time.}

\plot{ovc0}{width=0.7\columnwidth}{Slope decomposed initial migration for toy model data using $v_0=0.5 \frac{km}{s}$.  The side pane shows the slope decomposed data centered at $0.5\ km$, the diffractor, and the front pane shows a partial image for data containing energy with a slope of $2.998 \frac{km}{s^2}$, that of the top reflector. }

Examining the \old{initial migration's }slope gather \new{of the initial migration }in Figure~\ref{fig:cig0},
the three panels contain points of energy corresponding, from top to bottom,
to the top reflector, the diffractor, and the bottom reflector. 
\new{The energy of each reflector}\old{Each reflector's energy} is contained at the same lateral position in the three panels due to the constant slope of the reflector, although the \new{vertical position of the }top dipping reflector\old{'s vertical position} changes through the panels because \old{it}\new{the reflector} dips downward to the right.
\par
The diffraction has a hyperbolic moveout rather than a constant slope\new{, so energy appears at different slopes in}\old{.  This causes the diffraction energy's slope to change when viewed from}  different \new{slope }gathers, with zero slope in the gather centered over the diffractor at 0.5 km.  As we propagate data through velocity, this pattern holds: the reflection energy is stationary at its slope location for all gathers and diffraction energy has zero slope when viewed in the gather above the diffractor and non-zero slope for other gathers.
\par
\new{The initially migrated image is propagated to the higher time migration velocity of $0.75\ \frac{km}{s}$ using oriented velocity continuation,
and slope gathers are illustrated}\old{We propagate the data through OVC to a higher velocity of $0.75\ [km/s]$ and show slope gathers} in Figure~\ref{fig:cig1}.
Reflection energy now bends upward about the \old{reflection's }stationary point\new{ of each reflection} in ``smiles" that become
more accentuated with larger migration velocities. The diffraction event bends upward as well,
but this only holds for the current case of under-migration.  
\par
Figure~\ref{fig:cig2} shows the image propagated to the correct migration velocity.
Diffraction energy is planar in all three gathers and flat in the middle gather centered above the diffractor.
\new{Stacking the energy in each gather over slope collapses the flat diffraction energy in the central gather to a point at the location of the diffractor.}\old{ Only the flat diffraction energy in the center panel is stationary on stacking, collapsing to a point.}
\old{The right and left panels' s}\new{S}loping diffraction energy\new{ in the right and left panels} cancels out when summed over slope.
This flatness is essential to using oriented velocity continuation as a tool for determining the correct migration velocity.  
\par
When data are further propagated to over-migration with velocity of $1.25\ $\new{$\frac{km}{s}$}\old{$[km/s]$} in Figure~\ref{fig:cig3},
the diffraction event bows downward in a ``frown" juxtaposed against the upward bending reflection ``smiles".
\par
Stacking over slope provides images for these four velocities (Figure~\ref{fig:vc}).
In these images, the diffraction event incrementally evolves from having a hyperbolic downward character in the top under-migrated row, to collapsing to a point in the bottom left panel with the correct migration velocity, to bowing hyperbolically upward in the bottom right over-migrated image.
\par
The changing geometry of diffraction energy
in the gathers can be harnessed to determine the proper migration velocity \cite[]{landa08,reshef09}.
The correct migration velocity will be the one that maximizes the ``flatness" of
slope\new{ }\old{-}decomposed diffraction events, as measured by coherence or another appropriate metric.
Selecting $1.0\ $\new{$\frac{km}{s}$, the}\old{$[km/s]$} velocity that produces flat diffraction energy,
provides us with a properly migrated image (Figure~\ref{fig:mig}).
\par
To properly estimate migration velocity using diffraction flatness,
reflection events must first be filtered out from the diffraction data,
or else the contribution of reflection ``smiles" may dominate and bias the flatness measure.

\multiplot{4}{cig0,cig1,cig2,cig3}{width=0.45\columnwidth}{Slope\old{-decomposed angle gathers} gathers centered at $0.25\ km$ (left), $0.5\ km$ (center), and $0.75\ km$ (right) for migration velocities of: (a) initial under-migration with $0.5 \frac{km}{s}$; (b) under-migration with $0.75 \frac{km}{s}$; (c) the correct migration velocity of $1.0 \frac{km}{s}$; (d) over-migration with $1.25 \frac{km}{s}$.}

\plot{vc}{width=0.7\columnwidth}{Toy model data propagated through oriented velocity continuation for different migration velocities}





\plot{mig}{width=0.7\columnwidth}{Toy model image using $1.0 \frac{km}{s}$ migration velocity} 


\subsection{Synthetic Example}

To test oriented velocity continuation and its velocity resolution
we generate a synthetic dataset using a model with a constant velocity gradient beginning
with a $2.0\ $\new{$\frac{km}{s}$}\old{$[km/s]$} surface velocity.
Diffractors are created as reflectivity spikes within the model with random spatial and magnitude distributions.  Kirchhoff forward modeling is used to generate 24 offsets with a $50\ m$ interval.

Zero-offset data are shown in Figure~\ref{fig:a-zo}, and $1.0\ km$ common-offset data appear in Figure~\ref{fig:a-co}. A time shift between the data is noticeable. 

Both zero and common-offset data are warped to squared time, slope decomposed,
and migrated with a $2.0\ $\new{$\frac{km}{s}$}\old{$[km/s]$} initial velocity.
The initially migrated slope\old{-}\new{ }decomposed images are propagated
through a range of plausible migration velocities using oriented velocity continuation. 
Common-offset partial images are then stacked over offset for each continuation velocity.

\old{We perform velocity analysis by testing the semblance, or flatness, of diffraction events in common image
gathers over the range of velocities.
Because velocity does not vary laterally in our synthetic model,
we average the semblance across midpoints to generate semblance panels
for the zero-offset and common-offset cases (Figures~\ref{fig:a-semb-zo} and \ref{fig:a-semb-fo} respectively).
The common-offset semblance panel appears to have higher spatial and vertical resolution
than the zero-offset case.  This higher spatial resolution can be attributed to the improved
illumination of scattering objects with the full range of offsets.}

Figure~\ref{fig:a-cigs-zo,a-cigs-co} illustrates \old{common image}\new{slope} gathers for zero and $1.15\ km$
offsets generated for the image location $x = 2.32\ km$ with migration velocities $2.1,\ 3.0$, and $3.9\ $\new{$\frac{km}{s}$}\old{$[km/s]$}. $3.0\ $\new{$\frac{km}{s}$}\old{$[km/s]$} is the correct velocity for the diffractor at $1.4\ s$ located directly underneath the midpoint of this gather.
When a different velocity is used,
the \old{event's }shape \new{of the event }deviates from planar.
\new{We perform velocity analysis by testing the semblance, or flatness, of diffraction events
in slope gathers over the range of velocities.
Because velocity does not vary laterally in our synthetic model,
we average the semblance across midpoints to generate semblance panels
for the zero-offset and common-offset cases (Figures~\ref{fig:a-semb-zo} and \ref{fig:a-semb-fo} respectively).}

\new{As seen from the slope gathers (Figure~\ref{fig:a-cigs-zo,a-cigs-co})}, for the zero-offset case, there is a stationary
point corresponding to diffraction energy with zero slope which does not shift vertically under velocity
perturbations. For the $1.15\ km$ common-offset case, perturbing velocity changes the slope\old{-}\new{ }decomposed diffraction shape and shifts it vertically.
\old{Common image}\new{Slope} gathers with incorrect velocities (Figure~\ref{fig:a-cigs-co})
are time shifted with respect to those generated for the zero-offset case (Figure~\ref{fig:a-cigs-zo}). When the correct migration
velocity is used, horizontal common-offset diffraction energy appears at the same time as for the zero-offset case. 
The vertical shift of incorrectly migrated common-offset data leads to a sharper change in estimated flatness values while converging on the correct migration velocity and therefore improves velocity resolution.
\new{Therefore,} common-offset semblance panel appears to have higher spatial and vertical resolution than the zero-offset case.
This higher spatial resolution can be attributed to the improved illumination of scattering
objects with the full range of offsets.

Final images for zero-offset and stacked common-offset cases using migration velocities estimated from the semblance panels
are shown in Figure~\ref{fig:a-slice-zo,a-slice-fo}.
Differences between the two images are too small to easily detect in this example.
However, due to the higher velocity resolution\new{ visible in the semblance panel} resulting from the consideration of multiple offsets,\new{ Figure~\ref{fig:a-semb-fo}, }
we expect the stacked common-offset image to be better resolved and less prone to noise than zero-offset case when applied to field datasets. 

\inputdir{psovc}
\multiplot{2}{a-zo,a-co}{width=0.41\columnwidth}{Modeled data sections: (a) zero-offset; (b) $1.0\ km$ offset section.}

\multiplot{2}{a-semb-zo,a-semb-fo}{width=0.41\columnwidth}{Velocity scan semblance panels calculated for: (a) zero-offset; (b) all 24 offsets ($50\ m$ interval).}

\multiplot{2}{a-slice-zo,a-slice-fo}{width=0.41\columnwidth}
{Images: (a) zero-offset OVC (velocity from Figure~\ref{fig:a-semb-zo}); (b) stacked common-offset OVC (velocity from Figure~\ref{fig:a-semb-fo})}

\multiplot{2}{a-cigs-zo,a-cigs-co}{width=0.7\columnwidth}{\old{Dip-angle domain common image}\new{Slope} gathers: (a) zero-offset; (b) $1.15\ km$ offset}

\inputdir{nankai}
\new{\subsection{Field Data Example}}
\new{We demonstrate an application of zero-offset (post-stack) oriented velocity continuation on a deep water 2D line acquired to image
the Nankai Trough subduction zone. Data acquisition parameters as well as processing
results can be found in \cite{moore1990structure},
where the line is referred to as NT62-8.
Structural interpretation can be found in
\cite{moore1993character}. Here, we consider a fragment of 
the line (CMPs 900-1301) used previously by \cite{forel2005seismic}.}

\new{Conventional velocity analysis resolution suffers in this dataset from the limitations
imposed by the depth of a seabed in the area (average of $\approx 4.5\ km$)
and a relatively short $2\ km$ streamer length. For deep water datasets
diffractions may exhibit better illumination than reflections because 
diffraction aperture is not restricted to the recording array length, enabling them to provide a potentially more detailed velocity distribution. This behavior makes OVC migration velocity
analysis appealing.}

\new{The DMO stacked section considered in this study is shown in Figure~\ref{fig:slice}.
Diffractions are extracted via plane-wave destruction (Figure~\ref{fig:dif}),
warped to squared time, and decomposed into slope.
Figure~\ref{fig:tpx} shows slope decomposed data warped back to regular
time for ease of comparison with slope decomposed images appearing later.
Next, we take the decomposed data through oriented velocity
continuation over a range of sixty constant migration velocities beginning
with $v_0 = 1.4\ \frac{km}{s}$ using a $20\ \frac{m}{s}$ step. 
Diffraction events bend upward in the slope gather centered above $x=4100.06\ m$
with the minimum tested migration velocity (Figure~\ref{fig:txp14-slice-center-s}), indicating under-migration.
Diffraction events in the slope gather centered above the same location with the maximum
tested migration velocity  (Figure~\ref{fig:txp25-slice-center-s}) bend downward, indicating over-migration.}


\new{Gather semblance is calculated for each continuation velocity, and migration velocity
is automatically picked by attempting to maximize semblance for plausible velocity values
at each CMP location.
Semblance panels with superimposed picks are shown
in Figure~\ref{fig:g-picking}.
Anomalies corresponding to higher velocities than
the picked trend may correspond to reflections
with high curvature, like the one located between $x=3000$ and $4000\ m$ between $t=6.0$ and $6.5\ s$ in Figures~\ref{fig:slice} and \ref{fig:dif}. Highly curved reflections
have a similar behavior to diffractions in response to migration velocity perturbation \cite[]{sava05}, but focus at a higher velocity than the correct one.
Intersection of over-migrated reflection and diffraction
tails from the rugose seabed, some of which are out of plane, leads to
diffraction-like events, another cause of false semblance highs. These are visible in the
three semblance panels of Figure~\ref{fig:g-picking} above $~6.4\ s$.  Low velocity semblance
anomalies corresponding to the flattening of out of plane diffractions are also visible in the
middle interval of the semblance panels, particularly near $t \approx 6.8\ s$ in the central and
right panels centered above $x=5000$ and $6500\ m$.}

\new{Combining the semblance velocity picks from each CMP provides a time-migration velocity field, shown in Figure~\ref{fig:vpick-semb}.  As noted above, several anomalously low velocity zones exist in the picked field, primarily between $t=6.5$ and $7\ s$ where the attempted flattening of out of plane diffractions leads to a low picked velocity.}

\new{Gathers corresponding to the picked velocity are selected. Examining a slope gather
from $x=4100.06\ m$ generated using the picked migration velocity (Figure~\ref{fig:txp-slice-center-s}), diffraction events now appear flat, particularly the one located near $t=6.4\ s$ , indicating that they have been correctly migrated.
%Some events between $t=6.2$ and $6.4\ s$ are planar and slope downward to the right because they are centered below a different midpoint.
}

\new{Stacking gathers generated from the the picked velocity over slope provides the
diffraction image in Figure~\ref{fig:vc-slice-semb}.  We apply oriented velocity continuation
to the DMO stacked data from Figure~\ref{fig:slice} and stack over gathers
selected with the appropriate velocity to generate the image of reflections and diffractions in
Figure~\ref{fig:vc-fw-slice-semb}.  Both images highlight fault surfaces.
Finer discontinuities, such as those associated with the rough surface of the subducting plate crust, located near $t \approx 7.5\ s$ \cite[]{moore1993character}, are more prominent on the diffraction image and tend to be well focused, supporting the accuracy of the picked velocity.}

%\new{\multiplot{3}{slice,dif,tpx}{width=0.4\columnwidth}{Nankai field data example: (a) DMO stacked section; (b) separated diffractions; (c) slope decomposition of diffraction data.}}
\new{\plot{slice}{width=0.8\columnwidth}{Nankai DMO stacked section.}}
\new{\plot{dif}{width=0.8\columnwidth}{Nankai separated diffractions.}}
\new{\plot{tpx}{width=0.8\columnwidth}{Slope decomposition of Nankai diffraction data}}
\new{\plot{txp-migrated}{width=0.8\columnwidth}{Slope decomposition of Nankai diffraction image}}
\new{\multiplot{3}{txp14-slice-center-s,txp25-slice-center-s,txp-slice-center-s}{width=0.4\columnwidth}{Slope gathers centered above $x=4100.06\ m$ migrated with: (a) $1.4\ \frac{km}{s}$, (b) $2.5\ \frac{km}{s}$ and (c) picked migration velocity.}}
\new{\plot{g-picking}{width=0.7\columnwidth}{Velocity scan semblance panels with superimposed picks from left to right for CMPs at $2700\ m$, $5000\ m$, and $6500\ m$.}}
%\plot{cigs}{width=0.7\columnwidth}{Dip-angle domain common image gathers migrated with: (a)
%1.5 km/s, (b) 2.5 km/s and (c) picked (Figure~\ref{fig:vpick-semb}) migration velocity.}

\new{\plot{vpick-semb}{width=0.8\columnwidth}{Velocity picked from slope-gather flattening}}  
\new{\plot{vc-slice-semb}{width=0.8\columnwidth}{Diffraction image generated with the velocity from the Figure~\ref{fig:vpick-semb}.} }
\new{\plot{vc-fw-slice-semb}{width=0.8\columnwidth}{Conventional image generated with the velocity from the Figure~\ref{fig:vpick-semb}.}}

\section{Conclusions}
We have developed and demonstrated \old{an efficient}\new{a highly parallel} and constructive
procedure for time-domain velocity estimation.  The method operates
by decomposing data by slope and propagating slope components in velocity in the
midpoint-time-slope domain.  Semblance in \old{dip-angle}\new{slope} gathers is used as
a measure for selecting velocities that correspond to correctly
migrated flat diffraction events, which applies
even for single-offset data.  This semblance measure \old{achieves}\new{is observed to achieve} higher resolution when multiple offsets are considered in the oriented velocity continuation process. 
If data with multiple offsets are available, oriented velocity continuation could
be used to better constrain the velocity model when used in conjunction with traditional
reflection moveout analysis.  Chosen velocities can be used to generate both diffraction and reflection images. 
The powerful ability of oriented velocity continuation to operate with only zero-offset
data enables accurate migration velocity analysis in situations where only limited-offset data are available. % as with ground penetrating radar or p-cable data.
\par
Flatness of slope-decomposed diffraction events is more responsive to
velocity perturbation than diffraction focusing, because it does not
require smoothing or windowing in space. Therefore, the proposed
method has the potential for diffraction velocity estimation with
superior resolution when compared to methods
based on diffraction focusing.

\new{Oriented Velocity Continuation is formulated as a type of time-migration, and is thus subject to constraints relating to image distortion from horizontal velocity changes in the subsurface.  The presence of strong lateral velocity variations may alter diffraction moveout, making event slope change with azimuth. In such a case, OVC would be unable to completely flatten the diffraction signal in slope gathers and locally determine the correct migration velocity.}
\par 
This method can be extended to three dimensions using data
decomposition by \old{inline and crossline slopes, which}\new{azimuth and inclination for each image point.  } \new{Operating on three dimensional data }should improve velocity resolution \old{because three dimensional data overcomes}\new{by overcoming} out of plane artifacts in the seismic image. Although
\old{such an} extension\new{ to 3D} adds \old{an}\new{the} expense of additional \new{spatial and slope }dimensions, \new{the Fourier-domain computation would also be parallel in these new dimensions, making the operation}\old{the high
parallelism of Fourier-domain operations can make it} feasible in practice\new{ using computer clusters}.

\section{Acknowledgments}
We thank KAUST, Shell, and sponsors of TCCS for financial support of this work\new{ and Evgeny Landa
for inspiring discussions}.

%\onecolumn

\bibliographystyle{seg}
\bibliography{SEG,ovc}
