\published{Geophysics, 78, no. 4, S211-S219, (2013)}

\title{Kirchhoff migration using eikonal-based computation of traveltime source-derivatives}

\author{Siwei Li and Sergey Fomel}

%\ms{GEO-2012}

\address{Bureau of Economic Geology \\
John A. and Katherine G. Jackson School of Geosciences \\
The University of Texas at Austin \\
University Station, Box X \\
Austin, TX 78713-8924}

\lefthead{Li \& Fomel}
\righthead{Kirchhoff with Traveltime Source-derivative}

\maketitle

\begin{abstract}
The computational efficiency of Kirchhoff-type migration can be enhanced by employing accurate 
traveltime interpolation algorithms. We address the problem of interpolating between a sparse source 
sampling by using the derivative of traveltime with respect to the source location. We adopt a 
first-order partial differential equation that originates from differentiating the eikonal equation 
to compute the traveltime source-derivatives efficiently and conveniently. Unlike methods that rely 
on finite-difference estimations, the accuracy of the eikonal-based derivative does not depend on 
input source sampling. For smooth velocity models, the first-order traveltime source-derivatives 
enable a cubic Hermite traveltime interpolation that takes into consideration the curvatures of 
local wave-fronts and can be straight-forwardly incorporated into Kirchhoff anti-aliasing schemes. 
We provide an implementation of the proposed method to first-arrival traveltimes by modifying the 
fast-marching eikonal solver. Several simple synthetic models and a semi-recursive Kirchhoff migration 
of the Marmousi model demonstrate the applicability of the proposed method.
\end{abstract}

\section{Introduction}

Over the years, there have been significant efforts and progress in traveltime computations. The 
quality of traveltimes has a direct influence on Kirchhoff-type migrations since it determines the 
kinematic behaviors of the imaged wavefields. One can use either ray-tracing approaches or 
finite-difference solutions of the eikonal equation. The first option naturally handles multi-arrivals 
and can be extended to other wavefield approximations, such as Gaussian beams \cite[]{hill1,hill2,albertin,gray}, 
but is at the same time usually subject to the necessities of ray-coordinate and migration-grid mapping 
and irregular interpolation between rays in the presence of shadow zones in complex velocity media 
\cite[]{sava}. Two popular methods from the second option are the \textit{fast-marching method} (FMM) 
\cite[]{sethian1,sethian2} and the \textit{fast-sweeping method} (FSM) \cite[]{zhao}. They both rely 
on an ordered update to recover the causality behind expanding wave-fronts in a general medium, and 
are thus limited to first-arrival computations. Several works attempt to overcome the single-arrival 
drawback of the finite-difference eikonal solvers, for example multi-phase computation \cite[]{engquist}, 
phase-space escape equations \cite[]{fomel1}, and slowness marching \cite[]{symes}.

In practice, traveltime tables can be pre-computed on coarse grids and saved on disk, then serve as 
a dictionary when read by Kirchhoff migration algorithms. It is common to carry out a certain interpolation 
in this process in order to satisfy the needs of depth migration for fine-gridded traveltime tables 
at a large number of source locations \cite[]{mendes,vanelle,alkhalifah5}. Kirchhoff migrations with 
traveltime tables computed on the fly face the same issue. During the traveltime computation stage, 
accuracy requirements from eikonal solvers may lead to a fine model sampling. Combined with a large 
survey, traveltime computation for each shot can be costly. Because all traveltime computations handle 
one shot at a time, the overall cost increases linearly with the number of sources. Moreover, we need 
to store a large amount of traveltimes out of a dense source sampling. Therefore a sparse source 
sampling is preferred. In this paper, we try to address the problem of traveltime table interpolation 
between sparse source samples. The traveltime table estimated with simple nearest-neighbor or linear 
interpolation could not provide satisfying accuracy unless the velocity model has small variations. 
One possible improvement is to include derivatives in interpolation. During ray tracing, traveltime 
source-derivatives are directly connected to the slowness vector at the source and stay constant along 
individual rays, thus could be outputted as a by-product of traveltimes. For finite-difference eikonal 
solvers, such a convenience is not easily available. In these cases, we would like to avoid an extra 
differentiation on traveltime tables along the source dimension to compute such derivatives \cite[]{vanelle}, 
because its accuracy in turn relies on a dense source sampling and induces additional computations. 
\cite{alkhalifah3} derived an equation for the traveltime perturbation with respect to the source 
location changes. The governing equation is a first-order partial differential equation (PDE) that 
describes traveltime source-derivatives in a relative coordinate moving along with the source. In 
this paper, we show that the traveltime source-derivative desired by interpolation is related to this 
relative-coordinate quantity by a simple subtraction with the slowness vector. Unlike a finite-difference 
approach, traveltime source-derivatives computed by the PDE method are source-sampling independent. 
The extra costs are rather inexpensive. In this paper, we apply this method to Kirchhoff migration 
with first-arrival traveltimes computed by the FMM eikonal solver.

The paper is organized as follows. In the first section, we review the theory and implementation of 
the eikonal-based traveltime source-derivatives. Next, we use both simple and complex synthetic models 
to demonstrate the accuracy of a cubic Hermite traveltime table interpolation using the source-derivatives, 
and show effects of incorporating such an interpolation into Kirchhoff migration. We focus mainly on 
the kinematics in these experiments by neglecting possible true-amplitude weights in Kirchhoff migration. 
Finally, we discuss limitations and possible extensions of the proposed method. 

\section{Theory and Implementation}

We consider the isotropic eikonal equation:
\begin{equation}
\label{eq:eiko}
\nabla T (\mathbf{x}) \cdot \nabla T (\mathbf{x}) = 
\frac{1}{v^2 (\mathbf{x})} \equiv W (\mathbf{x})\;,
\end{equation}
where $\mathbf{x}$ is a point in space, $T(\mathbf{x})$ is the traveltime and $v(\mathbf{x})$ is the 
velocity. For 2D models, $\mathbf{x}$ is a vector containing the depth and the inline position. 
For 3D models, $\mathbf{x}$ also includes the crossline position. For conciseness, we define $W(\mathbf{x})$ 
as slowness-squared. Equation \ref{eq:eiko} can be derived by inserting the ray-theory series 
into the wave-equation and setting the coefficient of the leading-order term to zero \cite[]{chapman}. 
We are interested in particular in point-source solutions of the eikonal equation, i.e. with the 
initial condition $T(\mathbf{x_s}) = 0$ where $\mathbf{x_s}$ denotes the source location. 

\subsection{Traveltime Source-derivative}

The point-source traveltime $T(\mathbf{x})$ clearly depends on the source location $\mathbf{x_s}$. To 
explicitly show such a dependency in the eikonal equation, we define a relative coordinate $\mathbf{q}$ 
as
\begin{equation}
\label{eq:relative1}
\mathbf{q = x - x_s}\;,
\end{equation}
and use $\hat{T}(\mathbf{q};\mathbf{x_s})$ to denote traveltime in the relative coordinates. After 
inserting this definition into equation \ref{eq:eiko}, we obtain
\begin{equation}
\label{eq:insert}
\nabla_{\mathbf{q}} \hat{T} \cdot \nabla_{\mathbf{q}} \hat{T} = W (\mathbf{q+x_s})\;.
\end{equation}
Here the differentiation $\nabla_{\mathbf{q}}$ \new{stands for gradient operator}\old{ is taken} in 
the relative coordinate $\mathbf{q}$ \new{and is taken }with a fixed source location $\mathbf{x_s}$. 
\new{In 3D, if $\mathbf{q} = (q_1,q_2,q_3)$ and denoting $\mathbf{e}_i$ with $i = \{1,2,3\}$ to be 
the unit vector in depth, inline and crossline directions, respectively, then}
\begin{eqnarray}
\nabla_{\mathbf{q}} \equiv 
\frac{\partial}{\partial q_1} \mathbf{e}_1 +
\frac{\partial}{\partial q_2} \mathbf{e}_2 +
\frac{\partial}{\partial q_3} \mathbf{e}_3\;.
\end{eqnarray}
Since we are interested in the traveltime derivative with respect to the source, i.e. 
$\partial T / \partial \mathbf{x_s}$, we take directional derivative 
$\partial / \partial \mathbf{x_s}$ to $\hat{T}(\mathbf{q};\mathbf{x_s})$ and apply the chain-rule 
according to equation \ref{eq:relative1}:
\begin{equation}
\label{eq:chain}
\frac{\partial T}{\partial \mathbf{x_s}} \equiv \frac{\partial \hat{T}}{\partial \mathbf{x_s}} 
= \frac{\partial \hat{T}}{\partial \mathbf{x}} \frac{\partial \mathbf{x}}{\partial \mathbf{x_s}} 
+ \frac{\partial \hat{T}}{\partial \mathbf{q}} \frac{\partial \mathbf{q}}{\partial \mathbf{x_s}} 
= \frac{\partial \hat{T}}{\partial \mathbf{x}} - \frac{\partial \hat{T}}{\partial \mathbf{q}}\;.
\end{equation}
Equation \ref{eq:chain} results in a vector that contains the traveltime source-derivatives in 
depth, inline and crossline directions. \new{In accordance with $\partial / \partial \mathbf{x_s}$, 
$\partial / \partial \mathbf{x}$ and $\partial / \partial \mathbf{q}$ are also directional 
derivatives. }All numerical examples in this paper are based on a typical 2D acquisition, where we 
assume a constant source depth and thus only the inline traveltime source-derivative is of interest. 
The quantity $\partial \hat{T} / \partial \mathbf{q}$ coincides with the slowness vector of the 
ray that originates from $\mathbf{x_s}$. For a finite-difference eikonal solver such as FMM and 
FSM, it is usually estimated by an upwind scheme during traveltime computations at each grid point 
and thus can be easily extracted. Applying $\partial / \partial \mathbf{x}$ to both sides of 
equation \ref{eq:insert}, we find 
\begin{equation}
\label{eq:part}
\nabla_{\mathbf{q}} \hat{T} \cdot 
\nabla_{\mathbf{q}} \frac{\partial \hat{T}}{\partial \mathbf{x}} 
= \frac{1}{2} \frac{\partial W}{\partial \mathbf{x}}\;.
\end{equation}
Equation \ref{eq:part} has the form of the linearized eikonal equation \cite[]{aldridge} and was 
previously derived, in a slightly different notation, by \cite{alkhalifah3}. It implies that 
$\partial \hat{T} / \partial \mathbf{x}$, as needed by equation \ref{eq:chain}, can be determined 
along the characteristics of $\hat{T}$. Since the right-hand side contains a slowness-squared 
derivative, the velocity model must be differentiable, as is usually required by traveltime 
computations. The derivation also indicates that the accuracy of an eikonal-based traveltime 
source-derivative is source-sampling independent but model-sampling dependent, as from equations 
\ref{eq:chain} and \ref{eq:part} $\partial / \partial \mathbf{x_s}$ relies on 
$\hat{T}$, $\partial / \partial \mathbf{q}$ and $\partial / \partial \mathbf{x}$. The accuracy 
from a direct finite-difference estimation on $\partial / \partial \mathbf{x_s}$, in comparison, 
is both source- and model-sampling dependent.

Continuing applying differentiation and the chain-rule to equation \ref{eq:chain} will 
result in higher-order traveltime source-derivatives. For example, the second-order 
derivative reads:
\begin{eqnarray}
\frac{\partial^2 T}{\partial \mathbf{x_s^2}} \equiv 
\frac{\partial^2 \hat{T}}{\partial \mathbf{x_s^2}} & = & 
\frac{\partial}{\partial \mathbf{x}} \frac{\partial \hat{T}}{\partial \mathbf{x}} 
\cdot \frac{\partial \mathbf{x}}{\partial \mathbf{x_s}} + 
\frac{\partial}{\partial \mathbf{q}} \frac{\partial \hat{T}}{\partial \mathbf{x}} 
\cdot \frac{\partial \mathbf{q}}{\partial \mathbf{x_s}} - 
\frac{\partial}{\partial \mathbf{x}} \frac{\partial \hat{T}}{\partial \mathbf{q}} 
\cdot \frac{\partial \mathbf{x}}{\partial \mathbf{x_s}} - 
\frac{\partial}{\partial \mathbf{q}} \frac{\partial \hat{T}}{\partial \mathbf{q}} 
\cdot \frac{\partial \mathbf{q}}{\partial \mathbf{x_s}} \nonumber \\
& = & \frac{\partial^2 \hat{T}}{\partial \mathbf{x}^2} 
- 2 \frac{\partial^2 \hat{T}}{\partial \mathbf{x} \partial \mathbf{q}} 
+ \frac{\partial^2 \hat{T}}{\partial \mathbf{q}^2}\;.
\end{eqnarray}
Further, differentiating equation \ref{eq:part} once more by $\mathbf{x}$ provides 
\begin{eqnarray}
\nabla_{\mathbf{q}} \frac{\partial \hat{T}}{\partial \mathbf{x}} \cdot 
\nabla_{\mathbf{q}} \frac{\partial \hat{T}}{\partial \mathbf{x}} 
+ \nabla_{\mathbf{q}} \hat{T} \cdot 
\nabla_{\mathbf{q}} \frac{\partial^2 \hat{T}}{\partial \mathbf{x}^2} 
= \frac{1}{2} \frac{\partial^2 W}{\partial \mathbf{x}^2}\;.
\end{eqnarray}
It is easy to verify that any order of the traveltime source-derivative will require the 
corresponding order of the slowness-squared derivative. An approximation based on Taylor 
expansions of the traveltime around a fixed source location can make use of these derivatives. 
For example, \cite{ursin} and \cite{bortfeld} introduced parabolic and hyperbolic traveltime 
approximations with the first- and second-order derivatives. Notice that the need for slowness-squared 
derivatives may cause instability unless the velocity model is sufficiently smooth. \cite{alkhalifah3} 
also proved the following relationship between $\partial W/\partial \mathbf{x}$ and 
$\partial \hat{T}/\partial \mathbf{q}$: 
\begin{eqnarray}
\nabla_{\mathbf{q}} \hat{T} \cdot 
\nabla_{\mathbf{q}} \frac{\partial \hat{T}}{\partial \mathbf{(q+x_s)}} 
= \frac{1}{2} \frac{\partial W}{\partial \mathbf{x}}\;,
\end{eqnarray}
which implies that the traveltime source-derivative can be computed from the given traveltime 
tables only. However, the velocity smoothness is still implicitly assumed as the second-order spatial 
derivatives of traveltimes appear in the equation. For this reason, we restrict our current implementation 
to the first-order derivative only.

In a ray-tracing eikonal solver, $\partial T / \partial \mathbf{x_s}$ is the slowness vector 
of a particular ray at $\mathbf{x_s}$ and holds constant along the trajectory. As it may also require 
irregular coordinate mappings, one may use the same strategy as for the traveltime tables. In this way, 
there is no necessity for any additional effort. On the other hand, equations \ref{eq:chain} and 
\ref{eq:part} and their second-order extensions provide important attributes for use in Gaussian beams, 
which are commonly calculated by the dynamic ray tracing \cite[]{cerveny}. They might be alternatively 
estimated by the eikonal-based source-derivative formulas but with the traveltime tables from a 
finite-difference eikonal solver. However, this application is beyond the scope of this paper. In the 
following sections, we consider only the source-derivative estimation from traveltimes computed by a 
finite-difference eikonal solver.

\subsection{Numerical Implementation}

Equation \ref{eq:part} is a linear first-order PDE suitable for upwind numerical methods 
\cite[]{franklin}. Since it does not change the non-linear nature of the eikonal equation, 
the resulting traveltime source-derivative can be related to any branch of multi-arrivals, 
if one supplies the corresponding traveltime in $\hat{T}$. The source-derivatives can be 
computed either along with traveltimes or separately. In Appendix A, we describe a first-arrival 
implementation based on a modification of FMM \cite[]{sethian1}.

The first-order traveltime source-derivative enables a cubic Hermite interpolation 
\cite[]{press}. Geometrically, such an interpolation is valid only when the selected 
wave-front in the interpolation interval is smooth and continuous. For a 2D model and a 
source interpolation along the inline direction only, the Hermite interpolation reads: 
\begin{equation}
\label{eq:Hermite}
\begin{array}{lcl}
T (z,x; z_s,x_s + \alpha \Delta x_s) 
& = & (2\alpha^3-3\alpha^2+1)\,T (z,x; z_s,x_s) \\
& + & (\alpha^3-2\alpha^2+\alpha)\,\frac{\partial T}{\partial x_s} (z,x; z_s,x_s) \\
& + & (-2\alpha^3+3\alpha^2)\,T (z,x; z_s,x_s + \Delta x_s) \\
& + & (\alpha^3-\alpha^2)\,\frac{\partial T}{\partial x_s} (z,x; z_s,x_s + \Delta x_s)\;,
\end{array}
\end{equation}
where $\alpha \in [0,1]$ controls the source position to be interpolated between known 
values at $(z_s,x_s)$ and $(z_s,x_s+\Delta x_s)$. For comparison, the linear interpolation 
can be represented by:
\begin{equation}
\label{eq:linear}
T (z,x; z_s,x_s + \alpha \Delta x_s) = 
(1-\alpha)\,T (z,x; z_s,x_s) + \alpha\,T (z,x; z_s,x_s + \Delta x_s)\;.
\end{equation}
The linear interpolation fixes the subsurface image point $(z,x)$. A possible 
improvement is to instead fix the vector that links the source with the image, such 
that on the right-hand side the traveltimes are taken at shifted image locations:
\begin{equation}
\label{eq:shift}
\begin{array}{lcl}
T (z,x; z_s,x_s + \alpha \Delta x_s) 
& = & (1-\alpha)\,T (z,x- \alpha \Delta x_s; z_s,x_s) \\
& + & \alpha\,T (z,x + (1-\alpha) \Delta x_s; z_s,x_s + \Delta x_s)\;.
\end{array}
\end{equation}
We will refer to scheme \ref{eq:shift} as shift interpolation. According to our definition 
of the relative coordinate $\mathbf{q}$ in equation \ref{eq:relative1}, shift interpolation 
amounts to a linear interpolation in $\hat{T} (\mathbf{q}; \mathbf{x_s})$. It is easy to verify 
that, for a constant-velocity medium, both Hermite and shift interpolations are accurate, 
while the linear interpolation is not. However, the accuracy of shift interpolation deteriorates 
with increasing velocity variations, as it assumes that the wave-front remains invariant in 
the relative coordinate. Equations \ref{eq:Hermite}-\ref{eq:shift} can be generalized to 3D by 
cascading the inline and crossline interpolations (for example equation \ref{eq:linear} in 3D 
case becomes bilinear interpolation). The interpolated source does not need to lie collinear 
with source samples.

The derivatives themselves can also be directly used for Kirchhoff anti-aliasing 
\cite[]{lumley,abma,fomel2}. Equations \ref{eq:Hermite}, \ref{eq:linear} 
and \ref{eq:shift} give rise to their corresponding source-derivative interpolations 
after applying the following chain-rule to both sides: 
\begin{equation}
\label{eq:dinterp}
\frac{\partial}{\partial (x_s + \alpha \Delta x_s)} = 
\frac{\partial}{\partial \alpha} 
\frac{\partial \alpha}{\partial (x_s + \alpha \Delta x_s)} = 
\frac{1}{\Delta x_s} \frac{\partial}{\partial \alpha}\;.
\end{equation}
The anti-aliasing application is summarized in Appendix B.

\section{Numerical Examples}

\subsection{Constant-velocity-gradient Model}
\inputdir{check}

In a 2D medium of linearly changing velocities, $v(z,x) = v_0 + a x + b z$ where x is the 
lateral position and z is the depth, the traveltimes and source-derivatives have analytical 
solutions \cite[]{slotnick}. Figure~\ref{fig:model} shows the model used in our numerical test 
and the analytical source-derivative for a source located at $(0,0)$ km. The domain is of 
size 4km $\times$ 4km with grid spacing $0.01$ km in both directions. We solve for the traveltime 
tables at five sources of uniform spacing $1$ km along the top domain boundary by FMM and 
their associated source-derivatives using the method described in Appendix A. Figure~\ref{fig:diff} 
compares the errors in computed source-derivative between the proposed approach and a centered 
second-order finite-difference estimation for the same source shown in Figure~\ref{fig:model}. 
The proposed method is sufficiently accurate except for the small region around the source. This 
is due to the source singularity of the eikonal equation and can be improved by adaptive or 
high-order upwind finite-difference methods \cite[]{qian} or by factoring the singularity 
\cite[]{fomel4}. Since we are aiming at using the interpolated traveltime tables for migration 
purposes and the reflection energy around the sources is usually low, these errors in current 
implementation can be neglected. In Figure~\ref{fig:ierror}, we interpolate the traveltime 
table for a source at location $(0,0.25)$ km from the nearby source samples at $(0,0)$ km and 
$(0,1)$ km by the cubic Hermite, linear and shift interpolations. We use the eikonal-based 
source-derivative in the cubic Hermite interpolation. The shift interpolation is not applicable 
for some $\mathbf{q}$ and $\mathbf{x_s}$ if $\mathbf{x = q+x_s}$ is beyond the computational 
domain. In these regions, we use a linear interpolation to fill the traveltime table. As expected, 
the cubic Hermite interpolation achieves the best result, while its misfits near the source 
are related to the errors in source-derivatives. The shift interpolation performs generally 
better than the linear interpolation, especially in the regions close to the source where the 
wave-fronts are simple.

\plot{model}{width=0.95\textwidth}{(Left) a constant-velocity-gradient 
model $v(z,x) = 2 + 0.5 x$ km/s and (right) its analytical traveltime 
source-derivative for a source at origin $\mathbf{x_s} = (0,0)$ km.}
\plot{diff}{width=0.95\textwidth}{Comparison of error in computed 
source-derivative by (left) the proposed method and (right) a centered 
second-order finite-difference estimation based on traveltime tables. 
The maximum absolute errors are $0.15$ s/km and $0.56$ s/km, respectively.}
\plot{ierror}{width=0.95\textwidth}{Traveltime interpolation error 
of three different schemes: (top left) the analytical traveltime of 
a source at location $(0,0.25)$ km; (top right) error of the cubic 
Hermite interpolation; (bottom left) error of the linear interpolation; 
(bottom right) error of the shift interpolation. Using derivatives 
in interpolation enables a significantly higher accuracy. The $l2$ 
norm of the error are $1.5$ s, $9.2$ s and $6.0$ s respectively.}

The difference between a cubic Hermite interpolation and a linear or shift one is in the usage of 
source-derivatives. In this regard, one may think of supplying the finite-difference estimated 
derivatives to the interpolation. Indeed, a refined source sampling and higher-order differentiation 
may lead to more accurate derivatives. However the additional computation is considerable. For the 
same model in Figure~\ref{fig:model}, we carry out both a source sampling refinement experiment and 
a model grid spacing refinement experiment. The results are shown in Figures~\ref{fig:sfddiff} and 
\ref{fig:gfddiff}. Both figures are plotted for the traveltime at subsurface location $(1.5,-0.5)$ km 
for the source at location $(0,0)$ km. Although the curves vary for different locations, the source 
sampling refinement experiment suggests the general need for approximately three times finer 
source-sampling than that of Figure~\ref{fig:diff} to achieve the same level of accuracy.

\plot{sfddiff}{width=0.95\textwidth}{Source-sampling refinement 
experiment. The plot shows, at a fixed model grid sampling of 
$0.01$ km and increasing source sampling, the error in source-derivative 
estimated by a first-order finite-difference (solid) and a centered 
second-order finite-difference scheme (dotted) decrease. The 
horizontal axis is the number of sources and the source sampling 
is uniform. The vertical axis is the natural logarithm of the 
absolute error. The flat line (dash) is from the proposed eikonal-based 
method and is source-sampling independent.}
\plot{gfddiff}{width=0.95\textwidth}{Gird-spacing refinement 
experiment. The plot shows, at a fixed source sampling of $1$ km 
and increasing model grid sampling, the error in source-derivative 
estimated by the proposed eikonal-based method decreases. Meanwhile, 
the errors of both first- and second-order finite-difference estimations 
do not improve noticeably. The horizontal axis is the number of grid 
points in both directions and the grid sampling is uniform. See 
Figure~\ref{fig:sfddiff} for descriptions of the vertical axis and 
the lines.}

\inputdir{migration}

Kirchhoff migration can use traveltime source-derivatives in two ways: for traveltime interpolation when 
the source and receiver of a trace does not lie on the source grid of pre-computed traveltime tables, 
and for anti-aliasing. Figure~\ref{fig:modl} shows a synthetic model of constant-velocity-gradient with 
five dome-shaped reflectors. The model has a $0.01$ km grid spacing in both directions. We solve for 
traveltimes and source-derivatives by the modified FMM introduced in Appendix A at 21 sparse shots of 
uniform spacing 0.5 km, and migrate synthetic zero-offset data. The interpolation of source-derivative 
for the anti-aliasing purpose follows the method described in Appendix B. 48 interpolations are carried 
out within each sparse source sampling interval. Figures~\ref{fig:hzodmig} and \ref{fig:lzodmig} compare 
the images obtained by three different interpolations and the effect of anti-aliasing. All images are 
plotted at the same scale. We do not limit migration aperture for all cases and adopt the anti-aliasing 
criteria suggested by \cite{abma} to filter the input trace before mapping a sample to the image, where 
the source-derivative and receiver-derivative (in the zero-offset case they coincide) determine the filter 
coefficients. As expected, the cubic Hermite interpolation with anti-aliasing leads to the most desirable 
image. The image could be further improved by considering not only the kinematics predicted by the traveltimes 
but also the amplitude factors \cite[]{dellinger,vanelle2}.

\plot{modl}{width=0.95\textwidth}{Constant-velocity-gradient background 
model $v(z,x) = 1.5 + 0.25 z + 0.25 x$ km/s with dome shaped reflectors.}
\plot{hzodmig}{width=0.95\textwidth}{Zero-offset Kirchhoff migration 
image with (top) the cubic Hermite interpolation and (bottom) the 
shift interpolation.}
\plot{lzodmig}{width=0.95\textwidth}{Zero-offset Kirchhoff migration 
image with (top) the linear interpolation and (bottom) the cubic 
Hermite interpolation without anti-aliasing.}

\subsection{Marmousi Model}
\inputdir{marm}

The Marmousi model \cite[]{versteeg} has large velocity variations and is challenging for Kirchhoff 
migration with first-arrivals \cite[]{geoltrain}. We apply a single-fold 2D triangular smoothing of 
radius $20$ m to the original model (see Figure~\ref{fig:vel}) to remove only sharp velocity discontinuities 
but retain the complex velocity structures. Because wave-fronts change shapes rapidly, the traveltime 
interpolation may be subject to inaccurate source-derivatives and provide less satisfying accuracy 
compared to that in a simple model. Although the derivative computation in the proposed eikonal-based 
method is source-sampling independent, in practice we should limit the interpolation interval to be 
sufficiently small, so that the traveltime curve could be well represented by a cubic spline. For the 
smoothed Marmousi model, we use a sparse source sampling of $0.2$ km based on observations of the 
horizontal width of major velocity structures. Figures~\ref{fig:vel} and \ref{fig:slice} compare the 
traveltime interpolation errors of three methods as in Figure~\ref{fig:ierror} for a source located 
at $(0,3.1)$ km from nearby source samples at $(0,3)$ km and $(0,3.2)$ km. Figure~\ref{fig:curve} plots 
a reference traveltime curve for the fixed subsurface location $(2,3.3)$ km computed by a dense eikonal 
solving of $4$ m source spacing against curves produced by the interpolations. While these comparisons 
vary between different source intervals and subsurface locations, the cubic Hermite interpolation 
out-performs the linear and the shift interpolations except for the source singularity region. However 
in Figure~\ref{fig:vel} the errors are relatively large in the upper-left region. These errors occur 
due to the collapse of overlapping branches of the traveltime field \cite[]{xu} that causes wave-front 
discontinuities and undermines the assumptions of the proposed method.

\plot{vel}{width=0.95\textwidth}{(Top) the smoothed Marmousi model. 
The model has a $4$ m fine grid. (Bottom) the traveltime error by 
the cubic Hermite interpolation.}
\plot{slice}{width=0.95\textwidth}{The traveltime error by (top) 
the linear interpolation and (bottom) the shift interpolation.}
\plot{curve}{width=0.95\textwidth}{Traveltime interpolation for a 
fixed subsurface location. Compare between the result from a dense 
source sampling (solid blue), cubic Hermite interpolation (dotted 
magenta), linear interpolation (dashed cyan) and shift interpolation 
(dashed black). The $l2$ norm of the error (against the dense source 
sampling results) of $49$ evenly interpolated sources between interval 
$(0,3)$ km and $(0,3.2)$ km for all locations but the top $100$ m 
source singularity region are $3.9$ s, $9.2$ s and $11.6$ s 
respectively.}

One strategy for imaging multi-arrival wavefields with first-arrival traveltimes is the semi-recursive 
Kirchhoff migration \cite[]{bevc}. It breaks the image into several depth intervals, applies Kirchhoff 
redatuming to the next interval, performs Kirchhoff migration from there, and so on. The small redatuming 
depth effectively limits the maximum traveltime and the evolving of complex waveforms before the most 
energetic arrivals separate from first-arrivals. Since Kirchhoff redatuming also relies on traveltimes 
between datum levels, our method can be fully incorporated into the whole process. Again, for simplicity, 
we do not consider amplitude factors during migration. We use the Marmousi dataset with a source/receiver 
sampling of $25$ m. Due to the source and receiver reciprocity, the receiver side interpolations are 
equivalent to those on the source side. Figure~\ref{fig:dmig0d} is the result of a Kirchhoff migration 
with eikonal solvings at each source/receiver location, i.e. no interpolation performed. Only the upper 
portion is well imaged. Figure~\ref{fig:dmig0} shows the image after employing the cubic Hermite 
interpolation with a $0.2$ km sparse source/receiver sampling, which means $7$ source interpolations 
within each interval. Even though a $7$ times speed-up is not attainable in practice due to the extra 
computations in source-derivative and interpolation, we are still able to gain an approximately $5$-fold 
cost reduction in traveltime computations, while keeping the image quality comparable between 
Figures~\ref{fig:dmig0d} and \ref{fig:dmig0}. Next, following \cite{bevc}, we downward continue the data 
to a depth of $1.5$ km in three datuming steps. The downward continued data are then Kirchhoff migrated 
and combined with the upper portion of Figure~\ref{fig:dmig0}. We keep the same $0.2$ km sparse 
source/receiver sampling whenever eikonal solvings are required in this process. Figure~\ref{fig:dmig2} 
shows the image obtained by the semi-recursive Kirchhoff migration. The target zone at approximately 
$(2.5,6.5)$ km appears better imaged.

\plot{dmig0d}{width=0.95\textwidth}{Image of Kirchhoff migration with 
first-arrivals (no interpolation).}
\plot{dmig0}{width=0.95\textwidth}{Image of Kirchhoff migration with 
first-arrivals and a sparse source/receiver sampling.}
\plot{dmig2}{width=0.95\textwidth}{Image of semi-recursive Kirchhoff 
migration with a three-step redatuming from top surface to $1.5$ km depth 
and a $0.5$ km interval each time. The sparse source/receiver sampling 
is the same as in Figure~\ref{fig:dmig0}.}

\section{Discussion}

The proposed approach could be implemented either along with a finite-difference eikonal solver or 
separately. Our current implementation outputs both traveltime and source-derivative at the same time, 
with a roughly 30\% extra cost per eikonal solve compared to a FMM solver without the source-derivative 
functionality. An interpolation with these source-derivatives is superior to the ones without and thus 
enables an accurate traveltime-table compression. For 3D datasets, as both inline and crossline 
directions may benefit from the source-derivative and interpolation, the overall data compression could 
be significant. For instance, interpolating $10$ shots within each sparse source sampling interval in 
both inline and crossline directions leads to an approximately $100$-fold savings in traveltime storage. 
The method could be further combined with an interpolation within each source, for example from a coarse 
grid to a fine grid, for a greater data compression.

While our implementation is for first-arrivals only, the governing equations are valid also for other 
characteristic branches, for example the most energetic arrivals. However, an underlying assumption 
of the proposed method is a continuous change in the wave-front of selected arrivals within individual 
sources. For first-arrivals, this condition always holds valid. However, the most energetic wave-front 
can be more complicated than that of first-arrival, for example only piece-wise continuous, which may 
lead to a potential degradation in accuracy. For example, \cite{nichols} showed the most energetic 
wave-fronts in the Marmousi model. Another assumption is that the traveltime source-derivatives are 
continuous between nearby sources. This condition breaks down when multi-pathing takes place. \cite{vanelle} 
suggested to smooth traveltimes around the discontinuities in order to overcome this limitation. In theory, 
one can try to identify the discontinuities and only perform interpolation within individual continuous 
pieces by using the eikonal-based source-derivatives. By doing so, one should be able to recover branch 
jumping in interpolated traveltimes, but only for those locations within the identified continuous pieces. 
For the discontinuities themselves as well as the gaps between them, additional eikonal solving may be 
required. An efficient implementation of this strategy remains open for future research.

\section{Conclusion}

We have shown an application of computing traveltime source-derivatives in Kirchhoff migration. For 
first-arrivals, a cubic Hermite traveltime interpolation using the first-order source-derivatives 
speeds up computation and reduces storage without noticeably sacrificing accuracy. Anti-aliasing is 
another direct application of traveltime source-derivatives that can be easily incorporated into 
Kirchhoff migration.

Generalization of the method to 3D is straightforward. The computed derivative attributes may 
benefit other areas besides the kinematic-only Kirchhoff migration shown in this paper. An extension 
to multi-arrival traveltimes needs further investigation.

\section{Acknowledgments}

We thank the editors, Samuel Gray, and three anonymous reviewers for constructive 
suggestions that helped improving the quality of the manuscript. We thank Tariq Alkhalifah and 
Alexander Vladimirsky for useful discussions and sponsors of the Texas Consortium for Computational 
Seismology (TCCS) for financial support of this research. \new{This publication is authorized by 
the Director, Bureau of Economic Geology, The University of Texas at Austin.}

\append{FMM implementation of source-derivatives}

The FMM is a non-iterative eikonal solver with $O(N \log N)$ complexity, 
where $N$ is the total number of grid points of the discretized domain. 
It relies on a heap data structure to keep the updating sequence, and 
a local one-sided upwind finite-difference scheme for ensuring the 
causality \cite[]{sethian1}. Consider in 3D a cubic domain discretized into Cartesian grids, 
with uniform grid size of $(\Delta x,\Delta y,\Delta z)$. Let $\hat{T}_{i,j}^k$ 
be the traveltime value at vertices $\mathbf{x}_{i,j}^k = (x_i,y_j,z_k)$ 
and define difference operator $D_x^{\pm}$ for $x$ direction as 
\begin{equation}
\label{eq:upwind1}
D_x^{\pm} \hat{T}_{i,j}^k = 
\pm \frac{\hat{T}_{i \pm 1,j}^k - \hat{T}_{i,j}^k}{\Delta x}\;,
\end{equation}
The causality condition requires picking \old{a downwind}\new{an upwind} neighbor in all 
directions at $\mathbf{x}_{i,j}^k$. 
\begin{equation}
\label{eq:upwind2}
\hat{D}_x \hat{T}_{i,j}^k = 
\max \left( D_x^- \hat{T}_{i,j}^k, -D_x^+ \hat{T}_{i,j}^k, 0 \right)\;.
\end{equation}
After similar definitions for $\hat{D}_y$ and $\hat{D}_z$, the local 
upwind scheme in FMM for equation \ref{eq:insert} reads 
\begin{equation}
\label{eq:update}
\left( \hat{D}_x \hat{T}_{i,j}^k \right)^2 +
\left( \hat{D}_y \hat{T}_{i,j}^k \right)^2 +
\left( \hat{D}_z \hat{T}_{i,j}^k \right)^2 = W_{i,j}^k\;.
\end{equation}
For $\partial \hat{T} / \partial \mathbf{x}$ in equation \ref{eq:part} 
and $\partial \hat{T} / \partial \mathbf{q}$ in equation \ref{eq:chain}, 
we can apply the same upwind strategy: 
\begin{equation}
\label{eq:solve1}
\hat{D}_x \hat{T}_{i,j}^k \cdot 
\hat{D}_x \left( \frac{\partial \hat{T}}{\partial \mathbf{x}} \right)_{i,j}^k +
\hat{D}_y \hat{T}_{i,j}^k \cdot 
\hat{D}_y \left( \frac{\partial \hat{T}}{\partial \mathbf{x}} \right)_{i,j}^k +
\hat{D}_z \hat{T}_{i,j}^k \cdot 
\hat{D}_z \left( \frac{\partial \hat{T}}{\partial \mathbf{x}} \right)_{i,j}^k =
\frac{1}{2} \left( \frac{\partial W}{\partial \mathbf{x}} \right)_{i,j}^k\;,
\end{equation}
\begin{equation}
\label{eq:solve2}
\left( \frac{\partial \hat{T}}{\partial \mathbf{q}} \right)_{i,j}^k = 
\hat{D}_\mathbf{q} \hat{T}_{i,j}^k,\,\,\mathbf{q} = (x,y,z)\;.
\end{equation}
where in equation \ref{eq:solve1} $\hat{D}_x$, $\hat{D}_y$ and $\hat{D}_z$ 
are chosen according to $\hat{T}_{i,j}^k$, regardless of $\partial \hat{T} 
/ \partial \mathbf{x}$. Finally, 
\begin{equation}
\label{eq:final}
\left( \frac{\partial T}{\partial \mathbf{x_s}} \right)_{i,j}^k = 
\left( \frac{\partial \hat{T}}{\partial \mathbf{x}} \right)_{i,j}^k - 
\left( \frac{\partial \hat{T}}{\partial \mathbf{q}} \right)_{i,j}^k\;.
\end{equation}
To incorporate the computation of traveltime source-derivatives into 
FMM, one only needs to add equations \ref{eq:solve1}, \ref{eq:solve2} and 
\ref{eq:final} after \ref{eq:update}. An extra upwind sorting and 
solving after pre-computing $\hat{T}$ is not necessary. The total complexity of 
FMM with the auxiliary output of traveltime source-derivative remains $O(N \log N)$.

\append{Interpolation of source-derivatives}

Applying the chain-rule \ref{eq:dinterp} to equation \ref{eq:Hermite}, we arrive at the interpolation 
equation for source-derivatives in the cubic Hermite scheme:
\begin{equation}
\label{eq:hdinterp}
\begin{array}{lcl}
\Delta x_s\,
\frac{\partial T (z,x; z_s,x_s + \alpha \Delta x_s)}{\partial (x_s + \alpha \Delta x_s)} 
& = & (6\alpha^2-6\alpha)\,T (z,x; z_s,x_s) \\
& + & (3\alpha^2-4\alpha+1)\,\frac{\partial T}{\partial x_s} (z,x; z_s,x_s) \\
& + & (-6\alpha^2+6\alpha)\,T (z,x; z_s,x_s + \Delta x_s) \\
& + & (3\alpha^2-2\alpha)\,\frac{\partial T}{\partial x_s} (z,x; z_s,x_s + \Delta x_s)\;.
\end{array}
\end{equation}
Analogously, the interpolation of source-derivatives in the linear scheme \ref{eq:linear} reads:
\begin{equation}
\label{eq:ldinterp}
\begin{array}{lcl}
\Delta x_s\,
\frac{\partial T (z,x; z_s,x_s + \alpha \Delta x_s)}
{\partial (x_s + \alpha \Delta x_s)} = 
-T (z,x; z_s,x_s) + T (z,x; z_s,x_s + \Delta x_s)\;.
\end{array}
\end{equation}
which is a simple first-order finite-difference estimation. Finally, in the case of shift scheme 
\ref{eq:shift}, the partial derivative $\partial / \partial \alpha$ must be applied to the shifted 
traveltime terms at the same time: 
\begin{equation}
\label{eq:sdinterp}
\begin{array}{lcl}
\Delta x_s\,
\frac{\partial T (z,x; z_s,x_s + \alpha \Delta x_s)}
{\partial (x_s + \alpha \Delta x_s)} 
& = & -T (z,x- \alpha \Delta x_s; z_s,x_s) \\
& - & (1-\alpha) \Delta x_s\,
\frac{\partial T (z,x-\alpha \Delta x_s; z_s,x_s)}{\partial (x-\alpha \Delta x_s)} \\
& + & T (z,x + (1-\alpha) \Delta x_s; z_s,x_s + \Delta x_s) \\
& - & \alpha \Delta x_s\,
\frac{\partial T (z,x+(1-\alpha) \Delta x_s; z_s,x_s+\Delta x_s)}
{\partial (x+(1-\alpha) \Delta x_s)}\;.
\end{array}
\end{equation}
The required spatial derivatives can be estimated from the traveltime table by means of 
finite-differences, for example by using the upwind approximation \ref{eq:upwind2}.

\bibliographystyle{seg}
\bibliography{eikodsf}
