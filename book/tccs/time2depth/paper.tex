\published{Geophysics, 73, no. 5, VE205-VE210, (2008)}
\title{Time-to-depth conversion and seismic velocity estimation using time-migration velocity}

\renewcommand{\thefootnote}{\fnsymbol{footnote}} 

%\ms{GEO-2008-0206}

\address{
\footnotemark[1]Department of Mathematics, \\
Courant Institute of Mathematical Science, \\
New York University,\\
251 Mercer Street, New York, NY 10012 \\
\footnotemark[2]Bureau of Economic Geology, \\
John A. and Katherine G. Jackson School of Geosciences \\
The University of Texas at Austin \\
University Station, Box X \\
Austin, TX 78713-8972 \\
\footnotemark[3]Department of Mathematics, \\
University of California, Berkeley, \\
Berkeley, CA, 94720}
\author{Maria Cameron\footnotemark[1], Sergey Fomel\footnotemark[2], and James Sethian\footnotemark[3]}

\footer{Seismic Velocity Estimation}
\lefthead{Cameron, Fomel, Sethian}
\righthead{Velocity estimation from time migration}

\maketitle

\newpage

\begin{abstract}
  The objective of this work is to build an efficient algorithm (a) to
  estimate seismic velocity from time-migration velocity, and (b)
  to convert time-migrated images to depth. We establish theoretical
  relations between the time-migration velocity and the seismic
  velocity in 2-D and 3-D using paraxial ray tracing theory.  The
  relation in 2-D implies that the conventional Dix velocity is the
  ratio of the interval seismic velocity and the geometrical spreading
  of the image rays.  We formulate an inverse problem of finding
  seismic velocity from the Dix velocity and develop a numerical
  procedure for solving it. This procedure consists of two steps: (1)
  computation of the geometrical spreading of the image rays and the
  true seismic velocity in the time-domain coordinates from the Dix
  velocity; (2) conversion of the true seismic velocity from the
  time domain to the depth domain and computation of the transition
  matrices from time-domain coordinates to depth.

  For step 1, we derive a partial differential equation (PDE) in 2-D
  and 3-D relating the Dix velocity and the geometrical spreading of
  the image rays to be found.  This is a nonlinear elliptic PDE.  The
  physical setting allows us to pose a Cauchy problem for it. This
  problem is ill-posed.  However we are able to solve it numerically
  in two ways on the required interval of time.  One way is a finite
  difference scheme inspired by the Lax-Friedrichs method.  The second
  way is a spectral Chebyshev method. For step 2, we develop an efficient
  Dijkstra-like solver motivated by Sethian's Fast Marching Method.


  We test our numerical procedures on a synthetic data example and
  apply them to a field data example. We demonstrate that our
  algorithms give significantly more accurate estimate of the seismic
  velocity than the conventional Dix inversion. Our velocity estimate
  can be used as a reasonable first guess in building velocity models
  for depth imaging.
 
\end{abstract}

%%%%%%%%%%%%%%%%%%%%%%%%%%%%%%%%%%%%%%%%

\newpage

\section{Introduction}
Time-domain seismic imaging is a robust and efficient process
routinely applied to seismic data \cite[]{yilmaz,robein}. Rapid
scanning and determination of time-migration velocity can be
accomplished either by repeated migrations \cite[]{yilmaz2} or by
velocity continuation
\cite[]{fomel2003}.  Time migration is considered adequate for seismic
imaging in areas with mild lateral velocity variations. However, even
mild variations can cause structural distortions of time-migrated
images and render them inadequate for accurate geological
interpretation of subsurface structures.

To remove structural errors inherent in time migration, it is
necessary to convert time-migrated images into the depth domain either
by migrating the original data with a prestack depth migration
algorithm or by depth migrating post-stack data after time demigration
\cite[]{kim}. Each of these options requires converting the time migration
velocity to a velocity model in depth.

The connection between the time- and depth-domain coordinates is
provided by the concept of \emph{image ray}, introduced by
\cite{hubral1977}. Image rays are seismic rays that arrive normal to
the Earth's surface.  Hubral's theory explains how a depth velocity
model can be converted to the time coordinates. However, it does not
explain how a depth velocity model can be converted to the
time-migration velocity.  Moreover, image-ray tracing is a
numerically inconvenient procedure for achieving uniform coverage of
the subsurface. This may explain why simplified image-ray-tracing
algorithms \cite[]{larner,hatton} did not find widespread application
in practice. Other limitations of image rays are related to the
inability of time migration to handle large lateral variations in
velocity \cite[]{plumes,robein}.
 
The objective of the present work is to find an efficient method for
building a velocity model from time-migration velocity.  We
establish new ray-theoretic connections between time-migration
velocity and seismic velocity in 2-D and 3-D.  These results are
based on the image ray theory and the paraxial ray tracing theory
\cite[]{popov-psencik,cerveny,popov}. Our results can be viewed as an
extension of the Dix formula \cite[]{dix1955} to laterally
inhomogeneous media.  We show that the Dix velocity is seismic
velocity divided by the geometrical spreading of the image
rays. Hence, we use the Dix velocity instead of time migration
velocity as a more convenient input.  We develop a numerical
approach to find (a) seismic velocity from the Dix velocity, and
(b) transition matrices from the time-domain coordinates to the
depth-domain coordinates.  We test our approach on synthetic and field
data examples.

Our approach is complementary to more traditional velocity estimation
methods and can be used as the first step in a velocity model building
process.

%%%%%%%%%%%%%%%%%%%%%%%%%%%%%%%%%%%%%%%%

\section{Time Migration Velocity}

Kirchhoff prestack time migration is commonly based on the following
travel time approximation \cite[]{yilmaz}. Let $\mathbf{s}$ be a
source, $\mathbf{r}$ be a receiver, and $\mathbf{x}$ be the reflection
subsurface point. Then the total travel time from $\mathbf{s}$ to
$\mathbf{x}$ and from $\mathbf{x}$ to $\mathbf{r}$ is approximated as
\begin{equation}
\label{tsxr}
T(\mathbf{s},\mathbf{x})+T(\mathbf{x},\mathbf{r})
\approx\hat{T}(\mathbf{x}_0,t_0,\mathbf{s},\mathbf{r})
\end{equation}
where $\mathbf{x}_0$ and $t_0$ are effective parameters 
of the subsurface point $\mathbf{x}$. The approximation $\hat{T}$ 
usually takes the form of the double-square-root equation
\begin{equation}
\label{ttappr}
\hat{T}(\mathbf{x}_0,t_0,\mathbf{s},\mathbf{r})=
\sqrt{t_0^2+\frac{|\mathbf{x}_0-\mathbf{s}|^2}{v_m^2(\mathbf{x}_0,t_0)}}+
\sqrt{t_0^2+\frac{|\mathbf{x}_0-\mathbf{r}|^2}{v_m^2(\mathbf{x}_0,t_0)}},
\end{equation}
where $\mathbf{x}_0$ and $t_0$ are the escape location and the travel
time of the \emph{image ray} \cite[]{hubral1977} from the subsurface point
$\mathbf{x}$.  Regarding this approximation, let us list four cases
depending on the seismic velocity $v$ and the dimension of the
problem:
\begin{description}
\item[2-D and 3-D, velocity $v$ is constant.] 
Equation \ref{ttappr} is exact, and 
$v_m=v$.
\item[2-D and 3-D, velocity $v$ depends only on the depth $z$.]
Equation \ref{ttappr} is a consequence of the truncated Taylor expansion
for the travel time around the surface point $\mathbf{x}_0$. 
Velocity $v_m$ depends only on $t_0$ and is the root-mean-square velocity:
\begin{equation}
\label{eq:fdix}
v_m(t_0)=\sqrt{\frac{1}{t_0}\int_0^{t_0}v^2(z(t))dt}.
\end{equation}
In this case, the Dix inversion formula \cite[]{dix1955} is exact. We
formally define the Dix velocity $v_{Dix}(t)$ by inverting
equation~\ref{eq:fdix}, as follows:
\begin{equation}
\label{eq:dix}
v_{Dix}(t)=\sqrt{\frac{d}{d\,t_0}\left(t_0 v_m^2(t_0)\right)}\;.
\end{equation}
\item[2-D, velocity is arbitrary.]
Equation \ref{ttappr} is a consequence of the truncated Taylor expansion
for the travel time around the surface point $x_0$. 
Velocity $v_m(x_0,t_0)$ is a certain kind of mean velocity, and
we establish its exact meaning in the next section.
\item[3-D, velocity is arbitrary.]
Equation \ref{ttappr} is  heuristic and is not a consequence of the
truncated Taylor expansion. In order to write an analog of 
travel time approximation \ref{ttappr} for 3-D, we use the
relation \cite[]{hubralkrey} 
\begin{equation}
\label{Gammadef}
\tensor{\Gamma}=[v(\mathbf{x}_0)\tensor{R}(\mathbf{x}_0,t_0)]^{-1},
\end{equation}
where $\tensor{\Gamma}$ is the matrix
of the second derivatives of the travel times 
from a subsurface point $\mathbf{x}$ to the surface,
$\tensor{R}$ is the matrix of radii of curvature of the emerging wave front
from the point source $\mathbf{x}$, and $v(\mathbf{x}_0)$ is 
the velocity at the surface point $\mathbf{x}_0$.
For convenience, we prefer to deal with matrix 
$\tensor{K}\equiv\tensor{\Gamma}^{-1}$,
which, according to equation \ref{Gammadef} is
\begin{equation}
\label{Kdef}
\tensor{K}(\mathbf{x}_0,t_0)\equiv v(\mathbf{x}_0)\tensor{R}(\mathbf{x}_0,t_0).
\end{equation}
The travel time approximation for 3-D implied by the Taylor expansion is
\begin{eqnarray}
\label{3dappr}
&\hat{T}(\mathbf{x}_0,t_0,\mathbf{s},\mathbf{r})\nonumber \\
=&
\sqrt{t_0^2+t_0(\mathbf{x}_0-\mathbf{s})^T
[\tensor{K}(\mathbf{x}_0,t_0)]^{-1}(\mathbf{x}_0-\mathbf{s})}\\
+&\sqrt{t_0^2+t_0(\mathbf{x}_0-\mathbf{r})^T
[\tensor{K}(\mathbf{x}_0,t_0)]^{-1}(\mathbf{x}_0-\mathbf{r})}.\nonumber
\end{eqnarray}
The entries of the matrix $\frac{1}{t_0}\tensor{K}(\mathbf{x}_0,t_0)$
have dimension of squared velocity and can be chosen optimally in the
process of time migration.  
%In 3-D, we will call the square roots of
%the entries of this matrix time-migration velocity.
It is possible to show, however, that one needs only the values of
\begin{equation}
\label{3Dvdix}
\det\left(\frac{\partial }{\partial t_0}\tensor{K}(\mathbf{x}_0,t_0)\right)
\end{equation}
to perform the inversion. This means that the conventional 3-D
prestack time migration with traveltime approximation~\ref{ttappr}
provides sufficient input for our inversion procedure in 3-D.  The
determinant in equation~\ref{3Dvdix} is well approximated by the
square of the Dix velocity obtained from the 3-D prestack time
migration using the approximation given by equation~\ref{ttappr}.
\end{description}
\new{One can employ more complex and accurate approximations than the double-square-root equations~\ref{ttappr} and~\ref{3dappr}, i.e. the shifted hyperbola approximation 
\cite[]{siliqi}. However, other known approximations also involve parameters equivalent to $v_m$ or $\tensor{K}$.}

%%%%%%%%%%%%%%%%%%%%%%%%%%%%%%%%%%%%%%%%%%%%%%

\section{Seismic Velocity}
In this section, we will establish theoretical relationships between
time-migration velocity and seismic velocity in 2-D and 3-D.

The seismic velocity and the Dix
velocity are connected through the quantity $\mathbf{Q}$, the
geometrical spreading of image rays.  $\mathbf{Q}$ is a scalar in 2-D
and a $2\times 2$ matrix in 3-D.  The simplest way to introduce
$\mathbf{Q}$ is the following. Trace an image ray
$\mathbf{x}(\mathbf{x}_0,t)$. $\mathbf{x}_0$ is the
starting surface point, $t$ is the traveltime. Call this ray
\emph{central}. Consider a small tube of rays around it.  All these
rays start from a small neighborhood $d\mathbf{x}_0$ of the point
$\mathbf{x}_0$ perpendicular to the earth surface.  Thus, they
represent a fragment of a plane wave propagating downward.  Consider
the fragment of the wave front defined by this ray tube at time $t_0$.
Let $d\mathbf{q}$ be the fragment of the tangent to the front at the
point $\mathbf{x}(\mathbf{x}_0,t_0)$ reached by the central ray at
time $t_0$, bounded by the ray tube (Figure~\ref{fig:Qdef}).  Then, in
2-D, $Q$ is the derivative $Q(x_0,t_0)=\frac{dq}{dx_0}$.  In 3-D,
$\mathbf{Q}$ is the matrix of the derivatives
$\mathbf{Q}_{ij}(\mathbf{x}_0,t_0)=
\frac{d\mathbf{q}_i}{d\mathbf{x}_{0j}}$, $i,j=1,2$, where
derivatives are taken along certain mutually orthogonal directions
$\mathbf{e}_1$, $\mathbf{e}_2$
\cite[]{popov-psencik,cerveny,popov}.

\inputdir{.}
\plot{Qdef}{width=0.6\columnwidth}{Illustration for the definition of geometrical spreading.}


The time evolution of the matrices $\tensor{Q}$ and $\tensor{P}$  
is given by
\begin{equation}
\label{QPevol}
\frac{d}{dt}\left(\begin{array}{c}\tensor{Q}\\ \tensor{P}\end{array}\right)=
\left(\begin{array}{cc}\tensor{0}&v_0^2\tensor{I}\\-\frac{1}{v_0}\tensor{V}&\tensor{0}
\end{array}\right)
\left(\begin{array}{c}\tensor{Q}\\ \tensor{P}\end{array}\right),
\end{equation}
where $v_0$ it the velocity at the central ray at time $t$,
$\tensor{V}=\left(\frac{\partial^2 v}{\partial q_i\partial
    q_j}\right)_{i,j=1,2}$, and $\tensor{I}$ is the $2\times 2$
identity matrix.  The absolute value of $\det \mathbf{Q}$ has a simple
meaning: it is the geometrical spreading of the image rays
\cite[]{popov-psencik,cerveny,popov}.  The matrix $\tensor{\Gamma}$,
introduced in the previous section, relates to $\tensor{Q}$ and
$\tensor{P}$ as $\tensor{\Gamma}=\tensor{P}\tensor{Q}^{-1}$.  Hence,
$\tensor{K}=\tensor{Q}\tensor{P}^{-1}$.

In \cite[]{cfs}, we have proven that 
\begin{equation}
\label{2Dvdix-v}
v_{Dix}(x_0,t_0)\equiv\sqrt{\frac{\partial}{\partial t_0}
\left(t_0v_m^2(x_0,t_0)\right)}=
\frac{v(x(x_0,t_0),z(x_0,t_0))}{|Q(x_0,t_0)|}
\end{equation}
in 2-D, where $v_m(x_0,t_0)$ is the time-migration velocity, and
\begin{equation}
\label{3Dvdix-v}
\frac{\partial}{\partial t_0}
\left(\tensor{K}(\mathbf{x}_0,t_0)\right)=
v(\mathbf{x}(\mathbf{x}_0,t_0))\left(\mathbf{Q}(\mathbf{x}_0,t_0)
\mathbf{Q}^T(\mathbf{x}_0,t_0)\right)^{-1}
\end{equation}
in 3-D, $\tensor{K}$ is defined by equation~\ref{Kdef}
and can be determined from equation~\ref{3dappr}.


\section{Partial differential equations for the geometrical spreading of image rays}
In this section, we derive the partial differential equations for $Q$
in 2-D and 3-D.  From now on, we will denote the square of the Dix
velocity by $f$ in 2-D and the corresponding matrix by
$\mathbf{F}$ in 3-D, to avoid the subscript:
\begin{equation}
\tensor{F}\equiv \label{Fdef}\frac{\partial}{\partial t_0}
\left(\tensor{K}(\mathbf{x}_0,t_0)\right).
\end{equation}
Furthermore, we imply that $t_0$ denotes the one-way
traveltime  along the image rays.
Finally, we assume that our domain does 
not contain caustics, i.e., the image rays do not cross
on the interval of time we consider.

\subsection{2-D case}
Consider a set of image rays coming to the surface.
Suppose we are tracing them all backwards in time 
together with the quantities $Q$ and $P$. 
Let us eliminate the unknown velocity $v$ in system~\ref{QPevol}
using equation~\ref{2Dvdix-v}. 
Moreover, let us eliminate the differentiation 
in $q$ using the definition
of $Q$ and rewrite it in the time-domain 
coordinates $x_0,t_0)$.
Indeed, $Q=\frac{dq}{dx_0}$, hence 
$\frac{d}{dq}=\frac{d}{dx_0}\frac{dx_0}{dq}=Q^{-1}\frac{d}{dx_0}$.
Therefore, system~\ref{QPevol} becomes
\begin{equation}
\label{2DQPsys}
Q_{t_0}=(fQ)^2P,\qquad 
P_{t_0}=-\frac{1}{fQ}\left(\frac{(fQ)_{x_0}}{Q}\right)_{x_0}.
\end{equation}
Eliminating $P$ in system~\ref{2DQPsys}, we get 
the following partial differential equation (PDE) for $Q$
\begin{equation}
\label{2DQpde}
\left(\frac{Q_{t_0}}{f^2Q^2}\right)_{t_0}=
-\frac{1}{fQ}\left(\frac{(fQ)_{x_0}}{Q}\right)_{x_0}.
\end{equation}
The initial conditions are $Q(x_0,0)=1$, $Q_{t_0}(x_0,0)=0$.
Equation~\ref{2DQpde} simplifies in terms of the negative 
reciprocal of $Q$. Introduce $y=-\frac{1}{Q}$. Then
equation~\ref{2DQpde} becomes
\begin{equation}
\label{2Dy}
\left(\frac{y_{t_0}}{f^2}\right)_{t_0}=
\frac{y}{f}\left(\left(\frac{f}{y}\right)_{x_0}y\right)_{x_0}.
\end{equation}
In the expanded form, equation~\ref{2Dy} is
\begin{equation}
\label{2Dyexpanded}
\frac{y_{t_0t_0}}{f^2}-2\frac{y_{t_0}f_{t_0}}{f^3}
=y\frac{f_{x_0x_0}}{f}-y_{x_0}\frac{f_{x_0}}{f}-y_{x_0x_0}+
\frac{y_{x_0}^2}{y}.
\end{equation}

\subsection{3-D case}
Equation~\ref{3Dvdix-v} can be rewritten in the following form
\begin{equation}
\label{3Dv}
v=\sqrt[4]{\det\mathbf{F}(\det\mathbf{Q})^2},
\end{equation}
where $\mathbf{F}$ is the left-hand side of equation~\ref{3Dvdix-v}.
As in 2-D, we rewrite system~\ref{QPevol} in the time-domain
coordinates $(\mathbf{x}_0,t_0)$. Then we get
\begin{eqnarray}
\mathbf{Q}_{t_0}&=&v^2\mathbf{P},\label{3DQ}\\
\mathbf{P}_{t_0}&=&-\frac{1}{v}\mathbf{Q}^{-1}\left[
\nabla\left(\mathbf{Q}^{-1}\nabla v\right)^T\right]\mathbf{Q},\label{3DP}
\end{eqnarray}
where $v$ is given by equation~\ref{3Dv}, and the gradients 
are taken with respect to $\mathbf{x}_0$.
Then the PDE for $\mathbf{Q}$ is
\begin{equation}
\label{3DQpde}
\left(\frac{1}{v^2}\mathbf{Q}_{t_0}\right)_{t_0}=
-\frac{1}{v}\mathbf{Q}^{-1}\left[
\nabla\left(\mathbf{Q}^{-1}\nabla v\right)^T\right]\mathbf{Q}.
\end{equation}
The initial conditions are $\mathbf{Q}(\mathbf{x}_0,0)=\mathbf{I}_2$,
$\mathbf{Q}_{t_0}(\mathbf{x}_0,0)=\mathbf{0}$.
The required input $\sqrt{\det\mathbf{F}}$ is well-approximated
by the squares of the Dix velocity obtained
from the 3-D prestack time migration.
We emphasize that despite the fact that $\mathbf{Q}$ is a 
matrix in 3-D, scalar data are enough for its computation.

\subsection{Cauchy problem for elliptic equations}
Equations~\ref{2DQpde} and \ref{3DQpde} reveal the nature of the
instabilities in the problem in hand.  These PDEs are elliptic, while
the physical setting allows us to pose only a Cauchy problem for
them, which is well-known to be ill-posed.  Furthermore, the fact that
the PDEs involve not only the Dix velocity itself, but also its first
and second derivatives, leads to high sensitivity of the solutions to
input data.
 
Nonetheless, we found two ways 
for solving these PDEs numerically on the required,
 and relatively short, 
interval of time: namely,
a finite difference scheme inspired by the Lax-Friedrichs 
method and a spectral Chebyshev method. 
%In order to justify our methods, we have conducted
A detailed analysis of the problem shows that our methods work thanks to
\begin{enumerate}
\item{the special input $v_{Dix}$, corresponding 
to a positive finite seismic velocity;
}
\item{the special initial conditions $Q(x_0,t_0=0)=1$, $Q_t(x_0,t_0=0)=0$
 corresponding to the image rays;
}
\item{the fact that
our finite difference method contains error 
terms which damp the high harmonics;
truncation of the polynomial series in the 
spectral Chebyshev method which is similar to truncation
of the high harmonics; and
}
\item{the short interval of time, in which we need to compute the solution, 
so that the growing low harmonics fail to develop significantly.
}
\end{enumerate}
Items 1 and 2 say that the exact solutions of our PDEs for the
hypothetical perfect Dix velocity given by equations~\ref{2Dvdix-v}
and~\ref{3Dvdix-v} are finite and nonzero.  Items 3 and 4 say that the
numerical methods take care of the imperfection of the data and
computations on a short enough time interval.

\section{Inversion Methods}
Our numerical reconstruction of true seismic velocity
$v(\mathbf{x})$ in depth-domain coordinates from the Dix velocity
given in the time-domain coordinates $(\mathbf{x}_0,t_0)$ consists of
two steps:
\begin{description}
\item[Step 1.] Compute the geometrical spreading of the image rays in
  the time-domain coordinates from the Dix velocity by solving
  equation~\ref{2DQpde} in 2-D and~\ref{3DQpde} in 3-D.  Then find
  $v(\mathbf{x}_0,t_0)$ from equation~\ref{2Dvdix-v} in 2-D and
  equation~\ref{3Dv} in 3-D.
\item[Step 2.] Convert the seismic velocity $v(\mathbf{x}_0,t_0)$ in
  the time-domain coordinates to the depth-domain coordinates
  $\mathbf{x}$ using the time-to-depth conversion algorithm, which was
  presented by \cite{cfs}.  It is a fast and robust Dijkstra-like
  solver motivated by the Fast Marching method \cite[]{Sethian1996,SethFast}.
\end{description}

We performed step 1 in two ways: a finite difference method and 
a spectral Chebyshev method.

\subsection{Finite difference method}
This method was inspired by the Lax-Friedrichs method for hyperbolic
conservation laws \cite{lax} due to its total variation diminishing property.
We use the ``Lax-Friedrichs averaging'' and the wide 5-point
stencil in space. The scheme is given by
\begin{eqnarray}
P^{n+1}_j&=&\frac{P_{j+1}^n+P_{j-1}^n}{2}
-\frac{\Delta t}{4\Delta x}
\frac{1}{v^n_j}\left(\frac{v^n_{j+2}-v^n_j}{Q^n_{j+1}}-
\frac{v^n_{j}-v^n_{j-2}}{Q^n_{j-1}}\right),\label{2dlfp}\\
-\frac{1}{Q^{n+1}_j}&=&-\frac{1}{Q^n_j}+
\frac{\Delta t}{2}\left((f_j^n)^2P_j^n+(f_j^{n+1})^2P_j^{n+1}\right),
\label{2dlfq}
\end{eqnarray}
where $v\equiv fQ$.
We impose the following boundary conditions $Q^n_0=Q^n_{nx-1}=1$, 
$P_0^n=P^n_{nx-1}=0$
corresponding the straight boundary rays.
We set the initial conditions 
$Q_j^0=1$, $P^0_j=0$ corresponding to the initial conditions 
for the image rays traced backward: $Q=1$, $P=0$.
%We introduced this scheme and performed an error and 
%perturbation analysis for it and extended it for 3-D in \cite[]{camseth}.


\subsection{Spectral Chebyshev method}
Alternatively, we solve our PDE
in the form given by equation~\ref{2Dy}
by a spectral Chebyshev method \cite{boyd}.
Using cubic splines, we define the input data at $N_{coef}$ 
Chebyshev points.
We compute the Chebyshev coefficients and the coefficients of the
derivatives in the right-hand side of equation~\ref{2Dy}. 
Then we use a smaller number $N_{eval}$ of the 
coefficients for function evaluation. 
We need to do such Chebyshev differentiation twice.
Finally we perform the time step using 
the stable third-order Adams-Bashforth method
\cite{boyd},
which is
\begin{equation}
\label{adams}
u^{n+1}=u^n+\Delta t\left(\frac{23}{12}F^n-
\frac{4}{3}F^{n-1}+\frac{5}{12}F^{n-2}\right),
\end{equation}
where $F^n\equiv F(u^{n},x,t^{n})$ is the right-hand side.
In numerical examples, we tried $N_{coef}\ge 100$ and $N_{eval}\le 25$.
This method allows larger time steps than the finite difference, and 
has the adjustable parameter $N_{eval}$.


For step 2, we use a Dijkstra-like solver introduced in \cite{cfs}.
It is an efficient time-to-depth conversion algorithm motivated by
the Fast Marching Method \cite[]{Sethian1996}.  
The input for this algorithm 
is $v(x_0,t_0)$ and the outputs are 
the seismic velocity $v(x,z)$ and the transition matrices from time-domain 
to depth-domain coordinates $x_0(x,z)$ and $t_0(x,z)$. We solve the
eikonal equation with an unknown right-hand side coupled
with the orthogonality relation
\begin{equation}
\label{t2d_system}
|\nabla t_0|=\frac{1}{v(x_0(x,z),t_0(x,z))},\quad \nabla t_0\cdot\nabla x_0=0.
\end{equation}
The orthogonality relation means that the image rays are orthogonal to
the wavefronts.  Such time-to-depth conversion is very fast and
produces the outputs directly on the depth-domain grid.

%%%%%%%%%%%%%%%%%%%%%%%%%%%%%%%%%%%%

\section{Examples}

\subsection{Synthetic data example}
\inputdir{msyn2}

\multiplot{3}{ve,vd2,vf2}{width=0.8\textwidth}{Synthetic test on interval
velocity estimation. (a) Exact velocity model. (b) Dix velocity
converted to depth. (c) Estimated velocity model and the corresponding
image rays. The image ray spreading causes significant differences
between Dix velocity and \old{true} \new{estimated} velocity.}

Figure~\ref{fig:ve} shows a synthetic velocity model. The model
contains a high velocity anomaly that is asymmetric and decays
exponentially. The corresponding Dix velocity mapped from time to
depth is shown in Figure~\ref{fig:vd2}. There is a significant
difference between both the value and the shape of the velocity
anomaly recovered by the Dix method and the true anomaly. The
difference is explained by taking into account geometrical spreading
of image rays. Figure~\ref{fig:vf2} shows the velocity recovered by our
method and the corresponding family of image rays. \new{An analogous
  3-D example is provided in \cite[]{cfs}.}

%%%%%%%%%%%%%%%%%%%%%%%%%%%%%%%%%%%%%%

\subsection{Field data example}

\inputdir{masha2}

\plot{elf-fmg2}{width=\columnwidth}{(a) Seismic image from North Sea
obtained by prestack time migration using velocity continuation
(Fomel, 2003). (b) Corresponding time-migration velocity.}

\multiplot{2}{vels,elfvxz}{width=0.8\textwidth}{Field data example of interval
velocity estimation. (a) Dix velocity
converted to depth. (b) Estimated velocity model and the corresponding
image rays. The image-ray spreading causes significant differences
between Dix velocity and true velocity.}

\multiplot{3}{img0,img,fmg2}{width=0.8\textwidth}{Migrated images of
the field data example. (a) Poststack migration using Dix
velocity. (b) Poststack migration using estimated velocity.  (c)
Prestack time migration converted to depth with our algorithm.}

Figure~\ref{fig:elf-fmg2}, taken from \cite[]{fomel2003}, shows a
prestack time migrated image from the North Sea and the corresponding
time-migration velocity obtained by velocity continuation. The most
prominent feature in the image is a salt body which causes significant
lateral variations of velocity. Figure~\ref{fig:vels,elfvxz} compares
the Dix velocity converted to depth with the interval velocity model
recovered by our method. As in the synthetic example, there is a
significant difference between the two velocity caused by the
geometrical spreading of image rays. The middle part of the velocity
model may not be recovered properly. The true structure should include
a salt body visible in the image. The inability of our method to
recover it exactly shows the limitation of the proposed approach in the areas
of significant lateral velocity variations, which invalidate the
assumptions behind time migration
\cite[]{robein}. Figure~\ref{fig:img0,img,fmg2} compares three images:
post-stack depth-migration image using Dix velocity, post-stack
depth-migration image using the velocity estimated by our method, and
prestack time-migration image converted to depth with our
algorithm. The evident structural improvements in Figure~\ref{fig:img}
in comparison with Figure~\ref{fig:img0}\new{, in particular near salt
  flanks,} and a good structural agreement between
Figures~\ref{fig:img} and~\ref{fig:fmg2} serve as an indirect evidence
of the algorithm success. An ultimate validation should come from
prestack depth migration velocity analysis, which is significantly
more expensive.

\section{Conclusions}

We have applied the recently established theorem that the Dix velocity
obtainable from the time-migration velocity is the true interval
velocity divided by the geometrical spreading of image rays to pose
the corresponding inverse problem. We have suggested a set of
numerical algorithms for solving the problem numerically. We have
tested these algorithms on a synthetic data example with laterally
heterogeneous velocity and demonstrated that they produce
significantly better results than simple Dix inversion followed by
time-to-depth conversion. Moreover, the Dix velocity may qualitatively
differ from the output velocity. We have also tested our algorithm on a
field data example and validated it by comparing prestack time
migration image mapped to depth with post-stack depth migrated
images. Our approach is complementary to velocity estimation methods
that work directly in the depth domain\old{ and}. \new{Therefore, it}
can serve as an efficient first step in seismic velocity model
building.

\bibliographystyle{seg}
\bibliography{Qpde,t2d}

