\title{Plane-wave Sobel attribute for discontinuity enhancement in seismic images}
\author{Mason Phillips and Sergey Fomel}
\address{The University of Texas at Austin}

\lefthead{Phillips \& Fomel}
\righthead{Plane-wave Sobel attribute}

\maketitle
\begin{abstract}
One of the major challenges of interpretating seismic images is the delineation of reflection discontinuities that are related to geologic features, such as faults, channels, salt boundaries, and unconformities. 
Visually prominent reflection features often overshadow these subtle discontinuous features which are critical to understanding the structural and depositional environment of the subsurface. 
For this reason, precise manual interpretation of these reflection discontinuities in seismic images can be tedious and time-consuming, especially when data quality is poor. 
Discontinuity enhancement attributes are commonly used to facilitate the interpretation process by enhancing edges in seismic images and providing a quantitative measure of the significance of discontinuous features. 
These attributes require careful pre-processing to maintain geologic features and suppress acquisition and processing artifacts which may be artificially detected as a geologic edge. 

In this paper, we propose the plane-wave Sobel attribute, a modification of the classic Sobel filter, by orienting the filter along seismic structures using plane-wave destruction and plane-wave shaping to compute an enhanced discontinuity attribute. 
The plane-wave Sobel attribute can be applied directly to a seismic image to efficiently and effectively enhance discontinuous features, or to a coherence image to create a sharper and more detailed image. 
We demonstrate the effectiveness of this method by applying it to two field data examples with many faults and channel features from offshore New Zealand and offshore Nova Scotia and compare the results to other coherence attributes.
\end{abstract}

\section{Introduction}
Discontinuity enhancement attributes are among the most widely used seismic attributes today. 
These attributes are generally post-stack image domain calculations of the similarity or dissimilarity along a horizon or time-slice between a neighborhood of adjacent seismic traces. 
Discontinuous features, such as faults, channels, salt boundaries, unconformities, mass-transport complexes, and subtle stratigraphic features can be identified as area of low similarity. 
Such attributes are powerful interpretation tools that make detailed interpretation of previously indistinguishable features possible.

\cite{bahorich} proposed the first celebrated discontinuity enhancement attribute and coined the term ``coherence". 
It produces images of the normalized local cross-correlations between adjacent seismic traces and combines them to estimate coherence. 
This algorithm provided the framework for semblance, the generalization to an arbitrary number of traces, proposed by \cite{marfurt98}. 
Using multidimensional correlation, this approach provides better vertical resolution.
Both of these correlation-based methods are sensitive to lateral amplitude variations, which may obscure features such as faults and channels.

The local covariance matrix measures the uniformity of a seismic image in each dimension. 
Decomposing this matrix into its eigenvectors and eigenvalues provides a quantitative measure of local variations of seismic structures. 
\cite{gersztenkorn} propose to compute the ratio of the largest eigenvalue and the sum of all eigenvalues of the covariance matrix at each sample, highlighting areas where there is no dominant texture in the seismic image.
This implementation is commonly called ``eigenstructure coherence" and is only sensitive to lateral changes in phase.
A similar decomposition can be applied to the structure-tensor.
The structure-tensor measures the local covariance of the image in each dimension.
Local linearity and planarity can be computed from the eigenvalues of the structure-tensor \cite[]{randen00,randen01,bakker}.
\cite{wu17} proposes to modify the traditional structure-tensor decomposition by orienting the image gradient along seismic structures.
Discontinuous features are highlighted further by smoothing along discontinuities.

Information about reflection dip in seismic images allow filters to be oriented along seismic reflections.
Variance is a simple, but effective attribute which highlights unpredictable signal associated with discontinuous features.
The local variance calculation is oriented along structure using the eigenvectors of the structure-tensor \cite[]{randen01}.
\cite{hale09} also orients semblance along seismic reflections using the eigenvectors of the structure-tensor, and additionally smooths along directions perpendicular to the reflections to provide an enhanced image.
\cite{karimi} uses plane-wave destruction to paint multiple predictions of local structures in seismic images.
The difference between the predicted and real data provides an image with isolated discontinuities.

To compute a discontinuity enhancement image for detection and extraction of fault surfaces, semblance can be computed along fault strike and dip orientations.
\cite{cohen} use the normalized differential entropy attribute to enhance faults.
Local fault planes are separated and extracted using an adaptive image-binarization-and-skeletonization algorithm.
This method effectively extracts fault surfaces by segmenting the coherence image.
\cite{hale13} and \cite{wu16a} propose to scan through fault strikes and dips to maximize the semblance attribute.
Fault surfaces are constructed by picking along the ridges of the liklihood attribute.
Additionally, images can be unfaulted by estimating fault throws by correlating seismic reflections across fault surfaces \cite[]{wu16b}.

The traditional Sobel filter can be used to efficiently compute an image with enhanced discontinuities \cite[]{sobel}.
The Sobel filter is an edge detector which computes an approximation of the gradient of the image intensity function at each point by convolving the data with a zero-phase discrete differential operator and a perpendicular triangular smoothing filter.
This 2D filter is small and integer-valued in each direction, making it computationally inexpensive to apply to images \cite[]{ogorman}.
\cite{luo} first proposed the applications of Sobel filters to seismic images.
Since, modifications of the Sobel filter have been proposed for edge detection in seismic images by orienting the filter along local slopes estimated by maximizing local cross-correlation and dynamically adapting the size of the filter based on local frequency content \cite[]{aqrawi11a,aqrawi11b,aqrawi14}.
Dip-oriented Sobel filters can be applied directly to a seismic image to compute an image with enhanced edges, or to coherence images to further sharpen previously enhanced edges \cite[]{chopra14}.

We propose to modify the Sobel filter to explicitly follow seismic structures. 
\cite{phillips} modify the Sobel filter by replacing the discrete differential operator with linear plane-wave destruction \cite[]{fomel02} and triangular smoothing with plane-wave shaping \cite[]{fomel07,swindeman}.
This method is particularly efficient because it does not require computation of the eigenvectors of the covariance matrix or structure-tensor.
Local slopes are instead estimated using accelerated plane-wave destruction \cite[]{apwd}.
We further modify the Sobel filter by orienting the filter along the azimuth perpendicular to discontinuities by following the azimuth scanning workflow proposed by \cite{merzlikin}.
We test our modification on 3{D} seismic images from offshore New Zealand and Nova Scotia, Canada and compare the results with those from coherence attributes.

\section{Theory}
The traditional Sobel operator approximates the smoothed gradient of the image intensity function. 
It is defined as the convolution of an image with two 3$\times$3 filters.
The first of these filters ($S_i$) differentiates in the inline direction and smooths in the crossline direction.
The second filter ($S_x$) differentiates in the crossline direction and smooths in the inline direction.
\begin{equation}
S_i=\left[
\begin{array}{rrr}
-1 & 0 & 1 \\
-2 & 0 & 2 \\
-1 & 0 & 1
\end{array}
\right]=\left[
\begin{array}{c}
1 \\
2 \\
1
\end{array}
\right]\left[
\begin{array}{ccc}
-1 & 0 & 1
\end{array}
\right]
\end{equation}

\begin{equation}
S_x=S_i^T=\left[
\begin{array}{rrr}
-1 & -2 & -1 \\
0 & 0 & 0 \\
1 & 2 & 1
\end{array}
\right]
\end{equation}

In the $Z$-transform notation, filters (1) and (2) are expressed as

\begin{equation}
\begin{array}{l}
S_i(Z_i,Z_x)=(Z_x+2+Z_x^{-1})(Z_i-Z_i^{-1}) \\
S_x(Z_i,Z_x)=(Z_i+2+Z_i^{-1})(Z_x-Z_x^{-1})
\end{array}
\end{equation}

where $Z_j$ is a phase shift in the $j$ direction.

The inline and crossline images are combined to approximate the magnitude of the image gradient \cite[]{chopra07} where $\mathbf{d}$ is the data and $\mathbf{S}_i$ and $\mathbf{S}_x$ are convolution operators with the filters $S_i$ and $S_x$, respectively.
\begin{equation}
\|\mathbf{\nabla}\mathbf{d}\|\approx \sqrt{(\mathbf{S}_i\mathbf{d})^2+(\mathbf{S}_x\mathbf{d})^2}
\end{equation}

We propose to modify the filter for application to 3D seismic images by orienting the filter along the structure of seismic reflectors.
Local slopes of seismic reflections are estimated in the inline and crossline directions using accelerated plane-wave destruction filters \cite[]{apwd}.
The local-plane	wave model assumes seismic traces can be effectively predicted by dynamically shifting adjacent seismic traces.
This dynamic shift corresponds to the dip of seismic reflections.
This model is useful for seismic data characterization and is the basis for plane-wave destruction filters.
The local plane-wave differential equation is defined by \cite{claerbout} as

\begin{equation}
\frac{\partial u}{\partial x}+p\frac{\partial u}{\partial t} = 0 \ ,
\end{equation}

where $u$ is the seismic wavefield and $p$ is the temporally and spatially variable local slope.
The optimal local slopes in the inline	and crossline directions are determined	by minimizing the regularized plane-wave residual \cite[]{fomel02}.

We further modify the Sobel filter by replacing the derivative operation with plane-wave destruction \cite[]{fomel02} and the smoothing operation with plane-wave shaping \cite[]{fomel07,swindeman}.
High order plane-wave destruction filters are described in the $Z$-transform notation as

\begin{equation}
\begin{array}{l}
C_i(p)=B(p,Z_x^{-1})-Z_iB(p,Z_x)\\
C_x(q)=B(q,Z_i^{-1})-Z_xB(q,Z_i)
\end{array}
\end{equation}

where $C$ is the plane-wave destruction filter, $B$ is an all-pass filter, and $p$ and $q$ are the local slopes in the inline and crossline directions.

Inline and crossline shaping filters ($T_i$ and $T_x$) are applied to the crossline and inline plane-wave destruction images, respectively.
Thus, the plane-wave Sobel filter modifies equation (3) to

\begin{equation}
\begin{array}{l}
S_i(p,q)=T_x(q)C_i(p)\\
S_x(p,q)=T_i(p)C_x(q)
\end{array}
\end{equation}

In the conventional implementation, the inline and crossline images are combined to produce the final image (equation 4).
We proposed an alternative approach based on the efficient azimuth scanning workflow proposed by \cite{merzlikin}. 
We scan through a window of azimuths and compute the absolute value of linear combinations of the inline and crossline images weighted by the sine and cosine of the azimuth.
The azimuth which produces the best image at each point is picked on a semblance-like panel using a regularized automatic picking algorithm \cite[]{fomel09}.
The ensemble of images is then sliced using the pick to generate the optimal image

\begin{equation}
\max_\alpha\|\mathbf{S}_i(p,q)\mathbf{d}\cos\alpha+\mathbf{S}_x(p,q)\mathbf{d}\sin\alpha\|^2
\end{equation}

where $\alpha$ is the azimuth of discontinuous features in the image and $\mathbf{S}_i(p,q)$ and $\mathbf{S}_x(p,q)$ correspond to convolution operators with the filters $S_i(p,q)$ and $S_x(p,q)$, respectively. This improves the resolution of discontinuous features by effectively orienting the plane-wave Sobel filter perpendicular to edges in the seismic image.

\section{Example I}
Our first example is a subset of the Parihaka seismic data (full-stack, anisotropic, Kirchhoff prestack time migrated).
This marine 3{D} seismic volume was acquired offshore New Zealand and is available on the SEG open data repository.

\inputdir{pari}
\plot{sub}{width=\columnwidth}{(a) The Parihaka seismic data.}

This image contains complex geologic structures, including multiple generations of faulting, meandering channel systems, and prominent unconformities.
We focus on a particularly interesting time-slice (1.311 s) containing many faults and channels (Figure~\ref{fig:sub}).

We first apply the traditional Sobel filter.
This attribute enhances discontinuous geologic features, but also enhances dipping reflectors (Figure~\ref{fig:flat}).

In order to optimally enhance discontinuous features, it is important to orient the filter along local slopes.
We compute the structural dip in the inline (Figure~\ref{fig:idip}) and crossline (Figure~\ref{fig:xdip}) directions using accelerated plane-wave destruction \cite[]{apwd}.
Using the local slopes, we apply structure-oriented smoothing to enhance seismic structures and attenuate noise without blurring geologic edges \cite[]{liu}.
We subsequently apply the proposed plane-wave Sobel filter and compute the magnitude of the inline and crossline plane-wave Sobel images.
Discontinuous geologic features, most prominently faults and channels, are enhanced, revealing subtle details which would be difficult to interpret from the original seismic image (Figure 3b).

We compute a more segmented image by cascading another iteration of filtering to the Sobel image, this time orienting the filter along both the dip of seismic reflections and the azimuth of the faults and channels.
The plane-wave Sobel filter is applied to the Sobel filter image using structural information derived from the original seismic image.
We compute linear combinations of these inline and crossline images weighted by the cosine and sine of the azimuth.
The azimuth of the faults and channels correspond to the orientation which creates the optimal image at each point (Figure~\ref{fig:az}).
Faults and channels are further segmented in the cascaded image by orienting the Sobel filter along geologic structures (Figure~\ref{fig:slice}).

\multiplot{3}{idip,xdip,az}{width=0.45\columnwidth}{(a) Inline and (b) crossline reflection slopes computed using accelerated plane-wave destruction and (c) azimuth of faults and channels estimated using azimuthal plane-wave destruction.}
\multiplot{3}{flat,sobel,slice}{width=0.45\columnwidth}{(a) The traditional Sobel filter, (b) proposed plane-wave Sobel filter, and (c) cascaded plane-wave Sobel filter applied to the Parihaka seismic data.}

\section{Example II}
The next example is a subset of the Penobscot data used previously by \cite{kington} (Figure~\ref{fig:pen}).
This small marine 3{D} seismic volume was acquired offshore Nova Scotia, Canada and is available on the SEG open data repository.

\inputdir{penobscot}
\plot{pen}{width=\columnwidth}{Penobscot 3D seismic data}

We apply cross-correlation coherence, semblance, eigenstructure coherence, gradient-structure-tensor (GST) coherence, and predictive coherence attributes to the data and compare the results to the proposed plane-wave Sobel attribute. 
Correlation-based coherence attributes \cite[]{bahorich} produce an image of normalized local cross-correlation between adjacent seismic traces and combines them to estimate coherence.
This attribute is more efficient than most coherence attributes, but lacks robustness and has poor vertical resolution (Figure~\ref{fig:coh0}).
Semblance \cite[]{marfurt98} provides better vertical resolution by incorporating a local window of traces (Figure~\ref{fig:coh1}).
Eigendecomposition of the local covariance matrix \cite[]{gersztenkorn} or gradient-structure-tensor \cite[]{randen00} allows information about local structures to be incorporated into the coherence calculation.
These attributes provide significantly better vertical and lateral resolution (Figures~\ref{fig:coh2} and \ref{fig:coh}) compared to correlation-based coherence; however, calculation and decomposotion of the local covariance matrix or gradient-structure-tensor introduces significant computational cost.
Further, all of these attributes contain significant noise contamination in coherent sections of the image.
Predictive coherence \cite[]{karimi} uses plane-wave destruction to compute residuals between adjacent traces predicted by painting along local slopes.
Discontinuities are enhanced in this image with minimal noise contamination compared to the previous attributes (Figure~\ref{fig:pcoh}).

We compare these results to the Sobel attributes.
As expected, the proposed plane-wave Sobel filter enhances the faults and channels in the seismic image without significant noise contamination or highlighting dipping reflectors (Figure 5f).

\multiplot*{6}{coh0,coh1,coh2,coh,pcoh,sobel}{width=\columnwidth}{Comparison of discontinuity enhancement attributes: (a) cross-correlation, (b) semblance, (c) eigenstructure, (d) gradient-structure-tensor, and (e) predictive coherence, and the (f) plane-wave Sobel filter.}

\end{comment}

\section{Conclusions}
We have modified the Sobel filter by orienting it along the dip of seismic reflections and the azimuth of discontinuous features.
We find that the proposed plane-wave Sobel filter is a straightforward and inexpensive means for enhancing discontinuous features in 3{D} seismic images. 
Many popular coherence attributes come with significant computational cost because they require calculation and eigendecomposition of the local covariance matrix or structure tensor at each point in the 3{D} image.
The significant cost of eigendecomposition can be partially alleviated in practice by parallelization.
One of the key benefits of this method is its superior efficiency in comparison with other similar attributes.
The main costs of this attribute are the estimation of local slopes and azimuth scanning.
Local slopes can be estimated efficiently using accelerated plane-wave destruction and azimuth scanning is easy to parallelize.
As demonstrated in this paper, the proposed plane-wave Sobel attribute can help expedite and improve geological interpretations of subsurface faults and channels.
This attribute can likely be used to enhance other discontinuous or chaotic features commonly interpreted in seismic images, such as unconformities, salt boundaries, and mass transport complexes.

\section{Acknowledgements}
We thank Nova Scotia Department of Energy, Canada Nova Scotia Offshore Petroleum Board, and New Zealand Petroleum and Minerals for providing data used in this paper to the SEG open data repository.
We thank sponsors of the Texas Consortium for Computational Seismology (TCCS) for financial support.
We thank Xinming Wu for helpful comments and discussion.
The computational examples reported in this paper are reproducible using the Madagascar open-source software package \cite[]{fomel13}.

%\onecolumn
\bibliographystyle{seg}
\bibliography{paper,SEP2}
