\append{Annotated Parameter Files}

All IWAVE applications are parameter-driven: that is, they accept as
input a {\em map} or associative array, defined by 
a list of {\tt key = value} pairs. These parameter specifications can
be included on the command line. However, because the number of such
parameter specifications is rather large, it's convenient to store
them in a parameter file (``par file''). The use of a par file has the
added advantage that the file may include annotations and white space to
improve readability. 

The examples displayed in this paper are created in the directory {\tt
  \$TOP/demo/data}. The par file {\tt parfile} is a by-product of data
creation - the SConstruct script text-processes it from prototype
files including macros, which are resolved when the scripts are
run. Four such prototype par files are present in {\tt data}, each one defining
a modeling task corresponding to a given level of grid refinement. 
The actual input to the modeling command is {\tt parfile}. 

The meaning of each parameter in the par file is described in the IWAVE
web documentation \cite[]{IWAVE}.
This appendix gives a brief description of the parameter assignments
appearing in the {\tt parfile} generated for the 20 m grid example. To
run this example, and coincidentally generate its parameter file,
\begin{itemize}
\item {\tt cd \$TOP/demo/data}
\item {\tt scons demo20m}
\end{itemize}

The file {\tt parfile} groups job parameters into blocks. The first
block looks like this:
\begin{verbatim}
INPUT DATA FOR iwave

------------------------------------------------------------------------
FD:

         order = 2          spatial half-order
           cfl = 0.4        proportion of max dt/dx
          cmin = 1.0
          cmax = 4.5
          dmin = 0.5
          dmax = 5.0
         fpeak = 0.010      central frequency

\end{verbatim}
Note that comments, block labels, and typographical separators are all
accommodated. The IWAVE parameter parser identifies parameter
specifications by strings of the form 
\begin{verbatim}
          key = value
\end{verbatim}
consisting of a string with no embedded whitespace, followed by an
{\tt =} sign surrounded by any amount of whitespace on either side,
followed by another string with no embedded whitespace. Strings with embedded
whitespace are also allowed, provided that they are double-quoted -
thus {\tt "this is a value"} is a legitimate value expression. Other
capabilities of the parser are described in its html
documentation. All values are first read as strings, then
converted to other types as required.

The parameters appearing in {\tt parfile} are as follows:

\begin{itemize}
\item {\tt order = 2}: half-order of the spatial difference scheme -
  {\tt asg} implements schemes of order 2 in time, and 2k in space,
  for certain values of k, the spatial half-order, which is the value
  associated to the key {\tt order}. Permissible values
  of {\tt order} in the current release are 1, 2, and 4.
\item {\tt cfl = 0.4}: max time step is computed using one osf several
  criteria - see html docs for details. This number is the fraction of
  the max step used. Must lie between 0.0 and 1.0.
\item{\tt cmin, cmax, dmin, dmax}: sanity checks on density and
  velocity values. The max permitted velocity {\tt cmax} also figures
  in two of the max time step criteria. Violation of these bounds
  causes an informative error message with traceback information to be
  written to the output file {\tt cout[rk].txt}, where {\tt rk} is the
  MPI global rank ( = 0 for serial execution), and the program to
  exit. IWAVE handles all trappable fatal errors in this way.
\item {\tt fpeak = 0.010}: nominal central frequency, in kHz. Used in
  two ways: (1) to set the width of absorbing boundary layers by
  defining a wavelength at max velocity {\tt cmax}, and (2) as the
  center frequency of a Ricker wavelet in case the point source Ricker
  option is chosen for source generation. Plays no other role.
\item {\tt nl1, nr1, nl2, nr2}: specify 2D PML layer thicknesses: {\tt
    nl1} describes the layer thickness, in wavelengths determined as
  described in the preceding bullet, of the {\em left} boundary (with
  lesser coordinate) in the axis 1 direction, etc. Set = 0.0 for no
  layer, in which case the free (pressure-release) boundary condition
  is applied.
\item {\tt srctype = point}: this application implements two source
  representations, a point source with amplitude options and a very flexible array
  source option.
\item{\tt "source  = ...."}: a quoted parameter spec is just a
  string, from the IWAVE parser's point of view, so does not define
  anything: this parameter is commented out. If it were not quoted, it
  would define the pathname to an SU file containing source data -
  either a  wavelet (first trace) for point source, or an array source specified
  by a number of traces (if {\tt srctype=array}). If a source wavelet is not specified (as
  it is not for the scripted examples), the application creates a Ricker wavelet of central
  frequency {\tt fpeak}.
\item{\tt sampord = 0}: order of spatial interpolation. Legal values
  are 0 and 1. 0 signifies rounding down the source coordinates to the
  nearest gridpoint with smaller coordinates. 1 signifies piecewise
  multilinear interpolation (or adjoint interpolation, for the
  source), so that a point source at ${\bf x}_s$ is
  represented as a convex linear combination of point sources at
  the corners of the grid cube in which ${\bf x}_s$ lies, and receiver values
  are similar convex combinations of nearby grid function values. The first option is
  appropriate for synthetic examples in which sources and receivers
  lie on the grid. The second permits arbitrary placement of sources
  and receivers, and is compatible with the overall second-order
  accuracy of {\tt asg}.
\item {\tt refdist = 1000.0, refamp = 1.0}: point source calibration
  rule developed for the SEAM project - the wavelet is
  adjusted to produce the pulse shape read from the file specified by
  the {\tt source} parameter, or a Ricker wavelet of central frequency
  {\tt fpeak} if the {\tt source} parameter is not assigned a value,
  with amplitude (in GPa) given by {\tt refamp} at the
  prescribed distance (in m) given by {\tt refdist}, assuming a homogenous medium with
  parameters the same as those at the source point, and absorbing
  boundary conditions. If {\tt refdist} is set to $0.0$, then source
  pulse (either read from a file, if {\tt source} is set, or a Ricker
  of peak frequency {\tt fpeak} otherwise) is simply used as the time
  function in the discrete point source radiator. 
\item {\tt hdrfile = ...}: IWAVE specifies acquisition parameters
  such as source and receiver locations, time sample rates and delays,
  and so on, by supplying trace headers in a file: the traces produced
  in simulation have the same headers. At present, the only
  implemented option for specifying headers is via a path to an SU
  file, that is, a SEGY-formatted file with reel header stripped
  off. Other options are planned for future releases.
\item {\tt datafile = ...}: pathname for output data file; on normal
  completion of run, contains traces with
  same headers as in {\tt hdrfile}, computed trace samples. Note that
  sample rate of output traces is whatever is specified in {\tt
    hdrfile}, and generally is not the same as the time step used in
  the simulation, the trace samples being resampled on output. Note
  also that pathnames may be either fully qualified (as in the {\tt hdrfile}
  entry) or relative.
\item {\tt velocity = ..., density = ...}: pathnames to rsf header
  files for velocity and density. Other combinations of physical
  parameters are admissible, such as bulk modulus and density, bulk
  modulus and bouyancy (reciprocal density), velocity and
  buoyancy. Data stored in RSF disk format, described in Madagascar
  web documentation. Current proxy for unit conversions: scaling
  during read/write by power of 10, given by {\tt scale} keyword
  (extension to standard RSF). Must be chosen so that output is in
  m/ms or km/s for velocity, g/cm$^3$ for density, or compatible units
  for other parameters (eg. GPa for bulk modulus).
\item {\tt mpi\_np1 = ..., partask = 1}: parallelism parameters - {\tt
    mpi\_np1} gives the number of domains along axis 1, etc. (loop or
  domain decomposition), and
  {\tt partask} gives the number of shots to load simultaneously
  (task parallelization over shots). Domain decomposition and task
  parallelization may be used alone or in combination. Setting the
  value = 1 for all of these parameters signifies serial execution, even if the
  code is compiled with MPI. To execute in parallel, compilation with
  MPI is a precondition - see installation instructions.
\item {\tt dump\_pi = ...}: dump parameters regulate verbosity, with
  output being sent to text files {\tt cout0.txt} (serial) or {\tt
    cout[rk].txt}, {\tt rk} = MPI global rank encoded with uniform field width for
  parallel execution. Individual parameters described in the html
  documentation. If all dump parameters are set to zero, {\tt asg} is
  silent, i.e. all {\tt cout...} files will be empty on completion. 
\end{itemize}

The parameters described here represent one common use case of IWAVE's
acoustic application. The web documentation describes a number of
other use cases.

\append{Downloading and Installing IWAVE}

Download and installation instructions are available on the IWAVE web
site \cite[]{IWAVE}. In brief,
\begin{itemize}
\item The primary source for IWAVE is the SourceForge Subversion repository for
  Madagascar. To download IWAVE alone,
\begin{verbatim}
svn co https://rsf.svn.sourceforge.net/svnroot/rsf/trunk/iwave $TOP
\end{verbatim}
where {\tt \$TOP} is the full or relative pathname under which you
wish IWAVE source to be installed. To download the entire Madagascar
package (development version), simply leave off {\tt /iwave} in the
above URL. 
\item if a repository download is not possible for some reason,
  gzipped tar files are available. For the latest development version
  of IWAVE only, download via the link
\begin{verbatim}
http://www.trip.caam.rice.edu/software/iwave-dev.tar.gz
\end{verbatim}
The Madagascar web site includes a link for
download of the latest stable release of Madagascar as a gzipped
tar archive. At some point this stable release will include IWAVE.
\item to install with default options, 
\begin{itemize}
\item {\tt cd \$TOP}
\item {\tt scons}
\end{itemize}
\item to install with more agressively optimized compilation, create a
  configuration file per instructions in \cite[]{IWAVE}, and
  recompile.
\end{itemize}
