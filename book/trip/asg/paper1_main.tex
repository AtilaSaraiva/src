\begin{abstract}
  IWAVE is a framework for time-domain regular grid finite difference
  and finite element methods. The IWAVE package includes
  source code for infrastructure component, and 
  implementations of several wave physics modeling categories. This
  paper presents two sets of examples using IWAVE acoustic staggered grid
  modeling. The first set illustrates the effectiveness of a simple
  version of Perfectly Matched Layer absorbing
  boundary conditions. The second set reproduce illustrations from a  recent paper on error propagation for
  heterogeneous medium simulation using finite differences, and
  demostrate the interface error effect
  which renders all FD methods effectively first-order accurate.  
  The source code for these examples is packaged with the paper
  source, and supports the user in duplicating the results presented
  here and using IWAVE in other settings.
\end{abstract}

\section{Introduction}
Domain-specific simulation such as seismic modeling begs for
software re-use via modular design. All applications of this type have
the same structure: static fields are initialized, dynamic fields
updated, output extracted. A modular approach to code architecture is
implicit in this structure, and further specialization leads to even more opportunity
for code re-use via modular design. 
%The general topic of this paper, time-domain
%simulation on regular rectangular grids, presents specific openings
%for code re-use via abstraction: time steps may have internal
%structure which is repeated in a defined pattern, parallelism via
%domain decomposition may be computed from stencil shape information
%rather than hard-wired into the code, sampling rules for output
%buffering are simple to formulate and universal, and external data
%formats may be hidden behind uniform interfaces.

IWAVE is open source
software for finite difference or finite element time-domain
simulation on regular rectangular grids, written exclusively in the
C/C++. IWAVE is built around a core framework: that is,
a collection of separate software packages which together provide
essential services upon which applications may be built. These
service components completely define the interfaces to which
additional code must be written to formulate a complete
application. The core framework defines
\begin{itemize}
\item parameter-driven job control;
\item grid generation and memory allocation in 1D, 2D, and 3D space;
\item serial, loop-parallel, and task-parallel execution models,
scaling to thousands of threads; 
\item arbitrary source and receiver locations, and flexible
source specification including simultaneous source modeling (random,
plane-wave,...)
\item standard input and output data formats (SEGY, RSF)
\item predefined support for linearized (Born) modeling and adjoint
  linearized (RTM) modeling, both first and second order;
\item uniform interface to optimization and linear algebra for
  creation of inversion applications via the
  Rice Vector Library (``RVL'') \cite[]{RVLTOMS,GeoPros:11}.
\end{itemize}
\cite{trip14:struct} describes the desgn principles
underlying the IWAVE core framework, and illustrates the construction
of a complete acoustic modeling application using centered finite
differences for the second order acoustic constant denstity wave equation.

The primary purpose of this short paper is to illustrate 
synthetic seismogram generation using another finite
difference scheme implemented in IWAVE, the staggered grid approximation to
variable-density velocity-pressure acoustodynamics \cite[]{Vir:84}. 
Exactly the same framework supports this application as was described
in \cite{trip14:struct}; as explained there, only two data structures
and six principal functions need be defined to implement this (or any) finite
difference method in IWAVE. 

The examples illustrate two aspects of finite difference modeling.
The IWAVE staggered grid implementation includes a version of PML
absorbing boundary conditions \cite[]{Habashy:07}, permitting accurate
finite grid approximation of wave propagation in a full- or
half-space. The first set of examples demonstrates the effectiveness
of these very simple PML conditions.
The second set reproduce the examples presented in 
\cite{SymesVdovina:09}, and illustrate a fundamental limitation in the
use of straightforward finite-difference methods for modeling waves in
heterogeneous media.

IWAVE was used in a quality control role in the SEAM Phase I project -
see \cite{FehlerKeliher:11} for an account, including discussion of
the many difficulties of large scale numerical simulation of seismograms.

The internal details of IWAVE are not discussed here, except insofar
as is necessary to explain the use of the main
commands. As mentioned above, \cite{trip14:struct} overviews the design
of IWAVE and the main features of its internal
structure, and defines the elements necessary to compile a new IWAVE application. \cite{GeoPros:11} briefly describe the IWAVE/RVL
mechanisms for coupling modeling with optimization packages
to produce inversion applications.

The paper begins with a brief review of the system of partial
differential equations solved (approximately) by IWAVE's acoustic
application, and the choice of finite difference method. The next
section evaluates the effectiveness of the PML absorbing boundary
conditions included in the IWAVE staggered grid acoustic application.
The following section presents the examples of \cite{SymesVdovina:09},
along with some additional examples based on the same distribution of
mechanical parameters which shed light on the impact of finite
difference order on solution accuracy. Instructions follow for recreating
these examples, and for using them as starting points for further
modeling exercises. The paper ends with a brief discussion of the
prospects for improvements in performance and accuracy in FD
technology, and the evolutionary advantages flowing from the modular,
or object, orientation of IWAVE. Two appendices describe the job
parameters used in the examples, and download and install
instructions.
 
\section{Acoustodynamics}
The pressure-velocity form of
acoustodynamics consists of two coupled first-order partial
differential equations:
\begin{eqnarray}
\label{awe}
\rho \frac{\partial {\bf v}}{\partial t} &=& - \nabla p \\
\frac{1}{\kappa}\frac{\partial p}{\partial t} &=& -\nabla \cdot {\bf v} + g
\end{eqnarray}
In these equations, $p({\bf x},t)$ is the pressure (excess, relative to an
ambient equilibrium pressure), ${\bf v}({\bf x},t)$ is the particle velocity,
$\rho({\bf x})$ and $\kappa({\bf x})$ are the density and bulk modulus
respectively. Bold-faced symbols denote vectors; the above formulation
applies in 1, 2, or 3D.

The inhomogeneous term $g$ represents externally supplied energy (a
``source''), via a defect in the acoustic constitutive relation. A
typical example is the {\em isotropic point source}
\[
g({\bf x},t) = w(t) \delta({\bf x}-{\bf x}_s)
\]
at source location ${\bf x}_s$.

\cite{Vir:84} introduced finite difference methods based on this
formulation of acoustodynamics to the active source seismic
community. \cite{Vir:86} extended the technique to elastodynamics, and
\cite{Lev:88} demonstrated the use of higher (than second) order
difference formulas and the consequent improvement in dispersion
error. Many further developments are described in the review paper
\cite{moczoetal:06}. IWAVE's acoustic application uses the principles introduced by
these authors to offer a suite of finite difference schemes, all
second order in time and of various orders of accuracy in space.

The bulk modulus and buoyancy (reciprocal density) are the natural
parameters whose grid samplings appear in the difference
formulae. These are the parameters displayed in the figures below,
rather than, say, velocity and density, which might seem more natural.

\section{PML Effectiveness}

The IWAVE acoustic staggered grid scheme implements the Perfectly
Matched Layer (PML) approach to absorbing boundary conditions, in one of the
simpler of its many guises (a split field approach -
\cite[]{Habashy:07}). After some manipulation, the acoustic PML system
for the physical velocity $\bv$ and an artificial vector pressure
$\bp$ takes the form
\begin{eqnarray}
\label{eqn:pml}
\rho \left(\frac{\partial v_k}{\partial t} + \eta_k(x_k)v_k\right) &=&
- \frac{\partial p_k}{\partial x_k}, \nonumber \\
\frac{1}{\kappa}\left(\frac{\partial p_k}{\partial t} + \eta_k(x_k)p_k\right) &=& -\nabla \cdot {\bf v} + g
\end{eqnarray}
in which the $k$th component of the attenuation profile vector ${\bf \eta}$ depends
only on $x_k$, and can be stored as a collection of 1D objects. Ordinary acoustic
wave propagation takes place where ${\bf \eta}={\bf 0}$, and if the
components of the vector pressure $\bp$ are all the same in this zone,
then they remain the same there, and any one of them may be regarded
as the same as the physical pressure field. Outside of the physical
domain, where waves are to be attenuated, ${\bf {\eta}}$ should
ibe positive; at the boundary of the physical domain, it should vanish to positve order. We
elected to make ${\bf {\eta}}$ cubic in distance to the boundary: for
a PML layer of width $L_{k,r}$, beginning at $x_k=x_{k,r}$ along the $k$th
coordinate axes,
\[
\eta(x_k) = \eta_0 \left(\frac{x_k-x_{k,r}}{L_{k,r}}\right)^3
\]
Thus there are four PML boundary layer thicknesses in 2D, six in 3D, one
for each side of the simulation cube. The IWAVE convention imposes
pressure-free boundary conditions on the exterior boundary of the PML
domain. Thus $L=0$ signifies a free surface boundary face. Any face of
the boundary may be assigned a zero-pressure condition ($L=0$) or a
PML zone of any width ($L>0$).

Many implementations of PML, especially for elasticity, confine the
extra PML fields (in this case, the extra pressure variables) to
explicitly constructed zones around the boundary, and use the standard
physical system in the domain interior. We judged that for acoustics little would be
lost in either memory or efficiency, and much code bloat avoided, if
we were to solve the system (\ref{eqn:pml}) in the entire domain.

Considerable experience and some theory
\cite[]{Habashy:07,moczoetal:06} suggest that the system \ref{eqn:pml} will
effectively absorb waves that impinge on the boundary, emulating free
space in the exterior of the domain, if the PML zones outside the
physical domain in which ${\bf {\eta}}$ are roughly a half-wavelength
wide, and $\eta_0=0$.

A simple 2D example illustrates the performance of this type of
PML. The physical 
domain is a 1.8 x 7.6 km; the same domain is used in the experiments
reported in the next section. A point source is placed at $z$=40 m,
$x=3.3$ km, with a Gaussian derivative time dependence with peak
amplitude at about 5 Hz, and signifcant energy at 3 Hz but little below. The acoustic velocity is 1.5 km/s throughout
the domain, so the effective maximum wavelength is roughly 500 m. The
density is also constant, at 1 g/${\rm cm}^3$. A
snapshot of the wavefield at 1.2 s after source onsiet
(Figure \ref{fig:frame13}), before the wave has reached the boundary of the
domain, shows the expected circular wavefront. At 4.0 s, a simulation
with zero-pressure boundary conditions on all sides of the physical
domain produces the expected reflections, Figure \ref{fig:frame40-1}. With
PML zones of 250 m on the bottom and sides of the domain, so that only
the top is a zero-pressure surface, and $\eta_0=1$, the wave and its
free-surfacec ghost both appear to leave the domain
(Figure \ref{fig:frame40-2}, plotted on the same grey scale). The
maximum amplitude visible in Figure \ref{fig:frame40-1} is roughly
$7.1 \times 10^{-2}$, whereas the maximum amplitude in Figure
\ref{fig:frame40-2} is $7.0 \times 10^{-5}$. The actual reflection
coefficient is likely less than $10^{-3}$, as the 2D free space field
does not have a lacuna behind the wavefront, but decays smoothly, so
the low end of the wavelet spectrum remains.

It is not possible to decrease the PML layer thickness much beyond the
nominal longest half-wavelength and enjoy such small
reflections. Figure \ref{fig:frame40-3} shows the field at 4.0 s with
PML zones of width 100 m on bottom and sides, and an apparently
optimal choice of $\eta_0$. The maximum amplitude is $2.3 \times
10^{-4}$, and a reflected wave is clearly visible at the same grey scale.

\section{All FD Schemes are First Order in Heterogeneous Media}

The bulk modulus and buoyancy models
depicted in Figures \ref{fig:bm1} and \ref{fig:by1} embed an anticline or dome in an otherwise
undisturbed package of layers. These
figures display sampled versions of the models with $\Delta x = \Delta
z = $ 5 m; the model fields are actually given analytically, and can
be sampled at any spatial rate. The IWAVE utility $sfstandardmodel$ (in the
Madagascar {\tt bin} directory) builds this example and a number of
others that can be sampled arbitrarily for grid refinement
studies. See its self-doc for usage instructions.

\cite{SymesVdovina:09} use the model depicted in Figures \ref{fig:bm1} and \ref{fig:by1} to illustrate the {\em
  interface error} phenomenon: the tendency, first reported by
\cite{Brown:84}, of all finite difference schemes for wave
propagation to exhibit first order error, regardless of formal order,
for models with material parameter discontinuities. 
Figure \ref{fig:data1} exhibits a shot gather, computed with a (2,4) (= 2nd order in time,
4th order in space) staggered grid scheme, $\Delta
x = \Delta z = $ 5 m (more than 20 gridpoints per wavelength at the
wavelength corresponding to the highest frequency, 12 Hz, with
significant energy, and the smallest $v_p=1.5$ km/s)  and an appropriate near-optimal time step, acquisition geometry as described in
caption. The same gather computed at different spatial sample rates
seem identical, at first glance, however in fact the sample rate has a considerable effect. Figures
\ref{fig:trace} and \ref{fig:wtrace} compare traces computed from models sampled
at four different spatial rates (20 m to 2.5 m), with proportional
time steps. The scheme used is formally 2nd order
convergent like the original 2nd order scheme suggested by
\cite{Vir:84}, but has better dispersion suppression due to the use of
4th order spatial derivative approximation. Nonetheless,
the figures clearly show the first order error, in the form of a
grid-dependent time shift, predicted by \cite{Brown:84}. 

Generally, even higher order approximation of spatial derivatives
yields less dispersive propagation error, which dominates the finite
difference error for smoothly varying material models. For
discontinuous models, the dispersive component of error is still
improved by use of a higher order spatial derivative approximation,
but the first order interface error eventually dominates as the grids
are refined. Figure \ref{fig:data8k1} shows the same shot gather as
displayed earlier, with the same spatial and temporal sampling and
acquisition geometry, but computed via the (2,8) (8th order in space)
scheme. The two gather figures are difficult to disinguish. The trace
details (Figures \ref{fig:trace8k}, \ref{fig:wtrace8k}) show clearly
that while the coarse grid simulation is more accurate than the (2,4)
result, but the convergence rate stalls out to 1st order as the grid
is refined, and for fine grids the (2,4) and (2,8) schemes produce
very similar results: dispersion error has been suppressed, and what
remains is due to the presence of model discontinuities.

See
\cite{SymesVdovina:09} for more examples, analysis, and discussion,
also \cite{FehlerKeliher:11} for an account of consequences for quality control in
large-scale simulation.

Note that the finest (2.5 m) grid consists of roughly 10 million
gridpoints. Consequently the modeling runs collectively take a
considerable time, from a minutes to a substantial fraction of an hour
depending on platform,
on a single thread. This example is computationally large enough that
parallelism via domain decomposition is worthwhile. IWAVE is designed
from the ground up to support parallel computation; a companion report
will demonstrate parallel use of IWAVE.

\section{Creating the examples - running IWAVE applications}
IWAVE builds with SConstruct ({\tt http://www.scons.org}), either as an
independent package or as part of Madagascar
\cite[]{Madagascar}. See the Madagascar web site 
\begin{verbatim}
http://www.ahay.org/wiki/Main_Page
\end{verbatim}
for download and install
instructions. Source for IWAVE and other TRIP software reside in the {\tt
  trip} subdirectory of the top-level Madagascar source directory. A
{\tt README} file describes how to install TRIP software independently
of the rest of Madagascar, which is useful to configuring TRIP
differently from other parts of the package (for example, with MPI support). 

The IWAVE acoustic staggered grid modeling command is
{\tt sfasg} for the Madagascar build, stored in the Madagascar {\tt
  \$RSFROOT/bin} directory, or 
\begin{verbatim}
$RSFSRC/trip/iwave/asg/main/asg.x
\end{verbatim}
for the standalone build. All
IWAVE commands self-document: entering the command path prints a usage
statement to the terminal, including descriptions of all parameters.

The paper you are currently reading follows the reproducible research pattern described on
the Madagascar web site, using Madagascar reproducible research tools. You can find the LaTeX source in the
subdirectory {\tt book/trip/asg} of the Madagascar source directory,
and the script for building the data in 
\begin{verbatim}
$RSFSRC/book/trip/asg/project/SConstruct
\end{verbatim} 
This script, together with the
self-doc for the acoustic staggered grid command and the remarks in
the remainder of this section, should enable you to
build your own examples after the pattern used in this project.

IWAVE applications currently expect model data files in the RSF
format of Madagascar \cite[]{Madagascar}. Data
from other sources will need conversion to this format. An RSF data
set consists of two files, an ascii header (grid metadata) file and a
flat binary data file. The data set is referenced by the header file
name; one of the parameters listed in the header file is the pathname
of the binary data file, with key {\tt in}. The header file is small
and easily created by hand with an editor, if necessary. Madagascar
commands add processing history information to header files, and
modify their parameters. By convention, the last value of a parameter
({\tt key=value} pair) appearing in the file is the current value. Many archival
data formats make the grid sample values available as a flat binary
file - this is true for instance of the gridded models output by GOCAD
({\tt http://www.gocad.org}), for which the {\tt vo} files contain virtually
the same information as (so may easily be translated to) RSF header
files in ascii form, and the {\tt vodat} files are flat binary files
which may be used unaltered as RSF binary files.

IWAVE uses two extensions of the Madagascar RSF standard. The first is
the optional inclusion of the {\tt dim} and {\tt gdim} keywords. These
permit IWAVE applications to treat an RSF file image as defining a {\tt
  gdim} dimensional data hypercube divided into {\tt dim} dimensional
slices. The second is the axis identification keyword set, {\tt id1}, {\tt id2},
etc.: these supply information on the physical meaning of various axes. For an IWAVE {\tt dim} space-dimensional modeling problem, axes labeled {\tt
  id1},...,{\tt id[dim-1]} are the spatial grid axes. If {\tt gdim} >
{\tt dim}, then {\tt id[dim]} labels the time axis, and {\tt id[n]},
{\tt n > dim}, axes other than those of space-time. The IWAVE
structure paper \cite[]{trip14:struct} explains the use of the
additional keywords in more detail.

An example of this construction appears in the script that builds the
PML examples above, which are actually frames of movies. The output of
the 2D simulations are 3D RSF files ({\tt gdim=3, dim=2}) with {\tt id3=2},
that is, the third axis is treated as time. Madagascar applications
ignore these keywords: in particular, you can view the 3D RSF simulation
output as a movie using
{\tt sfgrey} and {\tt xtpen} as usual. The presence of the additional keywords is necessary in order
for IWAVE to correctly interpret the data geometry. 

This example illustrates another important feature of IWAVE
applications: any output data files must exist prior to execution -
their data samples are overwritten. The {\tt SConstruct} for this
project uses {\tt sfmakevel} to create the movie output files and {\tt
  sfput} to add the IWAVE-specific keywords to the headers, before
invoking the IWAVE command.

By IWAVE convention, the dimension of the problem is that of the primary
model grid. In the acoustic staggered grid application, the primary
model grid is that associated with the bulk modulus data. This grid is also the primary grid of the
simulation: that is, the space steps used in the finite difference
method are precisely those of the bulk modulus data.
Thus the choice of simulation grid is made externally to IWAVE.

The IWAVE acoustic application uses specific internal scales - m/ms
for velocity, g/cm$^3$ for density, and corresponding units for other
parameters. To ensure that data in other (metric) units are properly
scaled during i/o, the RSF header file may specify a value for the {\tt scale}
key, equal to the power of 10 by which the data should be multiplied
on being read into the application, to convert to the internal
scale. For example, if velocities are given in m/s, the header file
should include the line {\tt scale = -3}. In forthcoming releases,
this device will be deprecated in favor of explicit unit
specifications.

The current release is configured to use Seismic Unix (``SU'') (SEGY without reel header)
format for trace data i/o. Units of
length and time are m and ms respectively, consistent with other
internal unit choices. Two peculiarities of which the user should be
aware: (i) {\em receiver}
coordinates ({\tt gx}, {\tt gy}, and {\tt gelev} keywords) {\em always} specify trace location , that is, the
location at which values are sampled in space-time,
and (ii) on input, traces are regarded as point sources, so that each
trace multiplies a discrete spatial delta (hence values are scaled by the
reciprocal grid cell volume). Both of these design choices stem from the
migration (adjoint modeling) and inversion uses of IWAVE, discussed
for example in \cite[]{GeoPros:11,trip14:struct}. 

Source traces must be modified to conform to this rubric. The {\tt
  sftowed\_array} application relieves the user of the necessity to
manually adjust the headers of an SU file containing source traces. It
accepts three arguments: (i) an input source source file containing {\tt gx}, {\tt gy}, and {\tt gelev}
values representing source trace location {\em relative} to a source
center location - the 
source coordinates of source traces are ignored; (ii)
a data file whose {\tt sx}, {\tt sy}, and {\tt selev} values are the
source center locations to be used - its receiver coordinates are ignored, and (iii) an output file (name), 
to which output source traces will be written, each with {\em source}
coordinates equal to those of a data trace, and {\em receiver}
coordinates equal to the sums of the source trace {\em receiver}
coordinates and the data trace {\em source} coordinates. The result is
a collection of source coordinate gathers with the same source
coordinates as the data file, but within each gather the same receiver
coordinates {\em relative} to the source coordinates as the source
file. Thus the source array is translated to each of the source
centers specified in the data file. Because the source file may
contain arbitrarily many traces with arbitrary relative locations, any
source radiation pattern may be approximated \cite[]{SantosaSymes:00}. 

The example scripts in the {\tt project}
subdirectory use Madagascar
commands to create these prototype trace files.

One of IWAVE's design criteria is that acquisition geometry parameters
should have no {\em a priori} relation to the computational grid geometry:
source and receiver locations may be specified anywhere in Euclidean
space. 

\section{Discussion and Conclusion}
The rather large and only slowly disappearing error revealed by the
examples from \cite{SymesVdovina:09} suggests strong limits for the
accuracy of regular grid finite difference methods. Finite element
methods suffer from the same limitations: accurate solution of
acoustodynamic or elastodynamic problems appears to demand
interface-fitted meshed \cite[]{Cohen:01}, with the attendant increase
in code and computational complexity.

The situation may not be so bleak, however. For one special case,
namely constant density acoustics, 
\cite{Terentyev:09} show that a regular grid finite difference method,
derived from a regular grid Galerkin finite element method, has
accuracy properties one would expect in homogeneous media (second
order convergence, reduction of grid dispersion through higher order
space differencing) even for discontinuous models: the interface error
effect is attenuated. This type of result actually goes quite far back
in computational geophysics (see for example \cite{Muiretal:92}),
though theoretical support has been slower in coming.

Pure regular grid methods cannot take advantage of changes in average
velocity across the model, and concommitant changes in
wavelength. Coupling of local regular grids is possible, however, and
can yield substantial computational efficiency through grid coarsening
in higher velocity zones - see \cite{moczoetal:06}. IWAVE already
accommodates multiple grids (in domain decomposition parallelism), and
extension to incommensurable multiple grids would be a significant
change, but in principle
straightforward. The use of logically rectangular but
geometrically irregular (``stretched'') grids is completely
straightforward, on the other hand. 

These and other extensions, both past and future, are eased by the
reusability designed into the IWAVE core framework. This design has produced
reasonably well-performing and easy-to-use applications, and has proven
extensible to new models and schemes. Moreover, as explained by
\cite{GeoPros:11}, the object-oriented design of IWAVE dovetails with
similarly designed optimization software to support the construction
of waveform inversion software. The inversion applications resulting
from this marriage inherit the features of IWAVE - parallel execution,
high-order stencils, efficient boundary conditions, simple job control
- without requiring that these aspects be reworked in the code extensions. 

\section{Acknowledgements}
IWAVE has been a team effort: the original design of the core
framework is due to Igor Terentyev, and Tanya Vdovina, Dong Sun, Marco
Enriquez, Xin Wang, Yin Huang, Mario Bencomo, and Muhong Zhou have each made major contributions.
Development of IWAVE was supported by the SEG Advanced Modeling (SEAM)
project, by the National Science Foundation under awards 0620821 and
0714193, and by the sponsors of The Rice Inversion Project. The IWAVE
project owes a great deal to several open source seismic software
packages (Seismic Un*x, SEPlib, Madagascar), debts which we gratefully
acknowledge. The author wishes to record his special gratitude to
Sergey Fomel, the architect of Madagascar, for his inspiring ideas and
his generous and crucial help in the integration of IWAVE into Madagascar.

\bibliographystyle{seg}
\bibliography{../masterref}


