\section{Conjugate-direction iteration for complex numbers (ccdstep.c)}
\index{conjugate direction method!complex data}




\subsection{{sf\_ccdstep\_init}}\label{sec:sf_ccdstep_init}
Creates a complex number list for internal storage.

\subsubsection*{Call}
\begin{verbatim}sf_ccdstep_init();\end{verbatim}

\subsubsection*{Definition}
\begin{verbatim}
void sf_ccdstep_init(void) 
/*< initialize internal storage >*/
{
   ...
}
\end{verbatim}




\subsection{{sf\_ccdstep\_close}}
Frees the space allocated for internal storage by \hyperref[sec:sf_ccdstep_init]{\texttt{sf\_ccdstep\_init}}.

\subsubsection*{Call}
\begin{verbatim}sf_ccdstep_close();\end{verbatim}

\subsubsection*{Definition}
\begin{verbatim}
void sf_ccdstep_close(void) 
/*< free internal storage >*/
{
   ...
}
\end{verbatim}




\subsection{{sf\_ccdstep}}
Calculates one step for the conjugate direction iteration, that is, it calculates the new conjugate gradient for the new line search direction. It works like \hyperref[sec:sf_cdstep]{\texttt{sf\_cdstep}} but for complex numbers.

\subsubsection*{Call}
\begin{verbatim}sf_ccdstep (forget, nx, ny, x, g, rr, gg);\end{verbatim}

\subsubsection*{Definition}
\begin{verbatim}
void sf_ccdstep(bool forget          /* restart flag */, 
                int nx               /* model size */, 
                int ny               /* data size */, 
                sf_complex* x        /* current model [nx] */, 
                const sf_complex* g  /* gradient [nx] */, 
                sf_complex* rr       /* data residual [ny] */, 
                const sf_complex* gg /* conjugate gradient [ny] */) 
/*< Step of conjugate-direction iteration. 
  The data residual is rr = A x - dat>*/
{
   ...
}
\end{verbatim}
\subsubsection*{Input parameters}
\begin{desclist}{\tt }{\quad}[\tt forget]
   \setlength\itemsep{0pt}
   \item[forget] restart flag (\texttt{bool}).  
   \item[nx]     model size (\texttt{int}).  
   \item[ny]     data size (\texttt{int}).  
   \item[x]      current model (\texttt{sf\_complex*}).  
   \item[g]      gradient. Must be of type \texttt{const sf\_complex*}.  
   \item[rr]     data residual (\texttt{sf\_complex*}).
   \item[gg]     conjugate gradient. Must be of type \texttt{const sf\_complex*}.  
\end{desclist}





\subsection{{saxpy}}
Multiplies a given complex number with an array of complex numbers and stores the cumulative products in another array.

\subsubsection*{Call}
\begin{verbatim}saxpy (n, a, x, y);\end{verbatim}

\subsubsection*{Definition}
\begin{verbatim}
static void saxpy(int n, sf_double_complex a, 
                  const sf_complex *x, 
                  sf_complex *y)
/* y += a*x */
{
   ...
}
\end{verbatim}

\subsubsection*{Input parameters}
\begin{desclist}{\tt }{\quad}[\tt x]
   \setlength\itemsep{0pt}
   \item[n] length of the array of complex number (\texttt{int}).  
   \item[a] a complex number. Must be of type \texttt{sf\_double\_complex}.  
   \item[x] an array complex numbers (\texttt{sf\_complex*}).  
   \item[y] location where the cumulative sum of a*x is to be stored (\texttt{sf\_complex*}).  
\end{desclist}





\subsection{{dsdot}}
Returns the Hermitian\index{dot product!Hermitian} dot product of two complex numbers or the sum of the dot products if the are two arrays of complex numbers.

\subsubsection*{Call}
\begin{verbatim}prod = dsdot(n, cx, cy);\end{verbatim}

\subsubsection*{Definition}
\begin{verbatim}
static sf_double_complex dsdot(int n, 
                               const sf_complex *cx, 
                               const sf_complex *cy)
/* Hermitian dot product */
{
   ...
}
\end{verbatim}

\subsubsection*{Input parameters}
\begin{desclist}{\tt }{\quad}[\tt cy]
   \setlength\itemsep{0pt}
   \item[n]  size of the array of complex numbers (\texttt{int}).  
   \item[cx] a complex number (\texttt{sf\_complex*}).  
   \item[cy] a complex number (\texttt{sf\_complex*}).  
\end{desclist}

\subsubsection*{Output}
\begin{desclist}{\tt }{\quad}[\tt ]
   \setlength\itemsep{0pt}  
   \item[prod] dot product of the complex numbers. It is of type \texttt{static sf\_double\_complex}.
\end{desclist}



