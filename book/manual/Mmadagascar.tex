\documentclass[10pt]{book}
% \documentclass[10pt]{memoir}


\usepackage{geometry}                % See geometry.pdf to learn the layout options. There are lots.
\geometry{a4paper}                   % ... or a4paper or a5paper or ... 
%\geometry{landscape}                % Activate for for rotated page geometry
%\usepackage[parfill]{parskip}    % Activate to begin paragraphs with an empty line rather than an indent

\usepackage{makeidx}
\makeindex 

\setcounter{tocdepth}{1}

\usepackage{graphicx}
\usepackage{amssymb,amsmath}
\usepackage{dsfont}
\usepackage{epstopdf}
\DeclareGraphicsRule{.tif}{png}{.png}{`convert #1 `dirname #1`/`basename #1 .tif`.png}

\usepackage{desclist}


\usepackage{url}
\usepackage[dvips, bookmarks, hyperindex=true, colorlinks=false, 
            linktocpage=true
            pdftitle={An example PDF file from LaTeX}, 
            pdfauthor={Christos Saragiotis}, 
            pdfsubject={Madagascar Manual}, 
            pdfkeywords={reproducible research, opensource, geophysics, Madagascar}]
            {hyperref}
\usepackage{hyperref}
%\usepackage{memhfixc}


\title{Madagascar Programming Reference Manual}
\author{Seismic Wave Analysis Group\\
               \texttt{http://swag.kaust.edu.sa}\\
               \\KAUST\\
               \\
               Mohammad Akbar Zuberi\\
               Tariq Alkhalifah}
\date{}                                           % Activate to display a given date or no date





\begin{document}

\maketitle
\cleardoublepage

\tableofcontents\addcontentsline{toc}{chapter}{Contents}
\cleardoublepage

\chapter*{Preface}
\addcontentsline{toc}{chapter}{Preface}
This document is a contribution to the Madagascar developers community and it is meant to help out with Programming in Madagascar. It also came out of our selfish need to learn more about the libraries described in Madagascar. This document specifically lists and describes the RSF API functions supported by Madagascar for C. 

The first chapter is an introduction, summarizing the key information needed to understand Madagascar and its history. In the second chapter there is an example program with explanation for every line of the code. The example program is a finite difference modeling code.

The RSF Function Library starts from Chapter \ref{sec:datatypes}, which is a description of the data types used in Madagascar. Some data types are defined in Madagascar and some are from the standard C library headers. 


The subsequent chapters group different subroutines (the .c files) which are used for a particular task. Chapter \ref{sec:input} lists the .c files which have the functions used in the preparation for input, such as \texttt{sf\_alloc}, which allocates the required space.
 
Chapter \ref{sec:files} is related to the handling of the \texttt{.rsf} files, for example file input/output, extracting or inserting a parameter from a file.
 
Chapter \ref{sec:error} is for the error handling subroutines. It lists the functions which print the required error messages.
 
Chapter \ref{sec:lop} lists the linear operators. There is a detailed introduction in the beginning of the chapter.
 
Chapter \ref{sec:analysis} is for data analysis subroutines, for example \texttt{kiss\_fftr.c} for the Fourier inverse and forward transform of real time signals.
 
Chapter \ref{sec:filter} lists the subroutines for the filtering and convolution.
 
Chapter \ref{sec:solvers} lists the solvers. It has the subroutines for tasks like solving a first order ODE using the Runge-Kutta solver. There are subroutines in this chapter which perform the iterations for the conjugate gradient method for real and complex data. Some other functions include the root finder and tridagonal matrix solver.
 
Chapter \ref{sec:interpolation} is for the interpolation subroutines. It has subroutines for 1D, 2D and 3D interpolation. There are functions for B-Spline interpolation, calculation for B-spline coefficients, ENO and ENO power-p interpolation.
 
Chapter \ref{sec:smoothing} has the subroutines for of smoothing and edge detection. It has functions for 1D and 2D triangle smoothing. 
 
Chapter \ref{sec:ray} lists the Ray Tracing functions. 
 
Chapter \ref{sec:general} has some general purpose tools like functions for evaluating a mathematical expression and generating random numbers.
 
Chapter \ref{sec:geometry} is concerned with the geometry. It has the functions which define vectors and points which can be used to define the source and receiver coordinates. It also has the axa.c file which defines the functions for creating and operating on axes.
 
Chapter \ref{sec:misc} is for the miscellaneous functions.
 
Chapter \ref{sec:system} lists the function which are system specific for example system.c file defines functions which can run a given command on the terminal from within the program. 
 

\chapter{Introduction}
\chapter*{Themes}

The main theme of this book is to take a good quality reflection
seismic data set from the Gulf of Mexico and guide you through the
basic geophysical data processing steps from raw data to the
best-quality final image. Secondary themes are to introduce you (1) to
cleaned up but real working % Fortran 
code that does the job, (2) to the
concept of “adjoint operator”, and (3) to the notion of electronic
document.

\subsection*{What it does, what it means, and how it works}
A central theme of this book is to merge the abstract with the
concrete by linking mathematics to runnable computer codes. The codes
are in a consistent style using nomenclature that resembles the
accompanying mathematics so the two illuminate each other. The code
shown is exactly that used to generate the illustrations. There is
little or no mathematics or code that is not carried through with
examples using both synthetic and real data.

%The code itself is in
%a dialect of Fortran more suitable for exposition than standard Fortran. (This "ratfor" dialect
%easily translates to standard Fortran). 

Some codes have been heavily tested while others have
only been tested by the preparation of the illustrations.

\subsection*{Imaging with adjoint (conjugate-transpose) operators}

A secondary theme of this book is to develop in the reader an
understanding of a universal linkage beween forward modeling and data
processing. Thus the codes here that incarnate linear operators are
written in a style that also incarnates the adjoint
(conjugate-transpose) operator thus enabling both modeling and data
processing with the same code. This style of coding, besides being
concise and avoiding redundancy, ensures the consistency required for
estimation by conjugate-gradient optimization as described in my other
books.  Adjoint operators link the modeling activity to the model
estimation activity. While this linkage is less sophisticated than
formal estimation theory (“inversion”), it is robust, easily
available, and does not put unrealistic demands on the data or
imponderable demands on the interpreter.


\subsection*{Electronic document}

A goal that we met with the 1992 CD-ROM version of this book was to
give the user a full copy, not only of the book, but of all the
software that built the book including not only the seismic data
processing codes but also the word processing, the data, and the whole
superstructure.  Although we succeeded for a while having a book that
ran on machines of all the major manufacturers, eventually we were
beaten down by a host of incompatibilities. This struggle
continues. With my colleagues, we are now working towards having books
on the World Wide Web where you can grab parts of a book that
generates illustrations and modify them to create your own
illustrations.

\subsection*{Acknowledgements}

I had the good fortune to be able to establish a summer 1992
collaboration with Jim Black of IBM in Dallas who, besides bringing
fresh eyes to the whole undertaking, wrote the first version of
chapter 8 on dip moveout, made significant contributions to the other
chapters, and organized the raw data.  In this book, as in my previous
(and later) books, I owe a great deal to the many students at the
Stanford Exploration Project. The local computing environment from my
previous book is still a benefit, and for this I thank Stew Levin,
Dave Hale, and Richard Ottolini. In preparing this book I am specially
indebted to Joe Dellinger for his development of the intermediate
graphics language vplot that I used for all the figures. I am grateful
to Kamal Al-Yahya for converting my thinking from the troff
typesetting language to LATEX. Bill Harlan offered helpful
suggestions. Steve Cole adapted vplot to Postscript and X. Dave
Nichols introduced our multivendor environment. Joel M. Schroeder and
Matthias Schwab converted from cake to gmake. Bob Clapp expanded
Ratfor for Fortran 90. Martin Karrenbach got us started with
CD-ROMs. Sergey Fomel upgraded the Latex version to “2e” and he
implemented the basic changes taking us from CD-ROM to the WWW, a
process which continues to this day in year 2000.

\noindent Jon Claerbout \\
Stanford University \\
December 26, 2000

\chapter{An example: Finite-Difference modeling}

To demonstrate the use of the RSF library, a time-domain finite-difference modeling program is explained in detail.

\section{Introduction}
This section presents time-domain finite-difference modeling [1] written with the RSF library. The program is demonstrated with the C, C++ and Fortran 90 interfaces. The acoustic wave-equation
\begin{gather*}
	\Delta U - \frac{1}{v^2}\frac{\partial^2U}{\partial t^2} = f(t)
\intertext{can be written as}
	|\Delta U -	 f(t)|v^2 =   \frac{\partial^2U}{\partial t^2},       
\end{gather*}   
where $\Delta$ is the Laplacian symbol, $f(t)$ is the source wavelet, $v$ is the velocity, and $U$ is a scalar wavefield. A discrete time-step involves the following computations:                 
\begin{gather*}
	U_{i+1} = [\Delta U - f(t)]v^2\Delta t^2 + 2U_i - U_{i-1},       
\end{gather*}            
where $U_{i ? 1}$, $U_i$ and $U_{i + 1}$ represent the propagating wavefield at various time steps.
In this exercise we shall use a discrete Laplacian accurate up to the fourth order and the second derivative of time is accurate up to the second order.

\section{C program}
\begin{verbatim}
  1    /* time-domain acoustic FD modeling */
  2    #include <rsf.h>
  3
  4    int main(int argc, char* argv[])
  5    {
  6        /* Laplacian coefficients */
  7        float c0=-30./12.,c1=+16./12.,c2=- 1./12.;
  8     
  9        bool verb;           /* verbose flag */
 10        sf_file Fw=NULL,Fv=NULL,Fr=NULL,Fo=NULL; /* I/O files */
 11        sf_axis at,az,ax;    /* cube axes */
 12        int it,iz,ix;        /* index variables */
 13        int nt,nz,nx;
 14        float dt,dz,dx,idx,idz,dt2;
 15
 16        float  *ww,**vv,**rr;     /* I/O arrays*/
 17        float **um,**uo,**up,**ud;/* tmp arrays */
 18
 19        sf_init(argc,argv);
 20        if(! sf_getbool("verb",&verb)) verb=0;
 21
 22        /* setup I/O files */
 23        Fw = sf_input ("in" );
 24        Fo = sf_output("out");
 25        Fv = sf_input ("vel");
 26        Fr = sf_input ("ref");
 27
 28        /* Read/Write axes */
 29        at = sf_iaxa(Fw,1); nt = sf_n(at); dt = sf_d(at);
 30        az = sf_iaxa(Fv,1); nz = sf_n(az); dz = sf_d(az);
 31        ax = sf_iaxa(Fv,2); nx = sf_n(ax); dx = sf_d(ax);
 32
 33        sf_oaxa(Fo,az,1); 
 34        sf_oaxa(Fo,ax,2); 
 35        sf_oaxa(Fo,at,3);
 36
 37        dt2 =    dt*dt;
 38        idz = 1/(dz*dz);
 39        idx = 1/(dx*dx);
 40 
 41        /* read wavelet, velocity & reflectivity */
 42        ww = sf_floatalloc(nt);     sf_floatread(ww   ,nt   ,Fw);
 43        vv = sf_floatalloc2(nz,nx); sf_floatread(vv[0],nz*nx,Fv);
 44        rr = sf_floatalloc2(nz,nx); sf_floatread(rr[0],nz*nx,Fr);
 45 
 46        /* allocate temporary arrays */
 47        um = sf_floatalloc2(nz,nx);
 48        uo = sf_floatalloc2(nz,nx);
 49        up = sf_floatalloc2(nz,nx);
 50        ud = sf_floatalloc2(nz,nx);
 51
 52        for (iz=0; iz<nz; iz++) {
 53            for (ix=0; ix<nx; ix++) {
 54                um[ix][iz]=0;
 55                uo[ix][iz]=0;
 56                up[ix][iz]=0;
 57                ud[ix][iz]=0;
 58            }
 59        }
 60
 61        /* MAIN LOOP */
 62        if(verb) fprintf(stderr,"\n");
 63        for (it=0; it<nt; it++) {
 64            if(verb) fprintf(stderr,"\b\b\b\b\b%d",it);
 65  
 66             /* 4th order laplacian */
 67             for (iz=2; iz<nz-2; iz++) {
 68                 for (ix=2; ix<nx-2; ix++) {
 69                     ud[ix][iz] = 
 70                       c0* uo[ix  ][iz  ] * (idx+idz) + 
 71                       c1*(uo[ix-1][iz  ] + uo[ix+1][iz  ])*idx +
 72                       c2*(uo[ix-2][iz  ] + uo[ix+2][iz  ])*idx +
 73                       c1*(uo[ix  ][iz-1] + uo[ix  ][iz+1])*idz +
 74                       c2*(uo[ix  ][iz-2] + uo[ix  ][iz+2])*idz;	  
 75                 }
 76             }
 77
 78             /* inject wavelet */
 79             for (iz=0; iz<nz; iz++) {
 80                 for (ix=0; ix<nx; ix++) {
 81                     ud[ix][iz] -= ww[it] * rr[ix][iz];
 82                 }
 83             }
 84 
 85             /* scale by velocity */
 86             for (iz=0; iz<nz; iz++) {
 87                 for (ix=0; ix<nx; ix++) {
 88                     ud[ix][iz] *= vv[ix][iz]*vv[ix][iz];
 89                 }
 90             }
 91
 92             /* time step */
 93             for (iz=0; iz<nz; iz++) {
 94                 for (ix=0; ix<nx; ix++) {
 95                     up[ix][iz] = 2*uo[ix][iz] 
 96                                  - um[ix][iz] 
 97                                  + ud[ix][iz] * dt2; 
 98                           
 99                     um[ix][iz] = uo[ix][iz];
100                     uo[ix][iz] = up[ix][iz];
101                   }
102               }
103     
104               /* write wavefield to output */
105               sf_floatwrite(uo[0],nz*nx,Fo);
106         }               
107         if(verb) fprintf(stderr,"\n");    
108         sf_close()
109         exit(0);
110     }
\end{verbatim}
	
\section{Explanation of the code}
\begin{itemize}
    \item [\bf 2-4:]
\begin{verbatim}	
  2    #include <rsf.h>
  3
  4    int main(int argc, char* argv[])
\end{verbatim}	
Line 2 is a preprocessor directive to include the \texttt{rsf.h} header file which contains the RSF library functions.

Line 4 has parameters in the main function. This is to enable the program to take command line arguments. \texttt{char* argv[]} defines the pointer to the array of type \texttt{char} and \texttt{int argc} is the length of that array.
	
    \item [\bf 7:]
\begin{verbatim}
  7        float c0=-30./12.,c1=+16./12.,c2=- 1./12.;
\end{verbatim}
As was mentioned earlier, the Laplacian is being evaluated with an accuracy of up to the fourth order. These coefficients arise as a result of using five terms in the discrete form of the Laplacian.   
	
	\item [\bf 9-14:]
\begin{verbatim}
  9        bool verb;           /* verbose flag */
 10        sf_file Fw=NULL,Fv=NULL,Fr=NULL,Fo=NULL; /* I/O files */
 11        sf_axis at,az,ax;    /* cube axes */
 12        int it,iz,ix;        /* index variables */
 13        int nt,nz,nx;
 14        float dt,dz,dx,idx,idz,dt2;
\end{verbatim}

Line 9 defines a variable \texttt{verb} of type \texttt{bool}. This variable will be used in the program to check for verbosity flag. Lines 10-11 define the variables of the abstract data type provided by the RSF API. These will be used to store the input and output files. Lines 12-14 are the variables of integer and float type defined to be used as running variables (\texttt{it}, \texttt{iz}, \texttt{ix}) for the main loop, length of the axes (\texttt{nt}, \texttt{nz}, \texttt{nx}), the sampling of the axes (\texttt{dt}, \texttt{dx}, \texttt{dz}) and the squares and inverse squares of the samples (\texttt{dt2}, \texttt{idz}, \texttt{idx}).

\item [\bf 16-17:]
\begin{verbatim}
 16        float  *ww,**vv,**rr;     /* I/O arrays*/
 17        float **um,**uo,**up,**ud;/* tmp arrays */
   \end{verbatim}
Lines 16-17 define pointers to the arrays, which will be used for \texttt{input (*ww , **vv , **rr)} and for temporary storage \texttt{ (**um,**uo,**up,**ud)}. 

\item [\bf 19-20:]
\begin{verbatim}
 19	        sf_init(argc,argv);
 20         if(! sf_getbool("verb",&verb)) verb=0;
\end{verbatim}
	
Line 19 initializes the symbol tables used to store the argument from the command line.
	  
Line 20 tests the verbosity flag specified in the command line arguments. If the verbosity flag in the command line is set to \texttt{n}, the variable \texttt{verb} (of type \texttt{bool}) is set to zero. This would allow the verbose output to be printed only if the user set the verbosity flag to \texttt{y} in the command line. 


\item [\bf 22-26:]
\begin{verbatim}
 22        /* setup I/O files */
 23   	   Fw = sf_input ("in" );
 24   	   Fo = sf_output("out");
 25   	   Fv = sf_input ("vel");
 26   	   Fr = sf_input ("ref");
\end{verbatim}

In these lines we use the \hyperref[sec:sf_input]{\texttt{sf\_input}} (see p.~\pageref{sec:sf_input}) and \hyperref[sec:sf_output]{\texttt{sf\_output}} (see p.~\pageref{sec:sf_input}) functions of	    the RSF API. These functions take a string as argument and return a variable of type \texttt{sf\_file}, we had already defined this type of variables earlier in the program. 
           
\item [\bf 28-32:]
\begin{verbatim}
 28         /* Read/Write axes */
 29         at = sf_iaxa(Fw,1); nt = sf_n(at); dt = sf_d(at);
 30         az = sf_iaxa(Fv,1); nz = sf_n(az); dz = sf_d(az);
 31         ax = sf_iaxa(Fv,2); nx = sf_n(ax); dx = sf_d(ax);
\end{verbatim}

Here we input axes (\texttt{at}, \texttt{az}, \texttt{ax}) using \hyperref[sec:sf_iaxa]{\texttt{sf\_iaxa}} (p.~\pageref{sec:sf_iaxa} of the RSF API. \texttt{sf\_iaxa} accepts a variables of type \texttt{sf\_file} (RSF API) and an integer. The first argument in \texttt{sf\_iaxa} is the input file and the second is the axis which we want to input. In the second column we use \hyperref[sec:sf_n]{\texttt{sf\_n}} (p.~\pageref{sec:sf_n})from RSF API to get the lengths of the respective axes.

In the third column we use \hyperref[sec:sf_d]{\texttt{sf\_d}} (p.~\pageref{sec:sf_d}) of the RSF API to get the sampling interval of the respective axes.

\item [\bf 33-35:]
\begin{verbatim}
 33         sf_oaxa(Fo,az,1); 
 34         sf_oaxa(Fo,ax,2); 
 35         sf_oaxa(Fo,at,3);
\end{verbatim}

Here we output axes (\texttt{at}, \texttt{az}, \texttt{ax}) using \hyperref[sec:sf_oaxa]{\texttt{sf\_oaxa}} (p.~\pageref{sec:sf_oaxa}) of the RSF API. \texttt{sf\_oaxa} accepts variables of type \texttt{sf\_file} (RSF API), \texttt{sf\_axis} (RSF API) and an integer. First argument is the output file, second argument is the name of the axis which we want to output and the third is the number of the axis in the output file (\texttt{n1} is the fastest axis).

\item [\bf 37-39:]
\begin{verbatim}
 37       dt2 =    dt*dt;
 38       idz = 1/(dz*dz);
 39       idx = 1/(dx*dx);
\end{verbatim}

These lines define the square of the time sampling \texttt{interval(dt2)} and the inverse squares of the sampling interval of the spatial axes.
          

\item [\bf 41-44:]
\begin{verbatim}
 41       /* read wavelet, velocity & reflectivity */
 42       ww = sf_floatalloc(nt);     sf_floatread(ww   ,nt   ,Fw);
 43       vv = sf_floatalloc2(nz,nx); sf_floatread(vv[0],nz*nx,Fv);
 44       rr = sf_floatalloc2(nz,nx); sf_floatread(rr[0],nz*nx,Fr);
\end{verbatim}

In the first column we allocate the memory required to hold the input wavelet, velocity and reflectivity. This is done using \hyperref[sec:sf_floatalloc]{\texttt{sf\_floatalloc}} (p.~\pageref{sec:sf_floatalloc}) and \hyperref[sec:sf_floatalloc2]{\texttt{sf\_floatalloc2}} (p.~\pageref{sec:sf_floatalloc2}) of the RSF API. \texttt{sf\_floatalloc} takes integers as arguments and from these integers it calculates an allocates a block of memory of appropriate size. \texttt{sf\_floatalloc2}         is the same as \texttt{sf\_floatalloc} except for the fact that the former allocates an array of two dimensions, size of the memory block assigned in this case is the product of the two integers given as arguments (e.g. \texttt{nz*nx} in this case).  

Then \hyperref[sec:sf_floatread]{\texttt{sf\_floatread}} (p.~\pageref{sec:sf_floatread}) of the RSF API is used to read the data from the files into the allocated memory blocks (arrays). The \texttt{sf\_floatread} takes the arrays, integers and files as arguments and returns arrays filled with the data from the files.


\item [\bf 46-50:]
\begin{verbatim}
 46        /* allocate temporary arrays */
 47        um = sf_floatalloc2(nz,nx);
 48        uo = sf_floatalloc2(nz,nx);
 49        up = sf_floatalloc2(nz,nx);
 50        ud = sf_floatalloc2(nz,nx);
\end{verbatim}

Just like the memory blocks were allocated for input files to be read in to, we now allocate memory for the temporary arrays which will be used just for the calculation, using \hyperref[sec:sf_floatalloc2]{\texttt{sf\_floatalloc2}} (p.~\pageref{sec:sf_floatalloc2}).


\item [\bf 52-59:]
\begin{verbatim}
 52        for (iz=0; iz<nz; iz++) {
 53            for (ix=0; ix<nx; ix++) {
 54                um[ix][iz]=0;
 55                uo[ix][iz]=0;
 56                up[ix][iz]=0;
 57                ud[ix][iz]=0;
 58            }
 59        }
\end{verbatim}
Lines 52-59 initialize the temporary arrays by assigning \texttt{0} to each element of every array.

\item [\bf 61-64:]
\begin{verbatim}
 61        /* MAIN LOOP */
 62        if(verb) fprintf(stderr,"\n");
 63        for (it=0; it<nt; it++) {
 64            if(verb) fprintf(stderr,"\b\b\b\b\b%d",it);
\end{verbatim}
Now the main loop starts. The if condition in line 61 prints the message specified in the \texttt{fprintf} argument. The \texttt{stderr} is a stream in C which is used to direct the output to the screen. In this case the input is just an escape sequence \texttt{\\n}, which will bring the cursor to the next line if the user opted \texttt{y} or \texttt{1} to verbose flag in the command line (\texttt{verb=y} of \texttt{verb=1}).

Then the loop over time starts. Right after the for statement (within the body of the loop) there          is another if condition like the first one but this time it prints the the current value of \texttt{it}. This has escape sequence \texttt{\b} occurring several times. This is when the loop starts the value of \texttt{it} which is \texttt{0}, is printed on the screen, when the loop returns to the start the new value of it is \texttt{1}, so \texttt{\\b} (backspace) removes the previous value \texttt{0}, which is already on the screen, and puts \texttt{1} instead. 

\item [\bf 66-76:]
\begin{verbatim}
 66             /* 4th order laplacian */
 67             for (iz=2; iz<nz-2; iz++) {
 68                 for (ix=2; ix<nx-2; ix++) {
 69                     ud[ix][iz] = 
 70                       c0* uo[ix  ][iz  ] * (idx+idz) + 
 71                       c1*(uo[ix-1][iz  ] + uo[ix+1][iz  ])*idx +
 72                       c2*(uo[ix-2][iz  ] + uo[ix+2][iz  ])*idx +
 73                       c1*(uo[ix  ][iz-1] + uo[ix  ][iz+1])*idz +
 74                       c2*(uo[ix  ][iz-2] + uo[ix  ][iz+2])*idz;	  
 75                 }
 76             }
\end{verbatim}

This is the calculation for the fourth order laplacian. By the term ``4th order'' we mean the order of the approximation not the order of the PDE itself which of course is a second order PDE. A second order partial derivative discretized to second order approximation is written as:
\begin{gather*}
    \frac{\partial^2U}{\partial x^2} = \frac{U_{i+1} + 2U_i + U_{i-1}}{\Delta x^2}
\end{gather*}
     
This is the central difference formula for the second order partial derivative with pivot at $i-$th value of $U$.Similarly for the $z$ direction we have:
\begin{gather*}
    \frac{\partial^2U}{\partial z^2} = \frac{U_{i+1} + 2U_i + U_{i-1}}{\Delta z^2}
\end{gather*}

By adding these two we get the central difference formula accurate to the second order for the Laplacian. But we are using a central difference accurate up to the fourth order so for that          we have:
\begin{gather*}
    \frac{\partial^2U}{\partial x^2} = \frac{1}{\Delta x^2}
                                       \left[-\frac{1}{12}U_{i+2}+ \frac{16}{12}U_{i+1} 
                                         -\frac{30}{12}U_i   + \frac{16}{12}U_{i-i} 
                                         - \frac{1}{12}U_{i-2}\right]
\end{gather*}

By writing down a similar equation for $z$ and adding the two  we get the fourth order approximation of the Laplacian or as we refer to it here ``4th order laplacian''.

Now returning back to the code, the first line is the start of the loop in the $z$ direction. Within the body of the \texttt{z} loop there is another loop which runs through all the values of \texttt{x} for one value of \texttt{z}. The second line is start of the for-loop for the \texttt{x} direction. 
           
Then in the body of the loop for x direction we use the $2\times2$ arrays which we defined earlier.  This is just the equation of the Laplacian accurate up to the fourth order, as discussed            above, with the common coefficients factored out.  Note that the loops for \texttt{x} and \texttt{z} start two units after 0 and end two units before \texttt{nx} and \texttt{nz}. This is because to evaluate the Laplacian at a particular point $(x,z)$ the farthest values which we are using are           two units behind and two units ahead of the current point $(x,z)$ if we include the points \texttt{iz=0,1 ; iz=nz-1}, \texttt{nz} and \texttt{ix=0,1;ix=nx-1},\texttt{nx} we will run out of bounds. To fill these we will need a boundary condition which we will get from the next loop for           inserting the wavelet.



\item [\bf 78-83:]
\begin{verbatim}
 78             /* inject wavelet */
 79             for (iz=0; iz<nz; iz++) {
 80                 for (ix=0; ix<nx; ix++) {
 81                     ud[ix][iz] -= ww[it] * rr[ix][iz];
 82                 }
 83             }
\end{verbatim}


These lines insert the wavelet, which means evaluating the expression $\Delta U - f(t)$.
$\Delta U$ was already calculated in the previous loop and is stored as the array \texttt{ud}. \texttt{ww} is the array of the wavelet but before subtracting it form the Laplacian (\texttt{ud}) we multiply the wavelet amplitude at current time with the reflectivity at every point in space $(x,z)$. This amounts to an initial condition:
\begin{gather*}
    f(x,z,0) = g(x,z) = ww(0)rr(x,z),
\end{gather*}
and thus serves the purpose of filling the values at \texttt{ix,iz=0} and \texttt{ix-2,ix-1=0} and \texttt{iz-2,iz-1=0}.
            
But the source wavelet is not an ideal impulse so it has amplitudes at future times so for each time the wavelet will be multiplied by the reflectivity at every point $(x,z)$.
Why multiply the wavelet with reflectivity? Well, this model assumes  a hypothetical situation that the source was set off at each and every point in space $(x,z)$ under consideration and scaled by the reflectivity at that point $(x,z)$. What this means is that the source was set off at all the points where there is a change in the acoustic impedance (because reflectivity is the ratio of the difference and sum of the acoustic impedances across an interface).  

\item [\bf 85-90:]
\begin{verbatim}
 85             /* scale by velocity */
 86             for (iz=0; iz<nz; iz++) {
 87                 for (ix=0; ix<nx; ix++) {
 88                     ud[ix][iz] *= vv[ix][iz]*vv[ix][iz];
 89                 }
 90             }
\end{verbatim}

Here we just multiply $\Delta U - f(t)$  by the velocity, that is, we evaluate $(\Delta U - f(t))v^2$
\item [\bf 92-102]
\begin{verbatim}
 92             /* time step */
 93             for (iz=0; iz<nz; iz++) {
 94                 for (ix=0; ix<nx; ix++) {
 95                     up[ix][iz] = 2*uo[ix][iz] 
 96                                  - um[ix][iz] 
 97                                  + ud[ix][iz] * dt2; 
 98                           
 99                     um[ix][iz] = uo[ix][iz];
100                     uo[ix][iz] = up[ix][iz];
101                   }
102               }
\end{verbatim}

Here we calculate the time step, that is,
\begin{gather*}
    U_{i+1} = [\Delta U - f(t)]v^2\Delta t^2 + 2U_i - U_{i-1}.
\end{gather*}


The first for-loop is for the $z$ direction and within the body of this loop is another for-loop for the $x$ direction. \texttt{up} is the array which holds the amplitude of the wave at the current time in the time loop. \texttt{uo} is the array which contains the amplitude at a time one unit before the current time and the array um holds the amplitude two units before. \texttt{ud} is the array we calculated earlier in the program, now it gets multiplied by $\Delta t^2$ (\texttt{dt2}) and included in the final equation. This completes the calculation for one value of  \texttt{it}.
Now the arrays need to be updated to represent the next time step. This is done in the last two: The first one says $U_{i-1} \rightarrow U_i $ and the second one says $U_i \rightarrow U_{i+1}$, that is, the array um is updated by \texttt{uo} and then the array \texttt{uo} itself gets updated by \texttt{up}.


\item [\bf 104-106:]
\begin{verbatim}
104               /* write wavefield to output */
105               sf_floatwrite(uo[0],nz*nx,Fo);
106         }
\end{verbatim}
After the calculations for one time step are complete we write the array \texttt{uo} (remember that \texttt{uo} was made equal to \texttt{up}, which is the current time step, in the previous line). To write the array in the output file we use \hyperref[sec:sf_floatwrite]{\texttt{sf\_floatwrite}} (p.~\pageref{sec:sf_floatwrite}) exactly the same way we used \texttt{sf\_floatread} to read in from the input files, only difference is that the array given as the argument is written into the file given in the last argument. The bracket close is for the time loop, after this the time loop will start all over again for the next time value.  


\item [\bf 107-109:]
\begin{verbatim}
107         if(verb) fprintf(stderr,"\n");    
108         sf_close()
109         exit(0);
110     }
\end{verbatim}
The first line puts the cursor in the new line on the screen after the time loop has run through all the time values.

The second line uses \hyperref[sec:sf_close]{\texttt{sf\_close}} (p.~\pageref{sec:sf_close}) from RSF API to remove the temporary files.

The third line uses the \texttt{exit()} function in C language to close the streams and return the control to the host environment. The \texttt{0} in the argument indicates normal termination of the program. The last bracket closes the main function.

\end{itemize}














%

%
%RSF API Dictionary

%

%sf_axis
%File and Path:  Source Directory/build/api/axa.c
%Definition      :  A data structure for axes.

%sf_close
%File and Path: Source Directory/build/api/file.c
%Definition      : Remove temporary files

%sf_d
%File and Path: Source Directory/build/api/axa.c
%Definition      : Access axis sampling

%sf_file
%File and Path: Source Directory/build/api/file.c
%Definition      : Data structure for an RSF file

%sf_floatalloc
%File and Path: Source Directory/build/api/alloc.c
%Definition      : Float allocation.

%sf_floatalloc2
%File and Path: Source Directory/build/api/alloc.c
%Definition      : Float 2-D allocation, out[0] points to a contiguous array.

%sf_floatread
%File and Path: Source Directory/build/api/file.c
%Definition      : Read a float array arr[size] from file

%sf_floatwrite
%File and Path: Source Directory/build/api/file.c
%Definition      : Write a float array arr[size] to file. 

%sf_getbool
%File and Path: Source Directory/build/api/getpar.c
%Definition      : Get a bool parameter from the command line.

%
%sf_iaxa
%File and Path: Source Directory/build/api/axa.c
%Definition      : Read axis

%sf_init
%File and Path: Source Directory/build/api/getpar.c
%Definition      : Initialize parameter table from command-line arguments.

%sf_input
%File and Path: Source Directory/build/api/file.c
%Definition      : Create an input file structure

%sf_n
%File and Path: Source Directory/build/api/axa.c
%Definition      : Access axis length

%sf_oaxa
%File and Path: Source Directory/build/api/axa.c
%Definition      : Write axis

%sf_output
%File and Path: Source Directory/build/api/file.c
%Definition      : Create an output file structure

%

%

% ########
%\part{The RSF library}

\chapter{Data types}\label{sec:datatypes}
  \section{Data types}
This chapter contains the descriptions of the data types used in the RSF API.



\subsection{Complex numbers and FFT}
This section lists the data types for the complex numbers and FFT.

\subsubsection*{kiss\_fft\_scalar}
This is a data type, which defines a scalar real value for the data type \texttt{kiss\_fft\_cpx} for complex numbers. It can be either of type \texttt{short} or \texttt{float}. Default is \texttt{float}.

\subsubsection*{kiss\_fft\_cpx}
This is a data type (a C structure), which defines a complex number. It has the real and imaginary parts of the complex numbers defined to be of type \texttt{kiss\_fft\_scalar}.

\subsubsection*{kiss\_fft\_cfg}
This is an object of type \texttt{kiss\_fft\_state} (which is a C data structure). 

\subsubsection*{kiss\_fft\_state}
The \texttt{kiss\_fft\_state} is a data type which defines the required variables for the Fourier transform and allocates the required space. For example the variable �inverse� of type int indicates whether the transform needs to be an inverse or forward. 

\subsubsection*{kiss\_fftr\_cfg}
This is an object of type \texttt{kiss\_fftr\_state} (which is a C data structure). 

\subsubsection*{kiss\_fftr\_state}
The \texttt{kiss\_fftr\_state} is a data type which defines the required variables for the Fourier transform and allocates the required space. This has the same purpose as \texttt{kiss\_fft\_state} but for the Fourier transform of the real signals. 

\subsubsection*{sf\_complex}
This is an object of type \texttt{kiss\_fft\_cfg} (which is an object of  C data structure). 

\subsubsection*{sf\_double\_complex}
This is a C data structure for complex numbers. It uses the type double for the real and imaginary parts of the complex numbers.



\subsection{Files}
This section lists the data types used to define the \texttt{.rsf} file structure.

\subsubsection*{sf\_file}
This is an object of type \texttt{sf\_File}. \texttt{sf\_File} is a data structure which defines the variables required for creating a \texttt{.rsf} file in Madagascar. It is defined in \hyperref[sec:file.c]{\texttt{file.c}}.  

\subsubsection*{sf\_datatype}
This is a C enumeration, which means that it contains new data types, which are not the fundamental types like int, float, \texttt{sf\_file} etc.  This data type is used in \texttt{sf\_File} data structure to set the type of a \texttt{.rsf} file, for example SF\_CHAR, SF\_INT etc. It is defined in \hyperref[sec:file.c]{\texttt{file.c}}.

\subsubsection*{sf\_dataform}
This is a C enumeration, which means that it contains new data types, which are not the fundamental types like int, float, \texttt{sf\_file} etc.  This data type is used in \texttt{sf\_File} data structure to set the format of an \texttt{.rsf} file, for example \texttt{SF\_ASCII}, \texttt{SF\_XDR} and \texttt{SF\_NATIVE}. It is defined in \hyperref[sec:file.c]{\texttt{file.c}}.



\subsection{Operators}
This section lists the data types used to define linear operators.

\subsubsection*{sf\_triangle}
This is an object of an abstract C datatype type \texttt{sf\_Triangle}. The \texttt{sf\_triangle} data type defines the variables of relevant types to store information about the triangle smoothing filter. It is defined in \hyperref[sec:triangle.c]{\texttt{triangle.c}}.

\subsubsection*{sf\_operator}\label{sec:sf_operator}
This is a C data type of type void. It is also a pointer to a function which takes the input parameters precisely as \texttt{(bool, bool, int, int, float*, float*)}. It is defined in \texttt{\_solver.h}.

\subsubsection*{sf\_solverstep}\label{sec:sf_solverstep}
This is a C data type of type void. It is also a pointer to a function which takes the input parameters precisely as \texttt{(bool, bool, int, int, float*, const float*, float*, const float*)}. It is defined in \texttt{\_solver.h}.

\subsubsection*{sf\_weight}\label{sec:sf_weight}
This is a C data type of type void. It is also a pointer to a function which takes the input parameters precisely as \texttt{(int, int, const sf\_complex*, float*)}. It is defined in \texttt{\_solver.h}.

\subsubsection*{sf\_coperator}
This is a C data type of type void. It is also a pointer to a function which takes the input parameters precisely as \texttt{(bool
 bool, int, int, sf\_complex*, sf\_complex*)}. It works just like \hyperref[sec:sf_operator]{\texttt{sf\_operator}} but does it for complex numbers. It is defined in \texttt{\_solver.h}.

\subsubsection*{sf\_csolverstep}
This is a C data type of type void. It is also a pointer to a function which takes the input parameters precisely as \texttt{(bool, bool, int, int, sf\_complex*, const sf\_complex*, sf\_complex*, const sf\_complex*)}. It works just like \hyperref[sec:sf_solverstep]{\texttt{sf\_solverstep}} but does it for complex numbers. It is defined in \texttt{\_solver.h}.

\subsubsection*{sf\_cweight}
This is a C data type of type void. It is also a pointer to a function which takes the input parameters precisely as \texttt{(int, int, const sf\_complex*, float*)}. It works just like \hyperref[sec:sf_weight]{\texttt{sf\_weight}} but does it for the complex numbers. It is defined in \texttt{\_solver.h}.

\subsubsection*{sf\_eno}
This is a C data structure, which contains the required variables for 1D ENO (Essentially Non Oscillatory) interpolation. It is defined in \hyperref[sec:eno.c]{\texttt{eno.c}}.

\subsubsection*{sf\_eno2}
This is a C data structure, which contains the required variables for 2D ENO (Essentially Non Oscillatory) interpolation. It is defined in \hyperref[sec:eno2.c]{\texttt{eno2.c}}.

\subsubsection*{sf\_bands}
This is a C data structure, which contains the required variables for storing a banded matrix. It is defined in \hyperref[sec:banded.c]{\texttt{banded.c}}.



\subsection{Geometry}
This section lists the data types used to define the geometry of the seismic data.

\subsubsection*{sf\_axa}
This is a C data structure which contains the variables of type int and float to store the length origin and sampling of the axis. It is defined in \hyperref[sec:axa.c]{\texttt{axa.c}}.

\subsubsection*{pt2d}
This is a C data structure which contains the variables of type double and float to store the location and value of a 2D point. It is defined in \hyperref[sec:point.c]{\texttt{point.c}}.

\subsubsection*{pt3d}
This is a C data structure which contains the variables of type double and float to store the location and value of a 3D point. It is defined in \hyperref[sec:point.c]{\texttt{point.c}}.

\subsubsection*{vc2d}
This is a C data structure which contains the variables of type double to store the components of a 2D vector. It is defined in \hyperref[sec:vector.c]{\texttt{vector.c}}.

\subsubsection*{vc3d}
This is a C data structure which contains the variables of type double to store the components of a 3D vector. It is defined in \hyperref[sec:vector.c]{\texttt{vector.c}}.



\subsection{Lists}
This section describes the data types used to create and operate on lists.

\subsubsection*{sf\_list}
This is a C data structure, which contains the required variables for storing the information about the list, for example . It uses another C data structure �Entry�. It is defined in \hyperref[sec:llist.c]{\texttt{llist.c}}.

\subsubsection*{Entry}
This is a C data structure, which contains the required variables for storing the elements and moving the pointer in the list. It is defined in \hyperref[sec:llist.c]{\texttt{llist.c}}.



\subsection{sys/types.h}
This section describes some of the data types used from the C header file \texttt{sys/types.h}.

\subsubsection*{off\_t}
This is a data type defined in the \texttt{sys/types.h} header file (of fundamental type \texttt{unsigned long}) and is used to measure the file offset in bytes from the beginning of the file. It is defined as a signed, 32-bit integer, but if the programming environment enables large files \texttt{off\_t} is defined to be a signed, 64-bit integer.

\subsubsection*{size\_t}
This is a data type defined in the \texttt{sys/types.h} header (of fundamental type \texttt{unsigned int}) and is used to measure the file size in units of character. It is used to hold the result of the sizeof operator in C, for example sizeof(int)=4, sizeof(char)=1, etc.


\chapter{Preparing for input}\label{sec:input}
   \section{Convenience allocation programs (alloc.c)}




\subsection{{sf\_alloc}}\label{sec:sf_alloc}
Checks whether the requested size for memory allocation is valid and if so it returns a pointer of void type, pointing to the allocated memory block. It takes the 'number of elements' and 'size of one element' as input arguments. Both arguments have to be of the of type \texttt{size\_t}.

\subsubsection*{Call}
\begin{verbatim}sf_alloc (n, size);\end{verbatim}

\subsubsection*{Definition}
\begin{verbatim}
void *sf_alloc (size_t n    /* number of elements */, 
                size_t size /* size of one element */)
          /*< output-checking allocation >*/
{
   ...    
}
\end{verbatim}

\subsubsection*{Input parameters}
\begin{desclist}{\tt }{\quad}[\tt size]
   \setlength\itemsep{0pt}
   \item[n]	   number of elements (\texttt{size\_t}).
   \item[size] size of each element, for example \texttt{sizeof(float)} (\texttt{size\_t}).
\end{desclist}

\subsubsection*{Output}
\begin{desclist}{\tt }{\quad}[\tt ptr]
   \setlength\itemsep{0pt}
   \item[ptr] a void pointer pointing to the allocated block of memory.
\end{desclist}




\subsection{{sf\_realloc}}
The same as \hyperref[sec:sf_alloc]{\texttt{sf\_alloc}} but it allocates new memory such that it appends the block previously assigned by \texttt{sf\_alloc}. It takes three parameters, first one is a void pointer to the old memory block. Second and third parameters are the same as for \texttt{sf\_alloc} but are used to determine the new block, which is to be appended.
\texttt{sf\_realloc} returns a void pointer pointing to the whole memory block (new + old).


\subsubsection*{Call}
\begin{verbatim}sf_realloc (ptr, n, size);\end{verbatim}

\subsubsection*{Definition}
\begin{verbatim}
void *sf_realloc (void* ptr   /* previous data */, 
                  size_t n    /* number of elements */, 
                  size_t size /* size of one element */)
/*< output-checking reallocation >*/
{
   ...
}
\end{verbatim}

\subsubsection*{Input parameters}
\begin{desclist}{\tt }{\quad}[\tt size]
   \setlength\itemsep{0pt}
   \item[ptr]  pointer to the previously assigned memory block.
   \item[n]	   number of elements (\texttt{size\_t}).
   \item[size] size of each element, for example \texttt{sizeof(float)} (\texttt{size\_t}).
\end{desclist}

\subsubsection*{Output}
\begin{desclist}{\tt }{\quad}[\tt ]
   \setlength\itemsep{0pt}
   \item[ptr]	pointer to the new aggregate block.
\end{desclist}




\subsection{{sf\_charalloc}}\label{sec:sf_charalloc}
Allocates the memory exactly like \hyperref[sec:sf_alloc]{\texttt{sf\_alloc}} but the size in this one is fixed which is the size of one character. Therefore \texttt{sf\_charalloc} allocates the memory for \texttt{n} elements which must be of character type. Because the size is fixed there is just one input parameter which is the number of elements (i.e.~characters). Output is a void pointer pointing to the block of memory allocated.

\subsubsection*{Call}
\begin{verbatim}ptr = sf_charalloc (n);\end{verbatim}

\subsubsection*{Definition}
\begin{verbatim}
char *sf_charalloc (size_t n /* number of elements */)
/*< char allocation >*/ 
{
   ...    
}
\end{verbatim}

\subsubsection*{Input parameters}
\begin{desclist}{\tt }{\quad}[\tt ]
   \setlength\itemsep{0pt}
   \item[n]	number of elements (\texttt{size\_t}).
\end{desclist}

\subsubsection*{Output}
\begin{desclist}{\tt }{\quad}[\tt ]
   \setlength\itemsep{0pt}
   \item[ptr]	a void pointer pointing to the allocated block of memory.
\end{desclist}




\subsection{{sf\_ucharalloc}}\label{sec:sf_ucharalloc}
The same as \hyperref[sec:sf_charalloc]{\texttt{sf\_charalloc}} but it only allocates the memory for the unsigned character type, that is, the size of the elements is \texttt{sizeof(unsigned char)}.

\subsubsection*{Call}
\begin{verbatim}ptr = sf_ucharalloc (n);\end{verbatim}

\subsubsection*{Definition}
\begin{verbatim}
unsigned char *sf_ucharalloc (size_t n /* number of elements */)
/*< unsigned char allocation >*/ 
{
   ...
}
\end{verbatim}

\subsubsection*{Input parameters}
\begin{desclist}{\tt }{\quad}[\tt ]
   \setlength\itemsep{0pt}
   \item[n]	number of elements (\texttt{size\_t}).
\end{desclist}

\subsubsection*{Output}
\begin{desclist}{\tt }{\quad}[\tt ]
   \setlength\itemsep{0pt}
   \item[ptr] a void pointer pointing to the allocated block of memory.
\end{desclist}




\subsection{{sf\_shortalloc}}
Allocates the memory for  the short integer type, that is,  the size of the elements is, for example \texttt{sizeof(short int)}. 

\subsubsection*{Call}
\begin{verbatim}ptr = sf_shortalloc (n);\end{verbatim}

\subsubsection*{Definition}
\begin{verbatim}
short *sf_shortalloc (size_t n /* number of elements */)
/*< short allocation >*/  
{
   ...
}
\end{verbatim}

\subsubsection*{Input parameters}
\begin{desclist}{\tt }{\quad}[\tt ]
   \setlength\itemsep{0pt}
   \item[n]  number of elements (\texttt{size\_t}).
\end{desclist}

\subsubsection*{Output}
\begin{desclist}{\tt }{\quad}[\tt ]
   \setlength\itemsep{0pt}
   \item[ptr] a void pointer pointing to the allocated block of memory.
\end{desclist}




\subsection{{sf\_intalloc}}\label{sec:sf_intalloc}
Allocates the memory for  the large integer type, that is,  the size of the elements is, for example \texttt{sizeof(int)}. 

\subsubsection*{Call}
\begin{verbatim}ptr = sf_intalloc (n);\end{verbatim}

\subsubsection*{Definition}
\begin{verbatim}
int *sf_intalloc (size_t n /* number of elements */)
          /*< int allocation >*/  
{
   ...
}
\end{verbatim}

\subsubsection*{Input parameters}
\begin{desclist}{\tt }{\quad}[\tt ]
   \setlength\itemsep{0pt}
   \item[n]	number of elements (\texttt{size\_t}).
\end{desclist}

\subsubsection*{Output}
\begin{desclist}{\tt }{\quad}[\tt ]
   \setlength\itemsep{0pt}
   \item[ptr] a void pointer pointing to the allocated block of memory.
\end{desclist}





\subsection{{sf\_largeintalloc}}
Allocates the memory for  the large integer type, that is,  the size of the elements is, for example \texttt{sizeof(large int)}. 

\subsubsection*{Call}
\begin{verbatim}ptr = sf_largeintalloc (n);\end{verbatim}

\subsubsection*{Definition}
\begin{verbatim}
off_t *sf_largeintalloc (size_t n /* number of elements */)
/*< sf_largeint allocation >*/  
{
   ...
}
\end{verbatim}

\subsubsection*{Input parameters}
\begin{desclist}{\tt }{\quad}[\tt ]
   \setlength\itemsep{0pt}
   \item[n]  number of elements (\texttt{size\_t}).
\end{desclist}

\subsubsection*{Output}
\begin{desclist}{\tt }{\quad}[\tt ]
   \setlength\itemsep{0pt}
   \item[ptr] a void pointer pointing to the allocated block of memory.
\end{desclist}




\subsection{{sf\_floatalloc}}\label{sec:sf_floatalloc}
Allocates the memory for  the floating point type, that is,  the size of the elements is, for example \texttt{sizeof(float)}. 

\subsubsection*{Call}
\begin{verbatim}ptr = sf_floatalloc (n);\end{verbatim}

\subsubsection*{Definition}
\begin{verbatim}
float *sf_floatalloc (size_t n /* number of elements */)
          /*< float allocation >*/ 
{
   ...
}
\end{verbatim}

\subsubsection*{Input parameters}
\begin{desclist}{\tt }{\quad}[\tt ]
   \setlength\itemsep{0pt}
   \item[n] number of elements (\texttt{size\_t}).
\end{desclist}

\subsubsection*{Output}
\begin{desclist}{\tt }{\quad}[\tt ]
   \setlength\itemsep{0pt}
   \item[ptr] a void pointer pointing to the allocated block of memory.
\end{desclist}




\subsection{{sf\_complexalloc}}\label{sec:sf_complexalloc}
Allocates the memory for the \texttt{sf\_complex} type, that is, the size of the elements is, for example \texttt{sizeof(sf\_complex)}. 


\subsubsection*{Call}
\begin{verbatim}ptr = sf_complexalloc (n);\end{verbatim}

\subsubsection*{Definition}
\begin{verbatim}
sf_complex *sf_complexalloc (size_t n /* number of elements */) 
/*< complex allocation >*/
{
   ...
}
\end{verbatim}

\subsubsection*{Input parameters}
\begin{desclist}{\tt }{\quad}[\tt ]
   \setlength\itemsep{0pt}
   \item[n] number of elements (\texttt{size\_t}).
\end{desclist}

\subsubsection*{Output}
\begin{desclist}{\tt }{\quad}[\tt ]
   \setlength\itemsep{0pt}
   \item[ptr] a void pointer pointing to the allocated block of memory.
\end{desclist}




\subsection{{sf\_complexalloc2}}\label{sec:sf_complexalloc2}
Allocates a 2D array in the memory for the  \texttt{sf\_complex} type. It works just like \hyperref[sec:sf_complexalloc]{\texttt{sf\_complexalloc}} but does it for two dimensions. This is done by making a pointer point to another pointer, which in turn points to a particular column (or row) of an allocated 2D block of memory of size \texttt{n1*n2}. \texttt{n1} is the fastest dimension.

\subsubsection*{Call}
\begin{verbatim}ptr = sf_complexalloc2 (n1, n2);\end{verbatim}

\subsubsection*{Definition}
\begin{verbatim}
sf_complex **sf_complexalloc2 (size_t n1 /* fast dimension */, 
                               size_t n2 /* slow dimension */)
/*< complex 2-D allocation, out[0] points to a contiguous array >*/ 
{
   ...
}
\end{verbatim}

\subsubsection*{Input parameters}
\begin{desclist}{\tt }{\quad}[\tt ]
   \setlength\itemsep{0pt}
   \item[n1] number of elements in the fastest dimension (\texttt{size\_t}).
   \item[n2] number of elements in the slower dimension (\texttt{size\_t}).
\end{desclist}

\subsubsection*{Output}
\begin{desclist}{\tt }{\quad}[\tt ]
   \setlength\itemsep{0pt}
   \item[ptr] a void pointer pointing to the allocated block of memory.
\end{desclist}




\subsection{{sf\_complexalloc3}}
Allocates a 3D array in the memory for the  \texttt{sf\_complex} type. It works just like \hyperref[sec:sf_complexalloc2]{\texttt{sf\_complexalloc2}} but does it for three dimensions. This is done by extending the same argument as for \texttt{sf\_complexalloc2} this time making a pointer such that \texttt{Pointer2 -> Pointer1 -> Pointer}. \texttt{n1} is the fastest dimension.

\subsubsection*{Call}
\begin{verbatim}ptr = sf_complexalloc3 (n1, n2, n3);\end{verbatim}

\subsubsection*{Definition}
\begin{verbatim}
sf_complex ***sf_complexalloc3 (size_t n1 /* fast dimension */, 
                                size_t n2 /* slower dimension */, 
                                size_t n3 /* slowest dimension */)
/*< complex 3-D allocation, out[0][0] points to a contiguous array >*/
{
   ...
}
\end{verbatim}

\subsubsection*{Input parameters}
\begin{desclist}{\tt }{\quad}[\tt ]
   \setlength\itemsep{0pt}
   \item[n1] number of elements in the fastest dimension (\texttt{size\_t}).
   \item[n2] number of elements in the slower dimension (\texttt{size\_t}).
   \item[n3] number of elements in the slower dimension (\texttt{size\_t}).
\end{desclist}

\subsubsection*{Output}
\begin{desclist}{\tt }{\quad}[\tt ]
   \setlength\itemsep{0pt}
   \item[ptr] a void pointer pointing to the allocated block of memory.
\end{desclist}




\subsection{{sf\_complexalloc4}}
Allocates a 4D array in the memory for the  \texttt{sf\_complex} type. It works just like \hyperref[sec:sf_complexalloc2]{\texttt{sf\_complexalloc2}} but does it for four dimensions. This is done by extending the same argument as for \texttt{sf\_complexalloc2} but this time making a pointer such that \texttt{Pointer3 -> Pointer2 -> Pointer1 -> Pointer}. \texttt{n1} is the fastest dimension.

\subsubsection*{Call}
\begin{verbatim}ptr = sf_complexalloc4 (n1, n2, n3, n4);\end{verbatim}

\subsubsection*{Definition}
\begin{verbatim}
sf_complex ****sf_complexalloc4 (size_t n1 /* fast dimension */, 
                                 size_t n2 /* slower dimension */, 
                                 size_t n3 /* slower dimension */, 
                                 size_t n4 /* slowest dimension */)
/*< complex 4-D allocation, out[0][0][0] points to a contiguous array >*/ 
{
   ...
}
\end{verbatim}

\subsubsection*{Input parameters}
\begin{desclist}{\tt }{\quad}[\tt ]
   \setlength\itemsep{0pt}
   \item[n1] number of elements in the fastest dimension (\texttt{size\_t}).
   \item[n2] number of elements in the slower dimension (\texttt{size\_t}).
   \item[n3] number of elements in the slower dimension (\texttt{size\_t}).
   \item[n4] number of elements in the slower dimension (\texttt{size\_t}).
\end{desclist}

\subsubsection*{Output}
\begin{desclist}{\tt }{\quad}[\tt ]
   \setlength\itemsep{0pt}
   \item[ptr] a void pointer pointing to the allocated block of memory.
\end{desclist}




\subsection{{sf\_boolalloc}}\label{sec:sf_boolalloc}
Allocates the memory for  the bool type, that is,  the size of the elements is, for example \texttt{sizeof(bool)}. 

\subsubsection*{Call}
\begin{verbatim}ptr = sf_boolalloc (n);\end{verbatim}

\subsubsection*{Definition}
\begin{verbatim}
bool *sf_boolalloc (size_t n /* number of elements */)
/*< bool allocation >*/
{
   ...
}
\end{verbatim}

\subsubsection*{Input parameters}
\begin{desclist}{\tt }{\quad}[\tt ]
   \setlength\itemsep{0pt}
   \item[n] number of elements (\texttt{size\_t}).
\end{desclist}

\subsubsection*{Output}
\begin{desclist}{\tt }{\quad}[\tt ]
   \setlength\itemsep{0pt}
   \item[ptr] a void pointer pointing to the allocated block of memory.
\end{desclist}




\subsection{{sf\_boolalloc2}}\label{sec:sf_boolalloc2}
Allocates a 2D array in the memory for the bool type. It works just like \hyperref[sec:sf_boolalloc]{\texttt{sf\_boolalloc}} but does it for two dimensions. This is done by making a pointer point to another pointer, which in turn points to a particular column (or row) of an allocated 2D block of memory of size \texttt{n1*n2}. \texttt{n1} is the fastest dimension.

\subsubsection*{Call}
\begin{verbatim}ptr = sf_boolalloc2 (n1, n2);\end{verbatim}

\subsubsection*{Definition}
\begin{verbatim}
bool **sf_boolalloc2 (size_t n1 /* fast dimension */, 
                      size_t n2 /* slow dimension */)
/*< bool 2-D allocation, out[0] points to a contiguous array >*/
{
   ...
}
\end{verbatim}

\subsubsection*{Input parameters}
\begin{desclist}{\tt }{\quad}[\tt ]
   \setlength\itemsep{0pt}
   \item[n1] number of elements in the fastest dimension (\texttt{size\_t}).
   \item[n2] number of elements in the slower dimension (\texttt{size\_t}).
\end{desclist}

\subsubsection*{Output}
\begin{desclist}{\tt }{\quad}[\tt ]
   \setlength\itemsep{0pt}
   \item[ptr] a void pointer pointing to the allocated block of memory.
\end{desclist}




\subsection{{sf\_boolalloc3}}
Allocates a 3D array in the memory for the bool type. It works just like \hyperref[sec:sf_boolalloc2]{\texttt{sf\_boolalloc2}} but does it for three dimensions. This is done by extending the same argument as for \texttt{sf\_boolalloc2} but this time making a pointer such that \texttt{Pointer2 -> Pointer1 -> Pointer}. \texttt{n1} is the fastest dimension.

\subsubsection*{Call}
\begin{verbatim}ptr = sf_boolalloc3 (n1, n2, n3);\end{verbatim}

\subsubsection*{Definition}
\begin{verbatim}
bool ***sf_boolalloc3 (size_t n1 /* fast dimension */, 
                       size_t n2 /* slower dimension */, 
                       size_t n3 /* slowest dimension */)
/*< bool 3-D allocation, out[0][0] points to a contiguous array >*/ 
{
   ...
}
\end{verbatim}

\subsubsection*{Input parameters}
\begin{desclist}{\tt }{\quad}[\tt ]
   \setlength\itemsep{0pt}
   \item[n1] number of elements in the fastest dimension (\texttt{size\_t}).
   \item[n2] number of elements in the slower dimension (\texttt{size\_t}).
   \item[n3] number of elements in the slower dimension (\texttt{size\_t}).
\end{desclist}

\subsubsection*{Output}
\begin{desclist}{\tt }{\quad}[\tt ]
   \setlength\itemsep{0pt}
   \item[ptr] a void pointer pointing to the allocated block of memory.
\end{desclist}




\subsection{{sf\_floatalloc2}}\label{sec:sf_floatalloc2}
Allocates a 2D array in the memory for the float type. It works just like \hyperref[sec:sf_floatalloc]{\texttt{sf\_floatalloc}} but does it for two dimensions. This is done by making a pointer point to another pointer, which in turn points to a particular column (or row) of an allocated 2D block of memory of size \texttt{n1*n2}. \texttt{n1} is the fastest dimension.

\subsubsection*{Call}
\begin{verbatim}ptr = sf_floatalloc2 (n1, n2);\end{verbatim}

\subsubsection*{Definition}
\begin{verbatim}
float **sf_floatalloc2 (size_t n1 /* fast dimension */, 
                        size_t n2 /* slow dimension */)
/*< float 2-D allocation, out[0] points to a contiguous array >*/ 
{
   ...
}
\end{verbatim}

\subsubsection*{Input parameters}
\begin{desclist}{\tt }{\quad}[\tt ]
   \setlength\itemsep{0pt}
   \item[n1] number of elements in the fastest dimension (\texttt{size\_t}).
   \item[n2] number of elements in the slower dimension (\texttt{size\_t}).
\end{desclist}

\subsubsection*{Output}
\begin{desclist}{\tt }{\quad}[\tt ]
   \setlength\itemsep{0pt}
   \item[ptr] a void pointer pointing to the allocated block of memory.
\end{desclist}




\subsection{{sf\_floatalloc3}}
Allocates a 3D array in the memory for the float type. It works just like \hyperref[sec:sf_floatalloc2]{\texttt{sf\_floatalloc2}} but does it for three dimensions. This is done by extending the same argument as for \texttt{sf\_floatalloc2} but this time making a pointer such that \texttt{Pointer2 -> Pointer1 -> Pointer}. \texttt{n1} is the fastest dimension.


\subsubsection*{Call}
\begin{verbatim}ptr = sf_floatalloc3 (n1, n2, n3);\end{verbatim}

\subsubsection*{Definition}
\begin{verbatim}
float ***sf_floatalloc3 (size_t n1 /* fast dimension */, 
                         size_t n2 /* slower dimension */, 
                         size_t n3 /* slowest dimension */)
/*< float 3-D allocation, out[0][0] points to a contiguous array >*/ 
{
   ...
}
\end{verbatim}

\subsubsection*{Input parameters}
\begin{desclist}{\tt }{\quad}[\tt ]
   \setlength\itemsep{0pt}
   \item[n1] number of elements in the fastest dimension (\texttt{size\_t}).
   \item[n2] number of elements in the slower dimension (\texttt{size\_t}).
   \item[n3] number of elements in the slower dimension (\texttt{size\_t}).
\end{desclist}

\subsubsection*{Output}
\begin{desclist}{\tt }{\quad}[\tt ]
   \setlength\itemsep{0pt}
   \item[ptr] a void pointer pointing to the allocated block of memory.
\end{desclist}




\subsection{{sf\_floatalloc4}}
Allocates a 4D array in the memory for the float type. It works just like \hyperref[sec:sf_floatalloc2]{\texttt{sf\_floatalloc2}} but does it for four dimensions. This is done by extending the same argument as for \texttt{sf\_floatalloc2} but this time making a pointer such that \texttt{Pointer3 -> Pointer2 -> Pointer1 -> Pointer}. \texttt{n1} is the fastest dimension.

\subsubsection*{Call}
\begin{verbatim}ptr = sf_floatalloc4 (n1, n2, n3, n4);\end{verbatim}

\subsubsection*{Definition}
\begin{verbatim}
float ****sf_floatalloc4 (size_t n1 /* fast dimension */, 
                          size_t n2 /* slower dimension */, 
                          size_t n3 /* slower dimension */, 
                          size_t n4 /* slowest dimension */)
/*< float 4-D allocation, out[0][0][0] points to a contiguous array >*/ 
{
   ...
}
\end{verbatim}

\subsubsection*{Input parameters}
\begin{desclist}{\tt }{\quad}[\tt ]
   \setlength\itemsep{0pt}
   \item[n1] number of elements in the fastest dimension (\texttt{size\_t}).
   \item[n2] number of elements in the slower dimension (\texttt{size\_t}).
   \item[n3] number of elements in the slower dimension (\texttt{size\_t}).
   \item[n4] number of elements in the slower dimension (\texttt{size\_t}).
\end{desclist}

\subsubsection*{Output}
\begin{desclist}{\tt }{\quad}[\tt ]
   \setlength\itemsep{0pt}
   \item[ptr] a void pointer pointing to the allocated block of memory.
\end{desclist}




\subsection{{sf\_floatalloc5}}
Allocates a 5D array in the memory for the float type. It works just like \hyperref[sec:sf_floatalloc2]{\texttt{sf\_floatalloc2}} but does it for four dimensions. This is done by extending the same argument as for \texttt{sf\_floatalloc2} but this time making a pointer such that \texttt{Pointer4 -> Pointer3 -> Pointer2 -> Pointer1 -> Pointer}. \texttt{n1} is the fastest dimension.


\subsubsection*{Call}
\begin{verbatim}ptr = sf_floatalloc5 (n1, n2, n3, n4, n5);\end{verbatim}


\subsubsection*{Definition}
\begin{verbatim}
float *****sf_floatalloc5 (size_t n1 /* fast dimension */, 
                           size_t n2 /* slower dimension */, 
                           size_t n3 /* slower dimension */, 
                           size_t n4 /* slower dimension */,
                           size_t n5 /* slowest dimension */)
/*< float 5-D allocation, out[0][0][0][0] points to a contiguous array >*/ 
{
   ...
}
\end{verbatim}

\subsubsection*{Input parameters}
\begin{desclist}{\tt }{\quad}[\tt ]
   \setlength\itemsep{0pt}
   \item[n1] number of elements in the fastest dimension (\texttt{size\_t}).
   \item[n2] number of elements in the slower dimension (\texttt{size\_t}).
   \item[n3] number of elements in the slower dimension (\texttt{size\_t}).
   \item[n4] number of elements in the slower dimension (\texttt{size\_t}).
   \item[n5] number of elements in the slower dimension (\texttt{size\_t}).
\end{desclist}

\subsubsection*{Output}
\begin{desclist}{\tt }{\quad}[\tt ]
   \setlength\itemsep{0pt}
   \item[ptr] a void pointer pointing to the allocated block of memory.
\end{desclist}




\subsection{{sf\_floatalloc6}}
Allocates a 6D array in the memory for the float type. It works just like \hyperref[sec:sf_floatalloc2]{\texttt{sf\_floatalloc2}} but does it for four dimensions. This is done by extending the same argument as for \texttt{sf\_floatalloc2} but this time making a pointer such that \texttt{Pointer5 -> Pointer4 -> Pointer3 -> Pointer2 -> Pointer1 -> Pointer}. \texttt{n1} is the fastest dimension.

\subsubsection*{Call}
\begin{verbatim}ptr = sf_floatalloc6 (n1, n2, n3, n4, n5, n6);\end{verbatim}

\subsubsection*{Definition}
\begin{verbatim}
float ******sf_floatalloc6 (size_t n1 /* fast dimension */, 
                            size_t n2 /* slower dimension */, 
                            size_t n3 /* slower dimension */, 
                            size_t n4 /* slower dimension */,
                            size_t n5 /* slower dimension */,
                            size_t n6 /* slowest dimension */)
/*< float 6-D allocation, out[0][0][0][0][0] points to a contiguous array >*/ 
{
   ...
}
\end{verbatim}

\subsubsection*{Input parameters}
\begin{desclist}{\tt }{\quad}[\tt ]
   \setlength\itemsep{0pt}
   \item[n1] number of elements in the fastest dimension (\texttt{size\_t}).
   \item[n2] number of elements in the slower dimension (\texttt{size\_t}).
   \item[n3] number of elements in the slower dimension (\texttt{size\_t}).
   \item[n4] number of elements in the slower dimension (\texttt{size\_t}).
   \item[n5] number of elements in the slower dimension (\texttt{size\_t}).
   \item[n6] number of elements in the slower dimension. Must be of type \texttt{size\_t}
\end{desclist}

\subsubsection*{Output}
\begin{desclist}{\tt }{\quad}[\tt ]
   \setlength\itemsep{0pt}
   \item[ptr] a void pointer pointing to the allocated block of memory.
\end{desclist}




\subsection{{sf\_intalloc2}}\label{sec:sf_intalloc2}
Allocates a 2D array in the memory for the float type. It works just like \hyperref[sec:sf_intalloc]{\texttt{sf\_intalloc}} but does it for two dimensions. This is done by making a pointer point to another pointer, which in turn points to a particular column (or row) of an allocated 2D block of memory of size \texttt{n1*n2}. \texttt{n1} is the fastest dimension.


\subsubsection*{Call}
\begin{verbatim}ptr = sf_intalloc2 (n1, n2);\end{verbatim}

\subsubsection*{Definition}
\begin{verbatim}
int **sf_intalloc2 (size_t n1 /* fast dimension */, 
                    size_t n2 /* slow dimension */)
/*< float 2-D allocation, out[0] points to a contiguous array >*/  
{
   ...
}
\end{verbatim}

\subsubsection*{Input parameters}
\begin{desclist}{\tt }{\quad}[\tt ]
   \setlength\itemsep{0pt}
   \item[n1] number of elements in the fastest dimension (\texttt{size\_t}).
   \item[n2] number of elements in the slower dimension (\texttt{size\_t}).
\end{desclist}

\subsubsection*{Output}
\begin{desclist}{\tt }{\quad}[\tt ]
   \setlength\itemsep{0pt}
   \item[ptr] a void pointer pointing to the allocated block of memory.
\end{desclist}





\subsection{{sf\_intalloc3}}
Allocates a 3D array in the memory for the float type. It works just like \hyperref[sec:sf_intalloc2]{\texttt{sf\_intalloc2}} but does it for three dimensions. This is done by extending the same argument as for \texttt{sf\_intalloc2} this time making a pointer such that \texttt{Pointer2 -> Pointer1 -> Pointer}. \texttt{n1} is the fastest dimension.

\subsubsection*{Call}
\begin{verbatim}ptr = sf_intalloc3 (n1, n2, n3);\end{verbatim}

\subsubsection*{Definition}
\begin{verbatim}
int ***sf_intalloc3 (size_t n1 /* fast dimension */, 
                     size_t n2 /* slower dimension */, 
                     size_t n3 /* slowest dimension */)
/*< int 3-D allocation, out[0][0] points to a contiguous array >*/ 
{
   ...
}
\end{verbatim}

\subsubsection*{Input parameters}
\begin{desclist}{\tt }{\quad}[\tt ]
   \setlength\itemsep{0pt}
   \item[n1] number of elements in the fastest dimension (\texttt{size\_t}).
   \item[n2] number of elements in the slower dimension (\texttt{size\_t}).
   \item[n3] number of elements in the slower dimension (\texttt{size\_t}).
\end{desclist}

\subsubsection*{Output}
\begin{desclist}{\tt }{\quad}[\tt ]
   \setlength\itemsep{0pt}
   \item[ptr] a void pointer pointing to the allocated block of memory.
\end{desclist}




\subsection{{sf\_intalloc4}}
Allocates a 4D array in the memory for the float type. It works just like \hyperref[sec:sf_intalloc2]{\texttt{sf\_intalloc2}} but does it for four dimensions. This is done by extending the same argument as for \texttt{sf\_intalloc2} but this time making a pointer such that \texttt{Pointer3 -> Pointer2 -> Pointer1 -> Pointer}. \texttt{n1} is the fastest dimension.

\subsubsection*{Call}
\begin{verbatim}ptr = sf_intalloc4 (n1, n2, n3, n4);\end{verbatim}

\subsubsection*{Definition}
\begin{verbatim}
int ****sf_intalloc4 (size_t n1 /* fast dimension */, 
                      size_t n2 /* slower dimension */, 
                      size_t n3 /* slower dimension */,
                      size_t n4 /* slowest dimension */ )
/*< int 4-D allocation, out[0][0][0] points to a contiguous array >*/ 
{
   ...
}
\end{verbatim}

\subsubsection*{Input parameters}
\begin{desclist}{\tt }{\quad}[\tt ]
   \setlength\itemsep{0pt}
   \item[n1] number of elements in the fastest dimension (\texttt{size\_t}).
   \item[n2] number of elements in the slower dimension (\texttt{size\_t}).
   \item[n3] number of elements in the slower dimension (\texttt{size\_t}).
   \item[n4] number of elements in the slower dimension (\texttt{size\_t}).
\end{desclist}

\subsubsection*{Output}
\begin{desclist}{\tt }{\quad}[\tt ]
   \setlength\itemsep{0pt}
   \item[ptr] a void pointer pointing to the allocated block of memory.
\end{desclist}





\subsection{{sf\_charalloc2}}
Allocates a 2D array in the memory for the float type. It works just like \hyperref[sec:sf_charalloc]{\texttt{sf\_charalloc}} but does it for two dimensions. This is done by making a pointer point to another pointer, which in turn points to a particular column (or row) of an allocated 2D block of memory of size \texttt{n1*n2}. \texttt{n1} is the fastest dimension.


\subsubsection*{Call}
\begin{verbatim}ptr = sf_charalloc2 (n1, n2);\end{verbatim}

\subsubsection*{Definition}
\begin{verbatim}
char **sf_charalloc2 (size_t n1 /* fast dimension */, 
                      size_t n2 /* slow dimension */) 
/*< char 2-D allocation, out[0] points to a contiguous array >*/
{
   ...
}
\end{verbatim}
\subsubsection*{Input parameters}
\begin{desclist}{\tt }{\quad}[\tt ]
   \setlength\itemsep{0pt}
   \item[n1] number of elements in the fastest dimension (\texttt{size\_t}).
   \item[n2] number of elements in the slower dimension (\texttt{size\_t}).
\end{desclist}

\subsubsection*{Output}
\begin{desclist}{\tt }{\quad}[\tt ]
   \setlength\itemsep{0pt}
   \item[ptr] a void pointer pointing to the allocated block of memory.
\end{desclist}




\subsection{{sf\_uncharalloc2}}\label{sec:sf_uncharalloc2}
Allocates a 2D array in the memory for the float type. It works just like \hyperref[sec:sf_ucharalloc]{\texttt{sf\_uncharalloc}} but does it for two dimensions. This is done by making a pointer point to another pointer, which in turn points to a particular column (or row) of an allocated 2D block of memory of size \texttt{n1*n2}. \texttt{n1} is the fastest dimension.

\subsubsection*{Call}
\begin{verbatim}ptr = sf_ucharalloc2 (n1, n2);\end{verbatim}

\subsubsection*{Definition}
\begin{verbatim}
unsigned char **sf_ucharalloc2 (size_t n1 /* fast dimension */, 
                                size_t n2 /* slow dimension */)
/*< unsigned char 2-D allocation, out[0] points to a contiguous array >*/
{
   ...
}
\end{verbatim}

\subsubsection*{Input parameters}
\begin{desclist}{\tt }{\quad}[\tt ]
   \setlength\itemsep{0pt}
   \item[n1] number of elements in the fastest dimension (\texttt{size\_t}).
   \item[n2] number of elements in the slower dimension (\texttt{size\_t}).
\end{desclist}

\subsubsection*{Output}
\begin{desclist}{\tt }{\quad}[\tt ]
   \setlength\itemsep{0pt}
   \item[ptr] a void pointer pointing to the allocated block of memory.
\end{desclist}





\subsection{{sf\_uncharalloc3}}
Allocates a 3D array in the memory for the float type. It works just like \hyperref[sec:sf_uncharalloc2]{\texttt{sf\_uncharalloc2}} but does it for three dimensions. This is done by extending the same argument as for \texttt{sf\_uncharalloc2} but this time making a pointer such that \texttt{Pointer2 -> Pointer1 -> Pointer}. \texttt{n1} is the fastest dimension.

\subsubsection*{Call}
\begin{verbatim}ptr = sf_ucharalloc3 (n1, n2, n3);\end{verbatim}

\subsubsection*{Definition}
\begin{verbatim}
unsigned char ***sf_ucharalloc3 (size_t n1 /* fast dimension */, 
                                 size_t n2 /* slower dimension */, 
                                 size_t n3 /* slowest dimension */)
/*< unsigned char 3-D allocation, out[0][0] points to a contiguous array >*/ 
{
   ...
}
\end{verbatim}

\subsubsection*{Input parameters}
\begin{desclist}{\tt }{\quad}[\tt ]
   \setlength\itemsep{0pt}
   \item[n1] number of elements in the fastest dimension (\texttt{size\_t}).
   \item[n2] number of elements in the slower dimension (\texttt{size\_t}).
   \item[n3] number of elements in the slower dimension (\texttt{size\_t}).
\end{desclist}

\subsubsection*{Output}
\begin{desclist}{\tt }{\quad}[\tt ]
   \setlength\itemsep{0pt}
   \item[ptr] a void pointer pointing to the allocated block of memory.
\end{desclist}




 % Convenience allocation programs
   \section{Simbol Table for parameters (simtab.c)}




\subsection{{sf\_simtab\_init}}\label{sec:sf_simtab_init}
Creates a table to store the parameters input either from command line or a file. It takes the required size (type \texttt{int}) of the table as input. The output is a pointer to the allocated table and it is of the defined data type \texttt{sf\_simtab}.

\subsubsection*{call}
\begin{verbatim}table = sf_simtab_init(size);\end{verbatim}

\subsubsection*{Definition}
\begin{verbatim}
sf_simtab sf_simtab_init(int size)
/*< Create simbol table. >*/
{
   ...
}
\end{verbatim}

\subsubsection*{Input parameters}
\begin{desclist}{\tt }{\quad}[\tt table]
   \setlength\itemsep{0pt}
   \item[size] size of the table to be allocated (\texttt{int}).
\end{desclist}

\subsubsection*{Output}
\begin{desclist}{\tt }{\quad}[\tt table]
   \setlength\itemsep{0pt}
   \item[table] a pointer of type \texttt{sf\_simtab} pointing to the allocated block of memory for the symbol table.
\end{desclist}




\subsection{{sf\_simtab\_close}}\label{sec:sf_simtab_close}
Frees the allocated space for the table.

\subsubsection*{Call}
\begin{verbatim}sf_simtab_close(table);\end{verbatim}

\subsubsection*{Definition}
\begin{verbatim}
void sf_simtab_close(sf_simtab table)
/*< Free allocated memory >*/
{
   ...
}
\end{verbatim}

\subsubsection*{Input parameters}
\begin{desclist}{\tt }{\quad}[\tt table]
   \setlength\itemsep{0pt}
   \item[table] the table whose allocated memory has to be deleted. Must be of type \texttt{sf\_simtab}.
\end{desclist}




\subsection{{sf\_simtab\_enter}}\label{sec:sf_simtab_enter}
Enters a value in the table, which was created by \texttt{sf\_simtab\_init}. In the input it must be told which table to enter the value in, this is the first input argument and is of type \texttt{sf\_simtab}. The second and the third arguments are the pointers of \texttt{const char*} type. The first one points to \texttt{key}, which would be the name of the argument from command line or file. Second argument is the pointer to the value to be input.

\subsubsection*{Call}
\begin{verbatim}sf_simtab_enter(table, key, val);\end{verbatim}

\subsubsection*{Definition}
\begin{verbatim}
void sf_simtab_enter(sf_simtab table, const char *key, const char* val)
/*< Add an entry key=val to the table >*/
{
    ...
}
\end{verbatim}

\subsubsection*{Input parameters}
\begin{desclist}{\tt }{\quad}[\tt table]
   \setlength\itemsep{0pt}
   \item[table] the table in which the the key value is to be stored. Must be of type \texttt{sf\_simtab}.
   \item[key]   pointer to the name of the key value to be input (\texttt{const char*}).
   \item[val]   pointer to the key value to be input (\texttt{const char*}).
\end{desclist}




\subsection{{sf\_simtab\_get}}\label{sec:sf_simtab_get}
Extracts the value of the input key from the symbol table. It is used in other functions such as \texttt{sf\_simtab\_getint}.

\subsubsection*{Call}
\begin{verbatim}val = sf_simtab_get(table, key); \end{verbatim}

\subsubsection*{Input parameters}
\begin{desclist}{\tt }{\quad}[\tt table]
   \setlength\itemsep{0pt}
   \item[table] the table from which the vale has to be extracted. Must be of type \texttt{sf\_simtab}.
   \item[key]   the name of the entry which has to be extracted (\texttt{const char*}).
\end{desclist}

\subsubsection*{Output}
\begin{desclist}{\tt }{\quad}[\tt NULL]
   \setlength\itemsep{0pt}
   \item[val] pointer of type \texttt{char} to the desired key value stored in the table. This is the output in case there is a match between the required key and a key in the table. If there is no match between the required key and the key stored in the table, then \texttt{NULL} is returned.
\end{desclist}




\subsection{{sf\_simtab\_getint}}\label{sec:sf_simtab_getint}
Extracts an integer from the table. If the extraction is successful returns a boolean true, otherwise returns a false.

\subsubsection*{Call}
\begin{verbatim}success = sf_simtab_getint (table, key, par);\end{verbatim}

\subsubsection*{Definition}
\begin{verbatim}
bool sf_simtab_getint (sf_simtab table, const char* key,/*@out@*/ int* par)
/*< extract an int parameter from the table >*/
{
   ...
}
\end{verbatim}

\subsubsection*{Input parameters}
\begin{desclist}{\tt }{\quad}[\tt table]
   \setlength\itemsep{0pt}
   \item[table] the table from which the vale has to be extracted. Must be of type \texttt{sf\_simtab}.
   \item[key]   the name of the entry which has to be extracted (\texttt{const char*}).
   \item[par]   pointer to the integer variable where the extracted value is to be copied.
\end{desclist}

\subsubsection*{Output}
\begin{desclist}{\tt }{\quad}[\tt success]
   \setlength\itemsep{0pt}
   \item[success] a boolean value. It is \texttt{true}, if the extraction was successful and \texttt{false} otherwise.
\end{desclist}




\subsection{{sf\_simtab\_getlargeint}}\label{sec:sf_simtab_getlargeint}
Extracts a large integer from the table. If the extraction is successful, it returns a boolean true, otherwise a false. 

\subsubsection*{Call}
\begin{verbatim}success = sf_simtab_getlargeint (table, key, par);\end{verbatim}

\subsubsection*{Definition}
\begin{verbatim}
bool sf_simtab_getlargeint (sf_simtab table, const char* key,/*@out@*/ off_t* pa
r)
/*< extract a sf_largeint parameter from the table >*/
{
   ...
}
\end{verbatim}

\subsubsection*{Input parameters}
\begin{desclist}{\tt }{\quad}[\tt table]
   \setlength\itemsep{0pt}
   \item[table] the table from which the vale has to be extracted. Must be of type \texttt{sf\_simtab}.
   \item[key]   the name of the entry which has to be extracted (\texttt{const char*}).
   \item[par]   pointer to the large integer variable where the extracted value is to be copied.
\end{desclist}

\subsubsection*{Output}
\begin{desclist}{\tt }{\quad}[\tt success]
   \setlength\itemsep{0pt}
   \item[success] a boolean value. It is \texttt{true}, if the extraction was successful and \texttt{false} otherwise.
\end{desclist}




\subsection{{sf\_simtab\_getfloat}}\label{sec:sf_simtab_getfloat}
Extracts a float value from the table. If the extraction is successful, it returns a boolean true, otherwise a false. 

\subsubsection*{Call}
\begin{verbatim}success = sf_simtab_getfloat (table, key, par);\end{verbatim}

\subsubsection*{Definition}
\begin{verbatim}
bool sf_simtab_getfloat (sf_simtab table, const char* key,/*@out@*/ float* par)
/*< extract a float parameter from the table >*/
{
   ...
}
\end{verbatim}

\subsubsection*{Input parameters}
\begin{desclist}{\tt }{\quad}[\tt table]
   \setlength\itemsep{0pt}
   \item[table] the table from which the vale has to be extracted. Must be of type \texttt{sf\_simtab}.
   \item[key]   the name of the entry which has to be extracted (\texttt{const char*}).
   \item[par]   pointer to the float type value variable where the extracted value is to be copied.
\end{desclist}

\subsubsection*{Output}
\begin{desclist}{\tt }{\quad}[\tt success]
   \setlength\itemsep{0pt}
   \item[success] a boolean value. It is \texttt{true}, if the extraction was successful and \texttt{false} otherwise successfully.
\end{desclist}




\subsection{{sf\_simtab\_getdouble}}\label{sec:sf_simtab_getdouble}
Extracts a double type value from the table. If the extraction is successful, it returns a boolean true, otherwise a false. 

\subsubsection*{Call}
\begin{verbatim}success = sf_simtab_getdouble (table, key, par);\end{verbatim}

\subsubsection*{Definition}
\begin{verbatim}
bool sf_simtab_getdouble (sf_simtab table, const char* key,/*@out@*/ double* par
)
/*< extract a double parameter from the table >*/
{
   ...
}
\end{verbatim}

\subsubsection*{Input parameters}
\begin{desclist}{\tt }{\quad}[\tt table]
   \setlength\itemsep{0pt}
   \item[table] the table from which the vale has to be extracted. Must be of type \texttt{sf\_simtab}.
   \item[key]   the name of the entry which has to be extracted (\texttt{const char*}).
   \item[par]   pointer to the double type value variable where the extracted value is to be copied.
\end{desclist}

\subsubsection*{Output}
\begin{desclist}{\tt }{\quad}[\tt success]
   \setlength\itemsep{0pt}
   \item[success] a boolean value. It is \texttt{true}, if the extraction was successful and \texttt{false} otherwise.
\end{desclist}




\subsection{{sf\_simtab\_getfloats}}\label{sec:sf_simtab_getfloats}
Extracts an array of float values from the table. If the extraction is successful, it returns a boolean \texttt{true}, otherwise a \texttt{false}. 

\subsubsection*{Call}
\begin{verbatim} success = sf_simtab_getfloats (table, key, par, n);\end{verbatim}

\subsubsection*{Definition}
\begin{verbatim}
bool sf_simtab_getfloats (sf_simtab table, const char* key,
                          /*@out@*/ float* par,size_t n)
/*< extract a float array parameter from the table >*/
{
   ... 
}
\end{verbatim}

\subsubsection*{Input parameters}
\begin{desclist}{\tt }{\quad}[\tt table]
   \setlength\itemsep{0pt}
   \item[table] the table from which the vale has to be extracted. Must be of type \texttt{sf\_simtab}.
   \item[key]   the name of the \texttt{float} array which has to be extracted (\texttt{const char*}).
   \item[par]   pointer to the array of \texttt{float} type value variable where the extracted value id to be copied.
   \item[n]     size of the array to be extracted (\texttt{size\_t}).
\end{desclist}

\subsubsection*{Output}
\begin{desclist}{\tt }{\quad}[\tt success]
   \setlength\itemsep{0pt}
   \item[success] a boolean value. It is \texttt{true}, if the extraction was successful and \texttt{false} otherwise.
\end{desclist}




\subsection{{sf\_simtab\_getstring}}\label{sec:sf_simtab_getstring}
Extracts a string pointed by the input key from the symbol table. If the value is \texttt{NULL} it will return \texttt{NULL}, otherwise it will allocate a new block of memory of char type and copy the memory block from the table to the new block and return a pointer to the newly allocated block of memory. 

\subsubsection*{Call}
\begin{verbatim}string = sf_simtab_getstring (table, key);\end{verbatim}

\subsubsection*{Definition}
\begin{verbatim}
char* sf_simtab_getstring (sf_simtab table, const char* key) 
/*< extract a string parameter from the table >*/
{
   ...
}
\end{verbatim}

\subsubsection*{Input parameters}
\begin{desclist}{\tt }{\quad}[\tt table]
   \setlength\itemsep{0pt}
   \item[table] the table from which the string has to be extracted. Must be of type \texttt{sf\_simtab}.
   \item[key]   the name of the string which has to be extracted (\texttt{const char*}).
\end{desclist}

\subsubsection*{Output}
\begin{desclist}{\tt }{\quad}[\tt success]
   \setlength\itemsep{0pt}
   \item[string] a pointer to allocated block of memory containing a string of characters. 
\end{desclist}




\subsection{{sf\_simtab\_getbool}}\label{sec:sf_simtab_getbool}
Extracts a boolean value from the table. If the extraction is successful, it returns a boolean \texttt{true}, otherwise a \texttt{false}.

\subsubsection*{Call}
\begin{verbatim}success = sf_simtab_getbool (table, key, par);\end{verbatim}

\subsubsection*{Definition}
\begin{verbatim}
bool sf_simtab_getbool (sf_simtab table, const char* key,/*@out@*/ bool *par)
/*< extract a bool parameter from the table >*/
{
   ...
}
\end{verbatim}

\subsubsection*{Input parameters}
\begin{desclist}{\tt }{\quad}[\tt table]
   \setlength\itemsep{0pt}
   \item[table] the table from which the value has to be extracted. Must be of type \texttt{sf\_simtab}.
   \item[key]   the name of the entry which has to be extracted (\texttt{const char*}).
   \item[par]   pointer to the bool variable where the extracted value is to be copied.
\end{desclist}

\subsubsection*{Output}
\begin{desclist}{\tt }{\quad}[\tt success]
   \setlength\itemsep{0pt}
   \item[success] a boolean value. It is \texttt{true}, if the extraction was successful and \texttt{false} otherwise.
\end{desclist}




\subsection{{sf\_simtab\_getbools}}\label{sec:sf_simtab_getbools}
Extracts an array of boolean values from the table. If the extraction is successful, it returns a boolean \texttt{true}, otherwise a \texttt{false}. 

\subsubsection*{Call}
\begin{verbatim}success = sf_simtab_getbools (table, key, par, n);\end{verbatim}

\subsubsection*{Definition}
\begin{verbatim}
sf_simtab_getbools (sf_simtab table, const char* key,/*@out@*/bool *par,size_t n)
/*< extract a bool array parameter from the table >*/
{
   ...
}
\end{verbatim}

\subsubsection*{Input parameters}
\begin{desclist}{\tt }{\quad}[\tt table]
   \setlength\itemsep{0pt}
   \item[table] the table from which the vale has to be extracted. Must be of type \texttt{sf\_simtab}.
   \item[key]   the name of the boolean array which has to be extracted. Must be of XXXXXXXXX  pointer to the array of \texttt{bool} type value variable where the extracted value is to be copied. 
   \item[n]     size of the array to be extracted (\texttt{size\_t}).
\end{desclist}

\subsubsection*{Output}
\begin{desclist}{\tt }{\quad}[\tt success]
   \setlength\itemsep{0pt}
   \item[success] a boolean value. It is \texttt{true}, if the extraction was successful and \texttt{false} otherwise.
\end{desclist}




\subsection{{sf\_simtab\_getints}}\label{sec:sf_simtab_getints}
Extracts an array of integer values from the table. If the extraction is successful, it returns a boolean \texttt{true}, otherwise a \texttt{false}. 

\subsubsection*{Call}
\begin{verbatim}success = sf_simtab_getints (table, key, par, n);\end{verbatim}

\subsubsection*{Definition}
\begin{verbatim}
bool sf_simtab_getints (sf_simtab table, const char* key,
                        /*@out@*/ int *par,size_t n)
/*< extract an int array parameter from the table >*/
{    
   ...
}
\end{verbatim}

\subsubsection*{Input parameters}
\begin{desclist}{\tt }{\quad}[\tt table]
   \setlength\itemsep{0pt}
   \item[table] the table from which the vale has to be extracted. Must be of type \texttt{sf\_simtab}.
   \item[key]   the name of the integer array which has to be extracted (\texttt{const char*}).
   \item[par]   pointer to the array of integer type value variable where the extracted  value id to be copied.
   \item[n]     size of the array to be extracted. Must be of \texttt{size\_t}.
\end{desclist}

\subsubsection*{Output}
\begin{desclist}{\tt }{\quad}[\tt success]
   \setlength\itemsep{0pt}
   \item[success] a boolean value. It is \texttt{true}, if the extraction was successful and \texttt{false} otherwise.
\end{desclist}




\subsection{{sf\_simtab\_getstrings}}\label{sec:sf_simtab_getstrings}
Extracts an array of strings from the table. is successful, it returns a boolean \texttt{true}, otherwise a \texttt{false}.

\subsubsection*{Call}
\begin{verbatim}success = sf_simtab_getstrings (table, key, par, n);\end{verbatim}

\subsubsection*{Definition}
\begin{verbatim}
bool sf_simtab_getstrings (sf_simtab table, const char* key,
                           /*@out@*/ char **par,size_t n)
/*< extract a string array parameter from the table >*/
{    
   ...
}
\end{verbatim}

\subsubsection*{Input parameters}
\begin{desclist}{\tt }{\quad}[\tt table]
   \setlength\itemsep{0pt}
   \item[table] the table from which the vale has to be extracted. Must be of type \texttt{sf\_simtab}.
   \item[key]   the name of the string array which has to be extracted (\texttt{const char*}).
   \item[par]   pointer to the pointer to array of integer type value variable where the extracted value is to be copied.
   \item[n]     size of the array to be extracted. Must be of \texttt{size\_t}.
\end{desclist}

\subsubsection*{Output}
\begin{desclist}{\tt }{\quad}[\tt success]
   \setlength\itemsep{0pt}
   \item[success] a boolean value. It is \texttt{true}, if the extraction was successful and \texttt{false} otherwise.
\end{desclist}




\subsection{{sf\_simtab\_put}}\label{sec:sf_simtab_put}
Writes a new key together with its value to the symbol table. The new entry must be in the form \texttt{key=val} and must be of the \texttt{const char*} type, that is, this function must be given a pointer to \texttt{key=val}. Since the type of the pointer is \texttt{const char*} this can be a direct input from the command line and in that case the pointer will be \texttt{acgv[n]} where \texttt{n} specifies the position in the command line.  

\subsubsection*{Call}
\begin{verbatim}sf_simtab_put (table, keyval);\end{verbatim}

\subsubsection*{Definition}
\begin{verbatim}
void sf_simtab_put (sf_simtab table, const char *keyval) 
/*< put a key=val string to the table >*/
{
   ...
}
\end{verbatim}

\subsubsection*{Input parameters}
\begin{desclist}{\tt }{\quad}[\tt keyval]
   \setlength\itemsep{0pt}
   \item[table] the table in which the value has to be entered. Must be of type \texttt{sf\_simtab}. 
   \item[keyval] pointer to \texttt{key=val} which is to be entered.
\end{desclist}




\subsection{{sf\_simtab\_input}}\label{sec:sf_simtab_input}
Inputs a table from one file and copies it into another and also adds the new entry into the internal table using \hyperref[sec:sf_simtab_put]{\texttt{sf\_simtab\_put}}.

\subsubsection*{Call}
\begin{verbatim}sf_simtab_input ( table, fp, out);\end{verbatim}

\subsubsection*{Definition}
\begin{verbatim}
void sf_simtab_input (sf_simtab table, FILE* fp, FILE* out) 
/*< extract parameters from a file >*/
{
   ...
}
\end{verbatim}

\subsubsection*{Input parameters}
\begin{desclist}{\tt }{\quad}[\tt table]
   \setlength\itemsep{0pt}
   \item[table] the table in which the value has to be entered. Must be of type \texttt{sf\_simtab}. 
   \item[fp]    pointer to the file from which the parameter is to be read. It must be of type\texttt{FILE*}.
   \item[out]   pointer to the file in which the parameter is to be written. It must be of type FILE*.
\end{desclist}




\subsection{{sf\_simtab\_output}}\label{sec:sf_simtab_output}
Reads the parameters from the internal table and writes them to a file.

\subsubsection*{Call}
\begin{verbatim}sf_simtab_output ( table, fp);\end{verbatim}

\subsubsection*{Definition}
\begin{verbatim}
void sf_simtab_output (sf_simtab table, FILE* fp) 
/*< output parameters to a file >*/
{
   ...
}
\end{verbatim}

\subsubsection*{Input parameters}
\begin{desclist}{\tt }{\quad}[\tt table]
   \setlength\itemsep{0pt}
   \item[table] the table in which the value has to be entered. Must be of type \texttt{sf\_simtab}. 
   \item[fp]    pointer to the file in which the parameter is to be written. It must be of type \texttt{FILE*}.
\end{desclist}




       % Simbol Table for parameters
   \section{Parameter handling (getpar.c)}




\subsection{{sf\_stdin}}
Checks whether there is an input in the command line, if not it returns a false. It reads the first character in the file: if it is an \texttt{EOF}, \texttt{false} is returned and if not, then \texttt{true} is the return value. It takes no input parameters and returns a boolean value.

\subsubsection*{Call}
\begin{verbatim}hasinp sf_stdin();\end{verbatim}

\subsubsection*{Definition}
\begin{verbatim}
bool sf_stdin(void)
/*< returns true if there is an input in stdin >*/
{
   ...
}
\end{verbatim}




\subsection{{sf\_init}}\label{sec:sf_init}
Initializes a parameter table which is created using \hyperref[sec:sf_simtab_init]{\texttt{sf\_simtab\_init}}.
The input arguments are same as used in the main function in c when there is some arguments are to be input from the command line.

\subsubsection*{Input parameters}
\begin{desclist}{}{\quad}[\tt argv12]
   \setlength\itemsep{0pt}
   \item[\texttt{argc}] size of the table to be allocated (\texttt{int}). 
   \item[\texttt{argv[]}] pointer to the character array from the command line (\texttt{char*}).
\end{desclist}

\subsubsection*{Call}
\begin{verbatim}sf_init(argc, argv[]);\end{verbatim}

\subsubsection*{Definition}
\begin{verbatim}
void sf_init(int argc,char *argv[]) 
/*< initialize parameter table from command-line arguments >*/
{
   ...
}
\end{verbatim}




\subsection{{sf\_par\_close}}
Frees the allocated space for the table. It uses \hyperref[sec:sf_simtab_close]{\texttt{sf\_simtab\_close}} to close the table. It does not take any input parameters but passes a pointer \texttt{pars} defined by \texttt{sf\_init}.

\subsubsection*{Call}
\begin{verbatim}sf_parclose ();\end{verbatim}

\subsubsection*{Definition}
\begin{verbatim}
void sf_parclose (void)
/*< close parameter table and free space >*/
{
   ...
}
\end{verbatim}




\subsection{{sf\_parout}}
Reads the parameters from the internal table and writes them to a file. It uses \hyperref[sec:sf_simtab_output]{\texttt{sf\_simtab\_output}}. It takes a pointer to the file, in which the parameters are to be written.

\subsubsection*{Call}
\begin{verbatim}sf_parout (file);\end{verbatim}

\subsubsection*{Definition}
\begin{verbatim}
void sf_parout (FILE *file)
/*< write the parameters to a file >*/
{
   ...
}
\end{verbatim}

\subsubsection*{Input parameters}
\begin{desclist}{\tt }{\quad}[\tt parout]
   \setlength\itemsep{0pt}
   \item[parout] the table in which the value has to be entered (\texttt{FILE*}).
\end{desclist}




\subsection{{sf\_getprog}}
Outputs a pointer of \texttt{char} type to the name of the current running program. The pointer it returns is assigned a value in \hyperref[sec:sf_init]{\texttt{sf\_init}}.

\subsubsection*{Call}
\begin{verbatim}
prog = sf_getprog ();
\end{verbatim}

\subsubsection*{Definition}
\begin{verbatim}
char* sf_getprog (void) 
/*< returns name of the running program >*/ 
{
   ...
}
\end{verbatim}

\subsubsection*{Output}
\begin{desclist}{\tt }{\quad}[\tt prog]
   \setlength\itemsep{0pt}
   \item[prog] pointer to an array which contains the current program name (\texttt{char}).
\end{desclist}




\subsection{{sf\_getuser}}
Outputs a pointer of \texttt{char} type to the name of the current user. The pointer it returns is assigned a value in \hyperref[sec:sf_init]{\texttt{sf\_init}}.

\subsubsection*{Call}
\begin{verbatim}user = sf_getuser ();\end{verbatim}

\subsubsection*{Definition}
\begin{verbatim}
char* sf_getuser (void) 
/*< returns user name >*/
{
   ...
}
\end{verbatim}

\subsubsection*{Output}
\begin{desclist}{\tt }{\quad}[\tt user]
   \setlength\itemsep{0pt}
   \item[user] pointer to an array which contains the user name (\texttt{char}).
\end{desclist}




\subsection{{sf\_gethost}}
Outputs a pointer of \texttt{char} type to the name of the current host. The pointer it returns is assigned a value in \hyperref[sec:sf_init]{\texttt{sf\_init}}.

\subsubsection*{Call}
\begin{verbatim}host = sf_gethost ();\end{verbatim}

\subsubsection*{Definition}
\begin{verbatim}
char* sf_gethost (void) 
/*< returns host name >*/
{
   ...
}
\end{verbatim}

\subsubsection*{Output}
\begin{desclist}{\tt }{\quad}[\tt host]
   \setlength\itemsep{0pt}
   \item[host] pointer to an array which contains the host name (\texttt{char}).
\end{desclist}




\subsection{{sf\_getcdir}}
Outputs a pointer of \texttt{char} type to the name of the current working directory. The pointer it returns is assigned a value in \hyperref[sec:sf_init]{\texttt{sf\_init}}.

\subsubsection*{Call}
\begin{verbatim}cdir = sf_getcdir ();\end{verbatim}

\subsubsection*{Definition}
\begin{verbatim}
char* sf_getcdir (void) 
/*< returns current directory >*/
{
   ...
}
\end{verbatim}

\subsubsection*{Output}
\begin{desclist}{\tt }{\quad}[\tt cdir]
   \setlength\itemsep{0pt}
   \item[cdir] pointer to an array which contains the current working directory (\texttt{char}).
\end{desclist}




\subsection{{sf\_getint}}
Extracts an integer from the command line. If the extraction is successful, it returns a \texttt{true}, otherwise a \texttt{false}.  It uses \hyperref[sec:sf_simtab_getint]{\texttt{sf\_simtab\_getint}}.

\subsubsection*{Call}
\begin{verbatim}success = sf_getint (key, par);\end{verbatim}

\subsubsection*{Definition}
\begin{verbatim}
bool sf_getint (const char* key,/*@out@*/ int* par) 
/*< get an int parameter from the command line >*/
{
   ...
}
\end{verbatim}

\subsubsection*{Input parameters}
\begin{desclist}{\tt }{\quad}[\tt key]
   \setlength\itemsep{0pt}
   \item[key] the name of the entry which has to be extracted (\texttt{const char*}).
   \item[par] pointer to the integer variable where the extracted value is to be copied.
\end{desclist}

\subsubsection*{Output}
\begin{desclist}{\tt }{\quad}[\tt success]
   \setlength\itemsep{0pt}
   \item[success] a boolean value. It is \texttt{true}, if the extraction was successful and \texttt{false} otherwise. 
\end{desclist}




\subsection{{sf\_getlargeint}}
Extracts a large integer from the command line. If the extraction is successful, it returns a \texttt{true}, otherwise a \texttt{false}.  
It uses \hyperref[sec:sf_simtab_getlargeint]{\texttt{sf\_simtab\_getlargeint}}.

\subsubsection*{Call}
\begin{verbatim}sf_getlargeint (key, par);\end{verbatim}

\subsubsection*{Definition}
\begin{verbatim}
bool sf_getlargeint (const char* key,/*@out@*/ off_t* par) 
/*< get a large int parameter from the command line >*/
{
   ...
}
\end{verbatim}

\subsubsection*{Input parameters}
\begin{desclist}{\tt }{\quad}[\tt key]
   \setlength\itemsep{0pt}
   \item[key] the name of the entry which has to be extracted (\texttt{const char*}).
   \item[par] pointer to the large integer variable where the extracted value is to be copied. Must be of type \texttt{off\_t} which is defined in the header \texttt{<sys/types.h>}.
\end{desclist}

\subsubsection*{Output}
\begin{desclist}{\tt }{\quad}[\tt success]
   \setlength\itemsep{0pt}
   \item[success]  a boolean value. It is \texttt{true}, if the extraction was successful and \texttt{false} otherwise.
\end{desclist}




\subsection{{sf\_getints}}
Extracts an array of integer values from the command line. If the extraction is successful, it returns a \texttt{true}, otherwise a \texttt{false}. 

\subsubsection*{Call}
\begin{verbatim}success = sf_getints (key, par, n);\end{verbatim}

\subsubsection*{Definition}
\begin{verbatim}
bool sf_getints (const char* key,/*@out@*/ int* par,size_t n) 
/*< get an int array parameter (comma-separated) from the command line >*/
{
   ...
} 
\end{verbatim}

\subsubsection*{Input parameters}
\begin{desclist}{\tt }{\quad}[\tt key]
   \setlength\itemsep{0pt}
   \item[key] the name of the integer array which has to be extracted (\texttt{const char*}).
   \item[par] pointer to the array of integer type value variable where the extracted value id to be copied. 
   \item[n] size of the array to be extracted. Must be of \texttt{size\_t}.
\end{desclist}

\subsubsection*{Output}
\begin{desclist}{\tt }{\quad}[\tt success]
   \setlength\itemsep{0pt}
   \item[success]  a boolean value. It is \texttt{true}, if the extraction was successful and \texttt{false} otherwise. 
\end{desclist}




\subsection{{sf\_getfloat}}
Extracts a float value from the command line. If the extraction is successful, it returns a \texttt{true}, otherwise a \texttt{false}.  

\subsubsection*{Call}
\begin{verbatim}success = sf_getfloat (key, par);\end{verbatim}

\subsubsection*{Definition}
\begin{verbatim}
bool sf_getfloat (const char* key,/*@out@*/ float* par) 
/*< get a float parameter from the command line >*/
{
   ...
}
\end{verbatim}

\subsubsection*{Input parameters}
\begin{desclist}{\tt }{\quad}[\tt key]
   \setlength\itemsep{0pt}
   \item[key] the name of the entry which has to be extracted (\texttt{const char*}).
   \item[par] pointer to the float type value variable where the extracted value is to be copied.
\end{desclist}

\subsubsection*{Output}
\begin{desclist}{\tt }{\quad}[\tt success]
   \setlength\itemsep{0pt}
   \item[success]  a boolean value. It is \texttt{true}, if the extraction was successful and \texttt{false} otherwise. 
\end{desclist}




\subsection{{sf\_getdouble}}
Extracts a double type value from the command line. If the extraction is successful, it returns a \texttt{true}, otherwise a \texttt{false}.  

\subsubsection*{Call}
\begin{verbatim}success = sf_getdouble (key, par);\end{verbatim}

\subsubsection*{Definition}
\begin{verbatim}
bool sf_getdouble (const char* key,/*@out@*/ double* par) 
/*< get a double parameter from the command line >*/
{
   ...
}
\end{verbatim}

\subsubsection*{Input parameters}
\begin{desclist}{\tt }{\quad}[\tt table]
   \setlength\itemsep{0pt}
   \item[table] the table from which the vale has to be extracted. Must be of type \texttt{sf\_simtab}. 
   \item[key]   the name of the entry which has to be extracted (\texttt{const char*}).
   \item[par]   pointer to the double type value variable where the extracted value is to be copied.
\end{desclist}

\subsubsection*{Output}
\begin{desclist}{\tt }{\quad}[\tt success]
   \setlength\itemsep{0pt}
   \item[success]  a boolean value. It is \texttt{true}, if the extraction was successful and \texttt{false} otherwise. 
\end{desclist}




\subsection{{sf\_getfloats}}
Extracts an array of float values from the command line. If the extraction is successful, it returns a \texttt{true}, otherwise a \texttt{false}.  It uses \hyperref[sec:sf_simtab_getfloats]{\texttt{sf\_simtab\_getfloats}}.

\subsubsection*{Call}
\begin{verbatim}success = sf_getfloats (key, n);\end{verbatim}

\subsubsection*{Definition}
\begin{verbatim}
bool sf_getfloats (const char* key,/*@out@*/ float* par,size_t n) 
/*< get a float array parameter from the command line >*/
{
   ... 
}
\end{verbatim}

\subsubsection*{Input parameters}
\begin{desclist}{\tt }{\quad}[\tt key]
   \setlength\itemsep{0pt}
   \item[key] the name of the float array which has to be extracted (\texttt{const char*}).
   \item[par] pointer to the array of float type value variable where the extracted value id to be copied. 
   \item[n] size of the array to be extracted (\texttt{size\_t}).
\end{desclist}

\subsubsection*{Output}
\begin{desclist}{\tt }{\quad}[\tt success]
   \setlength\itemsep{0pt}
   \item[success]  a boolean value. It is \texttt{true}, if the extraction was successful and \texttt{false} otherwise.
\end{desclist}




\subsection{{sf\_getstring}}
Extracts a string pointed by the input key from the command line. If the value is \texttt{NULL} it will return \texttt{NULL}, otherwise it will allocate a new block of memory of \texttt{char} type and copy the memory block from the table to the new block and return a pointer to the newly allocated block of memory. It uses \hyperref[sec:sf_simtab_getstring]{\texttt{sf\_simtab\_getstring}}.

\subsubsection*{Call}
\begin{verbatim}string = sf_getstring (key);\end{verbatim}

\subsubsection*{Definition}
\begin{verbatim}
char* sf_getstring (const char* key) 
/*< get a string parameter from the command line >*/
{
   ...
}
\end{verbatim}

\subsubsection*{Input parameters}
\begin{desclist}{\tt }{\quad}[\tt key]
   \setlength\itemsep{0pt}
   \item[key] the name of the string which has to be extracted (\texttt{const char*}).
\end{desclist}

\subsubsection*{Output}
\begin{desclist}{\tt }{\quad}[\tt string]
   \setlength\itemsep{0pt}
   \item[string] a pointer to allocated block of memory containing a string of characters. 
\end{desclist}




\subsection{{sf\_getstrings}}
Extracts an array of strings from the command line. If the extraction is successful, it returns a \texttt{true}, otherwise a \texttt{false}. 

\subsubsection*{Call}
\begin{verbatim}success = sf_getstrings (key, par, n);\end{verbatim}

\subsubsection*{Definition}
\begin{verbatim}
bool sf_getstrings (const char* key,/*@out@*/ char** par,size_t n) 
/*< get a string array parameter from the command line >*/
{
   ...
}
\end{verbatim}

\subsubsection*{Input parameters}
\begin{desclist}{\tt }{\quad}[\tt key]
   \setlength\itemsep{0pt}
   \item[key] the name of the string array which has to be extracted (\texttt{const char*}).
   \item[par] pointer to the pointer to array of integer type value variable where the extracted value is to be copied. 
   \item[n] size of the array to be extracted. Must be of \texttt{size\_t}.
\end{desclist}

\subsubsection*{Output}
\begin{desclist}{\tt }{\quad}[\tt success]
   \setlength\itemsep{0pt}
   \item[success]  a boolean value. It is \texttt{true}, if the extraction was successful and \texttt{false} otherwise.
\end{desclist}




\subsection{{sf\_getbool}}
Extracts a boolean value from the command line. If the extraction is successful, it returns a \texttt{true}, otherwise a \texttt{false}. 

\subsubsection*{Call}
\begin{verbatim}success = sf_getbool (key, par);\end{verbatim}

\subsubsection*{Definition}
\begin{verbatim}
bool sf_getbool (const char* key,/*@out@*/ bool* par)
/*< get a bool parameter from the command line >*/
{
   ...
}
\end{verbatim}

\subsubsection*{Input parameters}
\begin{desclist}{\tt }{\quad}[\tt key]
   \setlength\itemsep{0pt}
   \item[key] the name of the entry which has to be extracted (\texttt{const char*}).
   \item[par] pointer to the bool variable where the extracted value is to be copied.
\end{desclist}

\subsubsection*{Output}
\begin{desclist}{\tt }{\quad}[\tt success]
   \setlength\itemsep{0pt}
   \item[success]  a boolean value. It is \texttt{true}, if the extraction was successful and \texttt{false} otherwise.
\end{desclist}




\subsection{{sf\_getbools}}
Extracts an array of bool values from the command line. If the extraction is successful, it returns a \texttt{true}, otherwise a \texttt{false}. It uses \hyperref[sec:sf_simtab_getbools]{\texttt{sf\_simtab\_getbools}}.

\subsubsection*{Call}
\begin{verbatim}success = sf_getbools (key, par, n);\end{verbatim}

\subsubsection*{Definition}
\begin{verbatim}
bool sf_getbools (const char* key,/*@out@*/ bool* par,size_t n) 
/*< get a bool array parameter from the command line >*/
{
    return sf_simtab_getbools(pars,key,par,n);
} 
\end{verbatim}

\subsubsection*{Input parameters}
\begin{desclist}{\tt }{\quad}[\tt par]
   \setlength\itemsep{0pt}
   \item[key] the name of the bool array which has to be extracted (\texttt{const char*}).
   \item[par] pointer to the array of bool type value variable where the extracted value id to be copied. 
   \item[n] size of the array to be extracted. Must be of \texttt{size\_t}.
\end{desclist}

\subsubsection*{Output}
\begin{desclist}{\tt }{\quad}[\tt success]
   \setlength\itemsep{0pt}
   \item[success]  a boolean value. It is \texttt{true}, if the extraction was successful and \texttt{false} otherwise. 
\end{desclist}




\subsection{{sf\_getpars}}
This function  returns a pointer to the parameter table which must have been initialized earlier in the program using \hyperref[sec:sf_init]{\texttt{sf\_init}}.

\subsubsection*{Call}
\begin{verbatim}pars = sf_getpars (void);\end{verbatim}

\subsubsection*{Definition}
\begin{verbatim}
sf_simtab sf_getpars (void)
/*< provide access to the parameter table >*/
{
   ...
}
\end{verbatim}

\subsubsection*{Output}
\begin{desclist}{\tt }{\quad}[\tt pars]
   \setlength\itemsep{0pt}
   \item[pars] a pointer to the parameter table. 
\end{desclist}



       % Parameter handling
   
\chapter{Operations with RSF files}\label{sec:files}
   \section{Main operations with RSF files (file.c)}\label{sec:file.c}




\subsection{{sf\_file\_error}}
Sets an error on opening files. It sets the value of \texttt{error} (a static variable of type \texttt{bool}). This variable is used in the \texttt{sf\_input\_error} as an if-condition.

\subsubsection*{Call}
\begin{verbatim}sf_file_error(err);\end{verbatim}

\subsubsection*{Definition}
\begin{verbatim}
void sf_file_error(bool err)
/*< set error on opening files >*/
{
   ...
}
\end{verbatim}

\subsubsection*{Input parameters}
\begin{desclist}{\tt }{\quad}[\tt ]
   \setlength\itemsep{0pt}
   \item[err] value of type \texttt{bool} which is to be assigned to the static variable \texttt{error}.
\end{desclist}




\subsection{{sf\_error}}
Outputs an error message to the \texttt{stderr} (usually the screen) if the file cannot be opened. The '\texttt{:}' after the format specifiers in the call to \texttt{sf\_error} ensures that any system errors are also included in the output. 

\subsubsection*{Call}
\begin{verbatim}sf_input_error(file, message, name);\end{verbatim}

\subsubsection*{Definition}
\begin{verbatim}
static void sf_input_error(sf_file file, const char* message, const char* name)
{
   ...
}
\end{verbatim}

\subsubsection*{Input parameters}
\begin{desclist}{\tt }{\quad}[\tt message]
   \setlength\itemsep{0pt}
   \item[file]    pointer to the input file structure (\texttt{sf\_file}). 
   \item[message] the error message to be output to \texttt{stderr}. 
   \item[name]    name of the file which was to be opened.    
\end{desclist}




\subsection{{sf\_input}}\label{sec:sf_input}
Creates an input file structure and returns a pointer to that file structure. It will create the symbol table, input parameters to the table and write them in a temporary file and check for any errors in the input of the parameters. It will also set the format of the file and then return a pointer to the file structure.

\subsubsection*{Call}
\begin{verbatim}file = sf_input(tag);\end{verbatim}

\subsubsection*{Definition}
\begin{verbatim}
sf_file sf_input (/*@null@*/ const char* tag)
/*< Create an input file structure >*/
{
   ...
}
\end{verbatim}

\subsubsection*{Input parameters}
\begin{desclist}{\tt }{\quad}[\tt ]
   \setlength\itemsep{0pt}
   \item[tag] a tag for the input file (\texttt{const char*}).  
\end{desclist}

\subsubsection*{Output}
\begin{desclist}{\tt }{\quad}[\tt file]
   \setlength\itemsep{0pt}
   \item[file] a pointer to the input file structure.
\end{desclist}




\subsection{{sf\_output}}\label{sec:sf_output}
Creates an output file structure and returns a pointer to that file structure. It will create the symbol table, a header file and put the path of the data file in the header with the key "\texttt{in}". 

\subsubsection*{Call}
\begin{verbatim}file = sf_output(tag);\end{verbatim}

\subsubsection*{Definition}
\begin{verbatim}
sf_file sf_output (/*@null@*/ const char* tag)
/*< Create an output file structure.
---
Should do output after sf_input. >*/
{
   ...
}
\end{verbatim}

\subsubsection*{Input parameters}
\begin{desclist}{\tt }{\quad}[\tt ]
   \setlength\itemsep{0pt}
   \item[tag] a tag for the output file (\texttt{const char*}).  
\end{desclist}

\subsubsection*{Output}
\begin{desclist}{\tt }{\quad}[\tt file]
   \setlength\itemsep{0pt}
   \item[file] a pointer to the input file structure.
\end{desclist}




\subsection{{sf\_gettype}}
Returns the type of  the file, e.g.~\texttt{SF\_INT}, \texttt{SF\_FLOAT}, \texttt{SF\_COMPLEX} etc.

\subsubsection*{Call}
\begin{verbatim}type = sf_gettype (file);\end{verbatim}

\subsubsection*{Definition}
\begin{verbatim}
sf_datatype sf_gettype (sf_file file)
/*< return file type >*/
{
   ...
}
\end{verbatim}

\subsubsection*{Input parameters}
\begin{desclist}{\tt }{\quad}[\tt file]
   \setlength\itemsep{0pt}
   \item[file] a pointer to the file structure whose type is required (\texttt{sf\_file}).  
\end{desclist}

\subsubsection*{Output}
\begin{desclist}{\tt }{\quad}[\tt ]
   \setlength\itemsep{0pt}
   \item[file->type] type of the file structure.
\end{desclist}




\subsection{{sf\_getform}}
Returns the file form, e.g.~\texttt{SF\_ASCII}, \texttt{SF\_XDR}.

\subsubsection*{Call}
\begin{verbatim}form = sf_getform (file);\end{verbatim}

\subsubsection*{Definition}
\begin{verbatim}
sf_dataform sf_getform (sf_file file)
/*< return file form >*/
{
   ...
}
\end{verbatim}

\subsubsection*{Input parameters}
\begin{desclist}{\tt }{\quad}[\tt file]
   \setlength\itemsep{0pt}
   \item[file] a pointer to the file structure whose type is required (\texttt{sf\_file}).  
\end{desclist}

\subsubsection*{Output}
\begin{desclist}{\tt }{\quad}[\tt ]
   \setlength\itemsep{0pt}
   \item[file->form] form of the file structure.
\end{desclist}




\subsection{{sf\_esize}}
Returns the size of the element type of the file, e.g.~\texttt{SF\_INT}, \texttt{SF\_FLOAT}, \texttt{SF\_COMPLEX} etc.

\subsubsection*{Call}
\begin{verbatim}size = sf_esize(file);\end{verbatim}

\subsubsection*{Definition}
\begin{verbatim}
size_t sf_esize(sf_file file)
/*< return element size >*/
{
   ...
}
\end{verbatim}

\subsubsection*{Input parameters}
\begin{desclist}{\tt }{\quad}[\tt file]
   \setlength\itemsep{0pt}
   \item[file] a pointer to the file structure whose type is required (\texttt{sf\_file}).  
\end{desclist}

\subsubsection*{Output}
\begin{desclist}{\tt }{\quad}[\tt ]
   \setlength\itemsep{0pt}
   \item[size] size in bytes of the type of the file structure.
\end{desclist}




\subsection{{sf\_settype}}
Sets the type of the file, e.g.~\texttt{SF\_INT}, \texttt{SF\_FLOAT}, \texttt{SF\_COMPLEX} etc.

\subsubsection*{Call}
\begin{verbatim}sf_settype (file,type);\end{verbatim}

\subsubsection*{Definition}
\begin{verbatim}
void sf_settype (sf_file file, sf_datatype type)
/*< set file type >*/
{
   ...
}
\end{verbatim}

\subsubsection*{Input parameters}
\begin{desclist}{\tt }{\quad}[\tt file]
   \setlength\itemsep{0pt}
   \item[file] a pointer to the file structure whose type is to be set (\texttt{sf\_file}). 
   \item[type] the type to be set. Must be of type \texttt{sf\_datatype}, e.g.~\texttt{SF\_INT}.
\end{desclist}




\subsection{{sf\_setpars}}
Changes the parameter table from that of the file to the one which has parameters from the command line.

\subsubsection*{Call}
\begin{verbatim}sf_setpars (file);\end{verbatim}

\subsubsection*{Definition}
\begin{verbatim}
void sf_setpars (sf_file file)
/*< change parameters to those from the command line >*/
{
   ...
}
\end{verbatim}

\subsubsection*{Input parameters}
\begin{desclist}{\tt }{\quad}[\tt file]
   \setlength\itemsep{0pt}
   \item[file] a pointer to the file structure whose parameter table is to be closed (\texttt{sf\_file}).  
\end{desclist}




\subsection{{sf\_bufsiz}}
Returns the size of the buffer associated with the file. It gets the buffer size using the file descriptor of the file and the predefined structure \texttt{stat}. This provides control over the I/O operations, making them more efficient.

\subsubsection*{Call}
\begin{verbatim}bufsiz = sf_bufsiz(file);\end{verbatim}

\subsubsection*{Definition}
\begin{verbatim}
size_t sf_bufsiz(sf_file file)
/*< return buffer size for efficient I/O >*/
{
   ...    
}       
\end{verbatim}

\subsubsection*{Input parameters}
\begin{desclist}{\tt }{\quad}[\tt file]
   \setlength\itemsep{0pt}
   \item[file] a pointer to the file structure whose buffer size is required (\texttt{sf\_file}).  
\end{desclist}

\subsubsection*{Output}
\begin{desclist}{\tt }{\quad}[\tt ]
   \setlength\itemsep{0pt}
   \item[bufsiz] size of the buffer of the file structure.
\end{desclist}




\subsection{{sf\_setform}}
Sets the form of the file, i.e.~\texttt{SF\_ASCII}, \texttt{SF\_XDR}, \texttt{SF\_NATIVE}.

\subsubsection*{Call}
\begin{verbatim}sf_setform (file, form);\end{verbatim}

\subsubsection*{Definition}
\begin{verbatim}
void sf_setform (sf_file file, sf_dataform form)
/*< set file form >*/
{
   ...
}
\end{verbatim}

\subsubsection*{Input parameters}
\begin{desclist}{\tt }{\quad}[\tt form]
   \setlength\itemsep{0pt}
   \item[file] a pointer to the file structure whose form is to be set (\texttt{sf\_file}). 
   \item[form] the type to be set. Must be of type \texttt{sf\_datatype}, e.g.~\texttt{SF\_ASCII}.
\end{desclist}




\subsection{{sf\_setformat}}
Sets the format of the file, e.g.~\texttt{SF\_INT}, \texttt{SF\_FLOAT}, \texttt{SF\_COMPLEX} etc. Format is the combination of file form and its type, e.g.~\texttt{ASCII\_INT}.

\subsubsection*{Call}
\begin{verbatim}sf_setformat (file, format);\end{verbatim}

\subsubsection*{Definition}
\begin{verbatim}
void sf_setformat (sf_file file, const char* format)
/*< Set file format.
---
format has a form "form_type", i.e. native_float, ascii_int, etc.
>*/
{    
   ...
}
\end{verbatim}

\subsubsection*{Input parameters}
\begin{desclist}{\tt }{\quad}[\tt format]
   \setlength\itemsep{0pt}
   \item[file] a pointer to the file structure whose format is to be set (\texttt{sf\_file}). 
   \item[format] the type to be set (\texttt{const char*}).
\end{desclist}




\subsection{{sf\_getfilename}}
Returns a boolean value (true or false), depending on whether it was able to find the filename of an open file or not. The search is based on finding the file descriptor of an open file, if it is found the return value is \texttt{true}, otherwise \texttt{false}. Once the file name is found it is copied to the value pointed by the pointer \texttt{filename} which is given as input and is already defined in the \texttt{sf\_input}. 

\subsubsection*{Call}
\begin{verbatim}success = getfilename (fp, filename);\end{verbatim}

\subsubsection*{Definition}
\begin{verbatim}
static bool getfilename (FILE* fp, char *filename)
/* Finds filename of an open file from the file descriptor.

Unix-specific and probably non-portable. */
{
   ...
}
\end{verbatim}

\subsubsection*{Input parameters}
\begin{desclist}{\tt }{\quad}[\tt filename]
   \setlength\itemsep{0pt}
   \item[fp] a pointer to the file structure whose file name is required (\texttt{FILE*}).
   \item[filename] pointer to the parameter on which the found file name is to be stored.    
\end{desclist}

\subsubsection*{Output}
\begin{desclist}{\tt }{\quad}[\tt ]
   \setlength\itemsep{0pt}
   \item[success] a boolean value which is true if the filename is found otherwise false.
\end{desclist}




\subsection{{sf\_gettmpdatapath}}
Returns the path of temporary data. It takes no input parameters. The places it looks for the temporary data path are listed in the function definition comment.

\subsubsection*{Call}
\begin{verbatim}path gettmpdatapath ();\end{verbatim}

\subsubsection*{Definition}
\begin{verbatim}
static char* gettmpdatapath (void) 
/* Finds temporary datapath.

Datapath rules:
1. check tmpdatapath= on the command line
2. check TMPDATAPATH environmental variable
3. check .tmpdatapath file in the current directory
4. check .tmpdatapath in the home directory
5. return NULL
*/
{
   ...
}
\end{verbatim}

\subsubsection*{Output}
\begin{desclist}{\tt }{\quad}[\tt ]
   \setlength\itemsep{0pt}
   \item[path] a pointer (of type \texttt{char}) to the value of the \texttt{tmpdatapath}.
\end{desclist}




\subsection{{sf\_getdatapath}}
Returns the path of the data. It takes no input parameters. The places it looks for the temporary data path are listed in the function definition comment.

\subsubsection*{Call}
\begin{verbatim}path =  getdatapath();\end{verbatim}

\subsubsection*{Definition}
\begin{verbatim}
static char* getdatapath (void) 
/* Finds datapath.

Datapath rules:
1. check datapath= on the command line
2. check DATAPATH environmental variable
3. check .datapath file in the current directory
4. check .datapath in the home directory
5. use '.' (not a SEPlib behavior)
*/
{
   ...
}
\end{verbatim}

\subsubsection*{Output}
\begin{desclist}{\tt }{\quad}[\tt ]
   \setlength\itemsep{0pt}
   \item[path] a pointer (of type char) to the value of the \texttt{datapath}.
\end{desclist}




\subsection{{sf\_readpathfile}}
Returns a boolean value (true or false), depending on whether it was able to find the data path in an open file or not. Once the \texttt{datapath} is found it is copied to the value pointed by the pointer \texttt{datapath} which is given as input and is already defined in the \texttt{sf\_input}. 

\subsubsection*{Call}
\begin{verbatim}success = readpathfile (filename, datapath);\end{verbatim}

\subsubsection*{Definition}
\begin{verbatim}
static bool readpathfile (const char* filename, char* datapath) 
/* find datapath from the datapath file */
{
   ...
}
\end{verbatim}

\subsubsection*{Input parameters}
\begin{desclist}{\tt }{\quad}[\tt ]
   \setlength\itemsep{0pt}
   \item[filename] a pointer to the file name in which datapath is to be found (\texttt{const char}). 
   \item[datapath] pointer to the parameter which is being looked for.
\end{desclist}

\subsubsection*{Output}
\begin{desclist}{\tt }{\quad}[\tt ]
   \setlength\itemsep{0pt}
   \item[success] a boolean value which is true if the filename is found otherwise false.
\end{desclist}




\subsection{{sf\_fileclose}}
Closes the file and frees any allocated space, like the temporary file and buffer.

\subsubsection*{Call}
\begin{verbatim}sf_fileclose (file);\end{verbatim}

\subsubsection*{Definition}
\begin{verbatim}
void sf_fileclose (sf_file file) 
/*< close a file and free allocated space >*/
{
   ...    
}
\end{verbatim}

\subsubsection*{Input parameters}
\begin{desclist}{\tt }{\quad}[\tt file]
   \setlength\itemsep{0pt}
   \item[file] the file which is to be closed (\texttt{sf\_file}).
\end{desclist}




\subsection{{sf\_histint}}
Extracts an integer from the file. If the extraction is successful, it returns a \texttt{true}, otherwise a \texttt{false}. It uses \hyperref[sec:sf_simtab_getint]{\texttt{sf\_simtab\_getint}}.

\subsubsection*{Call}
\begin{verbatim}success = sf_histint (file, key, par);\end{verbatim}

\subsubsection*{Definition}
\begin{verbatim}
bool sf_histint (sf_file file, const char* key,/*@out@*/ int* par)
/*< read an int parameter from file >*/ 
{
   ...
}
\end{verbatim}

\subsubsection*{Input parameters}
\begin{desclist}{\tt }{\quad}[\tt file]
   \setlength\itemsep{0pt}
   \item[file] file from which an integer is to be extracted (\texttt{sf\_file}).
   \item[key]  the name of the entry which has to be extracted (\texttt{const char*}).
   \item[par]  pointer to the integer variable where the extracted value is to be copied.
\end{desclist}

\subsubsection*{Output}
\begin{desclist}{\tt }{\quad}[\tt ]
   \setlength\itemsep{0pt}
   \item[success] a boolean value. It is \texttt{true}, if the extraction was successful and \texttt{false} otherwise.
\end{desclist}




\subsection{{sf\_histints}}
Extracts an array of integer values from the file. If the extraction is successful, it returns a \texttt{true}, otherwise a \texttt{false}.

\subsubsection*{Call}
\begin{verbatim}success = sf_histints (file, key, par, n);\end{verbatim}

\subsubsection*{Definition}
\begin{verbatim}
bool sf_histints (sf_file file, const char* key,/*@out@*/ int* par,size_t n) 
/*< read an int array of size n parameter from file >*/ 
{
   ...
}
\end{verbatim}

\subsubsection*{Input parameters}
\begin{desclist}{\tt }{\quad}[\tt file]
   \setlength\itemsep{0pt}
   \item[file] file from which an integer array is to be extracted (\texttt{sf\_file}).
   \item[key]  the name of the integer array which has to be extracted (\texttt{const char*}).
   \item[par]  pointer to the array of integer type value variable where the extracted value is to be copied. 
   \item[n] size of the array to be extracted. Must be of \texttt{size\_t}.
\end{desclist}

\subsubsection*{Output}
\begin{desclist}{\tt }{\quad}[\tt success]
   \setlength\itemsep{0pt}
   \item[success] a boolean value. It is \texttt{true}, if the extraction was successful and \texttt{false} otherwise.
\end{desclist}




\subsection{{sf\_histlargeint}}
Extracts a large integer from the file. If the extraction is successful, it returns a \texttt{true}, otherwise a \texttt{false}. 
It uses \hyperref[sec:sf_simtab_getlargeint]{\texttt{sf\_simtab\_getlargeint}}.

\subsubsection*{Call}
\begin{verbatim}success = sf_histlargeint ( file, key, par);\end{verbatim}

\subsubsection*{Definition}
\begin{verbatim}
bool sf_histlargeint (sf_file file, const char* key,/*@out@*/ off_t* par)
/*< read a sf_largeint parameter from file >*/ 
{
   ...
}
\end{verbatim}

\subsubsection*{Input parameters}
\begin{desclist}{\tt }{\quad}[\tt file]
   \setlength\itemsep{0pt}
   \item[file] file from which a large integer is to be extracted (\texttt{sf\_file}).
   \item[key]  the name of the entry which has to be extracted (\texttt{const char*}).
   \item[par]  pointer to the large integer variable where the extracted value is to be copied. Must be of type \texttt{off\_t} which is defined in the header \texttt{<sys/types.h>}.
\end{desclist}

\subsubsection*{Output}
\begin{desclist}{\tt }{\quad}[\tt success]
   \setlength\itemsep{0pt}
   \item[success] a boolean value. It is \texttt{true}, if the extraction was successful and \texttt{false} otherwise.
\end{desclist}




\subsection{{sf\_histfloat}}
Extracts a float value from the file. If the extraction is successful, it returns a \texttt{true}, otherwise a \texttt{false}. 

\subsubsection*{Call}
\begin{verbatim}success = sf_histfloat (file, key, par);\end{verbatim}

\subsubsection*{Definition}
\begin{verbatim}
bool sf_histfloat (sf_file file, const char* key,/*@out@*/ float* par) 
/*< read a float parameter from file >*/
{
   ...
}
\end{verbatim}

\subsubsection*{Input parameters}
\begin{desclist}{\tt }{\quad}[\tt file]
   \setlength\itemsep{0pt}
   \item[file] file from which a floating point number is to be extracted (\texttt{sf\_file}).
   \item[key]  the name of the entry which has to be extracted (\texttt{const char*}).
   \item[par]  pointer to the float type value variable where the extracted value is to be copied.
\end{desclist}

\subsubsection*{Output}
\begin{desclist}{\tt }{\quad}[\tt success]
   \setlength\itemsep{0pt}
   \item[success] a boolean value. It is \texttt{true}, if the extraction was successful and \texttt{false} otherwise.
\end{desclist}



\subsection{{sf\_histdouble}}
Extracts a double type value from the file. If the extraction is successful, it returns a \texttt{true}, otherwise a \texttt{false}. 

\subsubsection*{Call}
\begin{verbatim}success = sf_histdouble (file, key, par);\end{verbatim}

\subsubsection*{Definition}
\begin{verbatim}
bool sf_histdouble (sf_file file, const char* key,/*@out@*/ double* par) 
/*< read a float parameter from file >*/
{
   ...
}
\end{verbatim}

\subsubsection*{Input parameters}
\begin{desclist}{\tt }{\quad}[\tt table]
   \setlength\itemsep{0pt}
   \item[file]  file from which a double type value is to be extracted (\texttt{sf\_file}). 
   \item[table] the table from which the vale has to be extracted. Must be of type \texttt{sf\_simtab}.
   \item[key]   the name of the entry which has to be extracted (\texttt{const char*}).
   \item[par]   pointer to the double type value variable where the extracted value is to be copied.
\end{desclist}

\subsubsection*{Output}
\begin{desclist}{\tt }{\quad}[\tt success]
   \setlength\itemsep{0pt}
   \item[success] a boolean value. It is \texttt{true}, if the extraction was successful and \texttt{false} otherwise.
\end{desclist}




\subsection{{sf\_histfloats}}
Extracts an array of float values from the file. If the extraction is successful returns a \texttt{true}. It uses \hyperref[sec:sf_simtab_getfloats]{\texttt{sf\_simtab\_getfloats}}.

\subsubsection*{Call}
\begin{verbatim}success = sf_histfloats(file, key, par, n);\end{verbatim}

\subsubsection*{Definition}
\begin{verbatim}
bool sf_histfloats (sf_file file, const char* key,
          /*@out@*/ float* par,   size_t n) 
/*< read a float array of size n parameter from file >*/ 
{
   ...
}
\end{verbatim}

\subsubsection*{Input parameters}
\begin{desclist}{\tt }{\quad}[\tt file]
   \setlength\itemsep{0pt}
   \item[file] file from which a float type array is to be extracted (\texttt{sf\_file}). 
   \item[key]  the name of the float array which has to be extracted (\texttt{const char*}).
   \item[par]  pointer to the array of float type value variable where the extracted value is to be copied. 
   \item[n]    size of the array to be extracted  (\texttt{size\_t}).
\end{desclist}




\subsection{{sf\_histbool}}
Extracts a boolean value from the file. If the extraction is successful, it returns a \texttt{true}, otherwise a \texttt{false}.

\subsubsection*{Call}
\begin{verbatim}success = sf_histbool(file, key, par);\end{verbatim}

\subsubsection*{Definition}
\begin{verbatim}
bool sf_histbool (sf_file file, const char* key,/*@out@*/ bool* par) 
/*< read a bool parameter from file >*/
{
   ...
}
\end{verbatim}

\subsubsection*{Input parameters}
\begin{desclist}{\tt }{\quad}[\tt file]
   \setlength\itemsep{0pt}
   \item[file] file from which a bool type value is to be extracted (\texttt{sf\_file}).
   \item[key]  the name of the entry which has to be extracted (\texttt{const char*}).
   \item[par]  pointer to the bool variable where the extracted value is to be copied.
\end{desclist}

\subsubsection*{Output}
\begin{desclist}{\tt }{\quad}[\tt success]
   \setlength\itemsep{0pt}
   \item[success] a boolean value. It is \texttt{true}, if the extraction was successful and \texttt{false} otherwise.
\end{desclist}




\subsection{{sf\_histtbools}}
Extracts an array of bool values from the file. If the extraction is successful, it returns a \texttt{true}, otherwise a \texttt{false}.It uses \hyperref[sec:sf_simtab_getbools]{\texttt{sf\_simtab\_getbools}}.

\subsubsection*{Call}
\begin{verbatim}success = sf_histbools(file, key, par, n);\end{verbatim}

\subsubsection*{Definition}
\begin{verbatim}
bool sf_histbools (sf_file file, const char* key,
                   /*@out@*/ bool* par, size_t n) 
/*< read a bool array of size n parameter from file >*/ 
{
   ...
}
\end{verbatim}

\subsubsection*{Input parameters}
\begin{desclist}{\tt }{\quad}[\tt file]
   \setlength\itemsep{0pt}
   \item[file] file from which an array of bool  value is to be extracted (\texttt{sf\_file}).
   \item[key]  the name of the bool array which has to be extracted (\texttt{const char*}).
   \item[par]  pointer to the array of bool type value variable where the extracted value is to be copied. 
   \item[n]    size of the array to be extracted. Must be of \texttt{size\_t}.
\end{desclist}

\subsubsection*{Output}
\begin{desclist}{\tt }{\quad}[\tt success]
   \setlength\itemsep{0pt}
   \item[success] a boolean value. It is \texttt{true}, if the extraction was successful and \texttt{false} otherwise.
\end{desclist}




\subsection{{sf\_histstring}}
Extracts a string pointed by the input key from the file. If the value is \texttt{NULL} it will return \texttt{NULL}, otherwise it will allocate a new block of memory of \texttt{char} type and copy the memory block from the table to the new block and return a pointer to the newly allocated block of memory. It uses \hyperref[sec:sf_simtab_getstring]{\texttt{sf\_simtab\_getstring}}.

\subsubsection*{Call}
\begin{verbatim}string = sf_histstring(file, key);\end{verbatim}

\subsubsection*{Definition}
\begin{verbatim}
char* sf_histstring (sf_file file, const char* key) 
/*< read a string parameter from file (returns NULL on failure) >*/ 
{
   ...
}
\end{verbatim}

\subsubsection*{Input parameters}
\begin{desclist}{\tt }{\quad}[\tt file]
   \setlength\itemsep{0pt}
   \item[file] file from which a string is to be extracted (\texttt{sf\_file}).
   \item[key]  the name of the string which has to be extracted (\texttt{const char*}).
\end{desclist}

\subsubsection*{Output}
\begin{desclist}{\tt }{\quad}[\tt ]
   \setlength\itemsep{0pt}
   \item[string] a pointer to allocated block of memory containing a string of characters. 
\end{desclist}




\subsection{{sf\_fileflush}}
Outputs the parameters from source file to the output file. It sets the data format in the output file and prepares the file for writing binary data.

\subsubsection*{Call}
\begin{verbatim}sf_fileflush( file, src);\end{verbatim}

\subsubsection*{Definition}
\begin{verbatim}
void sf_fileflush (sf_file file, sf_file src)
/*< outputs parameter to a file (initially from source src)
---
Prepares file for writing binary data >*/ 
{
...
}
\end{verbatim}

\subsubsection*{Input parameters}
\begin{desclist}{\tt }{\quad}[\tt file]
   \setlength\itemsep{0pt}
   \item[file] pointer to the output file (\texttt{sf\_file}). 
   \item[src]  a pointer to the input file structure (\texttt{sf\_file}).
\end{desclist}




\subsection{{sf\_putint}}
Enters an integer value in the file. It uses \hyperref[sec:sf_simtab_enter]{\texttt{sf\_simtab\_enter}}.  

\subsubsection*{Call}
\begin{verbatim}sf_putint (file, key, par);\end{verbatim}

\subsubsection*{Definition}
\begin{verbatim}
void sf_putint (sf_file file, const char* key, int par)
/*< put an int parameter to a file >*/
{
   ...
}
\end{verbatim}

\subsubsection*{Input parameters}
\begin{desclist}{\tt }{\quad}[\tt file]
   \setlength\itemsep{0pt}
   \item[file] the file in which the the key value is to be stored (\texttt{sf\_file}).
   \item[key]  pointer to the name of the key value to be input (\texttt{const char*}).
   \item[par]  integer parameter which is to be written.
\end{desclist}




\subsection{{sf\_putints}}
Enters an array of integer values in the file. It uses \hyperref[sec:sf_simtab_enter]{\texttt{sf\_simtab\_enter}}.  

\subsubsection*{Call}
\begin{verbatim}sf_putints (file, key, par, n);\end{verbatim}

\subsubsection*{Definition}
\begin{verbatim}
void sf_putints (sf_file file, const char* key, const int* par, size_t n)
/*< put an int array of size n parameter to a file >*/
{
   ...
}
\end{verbatim}

\subsubsection*{Input parameters}
\begin{desclist}{\tt }{\quad}[\tt file]
   \setlength\itemsep{0pt}
   \item[file] the file in which the the key value is to be stored (\texttt{sf\_file}).
   \item[key]  pointer to the name of the key value to be input (\texttt{const char*}).
   \item[par]  pointer to integer parameter array which is to be written. 
   \item[n]    size of the array to be written (\texttt{size\_t}).
\end{desclist}



\subsection{{sf\_putlargeint}}
Enters a long integer value in the file. It uses \hyperref[sec:sf_simtab_enter]{\texttt{sf\_simtab\_enter}}.  

\subsubsection*{Call}
\begin{verbatim}sf_putlargeint (file, key, par);\end{verbatim}

\subsubsection*{Definition}
\begin{verbatim}
void sf_putlargeint (sf_file file, const char* key, off_t par)
/*< put a sf_largeint parameter to a file >*/
{
   ...
}
\end{verbatim}

\subsubsection*{Input parameters}
\begin{desclist}{\tt }{\quad}[\tt file]
   \setlength\itemsep{0pt}
   \item[file] the file in which the the key value is to be stored (\texttt{sf\_file}).
   \item[key]  pointer to the name of the key value to be input (\texttt{const char*}).
   \item[par]  integer parameter which is to be written.
\end{desclist}





\subsection{{sf\_putfloat}}
Enters a float value in the file. It uses \hyperref[sec:sf_simtab_enter]{\texttt{sf\_simtab\_enter}}.  

\subsubsection*{Call}
\begin{verbatim}sf_putfloat (file, key, par);\end{verbatim}

\subsubsection*{Definition}
\begin{verbatim}
void sf_putfloat (sf_file file, const char* key,float par)
/*< put a float parameter to a file >*/
{
   ...
}
\end{verbatim}

\subsubsection*{Input parameters}
\begin{desclist}{\tt }{\quad}[\tt file]
   \setlength\itemsep{0pt}
   \item[file] the file in which the the key value is to be stored (\texttt{sf\_file}).
   \item[key]  pointer to the name of the key value to be input (\texttt{const char*}).
   \item[par]  floating point parameter which is to be written.
\end{desclist}

\subsubsection*{Definition}
\begin{verbatim}
void sf_putfloat (sf_file file, const char* key,float par)
/*< put a float parameter to a file >*/
{
   ...
}
\end{verbatim}




\subsection{{sf\_putstring}}
Enters a string in to the file. It uses \hyperref[sec:sf_simtab_enter]{\texttt{sf\_simtab\_enter}}.  

\subsubsection*{Call}
\begin{verbatim}sf_putstring (file, key, par);\end{verbatim}

\subsubsection*{Definition}
\begin{verbatim}
void sf_putstring (sf_file file, const char* key,const char* par)
/*< put a string parameter to a file >*/
{
   ...
}
\end{verbatim}

\subsubsection*{Input parameters}
\begin{desclist}{\tt }{\quad}[\tt file]
   \setlength\itemsep{0pt}
   \item[file] the file in which the key value is to be stored (\texttt{sf\_file}).
   \item[key]  pointer to the name of the key value to be input (\texttt{const char*}).
   \item[par]  pointer to the string parameter which is to be written.
\end{desclist}




\subsection{{sf\_putline}}
Enters a string line in to the file.  

\subsubsection*{Call}
\begin{verbatim}sf_putline (file, line);\end{verbatim}

\subsubsection*{Definition}
\begin{verbatim}
void sf_putline (sf_file file, const char* line)
/*< put a string line to a file >*/
{
   ...
}
\end{verbatim}

\subsubsection*{Input parameters}
\begin{desclist}{\tt }{\quad}[\tt line]
   \setlength\itemsep{0pt}
   \item[file] the file in which the string line is to be stored (\texttt{sf\_file}). 
   \item[line] pointer to the which is to be written. 
\end{desclist}




\subsection{{sf\_setaformat}}
Sets number format specifiers for ASCII output. This can be used in \texttt{sf\_complexwrite}, for example.  

\subsubsection*{Call}
\begin{verbatim}sf_setaformat (format, line);\end{verbatim}

\subsubsection*{Definition}
\begin{verbatim}
void sf_setaformat (const char* format /* number format (.i.e "%5g") */, 
                    int line /* numbers in line */ )
/*< Set format for ascii output >*/
{
   ...    
}
\end{verbatim}

\subsubsection*{Input parameters}
\begin{desclist}{\tt }{\quad}[\tt format]
   \setlength\itemsep{0pt}
   \item[format] a number format, e.g.~\%5g. 
   \item[line]   numbers in the ASCII line.
\end{desclist}




\subsection{{sf\_complexwrite}}
Writes a complex array to the file, according to the value of the form of the file, i.e.~\texttt{SF\_ASCII}, \texttt{SF\_XDR} or \texttt{SF\_NATIVE}.    

\subsubsection*{Call}
\begin{verbatim}sf_complexwrite (arr, size, file);\end{verbatim}

\subsubsection*{Definition}
\begin{verbatim}
void sf_complexwrite (sf_complex* arr, size_t size, sf_file file)
/*< write a complex array arr[size] to file >*/
{
   ...
}
\end{verbatim}

\subsubsection*{Input parameters}
\begin{desclist}{\tt }{\quad}[\tt size]
   \setlength\itemsep{0pt}
   \item[arr] a pointer to the array which is to be written (\texttt{sf\_complex}). 
   \item[size] size of the array (\texttt{size\_t}). 
   \item[file] a file in which the array is to be written (\texttt{sf\_file}).
\end{desclist}




\subsection{{sf\_complexread}}
Reads a complex array from the file, according to the value of the form of the file, i.e.~\texttt{SF\_ASCII}, \texttt{SF\_XDR} or \texttt{SF\_NATIVE}.    

\subsubsection*{Call}
\begin{verbatim}sf_complexread (arr, size, file);\end{verbatim}

\subsubsection*{Definition}
\begin{verbatim}
void sf_complexread (/*@out@*/ sf_complex* arr, size_t size, sf_file file)
/*< read a complex array arr[size] from file >*/
{
   ...
}
\end{verbatim}

\subsubsection*{Input parameters}
\begin{desclist}{\tt }{\quad}[\tt size]
   \setlength\itemsep{0pt}
   \item[arr] a pointer to the array to which the array from the file is to be copied (\texttt{sf\_complex}). 
   \item[size] size of the array (\texttt{size\_t}). 
   \item[file] a file from which the array is to be read (\texttt{sf\_file}).
\end{desclist}




\subsection{{sf\_charwrite}}
Writes a character array to the file, according to the value of the form of the file, i.e.~\texttt{SF\_ASCII}, \texttt{SF\_XDR} or \texttt{SF\_NATIVE}.    

\subsubsection*{Call}
\begin{verbatim}sf_charwrite (arr, size, file);\end{verbatim}

\subsubsection*{Definition}
\begin{verbatim}
void sf_charwrite (char* arr, size_t size, sf_file file)
/*< write a char array arr[size] to file >*/
{
   ...
}
\end{verbatim}

\subsubsection*{Input parameters}
\begin{desclist}{\tt }{\quad}[\tt size]
   \setlength\itemsep{0pt}
   \item[arr]  a pointer to the array which is to be written (\texttt{sf\_complex}). 
   \item[size] size of the array (\texttt{size\_t}). 
   \item[file] a file in which the array is to be written (\texttt{sf\_file}).
\end{desclist}




\subsection{{sf\_uncharwrite}}
Writes a unsigned character array to the file, according to the value of the form of the file, i.e.~\texttt{SF\_ASCII}, \texttt{SF\_XDR} or \texttt{SF\_NATIVE}.        

\subsubsection*{Call}
\begin{verbatim}sf_ucharwrite (arr, size, file);\end{verbatim}

\subsubsection*{Definition}
\begin{verbatim}
void sf_ucharwrite (unsigned char* arr, size_t size, sf_file file)
/*< write an unsigned char array arr[size] to file >*/
{
   ...
}
\end{verbatim}

\subsubsection*{Input parameters}
\begin{desclist}{\tt }{\quad}[\tt size]
   \setlength\itemsep{0pt}
   \item[arr]  a pointer to the array which is to be written (\texttt{sf\_complex}). 
   \item[size] size of the array (\texttt{size\_t}). 
   \item[file] a file in which the array is to be written (\texttt{sf\_file}).
\end{desclist}




\subsection{{sf\_charread}}
Reads a character array from the file, according to the value of the form of the file, i.e.~\texttt{SF\_ASCII}, \texttt{SF\_XDR} or \texttt{SF\_NATIVE}.        

\subsubsection*{Call}
\begin{verbatim}sf_charread (arr, size, file);\end{verbatim}

\subsubsection*{Definition}
\begin{verbatim}
void sf_charread (/*@out@*/ char* arr, size_t size, sf_file file)
/*< read a char array arr[size] from file >*/
{
   ...
}
\end{verbatim}

\subsubsection*{Input parameters}
\begin{desclist}{\tt }{\quad}[\tt size]
   \setlength\itemsep{0pt}
   \item[arr] a pointer to the array to which the array from the file is to be copied (\texttt{sf\_complex}). 
   \item[size] size of the array (\texttt{size\_t}). 
   \item[file] a file from which the array is to be read (\texttt{sf\_file}).
\end{desclist}




\subsection{{sf\_uncharread}}
Reads an unsigned character array from the file, according to the value of the form of the file, i.e.~\texttt{SF\_ASCII}, \texttt{SF\_XDR} or \texttt{SF\_NATIVE}.    

\subsubsection*{Call}
\begin{verbatim}sf_ucharread (arr, size, file);\end{verbatim}

\subsubsection*{Definition}
\begin{verbatim}
void sf_ucharread (/*@out@*/ unsigned char* arr, size_t size, sf_file file)
/*< read a uchar array arr[size] from file >*/
{
   ...
}
\end{verbatim}

\subsubsection*{Input parameters}
\begin{desclist}{\tt }{\quad}[\tt size]
   \setlength\itemsep{0pt}
   \item[arr]  a pointer to the array to which the array from the file is to be copied (\texttt{sf\_complex}). 
   \item[size] size of the array (\texttt{size\_t}). 
   \item[file] a file from which the array is to be read (\texttt{sf\_file}).
\end{desclist}




\subsection{{sf\_intwrite}}
Writes an integer array to the file, according to the value of the form of the file, i.e.~\texttt{SF\_ASCII}, \texttt{SF\_XDR} or \texttt{SF\_NATIVE}.        

\subsubsection*{Call}
\begin{verbatim}sf_intwrite (arr, size, file);\end{verbatim}

\subsubsection*{Input parameters}
\begin{desclist}{\tt }{\quad}[\tt size]
   \setlength\itemsep{0pt}
   \item[arr] a pointer to the array which is to be written (\texttt{sf\_complex}). 
   \item[size] size of the array (\texttt{size\_t}). 
   \item[file] a file in which the array is to be written (\texttt{sf\_file}).
\end{desclist}




\subsection{{sf\_intread}}
Reads an integer array from the file, according to the value of the form of the file, i.e.~\texttt{SF\_ASCII}, \texttt{SF\_XDR} or \texttt{SF\_NATIVE}.        

\subsubsection*{Call}
\begin{verbatim}sf_intread (arr, size, file);\end{verbatim}

\subsubsection*{Definition}
\begin{verbatim}
void sf_intread (/*@out@*/ int* arr, size_t size, sf_file file)
/*< read an int array arr[size] from file >*/
{
   ...
}
\end{verbatim}

\subsubsection*{Input parameters}
\begin{desclist}{\tt }{\quad}[\tt size]
   \setlength\itemsep{0pt}
   \item[arr]  a pointer to the array to which the array from the file is to be copied (\texttt{sf\_complex}). 
   \item[size] size of the array (\texttt{size\_t}). 
   \item[file] a file from which the array is to be read (\texttt{sf\_file}).
\end{desclist}




\subsection{{sf\_shortread}}
Reads an short array from the file, according to the value of the form of the file, i.e.~\texttt{SF\_ASCII}, \texttt{SF\_XDR} or \texttt{SF\_NATIVE}.        

\subsubsection*{Call}
\begin{verbatim}sf_shortread (arr, size, file);\end{verbatim}

\subsubsection*{Definition}
\begin{verbatim}
void sf_shortread (/*@out@*/ short* arr, size_t size, sf_file file)
/*< read a short array arr[size] from file >*/
{
   ...
}
\end{verbatim}

\subsubsection*{Input parameters}
\begin{desclist}{\tt }{\quad}[\tt file]
   \setlength\itemsep{0pt}
   \item[arr ] a pointer to the array to which the array from the file is to be copied (\texttt{sf\_complex}). 
   \item[size] size of the array (\texttt{size\_t}). 
   \item[file] a file from which the array is to be read (\texttt{sf\_file}).
\end{desclist}




\subsection{{sf\_shortwrite}}
Writes an short array to the file, according to the value of the form of the file, i.e.~\texttt{SF\_ASCII}, \texttt{SF\_XDR} or \texttt{SF\_NATIVE}.

\subsubsection*{Call}
\begin{verbatim}sf_shortwrite (arr, size, file);\end{verbatim}

\subsubsection*{Definition}
\begin{verbatim}
void sf_shortwrite (short* arr, size_t size, sf_file file)
/*< write a short array arr[size] to file >*/
{
   ...
}
\end{verbatim}

\subsubsection*{Input parameters}
\begin{desclist}{\tt }{\quad}[\tt size]
   \setlength\itemsep{0pt}
   \item[arr]  a pointer to the array which is to be written (\texttt{sf\_complex}). 
   \item[size] size of the array (\texttt{size\_t}). 
   \item[file] a file in which the array is to be written (\texttt{sf\_file}).
\end{desclist}




\subsection{{sf\_floatwrite}}\label{sec:sf_floatwrite}
Writes an float array to the file, according to the value of the form of the file, i.e.~\texttt{SF\_ASCII}, \texttt{SF\_XDR} or \texttt{SF\_NATIVE}.

\subsubsection*{Call}
\begin{verbatim}sf_floatwrite (arr, size, file);\end{verbatim}

\subsubsection*{Definition}
\begin{verbatim}
void sf_floatwrite (float* arr, size_t size, sf_file file)
/*< write a float array arr[size] to file >*/
{
   ...
}
\end{verbatim}

\subsubsection*{Input parameters}
\begin{desclist}{\tt }{\quad}[\tt size]
   \setlength\itemsep{0pt}
   \item[arr]  a pointer to the array which is to be written (\texttt{sf\_complex}). 
   \item[size] size of the array (\texttt{size\_t}). 
   \item[file] a file in which the array is to be written (\texttt{sf\_file}).
\end{desclist}




\subsection{{sf\_floatread}}\label{sec:sf_floatread}
Reads a float array from the file, according to the value of the form of the file, i.e.~\texttt{SF\_ASCII}, \texttt{SF\_XDR} or \texttt{SF\_NATIVE}.

\subsubsection*{Call}
\begin{verbatim}sf_floatread (arr, size, file);\end{verbatim}

\subsubsection*{Definition}
\begin{verbatim}
void sf_floatread (/*@out@*/ float* arr, size_t size, sf_file file)
/*< read a float array arr[size] from file >*/
{
   ...
}
\end{verbatim}

\subsubsection*{Input parameters}
\begin{desclist}{\tt }{\quad}[\tt size]
   \setlength\itemsep{0pt}
   \item[arr]  a pointer to the array to which the array from the file is to be copied (\texttt{sf\_complex}). 
   \item[size] size of the array (\texttt{size\_t}). 
   \item[file] a file from which the array is to be read (\texttt{sf\_file}).
\end{desclist}




\subsection{{sf\_bytes}}
Returns the size of the file in bytes.

\subsubsection*{Call}
\begin{verbatim}size = sf_bytes (file);\end{verbatim}

\subsubsection*{Definition}
\begin{verbatim}
off_t sf_bytes (sf_file file)
/*< Count the file data size (in bytes) >*/
{
   ...
}
\end{verbatim}

\subsubsection*{Input parameters}
\begin{desclist}{\tt }{\quad}[\tt file]
   \setlength\itemsep{0pt}
   \item[file] a pointer to the file structure whose size is required (\texttt{sf\_file}).  
\end{desclist}

\subsubsection*{Output}
\begin{desclist}{\tt }{\quad}[\tt ]
   \setlength\itemsep{0pt}
   \item[size] the size of the file structure in bytes.
\end{desclist}




\subsection{{sf\_tell}}
Returns the current value of the position indicator of the file.

\subsubsection*{Call}
\begin{verbatim}val = sf_tell (file);\end{verbatim}

\subsubsection*{Definition}
\begin{verbatim}
off_t sf_tell (sf_file file)
/*< Find position in file >*/
{
   ...
}
\end{verbatim}

\subsubsection*{Input parameters}
\begin{desclist}{\tt }{\quad}[\tt file]
   \setlength\itemsep{0pt}
   \item[file] a pointer to the file structure the value of whose position indicator required (\texttt{sf\_file}).
\end{desclist}                 

\subsubsection*{Output}
\begin{desclist}{\tt }{\quad}[\tt ]
   \setlength\itemsep{0pt}
   \item[] Current value of the position indicator of the file. It is of type \texttt{off\_t}.
\end{desclist}




\subsection{{sf\_tempfile}}
Creates a temporary file with a unique file name.

\subsubsection*{Call}
\begin{verbatim}tmp = sf_tempfile(dataname, mode);\end{verbatim}

\subsubsection*{Definition}
\begin{verbatim}
FILE *sf_tempfile(char** dataname, const char* mode)
/*< Create a temporary file with a unique name >*/
{
   ...
}
\end{verbatim}

\subsubsection*{Input parameters}
\begin{desclist}{\tt }{\quad}[\tt dataname]
   \setlength\itemsep{0pt}
   \item[dataname] a pointer to the value of the name of the temporary file (\texttt{char**}).
   \item[mode] mode of the file to be created, e.g.~\texttt{w+}.
 \end{desclist}

\subsubsection*{Output}
\begin{desclist}{\tt }{\quad}[\tt ]
   \setlength\itemsep{0pt}
   \item[tmp] a pointer to the temporary file.
\end{desclist}




\subsection{{sf\_seek}}

\subsubsection*{Call}
\begin{verbatim}sf_seek (file, offset, whence);\end{verbatim}

\subsubsection*{Definition}
\begin{verbatim}
void sf_seek (sf_file file, off_t offset, int whence)
/*< Seek to a position in file. Follows fseek convention. >*/
{
   ...
}
\end{verbatim}




\subsection{{sf\_unpipe}}
Redirects a pipe input to a directly accessible file.

\subsubsection*{Call}
\begin{verbatim}sf_unpipe (file, size);\end{verbatim}

\subsubsection*{Definition}
\begin{verbatim}
void sf_unpipe (sf_file file, off_t size) 
/*< Redirect a pipe input to a direct access file >*/
{
   ...
} 
\end{verbatim}

\subsubsection*{Input parameters}
\begin{desclist}{\tt }{\quad}[\tt file]
   \setlength\itemsep{0pt}
   \item[file] a pointer to the file structure which is to be unpiped (\texttt{sf\_file}).
\end{desclist}



\subsection{{sf\_close}}\label{sec:sf_close}
Removes temporary files.

\subsubsection*{Call}
\begin{verbatim}sf_close();\end{verbatim}

\subsubsection*{Definition}
\begin{verbatim}
void sf_close(void)
/*< Remove temporary files >*/
{
   ...   
}
\end{verbatim}




          % Main operations with RSF files 
   \section{Additional operations with RSF files (files.c)}




\subsection{{sf\_filedims}}\label{sec:sf_filedims}
Returns the dimensions of the file.

\subsubsection*{Call}
\begin{verbatim}dim = sf_filedims (file, n);\end{verbatim}

\subsubsection*{Definition}
\begin{verbatim}
int sf_filedims (sf_file file, /*@out@*/ int *n) 
/*< Find file dimensions.
--- 
Outputs the number of dimensions dim and a dimension array n[dim] >*/
{
   ...
}
\end{verbatim}

\subsubsection*{Input parameters}
\begin{desclist}{\tt}{\quad}[file]
   \setlength\itemsep{0pt}
   \item[file] a pointer to the file structure whose dimensions are required (\texttt{sf\_file}). 
   \item[n]    an array where the dimensions will be stored (\texttt{int}).
\end{desclist}

\subsubsection*{Output}
\begin{desclist}{}{\quad}[nddimm]
   \setlength\itemsep{0pt}
   \item[\texttt{dim}] number of dimensions in the file (\texttt{int}). 
   \item[\texttt{n[dim]}] the array of dimensions (\texttt{int}).
\end{desclist}




\subsection{{sf\_largefiledims}}
Returns the dimensions of the file. It is exactly like \hyperref[sec:sf_filedims]{\texttt{sf\_filedims}} but \texttt{n} in the input is the offset in bytes in the input file (type \texttt{off\_t}) rather than just an integer.

\subsubsection*{Call}
\begin{verbatim}dim = sf_largefiledims (file, n);\end{verbatim}

\subsubsection*{Definition}
\begin{verbatim}
int sf_largefiledims (sf_file file, /*@out@*/ off_t *n) 
/*< Find file dimensions.
--- 
Outputs the number of dimensions dim and a dimension array n[dim] >*/
{
   ...
}
\end{verbatim}

\subsubsection*{Input parameters}
\begin{desclist}{\tt }{\quad}[\tt file]
   \setlength\itemsep{0pt}
   \item[file] a pointer to the file structure whose dimensions are required (\texttt{sf\_file}). 
   \item[n] an array where the dimensions will be stored (\texttt{off\_t}).
\end{desclist}

\subsubsection*{Output}
\begin{desclist}{}{\quad}[\tt nddimm]
   \setlength\itemsep{0pt}
   \item[\texttt{dim}]    number of dimensions in the file (\texttt{off\_t}). 
   \item[\texttt{n[dim]}] the array of dimensions (\texttt{off\_t}).
\end{desclist}




\subsection{{sf\_memsize}}
Returns the memory size defined in the environment variable \texttt{RSFMEMSIZE}. If there is an invalid value the function will print an error message an assign a default value of 100 Mbytes.

\subsubsection*{Call}
\begin{verbatim}memsize = sf_memsize();\end{verbatim}

\subsubsection*{Definition}
\begin{verbatim}
int sf_memsize()
/*< Returns memory size by:
  1. checking RSFMEMSIZE environmental variable
  2. using hard-coded "def" constant
  >*/
{

   ...
    return memsize;
}
\end{verbatim}




\subsection{{sf\_filesize}}
Returns the size of the file, that is, the product of the dimensions. It uses the function \hyperref[sec:sf_leftsize]{\texttt{sf\_leftsize}}.

\subsubsection*{Call}
\begin{verbatim}size = sf_filesize (file);\end{verbatim}

\subsubsection*{Definition}
\begin{verbatim}
off_t sf_filesize (sf_file file) 
/*< Find file size (product of all dimensions) >*/
{    
   ...
}
\end{verbatim}

\subsubsection*{Input parameters}
\begin{desclist}{\tt }{\quad}[\tt ]
   \setlength\itemsep{0pt}
   \item[file] a pointer to the file structure whose size is required (\texttt{sf\_file}).  
\end{desclist}

\subsubsection*{Output}
\begin{desclist}{\tt }{\quad}[\tt ]
   \setlength\itemsep{0pt}
   \item[size] product of dimensions in the file (\texttt{off\_t}).
\end{desclist}




\subsection{{sf\_leftsize}}\label{sec:sf_leftsize}
Returns the size of the file, that is, the product of the dimensions but only for the dimensions greater than the input integer dim. It uses the function \hyperref[sec:sf_leftsize]{\texttt{sf\_leftsize}}.

\subsubsection*{Call}
\begin{verbatim}size = sf_leftsize (file, dim);\end{verbatim}

\subsubsection*{Definition}
\begin{verbatim}
off_t sf_leftsize (sf_file file, int dim) 
/*< Find file size for dimensions greater than dim >*/
{
   ...
}
\end{verbatim}

\subsubsection*{Input parameters}
\begin{desclist}{\tt }{\quad}[\tt file]
   \setlength\itemsep{0pt}
   \item[file] the file whose size is required (\texttt{sf\_file}). 
   \item[dim] a pointer to the file structure whose size is required (\texttt{sf\_file}).  
\end{desclist}

\subsubsection*{Output}
\begin{desclist}{\tt }{\quad}[\tt ]
   \setlength\itemsep{0pt}
   \item[size] product of dimensions greater than dim in the file (\texttt{off\_t}).
\end{desclist}




\subsection{{sf\_cp}}
Copies the input file \texttt{in} to the output file \texttt{out}.

\subsubsection*{Call}
\begin{verbatim}sf_cp (in, out);\end{verbatim}

\subsubsection*{Definition}
\begin{verbatim}
void sf_cp(sf_file in, sf_file out)
/*< Copy file in to file out >*/
{
   ...
}
\end{verbatim}

\subsubsection*{Input parameters}
\begin{desclist}{\tt }{\quad}[\tt ]
   \setlength\itemsep{0pt}
   \item[in] the file which is to be copied (\texttt{sf\_file}). 
   \item[out] the file to which \texttt{in} file is to be copied (\texttt{sf\_file}).  
\end{desclist}




\subsection{{sf\_rm}}
Removes the RSF file. There are options to force removal (files are deleted even if protected), to inquire before removing a file and whether or to require verbose output.

\subsubsection*{Call}
\begin{verbatim}sf_rm (filename, force, verb, inquire);\end{verbatim}

\subsubsection*{Definition}
\begin{verbatim}
void sf_rm(const char* filename, bool force, bool verb, bool inquire)
/*< Remove an RSF file.
---
force, verb, and inquire flags should behave similar to the corresponding flags 
in the Unix "rm" command. >*/
{
   ...
}
\end{verbatim}

\subsubsection*{Input parameters}
\begin{desclist}{\texttt }{\quad}[filename]
   \setlength\itemsep{0pt}
   \item[filename] name of the file which is to be removed (\texttt{sf\_file}). 
   \item[force]    remove forcefully or not (\texttt{sf\_file}). 
   \item[inquire]  ask before removing or not (\texttt{sf\_file}).  
\end{desclist}




\subsection{{sf\_shiftdim}}
Shifts the grid by one dimension after the axis defined in the input parameters (axis). 

\subsubsection*{Call}
\begin{verbatim}n3 = sf_shiftdim(in, out, axis);\end{verbatim}

\subsubsection*{Definition}
\begin{verbatim}
off_t sf_shiftdim(sf_file in, sf_file out, int axis) 
/*< shift grid after axis by one dimension forward >*/
{
   ...
}
\end{verbatim}

\subsubsection*{Input parameters}
\begin{desclist}{\tt }{\quad}[\tt axis]
   \setlength\itemsep{0pt}
   \item[in]   a pointer to the input file structure (\texttt{sf\_file}). 
   \item[out]  a pointer to the output file structure (\texttt{sf\_file}). 
   \item[axis] the axis after which the grid is to be shifted (\texttt{sf\_file}).
\end{desclist}

\subsubsection*{Output}
\begin{desclist}{\tt }{\quad}[\tt ]
   \setlength\itemsep{0pt}
   \item[n3] the file size (product of dimensions) after the shift (\texttt{off\_t}).
\end{desclist}




\subsection{{sf\_unshiftdim}}
Shifts the grid backward by one dimension after the axis defined in the input parameters (axis). 

\subsubsection*{Call}
\begin{verbatim}n3 = sf_unshiftdim (in, out, axis);\end{verbatim}

\subsubsection*{Definition}
\begin{verbatim}
off_t sf_unshiftdim(sf_file in, sf_file out, int axis) 
/*< shift grid after axis by one dimension backward >*/
{
   ...
}
\end{verbatim}

\subsubsection*{Input parameters}
\begin{desclist}{\tt }{\quad}[\tt axis]
   \setlength\itemsep{0pt}
   \item[in]   a pointer to the input file structure (\texttt{sf\_file}). 
   \item[out]  a pointer to the output file structure (\texttt{sf\_file}). 
   \item[axis] the axis after which the grid is to be shifted (\texttt{sf\_file}).
\end{desclist}

\subsubsection*{Output}
\begin{desclist}{\tt }{\quad}[\tt n3]
   \setlength\itemsep{0pt}
   \item[n3] the file size (product of dimensions) after the backward shift (\texttt{off\_t}).
\end{desclist}




\subsection{{sf\_endian}}
Returns true if the machine is little endian. 

\subsubsection*{Definition}
\begin{verbatim}little_endian = sf_endian (); \end{verbatim}

\subsubsection*{Definition}
\begin{verbatim}
bool sf_endian (void)
/*< Endianness test, returns true for little-endian machines >*/
{
   ...    
}
\end{verbatim}

\subsubsection*{Output}
\begin{desclist}{\tt }{\quad}[\tt ]
   \setlength\itemsep{0pt}
   \item[little\_endian] a boolean parameter which is true if the machine is little endian.
\end{desclist}





         % Additional operations with RSF files
   \section{Complex number operations (komplex.c)}




\subsection{{creal}}
Returns the real part of the complex number.

\subsubsection*{Call}
\begin{verbatim}r = sf_creal(c);\end{verbatim}

\subsubsection*{Definition}
\begin{verbatim}
double sf_creal(sf_double_complex c)
/*< real part >*/
{
   ...
}
\end{verbatim}

\subsubsection*{Input parameters}
\begin{desclist}{\tt }{\quad}[\tt ]
   \setlength\itemsep{0pt}
   \item[c] a complex number. Must be of type \texttt{sf\_double\_complex}.  
\end{desclist}

\subsubsection*{Output}
\begin{desclist}{\tt }{\quad}[\tt ]
   \setlength\itemsep{0pt}
   \item[r] real part of the complex number. It is of type \texttt{double}.
\end{desclist}




\subsection{{cimag}}
Returns the imaginary part of the complex number.

\subsubsection*{Call}
\begin{verbatim}im = sf_cimag(c);\end{verbatim}

\subsubsection*{Definition}
\begin{verbatim}
double sf_cimag(sf_double_complex c)
/*< imaginary part >*/
{
   ...
}
\end{verbatim}

\subsubsection*{Input parameters}
\begin{desclist}{\tt }{\quad}[\tt ]
   \setlength\itemsep{0pt}
   \item[c] a complex number. Must be of type \texttt{sf\_double\_complex}.  
\end{desclist}

\subsubsection*{Output}
\begin{desclist}{\tt }{\quad}[\tt ]
   \setlength\itemsep{0pt}
   \item[im] imaginary part of the complex number (\texttt{double}).
\end{desclist}




\subsection{{dcneg}}
Returns the negative complex number.

\subsubsection*{Call}
\begin{verbatim}n = sf_dcneg(a);\end{verbatim}

\subsubsection*{Definition}
\begin{verbatim}
sf_double_complex sf_dcneg(sf_double_complex a)
/*< unary minus >*/
{
   ...
}
\end{verbatim}

\subsubsection*{Input parameters}
\begin{desclist}{\tt }{\quad}[\tt ]
   \setlength\itemsep{0pt}
   \item[a] a complex number. Must be of type \texttt{sf\_double\_complex}.  
\end{desclist}

\subsubsection*{Output}
\begin{desclist}{\tt }{\quad}[\tt ]
   \setlength\itemsep{0pt}
   \item[n] negative of the complex number. It is of type \texttt{sf\_double\_complex}.
\end{desclist}




\subsection{{dcadd}}
Adds two complex numbers.

\subsubsection*{Call}
\begin{verbatim}c = sf_dcadd(a, b);\end{verbatim}

\subsubsection*{Definition}
\begin{verbatim}
sf_double_complex sf_dcadd(sf_double_complex a, sf_double_complex b)
/*< complex addition >*/
{
   ...
}
\end{verbatim}

\subsubsection*{Input parameters}
\begin{desclist}{\tt }{\quad}[\tt ]
   \setlength\itemsep{0pt}
   \item[a] a complex number. Must be of type \texttt{sf\_double\_complex}. 
   \item[b] a complex number. Must be of type \texttt{sf\_double\_complex}.  
\end{desclist}

\subsubsection*{Output}
\begin{desclist}{\tt }{\quad}[\tt ]
   \setlength\itemsep{0pt}
   \item[c] $a+b$. It is of type \texttt{sf\_double\_complex}.
\end{desclist}




\subsection{{dcsub}}
Subtracts two complex numbers.

\subsubsection*{Call}
\begin{verbatim}c = sf_dcsub(a, b);\end{verbatim}

\subsubsection*{Definition}
\begin{verbatim}
sf_double_complex sf_dcsub(sf_double_complex a, sf_double_complex b)
/*< complex subtraction >*/
{
   ...
}
\end{verbatim}

\subsubsection*{Input parameters}
\begin{desclist}{\tt }{\quad}[\tt ]
   \setlength\itemsep{0pt}
   \item[a] a complex number. Must be of type \texttt{sf\_double\_complex}. 
   \item[b] a complex number. Must be of type \texttt{sf\_double\_complex}.  
\end{desclist}

\subsubsection*{Output}
\begin{desclist}{\tt }{\quad}[\tt ]
   \setlength\itemsep{0pt}
   \item[c] $a-b$. It is of type \texttt{sf\_double\_complex}.
\end{desclist}




\subsection{{dcmul}}
Multiplies two complex number.

\subsubsection*{Call}
\begin{verbatim}c = sf_dcmul(a, b);\end{verbatim}

\subsubsection*{Definition}
\begin{verbatim}
sf_double_complex sf_dcmul(sf_double_complex a, sf_double_complex b)
/*< complex multiplication >*/
{
   ...
}
\end{verbatim}

\subsubsection*{Input parameters}
\begin{desclist}{\tt }{\quad}[\tt ]
   \setlength\itemsep{0pt}
   \item[a] a complex number. Must be of type \texttt{sf\_double\_complex}.  
   \item[b] a complex number. Must be of type \texttt{sf\_double\_complex}.  
\end{desclist}

\subsubsection*{Output}
\begin{desclist}{\tt }{\quad}[\tt ]
   \setlength\itemsep{0pt}
   \item[c] the product $ab$. It is of type \texttt{sf\_double\_complex}.
\end{desclist}




\subsection{{dccmul}}
Multiplies two complex number. Its output type and one of the input parameters is of type \texttt{kiss\_fft\_cpx}.

\subsubsection*{Call}
\begin{verbatim}c = sf_dccmul(a, b);
\end{verbatim}

\subsubsection*{Definition}
\begin{verbatim}
kiss_fft_cpx sf_dccmul(sf_double_complex a, kiss_fft_cpx b)
/*< complex multiplication >*/
{
   ...
}
\end{verbatim}

\subsubsection*{Input parameters}
\begin{desclist}{\tt }{\quad}[\tt ]
   \setlength\itemsep{0pt}
   \item[a] a complex number. Must be of type \texttt{sf\_double\_complex}.  
   \item[b] a complex number. Must be of type \texttt{kiss\_fft\_cpx}.
\end{desclist}

\subsubsection*{Output}
\begin{desclist}{\tt }{\quad}[\tt ]
   \setlength\itemsep{0pt}
   \item[c] the product $ab$. It is of type \texttt{sf\_double\_complex}.
\end{desclist}




\subsection{{dcdmul}}
Multiplies two complex number. One of the input parameters is \texttt{kiss\_fft\_cpx}. This means that it should only be used if \texttt{complex.h} header is not used.

\subsubsection*{Call}
\begin{verbatim}c = sf_dcdmul(a, b);\end{verbatim}

\subsubsection*{Definition}
\begin{verbatim}
sf_double_complex sf_dcdmul(sf_double_complex a, kiss_fft_cpx b)
/*< complex multiplication >*/
{
   ...
}
\end{verbatim}

\subsubsection*{Input parameters}
\begin{desclist}{\tt }{\quad}[\tt ]
   \setlength\itemsep{0pt}
   \item[a] a complex number. Must be of type \texttt{sf\_double\_complex}.  
   \item[b] a complex number. Must be of type \texttt{kiss\_fft\_cpx}.
\end{desclist}

\subsubsection*{Output}
\begin{desclist}{\tt }{\quad}[\tt ]
   \setlength\itemsep{0pt}
   \item[c] product $ab$ of the two complex numbers $a$ and $b$. It is of type \texttt{sf\_double\_complex}.
\end{desclist}



\subsection{{dcrmul}}
Multiplies a complex number with a real number of type \texttt{double}. 

\subsubsection*{Call}
\begin{verbatim}c = sf_dcrmul(a, b);\end{verbatim}

\subsubsection*{Definition}
\begin{verbatim}
sf_double_complex sf_dcrmul(sf_double_complex a, double b)
/*< complex by real multiplication >*/
{
   ...
}
\end{verbatim}

\subsubsection*{Input parameters}
\begin{desclist}{\tt }{\quad}[\tt ]
   \setlength\itemsep{0pt}
   \item[a] a complex number. Must be of type \texttt{sf\_double\_complex}.  
   \item[b] a real number (\texttt{double}).  
\end{desclist}

\subsubsection*{Output}
\begin{desclist}{\tt }{\quad}[\tt ]
   \setlength\itemsep{0pt}
   \item[c] product of the complex number $a$ and the real number $b$. It is of type \texttt{sf\_double\_complex}.
\end{desclist}




\subsection{{dcdiv}}
Divides two complex numbers.

\subsubsection*{Call}
\begin{verbatim}c = sf_dcdiv(a, b);\end{verbatim}

\subsubsection*{Definition}
\begin{verbatim}
sf_double_complex sf_dcdiv(sf_double_complex a, sf_double_complex b)
/*< complex division >*/
{
   ...
}
\end{verbatim}

\subsubsection*{Input parameters}
\begin{desclist}{\tt }{\quad}[\tt ]
   \setlength\itemsep{0pt}
   \item[a] a complex number. Must be of type \texttt{sf\_double\_complex}.  
   \item[b] a complex number. Must be of type \texttt{sf\_double\_complex}.  
\end{desclist}

\subsubsection*{Output}
\begin{desclist}{\tt }{\quad}[\tt ]
   \setlength\itemsep{0pt}
   \item[c] $\frac{a}{b}$. It is of type \texttt{sf\_double\_complex}.
\end{desclist}




\subsection{{cabs}}
Returns the absolute value (magnitude) of a complex number. It uses the \texttt{hypot} function from the C library.

\subsubsection*{Call}
\begin{verbatim}a = sf_cabs(z);\end{verbatim}

\subsubsection*{Definition}
\begin{verbatim}
double sf_cabs(sf_double_complex z)
/*< replacement for cabsf >*/
{
   ...
}
\end{verbatim}

\subsubsection*{Input parameters}
\begin{desclist}{\tt }{\quad}[\tt ]
   \setlength\itemsep{0pt}
   \item[z] a complex number. Must be of type \texttt{sf\_double\_complex}.  
\end{desclist}

\subsubsection*{Output}
\begin{desclist}{\tt }{\quad}[\tt ]
   \setlength\itemsep{0pt}
   \item[hypot(z.r,z.i)] absolute value of the complex number. 
\end{desclist}




\subsection{{cabs}}
Returns the argument of a complex number. It uses the \texttt{atan2} function from the C library.

\subsubsection*{Call}
\begin{verbatim}u = sf_carg(z);\end{verbatim}

\subsubsection*{Definition}
\begin{verbatim}
double sf_carg(sf_double_complex z)
/*< replacement for cargf >*/
{
   ...
}
\end{verbatim}

\subsubsection*{Input parameters}
\begin{desclist}{\tt }{\quad}[\tt ]
   \setlength\itemsep{0pt}
   \item[z] a complex number. Must be of type \texttt{sf\_double\_complex}.  
\end{desclist}

\subsubsection*{Output}
\begin{desclist}{\tt }{\quad}[\tt ]
   \setlength\itemsep{0pt}
   \item[atan2(z.r,z.i)] argument of the complex number. 
\end{desclist}




\subsection{{crealf}}
Returns the real part of the complex number.

\subsubsection*{Call}
\begin{verbatim}r = sf_crealf(c);\end{verbatim}

\subsubsection*{Definition}
\begin{verbatim}
float sf_crealf(kiss_fft_cpx c)
/*< real part >*/
{
   ...
}
\end{verbatim}

\subsubsection*{Input parameters}
\begin{desclist}{\tt }{\quad}[\tt ]
   \setlength\itemsep{0pt}
   \item[c] a complex number. Must be of type \texttt{kiss\_fft\_cpx}.
\end{desclist}

\subsubsection*{Output}
\begin{desclist}{\tt }{\quad}[\tt ]
   \setlength\itemsep{0pt}
   \item[r] real part of the complex number. It is of type \texttt{float}.
\end{desclist}




\subsection{{cimagf}}
Returns the imaginary part of the complex number.

\subsubsection*{Call}
\begin{verbatim}im = sf_cimagf(c);\end{verbatim}

\subsubsection*{Definition}
\begin{verbatim}
float sf_cimagf(kiss_fft_cpx c)
/*< imaginary part >*/
{
   ...
}
\end{verbatim}

\subsubsection*{Input parameters}
\begin{desclist}{\tt }{\quad}[\tt ]
   \setlength\itemsep{0pt}
   \item[c]      a complex number. Must be of type \texttt{kiss\_fft\_cpx}.
\end{desclist}

\subsubsection*{Output}
\begin{desclist}{\tt }{\quad}[\tt ]
   \setlength\itemsep{0pt}
   \item[im] imaginary part of the complex number. It is of type \texttt{float}.
\end{desclist}




\subsection{{cprint}}
Prints the complex number on the screen. This is done using the \hyperref[sec:sf_warning]{\texttt{sf\_warning}}.

\subsubsection*{Call}
\begin{verbatim}cprint(c);\end{verbatim}

\subsubsection*{Definition}
\begin{verbatim}
void cprint (sf_complex c)
/*< print a complex number (for debugging purposes) >*/
{
   ...
}
\end{verbatim}

\subsubsection*{Input parameters}
\begin{desclist}{\tt }{\quad}[\tt ]
   \setlength\itemsep{0pt}
   \item[c] a complex number (\texttt{sf\_complex}).  
\end{desclist}




\subsection{{cadd}}
Adds two complex numbers. The output is of type \texttt{kiss\_fft\_cpx}.

\subsubsection*{Call}
\begin{verbatim}c = sf_cadd(a, b);\end{verbatim}

\subsubsection*{Definition}
\begin{verbatim}
kiss_fft_cpx sf_cadd(kiss_fft_cpx a, kiss_fft_cpx b)
/*< complex addition >*/
{
   ...
}
\end{verbatim}

\subsubsection*{Input parameters}
\begin{desclist}{\tt }{\quad}[\tt ]
   \setlength\itemsep{0pt}
   \item[a] a complex number. Must be of type \texttt{kiss\_fft\_cpx}.
   \item[b] a complex number. Must be of type \texttt{kiss\_fft\_cpx}.
\end{desclist}

\subsubsection*{Output}
\begin{desclist}{\tt }{\quad}[\tt ]
   \setlength\itemsep{0pt}
   \item[c] the sum $a+b$ of the two complex numbers $a$, $b$. It is of type \texttt{kiss\_fft\_cpx}.
\end{desclist}




\subsection{{csub}}
Subtracts two complex numbers. The output is of type \texttt{kiss\_fft\_cpx}.

\subsubsection*{Call}
\begin{verbatim}c = sf_csub(a, b);\end{verbatim}

\subsubsection*{Definition}
\begin{verbatim}
kiss_fft_cpx sf_csub(kiss_fft_cpx a, kiss_fft_cpx b)
/*< complex subtraction >*/
{
   ...
}
\end{verbatim}

\subsubsection*{Input parameters}
\begin{desclist}{\tt }{\quad}[\tt ]
   \setlength\itemsep{0pt}
   \item[a] a complex number. Must be of type \texttt{kiss\_fft\_cpx}.
   \item[b] a complex number. Must be of type \texttt{kiss\_fft\_cpx}.
\end{desclist}

\subsubsection*{Output}
\begin{desclist}{\tt }{\quad}[\tt ]
   \setlength\itemsep{0pt}
   \item[c] difference of the two complex numbers $a$, $b$. It is of type \texttt{kiss\_fft\_cpx}.
\end{desclist}




\subsection{{csqrtf}}
Returns the square root of a complex number. The output is of type \texttt{kiss\_fft\_cpx}.

\subsubsection*{Call}
\begin{verbatim}c = sf_csqrtf (c);\end{verbatim}

\subsubsection*{Definition}
\begin{verbatim}
kiss_fft_cpx sf_csqrtf (kiss_fft_cpx c)
/*< complex square root >*/
{
   ...
}
\end{verbatim}

\subsubsection*{Input parameters}
\begin{desclist}{\tt }{\quad}[\tt ]
   \setlength\itemsep{0pt}
   \item[a] a complex number. Must be of type \texttt{kiss\_fft\_cpx}.
   \item[b] a complex number. Must be of type \texttt{kiss\_fft\_cpx}.
\end{desclist}

\subsubsection*{Output}
\begin{desclist}{\tt }{\quad}[\tt ]
   \setlength\itemsep{0pt}
   \item[c] square root of the complex number. It is of type \texttt{kiss\_fft\_cpx}.
\end{desclist}




\subsection{{cdiv}}
Divides two complex numbers. The output is of type \texttt{kiss\_fft\_cpx}.

\subsubsection*{Call}
\begin{verbatim}c = sf_cdiv(a, b);\end{verbatim}

\subsubsection*{Definition}
\begin{verbatim}
kiss_fft_cpx sf_cdiv(kiss_fft_cpx a, kiss_fft_cpx b)
/*< complex division >*/
{
   ...
}
\end{verbatim}

\subsubsection*{Input parameters}
\begin{desclist}{\tt }{\quad}[\tt ]
   \setlength\itemsep{0pt}
   \item[a] a complex number. Must be of type \texttt{sf\_double\_complex}.  
   \item[b] a complex number. Must be of type \texttt{sf\_double\_complex}.  
\end{desclist}

\subsubsection*{Output}
\begin{desclist}{\tt }{\quad}[\tt ]
   \setlength\itemsep{0pt}
   \item[c] $\frac{a}{b}$. It is of type \texttt{kiss\_fft\_cpx}.
\end{desclist}




\subsection{{cmul}}
Multiplies two complex numbers. The output is of type \texttt{kiss\_fft\_cpx}.

\subsubsection*{Call}
\begin{verbatim}c = sf_cmul(a, b);\end{verbatim}

\subsubsection*{Definition}
\begin{verbatim}
kiss_fft_cpx sf_cmul(kiss_fft_cpx a, kiss_fft_cpx b)
/*< complex multiplication >*/
{
   ...
}
\end{verbatim}

\subsubsection*{Input parameters}
\begin{desclist}{\tt }{\quad}[\tt ]
   \setlength\itemsep{0pt}
   \item[a] a complex number. Must be of type \texttt{sf\_double\_complex}.  
   \item[b] a complex number. Must be of type \texttt{sf\_double\_complex}.  
\end{desclist}

\subsubsection*{Output}
\begin{desclist}{\tt }{\quad}[\tt ]
   \setlength\itemsep{0pt}
   \item[c] product of the two complex numbers a and b. It is of type \texttt{kiss\_fft\_cpx}.
\end{desclist}




\subsection{{crmul}}
Multiplies a complex number with a real number. The output is of type \texttt{kiss\_fft\_cpx}.

\subsubsection*{Call}
\begin{verbatim}c = sf_crmul(a, b);\end{verbatim}

\subsubsection*{Definition}
\begin{verbatim}
kiss_fft_cpx sf_crmul(kiss_fft_cpx a, float b)
/*< complex by real multiplication >*/
{
   ...
}
\end{verbatim}

\subsubsection*{Input parameters}
\begin{desclist}{\tt }{\quad}[\tt ]
   \setlength\itemsep{0pt}
   \item[a] a complex number. Must be of type \texttt{sf\_double\_complex}.  
   \item[b] a real number (\texttt{float}).  
\end{desclist}

\subsubsection*{Output}
\begin{desclist}{\tt }{\quad}[\tt ]
   \setlength\itemsep{0pt}
   \item[c] the product $ab$ of a complex number $a$ and real number $b$. It is of type \texttt{kiss\_fft\_cpx}.
\end{desclist}




\subsection{{cneg}}
Returns negative of a complex number. The output is of type \texttt{kiss\_fft\_cpx}.

\subsubsection*{Call}
\begin{verbatim}b = sf_cneg(a);\end{verbatim}

\subsubsection*{Definition}
\begin{verbatim}
kiss_fft_cpx sf_cneg(kiss_fft_cpx a)
/*< unary minus >*/
{
   ...
}
\end{verbatim}

\subsubsection*{Input parameters}
\begin{desclist}{\tt }{\quad}[\tt ]
   \setlength\itemsep{0pt}
   \item[a] a complex number. Must be of type \texttt{sf\_double\_complex}.  
\end{desclist}

\subsubsection*{Output}
\begin{desclist}{\tt }{\quad}[\tt ]
   \setlength\itemsep{0pt}
   \item[a] negative of a complex number. It is of type \texttt{kiss\_fft\_cpx}.
\end{desclist}




\subsection{{conjf}}
Returns complex conjugate of a complex number. The output is of type \texttt{kiss\_fft\_cpx}.

\subsubsection*{Call}
\begin{verbatim}z1 = sf_conjf(z);\end{verbatim}

\subsubsection*{Definition}
\begin{verbatim}
kiss_fft_cpx sf_conjf(kiss_fft_cpx z)
/*< complex conjugate >*/
{
   ...
}
\end{verbatim}

\subsubsection*{Input parameters}
\begin{desclist}{\tt }{\quad}[\tt ]
   \setlength\itemsep{0pt}
   \item[z] a complex number. Must be of type \texttt{sf\_double\_complex}.  
\end{desclist}

\subsubsection*{Output}
\begin{desclist}{\tt }{\quad}[\tt ]
   \setlength\itemsep{0pt}
   \item[z1] complex conjugate of the complex number. It is of type \texttt{kiss\_fft\_cpx}.
\end{desclist}




\subsection{{cabsf}}
Returns the magnitude of a complex number. It uses a function \texttt{hypotf} from \texttt{c99.h}, which calls the \texttt{hypot} function from \texttt{math.h} in the C library.

\subsubsection*{Call}
\begin{verbatim}w = sf_cabsf(kiss_fft_cpx z);\end{verbatim}

\subsubsection*{Definition}
\begin{verbatim}
float sf_cabsf(kiss_fft_cpx z)
/*< replacement for cabsf >*/
{
   ...
}
\end{verbatim}

\subsubsection*{Input parameters}
\begin{desclist}{\tt }{\quad}[\tt ]
   \setlength\itemsep{0pt}
   \item[z] a complex number. Must be of type \texttt{kiss\_fft\_cpx}.
\end{desclist}

\subsubsection*{Output}
\begin{desclist}{\tt }{\quad}[\tt ]
   \setlength\itemsep{0pt}
   \item[w] magnitude of a complex number. It is of type \texttt{kiss\_fft\_cpx}.
\end{desclist}




\subsection{{cargf}}
Returns the argument of a complex number. It uses a function \texttt{atan2f} from \texttt{c99.h}, which calls the \texttt{atan2} function from \texttt{math.h} in the C library.

\subsubsection*{Call}
\begin{verbatim}u = sf_cargf(z);\end{verbatim}

\subsubsection*{Definition}
\begin{verbatim}
float sf_cargf(kiss_fft_cpx z)
/*< replacement for cargf >*/
{
   ...
}
\end{verbatim}

\subsubsection*{Input parameters}
\begin{desclist}{\tt }{\quad}[\tt ]
   \setlength\itemsep{0pt}
   \item[z] a complex number. Must be of type \texttt{kiss\_fft\_cpx}.
\end{desclist}

\subsubsection*{Output}
\begin{desclist}{\tt }{\quad}[\tt ]
   \setlength\itemsep{0pt}
   \item[u] argument of a complex number. It is of type \texttt{kiss\_fft\_cpx}.
\end{desclist}




\subsection{{ctanhf}}
Returns hyperbolic tangent of a complex number. It uses a function
like \texttt{coshf} and \texttt{sinhf} from \texttt{c99.h}, which call \texttt{cosh} and \texttt{sinh} functions from \texttt{math.h} in the C library.

\subsubsection*{Call}
\begin{verbatim}th = sf_ctanhf(z);\end{verbatim}

\subsubsection*{Definition}
\begin{verbatim}
kiss_fft_cpx sf_ctanhf(kiss_fft_cpx z)
/*< complex hyperbolic tangent >*/
{
   ...
}
\end{verbatim}

\subsubsection*{Input parameters}
\begin{desclist}{\tt }{\quad}[\tt ]
   \setlength\itemsep{0pt}
   \item[z] a complex number. Must be of type \texttt{kiss\_fft\_cpx}.
\end{desclist}

\subsubsection*{Output}
\begin{desclist}{\tt }{\quad}[\tt ]
   \setlength\itemsep{0pt}
   \item[th] hyperbolic tangent of a complex number. It is of type \texttt{kiss\_fft\_cpx}.
\end{desclist}




\subsection{{ccosf}}
Returns cosine of a complex number. It uses the functions like \texttt{coshf} and \texttt{sinhf} from \texttt{c99.h}, which call \texttt{cosh} and \texttt{sinh} functions from \texttt{math.h} in the C library.

\subsubsection*{Call}
\begin{verbatim}w =  sf_ccosf (z);\end{verbatim}

\subsubsection*{Definition}
\begin{verbatim}
kiss_fft_cpx sf_ccosf(kiss_fft_cpx z)
/*< complex cosine >*/
{
   ...
}
\end{verbatim}

\subsubsection*{Input parameters}
\begin{desclist}{\tt }{\quad}[\tt ]
   \setlength\itemsep{0pt}
   \item[z] a complex number. Must be of type \texttt{kiss\_fft\_cpx}.
\end{desclist}

\subsubsection*{Output}
\begin{desclist}{\tt }{\quad}[\tt ]
   \setlength\itemsep{0pt}
   \item[w] cosine of a complex number. It is of type \texttt{kiss\_fft\_cpx}.
\end{desclist}




\subsection{{ccoshf}}
Returns hyperbolic cosine of a complex number. It uses the functions like \texttt{coshf} and \texttt{sinhf} from \texttt{c99.h}, which call \texttt{cosh} and \texttt{sinh} functions from \texttt{math.h} in the C library.

\subsubsection*{Call}
\begin{verbatim}w = sf_ccoshf(z);\end{verbatim}

\subsubsection*{Definition}
\begin{verbatim}
kiss_fft_cpx sf_ccoshf(kiss_fft_cpx z)
/*< complex hyperbolic cosine >*/
{
   ...    
}
\end{verbatim}

\subsubsection*{Input parameters}
\begin{desclist}{\tt }{\quad}[\tt ]
   \setlength\itemsep{0pt}
   \item[z] a complex number. Must be of type \texttt{kiss\_fft\_cpx}.
\end{desclist}

\subsubsection*{Output}
\begin{desclist}{\tt }{\quad}[\tt ]
   \setlength\itemsep{0pt}
   \item[w] huperbolic cosine of a complex number. It is of type \texttt{kiss\_fft\_cpx}.
\end{desclist}




\subsection{{ccosf}}
Returns sine of a complex number. It uses the functions like \texttt{coshf} and \texttt{sinhf} from \texttt{c99.h}, which call \texttt{cosh} and \texttt{sinh} functions from \texttt{math.h} in the C library.

\subsubsection*{Call}
\begin{verbatim}w = sf_csinf(z);\end{verbatim}

\subsubsection*{Definition}
\begin{verbatim}
kiss_fft_cpx sf_csinf(kiss_fft_cpx z)
/*< complex sine >*/
{
   ...
}
\end{verbatim}

\subsubsection*{Input parameters}
\begin{desclist}{\tt }{\quad}[\tt ]
   \setlength\itemsep{0pt}
   \item[z] a complex number. Must be of type \texttt{kiss\_fft\_cpx}.
\end{desclist}

\subsubsection*{Output}
\begin{desclist}{\tt }{\quad}[\tt ]
   \setlength\itemsep{0pt}
   \item[w] sine of a complex number. It is of type \texttt{kiss\_fft\_cpx}.
\end{desclist}




\subsection{{csinhf}}
Returns hyperbolic cosine of a complex number. It uses the functions like \texttt{coshf} and \texttt{sinhf} from \texttt{c99.h}, which call \texttt{cosh} and \texttt{sinh} functions from \texttt{math.h} in the C library.

\subsubsection*{Call}
\begin{verbatim}w = sf_csinhf(z);\end{verbatim}

\subsubsection*{Definition}
\begin{verbatim}
kiss_fft_cpx sf_csinhf(kiss_fft_cpx z)
/*< complex hyperbolic sine >*/
{
   ...
}
\end{verbatim}

\subsubsection*{Input parameters}
\begin{desclist}{\tt }{\quad}[\tt ]
   \setlength\itemsep{0pt}
   \item[z] a complex number. Must be of type \texttt{kiss\_fft\_cpx}.
\end{desclist}

\subsubsection*{Output}
\begin{desclist}{\tt }{\quad}[\tt ]
   \setlength\itemsep{0pt}
   \item[w] huperbolic cosine of a complex number. It is of type \texttt{kiss\_fft\_cpx}.
\end{desclist}




\subsection{{clogf}}
Returns natural logarithm of a complex number. It uses the functions like \texttt{logf} and \texttt{hypotf} from \texttt{c99.h}, which call \texttt{log} and \texttt{hypot} functions from \texttt{math.h} in the C library.

\subsubsection*{Call}
\begin{verbatim}w = sf_clogf(z);\end{verbatim}

\subsubsection*{Definition}
\begin{verbatim}
kiss_fft_cpx sf_clogf(kiss_fft_cpx z)
/*< complex natural logarithm >*/
{
   ...    
}
\end{verbatim}

\subsubsection*{Input parameters}
\begin{desclist}{\tt }{\quad}[\tt ]
   \setlength\itemsep{0pt}
   \item[z] a complex number. Must be of type \texttt{kiss\_fft\_cpx}.
\end{desclist}

\subsubsection*{Output}
\begin{desclist}{\tt }{\quad}[\tt ]
   \setlength\itemsep{0pt}
   \item[w] natural logarithm of a complex number. It is of type \texttt{kiss\_fft\_cpx}.
\end{desclist}




\subsection{{cexpf}}
Returns exponential of a complex number. It uses the functions like \texttt{expf} and \texttt{cosf} from \texttt{c99.h}, which call \texttt{exp} and \texttt{cos} functions from \texttt{math.h} in the C library.

\subsubsection*{Call}
\begin{verbatim}w = sf_cexpf(z);\end{verbatim}

\subsubsection*{Definition}
\begin{verbatim}
kiss_fft_cpx sf_cexpf(kiss_fft_cpx z)
/*< complex exponential >*/
{
   ...
}
\end{verbatim}

\subsubsection*{Input parameters}
\begin{desclist}{\tt }{\quad}[\tt ]
   \setlength\itemsep{0pt}
   \item[z] a complex number. Must be of type \texttt{kiss\_fft\_cpx}.
\end{desclist}

\subsubsection*{Output}
\begin{desclist}{\tt }{\quad}[\tt ]
   \setlength\itemsep{0pt}
   \item[z] exponential of a complex number. It is of type \texttt{kiss\_fft\_cpx}.
\end{desclist}




\subsection{{ctanf}}
Returns tangent of a complex number. It uses the functions like \texttt{sinf} and \texttt{cosf} from \texttt{c99.h}, which call \texttt{sin} and \texttt{cos} functions from \texttt{math.h} in the C library.

\subsubsection*{Call}
\begin{verbatim}w = sf_ctanf(z);\end{verbatim}

\subsubsection*{Definition}
\begin{verbatim}
kiss_fft_cpx sf_ctanf(kiss_fft_cpx z)
/*< complex tangent >*/
{
   ...
}
\end{verbatim}

\subsubsection*{Input parameters}
\begin{desclist}{\tt }{\quad}[\tt ]
   \setlength\itemsep{0pt}
   \item[z] a complex number. Must be of type \texttt{kiss\_fft\_cpx}.
\end{desclist}

\subsubsection*{Output}
\begin{desclist}{\tt }{\quad}[\tt ]
   \setlength\itemsep{0pt}
   \item[w] tangent of a complex number. It is of type \texttt{kiss\_fft\_cpx}.
\end{desclist}




\subsection{{casinf}}
Returns hyperbolic arcsine of a complex number. It uses the function \texttt{asinf} from \texttt{c99.h}, which calls the \texttt{asin} function from \texttt{math.h} in the C library.

\subsubsection*{Call}
\begin{verbatim}w = sf_casinf(z);\end{verbatim}

\subsubsection*{Definition}
\begin{verbatim}
kiss_fft_cpx sf_casinf(kiss_fft_cpx z)
/*< complex hyperbolic arcsine >*/
{
   ...  
}
\end{verbatim}

\subsubsection*{Input parameters}
\begin{desclist}{\tt }{\quad}[\tt ]
   \setlength\itemsep{0pt}
   \item[z] a complex number. Must be of type \texttt{kiss\_fft\_cpx}.
\end{desclist}

\subsubsection*{Output}
\begin{desclist}{\tt }{\quad}[\tt ]
   \setlength\itemsep{0pt}
   \item[w] arcsine of a complex number. It is of type \texttt{kiss\_fft\_cpx}.
\end{desclist}




\subsection{{cacosf}}
Returns hyperbolic arccosine of a complex number. It uses \texttt{sf\_cacosf}.

\subsubsection*{Call}
\begin{verbatim}w = sf_cacosf(z);\end{verbatim}

\subsubsection*{Definition}
\begin{verbatim}
kiss_fft_cpx sf_cacosf(kiss_fft_cpx z)
/*< complex hyperbolic arccosine >*/
{
   ...
}
\end{verbatim}

\subsubsection*{Input parameters}
\begin{desclist}{\tt }{\quad}[\tt ]
   \setlength\itemsep{0pt}
   \item[z] a complex number. Must be of type \texttt{kiss\_fft\_cpx}.
\end{desclist}

\subsubsection*{Output}
\begin{desclist}{\tt }{\quad}[\tt ]
   \setlength\itemsep{0pt}
   \item[w] hyperbolic arccosine of a complex number. It is of type \texttt{kiss\_fft\_cpx}.
\end{desclist}




\subsection{{catanf}}
Returns arctangent of a complex number. It uses \texttt{sf\_clogf} and \texttt{sf\_cdiv}.

\subsubsection*{Call}
\begin{verbatim}w = sf_catanf(z);\end{verbatim}

\subsubsection*{Definition}
\begin{verbatim}
kiss_fft_cpx sf_catanf(kiss_fft_cpx z)
/*< complex arctangent >*/
{
   ...   
}     
\end{verbatim}

\subsubsection*{Input parameters}
\begin{desclist}{\tt }{\quad}[\tt ]
   \setlength\itemsep{0pt}
   \item[z] a complex number. Must be of type \texttt{kiss\_fft\_cpx}.
\end{desclist}

\subsubsection*{Output}
\begin{desclist}{\tt }{\quad}[\tt ]
   \setlength\itemsep{0pt}
   \item[w] arctangent of a complex number. It is of type \texttt{kiss\_fft\_cpx}.
\end{desclist}



\subsection{{catanhf}}
Returns hyperbolic arctangent of a complex number. It uses \texttt{sf\_catanf}.

\subsubsection*{Call}
\begin{verbatim}w =_cpx sf_catanhf(z);\end{verbatim}

\subsubsection*{Definition}
\begin{verbatim}
kiss_fft_cpx sf_catanhf(kiss_fft_cpx z)
/*< complex hyperbolic arctangent >*/
{
   ...
}     
\end{verbatim}

\subsubsection*{Input parameters}
\begin{desclist}{\tt }{\quad}[\tt ]
   \setlength\itemsep{0pt}
   \item[z] a complex number. Must be of type \texttt{kiss\_fft\_cpx}.
\end{desclist}

\subsubsection*{Output}
\begin{desclist}{\tt }{\quad}[\tt ]
   \setlength\itemsep{0pt}
   \item[z] hyperbolic arctangent of a complex number. It is of type \texttt{kiss\_fft\_cpx}.
\end{desclist}



\subsection{{casinhf}}
Returns hyperbolic arcsine of a complex number. It uses \texttt{sf\_casinf}.

\subsubsection*{Call}
\begin{verbatim}w = sf_casinhf(z);\end{verbatim}

\subsubsection*{Definition}
\begin{verbatim}
kiss_fft_cpx sf_casinhf(kiss_fft_cpx z)
/*< complex hyperbolic sine >*/
{
   ...
}     
\end{verbatim}

\subsubsection*{Input parameters}
\begin{desclist}{\tt }{\quad}[\tt ]
   \setlength\itemsep{0pt}
   \item[z] a complex number. Must be of type \texttt{kiss\_fft\_cpx}.
\end{desclist}

\subsubsection*{Output}
\begin{desclist}{\tt }{\quad}[\tt ]
   \setlength\itemsep{0pt}
   \item[z] hyperbolic arcsine of a complex number. It is of type \texttt{kiss\_fft\_cpx}.
\end{desclist}



\subsection{{cacoshf}}
Returns hyperbolic arccosine of a complex number. It uses \texttt{sf\_casinf}.

\subsubsection*{Call}
\begin{verbatim}w = sf_cacoshf(z);\end{verbatim}

\subsubsection*{Definition}
\begin{verbatim}
kiss_fft_cpx sf_cacoshf(kiss_fft_cpx z)
/*< complex hyperbolic cosine >*/
{
   ...
}
\end{verbatim}

\subsubsection*{Input parameters}
\begin{desclist}{\tt }{\quad}[\tt ]
   \setlength\itemsep{0pt}
   \item[z] a complex number. Must be of type \texttt{kiss\_fft\_cpx}.
\end{desclist}

\subsubsection*{Output}
\begin{desclist}{\tt }{\quad}[\tt ]
   \setlength\itemsep{0pt}
   \item[w] hyperbolic arccosine of a complex number. It is of type \texttt{kiss\_fft\_cpx}.
\end{desclist}




\subsection{{cpowf}}
Returns the complex base $a$ raised to complex power $b$.

\subsubsection*{Call}
\begin{verbatim}c = sf_cpowf(a, b);\end{verbatim}

\subsubsection*{Definition}
\begin{verbatim}
kiss_fft_cpx sf_cpowf(kiss_fft_cpx a, kiss_fft_cpx b)
/*< complex power >*/
{
   ...
}
\end{verbatim}

\subsubsection*{Input parameters}
\begin{desclist}{\tt }{\quad}[\tt ]
   \setlength\itemsep{0pt}
   \item[a] a complex number. Must be of type \texttt{kiss\_fft\_cpx}.
   \item[b] a complex number. Must be of type \texttt{kiss\_fft\_cpx}.
\end{desclist}

\subsubsection*{Output}
\begin{desclist}{\tt }{\quad}[\tt ]
   \setlength\itemsep{0pt}
   \item[c] $a^b$. It is a complex number of type \texttt{kiss\_fft\_cpx}.
\end{desclist}



       % Complex number operations

\chapter{Error handling}\label{sec:error}
   \section{Handling warning and error messages (error.c)}




\subsection{{sf\_error}}\label{sec:sf_error}
Outputs an error message to \texttt{stderr}, which is usually the screen. It uses \texttt{sf\_getprog} to get the name of the program which is causing the error and print it on the screen.  Uses \texttt{vfprintf}, which can take a variable number of arguments initialized by \texttt{va\_list}. This gives the user flexibility in choosing the number of arguments.

If there is a '\texttt{:}' at the end of format, information about the system errors is printed, this is done by using \texttt{strerror} to interpret the last error number \texttt{errno} in the system. Also, if there is a '\texttt{;}' at the end of a format the command prompt will not go to the next line.

\subsubsection*{Call}
\begin{verbatim}sf_error (format, ...);\end{verbatim}

\subsubsection*{Definition}
\begin{verbatim}
void sf_error( const char *format, ... )
/*< Outputs an error message to stderr and terminates the program. 
    ---
    Format and variable arguments follow printf convention. Additionally, a ':' at
    the end of format adds system information for system errors. >*/
{
   ...
}
\end{verbatim}

\subsubsection*{Input parameters}
\begin{desclist}{\tt }{\quad}[\tt format]
   \setlength\itemsep{0pt}
   \item[format] a string of \texttt{type \texttt{const char*}} containing the format specifiers for the arguments to be input from the next commands.
   \item[...]    variable number of arguments, which are to replace the format specifiers in the format.    
\end{desclist}

\subsubsection*{Output}
An error message output to \texttt{sterr} (usually printed on screen).




\subsection{{sf\_warning}}\label{sec:sf_warning}
Outputs a warning message to \texttt{stderr} which is usually the screen. It uses \texttt{sf\_getprog} to get the name of the program which is causing the error and print it on the screen.  Uses \texttt{vfprintf}, which can take a variable number of arguments initialized by \texttt{va\_list}. This gives the user flexibility in choosing the number of arguments.
If there is a '\texttt{:}' at the end of format, information about the system errors is printed, this is done by using \texttt{strerror} to interpret the last error number \texttt{errno} in the system. Also, if there is a '\texttt{;}' at the end of a format the command prompt will not go to the next line.

\subsubsection*{Call}
\begin{verbatim}sf_warning (format, ... );\end{verbatim}

\subsubsection*{Definition}
\begin{verbatim}
void sf_warning( const char *format, ... )
/*< Outputs a warning message to stderr. 
---
Format and variable arguments follow printf convention. Additionally, a ':' at
the end of format adds system information for system errors. >*/
{
   ...
}
\end{verbatim}

\subsubsection*{Input parameters}
\begin{desclist}{\tt }{\quad}[\tt format]
   \setlength\itemsep{0pt}
   \item[format] a string of \texttt{type \texttt{const char*}} containing the format specifiers for the arguments to be input from the next commands. 
   \item[...]    variable number of arguments, which are to replace the format specifiers in the format.     
\end{desclist}

\subsubsection*{Output}
A warning message output to \texttt{sterr} ( usually printed on screen).



         % Error handling

\chapter{Linear operators}\label{sec:lop}
   \section{Introduction}

This section contains a bunch of programs that implement operators. Therefore a short introduction on operators is in order.

\subsection{Definition of operators}
Mathematically speaking an operator is a function of a function, i.e.~a rule (or mapping) according to which a function  $f$ is transformed into another function $g$. We use the notation $g=R[f]$ or simply $g=Rf$, where $R$ denotes the operator. Examples of operators are the derivative, the integral, convolution (with a specific function), multiplication by a scalar and others. Note that in general the domains of $f$ and $g$ are not necessarily the same. For example, in the case of the derivative, the domain of $g=Rf$ is the subset of the domain of $f$, in which $f$ is smooth. In particular if $f=|x|$, $x\in[-1,1]$, then the domain of $g$ is $(-1,0)\cup(0,1)$. 

An important class of operators are the \textbf{linear\index{operator!linear} operators}. An operator $L$ is linear if for any two functions $f_1$, $f_2$ and any two scalars $a_1$, $a_2$, $L[a_1f_1+a_1f_2]=a_1Lf_1 + a_2Lf_2$. The derivative, integral, convolution and multiplication by scalar are all linear operators.

In the discrete world, operators act on vectors and linear operators are in fact matrices, with which the vectors are multiplied. (Multiplication by a matrix is a linear operation, since $\mathbf{M}(a_1\mathbf{x}_1+a_2\mathbf{x}_2) = a_1\mathbf{M}\mathbf{x}_1+a_2\mathbf{M}\mathbf{x}_2$). In fact many of the calculations performed routinely in science and engineering are essentially matrix multiplications in disguise. For example assume a vector $\mathbf{x}=[x_1\;x_2\;\ldots\;x_n]^T$  with length $n$ (superscript ${}^T$ denotes transpose). Padding this vector with $m$ zeros, produces another vector $\mathbf{y}$ with 
\begin{gather*}
    \mathbf{y} = \begin{bmatrix}x_1\\x_2\\\vdots \\x_n\\0\\\vdots\\0\end{bmatrix}
               = \begin{bmatrix}\mathbf{x}\\\mathbf{0}\end{bmatrix}, 
\intertext{where $\mathbf{0}$ is the zero vector of length $m$. One can readily verify that zero padding is a \index{zero-padding operator}\index{operator!zero padding} linear operation with operator matrix $\mathbf{L}=\begin{bmatrix}\mathbf{I}\\\mathbf{O}\end{bmatrix}$, where $\mathbf{I}$ is the $n\times n$ identity matrix and $\mathbf{O}$ is the $m\times n$ zero matrix, since}
    \mathbf{y} = \mathbf{L}\mathbf{x} = \begin{bmatrix}\mathbf{I}\\\mathbf{O}\end{bmatrix}\mathbf{x} 
               = \begin{bmatrix}\mathbf{x}\\\mathbf{0}\end{bmatrix}.
\end{gather*}
Note that as in the case of functions, the domains of $\mathbf{x}$ and $\mathbf{y}$ are different: $\mathbf{x}\in\mathbb{R}^n$ (or more generally $\mathbf{x}\in\mathbb{C}^n$), while $\mathbf{y}\in\mathbb{R}^{n+m}$ (or $\mathbb{C}^{n+m}$).

Similarly, one can define \index{convolution operator}\index{operator!convolution} convolution of $\mathbf{x}$ with $\mathbf{a}=[a_1\;a_2\;\dots\;a_m]^T$ as the multiplication of $\mathbf{x}$ with 
\begin{gather*}
    \mathbf{A} = \begin{bmatrix} a_1 &  0  &  0  & \cdots &  0  &  0\\
                                 a_2 & a_1 &  0  & \cdots &  0  &  0\\
                                 a_3 & a_2 & a_1 & \cdots &  0  &  0\\
                                 \vdots & \vdots & \vdots & \ddots & \vdots & \vdots\\
                                  0  &  0  &  0  & \cdots & a_{m-1} & a_{m-2} \\ 
                                  0  &  0  &  0  & \cdots & a_m & a_{m-1} \\ 
                                  0  &  0  &  0  & \cdots &  0  & a_m
                 \end{bmatrix}. 
\end{gather*}
and many other operations as matrix multiplications. Other operators are the identity\index{identity operator}\index{operator!identity} operator is the identity matrix $\mathbf{I}$ and is implemented by \hyperref[sec:copy]{\texttt{copy.c}} and \hyperref[sec:ccopy]{\texttt{ccopy.c}} and the null \index{null operator}\index{operator!null}operator (or zero matrix $\mathbf{O}$), which is implemented by \hyperref[sec:adjnull]{\texttt{adjnull.c}}. For the rest of this introduction, the boldface notation will imply specifically discrete operators, while the normal fonts will imply operators on either continuous or discrete mathematical entities.

\subsection{Products of operators}
The result of an operation on a function is another function, therefore we can naturally apply an operator on another operator. In other words, if $L_1$, $L_2$ are two operators, then we can define $L_1L_2$ as $L_1L_2[x] = L_1[L_2[x]]$, provided that $L_1[L_2[x]]$ makes sense mathematically. This is called the composition of the operators $L_1$ and $L_2$. Because in the discrete case the composition of operators is in fact the multiplication $L_1L_2$ of the two matrices $L_1$, $L_2$ the operator\index{operator!product}\index{product!of operators} composition is usually referred to as operator product and denoted by $L_1L_2$ is used. The composition of operators can be naturally extended to any finite product $L_1\cdots L_{n-1}L_n$. The product of up to 3 operators is implemented in \hyperref[sec:chain]{\texttt{chain.c}}.

\subsection{Adjoint operators}
A very important notion in data processing is the \textbf{adjoint operator}\footnote{The adjoint operator should not be confused with the (classical) adjoint or adjucate or adjunct matrix of a square matrix. The adjugate matrix of an invertible matrix is the inverse multiplied by its determinant. } $L^*$ of an operator $L$. In the discrete world, the adjoint\index{operator!adjoint}\index{adjoint} operator of $L$ is its (conjugate) transpose, i.e.~$L^*=L^H$. From this definition of the adjoint it is evident that {\it the adjoint of the adjoint is the original operator }(since $(L^*)^*=(L^H)^H=L$). Consider a vector $\mathbf{y}=[y_1\,y_2,\,\ldots,y_{n+m}]^T$ and the adjoint of the zero-padding operator, $\mathbf{L}^*=\mathbf{L}^H=\begin{bmatrix}\mathbf{I}\\\mathbf{O}\end{bmatrix}^T=\begin{bmatrix}\mathbf{I} & \mathbf{O}^T\end{bmatrix}$. Then
\begin{gather*}
    \mathbf{L}^{*}\mathbf{y} = \begin{bmatrix}\mathbf{I} & \mathbf{O}^T\end{bmatrix}\mathbf{y}
                     = \begin{bmatrix}\mathbf{I} & \mathbf{O}^T\end{bmatrix}
                       \begin{bmatrix}y_1\\y_2\\\vdots \\y_n\\y_{n+1}\\\vdots\\y_{n+m}\end{bmatrix}
                     = \begin{bmatrix}y_1\\y_2\\\vdots \\y_n\end{bmatrix}.
\end{gather*}
We conclude that the adjoint of data zero-padding is data \index{truncation operator}\index{operator!truncation}truncation. It is also easy (but tedious) to verify that the adjoint operation of the convolution between $\mathbf{a}$ and $\mathbf{x}$ is the \index{operator!crosscorrelation}\index{crosscorrelation operation} crosscorrelation of $\mathbf{a}$ with $\mathbf{y}^H$. One may also notice that for the specific zero-padding operator, $\mathbf{L}^{*}\mathbf{L}\mathbf{x} = \mathbf{x}$, i.e.~in this case the adjoint neutralizes the effect of the operator. It is tempting to say that the adjoint operation is the inverse operation, however this is not the case: it is not always the case that $L^*L = LL^*$. In fact such an equality is meaningless mathematically if $\mathbf{L}$ is not a square matrix (if $\mathbf{L}$ is a $n\times m$ matrix, then $\mathbf{L}\mathbf{L}^{*}$ is $n\times n$, while $\mathbf{L}^{*}\mathbf{L}$ is $m\times m$). $L^*$ is not even the left inverse of $L$: notice that in the case of the zero padding operator, $(\mathbf{L}^*)^*\mathbf{L}^{*}\mathbf{y}=\mathbf{L}\mathbf{L}^{*}\mathbf{y}=\tilde{\mathbf{y}} \neq \mathbf{y}$ (the last $m$ elements of $\tilde{\mathbf{y}}$ are zero). However it is often the case that the adjoint is an adequate. It is the case though quite often that the adjoint is adequate approximation to the inverse (sometimes within a scaling factor) and it is also quite probable that the adjoint will do a better job than the inverse in inverse problems. This is because the adjoint operator tolerates data imperfections, which the inverse does not.

From the definition of the adjoint operation as the left multiplication the complex conjugate matrix, it follows that {\it the adjoint of the product of two linear operators equals the product of the adjoints in reverse order}, i.e.~$(L_1L_2)^*=L_2^*L_1^*$. This is naturally extended to the product of any finite product of operators, i.e.~$(L_1L_2\cdots L_n)^*=L_n^*L_{n-1}^*\cdots L_1^*$. The adjoint of the product is also implemented in \hyperref[sec:chain]{\texttt{chain.c}}.


\subsection{The dot-product test}\label{sec:dot-product test}
The dot-product test is a valuable checkpoint, which can tell us whether the implementation of the adjoint operator is wrong (however it cannot guarantee that it is indeed correct). The concept is the following: Assuming that we have coded an operator $L$ and its adjoint $L^*$. Then for any two vectors or functions $a$ and $b$,
\begin{gather}\label{eq:dotproducttest}
	\langle a,Lb\rangle = \langle(L^*a)^*,b\rangle
\end{gather} 
where $\langle\,,\,{}\rangle$ denotes the dot product. Remember that the dot product of two functions $f,g\in \mathbb{L}_2$ is $\int fg^*\;dt$ while the dot product of two vectors $\mathbf{x}$ and $\mathbf{y}$ is $\mathbf{x}^H\mathbf{y}$. Notice that for vectors eq.~\eqref{eq:dotproducttest} becomes $\mathbf{x}^H\mathbf{Ly} = (\mathbf{L}^H\mathbf{y})^H\mathbf{y}$ which is obviously true.
The lhs of eq.~\eqref{eq:dotproducttest} is computed using $L$, while the rhs is computed using the adjoint $L^*$. For the dot-product test, one just needs to load the vectors \texttt{x} and \texttt{y} with random numbers and perform the two computations. If the two results are not equal (within machine precision), then the computation of either $L$ or $L^*$ is erroneous. Note that truncation errors have identical effects on both operators, so the two results should be almost equal. The dot-product test (for real operators only) is implemented by \hyperref[sec:sf_dot_test]{\texttt{sf\_dot\_test}}.

\subsection{Implementation of operators}
It should be evident by now that the implementation of an operator $L$ should have at least four arguments: a variable \texttt{x} from which the operand (entity on which $L$ is applied) $x$ is read along with its length \texttt{nx}, and the variable \texttt{y} in which the result $y=Lx$ is stored and its length $n_y$. 

Also, since every operator comes along with its adjoint, the implementation of the linear operators described later in this chapter, gives also the possibility to compute the adjoint operator. This is done through the boolean \texttt{adj} input argument. {\it When {\rm\texttt{adj}} is {\rm\texttt{true}}, the adjoint operator $L^*$ computed}. As discussed before, the domains of $x$ and $Lx$ are in general different, therefore $L^*$ cannot be applied on $x$. However it can always be applied on $Lx$ or some $y$, which has the same domain as $Lx$. For this reason, when \texttt{adj} is \texttt{true}, the operand is $y$ and the result is $x$ and thus, \texttt{y} is used as input and the result is stored in \texttt{x}. As an example if \hyperref[sec:sf_copy_lop]{\texttt{sf\_copy\_lop}} (the identity operator) is called, then the result is that $y\leftarrow x$. However if additionally \texttt{adj} is \texttt{true}, then the result will be $x\leftarrow y$. If \hyperref[sec:adjnull]{\texttt{adjnull}} (the null operator) is called, then the result is that $y\leftarrow 0$. However if additionally \texttt{adj} is \texttt{true}, then the result will be $x\leftarrow 0$.

Finally, it is often the case that we need to compute $y\leftarrow Lx$ but $y\leftarrow y+Lx$. For this reason another boolean argument, namely \texttt{add} is defined. If \texttt{add} is true, then $y\leftarrow y+Lx$. Considering the same example with the identity operator, if \texttt{sf\_copy\_lop} is called with \texttt{add} being \texttt{true}, then $y\leftarrow y+x$. If additionally \texttt{adj} is \texttt{true}, then $x\leftarrow y+x$. Or if \texttt{adjnull} is called with \texttt{add} being \texttt{true}, if \texttt{adj} is \texttt{false}, $y\leftarrow y$ and if \texttt{adj} is \texttt{true}, then $x\leftarrow x$ (so in essence, if \texttt{add} is \texttt{true}, no matter what the value of \texttt{adj}, nothing happens).

As a conclusion, the linear operators described in this chapter have all the following form:
\begin{center}\texttt{oper(adj, add, nx, ny, x, y)},\end{center} 
where \texttt{adj} and \texttt{add} are boolean, \texttt{nx} and \texttt{ny} are integers and \texttt{x} and \texttt{y} are pointers of various but the same data type. Table \ref{tab:adjadd} summarizes the effect of the \texttt{adj} and \texttt{add} variables.

% Requires the booktabs if the memoir class is not being used
\begin{table}[htbp]
   \centering
   %\topcaption{Table captions are better up top} % requires the topcapt package
   \caption{Returned values for linear operations.}
   \begin{tabular}{ccll} % Column formatting, @{} suppresses leading/trailing space
      \hline\hline
      \texttt{adj} & \texttt{add} & description & returns\\
      \hline\hline
       0  & 0 & normal operation                & $ y\leftarrow Lx$      \\\hline
       0  & 1 & normal operation with addition  & $ y\leftarrow y+Lx$    \\\hline
       1  & 0 & adjoint operation               & $ x\leftarrow L^*y$    \\\hline
       1  & 1 & adjoint operation with addition & $ x\leftarrow x +L^*y$ \\
      \hline\hline
   \end{tabular}
%   \caption{Returned values for linear operations.}
   \label{tab:adjadd}
\end{table}

%For example assume again the case of the zero-padding operator, $\mathbf{L}=\begin{bmatrix}\mathbf{I}\\\mathbf{O}\end{bmatrix}$. Its adjoint is $\mathbf{L}^*=\begin{bmatrix}\mathbf{I} & \mathbf{O}^T\end{bmatrix}$. The effect of $\mathbf{L}^*$ on $\mathbf{y}=\mathbf{L}\mathbf{x}$ is  
%\begin{gather*}
%    \tilde{\mathbf{x}} = \mathbf{L}^{*}\mathbf{y} 
%                     = \begin{bmatrix}\mathbf{I} & \mathbf{O}^T\end{bmatrix}\mathbf{y}
%                     = \begin{bmatrix}\mathbf{I} & \mathbf{O}^T\end{bmatrix}
%                       \begin{bmatrix}x_1\\x_2\\\vdots \\x_n\\0\\\vdots\\0\end{bmatrix}
%                     = \begin{bmatrix}x_1\\x_2\\\vdots \\x_n\end{bmatrix} = \mathbf{x}.
%\end{gather*}

















   \section{Adjoint zeroing (adjnull.c)}\label{sec:adjnull}
The null operator is defined by
\begin{gather*}
	y = 0x=0,  \qquad\textrm{with}\quad y_t\leftarrow 0.
\intertext{Its adjoint is}
	x = 0^*y=0,\qquad\textrm{with}\quad x_t\leftarrow 0.
\end{gather*}




\subsection{sf\_adjnull}\label{sec:sf_adjnull}

\subsubsection*{Call}
\begin{verbatim}sf_adjnull(adj, add, nx, ny, x, y);\end{verbatim}

\subsubsection*{Definition}
\begin{verbatim}
void sf_adjnull (bool adj /* adjoint flag */, 
                 bool add /* addition flag */, 
                 int nx   /* size of \texttt{x} */, 
                 int ny   /* size of \texttt{y} */, 
                 float* x, 
                 float* y) 
/*< Zeros out the output (unless add is true). 
    Useful first step for any linear operator. >*/
{
   ...
}
\end{verbatim}

\subsubsection*{Input parameters}
\begin{desclist}{\tt }{\quad}[\tt add]
   \setlength\itemsep{0pt}
   \item[adj] adjoint flag (\texttt{bool}). If \texttt{true}, then the adjoint is computed, i.e.~$x\leftarrow 0^*y$ or $x\leftarrow x+0^*y$. 
   \item[add] addition flag (\texttt{bool}). If \texttt{true}, then $y\leftarrow y+0x$ or $x\leftarrow x+0^*y$.  
   \item[nx]  size of \texttt{x} (\texttt{int}). 
   \item[ny]  size of \texttt{y} (\texttt{int}). 
   \item[x]   input data or output (\texttt{float*}).
   \item[y]   output or input data (\texttt{float*}).
\end{desclist}




\subsection{sf\_cadjnull}
The same as \hyperref[sec:adjnull]{\texttt{sf\_adjnull}} but for complex data.

\subsubsection*{Call}
\begin{verbatim}sf_cadjnull(adj, add, nx, ny, x, y);\end{verbatim}

\subsubsection*{Definition}
\begin{verbatim}
void sf_cadjnull (bool adj /* adjoint flag */, 
                  bool add /* addition flag */, 
                  int nx   /* size of \texttt{x} */, 
                  int ny   /* size of \texttt{y} */, 
                  sf_complex* x, 
                  sf_complex* y) 
/*< adjnull version for complex data. >*/
{
   ...    
}
\end{verbatim}

\subsubsection*{Input parameters}
\begin{desclist}{\tt }{\quad}[\tt add]
   \setlength\itemsep{0pt}
   \item[adj] adjoint flag (\texttt{bool}). If \texttt{true}, then the adjoint is computed, i.e.~$x\leftarrow 0^*y$ or $x\leftarrow x+0^*y$. 
   \item[add] addition flag (\texttt{bool}). If \texttt{true}, then $y\leftarrow y+0x$ or $x\leftarrow x+0^*y$.  
   \item[nx]  size of \texttt{x} (\texttt{int}). 
   \item[ny]  size of \texttt{y} (\texttt{int}). 
   \item[x]   input data or output (\texttt{sf\_complex*}).
   \item[y]   output or input data (\texttt{sf\_complex*}).
\end{desclist}





       % Adjoint zeroing
   \section[Identity operator (copy.c)]{Simple identity (copy) operator (copy.c)}\label{sec:copy}
The identity operator is defined by
\begin{gather*}
	y = 1x=x,  \qquad\textrm{with}\quad y_t\leftarrow x_t.
\intertext{Its adjoint is}
	x = 1^*y=y,\qquad\textrm{with}\quad x_t\leftarrow y_t.
\end{gather*}




\subsection{{sf\_copy\_lop}}\label{sec:sf_copy_lop}

\subsubsection*{Call}
\begin{verbatim}sf_copy_lop (adj, add, nx, ny, x, y);\end{verbatim}

\subsubsection*{Definition}
\begin{verbatim}
void sf_copy_lop (bool adj, bool add, int nx, int ny, float* xx, float* yy)
/*< linear operator >*/
{
   ...
}
\end{verbatim}

\subsubsection*{Input parameters}
\begin{desclist}{\tt }{\quad}[\tt add]
   \setlength\itemsep{0pt}
   \item[adj] adjoint flag (\texttt{bool}). If \texttt{true}, then the adjoint is computed, i.e.~$x\leftarrow 1^*y$ or $x\leftarrow x+1^*y$. 
   \item[add] addition flag (\texttt{bool}). If \texttt{true}, then $y\leftarrow y+1x$ or $x\leftarrow x+1^*y$.  
   \item[nx]  size of \texttt{x} (\texttt{int}). \texttt{nx} must equal \texttt{ny}.
   \item[ny]  size of \texttt{y} (\texttt{int}). \texttt{ny} must equal \texttt{nx}.
   \item[x]   input data or output (\texttt{float*}).
   \item[y]   output or input data (\texttt{float*}).
\end{desclist}





          % Simple identity operator
   \section[Identity operator (complex data) (ccopy.c)]{Simple identity (copy) operator for complex data (ccopy.c)}\label{sec:ccopy}
This is the same operator as \hyperref[sec:sf_copy_lop]{\texttt{sf\_copy\_lop}} but for complex data. In particular, the identity operator is defined by
\begin{gather*}
	y = 1x=x,  \qquad\textrm{with}\quad y_t\leftarrow x_t.
\intertext{Its adjoint is}
	x = 1^*y=y,\qquad\textrm{with}\quad x_t\leftarrow y_t.
\end{gather*}




\subsection{{sf\_ccopy\_lop}}

\subsubsection*{Call}
\begin{verbatim}sf_ccopy_lop (adj, add, nx, ny, x, y);\end{verbatim}

\subsubsection*{Definition}
\begin{verbatim}
void sf_ccopy_lop (bool adj, bool add, int nx, int ny, 
                   sf_complex* xx, sf_complex* yy)
/*< linear operator >*/
{
   ...
}
\end{verbatim}

\subsubsection*{Input parameters}
\begin{desclist}{\tt }{\quad}[\tt add]
   \setlength\itemsep{0pt}
   \item[adj] adjoint flag (\texttt{bool}). If \texttt{true}, then the adjoint is computed, i.e.~$x\leftarrow 1^*y$ or $x\leftarrow x+1^*y$. 
   \item[add] addition flag (\texttt{bool}). If \texttt{true}, then $y\leftarrow y+1x$ or $x\leftarrow x+1^*y$.  
   \item[nx]  size of \texttt{x} (\texttt{int}). \texttt{nx} must equal \texttt{ny}.
   \item[ny]  size of \texttt{y} (\texttt{int}). \texttt{ny} must equal \texttt{nx}.
   \item[x]   input data or output (\texttt{sf\_complex*}).
   \item[y]   output or input data (\texttt{sf\_complex*}).
\end{desclist}


         % Simple identity operator (for complex data)
   \section{Simple mask operator (mask.c)}
This mask operator is defined by
\begin{gather*}
	y = L_mx=mx,  \qquad\textrm{with}\quad y_t\leftarrow m_tx_t,
\intertext{where $m_t$ takes binary values, i.e.~$m_t=0$ or $1$. Its adjoint is}
	x = L_m^*y=my,\qquad\textrm{with}\quad x_t\leftarrow m_ty_t,
\end{gather*}




\subsection{{sf\_mask\_init}}
Initializes the static variable \texttt{m} with boolean values, to be used in the \texttt{sf\_mask\_lop} or \texttt{sf\_cmask\_lop}.

\subsubsection*{Call}
\begin{verbatim}sf_mask_init (m);\end{verbatim}

\subsubsection*{Definition}
\begin{verbatim}
void sf_mask_init(const bool *m)
/*< initialize with mask >*/
{
   ...
}
\end{verbatim}

\subsubsection*{Input parameters}
\begin{desclist}{\tt }{\quad}[\tt ]
   \setlength\itemsep{0pt}
   \item[m] a pointer to boolean values (\texttt{const bool*}). 
\end{desclist}




\subsection{{sf\_mask\_lop}}\label{sec:sf_mask_lop}

\subsubsection*{Call}
\begin{verbatim}sf_mask_lop (adj, add, nx, ny, x, y);\end{verbatim}

\subsubsection*{Definition}
\begin{verbatim}
void sf_mask_lop(bool adj, bool add, int nx, int ny, float *x, float *y)
/*< linear operator >*/
{
   ...    
}
\end{verbatim}

\subsubsection*{Input parameters}
\begin{desclist}{\tt }{\quad}[\tt add]
   \setlength\itemsep{0pt}
   \item[adj] adjoint flag (\texttt{bool}). If \texttt{true}, then the adjoint is computed, i.e.~$x\leftarrow L_m^*y$ or $x\leftarrow x+L_m^*y$. 
   \item[add] addition flag (\texttt{bool}). If \texttt{true}, then $y\leftarrow y+L_mx$ or $x\leftarrow x+L_m^*y$.  
   \item[nx]  size of \texttt{x} (\texttt{int}). \texttt{nx} must equal \texttt{ny}. 
   \item[ny]  size of \texttt{y} (\texttt{int}). \texttt{ny} must equal \texttt{nx}. 
   \item[x]   input data or output (\texttt{float*}).
   \item[y]   output or input data (\texttt{float*}).
\end{desclist}




\subsection{{sf\_cmask\_lop}}
The same as \hyperref[sec:sf_mask_lop]{\texttt{sf\_mask\_lop}} but for complex data.

\subsubsection*{Call}
\begin{verbatim}sf_cmask_lop (adj, add, nx, ny, x, y);\end{verbatim}

\subsubsection*{Definition}
\begin{verbatim}
void sf_cmask_lop(bool adj, bool add, int nx, int ny, 
                  sf_complex *x, sf_complex *y)
/*< linear operator >*/
{
   ...
}
\end{verbatim}

\subsubsection*{Input parameters}
\begin{desclist}{\tt }{\quad}[\tt add]
   \setlength\itemsep{0pt}
   \item[adj] adjoint flag (\texttt{bool}). If \texttt{true}, then the adjoint is computed, i.e.~$x\leftarrow L_m^*y$ or $x\leftarrow x+L_m^*y$. 
   \item[add] addition flag (\texttt{bool}). If \texttt{true}, then $y\leftarrow y+L_mx$ or $x\leftarrow x+L_m^*y$.  
   \item[nx]  size of \texttt{x} (\texttt{int}). \texttt{nx} must equal \texttt{ny}. 
   \item[ny]  size of \texttt{y} (\texttt{int}). \texttt{ny} must equal \texttt{nx}. 
   \item[x]   input data or output (\texttt{sf\_complex*}).
   \item[y]   output or input data (\texttt{sf\_complex*}).
\end{desclist}



          % Simple mask operator
   \section{Simple weight operator (weight.c)}
This weight operator is defined by
\begin{gather*}
	y = L_wx=wx,  \qquad\textrm{with}\quad y_t\leftarrow w_tx_t.
\intertext{Its adjoint is}
	x = L_w^*y=wy,\qquad\textrm{with}\quad x_t\leftarrow w_ty_t.
\end{gather*}
Note that for complex data the weight $w$ must still be real.

There is also an in-place ($x\leftarrow L_wx$) version of the operator, which multiplies the input data with the square of $w$ i.e.
\begin{gather*}
	x = L_wx=w^2x,\qquad\textrm{ with}\quad x_t\leftarrow w_t^2x_t.
\end{gather*}




\subsection{{sf\_weight\_init}}\label{sec:sf_weight_init}
Initializes the weights to be applied as linear operator, by assigning value to a static parameter.

\subsubsection*{Call}
\begin{verbatim}sf_weight_init(w);\end{verbatim}

\subsubsection*{Definition}
\begin{verbatim}
void sf_weight_init(float *w1)
/*< initialize >*/
{
    ..
}
\end{verbatim}


\subsubsection*{Input parameters}
\begin{desclist}{\tt }{\quad}[\tt ]
   \setlength\itemsep{0pt}
   \item[w] values of the weights (\texttt{float*}).
\end{desclist}




\subsection{{sf\_weight\_lop}}\label{sec:sf_weight_lop}
Applies the linear operator with the weights initialized by \texttt{sf\_weight\_init}.

\subsubsection*{Call}
\begin{verbatim}sf_weight_lop (adj, add, nx, ny, x, y);\end{verbatim}

\subsubsection*{Definition}
\begin{verbatim}
void sf_weight_lop (bool adj, bool add, int nx, int ny, float* xx, float* yy)
/*< linear operator >*/
{
   ...
}
\end{verbatim}

\subsubsection*{Input parameters}
\begin{desclist}{\tt }{\quad}[\tt add]
   \setlength\itemsep{0pt}
   \item[adj] adjoint flag (\texttt{bool}). If \texttt{true}, then the adjoint is computed, i.e.~$x\leftarrow L_w^*y$ or $x\leftarrow x+L_w^*y$. 
   \item[add] addition flag (\texttt{bool}). If \texttt{true}, then $y\leftarrow y+L_wx$ or $x\leftarrow x+L_w^*y$.  
   \item[nx]  size of \texttt{x} (\texttt{int}). \texttt{nx} must equal \texttt{ny}. 
   \item[ny]  size of \texttt{y} (\texttt{int}). \texttt{ny} must equal \texttt{nx}. 
   \item[x]   input data or output (\texttt{float*}).
   \item[y]   output or input data (\texttt{float*}).
\end{desclist}




\subsection{{sf\_cweight\_lop}}
The same as \hyperref[sec:sf_weight_lop]{\texttt{sf\_weight\_lop}} but for complex data.

\subsubsection*{Call}
\begin{verbatim}sf_cweight_lop (adj, add, nx, ny, x, y);\end{verbatim}

\subsubsection*{Definition}
\begin{verbatim}
void sf_cweight_lop (bool adj, bool add, int nx, int ny, 
                     sf_complex* xx, sf_complex* yy)
/*< linear operator >*/
{
   ...
}
\end{verbatim}

\subsubsection*{Input parameters}
\begin{desclist}{\tt }{\quad}[\tt add]
   \setlength\itemsep{0pt}
   \item[adj] adjoint flag (\texttt{bool}). If \texttt{true}, then the adjoint is computed, i.e.~$x\leftarrow L_w^*y$ or $x\leftarrow x+L_w^*y$. 
   \item[add] addition flag (\texttt{bool}). If \texttt{true}, then $y\leftarrow y+L_wx$ or $x\leftarrow x+L_w^*y$.  
   \item[nx]  size of \texttt{x} (\texttt{int}). \texttt{nx} must equal \texttt{ny}. 
   \item[ny]  size of \texttt{y} (\texttt{int}). \texttt{ny} must equal \texttt{nx}. 
   \item[x]   input data or output (\texttt{sf\_complex*}).
   \item[y]   output or input data (\texttt{sf\_complex*}).
\end{desclist}




\subsection{{sf\_weight\_apply}}\label{sec:sf_weight_apply}
Creates a product of the weights squared and the input \texttt{x}.

\subsubsection*{Call}
\begin{verbatim}sf_weight_apply (nx, x);\end{verbatim}

\subsubsection*{Definition}
\begin{verbatim}
void sf_weight_apply(int nx, float *xx)
/*< apply weighting in place >*/
{
   ...
}
\end{verbatim}

\subsubsection*{Input parameters}
\begin{desclist}{\tt }{\quad}[\tt nx]
   \setlength\itemsep{0pt}
   \item[nx] size of \texttt{x} (\texttt{int}). 
   \item[x]  input data and output (\texttt{float*}).                   
\end{desclist}




\subsection{{sf\_cweight\_apply}}
The same as the \hyperref[sec:sf_weight_apply]{\texttt{sf\_weight\_apply}} but for the complex numbers.

\subsubsection*{Call}
\begin{verbatim}sf_cweight_apply (nx, x);\end{verbatim}

\subsubsection*{Definition}
\begin{verbatim}
void sf_cweight_apply(int nx, sf_complex *xx)
/*< apply weighting in place >*/
{
   ...
}
\end{verbatim}

\subsubsection*{Input parameters}
\begin{desclist}{\tt }{\quad}[\tt nx]
   \setlength\itemsep{0pt}
   \item[nx] size of \texttt{x} (\texttt{int}). 
   \item[x]  input data and output (\texttt{sf\_complex*}).        
\end{desclist}





        % Simple weight operator
   \section{1-D finite difference (igrad1.c)}
The 1-D finite difference operator is defined by
\begin{gather*}
	y = Dx,  \qquad\textrm{with}\quad y_t\leftarrow x_{t+1}-x_t.
\intertext{Its adjoint is}
	x = D^*y,\qquad\textrm{with}\quad x_t\leftarrow -(y_t - y_{t-1}),\; x_0=-y_0.
\end{gather*}




\subsection{{sf\_igrad1\_lop}}

\subsubsection*{Call}
\begin{verbatim}sf_igrad1_lop(adj, add, nx, ny, x, y);\end{verbatim}

\subsubsection*{Definition}
\begin{verbatim}
void  sf_igrad1_lop(bool adj, bool add,
                    int nx, int ny, float *xx, float *yy)
/*< linear operator >*/
{
   ...
}
\end{verbatim}

\subsubsection*{Input parameters}
\begin{desclist}{\tt }{\quad}[\tt add]
   \setlength\itemsep{0pt}
   \item[adj] adjoint flag (\texttt{bool}). If \texttt{true}, then the adjoint is computed, i.e.~$x\leftarrow D^*y$ or $x\leftarrow x+D^*y$. 
   \item[add] addition flag (\texttt{bool}). If \texttt{true}, then $y\leftarrow y+Dx$ or $x\leftarrow x+D^*y$.  
   \item[nx]  size of \texttt{x} (\texttt{int}). 
   \item[ny]  size of \texttt{y} (\texttt{int}). 
   \item[x]   input data or output (\texttt{float*}).
   \item[y]   output or input data (\texttt{float*}).
\end{desclist}


        % 1-D finite difference
   \section{Causal integration (causint.c)}
This causal integration operator is defined by
\begin{gather*}
	y = Lx,  \qquad\textrm{with}\quad y_t \leftarrow \sum_{\tau=0}^tx_\tau.
\intertext{Its adjoint is}
	x = L^*y,\qquad\textrm{with}\quad x_t \leftarrow \sum_{\tau=t}^{T-1}y_\tau,
\end{gather*}
where $T$ is the total number of samples of $x$.




\subsection{{sf\_causint\_lop}}

\subsubsection*{Call}
\begin{verbatim}sf_causint_lop (adj, add, nx, ny, x, y);\end{verbatim}

\subsubsection*{Defintion}
\begin{verbatim}
sf_causint_lop (adj, add, nx, ny, x, y)
/*< linear operator >*/
{
   ...
}
\end{verbatim}

\subsubsection*{Input parameters}
\begin{desclist}{\tt }{\quad}[\tt add]
   \setlength\itemsep{0pt}
   \item[adj] adjoint flag (\texttt{bool}). If \texttt{true}, then the adjoint is computed, i.e.~$x\leftarrow L^*y$ or $x\leftarrow x+L^*y$. 
   \item[add] addition flag (\texttt{bool}). If \texttt{true}, then $y\leftarrow y+Lx$ or $x\leftarrow x+L^*y$.  
   \item[nx]  size of \texttt{x} (\texttt{int}). 
   \item[ny]  size of \texttt{y} (\texttt{int}). 
   \item[x]   input data or output (\texttt{float*}).
   \item[y]   output or input data (\texttt{float*}).
\end{desclist}


       % Causal integration
   \section{Chaining linear operators (chain.c)}\label{sec:chain}
Calculates products of operators.



\subsection{{sf\_chain}}\label{sec:sf_chain}
Chains two operators $L_1$ and $L_2$: 
\begin{gather*}
	d = (L_2L_1)m.
\intertext{Its adjoint is}
	m = (L_2L_1)^*d=L_1^*L_2^*d.
\end{gather*}

\subsubsection*{Call}
\begin{verbatim}sf_chain (oper1,oper2, adj,add, nm,nd,nt, mod,dat,tmp);\end{verbatim}

\subsubsection*{Definition}
\begin{verbatim}
void sf_chain( sf_operator oper1     /* outer operator */, 
               sf_operator oper2     /* inner operator */, 
               bool adj              /* adjoint flag */, 
               bool add              /* addition flag */, 
               int nm                /* model size */, 
               int nd                /* data size */, 
               int nt                /* intermediate size */, 
               /*@out@*/ float* mod  /* [nm] model */, 
               /*@out@*/ float* dat  /* [nd] data */, 
               float* tmp            /* [nt] intermediate */) 
/*< Chains two operators, computing oper1{oper2{mod}} 
  or its adjoint. The tmp array is used for temporary storage. >*/
{
   ...
}
\end{verbatim}

\subsubsection*{Input parameters}
\begin{desclist}{\tt }{\quad}[\tt oper2]
   \setlength\itemsep{0pt}
   \item[oper1] outer operator, $L_1$ (\texttt{sf\_operator}). 
   \item[oper2] inner operator, $L_2$ (\texttt{sf\_operator}). 
   \item[adj] adjoint flag (\texttt{bool}). If \texttt{true}, then the adjoint is computed, i.e.~$m\leftarrow (L_2L_1)^*d$ or $m\leftarrow m+(L_2L_1)^*d$. 
   \item[add] addition flag (\texttt{bool}). If \texttt{true}, then $d\leftarrow d+(L_2L_1)m$ or $m\leftarrow m+(L_2L_1)^*d$ is computed.  
   \item[nm]    size of the model \texttt{mod} (\texttt{int}). 
   \item[nd]    size of the data \texttt{dat} (\texttt{int}). 
   \item[nt]    size of the intermediate result \texttt{tmp}  (\texttt{int}). 
   \item[mod]   the model, $m$ (\texttt{float*}).
   \item[dat]   the data, $d$ (\texttt{float*}).
   \item[tmp]   intermediate result (\texttt{float*}).
\end{desclist}




\subsection{{sf\_cchain}}
The same as \hyperref[sec:sf_chain]{\texttt{sf\_chain}} but for complex data.

\subsubsection*{Call}
\begin{verbatim}sf_cchain (oper1,oper2, adj,add, nm,nd,nt, mod, dat, tmp);\end{verbatim}

\subsubsection*{Definition}
\begin{verbatim}
void sf_cchain( sf_coperator oper1         /* outer operator */, 
                sf_coperator oper2         /* inner operator */, 
                bool adj                   /* adjoint flag */, 
                bool add                   /* addition flag */, 
                int nm                     /* model size */, 
                int nd                     /* data size */, 
                int nt                     /* intermediate size */, 
                /*@out@*/ sf_complex* mod  /* [nm] model */, 
                /*@out@*/ sf_complex* dat  /* [nd] data */, 
                sf_complex* tmp            /* [nt] intermediate */) 
/*< Chains two complex operators, computing oper1{oper2{mod}} 
    or its adjoint. The tmp array is used for temporary storage. >*/
{
   ...    
}
\end{verbatim}

\subsubsection*{Input parameters}
\begin{desclist}{\tt }{\quad}[\tt oper2]
   \setlength\itemsep{0pt}
   \item[oper1] outer operator, $L_1$ (\texttt{sf\_coperator}). 
   \item[oper2] inner operator, $L_2$ (\texttt{sf\_coperator}). 
   \item[adj] adjoint flag (\texttt{bool}). If \texttt{true}, then the adjoint is computed, i.e.~$m\leftarrow (L_2L_1)^*d$ or $m\leftarrow m+(L_2L_1)^*d$. 
   \item[add] addition flag (\texttt{bool}). If \texttt{true}, then $d\leftarrow d+(L_2L_1)m$ or $m\leftarrow m+(L_2L_1)^*d$ is computed.  
   \item[nm]    size of the model \texttt{mod} (\texttt{int}). 
   \item[nd]    size of the data \texttt{dat} (\texttt{int}). 
   \item[nt]    size of the intermediate result \texttt{tmp}  (\texttt{int}). 
   \item[mod]   the model, $m$ (\texttt{sf\_complex*}).
   \item[dat]   the data, $d$ (\texttt{sf\_complex*}).
   \item[tmp]   intermediate result (\texttt{sf\_complex*}).
   \item[tmp]   the intermediate storage (\texttt{sf\_complex*}).
\end{desclist}




\subsection{{sf\_array}}
For two operators $L_1$ and $L_2$, it calculates: 
\begin{gather*}
	d = Lm,
\intertext{or its adjoint}
	m = L^*d,
\intertext{where}
	L = \begin{bmatrix}L_1\\L_2\end{bmatrix}\quad\textrm{and}\quad 
	d = \begin{bmatrix}d_1\\d_2\end{bmatrix}
\end{gather*}

\subsubsection*{Call}
\begin{verbatim}sf_array (oper1,oper2, adj,add, nm,nd1,nd2, mod,dat1,dat2);\end{verbatim}

\subsubsection*{Definition}
\begin{verbatim}
void sf_array( sf_operator oper1     /* top operator */, 
               sf_operator oper2     /* bottom operator */, 
               bool adj              /* adjoint flag */, 
               bool add              /* addition flag */, 
               int nm                /* model size */, 
               int nd1               /* top data size */, 
               int nd2               /* bottom data size */, 
               /*@out@*/ float* mod  /* [nm] model */, 
               /*@out@*/ float* dat1 /* [nd1] top data */, 
               /*@out@*/ float* dat2 /* [nd2] bottom data */) 
/*< Constructs an array of two operators, 
    computing {oper1{mod},oper2{mod}} or its adjoint. >*/
{
   ...
}
\end{verbatim}

\subsubsection*{Input parameters}
\begin{desclist}{\tt }{\quad}[\tt oper2]
   \setlength\itemsep{0pt}
   \item[oper1] top operator, $L_1$ (\texttt{sf\_operator}). 
   \item[oper2] bottom operator, $L_2$ (\texttt{sf\_operator}). 
   \item[adj]   adjoint flag (\texttt{bool}). If \texttt{true}, then the adjoint is computed, i.e.~$m\leftarrow L^*d$ or $m\leftarrow m + L^*d$. 
   \item[add] addition flag (\texttt{bool}). If \texttt{true}, then $d\leftarrow d + Lm$ or $m\leftarrow m + L^*d$ is computed.  
   \item[nm]    size of the model, \texttt{mod} (\texttt{int}). 
   \item[nd1]   size of the top data, \texttt{dat1} \texttt{dat1} (\texttt{int}). 
   \item[nd2]   size of the bottom data, \texttt{dat2} (\texttt{int}). 
   \item[mod]   the model, $m$ (\texttt{float*}).
   \item[dat1]  the top data, $d_1$ (\texttt{float*}).
   \item[dat2]  the bottom data, $d_2$ (\texttt{float*}).
\end{desclist}




\subsection{{sf\_normal}}
Applies a normal operator (self-adjoint) to the model, i.e.~it calculates
\begin{gather*}
	d = LL^*m.
\end{gather*}


\subsubsection*{Call}
\begin{verbatim}sf_normal (oper, add, nm,nd, mod,dat,tmp);\end{verbatim}

\subsubsection*{Definition}
\begin{verbatim}
void sf_normal (sf_operator oper /* operator */, 
                bool add         /* addition flag */, 
                int nm           /* model size */, 
                int nd           /* data size */, 
                float *mod       /* [nd] model */, 
                float *dat       /* [nd] data */, 
                float *tmp       /* [nm] intermediate */)
/*< Applies a normal operator (self-adjoint) >*/
{
   ...
}
\end{verbatim}

\subsubsection*{Input parameters}
\begin{desclist}{\tt }{\quad}[\tt oper]
   \setlength\itemsep{0pt}
   \item[oper]  the operator, $L$ (\texttt{sf\_operator}). 
   \item[add]   addition flag (\texttt{bool}). If \texttt{true}, then $d\leftarrow d+LL^*m$.  
   \item[nm]    size of the model, \texttt{mod} (\texttt{int}). 
   \item[nd]    size of the data, \texttt{dat} (\texttt{int}). 
   \item[mod]   the model, $m$ (\texttt{float*}).
   \item[dat]   the data, $d$ (\texttt{float*}).
   \item[tmp]   the intermediate result (\texttt{float*}).
\end{desclist}




\subsection{{sf\_chain3}}
Chains three operators $L_1$, $L_2$ and $L_3$: 
\begin{gather*}
	d = (L_3L_2L_1)m.
\intertext{Its adjoint is}
	m = (L_3L_2L_1)^*d=L_1^*L_2^*L_3^*d.
\end{gather*}

\subsubsection*{Call}
\begin{verbatim}
sf_chain3 (oper1,oper2,oper3, adj,add, nm,nt1,nt2,nd, mod,dat,tmp1,tmp2);
\end{verbatim}

\subsubsection*{Definition}
\begin{verbatim}
void sf_chain3 (sf_operator oper1 /* outer operator */, 
                sf_operator oper2 /* middle operator */, 
                sf_operator oper3 /* inner operator */, 
                bool adj          /* adjoint flag */, 
                bool add          /* addition flag */, 
                int nm            /* model size */, 
                int nt1           /* inner intermediate size */, 
                int nt2           /* outer intermediate size */, 
                int nd            /* data size */, 
                float* mod        /* [nm] model */, 
                float* dat        /* [nd] data */, 
                float* tmp1       /* [nt1] inner intermediate */, 
                float* tmp2       /* [nt2] outer intermediate */)
/*< Chains three operators, computing oper1{oper2{poer3{{mod}}} or its adjoint.
  The tmp1 and tmp2 arrays are used for temporary storage. >*/
{
   ...    
}
\end{verbatim}

\subsubsection*{Input parameters}
\begin{desclist}{\tt }{\quad}[\tt oper3]
   \setlength\itemsep{0pt}
   \item[oper1] outer operator (\texttt{sf\_operator}). 
   \item[oper2] middle operator (\texttt{sf\_operator}). 
   \item[oper3] inner operator (\texttt{sf\_operator}). 
   \item[adj]   adjoint flag (\texttt{bool}). If \texttt{true}, then the adjoint is computed, i.e.~$m\leftarrow L_1^*L_2^*L_3^*d$ or $m\leftarrow m+L_1^*L_2^*L_3^*d$. 
   \item[add]   addition flag (\texttt{bool}). If \texttt{true}, then $d\leftarrow d+L_3L_2L_1m$ or $m\leftarrow m+L_1^*L_2^*L_3^*d$.  
   \item[nm]    size of the model, \texttt{mod} (\texttt{int}). 
   \item[nt1]   inner intermediate size (\texttt{int}). 
   \item[nt2]   outer intermediate size (\texttt{int}). 
   \item[ny]    size of the data, \texttt{dat} (\texttt{int}). 
   \item[mod]   the model, $x$ (\texttt{float*}).
   \item[dat]   the data, $d$ (\texttt{float*}).
   \item[tmp1]  the inner intermediate result (\texttt{float*}).
   \item[tmp2]  the outer intermediate result (\texttt{float*}).
\end{desclist}






         % Chaining linear operators
   \section{Dot product test for linear operators (dottest.c)}
Performs the dot product test (see p.~\pageref{sec:dot-product test}), to check whether the adjoint of the operator is coded incorrectly. Coding is incorrect if any of 
\begin{gather*}
   \langle Lm_1,d_2\rangle = \langle m_1, L^*d_2\rangle\quad\textrm{or}\quad
   \langle d_1 + Lm_1,d_2\rangle = \langle m_1, m_2+L^*d_2\rangle
\end{gather*} 
does not hold (within machine precision). $m_1$ and $d_2$ are random vectors.




\subsection{{sf\_dot\_test}}\label{sec:sf_dot_test}
\texttt{dot1[0]} must equal \texttt{dot1[1]} and \texttt{dot2[0]} must equal \texttt{dot2[1]} for the test to pass.

\subsubsection*{Call}
\begin{verbatim}sf_dot_test (oper, nm, nd, dot1, dot2);\end{verbatim}

\subsubsection*{Definition}
\begin{verbatim}
void sf_dot_test(sf_operator oper /* linear operator */, 
                 int nm           /* model size */, 
                 int nd           /* data size */, 
                 float* dot1      /* first output */, 
                 float* dot2      /* second output */) 
{
   ...
}
\end{verbatim}

\subsubsection*{Input parameters}
\begin{desclist}{\tt }{\quad}[\tt oper]
   \setlength\itemsep{0pt}
   \item[oper] the linear operator, whose adjoint is to be tested (\texttt{sf\_operator}). 
   \item[nm]   size of the models (\texttt{int}). 
   \item[nd]   size of the data (\texttt{int}). 
   \item[dot1] first output dot product (\texttt{float*}).
   \item[dot2] second output dot product (\texttt{float*}).
\end{desclist}



       % Dot product test for linear operators


\chapter{Data analysis}\label{sec:analysis}
   \section{FFT (kiss\_fftr.c)}

\subsection{{sf\_kiss\_fftr\_alloc}}
Allocates the memory for the FFT and returns an object of type \texttt{kiss\_fftr\_cfg}.

\subsubsection*{Call}
\begin{verbatim}kiss_fftr_alloc(nfft, inverse_fft, mem, lenmem);\end{verbatim}

\subsubsection*{Definition}
\begin{verbatim}
kiss_fftr_cfg kiss_fftr_alloc(int nfft,int inverse_fft,void * mem,size_t * lenmem)
{
   ...
}
\end{verbatim}

\subsubsection*{Input parameters}
\begin{desclist}{\tt}{\quad}[inversewfft]
   \setlength\itemsep{0pt}
   \item[nfft]         length of the forward FFT (\texttt{int}).  
   \item[inverse\_fft] length of the inverse FFT (\texttt{int}).  
   \item[mem]          pointer to the memory allocated for FFT (\texttt{void*}).
   \item[lenmem]       size of the allocated memory (\texttt{size\_t}).
\end{desclist}

\subsubsection*{Input parameters}
\begin{desclist}{\tt}{\quad}[]
   \item[st] an object for FFT (\texttt{kiss\_fftr\_cfg}).
\end{desclist}
 



\subsection{{kiss\_fftr}}
Performs the forward FFT on the input \texttt{timedata} which is real, and stores the transformed complex \texttt{freqdata} in the location specified in the input.

\subsubsection*{Call}
\begin{verbatim}kiss_fftr( st, timedata, freqdata)\end{verbatim}

\subsubsection*{Definition}
\begin{verbatim}
void kiss_fftr(kiss_fftr_cfg st,const kiss_fft_scalar *timedata,kiss_fft_cpx *fr
eqdata)
{
    /* input buffer timedata is stored row-wise */

    /* The real part of the DC element of the frequency spectrum in st->tmpbuf
     * contains the sum of the even-numbered elements of the input time sequence
     * The imag part is the sum of the odd-numbered elements
     *
     * The sum of tdc.r and tdc.i is the sum of the input time sequence. 
     *      yielding DC of input time sequence
     * The difference of tdc.r - tdc.i is the sum of the input (dot product) [1,-1,1,-1... 
     *      yielding Nyquist bin of input time sequence
     */
 
   ...
}
\end{verbatim}

\subsubsection*{Input parameters}
\begin{desclist}{\tt}{\quad}[freqdata]
   \setlength\itemsep{0pt}
   \item[st]       an object for the forward FFT (\texttt{kiss\_fftr\_cfg}).  
   \item[timedata] time data which is to be transformed (\texttt{const kiss\_fft\_scalar*}).  
   \item[freqdata] location where the transformed data is to be stored (\texttt{kiss\_fft\_cpx*}).
\end{desclist}



\subsection{{kiss\_fftri}}
Performs the inverse FFT on the input \texttt{timedata} which is real, and stores the transformed complex \texttt{freqdata} in the location specified in the input.

\subsubsection*{Call}
\begin{verbatim}kiss_fftri(st, freqdata, timedata);\end{verbatim}

\subsubsection*{Definition}
\begin{verbatim}
void kiss_fftri(kiss_fftr_cfg st,const kiss_fft_cpx *freqdata,kiss_fft_scalar *t
imedata)
/* input buffer timedata is stored row-wise */
{
   ...
}
\end{verbatim}

\subsubsection*{Input parameters}
\begin{desclist}{\tt}{\quad}[freqdata]
   \setlength\itemsep{0pt}
   \item[st]       an object for the inverse FFT (\texttt{kiss\_fftr\_cfg}).  
   \item[timedata] location where the inverse time data is to be stored (\texttt{const kiss\_fft\_scalar*}).  
   \item[freqdata] complex frequency data which is to be inverse transformed (\texttt{kiss\_fft\_cpx*}).
\end{desclist}

     % FFT
   \section{Cosine window weighting function (tent2.c)}




\subsection{{sf\_tent2}}
Sets the weights for the windows defined for each dimension.

\subsubsection*{Call}
\begin{verbatim}sf_tent2 (dim, nwind, windwt);\end{verbatim}

\subsubsection*{Definition}
\begin{verbatim}
void sf_tent2 (int dim          /* number of dimensions */, 
               const int* nwind /* window size [dim] */, 
               float* windwt    /* window weight */)
/*< define window weight >*/
{
   ...
}
\end{verbatim}

\subsubsection*{Input parameters}
\begin{desclist}{\tt }{\quad}[\tt windwt]
   \setlength\itemsep{0pt}
   \item[dim]    number of dimensions (\texttt{int}). 
   \item[nwind]  window size [dim] (\texttt{const int}). 
   \item[windwt] window weight (\texttt{const int}).
\end{desclist}



         % Cosine window weighting function
   \section{Anisotropic diffusion, 2-D (impl2.c)}




\subsection{{sf\_impl2\_init}}\label{sec:sf_impl2_init}
Initializes the required variables and allocates the required space for the anisotropic diffusion.

\subsubsection*{Call}
\begin{verbatim}
void sf_impl2_init (r1,r2, n1_in,n2_in, tau, pclip, 
                    up_in, verb_in, dist_in, nsnap_in, snap_in);
\end{verbatim}

\subsubsection*{Definition}
\begin{verbatim}
void sf_impl2_init (float r1, float r2   /* radius */, 
                 int n1_in, int n2_in /* data size */, 
                 float tau            /* duration */, 
                 float pclip          /* percentage clip */, 
                 bool up_in           /* weighting case */,
                 bool verb_in         /* verbosity flag */,
                 float *dist_in       /* optional distance function */,
                 int nsnap_in         /* number of snapshots */,
                 sf_file snap_in      /* snapshot file */)
/*< initialize >*/
{
   ...
}
\end{verbatim}

\subsubsection*{Input parameters}
\begin{desclist}{\tt }{\quad}[\tt ]
   \setlength\itemsep{0pt}
   \item[r1]        radius in the first dimension (\texttt{float}).  
   \item[r2]        radius in the second dimension (\texttt{float}).  
   \item[n1\_in]    length of first dimension in the input data (\texttt{int}).  
   \item[n2\_in]    length of second dimension in the input data (\texttt{int}).  
   \item[tau]       duration (\texttt{float}).  
   \item[pclip]     percentage clip (\texttt{float}).  
   \item[up\_in]    weighting case (\texttt{bool}).  
   \item[verb\_in]  verbosity flag (\texttt{bool}).  
   \item[dist\_in]  optional distance function (\texttt{float*}).  
   \item[nsnap\_in] number of snapshots (\texttt{int}).  
   \item[snap\_in]  snapshot file (\texttt{sf\_file}).  
\end{desclist}




\subsection{{sf\_impl2\_close}}
Frees the space allocated by \hyperref[sec:sf_impl2_init]{\texttt{sf\_impl2\_init}}.

\subsubsection*{Definition}
\begin{verbatim}sf_impl2_close ();\end{verbatim}

\subsubsection*{Definition}
\begin{verbatim}
void sf_impl2_close (void)
/*< free allocated storage >*/
{
   ...
}
\end{verbatim}




\subsection{{sf\_impl2\_set}}
Computes the weighting function for the anisotropic diffusion.

\subsubsection*{Call}
\begin{verbatim}sf_impl2_set(x);\end{verbatim}

\subsubsection*{Definition}
\begin{verbatim}
void sf_impl2_set(float ** x)
/*< compute weighting function >*/
{
   ...
}
\end{verbatim}

\subsubsection*{Input parameters}
\begin{desclist}{\tt }{\quad}[\tt ]
   \setlength\itemsep{0pt}
   \item[x] data (\texttt{float**}).  
\end{desclist}




\subsection{{sf\_impl2\_set}}
Applies the anisotropic diffusion.

\subsubsection*{Call}
\begin{verbatim}sf_impl2_apply (x, set, adj);\end{verbatim}

\subsubsection*{Definition}
\begin{verbatim}
void sf_impl2_apply (float **x, bool set, bool adj)
/*< apply diffusion >*/
{
   ...  
}
\end{verbatim}

\subsubsection*{Input parameters}
\begin{desclist}{\tt }{\quad}[\tt set]
   \setlength\itemsep{0pt}
   \item[x]   data (\texttt{float*}).  
   \item[set] whether the weighting function needs to be computed (\texttt{bool}).  
   \item[adj] whether the weighting function needs to be applied (\texttt{bool}).  
\end{desclist}





\subsection{{sf\_impl2\_lop}}
Applies either \texttt{x} or \texttt{y} as linear operator to \texttt{y} or \texttt{x} and output \texttt{x} or \texttt{y}, depending on whether \texttt{adj} is true or false.

\subsubsection*{Call}
\begin{verbatim}sf_impl2_lop (adj, add, nx, ny, x, y);\end{verbatim}

\subsubsection*{Definition}
\begin{verbatim}
void sf_impl2_lop (bool adj, bool add, int nx, int ny, float* x, float* y)
/*< linear operator >*/
{
   ...
}
\end{verbatim}

\subsubsection*{Input parameters}
\begin{desclist}{\tt }{\quad}[\tt add]
   \setlength\itemsep{0pt}
   \item[adj] a parameter to determine whether the output is \texttt{y} or \texttt{x} (\texttt{bool}).
   \item[add] a parameter to determine whether the input needs to be zeroed (\texttt{bool}).
   \item[nx]  size of \texttt{x} (\texttt{int}).
   \item[ny]  size of \texttt{y} (\texttt{int}).
   \item[x]   data or operator, depending on whether \texttt{adj} is true or false (\texttt{sf\_complex*}).
   \item[y]   data or operator, depending on whether \texttt{adj} is true or false (\texttt{sf\_complex*}).
\end{desclist}



         % Anisotropic diffusion, 2-D


\chapter{Filtering}\label{sec:filter}
   \section{Frequency-domain filtering (freqfilt.c)}
\index{filtering!frequency domain}




\subsection{{sf\_freqfilt\_init}}\label{sec:sf_freqfilt_init}
Initializes the required variables and allocates the required space for frequency filtering.

\subsubsection*{Call}
\begin{verbatim}sf_freqfilt_init(nfft1, nw1);\end{verbatim}

\subsubsection*{Definition}
\begin{verbatim}
void sf_freqfilt_init(int nfft1 /* time samples (possibly padded) */, 
                      int nw1   /* frequency samples */)
/*< Initialize >*/
{
   ...
}
\end{verbatim}

\subsubsection*{Input parameters}
\begin{desclist}{\tt }{\quad}[\tt nfft1]
   \setlength\itemsep{0pt}
   \item[nfft1] number of time samples (\texttt{int}).  
   \item[nw1]   number of frequency samples (\texttt{int}).  
\end{desclist}




\subsection{{sf\_freqfilt\_set}}
Initializes a zero phase filter.

\subsubsection*{Call}
\begin{verbatim}sf_freqfilt_set(filt);\end{verbatim}

\subsubsection*{Definition}
\begin{verbatim}
void sf_freqfilt_set(float *filt /* frequency filter [nw] */)
/*< Initialize filter (zero-phase) >*/
{
   ...
}
\end{verbatim}

\subsubsection*{Input parameters}
\begin{desclist}{\tt }{\quad}[\tt ]
   \setlength\itemsep{0pt}
   \item[filt] the frequency filter (\texttt{float*}).  
\end{desclist}




\subsection{{sf\_freqfilt\_cset}}
Initializes a non-zero phase filter (filter with complex values).

\subsubsection*{Call}
\begin{verbatim}sf_freqfilt_cset (filt);\end{verbatim}

\subsubsection*{Definition}
\begin{verbatim}
void sf_freqfilt_cset(kiss_fft_cpx *filt /* frequency filter [nw] */)
/*< Initialize filter >*/
{
   ...
}
\end{verbatim}

\subsubsection*{Input parameters}
\begin{desclist}{\tt }{\quad}[\tt ]
   \setlength\itemsep{0pt}
   \item[filt] the frequency filter. Must be of type \texttt{kiss\_fft\_cpx*}.  
\end{desclist}




\subsection{{sf\_freqfilt\_close}}
Frees the space allocated by \hyperref[sec:sf_freqfilt2_init]{\texttt{sf\_freqfilt\_init}}.

\subsubsection*{Call}
\begin{verbatim}sf_freqfilt_close();\end{verbatim}

\subsubsection*{Definition}
\begin{verbatim}
void sf_freqfilt_close(void) 
/*< Free allocated storage >*/
{
   ...
}
\end{verbatim}




\subsection{{sf\_freqfilt}}
Applies the frequency filtering to the input data.

\subsubsection*{Call}
\begin{verbatim}sf_freqfilt(nx, x);\end{verbatim}

\subsubsection*{Definition}
\begin{verbatim}
void sf_freqfilt(int nx, float* x)
/*< Filtering in place >*/
{
   ...
}
\end{verbatim}

\subsubsection*{Input parameters}
\begin{desclist}{\tt }{\quad}[\tt xx]
   \setlength\itemsep{0pt}
   \item[nx] data length (\texttt{int}).  
   \item[x]  the data (\texttt{float*}).  
\end{desclist}




\subsection{{sf\_freqfilt\_lop}}
Applies the frequency filtering to either \texttt{y} or \texttt{x}, depending on whether \texttt{adj} is true of false and then applies the result to \texttt{x} or \texttt{y} as a linear operator.

\subsubsection*{Call}
\begin{verbatim}sf_freqfilt_lop (adj, add, nx, ny, x, y);\end{verbatim}

\subsubsection*{Definition}
\begin{verbatim}
void sf_freqfilt_lop (bool adj, bool add, int nx, int ny, float* x, float* y) 
/*< Filtering as linear operator >*/
{
   ...
}
\end{verbatim}

\subsubsection*{Input parameters}
\begin{desclist}{\tt }{\quad}[\tt add]
   \setlength\itemsep{0pt}
   \item[adj] a parameter to determine whether frequency filter applied to \texttt{y} or \texttt{x}.  Must be of type \texttt{bool}.
   \item[add]  a parameter to determine whether the input needs to be zeroed (\texttt{bool}).
   \item[nx]   size of \texttt{x} (\texttt{int}).
   \item[ny]   size of \texttt{y} (\texttt{int}).
   \item[x]    data or operator, depending on whether \texttt{adj} is true or false (\texttt{float*}).
   \item[y]    data or operator, depending on whether \texttt{adj} is true or false (\texttt{float*}).
\end{desclist}

      % Frequency-domain filtering
   \section{Frequency-domain filtering in 2-D (freqfilt.c)}
\index{filtering!frequency domain}




\subsection{{sf\_freqfilt2\_init}}\label{sec:sf_freqfilt2_init}
Initializes the required variables and allocates the required space for frequency filtering for 2D data.


\subsubsection*{Call}
\begin{verbatim}sf_freqfilt2_init (n1, n2, nw1);\end{verbatim}

\subsubsection*{Definition}
\begin{verbatim}
void sf_freqfilt2_init(int n1, int n2 /* data dimensions */, 
                       int nw1        /* number of frequencies */)
/*< initialize >*/
{
   ...
}
\end{verbatim}

\subsubsection*{Input parameters}
\begin{desclist}{\tt }{\quad}[\tt nw1]
   \setlength\itemsep{0pt}
   \item[n1]  number of time samples in first dimension (\texttt{int}).  
   \item[n2]  number of time samples in second dimension (\texttt{int}).  
   \item[nw1] number of frequencies (\texttt{int}).  
\end{desclist}




\subsection{{sf\_freqfilt2\_set}}
Initializes a zero phase filter.


\subsubsection*{Call}
\begin{verbatim}sf_freqfilt2_set (filt);\end{verbatim}

\subsubsection*{Input parameters}
\begin{desclist}{\tt }{\quad}[\tt ]
   \setlength\itemsep{0pt}
   \item[filt] the frequency filter (\texttt{float**}).  
\end{desclist}

\subsubsection*{Definition}
\begin{verbatim}
void sf_freqfilt2_set(float **filt)
/*< set the filter >*/
{
   ...
}
\end{verbatim}




\subsection{{sf\_freqfilt2\_close}}
Frees the space allocated by \hyperref[sec:sf_freqfilt2_init]{\texttt{sf\_freqfilt2\_init}}.

\subsubsection*{Call}
\begin{verbatim}sf_freqfilt2_close();\end{verbatim}

\subsubsection*{Definition}
\begin{verbatim}
void sf_freqfilt2_close(void) 
/*< free allocated storage >*/
{
   ...
}
\end{verbatim}




\subsection{{sf\_freqfilt2\_spec}}
This function 2D spectrum of the input data.

\subsubsection*{Call}
\begin{verbatim}sf_freqfilt2_spec (x, y);\end{verbatim}

\subsubsection*{Definition}
\begin{verbatim}
void sf_freqfilt2_spec (const float* x /* input */, float** y /* spectrum */) 
/*< compute 2-D spectrum >*/
{
   ...
}
\end{verbatim}

\subsubsection*{Input parameters}
\begin{desclist}{\tt }{\quad}[\tt ]
   \setlength\itemsep{0pt}
   \item[x] the data (\texttt{const float*}).  
   \item[y] the data (\texttt{float**}).  
\end{desclist}




\subsection{{sf\_freqfilt2\_lop}}
Applies the frequency filtering to either \texttt{y} or \texttt{x}, depending on whether \texttt{adj} is true of false and then applies the result to \texttt{x} or \texttt{y} as a linear operator.


\subsubsection*{Call}
\begin{verbatim}sf_freqfilt2_lop (adj, add, nx, ny, x, y);\end{verbatim}

\subsubsection*{Definition}
\begin{verbatim}
void sf_freqfilt2_lop (bool adj, bool add, int nx, int ny, float* x, float* y) 
/*< linear filtering operator >*/
{
   ...
}
\end{verbatim}

\subsubsection*{Input parameters}
\begin{desclist}{\tt }{\quad}[\tt add]
   \setlength\itemsep{0pt}
   \item[adj] a parameter to determine whether frequency filter applied to \texttt{y} or \texttt{x} (\texttt{bool}).
   \item[add] a parameter to determine whether the input needs to be zeroed (\texttt{bool}).
   \item[nx]  size of \texttt{x} (\texttt{int}).
   \item[ny]  size of \texttt{y} (\texttt{int}).
   \item[x]   data or operator, depending on whether \texttt{adj} is true or false (\texttt{float*}).
   \item[y]   data or operator, depending on whether \texttt{adj} is true or false (\texttt{float*}).
\end{desclist}


     % Frequency-domain filtering in 2-D
   \section{Helical convolution (helicon.c)}




\subsection{{sf\_helicon\_init}}
Initializes an object of type \texttt{sf\_filter} to be used in the linear operator function.


\subsubsection*{Call}
\begin{verbatim}sf_helicon_init(bb);\end{verbatim}

\subsubsection*{Definition}
\begin{verbatim}
void sf_helicon_init (sf_filter bb) 
/*<  Initialized with the filter. >*/
{
   ...
}

\end{verbatim}
\subsubsection*{Input parameters}
\begin{desclist}{\tt }{\quad}[\tt ]
   \setlength\itemsep{0pt}
   \item[bb] the filter object (\texttt{sf\_filter}).  
\end{desclist}




\subsection{{sf\_helicon\_lop}}
Does the helical convolution. It applies the filter to either \texttt{yy} or \texttt{xx}, depending on whether \texttt{adj} is true of false and then applies the result to \texttt{xx} or \texttt{yy} as a linear operator.

\subsubsection*{Call}
\begin{verbatim}sf_helicon_lop (adj, add, nx, ny, xx, yy);\end{verbatim}

\subsubsection*{Definition}
\begin{verbatim}
void sf_helicon_lop( bool adj, bool add, 
                     int nx, int ny, float* xx, float*yy) 
/*< linear operator >*/
{
   ...
}
\end{verbatim}

\subsubsection*{Input parameters}
\begin{desclist}{\tt }{\quad}[\tt add]
   \setlength\itemsep{0pt}
   \item[adj] a parameter to determine whether the filter is applied to \texttt{yy} or \texttt{xx} (\texttt{bool}).
   \item[add] a parameter to determine whether the input needs to be zeroed (\texttt{bool}).
   \item[nx]  size of \texttt{x}x (\texttt{int}).
   \item[ny]  size of \texttt{y}y (\texttt{int}).
   \item[xx]  data or operator, depending on whether \texttt{adj} is true or false (\texttt{float*}).
   \item[yy]  data or operator, depending on whether \texttt{adj} is true or false (\texttt{float*}).
\end{desclist}

       % Helical convolution

   \section{Helical filter definition and allocation (helix.c)}




\subsection{{sf\_allocatehelix}}\label{sec:sf_allocatehelix}
Initializes the filter.

\subsubsection*{Call}
\begin{verbatim}aa = sf_allocatehelix(nh);\end{verbatim}

\subsubsection*{Definition}
\begin{verbatim}
sf_filter sf_allocatehelix( int nh) 
/*< allocation >*/
{
   ...
}
\end{verbatim}

\subsubsection*{Input parameters}
\begin{desclist}{\tt }{\quad}[\tt ]
   \setlength\itemsep{0pt}
   \item[nh] filter length (\texttt{int}).  
\end{desclist}

\subsubsection*{Output}
\begin{desclist}{\tt }{\quad}[\tt ]
   \setlength\itemsep{0pt}  
   \item[aa] object for helix filter. It is of type \texttt{sf\_filter}.
\end{desclist}




\subsection{{sf\_deallocatehelix}}
Frees the space allocated by \hyperref[sec:sf_allocatehelix]{\texttt{sf\_allocatehelix}} for the filter.

\subsubsection*{Definition}
\begin{verbatim}sf_deallocatehelix (aa);\end{verbatim}

\subsubsection*{Call}
\begin{verbatim}
void sf_deallocatehelix( sf_filter aa) 
/*< deallocation >*/
{
   ...
}
\end{verbatim}

\subsubsection*{Input parameters}
\begin{desclist}{\tt }{\quad}[\tt ]
   \setlength\itemsep{0pt}
   \item[aa] the filter (\texttt{sf\_filter}).  
\end{desclist}




\subsection{{sf\_displayhelix}}
Displays the filter.

\subsubsection*{Call}
\begin{verbatim}sf_displayhelix(aa);\end{verbatim}

\subsubsection*{Definition}
\begin{verbatim}
void sf_displayhelix (sf_filter aa)
/*< display filter >*/
{
   ...
}
\end{verbatim}

\subsubsection*{Input parameters}
\begin{desclist}{\tt }{\quad}[\tt ]
   \setlength\itemsep{0pt}
   \item[aa] the filter (\texttt{sf\_filter}).  
\end{desclist}

         % Helical filter definition and allocation
   \section{Recursive convolution (polynomial division) (recfilt.c)}




\subsection{{sf\_recfilt\_init}}
Initializes the linear filter by allocating the required space and initializing the required variables.


\subsubsection*{Call}
\begin{verbatim}sf_recfilt_init(nd, nb, bb);\end{verbatim}

\subsubsection*{Definition}
\begin{verbatim}
void sf_recfilt_init( int nd    /* data size */, 
                   int nb    /* filter size */, 
                   float* bb /* filter [nb] */) 
/*< initialize >*/
{
   ...
}
\end{verbatim}

\subsubsection*{Input parameters}
\begin{desclist}{\tt }{\quad}[\tt bb]
   \setlength\itemsep{0pt}
   \item[nd] size of the data which is to be filtered (\texttt{int}).  
   \item[nb] size of the filter (\texttt{int}). 
   \item[bb] filter which is to be applied (\texttt{float*}).  
\end{desclist}




\subsection{{sf\_recfilt\_lop}}
Applies the linear operator to \texttt{xx} (or \texttt{yy}) and the result is applied to \texttt{yy} (or \texttt{xx}), depending on whether \texttt{adj} is false or true, with the operator initialized by \texttt{sf\_recfilt\_init}.

\subsubsection*{Call}
\begin{verbatim}sf_recfilt_lop (adj, add, nx, ny, xx, yy);\end{verbatim}

\subsubsection*{Definition}
\begin{verbatim}
void sf_recfilt_lop( bool adj, bool add, int nx, int ny, float* xx, float*yy) 
/*< linear operator >*/
{
   ...
}
\end{verbatim}

\subsubsection*{Input parameters}
\begin{desclist}{\tt }{\quad}[\tt add]
   \setlength\itemsep{0pt}
   \item[adj] a parameter to determine whether filter is applied to \texttt{yy} or \texttt{xx} (\texttt{bool}).
   \item[add] a parameter to determine whether the input needs to be zeroed (\texttt{bool}).
   \item[nx]  size of \texttt{x}x (\texttt{int}).
   \item[ny]  size of \texttt{y}y (\texttt{int}).
   \item[xx]  data or operator, depending on whether \texttt{adj} is true or false (\texttt{float*}).
   \item[yy]  data or operator, depending on whether \texttt{adj} is true or false (\texttt{float*}).
\end{desclist}




\subsection{{sf\_recfilt\_close}}
Frees the space allocated by \texttt{sf\_recfilt\_init}.

\subsubsection*{Call}
\begin{verbatim}sf_recfilt_close ();\end{verbatim}

\subsubsection*{Definition}
\begin{verbatim}
void sf_recfilt_close (void) 
/*< free allocated storage >*/
{
   ...
}
\end{verbatim}





       % Recursive convolution (polynomial division)
   \section{Cosine Fourier transform (cosft.c)}




\subsection{{sf\_cosft}}
Makes preparations for the cosine Fourier transform, by allocating the required spaces.

\subsubsection*{Call}
\begin{verbatim}sf_cosft_init(n1);\end{verbatim}

\subsubsection*{Definition}
\begin{verbatim}
void sf_cosft_init(int n1)
/*< initialize >*/ 
{
   ...
}
\end{verbatim}

\subsubsection*{Input parameters}
\begin{desclist}{\tt }{\quad}[\tt ]
   \setlength\itemsep{0pt}
   \item[n1\_in] length of the input (\texttt{int}).  
\end{desclist}



\subsection{{sf\_cosft\_close}}
Frees the allocated space.

\subsubsection*{Call}
\begin{verbatim}sf_cosft_close();\end{verbatim}

\subsubsection*{Definition}
\begin{verbatim}
void sf_cosft_close(void) 
/*< free allocated storage >*/
{
   ...
}
\end{verbatim}




\subsection{{sf\_cosft\_frw}}
This function performs the forward cosine Fourier transform.

\subsubsection*{Call}
\begin{verbatim}sf_cosft_frw (q, o1, d1);\end{verbatim}

\subsubsection*{Definition}
\begin{verbatim}
void sf_cosft_frw (float *q /* data */, 
                   int o1   /* first sample */, 
                   int d1   /* step */) 
/*< forward transform >*/
{
   ...
}
\end{verbatim}

\subsubsection*{Input parameters}
\begin{desclist}{\tt }{\quad}[\tt d2]
   \setlength\itemsep{0pt}
   \item[q]  input data (\texttt{float}). 
   \item[o1] first sample of the input data (\texttt{int}). 
   \item[d1] step size (\texttt{int}).  
\end{desclist}



\subsection{{sf\_cosft\_inv}}
This function performs the forward cosine Fourier transform.

\subsubsection*{Call}
\begin{verbatim}sf_cosft_inv (q, o1, d1);\end{verbatim}

\subsubsection*{Definition}
\begin{verbatim}
void sf_cosft_inv (float *q /* data */, 
                   int o1   /* first sample */, 
                   int d1   /* step */) 
/*< inverse transform >*/
{
   ...
}
\end{verbatim}

\subsubsection*{Input parameters}
\begin{desclist}{\tt }{\quad}[\tt d2]
   \setlength\itemsep{0pt}
   \item[q]  input data (\texttt{float}). 
   \item[o1] first sample of the input data (\texttt{int}). 
   \item[d1] step size (\texttt{int}).  
\end{desclist}





         % Cosine Fourier transform


\chapter{Solvers}\label{sec:solvers}
   \section{Banded matrix solver (banded.c)}\label{sec:banded.c}




\subsection{{sf\_banded\_init}}
Initializes an object of type \texttt{sf\_bands} for the banded matrix, that is, it allocates the required spaces and defines initializes the variables.


\subsubsection*{Call}
\begin{verbatim}slv = sf_banded_init (n, band);\end{verbatim}

\subsubsection*{Definition}
\begin{verbatim}
sf_bands sf_banded_init (int n    /* matrix size */, 
                         int band /* band size */)
/*< initialize >*/
{
   ...
}
\end{verbatim}

\subsubsection*{Input parameters}
\begin{desclist}{\tt }{\quad}[\tt band]
   \setlength\itemsep{0pt}
   \item[n]    size of the banded matrix (\texttt{int}). 
   \item[band] size of the band (\texttt{int}).  
\end{desclist}

\subsubsection*{Output}
\begin{desclist}{\tt }{\quad}[\tt slv]
   \setlength\itemsep{0pt}
   \item[slv] an object of type \texttt{sf\_bands}.
\end{desclist}




\subsection{{sf\_banded\_define}}
Defines the banded matrix.

\subsubsection*{Call}
\begin{verbatim}sf_banded_define (slv, diag, offd);\end{verbatim}

\subsubsection*{Input parameters}
\begin{desclist}{\tt }{\quad}[\tt offd]
   \setlength\itemsep{0pt}
   \item[slv]  an object of type \texttt{sf\_bands}. 
   \item[diag] diagonal entries in the matrix (\texttt{float**}).  
   \item[offd] off-diagonal entries in the matrix (\texttt{float**}).  
\end{desclist}

\subsubsection*{Definition}
\begin{verbatim}
void sf_banded_define (sf_bands slv, 
                       float* diag  /* diagonal [n] */, 
                       float** offd /* off-diagonal [band][n] */)
/*< define the matrix >*/
{
   ...
}
\end{verbatim}




\subsection{{sf\_banded\_const\_define}}
Defines a banded matrix with constant value in the diagonal.

\subsubsection*{Call}
\begin{verbatim}sf_banded_const_define (slv, diag, offd);\end{verbatim}

\subsubsection*{Definition}
\begin{verbatim}
void sf_banded_const_define (sf_bands slv, 
                             float diag        /* diagonal */, 
                             const float* offd /* off-diagonal [band] */)
/*< define matrix with constant diagonal coefficients >*/
{
   ...
}
\end{verbatim}

\subsubsection*{Input parameters}
\begin{desclist}{\tt }{\quad}[\tt offd]
   \setlength\itemsep{0pt}
   \item[slv]  an object of type \texttt{sf\_bands}. 
   \item[diag] diagonal entries in the matrix (\texttt{float**}).  
   \item[offd] off-diagonal entries in the matrix (\texttt{float**}).  
\end{desclist}




\subsection{{sf\_banded\_const\_define\_reflect}}
Defines the banded matrix with constant diagonal values for the reflecting boundary conditions.

\subsubsection*{Call}
\begin{verbatim}sf_banded_const_define_reflect (slv, diag, offd);\end{verbatim}

\subsubsection*{Definition}
\begin{verbatim}
void sf_banded_const_define_reflect (sf_bands slv, 
                                     float diag        /* diagonal */, 
                                     const float* offd /* off-diagonal [band] */)
/*< define matrix with constant diagonal coefficients  
    and reflecting b.c. >*/
{
   ...
}
\end{verbatim}

\subsubsection*{Input parameters}
\begin{desclist}{\tt }{\quad}[\tt ]
   \setlength\itemsep{0pt}
   \item[slv]  an object of type \texttt{sf\_bands}. 
   \item[diag] diagonal entries in the matrix (\texttt{float**}).  
   \item[offd] off-diagonal entries in the matrix (\texttt{float**}).  

\end{desclist}




\subsection{{sf\_banded\_solve}}
Inverts the banded matrix.

\subsubsection*{Call}
\begin{verbatim}sf_banded_solve (slv, b);\end{verbatim}

\subsubsection*{Definition}
\begin{verbatim}
void sf_banded_solve (const sf_bands slv, float* b)
/*< invert (in place) >*/
{
   ...
}
\end{verbatim}

\subsubsection*{Input parameters}
\begin{desclist}{\tt }{\quad}[\tt slv]
   \setlength\itemsep{0pt}
   \item[slv] an object of type \texttt{sf\_bands}. Must be of type \texttt{const sf\_bands}   
   \item[b]   the inverted matrix values (\texttt{float*}).  
\end{desclist}




\subsection{{sf\_banded\_close}}
Frees the space allocated for the \texttt{sf\_bands} object.

\subsubsection*{Call}
\begin{verbatim}sf_banded_close (slv);\end{verbatim}

\subsubsection*{Definition}
\begin{verbatim}
void sf_banded_close (sf_bands slv)
/*< free allocated storage >*/
{
   ...
}
\end{verbatim}

\subsubsection*{Input parameters}
\begin{desclist}{\tt }{\quad}[\tt ]
   \setlength\itemsep{0pt}
   \item[slv] an object of type \texttt{sf\_stack}.
\end{desclist}





       % Banded matrix solver
   
   \section{Claerbout's conjugate-gradient iteration (cgstep.c)}




\subsection{{sf\_cgstep}}
Evaluates one step of the conjugate gradient method iteration.

\subsubsection*{Call}
\begin{verbatim}sf_cgstep(forget, nx, ny, x, g, rr, gg);\end{verbatim}

\subsubsection*{Definition}
\begin{verbatim}
void sf_cgstep( bool forget     /* restart flag */, 
                int nx, int ny  /* model size, data size */, 
                float* x        /* current model [nx] */, 
                const float* g  /* gradient [nx] */, 
                float* rr       /* data residual [ny] */, 
                const float* gg /* conjugate gradient [ny] */) 
/*< Step of conjugate-gradient iteration. >*/
{
   ...
}
\end{verbatim}

\subsubsection*{Input parameters}
\begin{desclist}{\tt }{\quad}[\tt forget]
   \setlength\itemsep{0pt}
   \item[forget] restart flag (\texttt{bool}). 
   \item[nx]     size of the model (\texttt{int}). 
   \item[ny]     size of the data (\texttt{int}). 
   \item[g]      the gradient (\texttt{const float*}).  
   \item[rr]     the data residual (\texttt{float*}).  
   \item[gg]     the conjugate gradient (\texttt{const float*}).  
\end{desclist}

\subsubsection*{Output}
\begin{desclist}{\tt }{\quad}[\tt ]
   \setlength\itemsep{0pt}
   \item[c.r] real part of the complex number. It is of type \texttt{double}.
\end{desclist}




\subsection{{sf\_cgstep\_close}}
Frees the space allocated for the conjugate gradient step calculation.

\subsubsection*{Call}
\begin{verbatim}sf_cgstep_close ();\end{verbatim}

\subsubsection*{Definition}
\begin{verbatim}
void sf_cgstep_close (void) 
/*< Free allocated space. >*/ 
{
   ...
}
\end{verbatim}





       % Claerbout's conjugate-gradient iteration
   
   \section{Conjugate-gradient with shaping regularization (conjgrad.c)}




\subsection{{sf\_conjgrad\_init}}
Initializes the conjugate grad\index{conjugate gradient!solver}ient solver by initializing the required variables and allocating the required space.

\subsubsection*{Call}
\begin{verbatim}sf_conjgrad_init (np, nx, nd, nr, eps, tol, verb, hasp0);\end{verbatim}

\subsubsection*{Definition}
\begin{verbatim}
void sf_conjgrad_init(int np1     /* preconditioned size */, 
                      int nx1     /* model size */, 
                      int nd1     /* data size */, 
                      int nr1     /* residual size */, 
                      float eps1  /* scaling */,
                      float tol1  /* tolerance */, 
                      bool verb1  /* verbosity flag */, 
                      bool hasp01 /* if has initial model */) 
/*< solver constructor >*/
{
   ...
}
\end{verbatim}

\subsubsection*{Input parameters}
\begin{desclist}{\tt }{\quad}[\tt hasp01]
   \setlength\itemsep{0pt}
   \item[np1]    the size of the preconditioned data (\texttt{int}). 
   \item[nx1]    size of the model (\texttt{int}). 
   \item[nd1]    size of the data (\texttt{int}). 
   \item[nr1]    size of the residual (\texttt{int}). 
   \item[eps1]   the scaling parameter (\texttt{float}). 
   \item[tol1]   tolerance to the error in the solution  (\texttt{float}). 
   \item[verb1]  verbosity flag (\texttt{bool}). 
   \item[hasp01] if there is a initial model (\texttt{bool}).  
\end{desclist}




\subsection{{sf\_conjgrad\_close}}
Frees the space allocated for the conjugate gradient solver by \texttt{sf\_conjgrad\_init}.

\subsubsection*{Definition}
\begin{verbatim}
sf_conjgrad_close();\end{verbatim}

\subsubsection*{Definition}
\begin{verbatim}
void sf_conjgrad_close(void) 
/*< Free allocated space >*/
{
   ...
}
\end{verbatim}




\subsection{{sf\_conjgrad}}
Applies the conjugate gradient solver with the shaping filter to the input data.

\subsubsection*{Definition}
\begin{verbatim}sf_conjgrad (prec, oper, shape, p, x, dat, niter);\end{verbatim}

\subsubsection*{Definition}
\begin{verbatim}
void sf_conjgrad(sf_operator prec  /* data preconditioning */, 
                 sf_operator oper  /* linear operator */, 
                 sf_operator shape /* shaping operator */, 
                 float* p          /* preconditioned model */, 
                 float* x          /* estimated model */, 
                 float* dat        /* data */, 
                 int niter         /* number of iterations */) 
/*< Conjugate gradient solver with shaping >*/
{
   ...
}
\end{verbatim}

\subsubsection*{Input parameters}
\begin{desclist}{\tt }{\quad}[\tt shape]
   \setlength\itemsep{0pt}
   \item[prec]  preconditioning operator (\texttt{sf\_operator}). 
   \item[oper]  the operator (\texttt{sf\_operator}). 
   \item[shape] the shaping operator (\texttt{sf\_operator}). 
   \item[p]     the preconditioned model (\texttt{float*}).  
   \item[x]     estimated model  (\texttt{float*}).  
   \item[dat]   the data (\texttt{float*}).  
   \item[niter] number of iterations (\texttt{int}).  
\end{desclist}



     % Conjugate-gradient with shaping regularization
   \section[CG with preconditioning (conjprec.c)]{Conjugate-gradient with preconditioning (conjprec.c)}




\subsection{{sf\_conjprec\_init}}
Initializes the conjugate gradient solver, that is, it sets the required variables and allocates the required memory.

\subsubsection*{Call}
\begin{verbatim}sf_conjprec_init(int nx, nr, eps, tol, verb, hasp0;\end{verbatim}

\subsubsection*{Definition}
\begin{verbatim}
void sf_conjprec_init(int nx     /* preconditioned size */, 
                      int nr     /* residual size */, 
                      float eps  /* scaling */,
                      float tol  /* tolerance */, 
                      bool verb  /* verbosity flag */, 
                      bool hasp0 /* if has initial model */)
/*< solver constructor >*/ 
{
   ...
}
\end{verbatim}

\subsubsection*{Input parameters}
\begin{desclist}{\tt }{\quad}[\tt hasp01]
   \setlength\itemsep{0pt}
   \item[nx1]    the size of the preconditioned data (\texttt{int}). 
   \item[nr1]    size of the residual (\texttt{int}). 
   \item[eps]    the scaling parameter (\texttt{float}). 
   \item[tol1]   tolerance to the error in the solution  (\texttt{float}). 
   \item[verb1]  verbosity flag (\texttt{bool}). 
   \item[hasp01] if there is a initial model (\texttt{bool}).  
\end{desclist}



\subsection{{sf\_conjprec\_close}}
Frees the allocated space for the conjugate gradient solver.

\subsubsection*{Call}
\begin{verbatim}sf_conjprec_close();\end{verbatim}

\subsubsection*{Definition}
\begin{verbatim}
void sf_conjprec_close(void)
/*< Free allocated space >*/
{
   ...
}
\end{verbatim}




\subsection{{sf\_conjprec}}
Applies the conjugate grad\index{conjugate gradient!with preconditioning}ient method after preconditioning to the input data.

\subsubsection*{Call}
\begin{verbatim}
void sf_conjprec(oper, prec, p, x, dat, niter);\end{verbatim}

\subsubsection*{Definition}
\begin{verbatim}
void sf_conjprec(sf_operator oper  /* linear operator */, 
                 sf_operator2 prec /* preconditioning */, 
                 float* p          /* preconditioned */, 
                 float* x          /* model */, 
                 const float* dat  /* data */, 
                 int niter         /* number of iterations */)
/*< Conjugate gradient solver with preconditioning >*/
{
   ...
}
\end{verbatim}

\subsubsection*{Input parameters}
\begin{desclist}{\tt }{\quad}[\tt niter]
   \setlength\itemsep{0pt}
   \item[oper]  the operator (\texttt{sf\_operator}). 
   \item[prec]  preconditioning operator (\texttt{sf\_operator2}). 
   \item[p]     the preconditioned data (\texttt{float*}).  
   \item[x]     model (\texttt{float*}).  
   \item[dat]   the data (\texttt{const float*}).  
   \item[niter] number of iterations (\texttt{int}).  
\end{desclist}




     % Conjugate-gradient with preconditioning   
   \section[CG iteration (complex data) (cgstep.c)]{Claerbout's conjugate-gradient iteration for complex numbers (cgstep.c)}




\subsection{{sf\_ccgstep}}\label{sec:sf_ccgstep}
Evaluates one step of the Claerbout's conjugate-gradient iteration for complex numbers.

\subsubsection*{Call}
\begin{verbatim}sf_ccgstep(forget, nx, ny, x, g, rr, gg);\end{verbatim}

\subsubsection*{Definition}
\begin{verbatim}
void sf_ccgstep( bool forget          /* restart flag */, 
                 int nx               /* model size */, 
                 int ny               /* data size */, 
                 sf_complex* x        /* current model [nx] */,  
                 const sf_complex* g  /* gradient [nx] */, 
                 sf_complex* rr       /* data residual [ny] */,
                 const sf_complex* gg /* conjugate gradient [ny] */) 
/*< Step of Claerbout's conjugate-gradient iteration for complex operators. 
    The data residual is rr = A x - dat
>*/
{
   ...
}
\end{verbatim}

\subsubsection*{Input parameters}
\begin{desclist}{\tt }{\quad}[\tt forget]
   \setlength\itemsep{0pt}
   \item[forget] restart flag (\texttt{bool}).  
   \item[nx]     size of the model (\texttt{int}).  
   \item[ny]     size of the data (\texttt{int}).  
   \item[x]      current model (\texttt{sf\_complex*}).  
   \item[g]      the gradient. Must be of type \texttt{const sf\_complex*}.  
   \item[rr]     the data residual (\texttt{sf\_complex*}).  
   \item[gg]     the conjugate gradient. Must be of type \texttt{const sf\_complex*}.  
\end{desclist}




\subsection{{sf\_ccgstep\_close}}
Frees the space allocated for \hyperref[sec:sf_ccgstep]{\texttt{sf\_ccgstep}}.

\subsubsection*{Call}
\begin{verbatim}sf_ccgstep_close();\end{verbatim}

\subsubsection*{Definition}
\begin{verbatim}
void sf_ccgstep_close (void) 
/*< Free allocated space. >*/ 
{
   ...
}
\end{verbatim}




\subsection{{dotprod}}
Returns the dot product of two complex numbers or the sum of the dot products if the are two arrays of complex numbers.

\subsubsection*{Call}
\begin{verbatim}prod = dotprod (n, x, y);\end{verbatim}

\subsubsection*{Definition}
\begin{verbatim}
static sf_double_complex dotprod (int n, const sf_complex* x, 
                                         const sf_complex* y)
/* complex dot product */
{
   ...
}
\end{verbatim}

\subsubsection*{Input parameters}
\begin{desclist}{\tt }{\quad}[\tt ]
   \setlength\itemsep{0pt}
   \item[n]  size of the array of complex numbers (\texttt{int}).  
   \item[x]  a complex number (\texttt{sf\_complex*}).  
   \item[y]  a complex number (\texttt{sf\_complex*}).  
\end{desclist}

\subsubsection*{Output}
\begin{desclist}{\tt }{\quad}[\tt ]
   \setlength\itemsep{0pt}  
   \item[prod] dot product of the complex numbers. It is of type \texttt{static sf\_double\_complex}.
\end{desclist}



      % Claerbout's conjugate-gradient iteration for complex numbers.
   \section{Conjugate-gradient with shaping regularization for complex numbers (cconjgrad.c)}




\subsection{{norm}}
Returns the $L_2$ norm of the complex number with double-precision, or the sum of $L_2$ norms, if there is an array of complex numbers.

\subsubsection*{Call}
\begin{verbatim}prod = norm (n, x);\end{verbatim}

\subsubsection*{Definition}
\begin{verbatim}
static double norm (int n, const sf_complex* x) 
/* double-precision L2 norm of a complex number */
{
   ...
}
\end{verbatim}

\subsubsection*{Input parameters}
\begin{desclist}{\tt }{\quad}[\tt x]
   \setlength\itemsep{0pt}
   \item[n] size of the array of complex numbers (\texttt{int}).  
   \item[x] a complex number (\texttt{sf\_complex*}).  
\end{desclist}

\subsubsection*{Output}
\begin{desclist}{\tt }{\quad}[\tt ]
   \setlength\itemsep{0pt}  
   \item[prod] $L_2$ norm of the complex number. It is of type \texttt{static double}.
\end{desclist}





\subsection{{sf\_cconjgrad\_init}}\label{sec:sf_cconjgrad_init}
Initializes the complex conjugate gradient solver by initializing the required variables and allocating the required space.

\subsubsection*{Definition}
\begin{verbatim}sf_cconjgrad_init (np, nx, nd, nr, eps, tol, verb, hasp0);\end{verbatim}

\subsubsection*{Definition}
\begin{verbatim}
void sf_cconjgrad_init(int np     /* preconditioned size */, 
                       int nx     /* model size */, 
                       int nd     /* data size */, 
                       int nr     /* residual size */, 
                       float eps  /* scaling */,
                       float tol  /* tolerance */, 
                       bool verb  /* verbosity flag */, 
                       bool hasp0 /* if has initial model */) 
/*< solver constructor >*/
{
   ...
}
\end{verbatim}

\subsubsection*{Input parameters}
\begin{desclist}{\tt }{\quad}[\tt hasp01]
   \setlength\itemsep{0pt}
   \item[np]    the size of the preconditioned data (\texttt{int}).  
   \item[nx]    size of the model (\texttt{int}).  
   \item[nd]    size of the data (\texttt{int}).  
   \item[nr]    size of the residual (\texttt{int}).  
   \item[eps]   the scaling parameter (\texttt{float}).  
   \item[tol]   tolerance to the error in the solution  (\texttt{float}).  
   \item[verb]  verbosity flag (\texttt{bool}).  
   \item[hasp0] if there is a initial model (\texttt{bool}).  
\end{desclist}





\subsection{{sf\_cconjgrad\_close}}
Frees the space allocated for the complex conjugate gradient solver by \hyperref[sec:sf_cconjgrad_init]{\texttt{sf\_cconjgrad\_init}}.

\subsubsection*{Definition}
\begin{verbatim}sf_cconjgrad_close();\end{verbatim}

\subsubsection*{Definition}
\begin{verbatim}
void sf_cconjgrad_close(void) 
/*< Free allocated space >*/
{
   ...
}
\end{verbatim}




\subsection{{sf\_cconjgrad}}
Applies the complex conjugate gradient solver with the shaping filter to the input data.

\subsubsection*{Definition}
\begin{verbatim}sf_cconjgrad (prec, oper, shape, p, x, dat, niter);\end{verbatim}

\subsubsection*{Definition}
\begin{verbatim}
void sf_cconjgrad(sf_coperator prec     /* data preconditioning */, 
                  sf_coperator oper     /* linear operator */, 
                  sf_coperator shape    /* shaping operator */, 
                  sf_complex* p         /* preconditioned model */, 
                  sf_complex* x         /* estimated model */, 
                  const sf_complex* dat /* data */, 
                  int niter             /* number of iterations */)
/*< Conjugate gradient solver with shaping >*/
{
   ...
}
\end{verbatim}

\subsubsection*{Input parameters}
\begin{desclist}{\tt }{\quad}[\tt shape]
   \setlength\itemsep{0pt}
   \item[prec]  preconditioning operator (\texttt{sf\_coperator}).  
   \item[oper]  the operator (\texttt{sf\_coperator}).  
   \item[shape] the shaping operator (\texttt{sf\_coperator}).  
   \item[p]     the preconditioned model (\texttt{sf\_complex*}).  
   \item[x]     estimated model  (\texttt{sf\_complex*}).  
   \item[dat]   the data (\texttt{sf\_complex*}).  
   \item[niter] number of iterations (\texttt{int}).  
\end{desclist}


    % Conjugate-gradient with shaping regularization for complex numbers

   \section{Conjugate-direction iteration (cdstep.c)}
\index{conjugate direction method!real data}




\subsection{{sf\_cdstep\_init}}\label{sec:sf_cdstep_init}
Creates a list for internal storage.

\subsubsection*{Call}
\begin{verbatim}sf_cdstep_init();\end{verbatim}

\subsubsection*{Definition}
\begin{verbatim}
void sf_cdstep_init(void) 
/*< initialize internal storage >*/
{
   ...
}
\end{verbatim}




\subsection{{sf\_cdstep\_close}}
Frees the space allocated for internal storage by \hyperref[sec:sf_cdstep_init]{\texttt{sf\_cdstep\_init}}.

\subsubsection*{Call}
\begin{verbatim}sf_cdstep_close();\end{verbatim}

\subsubsection*{Definition}
\begin{verbatim}
void sf_cdstep_close(void) 
/*< free internal storage >*/
{
   ...
}
\end{verbatim}




\subsection{{sf\_cdstep}}\label{sec:sf_cdstep}
Calculates one step for the conjugate direction iteration, that is, it calculates the new conjugate gradient for the new line search direction.

\subsubsection*{Call}
\begin{verbatim}sf_cdstep(forget, nx, ny, x, g, rr, gg);\end{verbatim}

\subsubsection*{Definition}
\begin{verbatim}
void sf_cdstep(bool forget     /* restart flag */, 
               int nx          /* model size */, 
               int ny          /* data size */, 
               float* x        /* current model [nx] */, 
               const float* g  /* gradient [nx] */, 
               float* rr       /* data residual [ny] */, 
               const float* gg /* conjugate gradient [ny] */) 
/*< Step of conjugate-direction iteration. 
  The data residual is rr = A x - dat
>*/
{
   ...
}
\end{verbatim}

\subsubsection*{Input parameters}
\begin{desclist}{\tt }{\quad}[\tt forget]
   \setlength\itemsep{0pt}
   \item[forget] restart flag (\texttt{bool}).  
   \item[nx]     model size (\texttt{int}).  
   \item[ny]     data size (\texttt{int}).  
   \item[x]      current model (\texttt{float*}).  
   \item[g]      gradient (\texttt{const float*}).  
   \item[rr]     data residual (\texttt{float*}).
   \item[gg]     conjugate gradient (\texttt{const float*}).  
\end{desclist}




\subsection{{sf\_cdstep\_diag}}
Calculates the diagonal of the model resolution matrix.

\subsubsection*{Call}
\begin{verbatim}sf_cdstep_diag(nx, res);\end{verbatim}

\subsubsection*{Definition}
\begin{verbatim}
void sf_cdstep_diag(int nx, float *res /* [nx] */)
/*< compute diagonal of the model resolution matrix >*/
{
   ...
}
\end{verbatim}

\subsubsection*{Input parameters}
\begin{desclist}{\tt }{\quad}[\tt res]
   \setlength\itemsep{0pt}
   \item[nx]  model size (\texttt{int}).  
   \item[res] diagonal entries of the model resolution matrix (\texttt{float*}).  
\end{desclist}




\subsection{{sf\_cdstep\_mat}}
Calculates the complete model resolution matrix.

\subsubsection*{Call}
\begin{verbatim}sf_cdstep_mat (nx, res);\end{verbatim}

\subsubsection*{Definition}
\begin{verbatim}
void sf_cdstep_mat (int nx, float **res /* [nx][nx] */)
/*< compute complete model resolution matrix >*/
{
   ...
}
\end{verbatim}

\subsubsection*{Input parameters}
\begin{desclist}{\tt }{\quad}[\tt ]
   \setlength\itemsep{0pt}
   \item[nx]  model size (\texttt{int}).  
   \item[res] diagonal entries of the model resolution matrix (\texttt{float**}).  
\end{desclist}

       % Conjugate-direction iteration
   \section{Linked list for use in conjugate-direction-type methods (llist.c)}\label{sec:llist.c}




\subsection{{sf\_list\_init}}
Creates an empty list. It returns a pointer to the list.

\subsubsection*{Call}
\begin{verbatim}l = sf_llist_init();\end{verbatim}

\subsubsection*{Definition}
\begin{verbatim}
sf_list sf_llist_init(void)
/*< create an empty list >*/
{
   ...
}
\end{verbatim}

\subsubsection*{Output}
\begin{desclist}{\tt }{\quad}[\tt ]
   \setlength\itemsep{0pt}
   \item[l] an empty list (\texttt{sf\_list}).  
\end{desclist}




\subsection{{sf\_llist\_rewind}}
Rewinds the list, that is, it makes the pointer to the current position equal to the first entry position.

\subsubsection*{Call}
\begin{verbatim}sf_llist_rewind(l);\end{verbatim}

\subsubsection*{Definition}
\begin{verbatim}
void sf_llist_rewind(sf_list l)
/*< return to the start >*/
{
   ...
}
\end{verbatim}

\subsubsection*{Input parameters}
\begin{desclist}{\tt }{\quad}[\tt ]
   \setlength\itemsep{0pt}
   \item[l] a list (\texttt{sf\_list}).  
\end{desclist}




\subsection{{sf\_llist\_depth}}
Returns the depth (length) of the list.

\subsubsection*{Call}
\begin{verbatim}d = sf_llist_depth(l);\end{verbatim}

\subsubsection*{Definition}
\begin{verbatim}
int sf_llist_depth(sf_list l)
/*< return list depth >*/
{
   ...
}
\end{verbatim}

\subsubsection*{Input parameters}
\begin{desclist}{\tt }{\quad}[\tt ]
   \setlength\itemsep{0pt}
   \item[l] a list (\texttt{sf\_list}).  
\end{desclist}

\subsubsection*{Output}
\begin{desclist}{\tt }{\quad}[\tt ]
   \setlength\itemsep{0pt}
   \item[l->depth] depth (length) of the list (\texttt{int}).  
\end{desclist}




\subsection{{sf\_llist\_add}}
Adds an entry to the list.

\subsubsection*{Call}
\begin{verbatim}sf_llist_add(l, g, gn);\end{verbatim}

\subsubsection*{Definition}
\begin{verbatim}
void sf_llist_add(sf_list l, float *g, double gn) 
/*< add an entry in the list >*/
{    
   ...
}
\end{verbatim}

\subsubsection*{Input parameters}
\begin{desclist}{\tt }{\quad}[\tt gn]
   \setlength\itemsep{0pt}
   \item[l] a list (\texttt{sf\_list}). 
   \item[g] value which is to be entered in the list (\texttt{float*}).  
   \item[gn] name or key of the value which is to be entered in the list (\texttt{double}).  
\end{desclist}




\subsection{{sf\_llist\_down}}
Extracts an entry from the list.

\subsubsection*{Call}
\begin{verbatim}sf_llist_down(l, g, gn);\end{verbatim}

\subsubsection*{Definition}
\begin{verbatim}
void sf_llist_down(sf_list l, float **g, double *gn)
/*< extract and entry from the list >*/
{
   ...
}
\end{verbatim}

\subsubsection*{Input parameters}
\begin{desclist}{\tt }{\quad}[\tt gn]
   \setlength\itemsep{0pt}
   \item[l] a list (\texttt{sf\_list}). 
   \item[g] location where extracted value is to be stored (\texttt{float**}).  
   \item[gn] location where the name or key of the value is to be stored (\texttt{double}).  
\end{desclist}




\subsection{{sf\_llist\_close}}
Frees the space allocated for the \texttt{sf\_list} object (list).

\subsubsection*{Call}
\begin{verbatim}sf_llist_close(l);\end{verbatim}

\subsubsection*{Definition}
\begin{verbatim}
void sf_llist_close(sf_list l) 
/*< free allocated storage >*/
{
   ...
}
\end{verbatim}

\subsubsection*{Input parameters}
\begin{desclist}{\tt }{\quad}[\tt ]
   \setlength\itemsep{0pt}
   \item[l] a list (\texttt{sf\_list}).  
\end{desclist}




\subsection{{sf\_llist\_chop}}
Removes the first entry from the list.

\subsubsection*{Call}
\begin{verbatim}sf_llist_chop(l);\end{verbatim}

\subsubsection*{Definition}
\begin{verbatim}
void sf_llist_chop(sf_list l) 
/*< free the top entry from the list >*/
{
   ...
}
\end{verbatim}

\subsubsection*{Input parameters}
\begin{desclist}{\tt }{\quad}[\tt ]
   \setlength\itemsep{0pt}
   \item[l] a list (\texttt{sf\_list}).  
\end{desclist}





        % Linked list for use in conjugate-direction-type methods
   \section{Conjugate-direction iteration for complex numbers (ccdstep.c)}
\index{conjugate direction method!complex data}




\subsection{{sf\_ccdstep\_init}}\label{sec:sf_ccdstep_init}
Creates a complex number list for internal storage.

\subsubsection*{Call}
\begin{verbatim}sf_ccdstep_init();\end{verbatim}

\subsubsection*{Definition}
\begin{verbatim}
void sf_ccdstep_init(void) 
/*< initialize internal storage >*/
{
   ...
}
\end{verbatim}




\subsection{{sf\_ccdstep\_close}}
Frees the space allocated for internal storage by \hyperref[sec:sf_ccdstep_init]{\texttt{sf\_ccdstep\_init}}.

\subsubsection*{Call}
\begin{verbatim}sf_ccdstep_close();\end{verbatim}

\subsubsection*{Definition}
\begin{verbatim}
void sf_ccdstep_close(void) 
/*< free internal storage >*/
{
   ...
}
\end{verbatim}




\subsection{{sf\_ccdstep}}
Calculates one step for the conjugate direction iteration, that is, it calculates the new conjugate gradient for the new line search direction. It works like \hyperref[sec:sf_cdstep]{\texttt{sf\_cdstep}} but for complex numbers.

\subsubsection*{Call}
\begin{verbatim}sf_ccdstep (forget, nx, ny, x, g, rr, gg);\end{verbatim}

\subsubsection*{Definition}
\begin{verbatim}
void sf_ccdstep(bool forget          /* restart flag */, 
                int nx               /* model size */, 
                int ny               /* data size */, 
                sf_complex* x        /* current model [nx] */, 
                const sf_complex* g  /* gradient [nx] */, 
                sf_complex* rr       /* data residual [ny] */, 
                const sf_complex* gg /* conjugate gradient [ny] */) 
/*< Step of conjugate-direction iteration. 
  The data residual is rr = A x - dat>*/
{
   ...
}
\end{verbatim}
\subsubsection*{Input parameters}
\begin{desclist}{\tt }{\quad}[\tt forget]
   \setlength\itemsep{0pt}
   \item[forget] restart flag (\texttt{bool}).  
   \item[nx]     model size (\texttt{int}).  
   \item[ny]     data size (\texttt{int}).  
   \item[x]      current model (\texttt{sf\_complex*}).  
   \item[g]      gradient. Must be of type \texttt{const sf\_complex*}.  
   \item[rr]     data residual (\texttt{sf\_complex*}).
   \item[gg]     conjugate gradient. Must be of type \texttt{const sf\_complex*}.  
\end{desclist}





\subsection{{saxpy}}
Multiplies a given complex number with an array of complex numbers and stores the cumulative products in another array.

\subsubsection*{Call}
\begin{verbatim}saxpy (n, a, x, y);\end{verbatim}

\subsubsection*{Definition}
\begin{verbatim}
static void saxpy(int n, sf_double_complex a, 
                  const sf_complex *x, 
                  sf_complex *y)
/* y += a*x */
{
   ...
}
\end{verbatim}

\subsubsection*{Input parameters}
\begin{desclist}{\tt }{\quad}[\tt x]
   \setlength\itemsep{0pt}
   \item[n] length of the array of complex number (\texttt{int}).  
   \item[a] a complex number. Must be of type \texttt{sf\_double\_complex}.  
   \item[x] an array complex numbers (\texttt{sf\_complex*}).  
   \item[y] location where the cumulative sum of a*x is to be stored (\texttt{sf\_complex*}).  
\end{desclist}





\subsection{{dsdot}}
Returns the Hermitian\index{dot product!Hermitian} dot product of two complex numbers or the sum of the dot products if the are two arrays of complex numbers.

\subsubsection*{Call}
\begin{verbatim}prod = dsdot(n, cx, cy);\end{verbatim}

\subsubsection*{Definition}
\begin{verbatim}
static sf_double_complex dsdot(int n, 
                               const sf_complex *cx, 
                               const sf_complex *cy)
/* Hermitian dot product */
{
   ...
}
\end{verbatim}

\subsubsection*{Input parameters}
\begin{desclist}{\tt }{\quad}[\tt cy]
   \setlength\itemsep{0pt}
   \item[n]  size of the array of complex numbers (\texttt{int}).  
   \item[cx] a complex number (\texttt{sf\_complex*}).  
   \item[cy] a complex number (\texttt{sf\_complex*}).  
\end{desclist}

\subsubsection*{Output}
\begin{desclist}{\tt }{\quad}[\tt ]
   \setlength\itemsep{0pt}  
   \item[prod] dot product of the complex numbers. It is of type \texttt{static sf\_double\_complex}.
\end{desclist}



      % Conjugate-direction iteration for complex numbers
   \section[Linked list for CD-type methods (complex data) (clist.c)]{Linked list for conjugate-direction-type methods (complex data) (clist.c)}




\subsection{{sf\_clist\_init}}
Creates an empty list for complex numbers. It returns a pointer to the list.

\subsubsection*{Call}
\begin{verbatim}sf_clist_init();\end{verbatim}

\subsubsection*{Definition}
\begin{verbatim}
sf_clist sf_clist_init(void)
/*< create an empty list >*/
{
   ...
}
\end{verbatim}

\subsubsection*{Output}
\begin{desclist}{\tt }{\quad}[\tt ]
   \setlength\itemsep{0pt}
   \item[l] an empty list. Must be of type \texttt{sf\_clist}.  
\end{desclist}




\subsection{{sf\_clist\_rewind}}
Rewinds the list, that is, it makes the pointer to the current position equal to the first entry position.

\subsubsection*{Call}
\begin{verbatim}sf_clist_rewind(l);\end{verbatim}

\subsubsection*{Definition}
\begin{verbatim}
void sf_clist_rewind(sf_clist l)
/*< return to the start >*/
{
   ...
}
\end{verbatim}

\subsubsection*{Input parameters}
\begin{desclist}{\tt }{\quad}[\tt ]
   \setlength\itemsep{0pt}
   \item[l] a list. Must be of type \texttt{sf\_clist}.  
\end{desclist}




\subsection{{sf\_clist\_depth}}
Returns the depth (length) of the list.

\subsubsection*{Input parameters}
\begin{desclist}{\tt }{\quad}[\tt ]
   \setlength\itemsep{0pt}
   \item[l] a list. Must be of type \texttt{sf\_clist}.  
\end{desclist}

\subsubsection*{Output}
\begin{desclist}{\tt }{\quad}[\tt ]
   \setlength\itemsep{0pt}
   \item[l->depth] depth (length) of the list (\texttt{int}).  
\end{desclist}




\subsection{{sf\_clist\_add}}
Adds an entry to the list.

\subsubsection*{Call}
\begin{verbatim}sf_clist_depth(l);\end{verbatim}

\subsubsection*{Definition}
\begin{verbatim}
int sf_clist_depth(sf_clist l)
/*< return list depth >*/
{
   ...
}
\end{verbatim}

\subsubsection*{Call}
\begin{verbatim}sf_clist_add(l, g, gn);\end{verbatim}

\subsubsection*{Definition}
\begin{verbatim}
void sf_clist_add(sf_clist l, sf_complex *g, double gn) 
/*< add an entry in the list >*/
{    
   ...
}
\end{verbatim}

\subsubsection*{Input parameters}
\begin{desclist}{\tt }{\quad}[\tt gn]
   \setlength\itemsep{0pt}
   \item[l]  a list. Must be of type \texttt{sf\_clist}. 
   \item[g]  value which is to be entered in the list. Must be of type \	exttt{sf\_complex*}.  
   \item[gn] name or key of the value which is to be entered in the list (\texttt{double}).  
\end{desclist}




\subsection{{sf\_llist\_down}}
Extracts an entry from the list.

\subsubsection*{Call}
\begin{verbatim}sf_clist_down(l, g, gn);\end{verbatim}

\subsubsection*{Definition}
\begin{verbatim}
void sf_clist_down(sf_clist l, sf_complex **g, double *gn)
/*< extract and entry from the list >*/
{
   ...
}
\end{verbatim}

\subsubsection*{Input parameters}
\begin{desclist}{\tt }{\quad}[\tt gn]
   \setlength\itemsep{0pt}
   \item[l]  a list. Must be of type \texttt{sf\_clist}. 
   \item[g]  location where extracted value is to be stored (\texttt{sf\_complex**}). 
   \item[gn] location where the name or key of the value is to be stored (\texttt{double*}).  
\end{desclist}




\subsection{{sf\_clist\_close}}
Frees the space allocated for the \texttt{sf\_clist} object (list).

\subsubsection*{Call}
\begin{verbatim}sf_clist_close(l);\end{verbatim}

\subsubsection*{Definition}
\begin{verbatim}
void sf_clist_close(sf_clist l) 
/*< free allocated storage >*/
{
   ...
}
\end{verbatim}

\subsubsection*{Input parameters}
\begin{desclist}{\tt }{\quad}[\tt ]
   \setlength\itemsep{0pt}
   \item[l] a list. Must be of type \texttt{sf\_clist}.  
\end{desclist}




\subsection{{sf\_clist\_chop}}
Removes the first entry from the list.

\subsubsection*{Call}
\begin{verbatim}sf_clist_chop(l);\end{verbatim}

\subsubsection*{Definition}
\begin{verbatim}
void sf_clist_chop(sf_clist l) 
/*< free the top entry from the list >*/
{
   ...
}
\end{verbatim}

\subsubsection*{Input parameters}
\begin{desclist}{\tt }{\quad}[\tt ]
   \setlength\itemsep{0pt}
   \item[l] a list. Must be of type \texttt{sf\_clist}.  
\end{desclist}



        % Linked list for use in conjugate-direction-type methods with complex numbers
    
   \section{Solving quadratic equations (quadratic.c)}
Solves the equation $ax^2 + 2bx + c = 0$ and returns the smallest positive root.



\subsection{{sf\_quadratic\_solve}}

\subsubsection*{Call}
\begin{verbatim}x1 = sf_quadratic_solve (a, b, c);\end{verbatim}

\subsubsection*{Definition}
\begin{verbatim}
float sf_quadratic_solve (float a, float b, float c) 
/*< solves a x^2 + 2 b x + c == 0 for smallest positive x >*/
{
   ...
}
\end{verbatim}

\subsubsection*{Input parameters}
\begin{desclist}{\tt }{\quad}[\tt ]
   \setlength\itemsep{0pt}
   \item[a] coefficient of $x^2$ (\texttt{float}). 
   \item[b] coefficient of $x$ (\texttt{float}). 
   \item[c] constant term (\texttt{float}).  
\end{desclist}

\subsubsection*{Output}
\begin{desclist}{\tt }{\quad}[\tt ]
   \setlength\itemsep{0pt}
   \item[x1] solution of the quadratic equation (\texttt{float}).
\end{desclist}




    % Solving quadratic equations
   \section{Zero finder (fzero.c)}




\subsection{{sf\_zero}}
Returns the zero (root) of the input function, $f(x)$ in a specified interval $[a,b]$.

\subsubsection*{Call}
\begin{verbatim}b = sf_zero ((*func)(float), a, b, fa, fb, toler, verb);\end{verbatim}

\subsubsection*{Definition}
\begin{verbatim}
float sf_zero (float (*func)(float) /* function f(x) */, 
               float a, float b     /* interval */, 
               float fa, float fb   /* f(a) and f(b) */,
               float toler          /* tolerance */, 
               bool verb            /* verbosity flag */)
/*< Return c such that f(c)=0 (within tolerance). 
  fa and fb should have different signs. >*/
{
    float c, fc, m, s, p, q, r, e, d;
    char method[256];

   ...
    return b;
}
\end{verbatim}

\subsubsection*{Input parameters}
\begin{desclist}{\tt }{\quad}[\tt (*func)(float)]
   \setlength\itemsep{0pt}
   \item[(*func)(float)] function, the root of which is required. Must be of type \texttt{sf\_double\_complex}.  
   \item[a]     lower limit of the interval (\texttt{float}).
   \item[b]     upper limit of the interval (\texttt{float}).
   \item[fa]    function value at the lower limit (\texttt{float}).
   \item[fb]    function value at the upper limit (\texttt{float}).
   \item[toler] error tolerance (\texttt{float}).
   \item[verb]  verbosity flag (\texttt{bool}).
\end{desclist}

\subsubsection*{Output}
\begin{desclist}{\tt }{\quad}[\tt ]
   \setlength\itemsep{0pt}  
   \item[b] root of the input function. It is of type \texttt{float}.
\end{desclist}


        % Zero finder
   \section{Runge-Kutta ODE solvers (runge.c)}




\subsection{{sf\_runge\_init}}
Initializes the required variables and allocates the required space for the ODE solver for raytracing.

\subsubsection*{Call}
\begin{verbatim}sf_runge_init(dim1, n1, d1);\end{verbatim}

\subsubsection*{Definition}
\begin{verbatim}
void sf_runge_init(int dim1 /* dimensionality */, 
                   int n1   /* number of ray tracing steps */, 
                   float d1 /* step in time */)
/*< initialize >*/
{
   ...
}

\subsubsection*{Input parameters}
\begin{desclist}{\tt }{\quad}[\tt dim1]
   \setlength\itemsep{0pt}
   \item[dim1] dimension of the ODE (\texttt{int}). 
   \item[n1]   number of steps for performing the raytracing (\texttt{int}).  
   \item[d1]   number of time steps for performing the raytracing (\texttt{int}).  
\end{desclist}



\subsection{{sf\_runge\_close\_init}}
Frees all allocated memory.

\subsubsection*{Call}
\begin{verbatim}sf_runge_close();\end{verbatim}

\subsubsection*{Definition}
\begin{verbatim}
void sf_runge_close(void)
/*< free allocated storage >*/
{
   ...
}
\end{verbatim}




\subsection{{sf\_ode23}}
This function solves a first order ODE to calculate the travel time by raytracing.

\subsubsection*{Call}
\begin{verbatim}
f = sf_ode23 (float t, tol, y, par, 
                (*rhs)(void*,float*,float*), (*term)(void*,float*));
\end{verbatim}

\subsubsection*{Definition}
\begin{verbatim}
float sf_ode23 (float t    /* time integration */,
                float* tol /* error tolerance */,
                float* y   /* [dim] solution */, 
                void* par  /* parameters for function evaluation */,
                void (*rhs)(void*,float*,float*) 
                /* RHS function */, 
                int (*term)(void*,float*)
             /* function returning 1 if the ray needs to terminate */)
/*< ODE solver for dy/dt = f where f comes from rhs(par,y,f)
    Note: Value of y is changed inside the function.>*/
{
   ...
}
\end{verbatim}

\subsubsection*{Input parameters}
\begin{desclist}{\tt }{\quad}[\tt (*rhs)(void*,float*,float*)]
   \setlength\itemsep{0pt}
   \item[t]   total time for integration (\texttt{float}). 
   \item[tol] error tolerance (\texttt{float*}).  
   \item[y]   the solution, of dimension \texttt{dim} (\texttt{float*}).  
   \item[par] parameters to evaluate the rhs function (\texttt{void*}). 
   \item[(*rhs)(void*,float*,float*)] function which evaluates the rhs of the ODE. Its inputs the parameters, the solution and the k's in Runge-Kutta scheme. Output is the rhs function $f$ of the ODE (\texttt{void*}).
   \item[(*term)(void*,float*)] a function which returns 1 if the ray is to be terminated (\texttt{int}).
\end{desclist}

\subsubsection*{Output}
\begin{desclist}{\tt }{\quad}[\tt ]
   \setlength\itemsep{0pt}
   \item[t1] total travel time for the ray. It is of type \texttt{float}.
\end{desclist}




\subsection{{sf\_ode23\_step}}
Solves a first order ODE and returns trajectory calculated by raytracing.

\subsubsection*{Call}
\begin{verbatim}
it = sf_ode23_step (y, par, (*rhs)(void*,float*,float*), (*term)(void*,float*), traj);
\end{verbatim}

\subsubsection*{Definition}
\begin{verbatim}
int sf_ode23_step (float* y    /* [dim] solution */, 
                   void* par   /* parameters for function evaluation */,
                   void (*rhs)(void*,float*,float*) 
                   /* RHS function */, 
                   int (*term)(void*,float*)
                   /* function returning 1 if the ray needs to terminate */, 
                   float** traj /* [nt+1][dim] - ray trajectory (output) */) 
/*< ODE solver for dy/dt = f where f comes from rhs(par,y,f)
  Note:
  1. Value of y is changed inside the function.
  2. The output code for it = ode23_step(...)
  it=0 - ray traced to the end without termination
  it>0 - ray terminated
  The total traveltime along the ray is 
  nt*dt if (it = 0); it*dt otherwise 
  >*/
{
   ...  
}
\end{verbatim}

\subsubsection*{Input parameters}
\begin{desclist}{\tt }{\quad}[\tt (*rhs)(void*,float*,float*)]
   \setlength\itemsep{0pt}
   \item[y]   the solution, of dimension \texttt{dim} (\texttt{float*}).  
   \item[par] parameters to evaluate the rhs function (\texttt{void*}).  
   \item[(*rhs)(void*,float*,float*)] function which evaluates the rhs of the ODE. Its inputs the parameters, the solution and the k's in Runge-Kutta scheme. Output is the rhs function $f$ of the ODE (\texttt{void*}).
   \item[(*term)(void*,float*)] a function which returns 1 if the ray is to be terminated (\texttt{int}). 
   \item[traj] location where the output ray trajectory is to be stored (\texttt{float**}).  
\end{desclist}

\subsubsection*{Output}
\begin{desclist}{\tt }{\quad}[\tt ]
   \setlength\itemsep{0pt}
   \item[t1] total travel time for the ray. It is of type \texttt{int}.
\end{desclist}





        % Runge-Kutta ODE solvers

   \section{Solver function for iterative least-squares optimization (tinysolver.c)}




\subsection{{sf\_tinysolver}}
Performs the linear inversion for equations of the type $Lx=y$ to compute the model $x$.

\subsubsection*{Call}
\begin{verbatim}sf_tinysolver (Fop, stepper, nm, nd, m, m0, d, niter);\end{verbatim}


\subsubsection*{Definition}
\begin{verbatim}
void sf_tinysolver (sf_operator Fop       /* linear operator */, 
                    sf_solverstep stepper /* stepping function */, 
                    int nm                /* size of model */, 
                    int nd                /* size of data */, 
                    float* m              /* estimated model */,
                    const float* m0       /* starting model */,
                    const float* d        /* data */, 
                    int niter             /* iterations */)
/*< Generic linear solver. Solves oper{x} =~ dat >*/
{
   ...
}
\end{verbatim}


\subsubsection*{Input parameters}
\begin{desclist}{\tt }{\quad}[\tt stepper]
   \setlength\itemsep{0pt}
   \item[Fop]     a linear operator applied to the model $x$.  Must be of type \texttt{sf\_operator}. 
   \item[stepper] a stepping function to perform updates on the initial model (\texttt{sf\_solverstep}). 
   \item[nm]      size of the model (\texttt{int}). 
   \item[nd]      size of the data (\texttt{int}). 
   \item[m]       estimated model (\texttt{int}). 
   \item[mo]      initial model (\texttt{const float}). 
   \item[d]       data (\texttt{const float}). 
   \item[niter]   number of iterations (\texttt{int}).
\end{desclist}


   % Solver function for iterative least-squares optimization
   \section{Solver functions for iterative least-squares optimization (bigsolver.c)}




\subsection{{sf\_solver\_prec}}
Applies solves generic linear equations after preconditioning the data.

\subsubsection*{Call}
\begin{verbatim}
sf_solver_prec (oper, solv, prec, nprec, nx, ny, 
                x, dat, niter, eps, "x0",x0, ..., "end");
\end{verbatim}

\subsubsection*{Definition}
\begin{verbatim}
void sf_solver_prec (sf_operator oper   /* linear operator */, 
                     sf_solverstep solv /* stepping function */, 
                     sf_operator prec   /* preconditioning operator */, 
                     int nprec          /* size of p */, 
                     int nx             /* size of x */, 
                     int ny             /* size of dat */, 
                     float* x           /* estimated model */, 
                     const float* dat   /* data */, 
                     int niter          /* number of iterations */, 
                     float eps          /* regularization parameter */, 
                    ...                /* variable number of arguments */) 
/*< Generic preconditioned linear solver.
 ---
 Solves
 oper{x} =~ dat
 eps p   =~ 0
 where x = prec{p}
 ---
 The last parameter in the call to this function should be "end".
 Example: 
 ---
 sf_solver_prec (oper_lop,sf_cgstep,prec_lop,
 np,nx,ny,x,y,100,1.0,"x0",x0,"end");
 ---
 Parameters in ...:
 ... 
 "wt":     float*:         weight      
 "wght":   sf_weight wght: weighting function
 "x0":     float*:         initial model
 "nloper": sf_operator:    nonlinear operator  
 "mwt":    float*:         model weight
 "verb":   bool:           verbosity flag
 "known":  bool*:          known model mask
 "nmem":   int:            iteration memory
 "nfreq":  int:            periodic restart
 "xmov":   float**:        model iteration
 "rmov":   float**:        residual iteration
 "err":    float*:         final error
 "res":    float*:         final residual
 "xp":     float*:         preconditioned model
 >*/
{
   ...
}
\end{verbatim}

\subsubsection*{Input parameters}
\begin{desclist}{\tt }{\quad}[\tt nprec]
   \setlength\itemsep{0pt}
   \item[oper]  the operator. Must be of type \texttt{sf\_operator}
   \item[solv]  the stepping function (\texttt{sf\_solverstep}).  
   \item[prec]  preconditioning operator (\texttt{sf\_operator}).   
   \item[nprec] size of the preconditioned data (\texttt{int}).  
   \item[nx]    size of the estimated model  (\texttt{int}).  
   \item[ny]    size of the data  (\texttt{int}).  
   \item[x]     estimated model  (\texttt{float*}).  
   \item[dat]   the data (\texttt{const float*}).  
   \item[niter] number of iterations (\texttt{int}).  
   \item[eps]   regularization parameter (\texttt{float}).  
   \item[...]   variable number of arguments.
\end{desclist}



\subsection{{sf\_csolver\_prec}}
Applies solves generic linear equations for complex data after preconditioning the data.

\subsubsection*{Call}
\begin{verbatim}
sf_csolver_prec (oper, solv, prec, nprec, nx, ny,
                 x, dat, niter, eps, "x0",x0, ..., "end");
\end{verbatim}

\subsubsection*{Definition}
\begin{verbatim}
sf_csolver_prec (oper, solv, prec, nprec, nx, ny,
                 x, dat, niter, eps, niter, eps, "x0",x0, ..., "end");
\end{verbatim}

\subsubsection*{Input parameters}
\begin{desclist}{\tt }{\quad}[\tt nprec]
   \setlength\itemsep{0pt}
   \item[oper]  the operator (\texttt{sf\_coperator}).  
   \item[solv]  the stepping function (\texttt{sf\_csolverstep}).  
   \item[prec]  preconditioning operator (\texttt{sf\_coperator}).  
   \item[nprec] size of the preconditioned data (\texttt{int}).  
   \item[nx]    size of the estimated model  (\texttt{int}).  
   \item[ny]    size of the data  (\texttt{int}).  
   \item[x]     estimated model  (\texttt{sf\_complex*}).  
   \item[dat]   the data. Must be of type \texttt{const sf\_complex*}.  
   \item[niter] number of iterations (\texttt{int}).  
   \item[eps]   regularization parameter (\texttt{float}).  
   \item[...]   variable number of arguments.
\end{desclist}





\subsection{{sf\_solver\_reg}}
Applies solves generic linear equations after regularizing the data.

\subsubsection*{Call}
\begin{verbatim}
sf_solver_reg (oper, solv, reg, nreg, nx, ny,
               x, y, niter, eps, "x0",x0, ..., "end");
\end{verbatim}

\subsubsection*{Definition}
\begin{verbatim}
void sf_solver_reg (sf_operator oper   /* linear operator */, 
                    sf_solverstep solv /* stepping function */,
                    sf_operator reg    /* regularization operator */, 
                    int nreg           /* size of reg{x} */, 
                    int nx             /* size of x */, 
                    int ny             /* size of dat */, 
                    float* x           /* estimated model */, 
                    const float* dat   /* data */, 
                    int niter          /* number of iterations */, 
                    float eps          /* regularization parameter */, 
                   ...                /* variable number of arguments */) 
/*< Generic regularized linear solver.
  ---
  Solves
  oper{x}    =~ dat
  eps reg{x} =~ 0
  ---
  The last parameter in the call to this function should be "end".
  Example: 
  ---
  sf_solver_reg (oper_lop,sf_cgstep,reg_lop,
  np,nx,ny,x,y,100,1.0,"x0",x0,"end");
  ---
  Parameters in ...:
  
  "wt":     float*:         weight      
  "wght":   sf_weight wght: weighting function
  "x0":     float*:         initial model
  "nloper": sf_operator:    nonlinear operator  
  "nlreg":  sf_operator:    nonlinear regularization operator
  "verb":   bool:           verbosity flag
  "known":  bool*:          known model mask
  "nmem":   int:            iteration memory
  "nfreq":  int:            periodic restart
  "xmov":   float**:        model iteration
  "rmov":   float**:        residual iteration
  "err":    float*:         final error
  "res":    float*:         final residual
  "resm":   float*:         final model residual
  >*/
{
   ...
}
\end{verbatim}

\subsubsection*{Input parameters}
\begin{desclist}{\tt }{\quad}[\tt niter]
   \setlength\itemsep{0pt}
   \item[oper]  the linear operator (\texttt{sf\_operator}).
   \item[solv]  the stepping function (\texttt{sf\_solverstep}).  
   \item[prec]  regularization operator (\texttt{sf\_operator}).   
   \item[nreg]  size of the regularized data (\texttt{int}).  
   \item[nx]    size of the estimated model  (\texttt{int}).  
   \item[ny]    size of the data  (\texttt{int}).  
   \item[x]     estimated model  (\texttt{float*}).  
   \item[dat]   the data (\texttt{const float*}).  
   \item[niter] number of iterations (\texttt{int}).  
   \item[eps]   regularization parameter (\texttt{float}).  
   \item[...]   variable number of arguments.
\end{desclist}




\subsection{{sf\_solver}}
Solves generic linear equations.


\subsubsection*{Call}
\begin{verbatim}
sf_solver (oper, solv, nx, ny, x, dat, niter, "x0",x0, ..., "end");
\end{verbatim}

\subsubsection*{Definition}
\begin{verbatim}
void sf_solver (sf_operator oper   /* linear operator */, 
                sf_solverstep solv /* stepping function */, 
                int nx             /* size of x */, 
                int ny             /* size of dat */, 
                float* x           /* estimated model */, 
                const float* dat   /* data */, 
                int niter          /* number of iterations */, 
               ...                /* variable number of arguments */)
/*< Generic linear solver.
  ---
  Solves
  oper{x}    =~ dat
  ---
  The last parameter in the call to this function should be "end".
  Example: 
  ---
  sf_solver (oper_lop,sf_cgstep,nx,ny,x,y,100,"x0",x0,"end");
  ---
  Parameters in ...:
  ---
  "wt":     float*:         weight      
  "wght":   sf_weight wght: weighting function
  "x0":     float*:         initial model
  "nloper": sf_operator:    nonlinear operator
  "mwt":    float*:         model weight
  "verb":   bool:           verbosity flag
  "known":  bool*:          known model mask
  "nmem":   int:            iteration memory
  "nfreq":  int:            periodic restart
  "xmov":   float**:        model iteration
  "rmov":   float**:        residual iteration
  "err":    float*:         final error
  "res":    float*:         final residual
  >*/ 
{
   ...
}
\end{verbatim}

\subsubsection*{Input parameters}
\begin{desclist}{\tt }{\quad}[\tt niter]
   \setlength\itemsep{0pt}
   \item[oper]  the operator (\texttt{sf\_operator}).
   \item[solv]  the stepping function (\texttt{sf\_solverstep}).  
   \item[nx]    size of the estimated model  (\texttt{int}).  
   \item[ny]    size of the data  (\texttt{int}).  
   \item[x]     estimated model  (\texttt{float*}).  
   \item[dat]   the data (\texttt{const float*}).  
   \item[niter] number of iterations (\texttt{int}).  
   \item[eps]   regularization parameter (\texttt{float}).  
   \item[...]   variable number of arguments.
\end{desclist}




\subsection{{sf\_left\_solver}}
Solves generic linear equations for non-symmetric operators.

\subsubsection*{Call}
\begin{verbatim}sf_left_solver (oper, solv, nx, x, dat, niter, "x0",x0, ..., "end");\end{verbatim}

\subsubsection*{Definition}
\begin{verbatim}
void sf_left_solver (sf_operator oper   /* linear operator */, 
                     sf_solverstep solv /* stepping function */, 
                     int nx             /* size of \texttt{x} and dat */, 
                     float* x           /* estimated model */, 
                     const float* dat   /* data */, 
                     int niter          /* number of iterations */, 
                    ...                /* variable number of arguments */)
/*< Generic linear solver for non-symmetric operators.
  ---
  Solves
  oper{x}    =~ dat
  ---
  The last parameter in the call to this function should be "end".
  Example: 
  ---
  sf_left_solver (oper_lop,sf_cdstep,nx,ny,x,y,100,"x0",x0,"end");
  ---
  Parameters in ...:
  ---
  "wt":     float*:         weight      
  "wght":   sf_weight wght: weighting function
  "x0":     float*:         initial model
  "nloper": sf_operator:    nonlinear operator
  "mwt":    float*:         model weight
  "verb":   bool:           verbosity flag
  "known":  bool*:          known model mask
  "nmem":   int:            iteration memory
  "nfreq":  int:            periodic restart
  "xmov":   float**:        model iteration
  "rmov":   float**:        residual iteration
  "err":    float*:         final error
  "res":    float*:         final residual
  >*/ 
{
   ...
}
\end{verbatim}

\subsubsection*{Input parameters}
\begin{desclist}{\tt }{\quad}[\tt niter]
   \setlength\itemsep{0pt}
   \item[oper]  the operator (\texttt{sf\_operator}).
   \item[solv]  the stepping function (\texttt{sf\_solverstep}).  
   \item[nx]    size of the estimated model  (\texttt{int}).  
   \item[x]     estimated model  (\texttt{float*}).  
   \item[dat]   the data (\texttt{const float*}).  
   \item[niter] number of iterations (\texttt{int}).  
   \item[eps]   regularization parameter (\texttt{float}).  
   \item[...]   variable number of arguments.
\end{desclist}




\subsection{{sf\_csolver}}
Solves generic linear equations for complex data.

\subsubsection*{Call}
\begin{verbatim}sf_csolver (oper, solv, nx, ny, x, dat, niter, "x0",x0, ..., "end");\end{verbatim}

\subsubsection*{Definition}
\begin{verbatim}
void sf_csolver (sf_coperator oper        /* linear operator */, 
                 sf_csolverstep solv      /* stepping function */, 
                 int nx                   /* size of x */, 
                 int ny                   /* size of dat */, 
                 sf_complex* x            /* estimated model */, 
                 const sf_complex* dat    /* data */, 
                 int niter                /* number of iterations */, 
                ...                      /* variable number of arguments */) 
/*< Generic linear solver for complex data.
  ---
  Solves
  oper{x}    =~ dat
  ---
  The last parameter in the call to this function should be "end".
  Example: 
  ---
  sf_csolver (oper_lop,sf_cgstep,nx,ny,x,y,100,"x0",x0,"end");
  ---
  Parameters in ...:
  ---
  "wt":     float*:          weight      
  "wght":   sf_cweight wght: weighting function
  "x0":     sf_complex*:  initial model
  "nloper": sf_coperator:    nonlinear operator  
  "verb":   bool:            verbosity flag
  "known":  bool*:           known model mask
  "nmem":   int:             iteration memory
  "nfreq":  int:             periodic restart
  "xmov":   sf_complex**: model iteration
  "rmov":   sf_complex**: residual iteration
  "err":    float*:  final error
  "res":    sf_complex*:  final residual
  >*/ 
{
   ...
}
\end{verbatim}

\subsubsection*{Input parameters}
\begin{desclist}{\tt }{\quad}[\tt niter]
   \setlength\itemsep{0pt}
   \item[oper]  the operator (\texttt{sf\_coperator}).  
   \item[solv]  the stepping function (\texttt{sf\_csolverstep}).  
   \item[nx]    size of the estimated model  (\texttt{int}).  
   \item[ny]    size of the data  (\texttt{int}).  
   \item[x]     estimated model  (\texttt{sf\_complex*}).  
   \item[dat]   the data. Must be of type \texttt{const sf\_complex*}.  
   \item[niter] number of iterations (\texttt{int}).  
   \item[eps]   regularization parameter (\texttt{float}).  
   \item[... ]   variable number of arguments.
\end{desclist}

    % Solver functions for iterative least-squares optimization
   \section{Weighting for iteratively-reweighted least squares (irls.c)}




\subsection{{sf\_irls\_init}}\label{sec:sf_irls_init}
Allocates the space equal to the data size for iteratively-reweighted least squares.

\subsubsection*{Call}
\begin{verbatim}sf_irls_init(n);\end{verbatim}

\subsubsection*{Definition}
\begin{verbatim}
void sf_irls_init(int n) 
/*< Initialize with data size >*/
{
   ...
}
\end{verbatim}

\subsubsection*{Input parameters}
\begin{desclist}{\tt }{\quad}[\tt ]
   \setlength\itemsep{0pt}
   \item[n] size of the data (\texttt{int}).  
\end{desclist}




\subsection{{sf\_irls\_close}}
Frees the space allocated for the iteratively-reweighted least squares by \hyperref[sec:sf_irls_init]{\texttt{sf\_irls\_init}}.

\subsubsection*{Call}
\begin{verbatim}sf_irls_close();\end{verbatim}

\subsubsection*{Definition}
\begin{verbatim}
void sf_irls_close(void) 
/*< free allocated storage >*/
{
   ...
}
\end{verbatim}




\subsection{{sf\_l1}}
Calculates the weights for $L_1$ norm.

\subsubsection*{Call}
\begin{verbatim}sf_l1 (n, res, weight);\end{verbatim}

\subsubsection*{Definition}
\begin{verbatim}
void sf_l1 (int n, const float *res, float *weight)  
/*< weighting for L1 norm >*/
{
   ...
}
\end{verbatim}

\subsubsection*{Input parameters}
\begin{desclist}{\tt }{\quad}[\tt weight]
   \setlength\itemsep{0pt}
   \item[n]      size of the data (\texttt{int}).  
   \item[res]    data (\texttt{const float*}).  
   \item[weight] weights for $L_1$ norm (\texttt{float*}).  
\end{desclist}




\subsection{{sf\_cauchy}}
Calculates the weights for Cauchy norm.

\subsubsection*{Call}
\begin{verbatim}sf_cauchy (n, res, weight);\end{verbatim}

\subsubsection*{Definition}
\begin{verbatim}
void sf_cauchy (int n, const float *res, float *weight)  
/*< weighting for Cauchy norm >*/
{
   ...
}
\end{verbatim}

\subsubsection*{Input parameters}
\begin{desclist}{\tt }{\quad}[\tt weight]
   \setlength\itemsep{0pt}
   \item[n]      size of the data (\texttt{int}).  
   \item[res]    data (\texttt{const float*}).  
   \item[weight] weights for Cauchy norm (\texttt{float*}).  
\end{desclist}

         % Weighting for iteratively-reweighted least squares
   \section{Tridiagonal matrix solver (tridiagonal.c)}




\subsection{{sf\_tridiagonal\_init}}
Initializes the object of the abstract data of type \texttt{sf\_tris}, which contains a matrix of size \texttt{n} with separate variables for the diagonal and off-diagonal entries and also for the solution to the matrix equation which it will be used to solve.

\subsubsection*{Call}
\begin{verbatim}slv = sf_tridiagonal_init (n);\end{verbatim}

\subsubsection*{Definition}
\begin{verbatim}
sf_tris sf_tridiagonal_init (int n /* matrix size */)
/*< initialize >*/
{
   ...
}
\end{verbatim}

\subsubsection*{Input parameters}
\begin{desclist}{\tt }{\quad}[\tt ]
   \setlength\itemsep{0pt}
   \item[n] size of the matrix (\texttt{int}).
\end{desclist}

\subsubsection*{Output}
\begin{desclist}{\tt }{\quad}[\tt ]
   \setlength\itemsep{0pt}
   \item[slv] the tridiagonal solver. It is of type \texttt{sf\_tris}.
\end{desclist}




\subsection{{sf\_tridiagonal\_define}}\label{sec:sf_tridiagonal_define}
Fills in the diagonal and off-diagonal entries in the tridiagonal solver based on the input entries \texttt{diag} and \texttt{offd}.

\subsubsection*{Call}
\begin{verbatim}sf_tridiagonal_define (slv, diag, offd);\end{verbatim}

\subsubsection*{Definition}
\begin{verbatim}
void sf_tridiagonal_define (sf_tris slv /* solver object */, 
                            float* diag /* diagonal */, 
                            float* offd /* off-diagonal */)
/*< fill the matrix >*/
{
   ...
}
\end{verbatim}

\subsubsection*{Input parameters}
\begin{desclist}{\tt }{\quad}[\tt offd]
   \setlength\itemsep{0pt}
   \item[slv]  the solver object (\texttt{sf\_tris}). 
   \item[diag] the diagonal (\texttt{float*}).
   \item[offd] the off-diagonal (\texttt{float*}).
\end{desclist}




\subsection{{sf\_tridiagonal\_const\_define}}
Fills in the diagonal and off diagonal entries in the tridiagonal solver based on the input entries \texttt{diag} and \texttt{offd}. It works like \hyperref[sec:sf_tridiagonal_define]{\texttt{sf\_tridiagonal\_define}} but for the special case where the matrix is Toeplitz.

\subsubsection*{Call}
\begin{verbatim}sf_tridiagonal_const_define (slv, diag, offd, damp);\end{verbatim}

\subsubsection*{Definition}
\begin{verbatim}
void sf_tridiagonal_const_define (sf_tris slv /* solver object */, 
                                  float diag  /* diagonal */, 
                                  float offd  /* off-diagonal */,
                                  bool damp   /* damping */)
/*< fill the matrix for the Toeplitz case >*/
{
   ...
}
\end{verbatim}

\subsubsection*{Input parameters}
\begin{desclist}{\tt }{\quad}[\tt pffd]
   \setlength\itemsep{0pt}
   \item[slv]  the solver object. Must be of type \texttt{sf\_tris}. 
   \item[diag] the diagonal (\texttt{float*}).
   \item[offd] the off-diagonal (\texttt{float*}).
   \item[damp] damping (\texttt{bool}).
\end{desclist}




\subsection{{sf\_tridiagonal\_solve}}\label{sec:sf_tridiagonal_solve}
Solves the matrix equation (like $La=b$, where $b$ is the input for the solve function, $a$ is the output and $L$ is the matrix defined by \hyperref[sec:sf_tridiagonal_define]{\texttt{sf\_tridiagonal\_define}}) and stores the solution in the space allocated by the variable \texttt{x} in the \texttt{slv} object.


\subsubsection*{Call}
\begin{verbatim}sf_tridiagonal_solve (sf_tris slv, b);\end{verbatim}

\subsubsection*{Definition}
\begin{verbatim}
void sf_tridiagonal_solve (sf_tris slv /* solver object */, 
                           float* b /* in - right-hand side, out - solution */)
/*< invert the matrix >*/
{
   ...
}
\end{verbatim}

\subsubsection*{Input parameters}
\begin{desclist}{\tt }{\quad}[\tt slv]
   \setlength\itemsep{0pt}
   \item[slv] the solver object. Must be of type \texttt{sf\_tris}. 
   \item[b]   right hand side of the matrix equation $La=b$ (\texttt{float*}).
\end{desclist}

\subsubsection*{Definition}
\begin{verbatim}
void sf_tridiagonal_solve (sf_tris slv /* solver object */, 
                           float* b /* in - right-hand side, out - solution */)
/*< invert the matrix >*/
{
    int k;
    ...
}
\end{verbatim}




\subsection{{sf\_tridiagonal\_close}}
This function frees the allocated space for the \texttt{slv} object.

\subsubsection*{Call}
\begin{verbatim}sf_tridiagonal_close (slv);\end{verbatim}

\subsubsection*{Definition}
\begin{verbatim}
void sf_tridiagonal_close (sf_tris slv)
/*< free allocated storage >*/
{
   ...
}
\end{verbatim}

\subsubsection*{Input parameters}
\begin{desclist}{\tt }{\quad}[\tt ]
   \setlength\itemsep{0pt}
   \item[slv] the solver object. Must be of type \texttt{sf\_tris}.
\end{desclist}





  % Tridiagonal matrix solver
   

\chapter{Interpolation}\label{sec:interpolation}
   \section{1-D interpolation (int1.c)}
\index{interpolation!1D}




\subsection{{sf\_int1\_init}}\label{sec:sf_int1_init}
Initializes the required variables and allocates the required space for 1D interpolation.

\subsubsection*{Call}
\begin{verbatim}sf_int1_init (coord, o1, d1, n1, interp, nf_in, nd_in);\end{verbatim}

\subsubsection*{Definition}
\begin{verbatim}
void  sf_int1_init (float* coord               /* cooordinates [nd] */, 
                 float o1, float d1, int n1 /* axis */, 
                 sf_interpolator interp     /* interpolation function */, 
                 int nf_in                  /* interpolator length */, 
                 int nd_in                  /* number of data points */)
/*< initialize >*/
{
   ...
}
\end{verbatim}

\subsubsection*{Input parameters}
\begin{desclist}{\tt }{\quad}[\tt coord]
   \setlength\itemsep{0pt}
   \item[coord]  coordinates (\texttt{float*}).  
   \item[o1]     origin of the axis (\texttt{float}).  
   \item[d1]     sampling of the axis (\texttt{float}).  
   \item[n1]     length of the axis (\texttt{float}).  
   \item[interp] interpolation function (\texttt{sf\_interpolator}).  
   \item[nf\_in] interpolator length (\texttt{int}).  
   \item[nd\_in] number of data points (\texttt{int}).  
\end{desclist}




\subsection{{sf\_int1\_lop}}
Applies the linear operator for interpolation.

\subsubsection*{Call}
\begin{verbatim}sf_int1_lop (adj, add, nm, ny, x, ord);\end{verbatim}

\subsubsection*{Definition}
\begin{verbatim}
void  sf_int1_lop (bool adj, bool add, int nm, int ny, float* x, float* ord)
/*< linear operator >*/
{ 
   ...
}
\end{verbatim}

\subsubsection*{Input parameters}
\begin{desclist}{\tt }{\quad}[\tt ord]
   \setlength\itemsep{0pt}
   \item[adj] a parameter to determine whether the output is \texttt{x} or \texttt{ord} (\texttt{bool}).
   \item[add] a parameter to determine whether the input needs to be zeroed (\texttt{bool}).
   \item[nm]  size of \texttt{x} (\texttt{int}).
   \item[ny]  size of \texttt{ord} (\texttt{int}).
   \item[x]   output or operator, depending on whether \texttt{adj} is true or false (\texttt{float*}).
   \item[ord] output or operator, depending on whether \texttt{adj} is true or false (\texttt{float*}).
\end{desclist}




\subsection{{sf\_cint1\_lop}}
Applies the complex \index{interpolation!complex data}linear operator for interpolation of complex data.

\subsubsection*{Call}
\begin{verbatim}sf_cint1_lop (adj, add, nm, ny, x, ord);\end{verbatim}

\subsubsection*{Definition}
\begin{verbatim}
void  sf_cint1_lop (bool adj, bool add, int nm, int ny, sf_complex* x, sf_comple
x* ord)
/*< linear operator for complex numbers >*/
{ 
   ...
}
\end{verbatim}

\subsubsection*{Input parameters}
\begin{desclist}{\tt }{\quad}[\tt ord]
   \setlength\itemsep{0pt}
   \item[adj] a parameter to determine whether the output is \texttt{x} or \texttt{ord} (\texttt{bool}).
   \item[add] a parameter to determine whether the input needs to be zeroed (\texttt{bool}).
   \item[nm]  size of \texttt{x} (\texttt{int}).
   \item[ny]  size of \texttt{ord} (\texttt{int}).
   \item[x]   output or operator, depending on whether \texttt{adj} is true or false (\texttt{sf\_complex*}).
   \item[ord] output or operator, depending on whether \texttt{adj} is true or false (\texttt{sf\_complex*}).
\end{desclist}




\subsection{{sf\_int1\_close}}
Frees the space allocated for 1D interpolation by \hyperref[sec:sf_int1_init]{\texttt{sf\_int1\_init}}.

\subsubsection*{Call}
\begin{verbatim}sf_int1_close ();\end{verbatim}

\subsubsection*{Definition}
\begin{verbatim}
void sf_int1_close (void)
/*< free allocated storage >*/
{
   ...
}
\end{verbatim}


        % 1-D interpolation
   \section{2-D interpolation (int2.c)}
\index{interpolation!2D}




\subsection{{sf\_int2\_init}}\label{sec:sf_int2_init}
Initializes the required variables and allocates the required space for 2D interpolation.

\subsubsection*{Call}
\begin{verbatim}sf_int2_init (coord, o1, o2, d1, d2, n1, n2, interp, nf_in, nd_in);\end{verbatim}

\subsubsection*{Definition}
\begin{verbatim}
void  sf_int2_init (float** coord          /* coordinates [nd][2] */, 
                    float o1, float o2, 
                    float d1, float d2,
                    int   n1, int   n2     /* axes */, 
                    sf_interpolator interp /* interpolation function */, 
                    int nf_in              /* interpolator length */, 
                    int nd_in              /* number of data points */)
/*< initialize >*/
{
   ...
}
\end{verbatim}

\subsubsection*{Input parameters}
\begin{desclist}{\tt }{\quad}[\tt interp]
   \setlength\itemsep{0pt}
   \item[coord]   coordinates (\texttt{float**}).  
   \item[o1]     origin of the first axis (\texttt{float}).  
   \item[o2]     origin of the second axis (\texttt{float}).  
   \item[d1]     sampling of the first axis (\texttt{float}).  
   \item[d2]     sampling of the second axis (\texttt{float}).  
   \item[n1]     length of the first axis (\texttt{float}).  
   \item[n2]     length of the second axis (\texttt{float}).  
   \item[interp] interpolation function (\texttt{sf\_interpolator}).  
   \item[nf\_in] interpolator length (\texttt{int}).  
   \item[nd\_in] number of data points (\texttt{int}).  
\end{desclist}




\subsection{{sf\_int2\_lop}}
Applies the linear operator for 2D interpolation.

\subsubsection*{Call}
\begin{verbatim}sf_int2_lop (adj, add, nm, ny, x, ord);\end{verbatim}

\subsubsection*{Definition}
\begin{verbatim}
void  sf_int2_lop (bool adj, bool add, int nm, int ny, float* x, float* ord)
/*< linear operator >*/
{ 
   ...
}
\end{verbatim}

\subsubsection*{Input parameters}
\begin{desclist}{\tt }{\quad}[\tt add]
   \setlength\itemsep{0pt}
   \item[adj] a parameter to determine whether the output is \texttt{x} or \texttt{ord} (\texttt{bool}).
   \item[add] a parameter to determine whether the input needs to be zeroed (\texttt{bool}).
   \item[nm]  size of \texttt{x} (\texttt{int}).
   \item[ny]  size of \texttt{ord} (\texttt{int}).
   \item[x]   output or operator, depending on whether \texttt{adj} is true or false (\texttt{float*}).
   \item[ord] output or operator, depending on whether \texttt{adj} is true or false (\texttt{float*}).
\end{desclist}




\subsection{{sf\_int2\_close}}
Frees the space allocated for 2D interpolation by \hyperref[sec:sf_int2_init]{\texttt{sf\_int2\_init}}.

\subsubsection*{Call}
\begin{verbatim}sf_int2_close();\end{verbatim}

\subsubsection*{Definition}
\begin{verbatim}
void sf_int2_close (void)
/*< free allocated storage >*/
{
   ...
}
\end{verbatim}

        % 2-D interpolation
   \section{3-D interpolation (int3.c)}
\index{interpolation!3D}




\subsection{{sf\_int3\_init}}\label{sec:sf_int3_init}
Initializes the required variables and allocates the required space for 3D interpolation.

\subsubsection*{Call}
\begin{verbatim}
sf_int3_init (coord, o1,o2,o3, d1,d2,d3, 
              n1,n2,n3, interp, nf_in, nd_in);
\end{verbatim}

\subsubsection*{Definition}
\begin{verbatim}
void  sf_int3_init (float** coord          /* coordinates [nd][3] */, 
                    float o1, float o2, float o3,
                    float d1, float d2, float d3,
                    int   n1, int   n2,   int n3 /* axes */, 
                    sf_interpolator interp /* interpolation function */, 
                    int nf_in              /* interpolator length */, 
                    int nd_in              /* number of data points */)
/*< initialize >*/
{
   ...
}
\end{verbatim}

\subsubsection*{Input parameters}
\begin{desclist}{\tt }{\quad}[\tt interp]
   \setlength\itemsep{0pt}
   \item[coord]  coordinates (\texttt{float**}).  
   \item[o1]     origin of the first axis (\texttt{float}).  
   \item[o2]     origin of the second axis (\texttt{float}).  
   \item[o3]     origin of the third axis (\texttt{float}).  
   \item[d1]     sampling of the first axis (\texttt{float}).  
   \item[d2]     sampling of the second axis (\texttt{float}).  
   \item[d3]     sampling of the third axis (\texttt{float}).  
   \item[n1]     length of the first axis (\texttt{float}).  
   \item[n2]     length of the second axis (\texttt{float}).  
   \item[n3]     length of the third axis (\texttt{float}).  
   \item[interp] interpolation function (\texttt{sf\_interpolator}).  
   \item[nf\_in] interpolator length (\texttt{int}).  
   \item[nd\_in] number of data points (\texttt{int}).  
\end{desclist}




\subsection{{sf\_int3\_lop}}
Applies the linear operator for 3D interpolation.

\subsubsection*{Call}
\begin{verbatim}sf_int3_lop (adj, add, nm, ny, mm, dd);\end{verbatim}

\subsubsection*{Definition}
\begin{verbatim}
void  sf_int3_lop (bool adj, bool add, int nm, int ny, float* mm, float* dd)
/*< linear operator >*/
{ 
   ...
}
\end{verbatim}

\subsubsection*{Input parameters}
\begin{desclist}{\tt }{\quad}[\tt add]
   \setlength\itemsep{0pt}
   \item[adj] a parameter to determine whether the output is \texttt{x} or \texttt{ord} (\texttt{bool}).
   \item[add] a parameter to determine whether the input needs to be zeroed (\texttt{bool}).
   \item[nm]  size of \texttt{x} (\texttt{int}).
   \item[ny]  size of \texttt{ord} (\texttt{int}).
   \item[x]   output or operator, depending on whether \texttt{adj} is true or false (\texttt{float*}).
   \item[ord] output or operator, depending on whether \texttt{adj} is true or false (\texttt{float*}).
\end{desclist}




\subsection{{sf\_int3\_close}}
Frees the space allocated for 3D interpolation by \hyperref[sec:sf_int3_init]{\texttt{sf\_int3\_init}}.

\subsubsection*{Call}
\begin{verbatim}int3_close ();\end{verbatim}

\subsubsection*{Definition}
\begin{verbatim}
void int3_close (void)
/*< free allocated storage >*/
{
   ...
}
\end{verbatim}

        % 3-D interpolation
   \section{Basic interpolation functions (interp.c)}
\index{interpolation!nearest neighbor}




\subsection{{sf\_bin\_int}}
Computes the nearest neighbor interpolation function coefficients.

\subsubsection*{Call}
\begin{verbatim}sf_bin_int (x, n, w);\end{verbatim}

\subsubsection*{Definition}
\begin{verbatim}
void sf_bin_int (float x, int n, float* w) 
/*< nearest neighbor >*/
{
   ...
}
\end{verbatim}

\subsubsection*{Input parameters}
\begin{desclist}{\tt }{\quad}[\tt ]
   \setlength\itemsep{0pt}
   \item[x] data (\texttt{float}).  
   \item[n] number of interpolation points (\texttt{int}).  
   \item[w] interpolation coefficients (\texttt{float*}).  
\end{desclist}




\subsection{{sf\_lin\_int}}
Computes the linear interpolation function coefficients.

\subsubsection*{Call}
\begin{verbatim}sf_lin_int (x, n, w);
\end{verbatim}

\subsubsection*{Definition}
\begin{verbatim}
void sf_lin_int (float x, int n, float* w) 
/*< linear >*/
{
   ...
}
\end{verbatim}

\subsubsection*{Input parameters}
\begin{desclist}{\tt }{\quad}[\tt ]
   \setlength\itemsep{0pt}
   \item[x] data (\texttt{float}).  
   \item[n] number of interpolation points (\texttt{int}).  
   \item[w] interpolation coefficients (\texttt{float*}).  
\end{desclist}




\subsection{{sf\_lg\_int}}
Computes the Lagrangian interpolation function coefficients.

\subsubsection*{Call}
\begin{verbatim}sf_lg_int (x, n, w);\end{verbatim}

\subsubsection*{Definition}
\begin{verbatim}
void sf_lg_int (float x, int n, float* w) 
/*< Lagrangian >*/
{
   ...
}
\end{verbatim}

\subsubsection*{Input parameters}
\begin{desclist}{\tt }{\quad}[\tt ]
   \setlength\itemsep{0pt}
   \item[x] data (\texttt{float}).  
   \item[n] number of interpolation points (\texttt{int}).  
   \item[w] interpolation coefficients (\texttt{float*}).  
\end{desclist}




\subsection{{sf\_taylor\_int}}
Computes the taylor interpolation function coefficients.

\subsubsection*{Call}
\begin{verbatim}sf_taylor (x, n, w);\end{verbatim}

\subsubsection*{Definition}
\begin{verbatim}
void sf_taylor (float x, int n, float* w) 
/*< Taylor >*/
{
   ...
}
\end{verbatim}

\subsubsection*{Input parameters}
\begin{desclist}{\tt }{\quad}[\tt ]
   \setlength\itemsep{0pt}
   \item[x] data (\texttt{float}).  
   \item[n] number of interpolation points (\texttt{int}).  
   \item[w] interpolation coefficients (\texttt{float*}).  
\end{desclist}




      % Basic interpolation functions
   \section{Convert data to B-spline coefficients by fast B-spline transform (prefilter.c)}




\subsection{{sf\_prefilter\_init}}\label{sec:sf_prefilter_init}
Initializes the pre-filter for spline interpolation by initializing the required variables and allocating the required space.

\subsubsection*{Call}
\begin{verbatim}sf_prefilter_init (nw, nt_in, pad_in);\end{verbatim}

\subsubsection*{Definition}
\begin{verbatim}
void sf_prefilter_init (int nw     /* spline order */, 
                        int nt_in  /* temporary storage length */, 
                        int pad_in /* padding */)
/*< initialize >*/
{
   ...
}
\end{verbatim}

\subsubsection*{Input parameters}
\begin{desclist}{\tt }{\quad}[\tt pad\_in]
   \setlength\itemsep{0pt}
   \item[nw]      order of the spline (\texttt{int}). 
   \item[nt\_in]  length of the temporary storage (\texttt{int}). 
   \item[pad\_in] length of the padding required (\texttt{int}).  
\end{desclist}




\subsection{{sf\_prefilter\_apply}}
Applies the pre-filter to the input 1D data to convert it to the spline coefficients.

\subsubsection*{Call}
\begin{verbatim}sf_prefilter_apply (nd, dat);\end{verbatim}

\subsubsection*{Definition}
\begin{verbatim}
void sf_prefilter_apply (int nd     /* data length */, 
                         float* dat /* in - data, out - coefficients */)
/*< Convert 1-D data to spline coefficients >*/
{
   ...
}
\end{verbatim}

\subsubsection*{Input parameters}
\begin{desclist}{\tt }{\quad}[\tt ndat]
   \setlength\itemsep{0pt}
   \item[nd] length of the input data (\texttt{int}). 
   \item[dat] the input data, which is converted to spline coefficients as output (\texttt{float*}).  
\end{desclist}




\subsection{{sf\_prefilter}}
Applies the pre-filter to the input N dimensional data to convert it to the spline coefficients.

\subsubsection*{Call}
\begin{verbatim}sf_prefilter (dim, n, dat);\end{verbatim}

\subsubsection*{Definition}
\begin{verbatim}
void sf_prefilter (int dim    /* number of dimensions */, 
                   int* n     /* data size [dim] */, 
                   float* dat /* in - data, out - coefficients */)
/*< Convert N-D data to spline coefficients >*/
{
   ...
}
\end{verbatim}

\subsubsection*{Input parameters}
\begin{desclist}{\tt }{\quad}[\tt dat]
   \setlength\itemsep{0pt}
   \item[dim] number of dimensions in the input data (\texttt{int}). 
   \item[n] size of the input data (\texttt{int*}).  
   \item[dat] the input data, which is converted to spline coefficients as output (\texttt{float*}).  
\end{desclist}




\subsection{{sf\_prefilter\_close}}
Frees the space allocated for the pre-filer by \hyperref[sec:sf_prefilter_init]{\texttt{sf\_prefilter\_init}}.

\subsubsection*{Call}
\begin{verbatim}sf_prefilter_close();\end{verbatim}

\subsubsection*{Definition}
\begin{verbatim}
void sf_prefilter_close( void)
/*< free allocated storage >*/
{
   ...
}
\end{verbatim}
      




   % Convert data to B-spline coefficients by fast B-spline transform
   \section{B-spline interpolation (spline.c)}




\subsection{{sf\_spline\_init}}
Initializes and defines a banded matrix for spline interpolation.


\subsubsection*{Call}
\begin{verbatim}slv = sf_spline_init (nw, nd);\end{verbatim}


\subsubsection*{Definition}
\begin{verbatim}
sf_bands sf_spline_init (int nw /* interpolator length */, 
                         int nd /* data length */)
/*< initialize a banded matrix >*/
{
   ...
}
\end{verbatim}


\subsubsection*{Input parameters}
\begin{desclist}{\tt }{\quad}[\tt nw]
   \setlength\itemsep{0pt}
   \item[nw] length of the interpolator (\texttt{int}). 
   \item[nd] length of the data (\texttt{int}).  
\end{desclist}

\subsubsection*{Output}
\begin{desclist}{\tt }{\quad}[\tt ]
   \setlength\itemsep{0pt}
   \item[slv] an object of type \texttt{sf\_band}. It is of type \texttt{sf\_band}.
\end{desclist}



\subsection{{sf\_spline4\_init}}
Initializes and defines a tridiagonal matrix for cubic spline interpolation.

\subsubsection*{Call}
\begin{verbatim}slv = sf_spline4_init(nd);\end{verbatim}

\subsubsection*{Definition}
\begin{verbatim}
sf_tris sf_spline4_init (int nd /* data length */)
/*< initialize a tridiagonal matrix for cubic splines >*/
{
   ...
}
\end{verbatim}

\subsubsection*{Input parameters}
\begin{desclist}{\tt }{\quad}[\tt ]
   \setlength\itemsep{0pt}
   \item[nd] length of the data (\texttt{int}).  
\end{desclist}

\subsubsection*{Output}
\begin{desclist}{\tt }{\quad}[\tt ]
   \setlength\itemsep{0pt}
   \item[slv] an object of type \texttt{sf\_tri}. It is of type \texttt{sf\_tri}.
\end{desclist}




\subsection{{sf\_spline4\_post}}
Performs the cubic spline post filtering.

\subsubsection*{Call}
\begin{verbatim}sf_spline4_post (n, n1, n2, inp, out);\end{verbatim}

\subsubsection*{Definition}
\begin{verbatim}
void sf_spline4_post (int n            /* total trace length */, 
                      int n1           /* start point */, 
                      int n2           /* end point */, 
                      const float* inp /* spline coefficients */, 
                      float* out       /* function values */)
/*< cubic spline post-filtering >*/
{
   ...
}
\end{verbatim}

\subsubsection*{Input parameters}
\begin{desclist}{\tt }{\quad}[\tt out]
   \setlength\itemsep{0pt}
   \item[n]   total length of the trace (\texttt{int}). 
   \item[n1]  start point (\texttt{int}).  
   \item[n2]  end point (\texttt{int}). 
   \item[inp] spline coefficients (\texttt{const float*}).
   \item[out] function values (\texttt{float*}).
\end{desclist}




\subsection{{sf\_spline\_post}}
Performs the post filtering to convert spline coefficients to model.


\subsubsection*{Call}
\begin{verbatim}sf_spline_post(nw, o, d, n, modl, datr);\end{verbatim}

\subsubsection*{Definition}
\begin{verbatim}
void sf_spline_post (int nw, int o, int d, int n, 
                     const float *modl, float *datr)
/*< post-filtering to convert spline coefficients to model >*/
{
   ...
}
\end{verbatim}

\subsubsection*{Input parameters}
\begin{desclist}{\tt }{\quad}[\tt datr]
   \setlength\itemsep{0pt}
   \item[nw]   length of the interpolator (\texttt{int}). 
   \item[o]    start point (\texttt{int}).  
   \item[d]    step size (\texttt{int}). 
   \item[n]    total length of the trace (\texttt{int}). 
   \item[modl] spline coefficients, which have to be converted to model coefficients.  Must be of type \texttt{const float*}.
   \item[datr] model, it is the output (\texttt{float*}).
\end{desclist}




\subsection{{sf\_spline2}}
Performs pre-filtering for spline interpolation for 2D data.

\subsubsection*{call}
\begin{verbatim}sf_spline2 (slv1, slv2, n1, n2, dat, tmp);\end{verbatim}

\subsubsection*{Definition}
\begin{verbatim}
void sf_spline2 (sf_bands slv1, sf_bands slv2, 
                 int n1, int n2, float** dat, float* tmp)
/*< 2-D spline pre-filtering >*/
{
   ...
}
\end{verbatim}

\subsubsection*{Input parameters}
\begin{desclist}{\tt }{\quad}[\tt slv2]
   \setlength\itemsep{0pt}
   \item[slv1] first banded matrix. Must be of type \texttt{sf\_band}. 
   \item[slv2] second banded matrix. Must be of type \texttt{sf\_band}. 
   \item[n1]   data length on the first axis (\texttt{int}).  
   \item[n2]	  data length on the second axis (\texttt{int}). 
   \item[dat]  2D data. Must be of type \texttt{const float*}*.
   \item[tmp]  temporary arrays for calculation (\texttt{float*}).
\end{desclist}





      % B-spline interpolation
   \section{Inverse linear interpolation (stretch.c)}




\subsection{{sf\_stretch\_init}}
Initializes the object of the abstract data of type \texttt{sf\_map}, which will be used to define and transform (stretch) coordinates.

\subsubsection*{Call}
\begin{verbatim}sf_map sf_stretch_init (n1, o1, d1, nd, eps, narrow);\end{verbatim}

\subsubsection*{Definition}
\begin{verbatim}
sf_map sf_stretch_init (int n1, float o1, float d1 /* regular axis */, 
                        int nd                     /* data length */, 
                        float eps                  /* regularization */, 
                        bool narrow                /* if zero boundary */)
/*< initialize >*/
{
   ...
}
\end{verbatim}

\subsubsection*{Input parameters}
\begin{desclist}{\tt }{\quad}[\tt narrow]
   \setlength\itemsep{0pt}
   \item[n1]	     axis (\texttt{int}).
   \item[o1]     first sample on the axis (\texttt{int}).
   \item[d1]     step length to access the sample on the same axis (\texttt{int}).
   \item[eps]    regularizaton (\texttt{float}). 
   \item[narrow] is boundary value zero or not (\texttt{bool}).
\end{desclist}

\subsubsection*{Output}
\begin{desclist}{\tt }{\quad}[\tt ]
   \setlength\itemsep{0pt}
   \item[str] the \texttt{sf\_map} object. It is of type \texttt{sf\_map}.
\end{desclist}




\subsection{{sf\_stretch\_define}}
Defines the coordinates for mapping (which in this case is stretching. That is, it fills the required variables in the \texttt{sf\_map} object to map the input coordinates.


\subsubsection*{Call}
\begin{verbatim}sf_stretch_define (str, coord);\end{verbatim}

\subsubsection*{Definition}
\begin{verbatim}
void sf_stretch_define (sf_map str, const float* coord)
/*< define coordinates >*/
{
   ...
}
\end{verbatim}

\subsubsection*{Input parameters}
\begin{desclist}{\tt }{\quad}[\tt cool]
   \setlength\itemsep{0pt}
   \item[str] the \texttt{sf\_map} object. Must be of type \texttt{sf\_map}. 
   \item[coor] input coordinates (\texttt{const float}).
\end{desclist}




\subsection{{sf\_stretch\_apply}}\label{sec:sf_stretch_apply}
Converts the ordinates (\texttt{ord}) defined in the input to model (\texttt{mod}). It uses the \hyperref[sec:sf_tridiagonal_solve]{\texttt{sf\_tridiagonal\_solve}} function.

\subsubsection*{Call}
\begin{verbatim}sf_stretch_apply (str, ord, mod);\end{verbatim}

\subsubsection*{Definition}
\begin{verbatim}
void sf_stretch_apply (sf_map str, const float* ord, float* mod)
/*< convert ordinates to model >*/
{
   ...
}
\end{verbatim}

\subsubsection*{Input parameters}
\begin{desclist}{\tt }{\quad}[\tt ord]
   \setlength\itemsep{0pt}
   \item[str] the \texttt{sf\_map} object. Must be of type \texttt{sf\_map}. 
   \item[ord] input ordinates (\texttt{const float*}).
   \item[mod] model (\texttt{const float*}).
\end{desclist}




\subsection{{sf\_stretch\_invert}}
Converts model (mod) to ordinates by linear interpolation. It is the inverse of \hyperref[sec:sf_stretch_apply]{\texttt{sf\_stretch\_apply}}.

\subsubsection*{Call}
\begin{verbatim}sf_stretch_invert (str, ord, mod);\end{verbatim}

\subsubsection*{Definition}
\begin{verbatim}
void sf_stretch_invert (sf_map str, float* ord, const float* mod)
/*< convert model to ordinates by linear interpolation >*/
{
   ...
}
\end{verbatim}

\subsubsection*{Input parameters}
\begin{desclist}{\tt }{\quad}[\tt ord]
   \setlength\itemsep{0pt}
   \item[str] the \texttt{sf\_map} object. Must be of type \texttt{sf\_map}. 
   \item[ord] input ordinates (\texttt{const float*}).
   \item[mod] model (\texttt{const float*}).
\end{desclist}




\subsection{{sf\_stretch\_close}}
This function frees the allocated space for the \texttt{sf\_map} object.

\subsubsection*{Call}
\begin{verbatim}sf_stretch_close (str);\end{verbatim}

\subsubsection*{Definition}
\begin{verbatim}
void sf_stretch_close (sf_map str)
/*< free allocated storage >*/
{
   ...
}
\end{verbatim}

\subsubsection*{Input parameters}
\begin{desclist}{\tt }{\quad}[\tt ]
   \setlength\itemsep{0pt}
   \item[str] the \texttt{sf\_map} object. Must be of type \texttt{sf\_map}.
\end{desclist}





     % Inverse linear interpolation
   \section{1-D ENO interpolation (eno.c)}\label{sec:eno.c}




\subsection{{sf\_eno\_init}}\label{sec:sf_eno_init}
Initializes an object of type \texttt{sf\_eno} for interpolation.

\subsubsection*{Call}
\begin{verbatim}ent = sf_eno_init (order, n);\end{verbatim}

\subsubsection*{Definition}
\begin{verbatim}
sf_eno sf_eno_init (int order /* interpolation order */, 
              int n     /* data size */)
/*< Initialize interpolation object. >*/
{
   ...
}
\end{verbatim}

\subsubsection*{Input parameters}
\begin{desclist}{\tt }{\quad}[\tt order]
   \setlength\itemsep{0pt}
   \item[order] order of interpolation (\texttt{int}).  
   \item[n]     size of the data (\texttt{int}).  
\end{desclist}

\subsubsection*{Output}
\begin{desclist}{\tt }{\quad}[\tt ]
   \setlength\itemsep{0pt}  
   \item[ent] an object for interpolation. It is of type \texttt{sf\_eno}.
\end{desclist}




\subsection{{sf\_eno\_close}}
Frees the space allocated for the internal storage by \hyperref[sec:sf_eno_init]{\texttt{sf\_eno\_init}}.

\subsubsection*{Call}
\begin{verbatim}sf_eno_close (ent);\end{verbatim}

\subsubsection*{Definition}
\begin{verbatim}
void sf_eno_close (sf_eno ent)
/*< Free internal storage >*/
{
   ...
}
\end{verbatim}




\subsection{{sf\_eno\_set}}
Creates a table for interpolation.

\subsubsection*{Call}
\begin{verbatim}sf_eno_set (ent, c);\end{verbatim}

\subsubsection*{Definition}
\begin{verbatim}
void sf_eno_set (sf_eno ent, float* c /* data [n] */)
/*< Set the interpolation table. c can be changed or freed afterwords >*/
{
   ...
}
\end{verbatim}

\subsubsection*{Input parameters}
\begin{desclist}{\tt }{\quad}[\tt ent]
   \setlength\itemsep{0pt}
   \item[ent] an object for interpolation. It is of type \texttt{sf\_eno}.
   \item[c]   the data which is to be interpolated (\texttt{float*}).  
\end{desclist}




\subsection{{sf\_eno\_apply}}
Interpolates the data.

\subsubsection*{Call}
\begin{verbatim}sf_eno_apply (ent, i, x, f, f1, what);\end{verbatim}

\subsubsection*{Definition}
\begin{verbatim}
void sf_eno_apply (sf_eno ent, 
                int i     /* grid location */, 
                float x   /* offset from grid */, 
                float *f  /* output data value */, 
                float *f1 /* output derivative */, 
                der what  /* flag of what to compute */) 
/*< Apply interpolation >*/
{
  ...
}
\end{verbatim}

\subsubsection*{Input parameters}
\begin{desclist}{\tt }{\quad}[\tt ]
   \setlength\itemsep{0pt}
   \item[ent]  an object for interpolation. It is of type \texttt{sf\_eno}.
   \item[i]    location of the grid (\texttt{int}).
   \item[x]    offset from the grid (\texttt{float}).  
   \item[f]    output data value (\texttt{float*}).  
   \item[f1]   output derivative (\texttt{float*}).  
   \item[what] whether the function value or the derivative is required. Must be of type \texttt{der}.  
\end{desclist}

         % 1-D ENO interpolation
   \section{ENO interpolation in 2-D (eno2.c)}\label{sec:eno2.c}




\subsection{{sf\_eno2\_init}}\label{sec:sf_eno2_init}
Initializes an object of type \texttt{sf\_eno}2 for interpolation of 2D data.

\subsubsection*{Call}
\begin{verbatim}pnt =  sf_eno2_init (order, n1, n2);\end{verbatim}

\subsubsection*{Definition}
\begin{verbatim}
sf_eno2 sf_eno2_init (int order      /* interpolation order */, 
                      int n1, int n2 /* data dimensions */)
/*< Initialize interpolation object >*/
{
    sf_eno2 pnt;
    int i2;
    
   ...
    return pnt;
}
\end{verbatim}

\subsubsection*{Input parameters}
\begin{desclist}{\tt }{\quad}[\tt order]
   \setlength\itemsep{0pt}
   \item[order] interpolation order (\texttt{int}).  
   \item[n1]    first dimension of the data (\texttt{int}).  
   \item[n2]    second dimension of the data (\texttt{int}).  
\end{desclist}

\subsubsection*{Output}
\begin{desclist}{\tt }{\quad}[\tt ]
   \setlength\itemsep{0pt}  
   \item[pnt] object for interpolation. It is of type \texttt{sf\_eno}2.
\end{desclist}




\subsection{{sf\_eno2\_set}}
Sets the interpolation table for the 2D data in a 2D array.

\subsubsection*{Call}
\begin{verbatim}sf_eno2_set (pnt, c);\end{verbatim}

\subsubsection*{Definition}
\begin{verbatim}
void sf_eno2_set (sf_eno2 pnt, float** c /* data [n2][n1] */)
/*< Set the interpolation table. c can be changed or freed afterwords. >*/
{
   ...    
}
\end{verbatim}

\subsubsection*{Input parameters}
\begin{desclist}{\tt }{\quad}[\tt pnt]
   \setlength\itemsep{0pt}
   \item[pnt] object for interpolation. It is of type \texttt{sf\_eno}2.
   \item[c]   the data (\texttt{float**}).  
\end{desclist}





\subsection{{sf\_eno2\_set1}}
Sets the interpolation table for the 2D data in a 1D array, which is of size \texttt{n1*n2}.

\subsubsection*{Call}
\begin{verbatim}sf_eno2_set1 (pnt, c);\end{verbatim}

\subsubsection*{Definition}
\begin{verbatim}
void sf_eno2_set1 (sf_eno2 pnt, float* c /* data [n2*n1] */)
/*< Set the interpolation table. c can be changed or freed afterwords. >*/
{
   ...
}
\end{verbatim}

\subsubsection*{Input parameters}
\begin{desclist}{\tt }{\quad}[\tt pnt]
   \setlength\itemsep{0pt}
   \item[pnt] object for interpolation. It is of type \texttt{sf\_eno}2.
   \item[c]   the data (\texttt{float*}).  
\end{desclist}




\subsection{{sf\_eno2\_close}}
Frees the space allocated for the internal storage by \hyperref[sec:sf_eno2_init]{\texttt{sf\_eno2\_init}}.

\subsubsection*{Call}
\begin{verbatim}sf_eno2_close (pnt);\end{verbatim}

\subsubsection*{Definition}
\begin{verbatim}
void sf_eno2_close (sf_eno2 pnt)
/*< Free internal storage >*/
{
   ...
}
\end{verbatim}




\subsection{{sf\_eno2\_apply}}
Interpolates the 2D data.

\subsubsection*{Call}
\begin{verbatim}sf_eno2_apply (pnt, i, j, x, y, f, f1, what);\end{verbatim}

\subsubsection*{Definition}
\begin{verbatim}
void sf_eno2_apply (sf_eno2 pnt, 
                    int i, int j     /* grid location */, 
                    float x, float y /* offset from grid */, 
                    float* f         /* output data value */, 
                    float* f1        /* output derivative [2] */,
                    der what         /* what to compute [FUNC,DER,BOTH] */)
/*< Apply interpolation. >*/
{
   ...
}
\end{verbatim}

\subsubsection*{Input parameters}
\begin{desclist}{\tt }{\quad}[\tt ]
   \setlength\itemsep{0pt}
   \item[pnt]  an object for interpolation. It is of type \texttt{sf\_eno}2.
   \item[i]    location of the grid for first dimension (\texttt{int}).
   \item[j]    location of the grid for second dimension (\texttt{int}).
   \item[x]    offset from the grid for the first dimension (\texttt{float}).  
   \item[y]    offset from the grid for the second dimension (\texttt{float}).  
   \item[f]    output data value (\texttt{float*}).  
   \item[f1]   output derivative (\texttt{float*}).  
   \item[what] whether the function value or the derivative or both are required. Must be of type \texttt{der}.  
\end{desclist}


        % ENO interpolation in 2-D
   \section{1-D ENO power-p interpolation (pweno.c)}




\subsection{{sf\_pweno\_init}}
Initializes an object of type \texttt{sf\_pweno}.

\subsubsection*{Call}
\begin{verbatim}ent = sf_pweno sf_pweno_init (int order, n);\end{verbatim}

\subsubsection*{Definition}
\begin{verbatim}
sf_pweno sf_pweno_init (int order /* interpolation order */,
                        int n     /* data size */)
/*< Initialize interpolation object >*/
{
   ...
}
\end{verbatim}

\subsubsection*{Input parameters}
\begin{desclist}{\tt }{\quad}[\tt order]
   \setlength\itemsep{0pt}
   \item[order] order of interpolation (\texttt{int}). 
   \item[n] size of the data (\texttt{int}).  
\end{desclist}

\subsubsection*{Output}
\begin{desclist}{\tt }{\quad}[\tt ]
   \setlength\itemsep{0pt}
   \item[ent] object of type \texttt{sf\_pweno}.
\end{desclist}




\subsection{{sf\_pweno\_close}}
Frees the space allocated for the \texttt{sf\_pweno} object by \texttt{sf\_pweno\_init}.

\subsubsection*{Call}
\begin{verbatim}sf_pweno_close (ent);\end{verbatim}

\subsubsection*{Definition}
\begin{verbatim}
void sf_pweno_close (sf_pweno ent)
/*< Free internal storage >*/
{
   ...
}
\end{verbatim}

\subsubsection*{Input parameters}
\begin{desclist}{\tt }{\quad}[\tt ]
   \setlength\itemsep{0pt}
   \item[ent] object of type \texttt{sf\_pweno}.
\end{desclist}




\subsection{{powerpeno}}
Calculates the Power-p limiter for eno method using the input numbers \texttt{x} and \texttt{y}.

\subsubsection*{Call}
\begin{verbatim}power = powerpeno (x, y, p);\end{verbatim}

\subsubsection*{Definition}
\begin{verbatim}
float powerpeno (float x, float y, int p /* power order */)
/*< Limiter power-p eno >*/
{
   ...
}
\end{verbatim}

\subsubsection*{Input parameters}
\begin{desclist}{\tt }{\quad}[\tt ]
   \setlength\itemsep{0pt}
   \item[x] an input number (\texttt{float}). 
   \item[y] an input number (\texttt{float}). 
   \item[p] power order (\texttt{int}).  
\end{desclist}

\subsubsection*{Output}
\begin{desclist}{\tt }{\quad}[\tt ]
   \setlength\itemsep{0pt}
   \item[mins * power] limiter power-p. It is of type \texttt{float}.
\end{desclist}




\subsection{{sf\_pweno\_set}}
Sets the interpolation undivided difference table.

\subsubsection*{Call}
\begin{verbatim}void sf_pweno_set (sf_pweno ent, float* c /* data [n] */, int p);\end{verbatim}

\subsubsection*{Definition}
\begin{verbatim}
void sf_pweno_set (sf_pweno ent, float* c /* data [n] */, int p /* power order */)
/*< Set the interpolation undivided difference table. c can be changed or freed 
afterwards >*/
{
   ...
}
\end{verbatim}

\subsubsection*{Input parameters}
\begin{desclist}{\tt }{\quad}[\tt ent]
   \setlength\itemsep{0pt}
   \item[ent] the interpolation object. Must be of type \texttt{sf\_pweno}. 
   \item[c] input data (\texttt{float*}).  
   \item[p] power order (\texttt{int}).  
\end{desclist}




\subsection{{sf\_pweno\_apply}}
Applies the interpolation.

\subsubsection*{Call}
\begin{verbatim}sf_pweno_apply (ent, i, x, f, f1, what);\end{verbatim}

\subsubsection*{Definition}
\begin{verbatim}
void sf_pweno_apply (sf_pweno ent, 
                int i     /* grid location */, 
                float x   /* offset from grid */, 
                float *f  /* output data value */, 
                float *f1 /* output derivative */, 
                derr what /* flag of what to compute */) 
/*< Apply interpolation >*/
{
   ...
}
\end{verbatim}

\subsubsection*{Input parameters}
\begin{desclist}{\tt }{\quad}[\tt what]
   \setlength\itemsep{0pt}
   \item[i]    location of the grid (\texttt{int}).  
   \item[x]    offset from the grid (\texttt{float}). 
   \item[f]    output data value (\texttt{float*}).  
   \item[f1]   output derivative (\texttt{float*}).  
   \item[what] flag of what to compute. Must be of type \texttt{derr}.  
\end{desclist}





       % 1-D ENO power-p interpolation

\chapter{Smoothing}\label{sec:smoothing}
   \section{1-D triangle smoothing as a linear operator (triangle1.c)}




\subsection{{sf\_triangle1\_init}}
Initializes the triangle filter.

\subsubsection*{Call}
\begin{verbatim}sf_triangle1_init (nbox, ndat);\end{verbatim}

\subsubsection*{Definition}
\begin{verbatim}
void sf_triangle1_init (int nbox /* triangle size */, 
                        int ndat /* data size */)
/*< initialize >*/
{
   ...
}
\end{verbatim}

\subsubsection*{Input parameters}
\begin{desclist}{\tt }{\quad}[\tt inbox]
   \setlength\itemsep{0pt}
   \item[inbox] size of the triangle filter (\texttt{int}). 
   \item[ndat]  size of the data (\texttt{int}).
\end{desclist}




\subsection{{sf\_triangle1\_lop}}\label{sec:sf_triangle1_lop}
Applies the triangle smoothing to one of the input data and applies the smoothed data to the unsmoothed one as a linear operator.


\subsubsection*{Call}
\begin{verbatim}sf_triangle1_lop (adj, add, nx, ny, x, y);\end{verbatim}

\subsubsection*{Definition}
\begin{verbatim}
void sf_triangle1_lop (bool adj, bool add, int nx, int ny, float* x, float* y)
/*< linear operator >*/
{
   ...  
}
\end{verbatim}

\subsubsection*{Input parameters}
\begin{desclist}{\tt }{\quad}[\tt add]
   \setlength\itemsep{0pt}
   \item[adj] a parameter to determine whether weights are applied to \texttt{yy} or \texttt{xx} (\texttt{bool}). 
   \item[add] a parameter to determine whether the input needs to be zeroed (\texttt{bool}).
   \item[nx]  size of \texttt{x} (\texttt{int}). 
   \item[ny]  size of \texttt{y} (\texttt{int}).
   \item[x]   data or operator, depending on whether \texttt{adj} is true or false (\texttt{float}). 
   \item[y]   data or operator, depending on whether \texttt{adj} is true or false.  Must be of type \texttt{float}.
\end{desclist}

\subsubsection*{Output}
\begin{desclist}{\tt }{\quad}[\tt ]
   \setlength\itemsep{0pt}
   \item[\texttt{x} or \texttt{y}] the output depending on whether \texttt{adj} is true or false (\texttt{float}).
\end{desclist}




\subsection{{sf\_triangle1\_close}}
Frees the space allocated for the triangle smoothing filter.

\subsubsection*{Call}
\begin{verbatim}sf_triangle1_close();\end{verbatim}

\subsubsection*{Definition}
\begin{verbatim}
void sf_triangle1_close(void)
/*< free allocated storage >*/
{
   ...
}
\end{verbatim}





   % 1-D triangle smoothing as a linear operator
   \section{2-D triangle smoothing as a linear operator (triangle2.c)}




\subsection{{sf\_triangle2\_init}}
Initializes the triangle filter.

\subsubsection*{Call}
\begin{verbatim}sf_triangle2_init (nbox1,nbox2, ndat1,ndat2, nrep);\end{verbatim}

\subsubsection*{Definition}
\begin{verbatim}
void sf_triangle2_init (int nbox1, int nbox2 /* triangle size */, 
                        int ndat1, int ndat2 /* data size */,
                        int nrep /* repeat smoothing */)
/*< initialize >*/
{
   ...
}
\end{verbatim}

\subsubsection*{Input parameters}
\begin{desclist}{\tt }{\quad}[\tt inbox2]
   \setlength\itemsep{0pt}
   \item[inbox1] size of the triangle filter (\texttt{int}). 
   \item[inbox2] size of the second triangle filter (\texttt{int}). 
   \item[ndat1]  size of the data (\texttt{int}). 
   \item[ndat2]  size of the second data set (\texttt{int}). 
   \item[nrep]   number of times the smoothing is to be repeated (\texttt{int}).
\end{desclist}




\subsection{{sf\_triangle2\_lop}}
Applies the triangle smoothing to one of the input data and applies the smoothed data to the unsmoothed one as a linear operator. This is just like \hyperref[sec:sf_triangle1_lop]{\texttt{sf\_triangle1\_lop}} but with two triangle filters instead of one.

\subsubsection*{Call}
\begin{verbatim}sf_triangle2_lop (adj, add, nx, ny, x, y);\end{verbatim}

\subsubsection*{Definition}
\begin{verbatim}
void sf_triangle2_lop (bool adj, bool add, int nx, int ny, float* x, float* y)
/*< linear operator >*/
{
   ...
}
\end{verbatim}

\subsubsection*{Input parameters}
\begin{desclist}{\tt }{\quad}[\tt add]
   \setlength\itemsep{0pt}
   \item[adj] a parameter to determine whether weights are applied to \texttt{yy} or \texttt{xx} (\texttt{bool}). 
   \item[add] a parameter to determine whether the input needs to be zeroed (\texttt{bool}).
   \item[nx]  size of \texttt{x} (\texttt{int}). 
   \item[ny]  size of \texttt{y} (\texttt{int}).
   \item[x]   data or operator, depending on whether \texttt{adj} is true or false (\texttt{float}).
   \item[y]   data or operator, depending on whether \texttt{adj} is true or false (\texttt{float}).
\end{desclist}

\subsubsection*{Output}
\begin{desclist}{\tt }{\quad}[\tt ]
   \setlength\itemsep{0pt}
   \item[\texttt{x} or \texttt{y}] the output depending on whether \texttt{adj} is true or false (\texttt{float}).
\end{desclist}




\subsection{{sf\_triangle2\_close}}
Frees the space allocated for the triangle smoothing filters.

\subsubsection*{Call}
\begin{verbatim}sf_triangle2_close();\end{verbatim}

\subsubsection*{Definition}
\begin{verbatim}
void sf_triangle2_close(void)
/*< free allocated storage >*/
{
   ...
}
\end{verbatim}





   % 2-D triangle smoothing as a linear operator
   \section{Triangle smoothing (triangle.c)}\label{sec:triangle.c}




\subsection{{sf\_triangle\_init}}
Initializes the triangle smoothing filter.

\subsubsection*{Call}
\begin{verbatim}tr = sf_triangle_init (nbox, ndat);\end{verbatim}

\subsubsection*{Definition}
\begin{verbatim}
sf_triangle sf_triangle_init (int nbox /* triangle length */, 
                              int ndat /* data length */)
/*< initialize >*/
{
   ...
}
\end{verbatim}

\subsubsection*{Input parameters}
\begin{desclist}{\tt }{\quad}[\tt nbox]
   \setlength\itemsep{0pt}
   \item[nbox] an integer which specifies the length of the filter (\texttt{int}). 
   \item[ndat] an integer which specifies the length of the data (\texttt{int}).  
\end{desclist}

\subsubsection*{Output}
\begin{desclist}{\tt }{\quad}[\tt ]
   \setlength\itemsep{0pt}
   \item[tr] the triangle smoothing filter. It is of type \texttt{sf\_triangle}.
\end{desclist}




\subsection{{fold}}\label{sec:fold}
Folds the edges of the smoothed data, because when the data is convolved with the data, the length of the data increases and in most cases it is required that the smoothed data is of the same length as the input data.

\subsubsection*{Call}
\begin{verbatim}fold (o, d, nx, nb, np, x, tmp);\end{verbatim}

\subsubsection*{Definition}
\begin{verbatim}
static void fold (int o, int d, int nx, int nb, int np, 
                  const float *x, float* tmp)
{
   ...
}
\end{verbatim}

\subsubsection*{Input parameters}
\begin{desclist}{\tt }{\quad}[\tt tmp]
   \setlength\itemsep{0pt}
   \item[o]   first indices of the input data (\texttt{int}).  
   \item[d]   step size (\texttt{int}).
   \item[nx]  data length (\texttt{int}).        
   \item[nb]  filter length (\texttt{int}).        
   \item[np]  length of the tmp array in the \texttt{sf\_Triangle} data structure (\texttt{int}).
   \item[x]   a pointer to the input data (\texttt{const float}).  
   \item[tmp] a pointer to an array in the \texttt{sf\_Triangle} data structure. Must be of type \texttt{float}
\end{desclist}




\subsection{{fold2}}\label{sec:fold2}
Is the same as \hyperref[sec:fold]{\texttt{fold}} except for the fact that it copies from tmp to data, unlike the fold which does it the other way round.

\subsubsection*{Call}
\begin{verbatim}fold2 (o, d, nx, nb, np, x, tmp);\end{verbatim}

\subsubsection*{Definition}
\begin{verbatim}
static void fold2 (int o, int d, int nx, int nb, int np, 
                  float *x, const float* tmp)
{
   ...
}
\end{verbatim}

\subsubsection*{Input parameters}
\begin{desclist}{\tt }{\quad}[\tt tmp]
   \setlength\itemsep{0pt}
   \item[o]  first indices of the input data (\texttt{int}).  
   \item[d]  step size (\texttt{int}).
   \item[nx]	 data length (\texttt{int}).        
   \item[nb]	 filter length (\texttt{int}).        
   \item[np]  length of the tmp array in the \texttt{sf\_Triangle} data structure. Must be of type \texttt{int}
   \item[x]   a pointer to the input data (\texttt{const float}).  
   \item[tmp] a pointer to an array in the \texttt{sf\_Triangle} data structure (\texttt{float}).
\end{desclist}




\subsection{{doubint}}\label{sec:doubint}
Integrates the input data first in the backward direction and then if the input variable \texttt{der} is true it integrates the result forward.   

\subsubsection*{Call}
\begin{verbatim}doubint (nx, xx, der);\end{verbatim}

\subsubsection*{Definition}
\begin{verbatim}
static void doubint (int nx, float *xx, bool der)
{
   ...
}
\end{verbatim}

\subsubsection*{Input parameters}
\begin{desclist}{\tt }{\quad}[\tt der]
   \setlength\itemsep{0pt}
   \item[nx]  data length (\texttt{int}). 
   \item[xx]  a pointer to the input data (\texttt{const float}). 
   \item[der] a parameter to specify whether forward integration is required or not (\texttt{const float}).  
\end{desclist}




\subsection{{doubint2}}\label{sec:doubint2}
Unlike the \hyperref[sec:doubint]{\texttt{doubint}} function this function integrates the input data first in the forward direction and then if the input variable \texttt{der} is true it integrates the result backward direction.   

\subsubsection*{Call}
\begin{verbatim}doubint2 (nx, xx, der);\end{verbatim}

\subsubsection*{Definition}
\begin{verbatim}
static void doubint2 (int nx, float *xx, bool der)
{
   ...
}
\end{verbatim}

\subsubsection*{Input parameters}
\begin{desclist}{\tt }{\quad}[\tt der]
   \setlength\itemsep{0pt}
   \item[nx]  data length (\texttt{int}). 
   \item[xx]  a pointer to the input data (\texttt{const float}). 
   \item[der] a parameter to specify whether forward integration is required or not (\texttt{const float}).  
\end{desclist}




\subsection{{triple}}\label{sec:triple}
Does the smoothing to the input data.

\subsubsection*{Call}
\begin{verbatim}triple (o, d, nx, nb, x, tmp, box);\end{verbatim}

\subsubsection*{Definition}
\begin{verbatim}
static void triple (int o, int d, int nx, int nb, float* x, 
                    const float* tmp, bool box)
{
    ...
}
\end{verbatim}

\subsubsection*{Input parameters}
\begin{desclist}{\tt }{\quad}[\tt box]
   \setlength\itemsep{0pt}
   \item[o]   first indices of the input data (\texttt{int}).  
   \item[d]   step size (\texttt{int}).
   \item[nx]	  data length (\texttt{int}).        
   \item[nb]  filter length (\texttt{int}).        
   \item[np]	  length of the tmp array in the {sf\_Triangle} data structure. Must be of type \texttt{int}
   \item[x]   a pointer to the input data (\texttt{const float}). 
   \item[tmp] a pointer to an array in the \texttt{sf\_Triangle} data structure (\texttt{float}).
   \item[box] a parameter to specify whether a box filter is required (\texttt{bool}).
\end{desclist}




\subsection{{triple2}}\label{sec:triple2}
Does the smoothing to the input data.

\subsubsection*{Call}
\begin{verbatim}triple2 (o, d, nx, nb, x, tmp, box);\end{verbatim}

\subsubsection*{Definition}
\begin{verbatim}
static void triple2 (int o, int d, int nx, int nb, 
                    const float* x, float* tmp, bool box)
{
    ...
}
\end{verbatim}

\subsubsection*{Input parameters}
\begin{desclist}{\tt }{\quad}[\tt tmp]
   \setlength\itemsep{0pt}
   \item[o]   first indices of the input data (\texttt{int}).  
   \item[d]   step size (\texttt{int}).
   \item[nx]  data length (\texttt{int}).        
   \item[nb]  filter length (\texttt{int}).        
   \item[np]  length of the tmp array in the \texttt{sf\_Triangle} data structure. Must be of type \texttt{int}
   \item[x]   a pointer to the input data (\texttt{const float}). 
   \item[tmp] a pointer to an array in the \texttt{sf\_Triangle} data structure (\texttt{float}). 
   \item[box] a parameter to specify whether a box filter is required (\texttt{bool}).
\end{desclist}  




\subsection{{sf\_smooth}}
Smoothes the input data by first applying the \hyperref[sec:fold]{\texttt{fold}} function then \hyperref[sec:doubint]{\texttt{doubint}} and then \hyperref[sec:triple]{\texttt{triple}}.

\subsubsection*{Call}
\begin{verbatim}sf_smooth (tr, o, d, der, box, x);\end{verbatim}

\subsubsection*{Definition}
\begin{verbatim}
void sf_smooth (sf_triangle tr  /* smoothing object */, 
                int o, int d    /* trace sampling */, 
                bool der        /* if derivative */, 
                bool box        /* if box filter */,
                float *x        /* data (smoothed in place) */)
/*< apply triangle smoothing >*/
{
   ...   
}
\end{verbatim}

\subsubsection*{Input parameters}
\begin{desclist}{\tt }{\quad}[\tt box]
   \setlength\itemsep{0pt}
   \item[tr]  an object (filter) used for smoothing, box or triangle. Must be of type \texttt{sf\_triangle}. 
   \item[o]   first indices of the input data (\texttt{int}).  
   \item[d]   step size (\texttt{int}). 
   \item[der] a parameter to specify whether forward integration in \texttt{doubint} is required or not (\texttt{const float}).  
   \item[x]   a pointer to the input data (\texttt{float}). 
   \item[box] a parameter to specify whether a box filter is required (\texttt{bool}).
\end{desclist}  




\subsection{{sf\_smooth2}}
Smoothes the input data by first applying the \hyperref[sec:triple2]{\texttt{triple2}} function then \hyperref[sec:doubint2]{\texttt{doubint2}} and then \hyperref[sec:fold2]{\texttt{fold2}}.

\subsubsection*{Call}
\begin{verbatim}sf_smooth2 (tr, o, d, der, box, x);\end{verbatim}

\subsubsection*{Definition}
\begin{verbatim}
void sf_smooth2 (sf_triangle tr  /* smoothing object */, 
                int o, int d    /* trace sampling */, 
                bool der        /* if derivative */, 
                bool box        /* if box filter */,
                float *x        /* data (smoothed in place) */)
/*< apply adjoint triangle smoothing >*/
{
   ...   
}
\end{verbatim}

\subsubsection*{Input parameters}
\begin{desclist}{\tt }{\quad}[\tt box]
   \setlength\itemsep{0pt}
   \item[tr]  an object (filter) used for smoothing, box or triangle. Must be of type \texttt{sf\_triangle}. 
   \item[o]   first indices of the input data (\texttt{int}).  
   \item[d]   step size (\texttt{int}). 
   \item[der] a parameter to specify whether forward integration in \texttt{doubint} is required or not (\texttt{const float}).  
   \item[x]   a pointer to the input data (\texttt{float}). 
   \item[box] a parameter to specify whether a box filter is required (\texttt{bool}).
 \end{desclist}




\subsection{{sf\_triangle\_close}}
Frees the space allocated for the triangle smoothing filter.

\subsubsection*{Call}
\begin{verbatim}sf_triangle_close(tr);\end{verbatim}

\subsubsection*{Definition}
\begin{verbatim}
void  sf_triangle_close(sf_triangle tr)
/*< free allocated storage >*/
{
   ...
}
\end{verbatim}

\subsubsection*{Input parameters}
\begin{desclist}{\tt }{\quad}[\tt ]
   \setlength\itemsep{0pt}
   \item[tr] the triangle smoothing filter. Must be of type \texttt{sf\_triangle}.            
\end{desclist}





    % Triangle smoothing
   \section{Smooth gradient operations (edge.c)}




\subsection{{sf\_grad2}}
Calculates the gradient squared of the input with the centered finite-difference formula.


\subsubsection*{Call}
\begin{verbatim}sf_grad2 (n, x, w);\end{verbatim}

\subsubsection*{Definition}
\begin{verbatim}
void sf_grad2 (int n          /* data size */, 
               const float *x /* input trace [n] */, 
               float *w       /* output gradient squared [n] */)
/*< centered finite-difference gradient >*/
{
   ...    
}
\end{verbatim}

\subsubsection*{Input parameters}
\begin{desclist}{\tt }{\quad}[\tt ]
   \setlength\itemsep{0pt}
   \item[n] size of the data (\texttt{int}). 
   \item[x] input trace (\texttt{const float*}).  
   \item[w] output gradient squared (\texttt{float*}).  
\end{desclist}




\subsection{{sf\_sobel}}\label{sec:sf_sobel}
Calculates the 9-point Sobel's gradient for a 2D image.

\subsubsection*{Call}
\begin{verbatim}sf_sobel (n1, n2, x, w1, w2);\end{verbatim}

\subsubsection*{Definition}
\begin{verbatim}
void sf_sobel (int n1, int n2         /* data size */, 
               float **x              /* input data [n2][n1] */, 
               float **w1, float **w2 /* output gradient [n2][n1] */)
/*< Sobel's 9-point gradient >*/
{
   ...
}
\end{verbatim}

\subsubsection*{Input parameters}
\begin{desclist}{\tt }{\quad}[\tt w2]
   \setlength\itemsep{0pt}
   \item[n1]	 size of the data, first axis (\texttt{int}).  
   \item[n2]	 size of the data, second axis (\texttt{int}). 
   \item[x]  2D input data (\texttt{const float**}).  
   \item[w1] output gradient, first axis (\texttt{float**}).  
   \item[w2] output gradient, second axis (\texttt{float**}).  
\end{desclist}




\subsection{{sf\_sobel2}}\label{sec:sf_sobel2}
Calculates the Sobel's gradient squared for a 2D image. It works like \hyperref[sec:sf_sobel]{\texttt{sf\_sobel}} but outputs the gradient squared.

\subsubsection*{Call}
\begin{verbatim}sf_sobel2 (n1, n2, x, w);\end{verbatim}

\subsubsection*{Definition}
\begin{verbatim}
void sf_sobel2 (int n1, int n2  /* data size */, 
                float **x        /* input data [n2][n1] */, 
                float **w        /* output gradient squared [n2][n1] */)
/*< Sobel's gradient squared >*/
{
   ...
}
\end{verbatim}

\subsubsection*{Input parameters}
\begin{desclist}{\tt }{\quad}[\tt n2]
   \setlength\itemsep{0pt}
   \item[n1] size of the data, first axis (\texttt{int}).  
   \item[n2] size of the data, second axis (\texttt{int}). 
   \item[x]  2D input data (\texttt{const float**}).  
   \item[w]  output gradient squared (\texttt{float**}).  
\end{desclist}




\subsection{{sf\_sobel32}}
Calculates the Sobel's gradient squared for a 3D image. It works like \hyperref[sec:sf_sobel]{\texttt{sf\_sobel}} but outputs the gradient squared for 3D data.

\subsubsection*{Call}
\begin{verbatim}sf_sobel32 (n1, n2, n3, x, w);\end{verbatim}

\subsubsection*{Definition}
\begin{verbatim}
void sf_sobel32 (int n1, int n2, int n3  /* data size */, 
                 float ***x              /* input data [n3][n2][n1] */, 
                 float ***w              /* output gradient squared */)
/*< Sobel's gradient squared in 3-D>*/
{
   ...
}
\end{verbatim}

\subsubsection*{Input parameters}
\begin{desclist}{\tt }{\quad}[\tt n2]
   \setlength\itemsep{0pt}
   \item[n1]	 size of the data, first axis (\texttt{int}).  
   \item[n2]	 size of the data, second axis (\texttt{int}).  
   \item[n3]	 size of the data, third axis (\texttt{int}). 
   \item[x]  3D input data (\texttt{const float***}).  
   \item[w]  output gradient squared (\texttt{float***}).  
\end{desclist}



        % Smooth gradient operations


\chapter{Ray tracing}\label{sec:ray}
   \section{Cell ray tracing (celltrace.c)}




\subsection{{sf\_celltrace\_init}}
Initializes the object \texttt{sf\_celltrace} for ray tracing by initializing the required variables and allocating the required space.

\subsubsection*{Call}
\begin{verbatim}ct = sf_celltrace_init (order, nt, nz, nx, dz, dx, z0, x0, slow);\end{verbatim}

\subsubsection*{Definition}
\begin{verbatim}
sf_celltrace sf_celltrace_init (int order   /* interpolation accuracy */, 
                                int nt      /* maximum time steps */,
                                int nz      /* depth samples */, 
                                int nx      /* lateral samples */, 
                                float dz    /* depth sampling */, 
                                float dx    /* lateral sampling */, 
                                float z0    /* depth origin */, 
                                float x0    /* lateral origin */, 
                                float* slow /* slowness [nz*nx] */)
/*< Initialize ray tracing object >*/
{
  ...
} 
\end{verbatim}

\subsubsection*{Input parameters}
\begin{desclist}{\tt }{\quad}[\tt order]
   \setlength\itemsep{0pt}
   \item[order] accuracy of the interpolation (\texttt{int}). 
   \item[nt]    maximum number of time steps (\texttt{int}). 
   \item[nz]    number of depth samples (\texttt{int}). 
   \item[nx]    number of lateral samples (\texttt{int}). 
   \item[dz]    depth sampling interval (\texttt{float}). 
   \item[dx]    lateral sampling interval (\texttt{float}). 
   \item[z0]    depth origin (\texttt{float}). 
   \item[x0]    lateral origin (\texttt{float}). 
   \item[slow]  slowness (\texttt{float*}).  
\end{desclist}

\subsubsection*{Output}
\begin{desclist}{\tt }{\quad}[\tt ]
   \setlength\itemsep{0pt}
   \item[ct] the ray tracing object. It is of type \texttt{sf\_celltrace}.
\end{desclist}




\subsection{{sf\_celltrace\_close}}
Frees the space allocated for the \texttt{sf\_celltrace} object by \texttt{sf\_celltrace\_init}.

\subsubsection*{Call}
\begin{verbatim}sf_celltrace_close (ct);\end{verbatim}

\subsubsection*{Definition}
\begin{verbatim}
void sf_celltrace_close (sf_celltrace ct)
/*< Free allocated storage >*/
{
   ...
}
\end{verbatim}




\subsection{{sf\_cell\_trace}}
Traces the ray with the ray parameter specified in the input.

\subsubsection*{Call}
\begin{verbatim}t = sf_cell_trace (ct, xp, p, it, traj);\end{verbatim}

\subsubsection*{Definition}
\begin{verbatim}
float sf_cell_trace (sf_celltrace ct, 
                     float* xp    /* position */, 
                     float* p     /* ray parameter */, 
                     int* it      /* steps till boundary */, 
                     float** traj /* trajectory */)
/*< ray trace >*/
{
   ...
}
\end{verbatim}

\subsubsection*{Input parameters}
\begin{desclist}{\tt }{\quad}[\tt ]
   \setlength\itemsep{0pt}
   \item[ct]   the ray tracing object. It is of type \texttt{sf\_celltrace}. 
   \item[xp]   position (\texttt{float*}).  
   \item[p]    ray parameters (\texttt{float*}).  
   \item[it]   number steps till the boundary (\texttt{int*}).  
   \item[traj] trajectory of the ray (\texttt{float**}).  
\end{desclist}

\subsubsection*{Output}
\begin{desclist}{\tt }{\quad}[\tt ]
   \setlength\itemsep{0pt}
   \item[t] the travel time obtained by the ray tracing. It is of type \texttt{float}.
\end{desclist}


   % Cell ray tracing
   \section{Cell ray tracing (cell.c)}




\subsection{{sf\_cell1\_intersect}}
Intersects a straight ray with the cell boundary.

\subsubsection*{Call}
\begin{verbatim}sf_cell1_intersect (a, x, dy, p, sx, jx);\end{verbatim}

\subsubsection*{Definition}
\begin{verbatim}
void sf_cell1_intersect (float a, float x, float dy, float p, 
                         float *sx, int *jx)
/*< intersecting a straight ray with cell boundaries >*/
{
   ...
}
\end{verbatim}

\subsubsection*{Input parameters}
\begin{desclist}{\tt }{\quad}[\tt dy]
   \setlength\itemsep{0pt}
   \item[a]  gradient of slowness (\texttt{float}).  
   \item[x]  non-integer part of the position in the grid relative to grid origin. It is of type \texttt{float}.
   \item[dy] depth or lateral sampling divided by slowness. It is of type \texttt{float}.
   \item[p]  the ray parameter. It is of type \texttt{float}.
   \item[sx] distance traveled in the medium (cell) times the velocity of the medium (equivalent to the optical path length in optics). It is of type \texttt{float*}.
   \item[jx] the direction of the ray. It is of type \texttt{int*}.
\end{desclist}




\subsection{{sf\_cell1\_update1}}
Performs the first step of the first order symplectic method for ray tracing.

\subsubsection*{Call}
\begin{verbatim}tt = sf_cell1_update1 (dim, s, v, p, g);\end{verbatim}

\subsubsection*{Definition}
\begin{verbatim}
float sf_cell1_update1 (int dim, float s, float v, float *p, const float *g) 
/*< symplectic first-order: step 1 >*/
{
   ...
}
\end{verbatim}

\subsubsection*{Input parameters}
\begin{desclist}{\tt }{\quad}[\tt dim]
   \setlength\itemsep{0pt}
   \item[dim] dimension (\texttt{int}).  
   \item[s]   $\sigma$ (\texttt{float}).
   \item[v]   slowness. It is of type \texttt{float}.
   \item[p]   direction. It is of type \texttt{float*}.
   \item[g]   slowness gradient. It is of type \texttt{const float*}.
\end{desclist}

\subsubsection*{Output}
\begin{desclist}{}{\quad}[\tt ]
   \setlength\itemsep{0pt}  
   \item[0.5*v*v*s*(1. + s*pg)] travel time. It is of type \texttt{float}.
\end{desclist}




\subsection{{sf\_cell1\_update2}}
Performs the second step of the first order symplectic method for ray tracing.

\subsubsection*{Call}
\begin{verbatim}tt = sf_cell1_update2 (dim, s, v, p, g);\end{verbatim}

\subsubsection*{Definition}
\begin{verbatim}
float sf_cell1_update2 (int dim, float s, float v, float *p, const float *g) 
/*< symplectic first-order: step 2 >*/
{
   ...
}
\end{verbatim}

\subsubsection*{Input parameters}
\begin{desclist}{\tt }{\quad}[\tt dim]
   \setlength\itemsep{0pt}
   \item[dim] dimension (\texttt{int}).  
   \item[s]   $\sigma$ (\texttt{float}).
   \item[v]   slowness. It is of type \texttt{float}.
   \item[p]   direction. It is of type \texttt{float*}.
   \item[g]   slowness gradient. It is of type \texttt{const float*}.
\end{desclist}

\subsubsection*{Output}
\begin{desclist}{ }{\quad}[\tt ]
   \setlength\itemsep{0pt}  
   \item[0.5*v*v*s*(1. - s*pg)] travel time. It is of type \texttt{float}.
\end{desclist}




\subsection{{sf\_cell11\_intersect2}}
Intersects a straight ray with the cell boundary.

\subsubsection*{Call}
\begin{verbatim}sf_cell11_intersect2 (a, da, p, g, sp, jp);\end{verbatim}

\subsubsection*{Definition}
\begin{verbatim}
void sf_cell11_intersect2 (float a, float da, 
                           const float* p, const float* g, 
                           float *sp, int *jp)
/*< intersecting a straight ray with cell boundaries >*/
{
   ...
}
\end{verbatim}

\subsubsection*{Input parameters}
\begin{desclist}{\tt }{\quad}[\tt ]
   \setlength\itemsep{0pt}
   \item[a]  position in the grid (\texttt{float}).  
   \item[da] grid spacing. It is of type \texttt{float}.
   \item[p]  the ray parameter. It is of type \texttt{const float*}.
   \item[g]  gradient of slowness (\texttt{const float*}).  
   \item[sp] distance traveled in the medium (cell) times the velocity of the medium (equivalent to the optical path length in optics). It is of type \texttt{float*}.
   \item[jp] the direction of the ray. It is of type \texttt{int*}.
\end{desclist}




\subsection{{sf\_cell11\_update1}}
Performs the first step of the first order non-symplectic method for ray tracing.

\subsubsection*{Call}
\begin{verbatim}tt = sf_cell11_update1 (dim, s, v, p, g);\end{verbatim}

\subsubsection*{Definition}
\begin{verbatim}
float sf_cell11_update1 (int dim, float s, float v, float *p, const float *g) 
/*< nonsymplectic first-order: step 1 >*/
{
   ...
}
\end{verbatim}

\subsubsection*{Input parameters}
\begin{desclist}{\tt }{\quad}[\tt dim]
   \setlength\itemsep{0pt}
   \item[dim] dimension (\texttt{int}).  
   \item[s]   $\sigma$ (\texttt{float}).
   \item[v]   slowness. It is of type \texttt{float}.
   \item[p]   direction. It is of type \texttt{float*}.
   \item[g]   slowness gradient. It is of type \texttt{const float*}.
\end{desclist}

\subsubsection*{Output}
\begin{desclist}{\tt }{\quad}[\tt ]
   \setlength\itemsep{0pt}  
   \item[0.5*v*v*s*(1. + s*pg)] travel time. It is of type \texttt{float}.
\end{desclist}




\subsection{{sf\_cell11\_update2}}
Performs the second step of the first order non-symplectic method for ray tracing.


\subsubsection*{Call}
\begin{verbatim}tt = sf_cell11_update2 (dim, s, v, p, g);\end{verbatim}


\subsubsection*{Definition}
\begin{verbatim}
float sf_cell11_update2 (int dim, float s, float v, float *p, const float *g) 
/*< nonsymplectic first-order: step 2 >*/
{
   ...
}
\end{verbatim}

\subsubsection*{Input parameters}
\begin{desclist}{\tt }{\quad}[\tt dim]
   \setlength\itemsep{0pt}
   \item[dim] dimension (\texttt{int}).  
   \item[s]   $\sigma$ (\texttt{float}).
   \item[v]   slowness. It is of type \texttt{float}.
   \item[p]   direction. It is of type \texttt{float*}.
   \item[g]   slowness gradient. It is of type \texttt{const float*}.
\end{desclist}

\subsubsection*{Output}
\begin{desclist}{\tt }{\quad}[\tt ]
   \setlength\itemsep{0pt}  
   \item[0.5*v*v*s*(1. - s*pg)] travel time. It is of type \texttt{float}.
\end{desclist}




\subsection{{sf\_cell\_intersect}}
Intersects a parabolic ray with the cell boundary.


\subsubsection*{Call}
\begin{verbatim}sf_cell_intersect (a, x, dy, p, sx, jx);\end{verbatim}

\subsubsection*{Definition}
\begin{verbatim}
void sf_cell_intersect (float a, float x, float dy, float p, 
                        float *sx, int *jx)
/*< intersecting a parabolic ray with cell boundaries >*/
{
   ...
}
\end{verbatim}

\subsubsection*{Input parameters}
\begin{desclist}{\tt }{\quad}[\tt ]
   \setlength\itemsep{0pt}
   \item[a]  gradient of slowness (\texttt{float}).  
   \item[x]  non-integer part of the position in the grid relative to grid origin. It is of type \texttt{float}.
   \item[dy] depth or lateral sampling divided by slowness. It is of type \texttt{float}.
   \item[p]  the ray parameter. It is of type \texttt{float}.
   \item[sx] distance traveled in the medium (cell) times the velocity of the medium (equivalent to the optical path length in optics). It is of type \texttt{float*}.
   \item[jx]  the direction of the ray. It is of type \texttt{int*}.
\end{desclist}




\subsection{{sf\_cell\_snap}}
Terminates the ray at the nearest boundary.

\subsubsection*{Definition}
\begin{verbatim} b = sf_cell_snap (z, iz, eps);\end{verbatim}

\subsubsection*{Definition}
\begin{verbatim}
bool sf_cell_snap (float *z, int *iz, float eps)
/*< round to the nearest boundary >*/
{
   ...
}
\end{verbatim}

\subsubsection*{Input parameters}
\begin{desclist}{\tt }{\quad}[\tt eps]
   \setlength\itemsep{0pt}
   \item[z]   position (\texttt{float*}).  
   \item[iz]  sampling (\texttt{int*}).
   \item[eps] tolerance. It is of type \texttt{float}.
\end{desclist}

\subsubsection*{Output}
\begin{desclist}{\tt }{\quad}[\tt ]
   \setlength\itemsep{0pt}  
   \item[true/false] whether the ray is terminated or not. It is of type \texttt{bool}.
\end{desclist}




\subsection{{sf\_cell\_update1}}
Performs the first step of the second order symplectic method for ray tracing.


\subsubsection*{Call}
\begin{verbatim}tt = sf_cell_update1 (dim, s, v, p, g);\end{verbatim}

\subsubsection*{Definition}
\begin{verbatim}
float sf_cell_update1 (int dim, float s, float v, float *p, const float *g) 
/*< symplectic second-order: step 1 >*/
{
   ...
}
\end{verbatim}

\subsubsection*{Input parameters}
\begin{desclist}{\tt }{\quad}[\tt dim]
   \setlength\itemsep{0pt}
   \item[dim] dimension (\texttt{int}).  
   \item[s]   $\sigma$ (\texttt{float}).
   \item[v]   slowness. It is of type \texttt{float}.
   \item[p]   direction. It is of type \texttt{float*}.
   \item[g]   slowness gradient. It is of type \texttt{const float*}.
\end{desclist}

\subsubsection*{Output}
\begin{desclist}{\tt }{\quad}[ ]
   \setlength\itemsep{0pt}  
   \item[0.5*v*v*s*(1. + s*pg)] travel time. It is of type \texttt{float}.
\end{desclist}




\subsection{{sf\_cell\_update2}}
Performs the second step of the second order symplectic method for ray tracing.

\subsubsection*{Call}
\begin{verbatim}tt = sf_cell_update2 (dim, s, v, p, g);\end{verbatim}

\subsubsection*{Definition}
\begin{verbatim}
float sf_cell_update2 (int dim        /* number of dimensions */, 
                       float s        /* sigma */, 
                       float v        /* slowness */, 
                       float *p       /* in - ?, out - direction */, 
                       const float *g /* slowness gradient */) 
/*< symplectic second-order: step 2 >*/
{
   ...
}
\end{verbatim}

\subsubsection*{Input parameters}
\begin{desclist}{\tt }{\quad}[\tt dim]
   \setlength\itemsep{0pt}
   \item[dim] dimension (\texttt{int}).  
   \item[s]   $\sigma$ (\texttt{float}).
   \item[v]   slowness. It is of type \texttt{float}.
   \item[p]   direction. It is of type \texttt{float*}.
   \item[g]   slowness gradient. It is of type \texttt{const float*}.
\end{desclist}

\subsubsection*{Output}
\begin{desclist}{ }{\quad}[ ]
   \setlength\itemsep{0pt}  
   \item[0.5*v*v*s*(1. - s*pg)] travel time. It is of type \texttt{float}.
\end{desclist}





\subsection{{sf\_cell\_p2a}}
Converts the ray parameter to an angle.


\subsubsection*{Call}
\begin{verbatim}a = sf_cell_p2a (p);\end{verbatim}


\subsubsection*{Definition}
\begin{verbatim}
float sf_cell_p2a (float* p)
/*< convert ray parameter to angle >*/
{
   ...
}
\end{verbatim}

\subsubsection*{Input parameters}
\begin{desclist}{\tt }{\quad}[\tt ]
   \setlength\itemsep{0pt}
   \item[p] the ray parameter (\texttt{float*}).  
\end{desclist}

\subsubsection*{Output}
\begin{desclist}{\tt }{\quad}[\tt ]
   \setlength\itemsep{0pt}  
   \item[a] angle of the ray. It is of type \texttt{float*}.
\end{desclist}


        % Cell ray tracing

\chapter{General tools}\label{sec:general}
   \section{First derivative FIR filter (deriv.c)}




\subsection{{sf\_deriv\_init}}
Initializes the derivative calculation of the input trace, that is, it sets the required parameters and allocates the required space.


\subsubsection*{Call}
\begin{verbatim}sf_deriv_init (nt1, n1, c1);\end{verbatim}


\subsubsection*{Definition}
\begin{verbatim}
void sf_deriv_init(int nt1  /* transform length */, 
                   int n1   /* trace length */, 
                   float c1 /* filter parameter */)
{
   ...
}

\end{verbatim}
\subsubsection*{Input parameters}
\begin{desclist}{\tt }{\quad}[\tt ]
   \setlength\itemsep{0pt}
   \item[nt1] length of the transform (derivative) (\texttt{int}). 
   \item[n1]  length of the trace (\texttt{int}). 
   \item[c1]  filter parameter (\texttt{float}).  
\end{desclist}




\subsection{{sf\_deriv\_free}}
Frees the temporary space allocated for the derivative operator.


\subsubsection*{Call}
\begin{verbatim}sf_deriv_free ();\end{verbatim}


\subsubsection*{Definition}
\begin{verbatim}
void sf_deriv_free(void)
{
   ...
}
\end{verbatim}




\subsection{{sf\_deriv}}
Calculates the derivative of the input trace (\texttt{trace}) and outputs it to \texttt{trace2}.


\subsubsection*{Definition}
\begin{verbatim}sf_deriv (trace, trace2);\end{verbatim}


\subsubsection*{Definition}
\begin{verbatim}
void sf_deriv (const float* trace, float* trace2)
/*< derivative operator >*/
{
   ...
}
\end{verbatim}


\subsubsection*{Input parameters}
\begin{desclist}{\tt }{\quad}[\tt ]
   \setlength\itemsep{0pt}
   \item[trace] input trace whose derivative is required (\texttt{float*}).  
   \item[trace2] location where the derivative is to be stored (\texttt{float*}).  
\end{desclist}



      % First derivative FIR filter
   \section{Computing quantiles by Hoare's algorithm (quantile.c)}




\subsection{{sf\_quantile}}
Returns the quantile - which is specified in the input -  for the input array.

\subsubsection*{Call}
\begin{verbatim}k = sf_quantile(q, n, a);\end{verbatim}

\subsubsection*{Definition}
\begin{verbatim}
float sf_quantile(int q    /* quantile */, 
                  int n    /* array length */, 
                  float* a /* array [n] */) 
/*< find quantile (caution: a is changed) >*/ 
{
   ...
}
\end{verbatim}

\subsubsection*{Input parameters}
\begin{desclist}{\tt }{\quad}[\tt ]
   \setlength\itemsep{0pt}
   \item[q] the required quantile (\texttt{int}). 
   \item[n] length of the input array (\texttt{int}). 
   \item[a] the input array for which the quantile is required (\texttt{float*}).
\end{desclist}

\subsubsection*{Output}
\begin{desclist}{\tt }{\quad}[\tt ]
   \setlength\itemsep{0pt}
   \item[*k] the quantile. It is of type \texttt{float}.
\end{desclist}




   % Computing quantiles by Hoare's algorithm
   \section{Pseudo-random numbers: uniform and normally distributed (randn.c)}




\subsection{{sf\_randn1}}
Generates a normally distributed random number using the Box-Muller method.

\subsubsection*{Call}
\begin{verbatim}vset = sf_randn_one_bm ();\end{verbatim}

\subsubsection*{Definition}
\begin{verbatim}
float sf_randn_one_bm (void)
/*< return a random number (normally distributed, Box-Muller method) >*/
{
   ...
}
\end{verbatim}

\subsubsection*{Output}
\begin{desclist}{\tt }{\quad}[\tt ]
   \setlength\itemsep{0pt}
   \item[vset] the random number. It is of type \texttt{float}.
\end{desclist}




\subsection{{sf\_randn}}
Fills an array with normally distributed random numbers.

\subsubsection*{Call}
\begin{verbatim}sf_randn (nr, r);\end{verbatim}

\subsubsection*{Definition}
\begin{verbatim}
void sf_randn (int nr, float *r /* [nr] */)
/*< fill an array with normally distributed numbers >*/
{
   ...
}
\end{verbatim}

\subsubsection*{Input parameters}
\begin{desclist}{\tt }{\quad}[\tt ]
   \setlength\itemsep{0pt}
   \item[nr] size of the array where the random numbers are to be stored (\texttt{int}). 
   \item[r] the array where the random numbers are to be stored (\texttt{float*}).
\end{desclist}




\subsection{{sf\_random}}
Fills an array with uniformly distributed random numbers.

\subsubsection*{Call}
\begin{verbatim}sf_random (nr, r)\end{verbatim}

\subsubsection*{Definition}
\begin{verbatim}
void sf_random (int nr, float *r /* [nr] */)
/*< fill an array with uniformly distributed numbers >*/
{
   ...
}
\end{verbatim}

\subsubsection*{Input parameters}
\begin{desclist}{\tt }{\quad}[\tt ]
   \setlength\itemsep{0pt}
   \item[nr] size of the array where the random numbers are to be stored (\texttt{int}). 
   \item[nr] the array where the random numbers are to be stored (\texttt{float*}).
\end{desclist}



      % Pseudo-random numbers: uniform and normally distributed
   \section{Evaluating mathematical expressions (math1.c)}




\subsection{{myabs}}
Returns a complex number with zero imaginary value and the real non-zero real part is the absolute value of the input complex number.

\subsubsection*{Call}
\begin{verbatim}c = sf_complex myabs(c);\end{verbatim}

\subsubsection*{Definition}
\begin{verbatim}
static sf_complex myabs(sf_complex c)
{
   ...
}
\end{verbatim}

\subsubsection*{Input parameters}
\begin{desclist}{\tt }{\quad}[\tt ]
   \setlength\itemsep{0pt}
   \item[c] a complex number (\texttt{sf\_complex}).  
\end{desclist}

\subsubsection*{Output}
\begin{desclist}{\tt }{\quad}[\tt ]
   \setlength\itemsep{0pt}
   \item[c] a complex number with real part = absolute value of the input complex number and imaginary part = zero. It is of type \texttt{static sf\_complex}.
\end{desclist}




\subsection{{myconj}}
Returns the complex conjugate of the input complex number.

\subsubsection*{Call}
\begin{verbatim}c = myconj(sf_complex c);\end{verbatim}

\subsubsection*{Definition}
\begin{verbatim}
static sf_complex myconj(sf_complex c)
{
   ...
}
\end{verbatim}

\subsubsection*{Input parameters}
\begin{desclist}{\tt }{\quad}[\tt ]
   \setlength\itemsep{0pt}
   \item[c] a complex number (\texttt{sf\_complex}).  
\end{desclist}

\subsubsection*{Output}
\begin{desclist}{\tt }{\quad}[\tt ]
   \setlength\itemsep{0pt}
   \item[c] complex conjugate of the input complex number.
\end{desclist}




\subsection{{myarg}}
Returns the argument of the input complex number.

\subsubsection*{Call}
\begin{verbatim}c = myarg(c);\end{verbatim}

\subsubsection*{Definition}
\begin{verbatim}
static sf_complex myarg(sf_complex c)
{
   ...
}
\end{verbatim}

\subsubsection*{Input parameters}
\begin{desclist}{\tt }{\quad}[\tt ]
   \setlength\itemsep{0pt}
   \item[c] a complex number (\texttt{sf\_complex}).  
\end{desclist}

\subsubsection*{Output}
\begin{desclist}{\tt }{\quad}[\tt ]
   \setlength\itemsep{0pt}
   \item[c] argument of the input complex number.
\end{desclist}




\subsection{{sf\_math\_evaluate}}\label{sec:sf_math_evaluate}
Applies a mathematical function to the input stack. For example it could evaluate the exponents of the samples in the stack.

\subsubsection*{Call}
\begin{verbatim}sf_math_evaluate(len, nbuf, fbuf, fst);\end{verbatim}

\subsubsection*{Definition}
\begin{verbatim}
void sf_math_evaluate (int     len  /* stack length */, 
                       int     nbuf /* buffer length */, 
                       float** fbuf /* number buffers */, 
                       float** fst  /* stack */)
/*< Evaluate a mathematical expression from stack (float numbers) >*/
{
   ...
}
\end{verbatim}

\subsubsection*{Input parameters}
\begin{desclist}{\tt }{\quad}[\tt nbuf]
   \setlength\itemsep{0pt}
   \item[len]  length of the stack (\texttt{int}). 
   \item[nbuf] length of the buffer (\texttt{int}). 
   \item[fbuf] buffers for floating point numbers (\texttt{float**}).  
   \item[fst]  the stack (\texttt{float**}).  
\end{desclist}




\subsection{{sf\_complex\_math\_evaluate}}
Applies a mathematical function to the input stack. For example it could evaluate the exponents of the samples in the stack, It works like \hyperref[sec:sf_math_evaluate]{\texttt{sf\_math\_evaluate}} but does it for complex numbers.

\subsubsection*{Call}
\begin{verbatim}sf_complex_math_evaluate(len, nbuf, cbuf, cst);\end{verbatim}

\subsubsection*{Definition}
\begin{verbatim}
void sf_complex_math_evaluate (int          len  /* stack length */, 
                               int          nbuf /* buffer length */, 
                               sf_complex** cbuf /* number buffers */, 
                               sf_complex** cst  /* stack */)
/*< Evaluate a mathematical expression from stack (complex numbers) >*/
{
   ...
}
\end{verbatim}

\subsubsection*{Call}
\begin{verbatim}sf_complex_math_evaluate(len, nbuf, cbuf, cst)\end{verbatim}

\subsubsection*{Definition}
\begin{verbatim}
void sf_complex_math_evaluate (int          len  /* stack length */, 
                               int          nbuf /* buffer length */, 
                               sf_complex** cbuf /* number buffers */, 
                               sf_complex** cst  /* stack */)
/*< Evaluate a mathematical expression from stack (complex numbers) >*/
{
   ...
}
\end{verbatim}

\subsubsection*{Input parameters}
\begin{desclist}{\tt }{\quad}[\tt nbuf]
   \setlength\itemsep{0pt}
   \item[len]  length of the stack (\texttt{int}). 
   \item[nbuf] length of the buffer (\texttt{int}). 
   \item[fbuf] buffers for floating point numbers (\texttt{sf\_complex**}).  
   \item[fst]  the stack (\texttt{sf\_complex**}).  
\end{desclist}




\subsection{{sf\_math\_parse}}
Parses the mathematical expression and returns the stack length.

\subsubsection*{call}
\begin{verbatim}len = sf_math_parse(output, out, datatype);\end{verbatim}

\subsubsection*{Definition}
\begin{verbatim}
size_t sf_math_parse (char*       output /* expression */, 
                      sf_file     out    /* parameter file */,
                      sf_datatype datatype)
/*< Parse a mathematical expression, returns stack length >*/ 
{
   ...
}
\end{verbatim}

\subsubsection*{Input parameters}
\begin{desclist}{\tt }{\quad}[\tt datatype]
   \setlength\itemsep{0pt}
   \item[output]   the expression which is to be parsed (\texttt{char}). 
   \item[out]      parameter file (\texttt{sf\_file}). 
   \item[datatype] file datatype (\texttt{sf\_datatype}).  
\end{desclist}

\subsubsection*{Output}
\begin{desclist}{\tt }{\quad}[\tt ]
   \setlength\itemsep{0pt}
   \item[len] length of the stack. It is of type \texttt{size\_t}.
\end{desclist}




\subsection{{sf\_math\_parse}}
Checks for any syntax errors.

\subsubsection*{Call}
\begin{verbatim}check();\end{verbatim}

\subsubsection*{Definition}
\begin{verbatim}
static void check (void)
{
   ...   
}
\end{verbatim}



      % Evaluating mathematical expressions

\chapter{Geometry}\label{sec:geometry}
   \section{Construction of points (point.c)}\label{sec:point.c}




\subsection{{printpt2d}}
Prints the value and location of a 2D point (position vector).

\subsubsection*{Call}
\begin{verbatim}printpt2d(pt2d P);\end{verbatim}

\subsubsection*{Definition}
\begin{verbatim}
void printpt2d(pt2d P)
/*< print point2d info  >*/
{
   ...
}
\end{verbatim}

\subsubsection*{Input parameters}
\begin{desclist}{\tt }{\quad}[\tt ]
   \setlength\itemsep{0pt}
   \item[P] a point (position vector) (\texttt{pt3d}).     
\end{desclist}




\subsection{{printpt3d}}
Prints the value and location of a 3D point (position vector).

\subsubsection*{Call}
\begin{verbatim}printpt3d(P);\end{verbatim}

\subsubsection*{Definition}
\begin{verbatim}
void printpt3d(pt3d P)
/*< print point3d info  >*/
{
   ...
}
\end{verbatim}

\subsubsection*{Input parameters}
\begin{desclist}{\tt }{\quad}[\tt ]
   \setlength\itemsep{0pt}
   \item[P] a point (position vector) (\texttt{pt3d}).     
\end{desclist}




\subsection{{pt2dwrite1}}
Outputs a 1D array of 2D points to a file. It can be used to define the source or receiver arrays, for example.

\subsubsection*{Call}
\begin{verbatim}pt2dwrite1(F, v, n1, k);\end{verbatim}

\subsubsection*{Definition}
\begin{verbatim}
void pt2dwrite1(sf_file F, pt2d *v, size_t n1, int k)
/*< output point2d 1-D vector >*/
{
   ...    
}
\end{verbatim}

\subsubsection*{Input parameters}
\begin{desclist}{\tt }{\quad}[\tt File]
   \setlength\itemsep{0pt}
   \item[File] a file to which the 1D array of 2D points is to be output (\texttt{sf\_file}). 
   \item[v]    an array of 2D points which is to be output (\texttt{pt2d}). 
   \item[n1]   size of the array (\texttt{size\_t}). 
   \item[k]    a number, which if equal to 3, indicates that the value of the 2D points must also be included (\texttt{int}).     
\end{desclist}




\subsection{{pt2dwrite2}}
Outputs a 2D array of 2D points to a file. It can be used to define the source or receiver arrays, for example.

\subsubsection*{Call}
\begin{verbatim}pt2dwrite2(F, v, n1, n2, k);\end{verbatim}

\subsubsection*{Definition}
\begin{verbatim}
void pt2dwrite2(sf_file F, pt2d **v, size_t n1, size_t n2, int k)
/*< output point2d 2-D vector >*/
{
   ...
}
\end{verbatim}

\subsubsection*{Input parameters}
\begin{desclist}{\tt }{\quad}[\tt File]
   \setlength\itemsep{0pt}
   \item[File] a file to which the 2D array of 2D points is to be output (\texttt{sf\_file}). 
   \item[v]    an array of 2D points which is to be output (\texttt{pt2d}). 
   \item[n1]   size of one axis of the 2D array (\texttt{size\_t}).
   \item[n2]	   size of the other axis of the 2D array (\texttt{size\_t}). 
   \item[k]    a number, which if equal to 3, indicates that the value of the 2D points must also be included (\texttt{int}).     
\end{desclist}




\subsection{{pt3dwrite1}}
Outputs a 1D array of 3D points to a file. It can be used to define the source or receiver arrays, for example.

\subsubsection*{Call}
\begin{verbatim}pt3dwrite1(F, v, n1, k);\end{verbatim}

\subsubsection*{Definition}
\begin{verbatim}
void pt3dwrite1(sf_file F, pt3d *v, size_t n1, int k)
/*< output point3d 1-D vector >*/
{
   ...
}
\end{verbatim}

\subsubsection*{Input parameters}
\begin{desclist}{\tt }{\quad}[\tt File]
   \setlength\itemsep{0pt}
   \item[File] a file to which the 1D array of 3D points is to be output (\texttt{sf\_file}). 
   \item[v]    an array of 1D points which is to be output (\texttt{pt3d}). 
   \item[n1]   size of one axis of the 3D array (\texttt{size\_t}). 
   \item[k]    a number, which if equal to 4, indicates that the value of the 3D points must also be included (\texttt{int}).     
\end{desclist}




\subsection{{pt3dwrite2}}
Outputs a 2D array of 3D points to a file. It can be used to define the source or receiver arrays, for example.

\subsubsection*{Call}
\begin{verbatim}pt3dwrite2(F, v, n1, n2, k);\end{verbatim}

\subsubsection*{Definition}
\begin{verbatim}
void pt3dwrite2(sf_file F, pt3d *v, size_t n1, size_t n2, int k)
/*< output point3d 2-D vector >*/
{
   ... 
}
\end{verbatim}

\subsubsection*{Input parameters}
\begin{desclist}{\tt }{\quad}[\tt File]
   \setlength\itemsep{0pt}
   \item[File] a file to which the 2D array of 3D points is to be output (\texttt{sf\_file}). 
   \item[v]    an array of 2D points which is to be output (\texttt{pt3d}). 
   \item[n1]   size of one axis of the 2D array (\texttt{size\_t}).
   \item[n2]	   size of the other axis of the 2D array (\texttt{size\_t}). 
   \item[k]    a number, which if equal to 4, indicates that the value of the 3D points must also be included (\texttt{int}).     
\end{desclist}




\subsection{{pt2dread1}}
Reads a 1D array of 2D points from a file. It can be used to define the source or receiver arrays, for example.

\subsubsection*{Call}
\begin{verbatim}pt2dread1(F, v, n1, k);\end{verbatim}

\subsubsection*{Definition}
\begin{verbatim}
void pt2dread1(sf_file F, pt2d *v, size_t n1, int k)
/*< input point2d 1-D vector >*/
{
   ...
}
\end{verbatim}

\subsubsection*{Input parameters}
\begin{desclist}{\tt }{\quad}[\tt File]
   \setlength\itemsep{0pt}
   \item[File] a file from which the 1D array of 2D points is to be read (\texttt{sf\_file}). 
   \item[v]    an array of 2D points which is to be read (\texttt{pt2d}). 
   \item[n1]   size of the array (\texttt{size\_t}). 
   \item[k]    a number, which if equal to 3, indicates that the value of the 2D points must also be included (\texttt{int}).     
\end{desclist}




\subsection{{pt2dread2}}
Reads a 2D array of 2D points from a file. It can be used to define the source or receiver arrays, for example.

\subsubsection*{Call}
\begin{verbatim}pt2dread2(F, v, n1, n2, k);\end{verbatim}

\subsubsection*{Definition}
\begin{verbatim}
void pt2dread1(sf_file F, pt2d *v, size_t n1, int k)
/*< input point2d 1-D vector >*/
{
   ...
}
\end{verbatim}   

\subsubsection*{Input parameters}
\begin{desclist}{\tt }{\quad}[\tt File]
   \setlength\itemsep{0pt}
   \item[File] a file from which the 2D array of 2D points is to be read (\texttt{sf\_file}). 
   \item[v]    an array of 2D points which is to be read (\texttt{pt2d}). 
   \item[n1]   size of one axis of the 2D array (\texttt{size\_t}).
   \item[n2]   size of the other axis of the 2D array (\texttt{size\_t}). 
   \item[k]    a number, which if equal to 3, indicates that the value of the 2D points must also be included (\texttt{int}).     
\end{desclist}




\subsection{{pt3dread1}}
Reads a 1D array of 3D points from a file. It can be used to define the source or receiver arrays, for example.

\subsubsection*{Call}
\begin{verbatim}pt3dread1(F, v, n1, k);\end{verbatim}

\subsubsection*{Definition}
\begin{verbatim}
void pt3dread1(sf_file F, pt3d *v, size_t n1, int k)
/*< input point3d 1-D vector >*/
{
   ...
}
\end{verbatim}

\subsubsection*{Input parameters}
\begin{desclist}{\tt }{\quad}[\tt File]
   \setlength\itemsep{0pt}
   \item[File] a file from which the 1D array of 3D points is to be read (\texttt{sf\_file}). 
   \item[v]    an array of 1D points which is to be read (\texttt{pt3d}). 
   \item[n1]   size of one axis of the 3D array (\texttt{size\_t}). 
   \item[k]    a number, which if equal to 4, indicates that the value of the 3D points must also be included (\texttt{int}).     
\end{desclist}




\subsection{{pt3dread2}}
Reads a 2D array of 3D points from a file. It can be used to define the source or receiver arrays, for example.

\subsubsection*{Call}
\begin{verbatim}pt3dread2(F, v, n1, n2, k);\end{verbatim}

\subsubsection*{Definition}
\begin{verbatim}
void pt3dread2(sf_file F, pt3d **v, size_t n1, size_t n2, int k)
/*< input point3d 2-D vector >*/
{
    ...
}
\end{verbatim}

\subsubsection*{Input parameters}
\begin{desclist}{\tt }{\quad}[\tt File]
   \setlength\itemsep{0pt}
   \item[File] a file from which the 2D array of 3D points is to be read (\texttt{sf\_file}). 
   \item[v]    an array of 2D points which is to be read (\texttt{pt3d}). 
   \item[n1]   size of one axis of the 2D array (\texttt{size\_t}).
   \item[n2]   size of the other axis of the 2D array (\texttt{size\_t}). 
   \item[k]    a number, which if equal to 4, indicates that the value of the 3D points must also be included (\texttt{int}).     
\end{desclist}




\subsection{{pt2dalloc1}}
Allocates memory for 1D array of 2D points.

\subsubsection*{Call}
\begin{verbatim}ptr = pt2dalloc1(n1);\end{verbatim}

\subsubsection*{Definition}
\begin{verbatim}
pt2d* pt2dalloc1( size_t n1)
/*< alloc point2d 1-D vector >*/
{
   ...
}
\end{verbatim}

\subsubsection*{Input parameters}
\begin{desclist}{\tt }{\quad}[\tt ]
   \setlength\itemsep{0pt}
   \item[n1]	size of the array (\texttt{size\_t}).
\end{desclist}

\subsubsection*{Output}
\begin{desclist}{\tt }{\quad}[\tt ]
   \setlength\itemsep{0pt}
   \item[ptr] pointer to memory (\texttt{pt2d*}).
\end{desclist}




\subsection{{pt2dalloc2}}
Allocates memory for 2D array of 2D points.

\subsubsection*{Call}
\begin{verbatim}ptr = pt2dalloc2(n1,n2);\end{verbatim}

\subsubsection*{Definition}
\begin{verbatim}
pt2d** pt2dalloc2( size_t n1, size_t n2)
/*< alloc point2d 2-D vector >*/
{
   ...
}
\end{verbatim}

\subsubsection*{Input parameters}
\begin{desclist}{\tt }{\quad}[\tt n2]
   \setlength\itemsep{0pt}
   \item[n1]	size one axis of the 2D array (\texttt{size\_t}).
   \item[n2]	size of the other axis of the 2D array (\texttt{size\_t}).
\end{desclist}

\subsubsection*{Output}
\begin{desclist}{\tt }{\quad}[\tt ptr]
   \setlength\itemsep{0pt}
   \item[ptr] pointer to memory (\texttt{pt2d**}).
\end{desclist}





\subsection{{pt2dalloc3}}
Allocates memory for 3D array of 2D points.

\subsubsection*{Call}
\begin{verbatim}
pt2d*** pt2dalloc3(n1, n2, n3);\end{verbatim}

\subsubsection*{Definition}
\begin{verbatim}
pt2d*** pt2dalloc3(size_t n1, size_t n2, size_t n3)
/*< alloc point2d 3-D vector >*/
{
    ...
}
\end{verbatim}

\subsubsection*{Input parameters}
\begin{desclist}{\tt }{\quad}[\tt n3]
   \setlength\itemsep{0pt}
   \item[n1] size of first axis of the 3D array (\texttt{size\_t}).
   \item[n2] size of the second axis of the 3D array (\texttt{size\_t}).
   \item[n3] size of the third axis of the 3D array (\texttt{size\_t}).
\end{desclist}

\subsubsection*{Output}
\begin{desclist}{\tt }{\quad}[\tt ptr]
   \setlength\itemsep{0pt}
   \item[ptr] pointer to memory (\texttt{pt2d***}).
\end{desclist}




\subsection{{pt3dalloc1}}
Allocates memory for 1D array of 3D points.

\subsubsection*{Call}
\begin{verbatim}ptr = pt3dalloc1(n1);\end{verbatim}

\subsubsection*{Definition}
\begin{verbatim}
pt3d* pt3dalloc1( size_t n1)
/*< alloc point3d 1-D vector >*/
{
   ...
}
\end{verbatim}

\subsubsection*{Input parameters}
\begin{desclist}{\tt }{\quad}[\tt ]
   \setlength\itemsep{0pt}
   \item[n1]	size of the array (\texttt{size\_t}).
\end{desclist}

\subsubsection*{Output}
\begin{desclist}{\tt }{\quad}[\tt ptr]
   \setlength\itemsep{0pt}
   \item[ptr] pointer to memory (\texttt{pt3d*}).
\end{desclist}




\subsection{{pt3dalloc2}}
Allocates memory for 2D array of 3D points.

\subsubsection*{Call}
\begin{verbatim}
ptr = pt3dalloc2(n1,n2);\end{verbatim}

\subsubsection*{Definition}
\begin{verbatim}
pt3d** pt3dalloc2( size_t n1, size_t n2)
/*< alloc point3d 2-D vector >*/
{
   ...   
}
\end{verbatim}

\subsubsection*{Input parameters}
\begin{desclist}{\tt }{\quad}[\tt n2]
   \setlength\itemsep{0pt}
   \item[n1]	size of one axis of the 2D array (\texttt{size\_t}).
   \item[n2]	size of the other axis of the 2D array (\texttt{size\_t}).
\end{desclist}

\subsubsection*{Output}
\begin{desclist}{\tt }{\quad}[\tt ptr]
   \setlength\itemsep{0pt}
   \item[ptr] pointer to memory (\texttt{pt3d**}).
\end{desclist}





\subsection{{pt3dalloc3}}
Allocates memory for 3D array of 3D points.

\subsubsection*{Call}
\begin{verbatim}ptr =  pt3dalloc3(n1, n2, n3); \end{verbatim}

\subsubsection*{Definition}
\begin{verbatim}
pt3d*** pt3dalloc3(size_t n1, size_t n2, size_t n3)
/*< alloc point3d 3-D vector >*/
{
   ...
}
\end{verbatim}

\subsubsection*{Input parameters}
\begin{desclist}{\tt }{\quad}[\tt n3]
   \setlength\itemsep{0pt}
   \item[n1]	size of first axis of the 3D array (\texttt{size\_t}).
   \item[n2]	size of the second axis of the 3D array (\texttt{size\_t}).
   \item[n3]	size of the third axis of the 3D array (\texttt{size\_t}).
\end{desclist}

\subsubsection*{Output}
\begin{desclist}{\tt }{\quad}[\tt ptr]
   \setlength\itemsep{0pt}
   \item[ptr] pointer to memory (\texttt{pt3d***}).
\end{desclist}



      % Construction of points
   \begin{figure}
\centering
$
\left(\begin{array}{c}
 0 \\
 1 \\
 2 \\
 3 \\
 4 \\
 5 \\
 0  
\end{array}\right),\;\;\;\;\;\;
\left(\begin{array}{c}
 0 \\
 2 \\
 4 \\
 1 \\
 3 \\
 5 \\
 0 
\end{array}\right)
$
\caption{Overlay of the owning process ranks of a vector of height 7 
distributed over a 2 $\times$ 3 process grid in the $[V_C,\star]$ vector 
distribution (left) and the $[V_R,\star]$ vector distribution (right).}
\label{fig:vector}
\end{figure}
     % Construction of vectors
   \section{Conversion between line and Cartesian coordinates of a vector (decart.c)}




\subsection{{sf\_line2cart}}\label{sec:sf_line2cart}
Converts the line coordinates to Cartesian coordinates.


\subsubsection*{Call}
\begin{verbatim}sf_line2cart(dim, nn, i, ii);\end{verbatim}


\subsubsection*{Definition}
\begin{verbatim}
void sf_line2cart(int dim       /* number of dimensions */, 
                  const int* nn /* box size [dim] */, 
                  int i         /* line coordinate */, 
                  int* ii       /* cartesian coordinates [dim] */)
/*< Convert line to Cartesian >*/
{
   ...
}
\end{verbatim}


\subsubsection*{Input parameters}
\begin{desclist}{\tt }{\quad}[\tt dim]
   \setlength\itemsep{0pt}
   \item[dim] number of dimensions (\texttt{int}).  
   \item[nn]  box size (size of the data file) (\texttt{const int*}).  
   \item[i]   the line coordinate (\texttt{int}).  
   \item[ii]  the Cartesian coordinates (\texttt{int*}).  
\end{desclist}




\subsection{{sf\_cart2line}}
Converts the Cartesian coordinates to line coordinate.


\subsubsection*{Call}
\begin{verbatim}int sf_cart2line(dim, nn, ii);\end{verbatim}


\subsubsection*{Definition}
\begin{verbatim}
int sf_cart2line(int dim       /* number of dimensions */, 
                 const int* nn /* box size [dim] */, 
                 const int* ii /* cartesian coordinates [dim] */) 
/*< Convert Cartesian to line >*/
{
   ...
}
\end{verbatim}


\subsubsection*{Input parameters}
\begin{desclist}{\tt }{\quad}[\tt dim]
   \setlength\itemsep{0pt}
   \item[dim] number of dimensions (\texttt{int}).  
   \item[nn]  box size (size of the data file) (\texttt{const int*}).  
   \item[ii]  the Cartesian coordinates (\texttt{int*}).  
\end{desclist}

\subsubsection*{Output}
\begin{desclist}{\tt }{\quad}[\tt ]
   \setlength\itemsep{0pt}
   \item[i] line coordinate. It is of type \texttt{int}.
\end{desclist}




\subsection{{sf\_first\_index}}\label{sec:sf_first_index}
Returns the first index for a particular dimension.

\subsubsection*{Call}
\begin{verbatim}sf_first_index (i, j, dim, n, s);\end{verbatim}


\subsubsection*{Definition}
\begin{verbatim}
int sf_first_index (int i        /* dimension [0...dim-1] */, 
                    int j        /* line coordinate */, 
                    int dim      /* number of dimensions */, 
                    const int *n /* box size [dim] */, 
                    const int *s /* step [dim] */)
/*< Find first index for multidimensional transforms >*/
{
   ... 
}
\end{verbatim}


\subsubsection*{Input parameters}
\begin{desclist}{\tt }{\quad}[\tt dim]
   \setlength\itemsep{0pt}
   \item[i]   the dimension (\texttt{int}).  
   \item[j]   the line coordinate (\texttt{int}).  
   \item[dim] number of dimensions (\texttt{int}).  
   \item[n]   box size (size of the data file) (\texttt{const int*}).  
   \item[s]   the step size (\texttt{const int*}).  
\end{desclist}

\subsubsection*{Output}
\begin{desclist}{\tt }{\quad}[\tt ]
   \setlength\itemsep{0pt}
   \item[i0] first index for the given dimension. It is of type \texttt{int}.
\end{desclist}




\subsection{{sf\_large\_line2cart}}
Converts the line coordinate to Cartesian coordinates. It works exactly like \hyperref[sec:sf_line2cart]{\texttt{sf\_line2cart}} but in this one the line and Cartesian coordinates are of type \texttt{off\_t}, which means that they are given in terms of the offset in bytes in the data file.

\subsubsection*{Call}
\begin{verbatim}sf_large_line2cart(dim, nn, i, ii);\end{verbatim}

\subsubsection*{Definition}
\begin{verbatim}
void sf_large_line2cart(int dim         /* number of dimensions */, 
                        const off_t* nn /* box size [dim] */, 
                        off_t i         /* line coordinate */, 
                        off_t* ii       /* cartesian coordinates [dim] */)
/*< Convert line to Cartesian >*/
{
   ...
}
\end{verbatim}


\subsubsection*{Input parameters}
\begin{desclist}{\tt }{\quad}[\tt dim]
   \setlength\itemsep{0pt}
   \item[dim] number of dimensions (\texttt{int}). 
   \item[nn]  box size (size of the data file) (\texttt{const off\_t*}).  
   \item[i]   the line coordinate (\texttt{off\_t}). 
   \item[ii]  the Cartesian coordinates (\texttt{off\_t*}).  
\end{desclist}




\subsection{{sf\_large\_cart2line}}
Converts the Cartesian coordinates to line coordinate. It works exactly like \hyperref[sec:sf_line2cart]{\texttt{sf\_line2cart}} but in this one the line and Cartesian coordinates are of type \texttt{off\_t}, which means that they are given in terms of the offset in bytes in the data file.


\subsubsection*{Call}
\begin{verbatim}sf_large_cart2line(int, nn, ii);\end{verbatim}


\subsubsection*{Definition}
\begin{verbatim}
off_t sf_large_cart2line(int dim         /* number of dimensions */, 
                         const off_t* nn /* box size [dim] */, 
                         const off_t* ii /* cartesian coordinates [dim] */) 
/*< Convert Cartesian to line >*/
{
   ...
}
\end{verbatim}


\subsubsection*{Input parameters}
\begin{desclist}{\tt }{\quad}[\tt dim]
   \setlength\itemsep{0pt}
   \item[dim] number of dimensions (\texttt{int}). 
   \item[nn]  box size (size of the data file) (\texttt{const off\_t*}).  
   \item[ii]  the Cartesian coordinates (\texttt{const off\_t*}).  
\end{desclist}

\subsubsection*{Output}
\begin{desclist}{\tt }{\quad}[\tt ]
   \setlength\itemsep{0pt}
   \item[i] line coordinate. It is of type \texttt{off\_t}.
\end{desclist}




\subsection{{sf\_large\_first\_index}}
Returns the first index for a particular dimension. It works exactly like \hyperref[sec:sf_first_index]{\texttt{sf\_first\_index}} but in this one the line coordinate, box size, step size and the first index are of type \texttt{off\_t}, which means that they are given in terms of the offset in bytes in the data file.

\subsubsection*{Call}
\begin{verbatim}sf_large_first_index (i, j, dim, n, s);\end{verbatim}


\subsubsection*{Definition}
\begin{verbatim}
off_t sf_large_first_index (int i          /* dimension [0...dim-1] */, 
                            off_t j        /* line coordinate */, 
                            int dim        /* number of dimensions */, 
                            const off_t *n /* box size [dim] */, 
                            const off_t *s /* step [dim] */)
/*< Find first index for multidimensional transforms >*/
{
   ...    
}
\end{verbatim}


\subsubsection*{Input parameters}
\begin{desclist}{\tt }{\quad}[\tt dim]
   \setlength\itemsep{0pt}
   \item[i ]  the dimension (\texttt{int}). 
   \item[j]   the line coordinate (\texttt{off\_t}). 
   \item[dim] number of dimensions (\texttt{int}). 
   \item[n]   box size (size of the data file) (\texttt{const off\_t*}).  
   \item[s]   the step size (\texttt{const off\_t*}).  
\end{desclist}

\subsubsection*{Output}
\begin{desclist}{\tt }{\quad}[\tt ]
   \setlength\itemsep{0pt}
   \item[i0] first index for the given dimension. It is of type \texttt{int}.
\end{desclist}




     % Conversion between line and Cartesian coordinates of a vector
   \section{Axes (axa.c)}\label{sec:axa.c}




\subsection{{sf\_maxa}}
Creates a simple axis.

\subsubsection*{Call}
\begin{verbatim}AA = sf_maxa(n, o, d);\end{verbatim}

\subsubsection*{Definition}
\begin{verbatim}
sf_axis sf_maxa(int n   /* length */, 
                float o /* origin */, 
                float d /* sampling */)
/*< make a simple axis >*/
{
   ...
}
\end{verbatim}

\subsubsection{Input parameters}
\begin{desclist}{\tt }{\quad}[\tt ]
   \setlength\itemsep{0pt}
   \item[n] length of the axis (\texttt{int}).  
   \item[o] origin of the axis (\texttt{float}).  
   \item[d] sampling of the axis (\texttt{float}).  
\end{desclist}

\subsubsection*{Output}
\begin{desclist}{\tt }{\quad}[\tt ]
   \setlength\itemsep{0pt}  
   \item[AA] the axis. It is of type \texttt{sf\_axis}.
\end{desclist}




\subsection{{sf\_iaxa}}\label{sec:sf_iaxa}
Reads an axis from the file which is given in the input.

\subsubsection*{Call}
\begin{verbatim}AA = sf_iaxa(FF, i);\end{verbatim}

\subsubsection*{Definition}
\begin{verbatim}
sf_axis sf_iaxa(sf_file FF, int i) 
/*< read axis i >*/
{
   ...
}
\end{verbatim}

\subsubsection{Input parameters}
\begin{desclist}{\tt }{\quad}[\tt FF]
   \setlength\itemsep{0pt}
   \item[FF] the file from which the axis is to be read (\texttt{sf\_file}).  
   \item[i]  a number which specified which axis is to be read, for example \texttt{n1}, \texttt{n2}, \texttt{n3} etc (\texttt{int}).  
\end{desclist}

\subsubsection*{Output}
\begin{desclist}{\tt }{\quad}[\tt ]
   \setlength\itemsep{0pt}  
   \item[AA] location where the axis is stored. It is of type \texttt{sf\_axis}.
\end{desclist}




\subsection{{sf\_oaxa}}\label{sec:sf_oaxa}
Writes an axis, from the input location, to the file, which is also given in the input.

\subsubsection*{Call}
\begin{verbatim}sf_oaxa(FF, AA, i);\end{verbatim}

\subsubsection*{Definition}
\begin{verbatim}
void sf_oaxa(sf_file FF, const sf_axis AA, int i) 
/*< write axis i >*/
{
   ...
}
\end{verbatim}

\subsubsection{Input parameters}
\begin{desclist}{\tt }{\quad}[\tt AA]
   \setlength\itemsep{0pt}
   \item[FF] the file in which the axis is to be written (\texttt{sf\_file}).  
   \item[AA] the location from where the axis is to be read. Must be of type \texttt{const sf\_axis}.  
   \item[i]  a number which specified which axis is to be read, for example \texttt{n1}, \texttt{n2}, \texttt{n3} etc (\texttt{int}).  
\end{desclist}




\subsection{{sf\_raxa}}
Prints the information about the axis on the screen.

\subsubsection*{Call}
\begin{verbatim}sf_raxa(AA);\end{verbatim}

\subsubsection*{Definition}
\begin{verbatim}  
void sf_raxa(const sf_axis AA) 
/*< report information on axis AA >*/
{    
   ...
}
\end{verbatim}

\subsubsection{Input parameters}
\begin{desclist}{\tt }{\quad}[\tt ]
   \setlength\itemsep{0pt}
   \item[AA] the axis about which the information is required. Must be of type \texttt{const sf\_axis}.  
\end{desclist}




\subsection{{sf\_n}}\label{sec:sf_n}
Provides access to the length of the axis.

\subsubsection*{Call}
\begin{verbatim}AA->n = sf_n(AA);\end{verbatim}

\subsubsection*{Definition}
\begin{verbatim}  
int sf_n(const sf_axis AA) 
/*< access axis length >*/
{
   ...
}
\end{verbatim}

\subsubsection{Input parameters}
\begin{desclist}{\tt }{\quad}[\tt ]
   \setlength\itemsep{0pt}
   \item[AA] the axis whose length is to be accessed. Must be of type \texttt{const sf\_axis}.  
\end{desclist}

\subsubsection*{Output}
\begin{desclist}{\tt }{\quad}[\tt ]
   \setlength\itemsep{0pt}  
   \item[AA->n] length of the axis. It is of type \texttt{int}.
\end{desclist}




\subsection{{sf\_o}}
Provides access to the origin of the axis.

\subsubsection*{Call}
\begin{verbatim}AA->o = sf_o(AA);\end{verbatim}

\subsubsection*{Definition}
\begin{verbatim}  
float sf_o(const sf_axis AA) 
/*< access axis origin >*/
{
   ...
}
\end{verbatim}

\subsubsection{Input parameters}
\begin{desclist}{\tt }{\quad}[\tt ]
   \setlength\itemsep{0pt}
   \item[AA] the axis whose length is to be accessed. Must be of type \texttt{const sf\_axis}.  
\end{desclist}

\subsubsection*{Output}
\begin{desclist}{\tt }{\quad}[\tt ]
   \setlength\itemsep{0pt}  
   \item[AA->o] length of the axis. It is of type \texttt{float}.
\end{desclist}




\subsection{{sf\_d}}\label{sec:sf_d}
Provides access to the sampling of the axis.

\subsubsection*{Call}
\begin{verbatim}AA->d = sf_d(AA);\end{verbatim}

\subsubsection*{Definition}
\begin{verbatim}  
float sf_d(const sf_axis AA) 
/*< access axis sampling >*/
{
   ...
}
\end{verbatim}

\subsubsection{Input parameters}
\begin{desclist}{\tt }{\quad}[\tt ]
   \setlength\itemsep{0pt}
   \item[AA] the axis whose length is to be accessed. Must be of type \texttt{const sf\_axis}.  
\end{desclist}

\subsubsection*{Output}
\begin{desclist}{\tt }{\quad}[\tt ]
   \setlength\itemsep{0pt}  
   \item[AA->d] length of the axis. It is of type \texttt{float}.
\end{desclist}




\subsection{{sf\_nod}}
Copies the length, origin and sampling of the input axis to another place which is also an object of type \texttt{sf\_axis}.

\subsubsection*{Call}
\begin{verbatim}BB = sf_nod(AA);\end{verbatim}

\subsubsection*{Definition}
\begin{verbatim}  
sf_axa sf_nod(const sf_axis AA) 
/*< access length, origin, and sampling >*/
{
   ...
}
\end{verbatim}

\subsubsection{Input parameters}
\begin{desclist}{\tt }{\quad}[\tt ]
   \setlength\itemsep{0pt}
   \item[AA] the axis whose length, origin and sampling is to be accessed. Must be of type \texttt{const sf\_axis}.  
\end{desclist}

\subsubsection*{Output}
\begin{desclist}{\tt }{\quad}[\tt ]
   \setlength\itemsep{0pt}  
   \item[BB] the location where the length, origin and sampling are copied. It is of type \texttt{sf\_axis}.
\end{desclist}




\subsection{{sf\_setn}}
Changes the length of the axis.

\subsubsection*{Call}
\begin{verbatim}AA->n = sf_setn(AA, n);\end{verbatim}

\subsubsection*{Definition}
\begin{verbatim}  
void sf_setn(sf_axis AA, int n)
/*< change axis length >*/
{ AA->n=n; }
\end{verbatim}

\subsubsection{Input parameters}
\begin{desclist}{\tt }{\quad}[\tt AA]
   \setlength\itemsep{0pt}
   \item[AA] the axis whose length is to be changed (\texttt{sf\_axis}).  
   \item[n]  the new length which is to be set (\texttt{int}).  
\end{desclist}




\subsection{{sf\_seto}}
Changes the origin of the axis.

\subsubsection*{Call}
\begin{verbatim}AA->o = sf_seto(AA, o);\end{verbatim}

\subsubsection*{Definition}
\begin{verbatim}  
void sf_seto(sf_axis AA, float o)
/*< change axis origin >*/
{
   ...
}
\end{verbatim}

\subsubsection{Input parameters}
\begin{desclist}{\tt }{\quad}[\tt AA]
   \setlength\itemsep{0pt}
   \item[AA] the axis whose origin is to be changed (\texttt{sf\_axis}).  
   \item[o]  the new origin which is to be set (\texttt{float}).  
\end{desclist}




\subsection{{sf\_setd}}
Changes the sampling of the axis.

\subsubsection*{Call}
\begin{verbatim}AA->d = sf_setd(AA, d);\end{verbatim}

\subsubsection*{Definition}
\begin{verbatim}  
void sf_setd(sf_axis AA, float d)
/*< change axis sampling >*/
{
   ...
}
\end{verbatim}

\subsubsection{Input parameters}
\begin{desclist}{\tt }{\quad}[\tt ]
   \setlength\itemsep{0pt}
   \item[AA] the axis whose sampling is to be changed (\texttt{sf\_axis}).  
   \item[o]  the new sampling which is to be set (\texttt{float}).  
\end{desclist}




\subsection{{sf\_setlabel}}
Changes the label of the axis.

\subsubsection*{Call}
\begin{verbatim}sf_setlabel(AA, label);\end{verbatim}

\subsubsection*{Definition}
\begin{verbatim}  
void sf_setlabel(sf_axis AA, const char* label)
/*< change axis label >*/
{
   ...
}
\end{verbatim}

\subsubsection{Input parameters}
\begin{desclist}{\tt }{\quad}[\tt label]
   \setlength\itemsep{0pt}
   \item[AA]    the axis whose label is to be changed (\texttt{sf\_axis}).  
   \item[label] the new label which is to be set (\texttt{const char*}).  
\end{desclist}




\subsection{{sf\_setunit}}
Changes the unit of the axis.

\subsubsection*{Call}
\begin{verbatim}sf_setunit(AA, unit);\end{verbatim}

\subsubsection*{Definition}
\begin{verbatim}  
void sf_setunit(sf_axis AA, const char* unit)
/*< change axis unit >*/
{
   ...
}
\end{verbatim}

\subsubsection{Input parameters}
\begin{desclist}{\tt }{\quad}[\tt unit]
   \setlength\itemsep{0pt}
   \item[AA]   the axis whose unit is to be changed (\texttt{sf\_axis}).  
   \item[unit] the new unit which is to be set (\texttt{const char*}).  
\end{desclist}

        % Axes

\chapter{Miscellaneous}\label{sec:misc}
   \section{sharpening (sharpen.c)}




\subsection{{sf\_sharpen\_init}}
Initializes the sharpening operator by allocating the required and initializing the required operators.

\subsubsection*{Call}
\begin{verbatim}sf_sharpen_init(n1, perc);\end{verbatim}

\subsubsection*{Definition}
\begin{verbatim}
void sf_sharpen_init(int n1     /* data size */,
                     float perc /* quantile percentage */) 
/*< initialize >*/
{
   ...
}
\end{verbatim}

\subsubsection*{Input parameters}
\begin{desclist}{\tt }{\quad}[\tt perc]
   \setlength\itemsep{0pt}
   \item[n1]	   size  of the data (\texttt{int}).  
   \item[perc] the quantile percentage (\texttt{float}).  
\end{desclist}




\subsection{{sf\_sharpen\_close}}
Frees the allocated memory for the sharpening calculation.

\subsubsection*{Call}
\begin{verbatim}sf_sharpen_close();\end{verbatim}

\subsubsection*{Definition}
\begin{verbatim}
void sf_sharpen_close(void)
/*< free allocated storage >*/
{
   ...
}
\end{verbatim}




\subsection{{sf\_sharpen}}
Computes the weights for the sharpening regularization.


\subsubsection*{Call}
\begin{verbatim}wp = sf_sharpen(pp);\end{verbatim}

\subsubsection*{Definition}
\begin{verbatim}
float sf_sharpen(const float *pp) 
/*< compute weight for sharpening regularization >*/
{
   ...
}
\end{verbatim}

\subsubsection*{Input parameters}
\begin{desclist}{\tt }{\quad}[\tt ]
   \setlength\itemsep{0pt}
   \item[pp] an array for which the weights are to be calculated (\texttt{const float*}).  
\end{desclist}

\subsubsection*{Output}
\begin{desclist}{\tt }{\quad}[\tt ]
   \setlength\itemsep{0pt}
   \item[wp] weights for sharpening regularization. It is of type \texttt{float}.
\end{desclist}




\subsection{{sf\_csharpen}}
Computes the weights for the sharpening regularization for complex numbers.

\subsubsection*{Call}
\begin{verbatim}sf_csharpen(pp);\end{verbatim}

\subsubsection*{Definition}
\begin{verbatim}
void sf_csharpen(const sf_complex *pp) 
/*< compute weight for sharpening regularization >*/
{
   ...
}
\end{verbatim}

\subsubsection*{Input parameters}
\begin{desclist}{\tt }{\quad}[\tt ]
   \setlength\itemsep{0pt}
   \item[pp] an array for which the weights are to be calculated (\texttt{sf\_complex}).  
\end{desclist}





    % Sharpening
   \section{Sharpening inversion added Bregman iteration (sharpinv.c)}




\subsection{{sf\_csharpinv}}
Performs the sharp inversion to estimate the model from the data, for complex numbers.

\subsubsection*{Call}
\begin{verbatim}
sf_csharpinv(oper, scale, niter, ncycle, perc, verb, nq, np, qq, pp);
\end{verbatim}

\subsubsection*{Definition}
\begin{verbatim}
void sf_csharpinv(sf_coperator oper /* inverted operator */, 
                  float scale       /* extra operator scaling */,
                  int niter         /* number of outer iterations */,
                  int ncycle        /* number of iterations */,
                  float perc        /* sharpening percentage */,
                  bool verb         /* verbosity flag */,
                  int nq, int np    /* model and data size */,
                  sf_complex *qq    /* model */, 
                  sf_complex *pp    /* data */)
/*< sharp inversion for complex-valued operators >*/
{
   ...
}
\end{verbatim}

\subsubsection*{Input parameters}
\begin{desclist}{\tt }{\quad}[\tt ncycle]
   \setlength\itemsep{0pt}
   \item[oper]   the inverted operator (\texttt{sf\_operator}). 
   \item[scale]  extra operator scaling (\texttt{float}). 
   \item[niter]  number of outer iterations (\texttt{int}). 
   \item[ncycle] number of iterations (\texttt{int}). 
   \item[perc]   sharpening percentage (\texttt{float}). 
   \item[verb]   verbosity flag (\texttt{bool}). 
   \item[nq]     size of the model (\texttt{int}).
   \item[np]     size of the data (\texttt{int}). 
   \item[qq]     the model. Must be of type \	texttt{sf\_complex*}.
   \item[pp]     the data (\texttt{sf\_complex*}).
\end{desclist}




\subsection{{sf\_sharpinv}}
Performs the sharp inversion to estimate the model from the data.

\subsubsection*{Call}
\begin{verbatim}
void sf_sharpinv(oper, scale, niter, ncycle, perc, verb, nq, np, qq, pp);
\end{verbatim}

\subsubsection*{Definition}
\begin{verbatim}
void sf_sharpinv(sf_operator oper  /* inverted operator */, 
                 float scale       /* extra operator scaling */,
                 int niter         /* number of outer iterations */,
                 int ncycle        /* number of iterations */,
                 float perc        /* sharpening percentage */,
                 bool verb         /* verbosity flag */,
                 int nq, int np    /* model and data size */,
                 float *qq         /* model */, 
                 float *pp         /* data */)
/*< sharp inversion for real-valued operators >*/
{
   ...
}
\end{verbatim}

\subsubsection*{Input parameters}
\begin{desclist}{\tt }{\quad}[\tt ncycle]
   \setlength\itemsep{0pt}
   \item[oper]   the inverted operator (\texttt{sf\_operator}). 
   \item[scale]  extra operator scaling (\texttt{float}). 
   \item[niter]  number of outer iterations (\texttt{int}). 
   \item[ncycle] number of iterations (\texttt{int}). 
   \item[perc]   sharpening percentage (\texttt{float}). 
   \item[verb]   verbosity flag (\texttt{bool}). 
   \item[nq]     size of the model (\texttt{int}).
   \item[np]     size of the data (\texttt{int}). 
   \item[qq]     the model (\texttt{float*}).
   \item[pp]     the data (\texttt{float*}).
\end{desclist}



   % Sharpening inversion added Bregman iteration

\chapter{System}\label{sec:system}
   \section{Priority queue (heap sorting) (pqueue.c)}




\subsection{{sf\_pqueue\_init}}
Initializes the heap with the maximum size given in the input.

\subsubsection*{Call}
\begin{verbatim}sf_pqueue_init (n);\end{verbatim}

\subsubsection*{Definition}
\begin{verbatim}
void sf_pqueue_init (int n)
/*< Initialize heap with the maximum size >*/
{
   ...
}
\end{verbatim}

\subsubsection*{Input parameters}
\begin{desclist}{\tt }{\quad}[\tt ]
   \setlength\itemsep{0pt}
   \item[n] maximum size of the heap (\texttt{int}).  
\end{desclist}




\subsection{{sf\_pqueue\_start}}
Sets the starting values for the queue.

\subsubsection*{Call}
\begin{verbatim}sf_pqueue_start ();\end{verbatim}

\subsubsection*{Definition}
\begin{verbatim}
void sf_pqueue_start (void)
/*< Set starting values >*/
{
   ...
}
\end{verbatim}




\subsection{{sf\_pqueue\_close}}
Frees the space allocated by \texttt{sf\_pqueue\_init}.

\subsubsection*{Call}
\begin{verbatim}sf_pqueue_close();\end{verbatim}

\subsubsection*{Definition}
\begin{verbatim}
void sf_pqueue_close (void)
/*< Free the allocated storage >*/
{
   ...
}
\end{verbatim}




\subsection{{sf\_pqueue\_insert}}
Inserts an element in the queue. The smallest element goes first.

\subsubsection*{Call}
\begin{verbatim}sf_pqueue_insert (v);\end{verbatim}

\subsubsection*{Definition}
\begin{verbatim}
void sf_pqueue_insert (float* v)
/*< Insert an element (smallest first) >*/
{
   ...
}
\end{verbatim}

\subsubsection*{Input parameters}
\begin{desclist}{\tt }{\quad}[\tt ]
   \setlength\itemsep{0pt}
   \item[v] element to be inserted, smallest first (\texttt{float*}).  
\end{desclist}




\subsection{{sf\_pqueue\_insert2}}
Inserts an element in the queue. The largest element goes first.

\subsubsection*{Call}
\begin{verbatim}sf_pqueue_insert2 (v);\end{verbatim}

\subsubsection*{Definition}
\begin{verbatim}
void sf_pqueue_insert2 (float* v)
/*< Insert an element (largest first) >*/
{
   ...
}
\end{verbatim}

\subsubsection*{Input parameters}
\begin{desclist}{\tt }{\quad}[\tt ]
   \setlength\itemsep{0pt}
   \item[v] element to be inserted, largest first (\texttt{float*}).  
\end{desclist}




\subsection{{sf\_pqueue\_extract}}
Extracts the smallest element from the list.

\subsubsection*{Call}
\begin{verbatim}v = sf_pqueue_extract();\end{verbatim}

\subsubsection*{Definition}
\begin{verbatim}
float* sf_pqueue_extract (void)
/*< Extract the smallest element >*/
{
    unsigned int c;
    int n;
   ...
    return v;
}
\end{verbatim}

\subsubsection*{Output}
\begin{desclist}{\tt }{\quad}[\tt ]
   \setlength\itemsep{0pt}
   \item[v] the extracted smallest element (\texttt{float*}).  
\end{desclist}




\subsection{{sf\_pqueue\_extract2}}
Extracts the largest element from the list.

\subsubsection*{Call}
\begin{verbatim}v = sf_pqueue_extract2();\end{verbatim}

\subsubsection*{Definition}
\begin{verbatim}
float* sf_pqueue_extract2 (void)
/*< Extract the largest element >*/
{
   ...
}
\end{verbatim}

\subsubsection*{Output}
\begin{desclist}{\tt }{\quad}[\tt ]
   \setlength\itemsep{0pt}
   \item[v] the extracted largest element (\texttt{float*}).  
\end{desclist}





\subsection{{sf\_pqueue\_update}}
Updates the heap.

\subsubsection*{Call}
\begin{verbatim}sf_pqueue_update (v);\end{verbatim}

\subsubsection*{Definition}
\begin{verbatim}
void sf_pqueue_update (float **v)
/*< restore the heap: the value has been altered >*/ 
{
   ... 
}
\end{verbatim}

\subsubsection*{Input parameters}
\begin{desclist}{\tt }{\quad}[\tt ]
   \setlength\itemsep{0pt}
   \item[v] elements to be inserted, largest first (\texttt{float**}).  
\end{desclist}





     % Priority queue (heap sorting)
   \section{Simplified system command (system.c)}




\subsection{{sf\_system}}
Runs a system command given to it as an input.

\subsubsection*{Call}
\begin{verbatim}sf_system(command);\end{verbatim}

\subsubsection*{Definition}
\begin{verbatim}
void sf_system(const char *command)
/*< System command >*/
{
   ...
}
\end{verbatim}

\subsubsection*{Input parameters}
\begin{desclist}{\tt }{\quad}[\tt ]
   \setlength\itemsep{0pt}
   \item[command] the command which is to be run on the system (\texttt{const char}).
\end{desclist}



     % Simplified system command
   \section{Generic stack (FILO) structure operations (stack.c)}




\subsection{{sf\_stack\_init}}
Initializes the object of type \texttt{sf\_stack}, that is, it allocates the required memory for the data and also sets the size of the stack.

\subsubsection*{Call}
\begin{verbatim}s = sf_stack_init (size_t size);\end{verbatim}

\subsubsection*{Definition}
\begin{verbatim}
sf_stack sf_stack_init (size_t size)
/*< create a stack >*/
{
   ...
}
\end{verbatim}

\subsubsection*{Input parameters}
\begin{desclist}{\tt }{\quad}[\tt ]
   \setlength\itemsep{0pt}
   \item[size] size of the stack (\texttt{size\_t}).  
\end{desclist}

\subsubsection*{Output}
\begin{desclist}{\tt }{\quad}[\tt ]
   \setlength\itemsep{0pt}
   \item[s] a stack (an object of type \texttt{sf\_stack}). It is of type \texttt{sf\_stack}.
\end{desclist}




\subsection{{sf\_stack\_print}}
Prints the information about the stack on the screen. This may be used for debugging.

\subsubsection*{Call}
\begin{verbatim}sf_stack_print(s);\end{verbatim}

\subsubsection*{Definition}
\begin{verbatim}
void sf_stack_print (sf_stack s)
/*< print out a stack (for debugging) >*/ 
{
   ...
}
\end{verbatim}

\subsubsection*{Input parameters}
\begin{desclist}{\tt }{\quad}[\tt ]
   \setlength\itemsep{0pt}
   \item[s] a stack (an object of type \texttt{sf\_stack}). It is of type \texttt{sf\_stack}.
\end{desclist}




\subsection{{sf\_stack\_get}}
Extracts the length of the stack.

\subsubsection*{Call}
\begin{verbatim}l = sf_stack_get(s);\end{verbatim}

\subsubsection*{Definition}
\begin{verbatim}
int sf_stack_get (sf_stack s) 
/*< extract stack length >*/
{
   ... 
}
\end{verbatim}

\subsubsection*{Call}
\begin{verbatim}sf_stack_set(s, pos);\end{verbatim}

\subsubsection*{Definition}
\begin{verbatim}
void sf_stack_set (sf_stack s, int pos) 
/*< set stack position >*/
{
   ...
}
\end{verbatim}

\subsubsection*{Input parameters}
\begin{desclist}{\tt }{\quad}[\tt ]
   \setlength\itemsep{0pt}
   \item[s] a stack (an object of type \texttt{sf\_stack}). It is of type \texttt{sf\_stack}.
\end{desclist}

\subsubsection*{Output}
\begin{desclist}{\tt }{\quad}[\tt ]
   \setlength\itemsep{0pt}
   \item[s->top - s->entry] length of the stack. It is of type \texttt{int}.
\end{desclist}




\subsection{{sf\_stack\_set}}
Sets the position of the pointer in the stack to the specified in the input \texttt{pos}.

\begin{verbatim}sf_stack_set (s, pos);\end{verbatim}

\begin{verbatim}
void sf_stack_set (sf_stack s, int pos) 
/*< set stack position >*/
{
   ...
}
\end{verbatim}

\subsubsection*{Input parameters}
\begin{desclist}{\tt }{\quad}[\tt pos]
   \setlength\itemsep{0pt}
   \item[s]   a stack (an object of type \texttt{sf\_stack}). It is of type \texttt{sf\_stack}. 
   \item[pos] desired position of the pointer in the stack. It is of type \texttt{int}.
\end{desclist}




\subsection{{sf\_push}}
Inserts the data into the stack.

\subsubsection*{Call}
\begin{verbatim}sf_push(s, data, type);\end{verbatim}

\subsubsection*{Definition}
\begin{verbatim}
void sf_push(sf_stack s, void *data, int type)
/*< push data into stack (requires unique data for each push) >*/
{
   ...
}
\end{verbatim}

\subsubsection*{Input parameters}
\begin{desclist}{\tt }{\quad}[\tt type]
   \setlength\itemsep{0pt}
   \item[s ]   a stack (an object of type \texttt{sf\_stack}). It is of type \texttt{sf\_stack}. 
   \item[data] data which is to be written into the stack. It is of type \texttt{void*}.
   \item[type] type of the data. It is of type \texttt{int}.
\end{desclist}




\subsection{{sf\_pop}}
Extracts the data from the stack.

\subsubsection*{Call}
\begin{verbatim}dat = sf_pop(s);\end{verbatim}

\subsubsection*{Definition}
\begin{verbatim}
void* sf_pop(sf_stack s)
/*< pop data from stack >*/ 
{
   ...
}
\end{verbatim}

\subsubsection*{Input parameters}
\begin{desclist}{\tt }{\quad}[\tt ]
   \setlength\itemsep{0pt}
   \item[s] a stack (an object of type \texttt{sf\_stack}). It is of type \texttt{sf\_stack}.
\end{desclist}

\subsubsection*{Output}
\begin{desclist}{\tt }{\quad}[\tt ]
   \setlength\itemsep{0pt}
   \item[old->data] extracted data from the stack. It is of type \texttt{void*}.
\end{desclist}




\subsection{{sf\_full}}
Tests whether the stack is full or not.

\subsubsection*{Call}
\begin{verbatim} isfull = sf_full(s);\end{verbatim}

\subsubsection*{Definition}
\begin{verbatim}
bool sf_full (sf_stack s)
/*< test if the stack is full >*/
{
   ...
}
\end{verbatim}

\subsubsection*{Input parameters}
\begin{desclist}{\tt }{\quad}[\tt ]
   \setlength\itemsep{0pt}
   \item[s] a stack (an object of type \texttt{sf\_stack}). It is of type \texttt{sf\_stack}.
\end{desclist}

\subsubsection*{Output}
\begin{desclist}{\tt }{\quad}[\tt ]
   \setlength\itemsep{0pt}
   \item[s->top >= s->entry] true, if the stack is full, false otherwise. It is of type \texttt{bool}.
\end{desclist}




\subsection{{sf\_top}}
Returns the data type of the top entry of the stack.

\subsubsection*{Call}
\begin{verbatim}typ = sf_top(s);\end{verbatim}

\subsubsection*{Definition}
\begin{verbatim}
int sf_top(sf_stack s)
/*< return the top type >*/
{
   ...
}
\end{verbatim}

\subsubsection*{Input parameters}
\begin{desclist}{\tt }{\quad}[\tt ]
   \setlength\itemsep{0pt}
   \item[s] a stack (an object of type \texttt{sf\_stack}). It is of type \texttt{sf\_stack}.
\end{desclist}

\subsubsection*{Output}
\begin{desclist}{\tt }{\quad}[\tt ]
   \setlength\itemsep{0pt}
   \item[s->top->type] type of the top entry. It is of type \texttt{int}.
\end{desclist}




\subsection{{sf\_stack\_close}}
Frees the space allocated for the stack.

\subsubsection*{Call}
\begin{verbatim}sf_stack_close(s);\end{verbatim}

\subsubsection*{Definition}
\begin{verbatim}
void sf_stack_close(sf_stack s)
/*< free allocated memory >*/
{
   ...
}
\end{verbatim}

\subsubsection*{Input parameters}
\begin{desclist}{\tt }{\quad}[\tt ]
   \setlength\itemsep{0pt}
   \item[s] a stack (an object of type \texttt{sf\_stack}). It is of type \texttt{sf\_stack}.
\end{desclist}



      % Generic stack (FILO) structure operations
%    \section{Supplying compatibility with the C99 standard (c99.c)}




\subsection{{sf\_cmplx}}
Creates a complex number. The real and imaginary parts are of type \texttt{float}.

\subsubsection*{Input parameters}
\begin{desclist}{\tt }{\quad}[\tt re]
   \setlength\itemsep{0pt}
   \item[re] real part of the complex number (\texttt{float}).
   \item[im] imaginary part of the complex number (\texttt{float}).
\end{desclist}

\subsubsection*{Output}
\begin{desclist}{\tt }{\quad}[\tt c]
   \setlength\itemsep{0pt}
   \item[c] the complex number.
\end{desclist}

\subsubsection*{Definition}
\begin{verbatim}
float complex sf_cmplx(float re, float im)
/*< complex number >*/
{
    float complex c;
    __real__ c = re;
    __imag__ c = im;
    return c;
}
\end{verbatim}




\subsection{{sf\_dcmplx}}
Creates a complex number. The real and imaginary parts are of type \texttt{double}.

\subsubsection*{Input parameters}
\begin{desclist}{\tt }{\quad}[\tt re]
   \setlength\itemsep{0pt}
   \item[re] real part of the complex number (\texttt{double}).
   \item[im] imaginary part of the complex number (\texttt{double}).
\end{desclist}

\subsubsection*{Output}
\begin{desclist}{\tt }{\quad}[\tt ]
   \setlength\itemsep{0pt}
   \item[c] the complex number.
\end{desclist}

\subsubsection*{Definition}
\begin{verbatim}
double complex sf_dcmplx(double re, double im)
/*< complex number >*/
{
    double complex c;
    __real__ c = re;
    __imag__ c = im;
    return c;
}
\end{verbatim}




\subsection{{sf\_cmplx}}
Creates a complex number. The real and imaginary parts are of type \texttt{float}. Definition would return a complex number but of type \texttt{kiss\_fft\_cpx}.

\subsubsection*{Input parameters}
\begin{desclist}{\tt }{\quad}[\tt re]
   \setlength\itemsep{0pt}
   \item[re] real part of the complex number (\texttt{float}).
   \item[im] imaginary part of the complex number (\texttt{float}).
\end{desclist}

\subsubsection*{Output}
\begin{desclist}{\tt }{\quad}[\tt c]
   \setlength\itemsep{0pt}
   \item[c] the complex number. It is of type \texttt{kiss\_fft\_cpx}.
\end{desclist}

\subsubsection*{Definition}
\begin{verbatim}
kiss_fft_cpx sf_cmplx(float re, float im)
/*< complex number >*/
{
    kiss_fft_cpx c;
    c.r = re;
    c.i = im;
    return c;
}
\end{verbatim}




\subsection{{sf\_dcmplx}}
Creates a complex number. The real and imaginary parts are of type \texttt{float}. Definition would return a complex number but of type \texttt{sf\_double\_complex}.

\subsubsection*{Input parameters}
\begin{desclist}{\tt }{\quad}[\tt ]
   \setlength\itemsep{0pt}
   \item[re] real part of the complex number (\texttt{double}).
   \item[im] imaginary part of the complex number (\texttt{double}).
\end{desclist}

\subsubsection*{Output}
\begin{desclist}{\tt }{\quad}[\tt ]
   \setlength\itemsep{0pt}
   \item[c] the complex number. It is of type \texttt{sf\_double\_complex}.
\end{desclist}

\subsubsection*{Definition}
\begin{verbatim}
sf_double_complex sf_dcmplx(double re, double im)
/*< complex number >*/
{
    sf_double_complex c;
    c.r = re;
    c.i = im;
    return c;
}

#endif
/*^*/

#if !defined(__cplusplus) && !defined(SF_HAS_COMPLEX_H)
/*^*/

#if !defined(hpux) && !defined(__hpux)
/*^*/

float copysignf(float x, float y)
/*< float copysign >*/
{ return (float) copysign(x,y);}

#endif
/*^*/

float sqrtf(float x) 
/*< float sqrt >*/
{ return (float) sqrt(x);}

float logf(float x)  
/*< float log >*/
{ return (float) log(x);}

float log10f(float x) 
/*< float log10 >*/
{ return (float) log10(x);}

float expf(float x) 
/*< float exp >*/
{ return (float) exp(x);}

float erff(float x) 
/*< float erf >*/
{ return (float) erf(x);}

float erfcf(float x) 
/*< float erfc >*/
{ return (float) erfc(x);}

#if !defined(hpux) && !defined(__hpux)
/*^*/

float fabsf(float x) 
/*< float fabs >*/
{ return (float) fabs(x);}

#endif
/*^*/

float floorf(float x)
/*< float floor >*/
{ return (float) floor(x);}

float ceilf(float x) 
/*< float ceil >*/
{ return (float) ceil(x);}

float fmodf(float x, float y) 
/*< float fmod >*/
{ return (float) fmod(x,y);}

float cosf(float x) 
/*< float cos >*/
{ return (float) cos(x);}

float sinf(float x) 
/*< float sin >*/
{ return (float) sin(x);}

float tanf(float x) 
/*< float tan >*/
{ return (float) tan(x);}

float acosf(float x) 
/*< float acos >*/
{ return (float) acos(x);}

float asinf(float x) 
/*< float asin >*/
{ return (float) asin(x);}

float atanf(float x) 
/*< float atan >*/
{ return (float) atan(x);}

float atan2f(float x, float y) 
/*< float atan2 >*/
{ return (float) atan2(x,y);}

float coshf(float x) 
/*< float cosh >*/
{ return (float) cosh(x);}

float sinhf(float x) 
/*< float sinh >*/
{ return (float) sinh(x);}

float tanhf(float x) 
/*< float tanh >*/
{ return (float) tanh(x);}

float acoshf(float x) 
/*< float acosh >*/
{ extern double acosh(double x);
return (float) acosh(x);}

float asinhf(float x) 
/*< float asinh >*/
{ extern double asinh(double x);
return (float) asinh(x);}

float atanhf(float x) 
/*< float atanh >*/
{ extern double atanh(double x);
 return (float) atanh(x);}

float powf(float x, float y) 
/*< float pow >*/
{ return (float) pow(x,y);}

float hypotf(float x, float y) 
/*< float hypot >*/
{ extern double hypot(double x, double y);
 return (float) hypot(x,y);}

#if defined(hpux) || defined(__hpux)
/*^*/

static char     *digits = "0123456789abcdefghijklmnopqrstuvwxyz";

long long
strtoll(const char *ptr, const char **endptr, int base)
/*< strtoll replacement >*/
{
        const char      *cp;
        long long        ret;
        char            *dig;
        int              d;

        for (ret = 0, cp = ptr ; *cp && (dig = strchr(digits, *cp)) != NULL  && 
(d = (int)(dig - digits)) < base ; cp++) {
                ret = (ret * base) + d;
        }
        if (endptr != NULL) {
                *endptr = cp;
        }
        return ret;
}

unsigned long long
strtoull(const char *ptr, const char **endptr, int base)
/*< strtoull replacement >*/
{
        const char      *cp;
        unsigned long long       ret;
        char            *dig;
        int              d;

        for (ret = 0, cp = ptr ; *cp && (dig = strchr(digits, *cp)) != NULL  && 
(d = (unsigned int)(dig - digits)) < base ; cp++) {
                ret = (ret * base) + d;
        }
        if (endptr != NULL) {
                *endptr = cp;
        }
        return ret;
}

#define finite(x) isfinite(x)
#endif
/*^*/

#endif
/*^*/

/*      $Id: c99.c 3594 2008-05-14 03:09:16Z sfomel $    */
\end{verbatim}




      % Supplying compatibility with the C99 standard

\cleardoublepage


\printindex
\addcontentsline{toc}{chapter}{Index}
\cleardoublepage






\end{document}  