\section{Frequency-domain filtering (freqfilt.c)}
\index{filtering!frequency domain}




\subsection{{sf\_freqfilt\_init}}\label{sec:sf_freqfilt_init}
Initializes the required variables and allocates the required space for frequency filtering.

\subsubsection*{Call}
\begin{verbatim}sf_freqfilt_init(nfft1, nw1);\end{verbatim}

\subsubsection*{Definition}
\begin{verbatim}
void sf_freqfilt_init(int nfft1 /* time samples (possibly padded) */, 
                      int nw1   /* frequency samples */)
/*< Initialize >*/
{
   ...
}
\end{verbatim}

\subsubsection*{Input parameters}
\begin{desclist}{\tt }{\quad}[\tt nfft1]
   \setlength\itemsep{0pt}
   \item[nfft1] number of time samples (\texttt{int}).  
   \item[nw1]   number of frequency samples (\texttt{int}).  
\end{desclist}




\subsection{{sf\_freqfilt\_set}}
Initializes a zero phase filter.

\subsubsection*{Call}
\begin{verbatim}sf_freqfilt_set(filt);\end{verbatim}

\subsubsection*{Definition}
\begin{verbatim}
void sf_freqfilt_set(float *filt /* frequency filter [nw] */)
/*< Initialize filter (zero-phase) >*/
{
   ...
}
\end{verbatim}

\subsubsection*{Input parameters}
\begin{desclist}{\tt }{\quad}[\tt ]
   \setlength\itemsep{0pt}
   \item[filt] the frequency filter (\texttt{float*}).  
\end{desclist}




\subsection{{sf\_freqfilt\_cset}}
Initializes a non-zero phase filter (filter with complex values).

\subsubsection*{Call}
\begin{verbatim}sf_freqfilt_cset (filt);\end{verbatim}

\subsubsection*{Definition}
\begin{verbatim}
void sf_freqfilt_cset(kiss_fft_cpx *filt /* frequency filter [nw] */)
/*< Initialize filter >*/
{
   ...
}
\end{verbatim}

\subsubsection*{Input parameters}
\begin{desclist}{\tt }{\quad}[\tt ]
   \setlength\itemsep{0pt}
   \item[filt] the frequency filter. Must be of type \texttt{kiss\_fft\_cpx*}.  
\end{desclist}




\subsection{{sf\_freqfilt\_close}}
Frees the space allocated by \hyperref[sec:sf_freqfilt2_init]{\texttt{sf\_freqfilt\_init}}.

\subsubsection*{Call}
\begin{verbatim}sf_freqfilt_close();\end{verbatim}

\subsubsection*{Definition}
\begin{verbatim}
void sf_freqfilt_close(void) 
/*< Free allocated storage >*/
{
   ...
}
\end{verbatim}




\subsection{{sf\_freqfilt}}
Applies the frequency filtering to the input data.

\subsubsection*{Call}
\begin{verbatim}sf_freqfilt(nx, x);\end{verbatim}

\subsubsection*{Definition}
\begin{verbatim}
void sf_freqfilt(int nx, float* x)
/*< Filtering in place >*/
{
   ...
}
\end{verbatim}

\subsubsection*{Input parameters}
\begin{desclist}{\tt }{\quad}[\tt xx]
   \setlength\itemsep{0pt}
   \item[nx] data length (\texttt{int}).  
   \item[x]  the data (\texttt{float*}).  
\end{desclist}




\subsection{{sf\_freqfilt\_lop}}
Applies the frequency filtering to either \texttt{y} or \texttt{x}, depending on whether \texttt{adj} is true of false and then applies the result to \texttt{x} or \texttt{y} as a linear operator.

\subsubsection*{Call}
\begin{verbatim}sf_freqfilt_lop (adj, add, nx, ny, x, y);\end{verbatim}

\subsubsection*{Definition}
\begin{verbatim}
void sf_freqfilt_lop (bool adj, bool add, int nx, int ny, float* x, float* y) 
/*< Filtering as linear operator >*/
{
   ...
}
\end{verbatim}

\subsubsection*{Input parameters}
\begin{desclist}{\tt }{\quad}[\tt add]
   \setlength\itemsep{0pt}
   \item[adj] a parameter to determine whether frequency filter applied to \texttt{y} or \texttt{x}.  Must be of type \texttt{bool}.
   \item[add]  a parameter to determine whether the input needs to be zeroed (\texttt{bool}).
   \item[nx]   size of \texttt{x} (\texttt{int}).
   \item[ny]   size of \texttt{y} (\texttt{int}).
   \item[x]    data or operator, depending on whether \texttt{adj} is true or false (\texttt{float*}).
   \item[y]    data or operator, depending on whether \texttt{adj} is true or false (\texttt{float*}).
\end{desclist}

