\section{Cosine Fourier transform (cosft.c)}




\subsection{{sf\_cosft}}
Makes preparations for the cosine Fourier transform, by allocating the required spaces.

\subsubsection*{Call}
\begin{verbatim}sf_cosft_init(n1);\end{verbatim}

\subsubsection*{Definition}
\begin{verbatim}
void sf_cosft_init(int n1)
/*< initialize >*/ 
{
   ...
}
\end{verbatim}

\subsubsection*{Input parameters}
\begin{desclist}{\tt }{\quad}[\tt ]
   \setlength\itemsep{0pt}
   \item[n1\_in] length of the input (\texttt{int}).  
\end{desclist}



\subsection{{sf\_cosft\_close}}
Frees the allocated space.

\subsubsection*{Call}
\begin{verbatim}sf_cosft_close();\end{verbatim}

\subsubsection*{Definition}
\begin{verbatim}
void sf_cosft_close(void) 
/*< free allocated storage >*/
{
   ...
}
\end{verbatim}




\subsection{{sf\_cosft\_frw}}
This function performs the forward cosine Fourier transform.

\subsubsection*{Call}
\begin{verbatim}sf_cosft_frw (q, o1, d1);\end{verbatim}

\subsubsection*{Definition}
\begin{verbatim}
void sf_cosft_frw (float *q /* data */, 
                   int o1   /* first sample */, 
                   int d1   /* step */) 
/*< forward transform >*/
{
   ...
}
\end{verbatim}

\subsubsection*{Input parameters}
\begin{desclist}{\tt }{\quad}[\tt d2]
   \setlength\itemsep{0pt}
   \item[q]  input data (\texttt{float}). 
   \item[o1] first sample of the input data (\texttt{int}). 
   \item[d1] step size (\texttt{int}).  
\end{desclist}



\subsection{{sf\_cosft\_inv}}
This function performs the forward cosine Fourier transform.

\subsubsection*{Call}
\begin{verbatim}sf_cosft_inv (q, o1, d1);\end{verbatim}

\subsubsection*{Definition}
\begin{verbatim}
void sf_cosft_inv (float *q /* data */, 
                   int o1   /* first sample */, 
                   int d1   /* step */) 
/*< inverse transform >*/
{
   ...
}
\end{verbatim}

\subsubsection*{Input parameters}
\begin{desclist}{\tt }{\quad}[\tt d2]
   \setlength\itemsep{0pt}
   \item[q]  input data (\texttt{float}). 
   \item[o1] first sample of the input data (\texttt{int}). 
   \item[d1] step size (\texttt{int}).  
\end{desclist}





