\section{Introduction}

This section contains a bunch of programs that implement operators. Therefore a short introduction on operators is in order.

\subsection{Definition of operators}
Mathematically speaking an operator is a function of a function, i.e.~a rule (or mapping) according to which a function  $f$ is transformed into another function $g$. We use the notation $g=R[f]$ or simply $g=Rf$, where $R$ denotes the operator. Examples of operators are the derivative, the integral, convolution (with a specific function), multiplication by a scalar and others. Note that in general the domains of $f$ and $g$ are not necessarily the same. For example, in the case of the derivative, the domain of $g=Rf$ is the subset of the domain of $f$, in which $f$ is smooth. In particular if $f=|x|$, $x\in[-1,1]$, then the domain of $g$ is $(-1,0)\cup(0,1)$. 

An important class of operators are the \textbf{linear\index{operator!linear} operators}. An operator $L$ is linear if for any two functions $f_1$, $f_2$ and any two scalars $a_1$, $a_2$, $L[a_1f_1+a_1f_2]=a_1Lf_1 + a_2Lf_2$. The derivative, integral, convolution and multiplication by scalar are all linear operators.

In the discrete world, operators act on vectors and linear operators are in fact matrices, with which the vectors are multiplied. (Multiplication by a matrix is a linear operation, since $\mathbf{M}(a_1\mathbf{x}_1+a_2\mathbf{x}_2) = a_1\mathbf{M}\mathbf{x}_1+a_2\mathbf{M}\mathbf{x}_2$). In fact many of the calculations performed routinely in science and engineering are essentially matrix multiplications in disguise. For example assume a vector $\mathbf{x}=[x_1\;x_2\;\ldots\;x_n]^T$  with length $n$ (superscript ${}^T$ denotes transpose). Padding this vector with $m$ zeros, produces another vector $\mathbf{y}$ with 
\begin{gather*}
    \mathbf{y} = \begin{bmatrix}x_1\\x_2\\\vdots \\x_n\\0\\\vdots\\0\end{bmatrix}
               = \begin{bmatrix}\mathbf{x}\\\mathbf{0}\end{bmatrix}, 
\intertext{where $\mathbf{0}$ is the zero vector of length $m$. One can readily verify that zero padding is a \index{zero-padding operator}\index{operator!zero padding} linear operation with operator matrix $\mathbf{L}=\begin{bmatrix}\mathbf{I}\\\mathbf{O}\end{bmatrix}$, where $\mathbf{I}$ is the $n\times n$ identity matrix and $\mathbf{O}$ is the $m\times n$ zero matrix, since}
    \mathbf{y} = \mathbf{L}\mathbf{x} = \begin{bmatrix}\mathbf{I}\\\mathbf{O}\end{bmatrix}\mathbf{x} 
               = \begin{bmatrix}\mathbf{x}\\\mathbf{0}\end{bmatrix}.
\end{gather*}
Note that as in the case of functions, the domains of $\mathbf{x}$ and $\mathbf{y}$ are different: $\mathbf{x}\in\mathds{R}^n$ (or more generally $\mathbf{x}\in\mathds{C}^n$), while $\mathbf{y}\in\mathds{R}^{n+m}$ (or $\mathds{C}^{n+m}$).

Similarly, one can define \index{convolution operator}\index{operator!convolution} convolution of $\mathbf{x}$ with $\mathbf{a}=[a_1\;a_2\;\dots\;a_m]^T$ as the multiplication of $\mathbf{x}$ with 
\begin{gather*}
    \mathbf{A} = \begin{bmatrix} a_1 &  0  &  0  & \cdots &  0  &  0\\
                                 a_2 & a_1 &  0  & \cdots &  0  &  0\\
                                 a_3 & a_2 & a_1 & \cdots &  0  &  0\\
                                 \vdots & \vdots & \vdots & \ddots & \vdots & \vdots\\
                                  0  &  0  &  0  & \cdots & a_{m-1} & a_{m-2} \\ 
                                  0  &  0  &  0  & \cdots & a_m & a_{m-1} \\ 
                                  0  &  0  &  0  & \cdots &  0  & a_m
                 \end{bmatrix}. 
\end{gather*}
and many other operations as matrix multiplications. Other operators are the identity\index{identity operator}\index{operator!identity} operator is the identity matrix $\mathbf{I}$ and is implemented by \hyperref[sec:copy]{\texttt{copy.c}} and \hyperref[sec:ccopy]{\texttt{ccopy.c}} and the null \index{null operator}\index{operator!null}operator (or zero matrix $\mathbf{O}$), which is implemented by \hyperref[sec:adjnull]{\texttt{adjnull.c}}. For the rest of this introduction, the boldface notation will imply specifically discrete operators, while the normal fonts will imply operators on either continuous or discrete mathematical entities.

\subsection{Products of operators}
The result of an operation on a function is another function, therefore we can naturally apply an operator on another operator. In other words, if $L_1$, $L_2$ are two operators, then we can define $L_1L_2$ as $L_1L_2[x] = L_1[L_2[x]]$, provided that $L_1[L_2[x]]$ makes sense mathematically. This is called the composition of the operators $L_1$ and $L_2$. Because in the discrete case the composition of operators is in fact the multiplication $L_1L_2$ of the two matrices $L_1$, $L_2$ the operator\index{operator!product}\index{product!of operators} composition is usually referred to as operator product and denoted by $L_1L_2$ is used. The composition of operators can be naturally extended to any finite product $L_1\cdots L_{n-1}L_n$. The product of up to 3 operators is implemented in \hyperref[sec:chain]{\texttt{chain.c}}.

\subsection{Adjoint operators}
A very important notion in data processing is the \textbf{adjoint operator}\footnote{The adjoint operator should not be confused with the (classical) adjoint or adjucate or adjunct matrix of a square matrix. The adjugate matrix of an invertible matrix is the inverse multiplied by its determinant. } $L^*$ of an operator $L$. In the discrete world, the adjoint\index{operator!adjoint}\index{adjoint} operator of $L$ is its (conjugate) transpose, i.e.~$L^*=L^H$. From this definition of the adjoint it is evident that {\it the adjoint of the adjoint is the original operator }(since $(L^*)^*=(L^H)^H=L$). Consider a vector $\mathbf{y}=[y_1\,y_2,\,\ldots,y_{n+m}]^T$ and the adjoint of the zero-padding operator, $\mathbf{L}^*=\mathbf{L}^H=\begin{bmatrix}\mathbf{I}\\\mathbf{O}\end{bmatrix}^T=\begin{bmatrix}\mathbf{I} & \mathbf{O}^T\end{bmatrix}$. Then
\begin{gather*}
    \mathbf{L}^{*}\mathbf{y} = \begin{bmatrix}\mathbf{I} & \mathbf{O}^T\end{bmatrix}\mathbf{y}
                     = \begin{bmatrix}\mathbf{I} & \mathbf{O}^T\end{bmatrix}
                       \begin{bmatrix}y_1\\y_2\\\vdots \\y_n\\y_{n+1}\\\vdots\\y_{n+m}\end{bmatrix}
                     = \begin{bmatrix}y_1\\y_2\\\vdots \\y_n\end{bmatrix}.
\end{gather*}
We conclude that the adjoint of data zero-padding is data \index{truncation operator}\index{operator!truncation}truncation. It is also easy (but tedious) to verify that the adjoint operation of the convolution between $\mathbf{a}$ and $\mathbf{x}$ is the \index{operator!crosscorrelation}\index{crosscorrelation operation} crosscorrelation of $\mathbf{a}$ with $\mathbf{y}^H$. One may also notice that for the specific zero-padding operator, $\mathbf{L}^{*}\mathbf{L}\mathbf{x} = \mathbf{x}$, i.e.~in this case the adjoint neutralizes the effect of the operator. It is tempting to say that the adjoint operation is the inverse operation, however this is not the case: it is not always the case that $L^*L = LL^*$. In fact such an equality is meaningless mathematically if $\mathbf{L}$ is not a square matrix (if $\mathbf{L}$ is a $n\times m$ matrix, then $\mathbf{L}\mathbf{L}^{*}$ is $n\times n$, while $\mathbf{L}^{*}\mathbf{L}$ is $m\times m$). $L^*$ is not even the left inverse of $L$: notice that in the case of the zero padding operator, $(\mathbf{L}^*)^*\mathbf{L}^{*}\mathbf{y}=\mathbf{L}\mathbf{L}^{*}\mathbf{y}=\tilde{\mathbf{y}} \neq \mathbf{y}$ (the last $m$ elements of $\tilde{\mathbf{y}}$ are zero). However it is often the case that the adjoint is an adequate. It is the case though quite often that the adjoint is adequate approximation to the inverse (sometimes within a scaling factor) and it is also quite probable that the adjoint will do a better job than the inverse in inverse problems. This is because the adjoint operator tolerates data imperfections, which the inverse does not.

From the definition of the adjoint operation as the left multiplication the complex conjugate matrix, it follows that {\it the adjoint of the product of two linear operators equals the product of the adjoints in reverse order}, i.e.~$(L_1L_2)^*=L_2^*L_1^*$. This is naturally extended to the product of any finite product of operators, i.e.~$(L_1L_2\cdots L_n)^*=L_n^*L_{n-1}^*\cdots L_1^*$. The adjoint of the product is also implemented in \hyperref[sec:chain]{\texttt{chain.c}}.


\subsection{The dot-product test}\label{sec:dot-product test}
The dot-product test is a valuable checkpoint, which can tell us whether the implementation of the adjoint operator is wrong (however it cannot guarantee that it is indeed correct). The concept is the following: Assuming that we have coded an operator $L$ and its adjoint $L^*$. Then for any two vectors or functions $a$ and $b$,
\begin{gather}\label{eq:dotproducttest}
	\langle a,Lb\rangle = \langle(L^*a)^*,b\rangle
\end{gather} 
where $\langle\,,\,{}\rangle$ denotes the dot product. Remember that the dot product of two functions $f,g\in \mathds L_2$ is $\int fg^*\;dt$ while the dot product of two vectors $\mathbf{x}$ and $\mathbf{y}$ is $\mathbf{x}^H\mathbf{y}$. Notice that for vectors eq.~\eqref{eq:dotproducttest} becomes $\mathbf{x}^H\mathbf{Ly} = (\mathbf{L}^H\mathbf{y})^H\mathbf{y}$ which is obviously true.
The lhs of eq.~\eqref{eq:dotproducttest} is computed using $L$, while the rhs is computed using the adjoint $L^*$. For the dot-product test, one just needs to load the vectors \texttt{x} and \texttt{y} with random numbers and perform the two computations. If the two results are not equal (within machine precision), then the computation of either $L$ or $L^*$ is erroneous. Note that truncation errors have identical effects on both operators, so the two results should be almost equal. The dot-product test (for real operators only) is implemented by \hyperref[sec:sf_dot_test]{\texttt{sf\_dot\_test}}.

\subsection{Implementation of operators}
It should be evident by now that the implementation of an operator $L$ should have at least four arguments: a variable \texttt{x} from which the operand (entity on which $L$ is applied) $x$ is read along with its length \texttt{nx}, and the variable \texttt{y} in which the result $y=Lx$ is stored and its length $n_y$. 

Also, since every operator comes along with its adjoint, the implementation of the linear operators described later in this chapter, gives also the possibility to compute the adjoint operator. This is done through the boolean \texttt{adj} input argument. {\it When {\rm\texttt{adj}} is {\rm\texttt{true}}, the adjoint operator $L^*$ computed}. As discussed before, the domains of $x$ and $Lx$ are in general different, therefore $L^*$ cannot be applied on $x$. However it can always be applied on $Lx$ or some $y$, which has the same domain as $Lx$. For this reason, when \texttt{adj} is \texttt{true}, the operand is $y$ and the result is $x$ and thus, \texttt{y} is used as input and the result is stored in \texttt{x}. As an example if \hyperref[sec:sf_copy_lop]{\texttt{sf\_copy\_lop}} (the identity operator) is called, then the result is that $y\leftarrow x$. However if additionally \texttt{adj} is \texttt{true}, then the result will be $x\leftarrow y$. If \hyperref[sec:adjnull]{\texttt{adjnull}} (the null operator) is called, then the result is that $y\leftarrow 0$. However if additionally \texttt{adj} is \texttt{true}, then the result will be $x\leftarrow 0$.

Finally, it is often the case that we need to compute $y\leftarrow Lx$ but $y\leftarrow y+Lx$. For this reason another boolean argument, namely \texttt{add} is defined. If \texttt{add} is true, then $y\leftarrow y+Lx$. Considering the same example with the identity operator, if \texttt{sf\_copy\_lop} is called with \texttt{add} being \texttt{true}, then $y\leftarrow y+x$. If additionally \texttt{adj} is \texttt{true}, then $x\leftarrow y+x$. Or if \texttt{adjnull} is called with \texttt{add} being \texttt{true}, if \texttt{adj} is \texttt{false}, $y\leftarrow y$ and if \texttt{adj} is \texttt{true}, then $x\leftarrow x$ (so in essence, if \texttt{add} is \texttt{true}, no matter what the value of \texttt{adj}, nothing happens).

As a conclusion, the linear operators described in this chapter have all the following form:
\begin{center}\texttt{oper(adj, add, nx, ny, x, y)},\end{center} 
where \texttt{adj} and \texttt{add} are boolean, \texttt{nx} and \texttt{ny} are integers and \texttt{x} and \texttt{y} are pointers of various but the same data type. Table \ref{tab:adjadd} summarizes the effect of the \texttt{adj} and \texttt{add} variables.

% Requires the booktabs if the memoir class is not being used
\begin{table}[htbp]
   \centering
   %\topcaption{Table captions are better up top} % requires the topcapt package
   \caption{Returned values for linear operations.}
   \begin{tabular}{ccll} % Column formatting, @{} suppresses leading/trailing space
      \hline\hline
      \texttt{adj} & \texttt{add} & description & returns\\
      \hline\hline
       0  & 0 & normal operation                & $ y\leftarrow Lx$      \\\hline
       0  & 1 & normal operation with addition  & $ y\leftarrow y+Lx$    \\\hline
       1  & 0 & adjoint operation               & $ x\leftarrow L^*y$    \\\hline
       1  & 1 & adjoint operation with addition & $ x\leftarrow x +L^*y$ \\
      \hline\hline
   \end{tabular}
%   \caption{Returned values for linear operations.}
   \label{tab:adjadd}
\end{table}

%For example assume again the case of the zero-padding operator, $\mathbf{L}=\begin{bmatrix}\mathbf{I}\\\mathbf{O}\end{bmatrix}$. Its adjoint is $\mathbf{L}^*=\begin{bmatrix}\mathbf{I} & \mathbf{O}^T\end{bmatrix}$. The effect of $\mathbf{L}^*$ on $\mathbf{y}=\mathbf{L}\mathbf{x}$ is  
%\begin{gather*}
%    \tilde{\mathbf{x}} = \mathbf{L}^{*}\mathbf{y} 
%                     = \begin{bmatrix}\mathbf{I} & \mathbf{O}^T\end{bmatrix}\mathbf{y}
%                     = \begin{bmatrix}\mathbf{I} & \mathbf{O}^T\end{bmatrix}
%                       \begin{bmatrix}x_1\\x_2\\\vdots \\x_n\\0\\\vdots\\0\end{bmatrix}
%                     = \begin{bmatrix}x_1\\x_2\\\vdots \\x_n\end{bmatrix} = \mathbf{x}.
%\end{gather*}
















