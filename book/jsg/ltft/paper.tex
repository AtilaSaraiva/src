\published{Geophysical Prospecting, DOI 10.1111/j.1365-2478.2012.01062.x (2013)}
\title{Seismic data analysis using local time-frequency decomposition}

\renewcommand{\thefootnote}{\fnsymbol{footnote}}

\ms{GP-2010-0932-Final}

\address{
\footnotemark[1] College of Geo-exploration Science and Technology,\\
Jilin University \\
No.6 Xi minzhu street, \\
Changchun, China, 130026 \\
\footnotemark[2] Bureau of Economic Geology,\\
John A. and Katherine G. Jackson School of Geosciences \\
The University of Texas at Austin \\
University Station, Box X \\
Austin, TX, USA, 78713-8924}

\author{Yang Liu\footnotemark[1]\footnotemark[2], Sergey Fomel\footnotemark[2]}

\footer{GP-2010-0932-Final}
\lefthead{Liu and Fomel}
\righthead{Local time-frequency decomposition}

\maketitle
\begin{abstract}
Many natural phenomena, including geologic
events and geophysical data, are fundamentally nonstationary -
exhibiting statistical variation that changes in space and
time. Time-frequency characterization is useful for analyzing such
data, seismic traces in particular. 

We present a novel time-frequency
decomposition, which aims at depicting the 
nonstationary character of seismic data. The proposed
decomposition uses a Fourier basis to match the target
signal using regularized least-squares inversion. The decomposition is
invertible, which makes it suitable for
analyzing nonstationary data. The proposed method can provide more
flexible time-frequency representation than the classical S
transform. Results of applying the method to both synthetic and field
data examples demonstrate that the local time-frequency
decomposition can characterize nonstationary
variation of seismic data and be used in practical
applications, such as seismic ground-roll noise attenuation and
multicomponent data registration.
\end{abstract}

\section{Introduction}
Geological events and geophysical data often exhibit fundamentally
nonstationary variations. Therefore, time-frequency characterization
of seismic traces is useful for geophysical data analysis. A widely
used method of time-frequency analysis is the short-time Fourier
transform (STFT) \cite[]{Allen77}. However, the window function limits
the time-frequency resolution of STFT
\cite[]{Cohen95}. An alternative is the wavelet transform, which expands the
signal in terms of wavelet functions that are
localized in both time and frequency \cite[]{Chakraborty95}. However,
because a wavelet family is built by restricting its frequency
parameter to be inversely proportional to the scale, expansion
coefficients in a wavelet frame may not provide precise enough
estimates of the frequency content of waveforms, especially at high
frequencies \cite[]{Wang07}. Therefore, \cite{Sinha05,Sinha09}
developed a time-frequency continuous-wavelet transform (TFCWT) to
describe time-frequency map more accurately than the conventional
continuous-wavelet transform (CWT). The S transform
\cite[]{Stockwell96} is another generalization of 
STFT, which extends
CWT and overcomes some of its disadvantages. \cite{Pinnegar03}
developed a general version of the S transform by employing windows of
arbitrary and varying shape. The clarity of the S transform is worse
than the Wigner-Ville distribution function \cite[]{Wigner32}, which
achieves a higher resolution but is seldom used in practice because of
its well-known drawbacks, such as interference and aliasing. For
this reason, \cite{Li08} provided a smoothed Wigner-Wille
distribution (SWVD) to reduce the interference caused by the
cross-term interference. The matching pursuit method is yet
another approach to representing the time-frequency signature
\cite[]{Liu07,Wang07,Wang10}. Matching pursuit involves several
parameters and is a relatively expensive method. There are some other
approaches to spectral decomposition. \cite{Castagna06} compare
several different spectral-decomposition methods.

\cite{Liu09,Liu11} recently proposed a new method of time-varying 
frequency characterization of nonstationary seismic signals that is
based on regularized least-squares inversion. In this paper, we
expand the method of \cite{Liu11} by designing an invertible
nonstationary time-frequency
decomposition ---
 \emph{local time-frequency (LTF) decomposition}
and its extensions --- \emph{local time-frequency-wavenumber (LTFK)}
and \emph{local space-frequency-wavenumber (LXFK)
decompositions}. The key idea is to minimize the
error between the input signal and all its Fourier components
simultaneously using regularized nonstationary regression
\cite[]{Fomel09} with control on time resolution. This approach is
generic, in the sense that it is possible to
combine other basis functions, eg., fractional splines, with
regularization \cite[]{Herrmann01}. Although there is an iterative
inversion inside the algorithm, one can use LTF
decomposition as
an invertible "black box" transform from time to 
time-frequency, similar in properties to the S transform. The proposed
decompositions can provide local time-frequency 
or space-wavenumber representations for common seismic data-processing
tasks. We test the new method and compare it with the S transform by
using a classical benchmark signal with two crossing chirps. The
proposed LTF
decomposition appears to provide higher
resolution in both time and frequency when
an appropriate parameters of the shaping 
regularization operator \cite[]{Fomel07b} are used to constrain
the time resolution. Examples of ground-roll attenuation and
multicomponent image registration demonstrate that the method can be
effective in practical applications.

\section{Theory}
\subsection{Local time-frequency (LTF) decomposition}

The Fourier series is by definition an expansion of a function in terms of 
a sum of sines and cosines. Letting a causal signal, $f(x)$, be in range of
$[0,L]$, the Fourier series of the signal is given by
\begin{equation}
  \label{eq:eq1}
  f(x) = \frac{a_0}{2} + \sum_{n=1}^{\infty} 
  \left[a_n\cos\left(\frac{2\pi nx}{L}\right)+ 
    b_n\sin \left(\frac{2\pi nx}{L}\right)\right]\;.
\end{equation}

The notion of a Fourier series can also be extended to complex
coefficients as follows:
\begin{equation}
  \label{eq:eq2}
  f(x) = \sum_{n=-\infty}^{\infty}A_n \Psi_n(x)\;,
\end{equation}
where $A_n$ are the Fourier coefficients and $\Psi_n (x) = e^{i(2\pi
nx/L)}$. 

Nonstationary regression allows the coefficients $A_n$ to change with
$x$. In the linear notation, $A_n(x)$ can be obtained by solving the
least-squares minimization problem
\begin{equation}
  \label{eq:eq3}
  \min_{A_n}\,\|f(x)-\sum_n A_n(x) \Psi_n(x) \|_2^2\;.
\end{equation}
The minimization problem 
is ill posed because there are a lot more unknown variables than
constraints. Our solution is to include additional constraints in the
form of regularization, which limits the allowed variability of the
estimated coefficients
\cite[]{Fomel09}. Tikhonov's regularization \cite[]{Tikhonov63} can
modify the objective function to
%{\setlength\arraycolsep{2pt}
\begin{eqnarray}
  \label{eq:eq4}
  \widetilde{A_n}(x) = \arg\min_{A_n}\|f(x)-\sum_n 
            A_n(x)\Psi_n(x)\|_2^2
             +\, \epsilon^2\, \sum_n \|\mathbf{D}[A_n(x)]\|_2^2\;,
\end{eqnarray}%}
where $\mathbf{D}$ is the \emph{regularization} operator and
$\epsilon$ is a scaling parameter. One can define $\mathbf{D}$,
for example, as a
gradient operator that penalizes roughness of $A_n(x)$.

We use shaping regularization \cite[]{Fomel07b} 
instead of Tikhonov's regularization to constrain the 
least-squares inversion. Shaping is a general method for imposing
constraints by explicit mapping the estimated model to
the desired model, eg., smooth
model. Instead of trying to find
and specify an appropriate regularization operator, the user of the
shaping-regularization algorithm specifies a shaping operator, which
is often easier to design.

The absolute value of time-varying coefficients $|A_n(x)|$ provides a
time-frequency representation, and equation~\ref{eq:eq2} provides the
inverse calculation. In the discrete form,
a range of frequencies can be decided by the Nyquist frequency
\cite[]{Cohen95} or by the user's assignment. In a somewhat different
approach, \cite{Liu09} minimized the error between the input signal
and each frequency component independently. Their algorithm and the
proposed algorithm are equivalent when the decomposition is stationary
(or using a very large shaping radius), because they both reduce to
the regular Fourier transform. In the case of nonstationarity, their
approach does not guarantee invertability, because it processes each
frequency independently.

\subsection{Local $t$-$f$-$k$ (LTFK) and local $x$-$f$-$k$ (LXFK) decompositions}

\cite{Askari08} proposed $t$-$f$-$k$ and $x$-$f$-$k$ transforms that 
are based on the S transform. We define analogous LTFK and LXFK
decompositions by using the LTF decomposition. The new decompositions
can be used to design general time-varying or space-varying FK
filters. The key steps of the algorithm are illustrated schematically
in Figure~\ref{fig:LTFK,LXFK}.

\inputdir{.}
   \multiplot{2}{LTFK,LXFK}{width=0.85\columnwidth}{Schematic
              illustration of LTFK decomposition (a) and LXFK
              decomposition (b) by using the LTF decomposition.}

\subsection{Example of Time-frequency Characterization} 
A simple 1-D example is shown in Figure~\ref{fig:cchirps,st1}. The
input signal includes two crossing chirp signals and displays
nonstationary characteristics (Figure~\ref{fig:cchirps,st1}a).  We
applied the LTF decomposition to obtain a time-frequency
distribution. Figure~\ref{fig:ltft1,ltft12}a shows that the proposed
method recovers the linear frequency trend with high resolution in
both time and frequency. In comparison, the S transform has high
resolution near low frequencies but loses resolution at high
frequencies
(Figure~\ref{fig:cchirps,st1}b). Figure~\ref{fig:ltft1,ltft12}b
displays the LTF decomposition using a different smoothing parameter
(14 points) to demonstrate adjustable time-frequency characteristics
of the LTF decomposition.

  \inputdir{timefreq}

   \multiplot{2}{cchirps,st1}{width=0.75\textwidth}{Synthetic signal
                 with two crossing chirps (a) and time-frequency
                 spectra from S transform (b).}
   \multiplot{2}{ltft1,ltft12}{width=0.75\textwidth}{Time-frequency
                 spectra from LTF decomposition with different sizes of
                 the smoothing radius.  Smoothing radius of 7 points
                 (a) and smoothing radius of 14 points (b).}

 \section{Application to Ground-roll attenuation}
Seismic data always consist of signal and noise components. The
time-frequency denoising algorithm is an effective method for handling
noise problems \cite[]{Elboth10}. Ground roll is the main type of
coherent noise in land seismic surveys and is characterized by low
frequencies and high amplitudes. Current processing techniques for
attenuating ground roll include frequency filtering,
FK filtering \cite[]{Yilmaz01}, radon transform \cite[]{Liu04},
wavelet transform
\cite[]{GEO62-06-18961903}, and the curvelet transform
\cite[]{Yarham08}. \cite{Askari08} applied the S transform to
ground-roll attenuation. Here, we propose a similar strategy, except
that we are applying the proposed local time-frequency
decomposition instead of the S transform.

We applied our methods to a land shot gather contaminated by nearly
radial ground roll (Figure~\ref{fig:dat,siltft,iltft}a). All
time-domain images are obtained after automatic gain control (AGC). We
applied the forward LTF decomposition to each
trace to generate a time-frequency cube
(Figure~\ref{fig:ltft,mask}a). Note that the ground roll is
distributed at localized time-space (left-down section of
Figure~\ref{fig:ltft,mask}a) and time-frequency (right-down section of
Figure~\ref{fig:ltft,mask}a) positions. The LTF
decomposition is flexible, due to its adjustable
time-frequency resolution. Therefore, we designed a simple muting
filter to remove the noise components localized in both frequency and
space (Figure~\ref{fig:ltft,mask}b). The inverse LTF
decomposition brings the separated signal back 
to the original domain (Figure~\ref{fig:dat,siltft,iltft}).
Figure~\ref{fig:dat,siltft,iltft}c shows the difference between raw
data (Figure~\ref{fig:dat,siltft,iltft}a) and denoised result using
LTF decomposition
(Figure~\ref{fig:dat,siltft,iltft}b). It is possible to design more
complicated but more powerful masks. Without a time-space mask, our
method of simply muting selected frequencies would reduce to band-pass
filtering.

The LTFK and LXFK decompositions generate data
in different domains (Figure~\ref{fig:tfk,masktfk}a and
\ref{fig:fxk,maskfxk}a), which show the trend of ground-roll noise in
the frequency-wavenumber sections. Simple frequency-wavenumber masks
(Figure~\ref{fig:tfk,masktfk}b and
\ref{fig:fxk,maskfxk}b) can eliminate ground-roll noise in the
decomposition domains. The recovered signals using the inverse LTFK
and LXFK decompositions produce similar results
(Figure~\ref{fig:sitfk,sifxk}a and
\ref{fig:sitfk,sifxk}b, respectively). Furthermore, different
decompositions can be cascaded to improve their
denoising abilities. For comparison, we used a simple high-pass
filter. Figure~\ref{fig:brshot,ifk}a shows that the high-pass filter
fails in removing noise, a larger filter window can
damage the seismic signal. Another choice is FK filtering
(Figure~\ref{fig:brshot,ifk}b), which cannot remove the low-frequency
part of ground-roll noise. The result is similar to that of the LXFK
decomposition (Figure~\ref{fig:sitfk,sifxk}b),
but the proposed method tends to remove more noise than the standard
FK filter (especially near location of time 2.7s and offset 1.2km
in Figure~\ref{fig:sitfk,sifxk}b and
\ref{fig:brshot,ifk}b) because of the
decomposition's locality and its more flexible design. Radial trace
(RT) transform is another approach to deal with ground-roll noise,
which is a simple geometric re-mapping method of a seismic trace
gather. Idealized ground roll is transformed to small temporal
frequency by the RT transform and can be eliminated by applying the RT
transform, followed by high-pass filtering and the inverse RT
transform
\cite[]{Claerbout83,Henley99}. Figure~\ref{fig:resu8,csiltft}a shows
that the RT transform performs better than the high-pass filter or the
FK filter. However, it still has trouble separating signal and noise
near the source. Figure~\ref{fig:resu8,csiltft}b shows the denoised
result after cascading the proposed LXFK and LTF decompositions, which
achieved the best result in this case (especially at locations around
the bottom left corner).

\inputdir{groll}
    \multiplot{3}{dat,siltft,iltft}{width=0.55\textwidth}{Field
                  land data (a), denoised result using LTF 
                  decomposition
                  (b), and difference between raw data
                  (Figure~\ref{fig:dat,siltft,iltft}a) and denoised
                  result using LTF decomposition
                  (Figure~\ref{fig:dat,siltft,iltft}b) (c).}
    \multiplot{2}{ltft,mask}{width=0.75\textwidth}{Local $T$-$X$-$F$
                   spectra (a) and filter mask in $T$-$X$-$F$ domain (b).}
    \multiplot{2}{tfk,masktfk}{width=0.75\textwidth}{Local $F$-$K$-$T$ 
                  spectra (a) and filter mask in $F$-$K$-$T$ domain (b).}
    \multiplot{2}{fxk,maskfxk}{width=0.75\textwidth}{Local $F$-$K$-$X$ 
                  spectra (a) and filter mask in $F$-$K$-$X$ domain (b).}
    \multiplot{2}{sitfk,sifxk}{width=0.55\textwidth}{Denoised results
                  using different local decompositions. 
                  LTFK decomposition (a)
                  and LXFK decomposition (b).}
    \multiplot{2}{brshot,ifk}{width=0.55\textwidth}{Denoised data
                  using different methods (shown for
                  comparison). High-pass filter (a) and FK filter
                  (b).}
    \multiplot{2}{resu8,csiltft}{width=0.55\textwidth}{Denoised result
                  by using RT transform with high-pass filter (a) and
                  cascading LXFK and LTF decompositions 
                  (b).}

 \section{Application to multicomponent data registration} 
Multicomponent seismic data provide additional
information about subsurface physical characteristics
\cite[]{GEO68-01-00400057}. Joint interpretation of multiple image
components depends on our ability to identify and register reflection
events from similar reflectors. \cite{Fomel03} and \cite{warp} proposed a
multistep approach for registering 
PP and PS images, and identified spectral differences between PP and
PS images as a major problem that prevents an easy automatic
registration. The new LTF decomposition can
provide a natural domain for nonstationary spectral balancing of
multicomponent images.

\inputdir{vecta}
   \multiplot{2}{vpp,vss}{width=0.75\textwidth}{PP (a) and SS (b)
                 images from a nine-component land survey.}
   \plot{vnails}{width=0.75\textwidth}{Three ``nails'' for PP and SS
                 time correlation identified by initial image
                 interpretation and fitted to a straight line.}
   \multiplot{4}{vect-ppft-0a,vect-psft-0a,vect-ppft-0c,vect-psft-0c}{width=0.47\textwidth}{Time-frequency spectra in LTF decomposition 
                 domain. PP
                before balancing (a), SS after initial warping (b),
		PP after balancing (c), and warped SS after balancing (d).}
   \plot{vect-psw-0c}{width=0.75\textwidth}{Three stages for PP and SS
                registration. Initial warping (top), nonstationary
                spectral balancing (middle), and final registration after
                warping scan (bottom).}

Figure~\ref{fig:vpp,vss}a and b show seismic images from compressional
(PP) and shear (SS) reflections obtained by processing a land
nine-component survey \cite[]{Fomel07a}. One can use ``image warping''
\cite[]{wolberg} to squeeze the SS image to PP reflection
time and make the two images display in the same coordinate system.
Using initial interpretation and well-log analysis, we identified
three individual correlation ``nails'' in the terminology of
\cite{DeAngelo03}. Fitting a straight line through the nails suggests a 
constant initial $V_P/V_S$ ratio (Figure~\ref{fig:vnails}). For
illustration of spectral balancing, we select the 300th trace in the
PP and SS images and then warp (squeeze) SS time to PP time by using
the initial $V_P/V_S$ ratio. The corresponding local time-frequency
spectra are shown in
Figures~\ref{fig:vect-ppft-0a,vect-psft-0a,vect-ppft-0c,vect-psft-0c}a
and b. The SS-trace frequency appears higher in the shallow part of
the image because of a relatively low S-wave velocity but lower in the
deeper part of the image because of the apparently stronger
attenuation of shear waves. Spectral balancing essentially smoothes
the high-frequency image to match the low-frequency image. The LTF
decompositions provide a nonstationary domain for 
time-varying spectral balancing. Our spectral balancing works as
follows. For each time slice in LTF domains, we use three steps:
\begin{enumerate}
\item Match the PP and SS spectra by least-squares fitting 
with Ricker spectra 
\begin{equation}
\label{eq:ricker}
      R_i(f)=A^2_i\frac{f^2}{f^2_i}\,e^{-f^2/f^2_i}\;,
\end{equation}
where $f$ is frequency axis and get the dominant frequencies $f_1$ and
      $f_2$ ($f_2 > f_1$) and the corresponding amplitudes $A_1$ and
      $A_2$.
\item Use the estimated Ricker parameters to design a matching 
      Gaussian filter 
\begin{equation}
\label{eq:gauss}
G(f)=\frac{A_1 f^2_2}{A_2 f^2_1}\,e^{f^2 (1/f^2_2 -
      1/f^2_1)}\;.
\end{equation}
\item Shrink the high-frequency spectra to match the low-frequency spectra 
      by applying the Gaussian filter.
\end{enumerate}
The LTF spectra of PP and warped SS trace after nonstationary spectral
balancing are shown in
Figure~\ref{fig:vect-ppft-0a,vect-psft-0a,vect-ppft-0c,vect-psft-0c}c
and d, respectively, which shows a reasonable similarity between the
PP and SS traces for both shallow and deep parts. The inverse LTF
decomposition reconstructs balanced PP and SS 
waveforms in the
time domain. Figure~\ref{fig:vect-psw-0c} displays PP trace, SS trace,
and the difference between the two traces in time domain, which are
compared for three stages of automatic data registration
\cite[]{warp}. Residual $\gamma$ scan is an algorithm for rapid scanning 
of the field of possible registrations. After applying residual
$\gamma$ scan to update $V_P/V_S$ ratio, the difference between
balanced PP and registered SS traces is substantially reduced compared
to the initial registration. The final registration result is
visualized in Figure~\ref{fig:before,after}, which shows interleaved
traces from PP and SS images before and after registration. The
alignment of main seismic events (especially those at locations
``A'' and ``B'') is an indication of successful registration.

\inputdir{vecta}
   \multiplot{2}{before,after}{width=0.75\textwidth}{Interleaved
                 traces from PP and SS images before (a) and
                 after (b) multicomponent registration.}

 \section{Conclusion} 

We have introduced a new time-frequency
decomposition that uses regularized nonstationary
regression with Fourier bases to represent the time-frequency
variation of nonstationary signals. The
decomposition is invertible and provides an
explicit control on the time and frequency resolution of the
time-frequency representation. Experiments with synthetic and field
data show that the proposed local time-frequency
decomposition can depict nonstationary variation
and provide a useful domain for practical applications, such as
ground-roll noise attenuation and multicomponent image registration.

\section{Acknowledgments}
We thank Guochang Liu and Mirko van der Baan for stimulating
discussions. We thank Partha Routh, one anonymous associate editor,
and one anonymous reviewer for helpful suggestions, which improved the
quality of the paper. This work is supported in part by National
Natural Science Foundation of China (Grant No. 41004041) and 973
Programme of China (Grant No. 2009CB219301). This publication was
authorized by the Director, Bureau of Economic Geology, The University
of Texas at Austin.

\bibliographystyle{seg}
\bibliography{SEG,SEP2,gp2010}

%\end{document}

