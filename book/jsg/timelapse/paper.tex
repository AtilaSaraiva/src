\title{Time-lapse image registration using the local similarity attribute}
\author{Sergey Fomel and Long Jin}

\ms{GEO-2008-0257.R1}
\published{Geophysics, 74, no. 2, A7-A11, (2009)}
\address{
John A. and Katherine G. Jackson School of Geosciences \\
The University of Texas at Austin \\
University Station, Box X \\
Austin, TX 78713-8924}

\lefthead{Fomel \& Jin}
\righthead{Time-lapse image registration}

\maketitle

\begin{abstract}
  We present a method for registration of time-lapse seismic images
  based on the local similarity attribute. We define registration as
  an automatic point-by-point alignment of time-lapse images.
  Stretching and squeezing a monitor image and computing its local
  similarity to the base image allows us to detect an optimal
  registration even in the presence of significant velocity changes in
  the overburden.  A by-product of this process is an estimate of the
  ratio of the interval seismic velocities in the reservoir
  interval. We illustrate the proposed method and demonstrate its
  effectiveness using both synthetic experiments and real data from
  the Duri time-lapse experiment in Indonesia.
\end{abstract}

\section{Introduction}
Time-lapse seismic monitoring is an important technology for enhancing
hydrocarbon recovery \cite[]{GEO66-01-00500053}. At the heart of the
method is comparison between repeated seismic images with an attempt
to identify changes indicative of fluid movements in the reservoir.

In general, time-lapse image differences contain two distinct effects:
shifts of image positions in time caused by changes in seismic
velocities and amplitude differences caused by changes in seismic
reflectivity. The data processing challenge is to isolate changes in
the reservoir itself from changes in the surrounding
areas. Cross-equalization is a popular technique for this task
\cite[]{GEO66-04-10151025,stucchi}.  A number of different
cross-equalization techniques have been successfully applied in recent
years to estimate and remove time shifts between 
time-lapse images \cite[]{bertrand,aarre}. An analogous task 
exists in medical imaging, where it is known as the \emph{image
registration} problem \cite[]{medical}.

In this paper, we propose to use the local similarity attribute
\cite[]{attr} for automatic quantitative estimation and extraction of
variable time shifts between time-lapse seismic images. A similar
technique has been applied previously to multicomponent image
registration \cite[]{warp}.  As a direct quantitative measure of image
similarity, local attributes are perfectly suited for measuring
nonstationary time-lapse correlations. The extracted time shifts
 also provide a direct estimate of the seismic velocity
changes in the reservoir. We demonstrate an application of the
proposed method with synthetic and real data examples.

\section{Theory}
The correlation coefficient between two data sequences $a_t$ and $b_t$
is defined as 
\begin{equation}
  {c} = {\frac{\displaystyle \sum_t a_t\,b_t}{\displaystyle \sqrt{ \sum_t a_t^2\,\sum_t b_t^2}}}
\label{eq:c}
\end{equation}
and ranges between 1 (perfect correlation) and -1 (perfect correlation
of signals with different polarity). The definition of the local
similarity attribute \cite[]{attr} starts with the observation that
the squared correlation coefficient can be represented as the product
of two quantities $c^2 = p\,q$, where 
\[ 
p=\frac{\displaystyle \sum_t a_t\,b_t}{\displaystyle \sum_t b_t^2}
\]
is the solution of the least-squares minimization problem
\begin{equation}
  \label{eq:p}
  \min_p \sum_t \left(a_t - p\,b_t\right)^2\;,
\end{equation}
and 
\[
q = \frac{\displaystyle \sum_t a_t\,b_t}{\displaystyle \sum_t a_t^2}
\]
 is the solution of the least-squares minimization
\begin{equation}
  \label{eq:q}
  \min_q \sum_t \left(b_t - q\,a_t\right)^2\;.
\end{equation}
Analogously, the local similarity $\gamma_t$ is a variable signal
defined as the product of two variable signals $p_t$ and $q_t$ that
are the solutions of the regularized least-squares problems
\begin{eqnarray}
  \label{eq:pt}
  \min_{p_t} 
  \left(\sum\nolimits_t \left(a_t - p_t\,b_t\right)^2 + R\left[p_t\right]\right)\;, \\
   \label{eq:qt}
   \min_{q_t}
   \left(\sum\nolimits_t \left(b_t - q_t\,a_t\right)^2 + R\left[q_t\right]\right)\;,
\end{eqnarray}
where $R$ is a regularization operator designed to enforce a desired
behavior such as smoothness. Shaping regularization \cite[]{shape}
provides a particularly convenient method of enforcing smoothness in
iterative optimization schemes. If shaping regularization is applied
iteratively with Gaussian smoothing as a shaping operator, its first
iteration is equivalent to the fast local cross-correlation method of
\cite{hale}. Further iterations introduce relative amplitude
normalization and compensate for amplitude effects on the local image
similarity.  Choosing the amount of regularization (smoothness of the
shaping operator) affects the results.  In practice, we start with
strong smoothing and decrease it when the results stop changing and
before they become unstable.

The application of local similarity to the time-lapse image
registration problem consists of squeezing and stretching the monitor
image with respect to the base image while computing the local
similarity attribute. Next, we pick the strongest similarity trend
from the attribute panel and apply the corresponding shift to the image.

In addition to its use for image registration, the estimated local
time shift is a useful attribute by itself. Time shift analysis has
been widely applied to infer reservoir compaction
\cite[]{hatchell,tura,janssen,rickett}. Since the time shift has a
cumulative effect, it is helpful to compute the derivative of time
shift, which can relate the time shift change to the corresponding
reservoir layer. \cite{rickett} define the derivative of time shift
as \emph{time strain} and find it to be an
intuitive attribute for studying reservoir
compaction.

What is the exact physical meaning of the warping function $w(t)$ that
matches the monitor image $I_1(t)$ with the
base image $I_0(t)$ by applying the
transformation $I_1[w(t)]$? One can define the 
base traveltime as an integral in depth, as follows:
\begin{equation}
\label{eq:t0}
t = 2\,\int\limits_0^{H_0} \frac{dz}{v_0(z)}\;,
\end{equation} 
where $v_0(z)$ is the base velocity, and $H_0$
is the base depth. A similar event in the
monitor image appears at time
\begin{equation}
\label{eq:t1}
w(t) = 2\,\int\limits_0^{H_1} \frac{dz}{v_1(z)} = \int\limits_0^{t+\Delta t} \frac{\hat{v}_0(\tau)}{\hat{v}_1(\tau)}\,d\tau\;,
\end{equation} 
where $H_1$ is the monitor depth, $\hat{v}_0(t)$ and $\hat{v}_1(t)$
are seismic velocities as functions of time rather than depth, and
$\Delta t$ is the part of the time shift caused by the reflector
  movement:
\begin{equation}
\label{eq:dt}
\Delta t = 2\,\int\limits_{H_0}^{H_1} \frac{dz}{v_1(z)}
\end{equation}
 
In a situation where the change of $\Delta t$ with $t$ can be neglected,
 a simple differentiation of the function
$w(t)$ detected by the local similarity analysis provides an estimate
of the local ratio of the velocities:
\begin{equation}
{\frac{d w}{d t}} \approx {\frac{\hat{v}_0(t)}{\hat{v}_1(t)}}\;.
\label{eq:rat}
\end{equation}
If the registration is correct, the estimated velocity ratio outside
of the reservoir should be close to one. One can connect
the local velocity ratio to other physical attributes that are
related to changes in saturation, pore pressure, or compaction.

We demonstrate the proposed procedure in the next section using several
 examples.

\section{Examples}

\subsection{1-D synthetic data}
\inputdir{timelapse}      

\multiplot{2}{modl,rat}{width=0.6\columnwidth}{(a) 1-D synthetic velocity model before (solid line) and
  after (dashed line) reservoir production. (b) True (solid line) and estimated (dashed line) 
  interval velocity ratio.}
\multiplot{2}{data,warp}{width=0.7\columnwidth}{1-D synthetic seismic
  images and the time-lapse difference initially (a) and after image
  registration (b).}
\plot{scan100}{width=\columnwidth}{(a) Local similarity scan for detecting the warping
  function in the 1-D synthetic model. Red colors indicate large
   similarity. The black curve shows an automatically detected trend.}

 Figure~\ref{fig:modl} shows a simplistic five-layer velocity model,
 where we introduce a velocity increase in one of the layers to
 simulate a time-lapse effect. After generating synthetic image
 traces, we can observe, in Figure~\ref{fig:data}, that the time-lapse
 difference contains changes not only at the reservoir itself but also
 at interfaces below the reservoir. Additionally, the image amplitude
 and the wavelet shape at the reservoir bottom are incorrect.  These
 artifact differences are caused by time shifts resulting from the
 velocity change. After detecting the warping function $w(t)$ from the
 local similarity scan, shown in Figure~\ref{fig:scan100}, and
 applying it to the time-lapse image, the difference correctly
 identifies changes in reflectivity only at the top and the bottom of
 the producing reservoir [Figure~\ref{fig:warp}]. To implement the
 local similarity scan, we use the \emph{relative stretch} measure
 $s(t) = w(t)/t$. When the two images are perfectly aligned,
 $s(t)=1$. Deviations of $s(t)$ from one indicate possible
 misalignment. Finally, we apply equation~\ref{eq:rat} to estimate
 interval velocity changes in the reservoir and observe a reasonably
 good match with the exact synthetic model [Figure~\ref{fig:rat}].

\subsection{2-D synthetic data} 
\inputdir{long1}  
 
\multiplot{4}{time2-over,mdif-time2-over,dif-time2-over,dif2-time2-over}{width=0.47\columnwidth}{(a) Synthetic
  model. (b) Time-lapse change containing differences in both the
  reservoir interval and the shallow overburden. (c) Initial
  time-lapse difference image. (d) Time-lapse difference image
  after registration.}

\plot{scan-time2-over}{width=0.7\columnwidth}{Local similarity
  scan for the 2-D synthetic model.}

Figure~\ref{fig:time2-over} shows a more complicated 2-D synthetic
example. In this experiment, we assume that the changes occur both in
the reservoir and in the shallow subsurface
[Figure~\ref{fig:mdif-time2-over}]. The synthetic data were generated
by convolution modeling. After computing local similarity between the
two synthetic time-lapse images [Figure~\ref{fig:scan-time2-over}], we
apply the extracted stretch factor to register the images.
Figures~\ref{fig:dif-time2-over} and~\ref{fig:dif2-time2-over} compare
time-lapse difference images before and after
registration. Similarity-based registration effectively removes
artifact differences both above and below the synthetic reservoir. As
mentioned before, the local similarity cube is an important attribute
by itself and includes information on uncertainty bounds for the local
stretch factor, which reflects the uncertainty of the reservoir
parameter estimation.

\subsection{3-D field data}
\inputdir{duri}

\multiplot{2}{difc,difc2}{width=0.7\columnwidth}{Application to the
  Duri field data. Base image and time-lapse differences: (a)
  before registration, reproduced from~\cite{Lumley.sepphd.91}, (b)
  after registration.}
%\multiplot{2}{difb,difb2}{width=0.7\columnwidth}{Application to the
%  Duri field data. Time-lapse differences: (a) before registration,
%  reproduced from~\cite{Lumley.sepphd.91}, (b) after registration.}
%\multiplot{2}{b2,t2}{width=0.7\columnwidth}{Time slice of the Duri
%  image after 5 months of steam injection at 0.4~s (the reservoir
%  level) before registration (a) and after registration (b). The steam
%  injection well is located in the middle of the image.}

Finally, Figure~\ref{fig:difc,difc2} 
%and~\ref{fig:difb,difb2} 
shows an application of the proposed method to time-lapse images from
steam flood monitoring in the Duri field, reproduced from
\cite{Lumley.sepphd.91,SEG-1995-0203}. Before registration, real
differences in the monitor surveys after 2 months and 19 months are
obscured by coherent artifacts, which are caused by velocity changes
both in the shallow overburden and in the reservoir interval
[Figure~\ref{fig:difc}].  Similarly to the results of the synthetic
experiments, local-similarity registration succeeds in removing
artifact differences both above and below the reservoir level
[Figure~\ref{fig:difc2}]. After separating the time-shift effect from
amplitude changes, one can image the steam front propagation more
accurately using time-lapse seismic data.  We expect our method to
work even better on higher-quality marine data.

%Comparing the time slices before and after registration
%(Figure~\ref{fig:b2,t2}) indicates that accurate registration can
%change the interpretation of fluid flow effects in time-lapse seismic.

\section{Conclusions}

We propose a method of time-lapse image registration based on an
application of the local similarity attribute. The local attribute
provides a smooth continuous measure of similarity between two
images. Perturbing the monitor image by stretching and squeezing it in
time while picking its best match to the base image enables an
effective registration algorithm. The by-product of this process is an
estimate of the time-lapse seismic velocity ratios in the reservoir
interval.

Using synthetic and real data examples, we have demonstrated the
ability of our method to achieve an accurate time-domain image
registration and to remove artifact time-lapse differences caused by
velocity changes. Unlike some of the alternative cross-equalization methods,
the proposed method is not influenced by amplitude differences and can
account for velocity changes in the shallow overburden.


\section{Acknowledgments}

We thank David Lumley for the permission to use results from his
Ph.D. thesis. We also thank Aaron Janssen, Brackin Smith, and Ali
Tura for inspiring discussions and two anonymous reviewers for helpful
suggestions.

This publication is authorized by the Director, Bureau of Economic
Geology, The University of Texas at Austin.

\bibliographystyle{seg}
\bibliography{SEG,SEP2,timelapse}
