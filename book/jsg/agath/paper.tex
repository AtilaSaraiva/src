\title{Theory of 3-D angle gathers in wave-equation seismic imaging}
\author{Sergey Fomel}

\lefthead{Fomel}
\righthead{3-D angle gathers}

\maketitle

\begin{abstract}
  I present two methods for constructing angle gathers in 3-D seismic imaging
  by downward extrapolation. Angles in angle gathers refer to the scattering
  angle at the reflector and provide a natural access to analyzing migration
  velocity and amplitudes. In the first method, angle gathers are extracted at
  each downward-continuation step by mapping transformations in constant-depth
  frequency slices. In the second method, one extracts angle gathers after
  applying the imaging condition by transforming local offset gathers in the
  depth domain.  The second approach generalizes previously published
  algorithms for angle-gather construction in 2-D and common-azimuth imaging. 
%\keywords{Geophysics \and Seismic imaging \and Velocity analysis \and Amplitude
% analysis}
\end{abstract}

\section{Introduction}

Wave extrapolation provides an accurate method for seismic imaging in
structurally complex areas \cite[]{biondi,etgen}. Wave
extrapolation methods have several known advantages in comparison with
direct methods such as Kirchhoff migration thanks to their ability to
handle multi-pathing, strong velocity heterogeneities, and
finite-bandwidth wave-propagation effects
\cite[]{GEO66-05-16221640}.  However, velocity and amplitude analysis in the
prestack domain are not immediately available for wave extrapolation
methods. To overcome this limitation, several authors
\cite[]{GEO55-09-12231234,SEG-1999-08240827,SEG-2000-08300833,GEO67-03-08830889,wu,SEG-2003-08890892,GEO68-03-10651074,cig,tsic}
suggested methods for constructing angle gathers from downward-continued
wavefields. Angles in angle gathers are generally understood as the reflection
(scattering) angles at reflecting interfaces \cite[]{xu,hoop}. Angle gathers
facilitate velocity analysis \cite[]{SEG-2001-08850888,stork} and can be used in
principle for extracting angle-dependent reflectivity information directly at
the target reflectors \cite[]{SEG-2001-02960299}. \cite{stolk} assert that
angle gathers generated with wavefield extrapolation are genuinely free of
artifacts documented for Kirchhoff-generated angle gathers \cite[]{symes,symes2}.

There are two possible approaches to angle-gather construction with wavefield
continuation. In the first approach, one generates gathers at each depth level
converting offset-space-frequency planes into angle-space planes
simultaneously with applying the imaging condition. The offset in this case
refers to the local offset between source and receiver parts of the downward
continued prestack data. Such a construction was suggested, for example, by
\cite{SEG-1999-08240827}. This approach is attractive because of its
localization in depth. However, the method of \cite{SEG-1999-08240827}
produces gathers in the offset ray parameter as opposed to angle. As a result,
the angle-domain information becomes structure-dependent: the output depends
not only on the scattering angle but also on the structural dip.

In the second approach, one converts migrated images in offset-depth domain to
angle-depth gathers after imaging of all depth levels is completed.
\cite{GEO68-03-10651074} suggested a simple Radon-transform procedure for
extracting angle gathers from migrated images. The transformation is
independent of velocity and structure. \cite{GEO67-03-08830889} adopted it
for constructing angle gathers in the shot-gather migration.  \cite{new}
demonstrate that the method of \cite{GEO68-03-10651074} is strictly valid
in the 3-D case only in the absence of cross-line structural dips. They
present an extension of this method for the common-azimuth approximation
\cite[]{GEO61-06-18221832}.

In this paper, I present a more complete analysis of the angle-gather
construction in 3-D imaging by wavefield continuation. First, I show how to
remove the structural dependence in the depth-slice approach. The improved
mapping retains the velocity dependence but removes 
the effect of the structure. Additionally, I extend the second, post-migration
approach to a complete 3-D wide-azimuth situation.  Under the common-azimuth
approximation, this formulation reduces to the result of \cite{newseg}
and, in the absence of cross-line structure, it is equivalent to the Radon
construction of \cite{GEO68-03-10651074}. 

\section{Traveltime derivatives and dispersion relationships for a
  3-D dipping reflector}
%\inputdir{XFig}

Theoretical analysis of angle gathers in downward continuation methods can be
reduced to analyzing the geometry of reflection in the simple case of a
dipping reflector in a locally homogeneous medium. Considering the reflection
geometry in the case of a plane reflector is sufficient for deriving
relationships for local reflection traveltime derivatives in the vicinity of a
reflection point \cite[]{goldin}. Let the local reflection plane be described in
$\{x,y,z\}$ coordinates by the general equation
\begin{equation}
  \label{eq:plane}
  x\,\cos{\alpha} + y\,\cos{\beta} + z\,\cos{\gamma} = d\;,
\end{equation}
where the normal angles $\alpha$, $\beta$, and $\gamma$ satisfy
\begin{equation}
  \label{eq:norm}
  \cos^2{\alpha} + \cos^2{\beta} + \cos^2{\gamma} = 1\;,
\end{equation}
The geometry of the reflection ray paths is depicted in
Figure~\ref{fig:plane3b}.  The reflection traveltime measured on a
horizontal surface above the reflector is given by the known expression
\cite[]{LSC00-00-02680268,GEO36-03-05100516}
\begin{equation}
  \label{eq:ttime}
  t(h_x,h_y) =
  \frac{2}{v}\,\sqrt{D^2+h_x^2+h_y^2-
    \left(h_x\,\cos{\alpha} + h_y\,\cos{\beta}\right)^2}\;,
\end{equation}
where $D$ is the length of the normal to the reflector from the
midpoint (distance $MM'$ in Figure~\ref{fig:plane2b})
\begin{equation}
  \label{eq:dmm}
  D = d - m_x\,\cos{\alpha} - m_y\,\cos{\beta}\;,
\end{equation}
$m_x$ and $m_y$ are the midpoint coordinates, $h_x$ and $h_y$ are the
half-offset coordinates, and $v$ is the local propagation velocity.

\inputdir{XFig}
\plot{plane3b}{width=0.7\textwidth}{Reflection geometry in 3-D (a scheme).
  $S$ and $R$ and the source and the receiver positions at the surface. $O$ is
  the reflection point. $S'$ is the normal projection of the source to the
  reflector. $S''$ is the ``mirror'' source. The cumulative length of the
  incident and reflected rays is equal to the distance from $S''$ to $R$.}

According to elementary geometrical considerations
(Figures~\ref{fig:plane3b} and~\ref{fig:plane2b}), the reflection angle $\theta$
is related to the previously introduced quantities by the equation
\begin{equation}
  \label{eq:theta}
  \cos{\theta} = \frac{D}{\sqrt{D^2+h_x^2+h_y^2-
      \left(h_x\,\cos{\alpha} + h_y\,\cos{\beta}\right)^2}}\;.
\end{equation}

\plot{plane2b}{width=0.7\textwidth}{Reflection geometry in the reflection
  plane (a scheme). $M$ is the midpoint. As follows from the similarity of
  triangles $S''SR$ and $S'SM$, the distance from $M$ to $S'$ is twice smaller
  than the distance from $S''$ to $R$.}

Explicitly differentiating equation~(\ref{eq:ttime}) with respect to the
midpoint and offset coordinates and utilizing equation~(\ref{eq:theta})
leads to the equations
\begin{eqnarray}
  \label{eq:tmx}
  t_{m_x} & \equiv & \frac{\partial t}{\partial m_x} = 
  -\frac{2}{v}\,\cos{\theta}\,\cos{\alpha}\;,
  \\ \label{eq:tmy}
  t_{m_y} & \equiv & \frac{\partial t}{\partial m_y} =
  -\frac{2}{v}\,\cos{\theta}\,\cos{\beta}\;,
  \\ \label{eq:thx}
  t_{h_x} & \equiv & \frac{\partial t}{\partial h_x} = 
  \frac{4}{v^2\,t}\,\left(h_x\,\sin^2{\alpha} - 
    h_y\,\cos{\alpha}\,\cos{\beta}\right)\;,
  \\ \label{eq:thy}
  t_{h_y} & \equiv & \frac{\partial t}{\partial h_y} = 
  \frac{4}{v^2\,t}\,\left(h_y\,\sin^2{\beta} - 
    h_x\,\cos{\alpha}\,\cos{\beta}\right)\;.
\end{eqnarray}

Additionally, the traveltime derivative with respect to the depth of the
observation surface is given by
\begin{equation}
  \label{eq:tz}
  t_z \equiv \frac{\partial t}{\partial z} = 
  -\frac{2}{v}\,\cos{\theta}\,\cos{\gamma}
\end{equation}
and is related to the previously defined derivatives by the double-square-root
equation
\begin{eqnarray}
  \nonumber
  - v\,t_z & = &  
  \sqrt{1-
    \frac{v^2}{4}\,\left(t_{m_x} - t_{h_x}\right)^2 - 
    \frac{v^2}{4}\,\left(t_{m_y} - t_{h_y}\right)^2} \\
  & + & 
  \sqrt{1-
    \frac{v^2}{4}\,\left(t_{m_x} + t_{h_x}\right)^2 - 
    \frac{v^2}{4}\,\left(t_{m_y} + t_{h_y}\right)^2}\;.
  \label{eq:dsr}
\end{eqnarray}
In the frequency-wavenumber domain, equation~(\ref{eq:dsr}) serves as the
basis for 3-D shot-geophone downward-continuation imaging. In the Fourier
domain, each $t_x$ derivative translates into $-k_x/\omega$ ratio, where $k_x$
is the wavenumber corresponding to $x$ and $\omega$ is the temporal frequency.

Equations~(\ref{eq:tmx}), (\ref{eq:tmy}), and~(\ref{eq:tz}) immediately
produce the first important 3-D relationship for angle gathers
\begin{equation}
  \label{eq:theta3}
  \cos{\theta} = 
  \frac{v}{2\,\omega}\,\sqrt{
  k_{m_x}^2 +
  k_{m_y}^2 +
  k_z^2}\;.
\end{equation}
Expressing the depth derivative with the help of the double-square-root
equation~(\ref{eq:dsr}) and applying a number of algebraic transformations,
one can turn equation~(\ref{eq:theta3}) into the dispersion relationship
\begin{equation}
\boxed{
  \begin{gathered}
    \left(k_{m_x}^2 + k_{m_y}^2\right)\,\frac{\sin^2{\theta}}{v^2} +
    \left(k_{h_x}^2 + k_{h_y}^2\right)\,\frac{\cos^2{\theta}}{v^2}
    =  \\
    \frac{1}{4\,\omega^2}\,
    \left(k_{m_x}\,k_{h_y} - k_{m_y}\,k_{h_x}\right)^2 
    + 4\,\omega^2\,\,\frac{\cos^2{\theta}}{v^2}\,\frac{\sin^2{\theta}}{v^2}\;.
  \end{gathered}
}
\label{eq:down3fk}
\end{equation}
For each reflection angle $\theta$ and each frequency $\omega$,
equation~(\ref{eq:down3fk}) specifies the locations on the
four-dimensional \new{($k_{m_x}$, $k_{m_y}$, $k_{h_x}$, $k_{h_y}$)}
wavenumber hyperplane that contribute to the common-angle gather. In
the 2-D case, equation~(\ref{eq:down3fk}) simplifies by setting
$k_{h_y}$ and $k_y$ to zero. Using the notation $k_{m_x}=k_m$ and
$k_{h_x}=k_h$, the 2-D equation takes the form
\begin{equation}
  \label{eq:down2fk}
  k_m^2\,\sin^2{\theta} +
  k_h^2\,\cos^2{\theta}
  = \frac{4\,\omega^2}{v^2}\,\cos^2{\theta}\,\sin^2{\theta}
\end{equation}
and can be explicitly solved for $k_h$ resulting in the convenient 
2-D dispersion relationship
\begin{equation}
  \label{eq:khdown2}
  k_h = \frac{2\,\omega\,\sin{\theta}}{v}\,
  \sqrt{1-\frac{4\,k_m^2\,v^2}{\omega^2\,\cos^2{\theta}}}\;.
\end{equation}
In the next section, I show that a similar simplification is also valid
under the common-azimuth approximation. Equations~(\ref{eq:down3fk})
and~(\ref{eq:khdown2}) describe an effective migration of the
downward-continued data to the appropriate positions on midpoint-offset planes
to remove the structural dependence from the local image gathers.

Another important relationship follows from eliminating the local velocity $v$
from equations~(\ref{eq:dsr}) and~(\ref{eq:theta3}). Expressing $v^2$ from
equation~(\ref{eq:theta3}) and substituting the result in
equation~(\ref{eq:dsr}), we arrive (after a number of algebraical
transformations) to the frequency-independent equation
\begin{equation}
  \label{eq:post3fk}
  \boxed{
  \tan^2{\theta} = \frac{
    k_z^2\,\left(k_{h_x}^2 + k_{h_y}^2\right) +
    \left(k_{h_x}\,k_{m_x} + k_{h_y}\,k_{m_y}\right)^2}
  {k_z^2\,\left(k_{m_x}^2 + k_{m_y}^2 + k_z^2\right)}\;.
  }
\end{equation}
Equation~(\ref{eq:post3fk}) can be expressed in terms of ratios $k_{m_x}/k_z$
and $k_{m_y}/k_z$, which correspond at the zero local offset to local
structural dips ($z_{m_x}$ and $z_{m_y}$ partial derivatives), and ratios
$k_{h_x}/k_z$ and $k_{h_y}/k_z$, which correspond to local offset slopes. \new{As shown by \cite{cig}, it can be also expressed as
\begin{equation}
  \label{eq:cig}
  \tan^2{\theta} = \frac{k_{h_x}^2 + k_{h_y}^2 + k_{h_z}^2}{k_{m_x}^2 + k_{m_y}^2 + k_z^2}\;,
\end{equation}
where $k_{h_z}$ refers to the vertical offset between source and receiver wavefields \cite[]{SEG-2002-12841287}.}

In the 2-D case, equation~(\ref{eq:post3fk}) simplifies to the form,
independent of the structural dip:
\begin{equation}
  \label{eq:post2fk}
  \tan{\theta} = \frac{k_h}{k_z}\;,
\end{equation}
which is the equation suggested by \cite{GEO68-03-10651074}.
Equation~(\ref{eq:post2fk}) appeared previously in the theory of
migration-inversion \cite[]{GEO50-12-24582472}.
%I discuss a specialization
%to the case of the common-azimuth approximation in Appendix B.

\section{Common-azimuth approximation}

Common-azimuth migration \cite[]{GEO61-06-18221832} is a downward continuation
imaging method tailored for narrow-azimuth strea\-mer surveys that can be
transformed to a single common azimuth with the help of azimuth moveout
\cite[]{GEO63-02-05740588} Employing the common-azimuth approximation, one
assumes the reflection plane stays confined in the acquisition azimuth.
Although this assumption is strictly valid only in the case of constant
velocity \cite[]{Vaillant.sep.103.louis1}, the modest azimuth variation in
realistic situations justifies the use of the method \cite[]{naraz}. 
 
To restrict equations of the previous section to the common-azimuth
approximation, it is sufficient to set the cross-line offset $h_y$ to zero
assuming the $x$ coordinate is oriented along the acquisition azimuth. In
particular, from equations~(\ref{eq:thx}-\ref{eq:thy}), we obtain
\begin{equation}
  \label{eq:hxca}
  h_x\,\sin{\alpha} = \frac{v\,t}{2}\,\sin{\theta}
\end{equation}
\begin{eqnarray}
  \label{eq:thxca}
  t_{h_x} &  = &
  \frac{4\,h_x}{v^2\,t}\,\sin^2{\alpha} 
  = \frac{2}{v}\,\sin{\theta}\,\sin{\alpha}\;,
  \\ \label{eq:thyca}
  t_{h_y} &  = & 
  -\frac{4\,h_x}{v^2\,t}\,\cos{\alpha}\,\cos{\beta}
  = - \frac{2}{v}\,\sin{\theta}\,\cot{\alpha}\,\cos{\beta}\;.
\end{eqnarray}
With the help of equations~(\ref{eq:tmx}), \old{(\ref{eq:tmx})}
\new{(\ref{eq:tmy})}, and (\ref{eq:tz}), equation~(\ref{eq:thyca})
transforms to the form
\begin{eqnarray}
  \nonumber
  t_{h_y} & = & t_{m_y}\,\frac{\tan{\theta}}{\tan{\alpha}} \\
  & = &
  t_{m_y}\,\frac{
    \sqrt{1-\frac{v^2}{4}\,\left(t_{m_x}+t_{h_x}\right)^2} -
    \sqrt{1-\frac{v^2}{4}\,\left(t_{m_x}-t_{h_x}\right)^2}}
  {\sqrt{1-\frac{v^2}{4}\,\left(t_{m_x}+t_{h_x}\right)^2} +
    \sqrt{1-\frac{v^2}{4}\,\left(t_{m_x}-t_{h_x}\right)^2}}\;,
  \label{eq:biondo}
\end{eqnarray}
suggested by \cite{GEO61-06-18221832}. Combining equations~(\ref{eq:tmx}),
\old{(\ref{eq:tmx})} \new{(\ref{eq:tmy})}, (\ref{eq:tz}), and~(\ref{eq:thxca}) and transforming to the
frequency-wavenumber domain, we obtain the common-azimuth dispersion
relationship
\begin{equation}
  \label{eq:comaz}
  \left(k_{h_x}^2 + k_{m_y}^2 + k_z^2\right)\,
  \left(k_{m_x}^2 + k_{m_y}^2 + k_z^2\right) =
  \frac{4\,\omega^2}{v^2}\,\left(k_{m_y}^2 + k_z^2\right)\;,
\end{equation}
which shows that, under the common-azimuth approximation and in a laterally
homogeneous medium, 3-D seismic migration amounts to a cascade of a 
2-D prestack migrations in the in-line direction and a 2-D zero-offset
migration in the cross-line direction \cite[]{GEO61-02-04090421}.  

Under the common-azimuth approximation, the angle-dependent
relationship~(\ref{eq:down3fk}) takes the form
\begin{equation}
  \label{eq:down2ca}
  k_{m_x}^2\,\sin^2{\theta} +
  k_{h_x}^2\,\cos^2{\theta}
  = \frac{4\,\omega^2}{v^2}\,\cos^2{\theta}\,\sin^2{\theta}\;,
\end{equation}
which is identical to the 2-D equation~(\ref{eq:down2fk}). This proves that
under this approximation, one can perform the structural correction
independently for each cross-line wavenumber.

The post-imaging equation~(\ref{eq:post3fk}) transforms to the equation
\begin{equation}
  \label{eq:postcafk}
  \tan^2{\theta} = \frac{k_{h_x}^2}{k_{m_y}^2 + k_z^2}\;,
\end{equation}
obtained previously by \cite{newseg}. In the absence of cross-line
structural dips ($k_{m_y}=0$), it is equivalent to the 2-D
equation~(\ref{eq:post2fk}).

\section{Algorithm I: Angle gathers during downward continuation}
%\inputdir{zslice}

This algorithm follows from equation~(\ref{eq:down3fk}). It consists of the
following steps, applied at each propagation depth $z$:
\begin{enumerate} 
\item Generate local offset gathers and transform them to the wavenumber
  domain. In the double-square-root migration, the local offset wavenumbers
  are immediately available. In the shot gather migration, local offsets are
  generated by cross-correlation of the source and receiver wavefields
  \cite[]{GEO67-03-08830889}.
\item For each frequency $\omega$, transform the local offset wavenumbers
  $\{k_{h_x},k_{h_y}\}$ into the angle coordinates $\sin{\theta}/v$ according
  to equation~(\ref{eq:down3fk}). The angle coordinates depend on velocity but
  do not depend on the local structural dip. In the 2-D case, each frequency
  slice is simply the $\{k_m,k_h\}$ plane, and each angle coordinate
  corresponds to a circle in that plane centered at the origin and described
  by equation~(\ref{eq:down2fk}). Figure~\ref{fig:angle1} shows an example of
  a 2-D frequency slice transformed to angles.
\item Accumulate contributions from all frequencies to 
  apply the imaging condition in time.
\end{enumerate}

\inputdir{zslice}
\plot{angle1}{width=0.9\textwidth}{Constant-depth constant-frequency
  slice mapped to reflection angles according to the 2-D version of 
  Algorithm I. Zero offset wavenumber maps to zero (normal incidence)
  angle. The top right corner is the evanescent region.}

This algorithm is applicable for targets localized in depth. The local
offset gathers need to be computed for all lateral locations, but
there is no need to store them in memory, because conversion to angles
happens on the fly. The algorithm outputs not angles directly, but
velocity-dependent parameters $\sin{\theta}/v$. \cite{tariq} extend
this algorithm to transversally-isotropic media.

\section{Algorithm II: Post-migration angle gathers}
%\inputdir{radon}

The second algorithm follows from equation~(\ref{eq:post3fk}). It applies
after the imaging has completed and consists of the following steps applied at
each common-image location:
\begin{enumerate}
\item Generate and store local offset gathers.  In the double-square-root
  migration, the local offsets are immediately available.  In the shot gather
  migration, local offsets are generated by cross-correlation of the source
  and receiver wavefields.
\item Estimate the dominant local structural dips at the common image point by
  using one of the available dip estimation methods: local slant stack,
  plane-wave destruction, etc.
\item After the imaging has completed, transform local-offset gathers into the
  slant-stack domain either by slant-stacking in the $\{z,h_x,h_y\}$ physical
  domain or by radial-trace construction in the $\{k_z,k_{h_x},k_{h_y}\}$
  Fourier domain \cite[]{GEO68-03-10651074}.
\item Using estimated dips, convert slant stacks into angles by applying
  equation~(\ref{eq:post3fk}). The mapping from offset-depth slopes to angles
  is illustrated in Figure~\ref{fig:angl45-45}.
\end{enumerate}
The last two steps can be combined into one. It is sufficient to compute the
effective offset $\hat{h} = \sqrt{h_x^2+h_y^2+(h_x z_y - h_y z_x)^2}$ and
apply the basic 2-D angle extraction algorithm to the effective offset gather.

\inputdir{radon}
\plot{angl45-45}{width=0.9\textwidth}{Mapping from the offset slope plane
  to angles according to Algorithm II. Zero slopes map to zero
  (normal-incidence) angle.}

The second method is applicable to selected common-image gathers, which can be
spread on a sparse grid. The local offset gathers need to be computed and
stored at all depths. The method works independent of the velocity. The main
disadvantage is the need to estimate local structural dips. In the
common-azimuth approximation, only the cross-line dip is required
\cite[]{newseg}. In the 2-D case (zero cross-line dip), the method is
dip-independent \cite[]{GEO68-03-10651074}.

\section{Discussion}

\new{Since the first presentation of the 3-D angle-gather theory \cite[]{SEG-2004-10531056}, many new research results have appeared in the literature. By the end of 2000s, prestack 3-D reverse-time migration has become a standard tool for depth imaging in structurally-complex areas, and it is becoming feasible to generate 3-D angle gathers as part of routine processing \cite[]{exxon,western,veritas}. The most important new theoretical developments are the ability to extract angle information from time-shift angle gathers \cite[]{tsic,western}, the ability to extract not only reflection-angle but also azimuth information \cite[]{veritas}, and the extension of the angle-gather theory to anisotropy \cite[]{biondianis,tariq}.}

\section{Conclusions}

Angle gathers present a natural tool for analyzing velocities and
amplitudes in wave-equation imaging. I have discussed two approaches
for angle-gather construction. In the first approach, angle gather are
constructed on the fly at different depth steps of the wave
extrapolation process. In the second approach, angle gathers are
extracted from the local-offset gathers after imaging has
completed. The second method was previously presented for the 2-D case
and for the case of a common-azimuth approximation. Both approaches
have advantages and disadvantages. The preference depends on the
application and the input data configuration.

%Application examples will be included in the presentation.

\section{Acknowledgments}

I am grateful to Nanxun Dai, John Etgen, \fbox{Sergey Goldin}, and Paul Sava
for enlightening discussions. 

This publication is authorized by the Director, Bureau of Economic
Geology, The University of Texas at Austin.

\bibliographystyle{seg}
\bibliography{SEG,SEP2,agath}

