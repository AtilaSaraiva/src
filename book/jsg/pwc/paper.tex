\title{Regularizing seismic inverse
problems by model re-parameterization using plane-wave construction}

\renewcommand{\thefootnote}{\fnsymbol{footnote}}

\lefthead{Fomel \& Guitton}
\righthead{Plane-wave construction}

\author{Sergey Fomel\footnotemark[1] and
  Antoine Guitton\footnotemark[2]}

\address{
\footnotemark[1]Bureau of Economic Geology, \\
John A. and Katherine G. Jackson School of Geosciences \\
The University of Texas at Austin \\
University Station, Box X \\
Austin, TX 78713-8972 \\
\footnotemark[2]3DGeo Development Inc. \\
4633 Old Ironsides Drive, Suite 401 \\
Santa Clara, CA 95054}

\maketitle

\begin{abstract}
  We define plane-wave construction (PWC), an operator for generating
  data aligned along predefined locally variable slopes, as the
  inverse of plane-wave destruction, an operator used for measuring
  the slopes.  PWC can be applied for efficient \old{preconditioning}
  \new{regularization} of \old{geophysical} \new{seismic} estimation
  problems. Using simple examples, we demonstrate the applicability of
  PWC for enhancing the coherency of seismic images, improving
  velocity estimation methods, and separating primaries and multiples
  with a pattern-based approach.
\end{abstract}

\section{Introduction}

\cite{GEO67-06-19461960} describes applications of plane-wave
destruction filters, a concept originally introduced by \cite{pvi}.
High-order accurate destruction filters are used for estimating local
slopes of seismic events and can be applied for such problems as fault
detection, data regularization, and noise attenuation.  Estimating
local slopes can also replace traditional velocity analysis and enable
velocity-independent time-domain seismic imaging \cite[]{pmig}.

In this paper, we introduce \emph{plane-wave construction} (PWC), a
formal inverse of the plane-wave destruction operator. We show that
PWC can be used as an efficient \old{preconditioner} \new{regularizer}
for speeding up iterative optimization that involves models with local
plane-wave structure.  In that sense, one can view PWC as a
higher-order generalization of \emph{steering filters}
\cite[]{SEG-1998-1851,clapp}. We illustrate its applications with
field data examples.

The immediate effect of PWC is smoothing the data along dominant event
slopes. In application to coherency enhancement in seismic images,
iterative least-squares inversion with PWC \old{preconditioning}
\new{parameterization} extracts portions of the image aligned with the
dominant slopes. Using PWC in combination with iterative reweighting
allows us to preserve fault geometry in the coherency enhancing
process. When applied to the classic velocity estimation problem, PWC
enforces consistency between the velocity structure and reflector
geometry. In the multiple elimination problem, PWC
\old{preconditioning} \new{parameterization} enables separating
primary and multiple events on the basis of differences in their local
slopes.

\new{PWC requires seismic traces to be sequentially
ordered. Therefore, it may not be immediately applicable to 3-D
problems.  Transformation from plane-wave construction to plane-wave
shaping \cite[]{shape} addresses this problem \cite[]{us}.}

\begin{comment}
A more powerful approach to regularization involves shaping operators
that project the estimated model into the space of acceptable models
\cite[]{shape}. In the last section of this paper, we show how to
transform plane-wave construction into plane-wave shaping by following
an analogy with triangle filtering.
\end{comment}

\section{Plane-wave construction defined}

Let us represent a seismic section $\mathbf{s}$ as a collection of
traces: $\mathbf{s} = \left[\mathbf{s}_1 \; \mathbf{s}_2 \; \ldots \;
\mathbf{s}_N\right]^T$. A plane-wave destruction operator
\cite[]{GEO67-06-19461960} effectively predicts each trace from its
neighbor and subtracts the prediction from the original trace. In the
linear operator notation, we can write the plane-wave destruction
operation as
\begin{equation}
  \label{eq:pwd}
  \mathbf{r} = \mathbf{D\,s}\;,
\end{equation}
where $\mathbf{r}$ is the destruction residual, and $\mathbf{D}$ is the
destruction operator defined as
\begin{equation}
  \label{eq:d}
  \mathbf{D} = \mathbf{I}_N - \mathbf{P} =
  \left[\begin{array}{ccccc}
      \mathbf{I} & 0 & 0 & \cdots & 0 \\
      - \mathbf{P}_{1,2} & \mathbf{I} & 0 & \cdots & 0 \\
      0 & - \mathbf{P}_{2,3} & \mathbf{I} & \cdots & 0 \\
      \cdots & \cdots & \cdots & \cdots & \cdots \\
      0 & 0 & \cdots & - \mathbf{P}_{N-1,N} & \mathbf{I} \\
    \end{array}\right]\;,
\end{equation}
where $\mathbf{I}$ stands for the identity operator, $\mathbf{P}$ is
the prediction operator defined by \cite{GEO67-06-19461960}, and
$\mathbf{P}_{i,j}$ describes prediction of trace $j$ from trace
$i$. \new{Prediction of a trace consists of shifting the original
trace along the dominant event slope. The dominant slope is estimated
by minimizing the prediction error (the output of $\mathbf{D}$) by
regularized least-squares optimization. Regularization constraints the
estimated slopes to vary smoothly inside the data space.}

The plane-wave construction (PWC) operator $\mathbf{C}$ is simply the inverse of
$\mathbf{D}$:
\begin{eqnarray}
  \nonumber 
  \mathbf{C} & = & 
  \left[\begin{array}{ccccc}
      \mathbf{I} & 0 & 0 & \cdots & 0 \\
      - \mathbf{P}_{1,2} & \mathbf{I} & 0 & \cdots & 0 \\
      0 & - \mathbf{P}_{2,3} & \mathbf{I} & \cdots & 0 \\
      \cdots & \cdots & \cdots & \cdots & \cdots \\
      0 & 0 & \cdots & - \mathbf{P}_{N-1,N} & \mathbf{I} \\
    \end{array}\right]^{-1} \\
  \nonumber
  &  = & 
  \left[\begin{array}{ccccc}
      \mathbf{I} & 0 & 0 & \cdots & 0 \\
      \mathbf{P}_{1,2} & \mathbf{I} & 0 & \cdots & 0 \\
      \mathbf{P}_{1,2}\,\mathbf{P}_{2,3} & \mathbf{P}_{2,3} & \mathbf{I} & \cdots & 0 \\
      \cdots & \cdots & \cdots & \cdots & \cdots \\
      \mathbf{P}_{1,2} \cdots \mathbf{P}_{N-1,N} & 
      \mathbf{P}_{2,3} \cdots \mathbf{P}_{N-1,N} & \cdots & 
      \mathbf{P}_{N-1,N} & \mathbf{I} \\
    \end{array}\right] \\
  &  = & 
  \left[\begin{array}{ccccc}
      \mathbf{I} & 0 & 0 & \cdots & 0 \\
      \mathbf{P}_{1,2} & \mathbf{I} & 0 & \cdots & 0 \\
      \mathbf{P}_{1,3} & \mathbf{P}_{2,3} & \mathbf{I} & \cdots & 0 \\
      \cdots & \cdots & \cdots & \cdots & \cdots \\
      \mathbf{P}_{1,N} & 
      \mathbf{P}_{2,N} & \cdots & 
      \mathbf{P}_{N-1,N} & \mathbf{I} \\
    \end{array}\right]
  \;,
  \label{eq:c}
\end{eqnarray}

For efficiency, it is convenient to apply PWC as a recursive triangular inversion. The output
of
\begin{equation}
  \label{eq:pwc}
  \mathbf{c} = \mathbf{C\,s} = \left[\mathbf{c}_1 \; \mathbf{c}_2 \; \ldots \;
    \mathbf{s}_N\right]^T
\end{equation}
is computed recursively as follows:
\begin{equation}
  \label{eq:rec}
  \mathbf{c}_1 = \mathbf{s}_1\;,\quad
  \mathbf{c}_k = \mathbf{s}_k + \mathbf{P}_{k-1,k}\,\mathbf{c}_{k-1}\;\quad
  k=2,3,\ldots,N
\end{equation}
$\mathbf{C}$ is a smoothing operator along local
plane-waves. \new{Re-parameterization by $\mathbf{C}$ provides
effective regularization and} can help \old{accelerating}
\new{accelerate the} convergence of iterative optimization in inverse
problems \cite[]{harlan,GEO68-02-05770588}.  When applied for model
\old{preconditioning} \new{re-parameterization} in least-squares
inversion of the forward modeling operator
\begin{equation}
  \label{eq:for}
  \mathbf{d} = \mathbf{L\,m} = \mathbf{L\,C\,p}\;,
\end{equation}
plane-wave construction leads to the formal inversion
\begin{equation}
  \label{eq:inv}
  \widehat{\mathbf{m}} = \mathbf{C}\,\widehat{\mathbf{p}} =
\mathbf{C\,C}^T\,\left(\mathbf{L\,C\,C}^T\,\mathbf{L}^T + \epsilon^2\,\mathbf{I}_N\right)^{-1}\,\mathbf{d}\;.
\end{equation}
Here $\mathbf{m}$ is the model, $\mathbf{p}$ is the \old{preconditioned} \new{re-parameterized}
model, $\mathbf{d}$ is the observed data, $\epsilon$ is the
regularization parameter, and $\widehat{\mathbf{m}}$ is the
regularized model estimate. The $\mathbf{C}\,\mathbf{C}^T$ operator
plays the role of the model covariance operator. % \cite[]{tarantola}.
In large-scale
problems, the inversion in equation~(\ref{eq:inv}) is computed efficiently by
an iterative conjugate-gradient algorithm.  A different approach to recursive
filter preconditioning, based on the helix transformation
\cite[]{GEO63-05-15321541}, was suggested by \cite{GEO68-02-05770588}.

The total cost of plane-wave construction is simply proportional to the data
size and to the cost of an elementary prediction $\mathbf{P}_{k-1,k}$, which
operates at the speed of tridiagonal inversion for a matrix of the
trace-length size \cite[]{GEO67-06-19461960}. This operation is comfortably
efficient in practical applications.

\section{Applications}

\inputdir{beic}

In this section, we describe three practical applications of
plane-wave construction in solving inverse problems with
  model \old{preconditioning} \new{re-parameterization}.

\subsection{Coherency enhancement}

\plot{bei-slp}{width=0.9\columnwidth}{Left: seismic image after prestack time
  migration. Right: local dips estimated with plane-wave destruction.}
\plot{bei-vg}{width=0.9\columnwidth}{Left: seismic image after coherency
  enhancing. Right: difference between the coherency-enhanced image and the
  original image.}

The simplest kind of regularized inversion involves $\mathbf{L}$ as
the identity operator $\mathbf{I}_N$. The corresponding application of
PWC \old{preconditioning} \new{parameterization} results in an
effective deconvolution that enhances structural consistency and
continuity of seismic images.

A coherency enhancement operator is
\begin{equation}
  \label{eq:vg}
  \mathbf{h = H\,s} = \mathbf{C\,W\,C}^T\,\left(\mathbf{C\,W\,C}^T + \epsilon^2\,\mathbf{I}_N\right)^{-1}\,\mathbf{s}\;,
\end{equation}
where a diagonal weighting operator $\mathbf{W}$ is added to prevent
plane-wave smoothing across structural discontinuities. We define
$\mathbf{W}$ as a diagonalized magnitude of $\mathbf{p}$ from an
unweighted inversion in equation~(\ref{eq:inv}). When repeated
iteratively, this method corresponds to iteratively reweighted
least-squares with a simulated $L_1$ norm for vector $\mathbf{p}$
\cite[]{GEO68-01-03860399}.

We apply coherency enhancement to a time-migrated seismic image from a
historic Gulf of Mexico dataset \cite[]{bei}, shown in
Figure~\ref{fig:bei-slp}.  Estimating local event slopes (right plot
in Figure~\ref{fig:bei-slp}) defines the PWC operator. The output of
coherency enhancement using equation~(\ref{eq:vg}) and the
corresponding noise component removed from the data are shown in
Figure~\ref{fig:bei-vg}. Coherency enhancement highlights locally
continuous reflectors while preserving the geometry of faults. A
similar effect was described by \cite{TLE21-03-02380243}, who called
it ``Van Gogh'' filtering.

\subsection{Velocity estimation}

\plot{bei-pdxw}{width=0.88\textwidth}{Seismic image from
Figure~\ref{fig:bei-vg} overlaid on top of the interval velocity model
estimated by PWC-parameterized Dix inversion.}
\plot{bei-pdx}{width=0.9\columnwidth}{Left: migration velocity used for
  prestack time migration. Right: migration velocity predicted by regularized
  Dix inversion.}


%Plane-wave construction can be used as a model preconditioning
%operator \cite[]{GEO59-05-08180829,GEO68-02-05770588} for regularizing
%inverse problems in the cases, where estimated models have certain
%structure. In these applictions, plane-wave construction acts as an
%accurate steering filter in terminology of \cite{clapp}.

To demonstrate an application of PWC to the seismic velocity
estimation problem, we chose a simple Dix inversion formulation
\cite[]{GEO20-01-00680086} applied for interval velocity estimation
using the Gulf of Mexico dataset from Figure~\ref{fig:bei-slp}. When
Dix inversion is formulated as a regularized estimation problem, the
forward operator $\mathbf{L}$ \new{in equations~(\ref{eq:for})
and~(\ref{eq:inv})} turns into simple integration
\cite[]{Clapp.sep.97.bob1,alejandro}. \old{Preconditioning}
\new{Parameterization} by plane-wave construction using the dip field
estimated from the image forces the estimated velocity to follow the
geological structure.

Figure~\ref{fig:bei-pdxw} shows the resultant interval velocity model.
Figure~\ref{fig:bei-pdx} shows a comparison between the input and
predicted RMS (root-mean-square) migration velocity.  We can see that
both goals of model estimation are achieved: the estimated model
explains the observed data while following a geological structure
consistent with the seismic image.

\subsection{Multiple suppression}

Separation of primary and multiple reflections is one of the most
important tasks in seismic data processing. A distinguishable
characteristic of surface-related multiple events is different slopes
because of different apparent velocities. The advantage of using
slope-based prediction is that, for estimating dominant slopes of
multiple events, we can utilize models of the multiples that may have
incorrect amplitudes and wavelets as long as they correctly predict
event geometry \cite[]{antoine}. An example is shown in
Figure~\ref{fig:cmp}, which contains a multiple-infested CMP gather
from the Mobil AVO dataset \cite[]{CSI00-00-00010213}, its prediction
with the SRME (surface-related multiple elimination) method
\cite[]{GEO57-09-11661177} and two dominant slopes estimated from the
data and corresponding to primary and multiple reflections. Even
though SRME does not provide correct amplitudes and wavelets in
prediction of the multiple events, it can guide the slope estimation
method toward extracting the dominant slopes of multiple events.

\inputdir{cmp}

\plot{cmp}{width=\columnwidth}{a: CMP gather from the
Mobil AVO dataset, b: multiple model from SRME prediction, c: estimated
dominant slope of the primary reflection events, d: estimated dominant
slope of the multiple reflection events.}

The model of the data is now \cite[]{GEO65-02-05740583,FBR20-03-01610167}
\begin{equation}
\label{eq:splusn}
  \mathbf{d} = \mathbf{C}_p\,\mathbf{p}+\mathbf{C}_n\,\mathbf{n} = 
  \left[\begin{array}{cc} \mathbf{C}_p & \mathbf{C}_n \end{array}\right]\,
  \left[\begin{array}{c} \mathbf{p} \\ \mathbf{n} \end{array}\right]\;,
\end{equation}
where $\mathbf{C}_p$ is plane-wave construction along primary slopes
and \old{$\mathbf{C}_p$} \new{$\mathbf{C}_n$} is plane-wave
construction along multiple slopes. In accordance with
equation~(\ref{eq:inv}), the least-squares estimates of the multiples
is
\begin{equation}
\label{eq:mult}
  \widehat{\mathbf{m}} = \mathbf{C}_n\,\widehat{\mathbf{n}} =
  \mathbf{C}_n\,\mathbf{C}_n^T\,
  \left(\mathbf{C}_p\,\mathbf{C}_p^T + \mathbf{C}_n\,\mathbf{C}_n^T + \epsilon^2\,\mathbf{I}_N\right)^{-1}\,\mathbf{d}\;.
\end{equation}

Figure~\ref{fig:super} shows the estimated primary and multiple events
and comparison of velocity semblance scans before and after multiple
suppression. A large portion of the multiple energy is successfully
removed from the data.

\plot{super}{width=\columnwidth}{a: estimated primary
reflections (data with multiples removed), b: estimated multiple
reflections, c: velocity scan of the original gather, d: velocity scan of
the gather after multiple suppression.}

\begin{comment}

\section{From plane-wave construction to shaping}

Shaping \cite[]{shape} is a powerful approach to
regularization. Shaping acts as a smoothing operator that projects the
estimated model into the space of acceptable models.

To transform plane-wave construction to shaping, we can use the
prediction operator~$\mathbf{P}$ from equation~(\ref{eq:d}) and define
a box smoother of length $k$ as
\begin{equation}
  \label{eq:bk}
  \mathbf{B}_k = \frac{1}{k}\,\left(\mathbf{I}_N + \mathbf{P} + 
  \mathbf{P}^2 + \cdots +  \mathbf{P}^k\right)\;.
\end{equation}
Implementing equation~(\ref{eq:bk}) directly requires many computational
operations. Noting that
\begin{equation}
  \label{eq:rec}
  \left(\mathbf{I}_N - \mathbf{P}\right)\,\mathbf{B}_k =
    \frac{1}{k}\,\left(\mathbf{I}_N - \mathbf{P}^{k+1}\right)\;,
\end{equation}
we can rewrite equation~(\ref{eq:bk}) in the compact form
\begin{equation}
  \label{eq:bcomp}
  \mathbf{B}_k = 
  \frac{1}{k}\,\left(\mathbf{I}_N - \mathbf{P}\right)^{-1}\,
  \left(\mathbf{I}_N - \mathbf{P}^{k+1}\right) = \frac{1}{k}\,\mathbf{C}\,\left(\mathbf{I}_N - \mathbf{P}^{k+1}\right)\;,
\end{equation}
which can be implemented economically.  Finally, combining two
generalized box smoothers creates a symmetric generalized triangle smoothing operator suitable for shaping regularization
\begin{equation}
  \label{eq:tk}
  \mathbf{T}_k = \mathbf{B}_k^T\,\mathbf{B}_k\;.
\end{equation}
A triangle shaper uses local predictions from both the left and the
right neighbors of a record and averages with triangle weights.

The least-squares inversion regularized by shaping takes the form
\cite[]{shape}
\begin{equation}
  \label{eq:shape}
  \widehat{\mathbf{m}} = 
  \mathbf{B}_k\,\left[\epsilon^2\,\mathbf{I}_N + 
    \mathbf{B}_k^T\,\left(\mathbf{L}^T\,\mathbf{L} - 
      \epsilon^2\,\mathbf{I}_N\right)\,
    \mathbf{B}_k\right]^{-1}\,
  \mathbf{B}_k^T\,\mathbf{L}^T\,\,\mathbf{d}\;.
\end{equation}

Figure~\ref{fig:bei-smo} shows several impulse responses for shaping with
$k=20$ for the seismic image used in this study.

%\plot{bei-smo}{width=\columnwidth}{Impulse responses of PWC shaping
%  for different points inside the image space.}

\end{comment}

\section{Conclusions}

We have introduced plane-wave construction (PWC), an operator that
generates models aligned with predefined locally variable dips. PWC is
defined as the inverse of plane-wave destruction, an operator used for
measuring local dips.  It is applicable as a model
\old{preconditioner} \new{regularizer} in \old{geophysical}
\new{seismic} estimation problems such as coherency enhancement,
\old{seismic} velocity estimation, and multiple suppression. As an
efficient operator for characterizing data elongated over dominant
smoothly variable slopes, PWC can be incorporated for regularization
of any inversion problems that operate with locally planar data. We
anticipate more applications of the proposed method.

\old{An important limitation of the PWC method is that it is only applicable to
2-D problems, where seismic traces can be simply ordered. A possible
extension to 3-D problems is in transformation from plane-wave
construction to plane-wave shaping and in
application of shaping regularization}.

\section{Acknowledgments}

The first author thanks Norsk Hydro for partially supporting this
research. 

\bibliographystyle{seg}
\bibliography{SEG,SEP2,flat}

%%% Local Variables: 
%%% mode: latex
%%% TeX-master: t
%%% End: 
