%\documentclass[paper]{geophysics}
%\documentclass[paper,revised]{geophysics}
%\usepackage{times,graphicx,timet}

% Defining macros

%\begin{document}

%\usepackage{xcolor}


% ----------------------------------
% ARTICLE
% ----------------------------------


\title{Fractal heterogeneities in sonic logs\\ and low-frequency scattering attenuation}

\renewcommand{\thefootnote}{\fnsymbol{footnote}} 

%\ms{GEO-2007-0352} % paper number

\author{Thomas J. Browaeys and Sergey Fomel}

\footer{}
\lefthead{Browaeys \& Fomel}
\righthead{Fractals heterogeneities and attenuation}

\maketitle


%\pagebreak


\begin{abstract}
Cycles in sedimentary strata exist at different scales
and can be described by fractal statistics.
We use von K\'arm\'an's autocorrelation function
to model heterogeneities in sonic logs from a clastic reservoir and
propose a nonlinear parameter estimation.
Our method is validated using synthetic signals,
and when applied to real sonic logs, it extracts both
the fractal properties of high spatial frequencies and one dominant cycle between 2.5 and 7 m.
Results demonstrate non-Gaussian and antipersistent statistics of sedimentary layers.
We derive an analytical formula for the scattering attenuation of scalar waves 
by 3D isotropic fractal heterogeneities using the mean field theory.
Penetration of waves exhibits a high-frequency cutoff sensitive to heterogeneity size.
Therefore shear waves can be more attenuated than compressional waves because of their shorter wavelength.
\end{abstract}


\section{Introduction}


Propagation of waves in heterogeneous media involves attenuation and
dispersion by scattering.  Theoreticians are still challenged by the
phenomenon of wave propagation in random media. The mean field theory
\cite[]{Chernov_60,Karal_K64,Uscinski_77} is commonly used and
provides both dispersion and attenuation, depending on scattering
cross-sections of the heterogeneities
\cite[]{Waterman_T61,Wu_A85,Kanaun_L08}, which are described by their
statistical spatial autocorrelation.  Higher order correlations have
been more recently incorporated in a frequency-dependent effective
medium theory \cite[]{Chesnokov_KK98}.  One major advance, pointed out
by \cite{Wu_82,Wu_A85}, is the restriction of the validity of the mean
field formalism to low frequency.  The theory in fact includes
destructive interferences, caused by averaging different realizations
of the random medium, and overestimates attenuation at high
frequencies.  Alternative solutions have been proposed to remove this
artificial decoherence of the phase.  Two important examples are
radiative transfer theory \cite[]{Wu_93,Haney_WS05} and the Rytov
approximation \cite[]{Rytov_KT89}, which is more adequate than the
Born approximation when phase fluctuations are important.

In the area of seismic imaging, the layered structure of sediments led
\cite{Odoherty_A71} to introduce the fundamental concept of
stratigraphic filtering.  The empirical formula postulated by
O'Doherty and Anstey was demonstrated in 1D using mean field formalism
\cite[]{Banik_LS85,Resnick_90} and, alternately, using wave
localization theory \cite[]{Sheng_WZP86,Shapiro_Z93,Shapiro_H99}, with
a recent extension to a larger frequency band for acoustic waves in 3D
\cite[]{Muller_S01}.  The wave localization method utilizes phase, and
logarithm of the amplitude, which have the property of self-averaging
over some distance called the localization length, in a stationary
random medium.  These quantities are exactly the ones used in the
Rytov method and avoid the phenomenon of artificial phase decoherence
at high frequencies.  Multiple scattering of seismic waves remains a
complex and active research area.

The fractal property of subsurface heterogeneities was initially discussed by \cite{Hewett_86} 
for hydrocarbon reservoirs and has been confirmed in both vertical and horizontal wells \cite[]{Stefani_G01}.
The study of seismic scattering by 2D numerical wave propagation \cite[]{Frankel_C86} 
demonstrates the necessity of self-similar heterogeneities for modeling
both observations of coda waves and of traveltime anomalies.
A $1/f$ spectrum of heterogeneities was modeled 
using the von K\'arm\'an spatial autocorrelation function \cite[]{Vonkarman_48}
in order to obtain a constant quality factor at high frequencies.
For scatterers larger than the wavelength, multipathing was observed, whereas
3D effects were revealed to be important for scattering loss at low frequency.
\cite{Gist_94} tried to explain seismic-wave attenuation in VSP surveys 
by 3D scattering from fractal heterogeneities.
One of the first attempts to relate the statistics of well log data 
to seismic scattering used wave localization theory \cite[]{White_SN90}.
A more detailed study of acoustic-wave localization effects in 1D fractal media \cite[]{Vanderbaan_01}
shows that a constant quality factor is possible only for the $1/f$ fractal spectrum
and that localization can not occcur if the medium contains periodic layers involving resonance
and violating the ergodicity assumption.
Convergence of the localization effect in realistic 3D seismic surveys seems questionable.
Presence of strong cycles in well log data is causing difficulties 
when the fractal exponent is being estimated \cite[]{Dolan_BR98}
and is commonly attributed to Milankovitch cycles \cite[]{Anstey_D02a}.

Further investigation of the relationship between cycles, fractal
properties, and correlation lengths is necessary and low-frequency
scattering theories in 3D fractal media can be appropriate for
conventional seismic surveys.  In this paper, we propose a nonlinear
estimation method for fractal statistics of sonic-log heterogeneities
using von K\'arm\'an's model.  We attempt to identify different scales
of the sedimentation process as proposed by \cite{Odoherty_A71} and
\cite{Anstey_D02a}.  The inversion captures small-scale
heterogeneities while larger local cycles exist.  We use the mean
field theory to calculate analytical solution of low-frequency
attenuation by scattering from 3D fractal heterogeneities and predict
a shift of the dominant frequency with depth in seismic surveys.

The paper is organized into two parts.
In the first part we present a description of cycles in sediments
in connection with fractal statistics.
The von K\'arm\'an spatial autocorrelation function is introduced,
and we briefly review some features of fractal statistics.
We present our estimation method, validate it on synthetic signals,
apply it to our sonic-log data, and show that one can detect the high-frequency part
of the superposition of different geological scales.
The second part explains the derivation of scattering attenuation
for low-frequency acoustic waves by 3D isotropic fractal heterogeneities using the mean field theory.
Results comply with the Rayleigh regime
and the Backus effective medium for very low frequency.
We present analytical predictions of scattering attenuation and show the existence of
a cutoff frequency for the penetration of waves.
We use the frequency dependence of the penetration depth
to calculate the shift of the dominant frequency of a Ricker wavelet.
We conclude by suggesting further improvements.


\section{Statistical model of heterogeneities}


Let us consider the spatial fluctuations of seismic velocities to be small and to 
constitute a second-order stochastic process.
We describe the fluctuations by using different realizations of the random function $f(\mathbf{x})$
with the expectation value $\langle f \rangle = 0$ and with the spatial covariance 
depending on the relative distance $\mathbf{r}$ defined by
\begin{eqnarray}
\langle f(\mathbf{x})f(\mathbf{x}+\mathbf{r})\rangle & = & \sigma^2 N(\mathbf{r}), \label{eqn:covarf}
\end{eqnarray}
where $\sigma$ is the standard deviation and $N(\mathbf{r})$ is the spatial autocorrelation 
function with $N(\mathbf{0})=1$.
The energy spectrum $E^{(s)}(\mathbf{k})$ of the fluctuations in $s$ dimensions ($s=1,2,3$) is
related to the autocorrelation by the Wiener-Khintchine theorem \cite[]{Born_W64}:
\begin{eqnarray}
E^{(s)}(\mathbf{k}) = |F(\mathbf{k})|^2 & = & 
\sigma^2\int N(\mathbf{r})e^{-i\mathbf{k}\cdot\mathbf{r}}d\mathbf{r}, \label{eqn:wienerkh}\\
F(\mathbf{k}) & = & \int f(\mathbf{x})e^{-i\mathbf{k}\cdot\mathbf{x}}d\mathbf{x},
\end{eqnarray}
where $\mathbf{k}$ is the spatial wave vector and $F(\mathbf{k})$ is the Fourier transform of $f(\mathbf{x})$.
The energy spectrum in equation \ref{eqn:wienerkh} can be simplified, for an isotropic correlation function, to
\begin{eqnarray}
E^{(1)}(k) & = & 2~\sigma^2\int_{0}^{\infty}N(r)\cos(kr)dr, \label{eqn:etf1d}\\
E^{(3)}(k) & = & \frac{4\pi}{k}~\sigma^2\int_{0}^{\infty} rN(r)\sin(kr)dr, \label{eqn:etf3d}
\end{eqnarray}
where $k=|\mathbf{k}|$.  The von K\'arm\'an autocorrelation function
$N_{H,b}(\mathbf{r})$ describes a self-affine medium relevant for
geological structures
\cite[]{Goff_J88,Holliger_L92,Dolan_BR98,Sato_F98,Klimes_02,Goff_H03}.
This function was initially derived by \cite{Vonkarman_48} while
studying the velocity field in a turbulent fluid and has been used to
describe heterogeneous media \cite[]{Tatarski_61,Frankel_C86}.  The
Fourier transform of $N_{H,b}(\mathbf{r})$ was given by
\cite{Lord_54}.  The statistical autocorrelation $N_{H,b}(\mathbf{r})$
and the energy spectrum $E^{(s)}_{H,b}(\mathbf{k})$ in the Fourier
domain are
\begin{eqnarray}
N_{H,b}(\mathbf{r}) & = & \frac{2^{(1-H)}}{\Gamma(H)}~(r/b)^{H}K_{H}(r/b),\label{eqn:vkacf}\\
E^{(s)}_{H,b}(\mathbf{k}) & = & \sigma^2~C^{(s)}_{H}
\frac{\left(2b\right)^s}{\left(1+b^2k^2\right)^{H+\frac{s}{2}}},\label{eqn:vkes} \\
\mbox{with} & & C^{(s)}_{H}=\left|\frac{\Gamma(H+\frac{s}{2})}{\Gamma(H)}\right|\pi^{\frac{s}{2}}, \label{eqn:calpha}
\end{eqnarray}
where $r=|\mathbf{r}|$, $K_{H}$ is the modified Bessel function 
of the second kind with order $H$, and $\Gamma$ is the Gamma function. 
Parameters describing the heterogeneities are characteristic distance $b$,
below which the distribution is fractal,
and exponent $H$, characterizing the roughness of the medium.
We use the energy spectrum in equation \ref{eqn:vkes} with $s=1$ to analyze sonic logs
and with $s=3$ to predict 3D scattering attenuation.


\subsection{Fractal statistics}


Among different concepts introduced by the theory of fractals \cite[]{Mandelbrot_83},
self-affine property accounts for invariance of roughness of a curve observed at different scales.
Self-affine fractals can be characterized by the power-law dependence of their energy spectrum $E(f)$
on frequency $f$:
\begin{eqnarray}
E(f) & \propto & f^{-\beta}. \label{eqn:fracspec}
\end{eqnarray}
The exponent $\beta$, in the energy spectrum $E^{(s)}_{H,b}(\mathbf{k})$ from equation \ref{eqn:vkes}, is
\begin{eqnarray}
\beta & = & 2H+s. \label{eqn:fracexp}
\end{eqnarray}
For $\beta=0$, energy spectrum is constant and describes the familiar white noise.
Causal integration of Gaussian white noise produces the classical Brownian motion, or random walk,
characteristic of diffusion processes, and results in an energy spectrum with $\beta=2$. 
The autocorrelation function of Brownian motion signals is a decreasing exponential and
the autocorrelation in equation \ref{eqn:vkacf} properly reduces to $\exp{[-r/b]}$ for $H=0.5$.
Another interesting form of spectrum is for $\beta=1$.  
The associated signal is called Flicker noise \cite[]{Schottky_26,Dolan_BR98} and can be interpreted 
as the superposition of different relaxation processes.
For geological layers, such form of spectrum was interpreted as the expression
of quasi-cyclicity and blocky layering \cite[]{Shtatland_91}.
Generalization, including Gaussian white noise and Brownian motion, leads to two types 
of fractal signals \cite[]{Shtatland_91,Turcotte_97,Li_03}, namely
\begin{itemize} 
\item fractional Gaussian noise (fGn) defined as filtered Gaussian white noise with $-1 \leq \beta \leq 1$,
\item fractional Brownian motion (fBm) built by causal integration of fGn above and resulting in $1 \leq \beta \leq 3$.
\end{itemize}
The fGn is stationary and Gaussian, whereas the fBm is neither stationary nor Gaussian.
Exponent $\beta_{fBm}$ of the fBm is related to exponent $\beta_{fGn}$ of the fGn, used for integration, by 
$\beta_{fBm}=\beta_{fGn}+2$.

The significance of parameter $H$ in equation \ref{eqn:fracexp} is delicate and connected to the Hurst exponent $Hu$,
which measures the correlation of time series \cite[]{Hurst_51} by
\begin{eqnarray}
Hu & = & \frac{\log\left(R/\sqrt{S}\right)}{\log(T)}\,, \label{eqn:Hurst}
\end{eqnarray}
where $R$ and $S$ are respectively range of variations and variance
calculated for the length $T$ of the signal. The meaning of the value of $Hu$ is 
\begin{itemize} 
\item antipersistence for $0\leq Hu\leq 0.5$,
\item random process for $Hu=0.5$, and
\item persistence for $0.5\leq Hu\leq 1$.
\end{itemize}
Estimation of $Hu$ using formula \ref{eqn:Hurst} is relevant only for fGn signals \cite[]{Turcotte_97}.
For example, Gaussian white noise produces $Hu=0.5$.
Parameter $H$ defined in equation \ref{eqn:fracexp} is associated with the Hurst exponent by
\begin{itemize} 
\item $H=Hu-1$ for fGn with $-1\leq\beta\leq 1$;
\item $H$ equals the Hurst exponent of incremented fGn \cite[]{Li_03} for fBm with $1\leq\beta\leq 3$.
\end{itemize}
The different self-affine 1D fractal models are presented in Table~\ref{tbl:fractclass}
according to the nature and persistency of the signal.
A previous analysis of the logarithm of acoustic impedance from well data
by \cite{Walden_H85} shows $1/2 \leq \beta \leq 3/2$, promoting the so-called $1/f$ geology.
An improved solution, reproducing the complete well log sequence in sedimentary rocks, uses
a similar random process based on fractional L\'evy motion \cite[]{Painter_P94}, 
but fBm can be adequate at small scales, inside different facies \cite[]{Lu_MFC02}.
 
% Table fractal classification 
\tabl{fractclass}
{Classification of 1D fractal statistics according to the exponent $\beta$ and geological interpretation.}
{
\begin{center}
    \begin{tabular}{|c|r|r|c|}
      \hline
      {\it Fractal exponent} & Von K\'arm\'an exponent $H$ & {\it Description} & {Geology} \\[0.2mm]
      \hline
      $\beta=0$     & $-0.5$       & Gaussian white noise & Random process \\[0.3mm]
      \hline
      $0<\beta<1$   & $-0.5<H<0$   & Persistent fGn & \\[0.3mm]
      \hline
      $\beta=1$     & $0$          & Flicker noise & Blocky layers\\[0.3mm]
      \hline
      $1<\beta<2$   & $0<H<0.5$    & Antipersistent fBm & Quasi-cyclic deposition \\[0.3mm]
      \hline
      $\beta=2$     & $0.5$        & Brownian walk & Random deposition \\[0.3mm]
      \hline
      $2<\beta<3$   & $0.5<H<1$    & Persistent fBm & Transitional deposition \\
      \hline
    \end{tabular}
\end{center}
}

\subsection{Synthetic realizations}

% Table synthetic generation
\tabl{randmethod}
{Synthesis of correlated heterogeneous media by generating fractional Gaussian noise (fGn)
or fractional Brownian motion (fBm) with the spectrum $E^{(s)}_{H,b}(\mathbf{k})$.
The spatial Fourier transforms in $s$ dimensions
of the distributions $f(\mathbf{x})$ and $g(\mathbf{x})$
are respectively $F(\mathbf{k})$ and $G(\mathbf{k})$ with $|G(\mathbf{k})|=1$,
and $\mbox{sign}(\mathbf{k})=\mathbf{k}/|\mathbf{k}|$.}
{
\begin{center}
    \begin{tabular}{|c|l|c|c|}
      \hline
      {\it Steps} & {\it Domain} & {\it Operation for fGn} & {\it Operation for fBm} \\[0.2mm]
      \hline
      1 & Space & \multicolumn{2}{c|}{Generate Gaussian white noise $g(\mathbf{x})$ with zero mean and unit variance} \\[0.2mm]
      \hline
      2 & Fourier & \multicolumn{2}{c|}{Generate energy spectrum $E^{(s)}_{H,b}(\mathbf{k})$} \\[0.5mm]
      \hline
      3 & Fourier & 
      $F(\mathbf{k})=\sqrt{E^{(s)}_{H,b}(\mathbf{k})}~G(\mathbf{k})$ & 
      $F(\mathbf{k})=-i~\mbox{sign}(\mathbf{k})~\sqrt{E^{(s)}_{H,b}(\mathbf{k})}~G(\mathbf{k})$ \\[0.5mm] 
      \hline
      4 & Space & {Obtain correlated fGn $f(\mathbf{x})$} & {Obtain correlated fBm $f(\mathbf{x})$} \\
      \hline
    \end{tabular}
\end{center}
}

Our method of synthesizing correlated random media is summarized in Table~\ref{tbl:randmethod}
for fractional Gaussian noise (fGn) and fractional Brownian motion (fBm).
The Gaussian nature of the initial white noise is contained in the phase of its Fourier transform $G(\mathbf{k})$, 
whereas the amplitude is constant with frequency  because the noise is white.
The causal integration, to produce the fBm, is performed by a phase rotation in the Fourier domain, strictly equivalent to the Hilbert transform.
The fractal property, below spatial scale $b$, is imposed by the amplitude $\sqrt{E^{(s)}_{H,b}}$ of the von K\'arm\'an model.

\inputdir{enerd}
\multiplot{4}{signal,f2dfile,cgaussb,fcgaussbp}{width=0.47\textwidth}
{Variations of $V_S$ in a high-resolution reservoir model based 
on seismic and well data from a field in Canada
(a) and its spectral energy density (b).
Synthetic realization of 2D fGn
using the von K\'arm\'an spectral amplitude with the exponent $\beta=1$ 
and elliptical anisotropy (c,\,d).}

For dimension $s=2$, the exponent in equation \ref{eqn:vkes} is $\beta=2H+2$.
Therefore, the energy spectrum of the 2D fGn with $H=-0.5$ in Figure~\ref{fig:signal,f2dfile,cgaussb,fcgaussbp}
is $E(f)\propto 1/f$.
\cite{Klimes_02} in fact proposed using $-0.5\leq H\leq 0$ to synthesize geologically realistic 2D models
with the von K\'arm\'an function.
Because sediments are made up of layers, we consider the autocorrelation function to be vertical transverse isotropic.
We use two different correlation lengths $b_x$ and $b_z$ for horizontal and vertical directions,
and define the Riemannian relative distance \cite[]{Goff_J88}:
\begin{eqnarray}
r/b=\sqrt{(r_x/b_x)^2+(r_z/b_z)^2}.
\end{eqnarray}
Figure~\ref{fig:signal,f2dfile,cgaussb,fcgaussbp}
shows, for comparison, the signal and associated energy spectrum of the synthetic fGn 
and of a 2D section from a high-resolution model of a clastic reservoir in Canada.
The spectrum of the synthetic heterogeneous medium is similar to the one from the reservoir model,
but the synthetic fGn, although exhibiting some comparable roughness in the space domain,
does not contain coherent, large geological structures, i.e. folded beds.


\subsection{Nonlinear parameter estimation on sonic well logs}


We propose to use the synthesis of a random medium 
detailed in Table~\ref{tbl:randmethod} for $s$=1
as a basis for the procedure to estimate heterogeneity parameters from sonic logs.
We achieve optimization by using a weighted least-squares method in the spectral domain
on the logarithm of the amplitude, with the model derived from equation \ref{eqn:vkes}\,:
\begin{eqnarray}
\ln|F(k)| & = & \ln|F(0)| - p\ln\left[1+(kb)^2\right],\label{eqn:nlinvmethod} \\
p & = & \frac{H}{2}+\frac{1}{4}\ \ \ \mbox{and} \\
|F(0)| & = & \sigma~\sqrt{2b~C^{(1)}_{H}}\,. \label{eqn:nlinvconstant}
\end{eqnarray}
We estimate the three parameters, $b$, $H$, and $\sigma$, using a separable least-squares method 
\cite[]{Golub_P73} for $\ln|F(0)|$ and the slope $p$, and a Gauss Newton optimization algorithm on the nonlinear parameter $b^2$.
Parameter $\ln|F(0)|$ is included in the optimization algorithm
because it is difficult to estimate directly from the zero-frequency component in the data.
Standard deviation $\sigma$, extracted from relation \ref{eqn:nlinvconstant}, is confirmed by direct evaluation on the spatial signal.
When applying the method, we first substract the signal expectation and use it as a scaling factor.
We have tested the efficiency of the algorithm on synthetic fGn and fBm 
generated by the procedure in Table~\ref{tbl:randmethod} with a discrete length of 4056 points.
Three synthetic fractal signals and their parameter estimations are shown 
in Figure~\ref{fig:cgaussM025,lligaussfM025,cgauss025,lligaussf025,cgauss05,lligaussf05}.
Results of the validation tests are presented in Table~\ref{tbl:synthetictest}.

\inputdir{karman1}
\multiplot{6}{cgaussM025,lligaussfM025,cgauss025,lligaussf025,cgauss05,lligaussf05}{width=0.35\textwidth}
{Synthetic signals generated as fGn with $H=-0.25$, $b=10$\,m, $\sigma=20~\%$ in (a); 
fBm with $H=0.25$, $b=5$\,m, $\sigma=30~\%$ in (c); and fBm with $H=0.5$, $b=5$\,m, $\sigma=30~\%$ in (e).
Parameter estimations on the logarithm of the spectral amplitude are shown on the right (b,\,d,\,f).}

\inputdir{karmand}
\multiplot{6}{signalA1,signalC2,llisignalfA1,llisignalfC2,rllisignalfA1L,rllisignalfC2L}{width=0.35\textwidth}
{Sonic log $V_P$ (a) from well N$^{\circ}$1 and $V_S$ (b) from well N$^{\circ}$3, scaled by their respective average value $V_0$.
Parameter estimation on the logarithm of the spectral amplitude (c,\,d) shows the existence of different slopes 
for low, medium, and high frequencies.
These tool artefacts are removed by restricting the estimation method to low frequency (e,\,f).}


% Table synthetic tests
\tabl{synthetictest}
{Comparison of the stochastic medium parameters
used to generate synthetics fGn and fBm and their 
recovery by the nonlinear estimation method.}
{
\begin{center}
    \begin{tabular}{|l|rcc|}
      \hline
      {\it Parameters} & $H$ & $b$~(m) & $\sigma$~(\%) \\[0.2mm]
      \hline
     {Generated fGn} & -0.25 & 10.0 & 20 \\[-0.4mm]
     {Recovered}     & -0.21 & 11.0 & 18 \\[0.2mm]
      \hline
     {Generated fGn} & -0.25 &  5.0 & 20 \\[-0.4mm]
     {Recovered}     & -0.21 &  5.9 & 17 \\[0.2mm]
      \hline
     {Generated fBm} &  0.25 &  10.0 & 30 \\[-0.4mm]
     {Recovered}     &  0.26 &  10.2 & 23 \\[0.2mm]
      \hline
     {Generated fBm} &  0.50 & 5.0 & 40 \\[-0.4mm]
     {Recovered}     &  0.51 & 5.3 & 32 \\[0.2mm]
      \hline
     {Generated fBm} & 0.75 & 3.0 & 40 \\[-0.4mm]
     {Recovered}     & 0.79 & 2.7 & 30 \\[0.2mm]
      \hline
    \end{tabular}
\end{center}
}

Well log data come from a sandy channel reservoir with a clastic overburden,
and the facies evolves from silty sandstone to mudstone,
which is characteristic of alluvial deposition.
Velocities $V_P$ and $V_S$ were both measured with a spatial sampling of $0.125$~m. 
Figure~\ref{fig:signalA1,signalC2,llisignalfA1,llisignalfC2,rllisignalfA1L,rllisignalfC2L} shows
the parameter estimation for two sonic logs.
Comparison with the method applied to the synthetics in 
Figure~\ref{fig:cgaussM025,lligaussfM025,cgauss025,lligaussf025,cgauss05,lligaussf05}
uncovers the existence of different slopes for different frequencies in Figures~\ref{fig:llisignalfA1} and \ref{fig:llisignalfC2}.
We can reasonably delimit three domains, denoted (A) for low frequencies, (B) for medium frequencies, and (C) for very high frequencies.
These domains can be identified by parameters $r_{S}$ 
and $r_{I}$, representing specific values of relative distance $r$, namely
\begin{eqnarray}
\mbox{(A) for}\ \ r\geq r_{S}, & \mbox{(B) for}\ \ r_{S}\geq r\geq r_{I}, & \mbox{(C) for}\ \ r_{I}\geq r, \nonumber
\end{eqnarray}
%\begin{array}{lr}
%\mbox{(A)} & r \geq r_{S}, \nonumber\\
%\mbox{(B)} & r_{S} \geq r \geq r_{I}, \nonumber\\
%\mbox{(C)} & r_{I} \geq r, \nonumber
%\end{array}
where 1\,m~$<r_{S}<2$\,m and $r_{I}<1$\,m.
The sharp break (B) in the medium frequencies
followed by a white noise (C) at high frequencies is characteristic of the tool artefact.
Data acquisition involves a convolution 
with a box-car window \cite[]{Shiomi_SO97,Dolan_BR98}.
Application of the estimation method is thus restricted to relative distances $r>r_S$,
and results are shown in Figures~\ref{fig:rllisignalfA1L} and \ref{fig:rllisignalfC2L}.

Results are summarized in Table~\ref{tbl:sonicgeneral} 
for the four different sonic logs $V_P$ and $V_S$.
The ratio $\langle V_P \rangle / \langle V_S \rangle$ is almost constant
for the four well logs and roughly equal to two.
The updated estimation in the spatial wavelength domain (A) produces reasonable results in Table~\ref{tbl:sonicgeneral}.
Standard deviation $\sigma$ varies from 20 to 45~\% 
and is larger for $V_S$ logs than for $V_P$ logs.
Correlation length $b$ is about 5~m for both $V_P$ and $V_S$,
except for the well N$^{\circ}$4, which is 2.5~m.
Exponent $H$ for $V_P$ varies from 0.1 to 0.4
and for $V_S$ from 0.2 to 0.6.
In Figure~\ref{fig:rfsignalC2,fitfiltb025},
comparison of the frequency content of one real sonic log
with one realization of a synthetic fBm, generated using similar parameters, shows that
the sonic-log data contain higher peaks for very large wavelengths.
We detected in the different sonic well logs the recurrence of some particular spatial cycles 
at 2.5~m, 5~m, 10~m, and 20~m.

% Table soniclogs
\tabl{sonicgeneral}
{Parameters estimation from four wells in a clastic overburden, 
for the full sonic logs, and for limited spatial frequency bandwidths 
using the indicated restriction on the relative distance $r$ to remove the tool artefacts.
Relevant physical values are underlined.}
{
\begin{center}
\begin{tabular}{|r|c|c|c|c|}
      \hline
      {Well} \hfill {Log} & {\it b} (m) & $H$ & $\sigma$ (\%) & {$\langle{V}\rangle$ (m/s)}\\[1mm]
      \hline
N$^{\circ}$1 \hspace{4mm} $V_P$ full & 0.79 & 1.32 & 17 & \underline{2791} \\
            {$r>1.5$~m} & \underline{6.70} & \underline{0.13} & \underline{22} & \\ \cline{2-5}
             $V_S$ full & 1.05 & 1.16 & 32 & \underline{1218} \\
            {$r>2.0$~m} & \underline{5.92} & \underline{0.21} & \underline{35} & \\
      \hline
N$^{\circ}$2 \hfill $V_P$ full & 1.90 & 0.92 & 27 & \underline{2842} \\
              $r>1.6$~m & \underline{5.34} & \underline{0.38} & \underline{29} & \\ \cline{2-5}
             $V_S$ full & 2.84 & 0.83 &  45 & \underline{1240} \\
              $r>1.5$~m & \underline{3.08} & \underline{0.62} & \underline{44} & \\
    \hline
N$^{\circ}$3 \hfill $V_P$ full & 1.34 & 1.16 & 20 & \underline{2787}\\
              $r>1.9$~m & \underline{7.22} & \underline{0.18} & \underline{21} & \\ \cline{2-5}
             $V_S$ full & 1.25 & 1.23 & 32 & \underline{1216} \\
              $r>1.8$~m & \underline{5.01} & \underline{0.32} & \underline{36} & \\
      \hline
N$^{\circ}$4 \hfill $V_P$ full & 0.64 & 1.98 & 18 & \underline{2745} \\
              $r>1.4$~m & \underline{2.58} & \underline{0.39} & \underline{38} & \\ \cline{2-5}
             $V_S$ full & 0.57 & 2.26 & 32 & \underline{1247} \\
              $r>1.3$~m & \underline{2.46} & \underline{0.56} & \underline{33} & \\
     \hline
\end{tabular}
\end{center}
}

\inputdir{spectra}
\multiplot{2}{rfsignalC2,fitfiltb025}{width=0.44\textwidth}
{Fourier spectrum of the scaled $V_S$ sonic log from well N$^{\circ}$ 3 (a). The shape of the low-frequency 
content is different from that of the Fourier spectrum of the fractional Brownian motion (b)
synthesized with the von K\'arm\'an model using $H=0.25$, $b=5$\,m, and $\sigma=30~\%$.}

\subsection{Fractal heterogeneities and cycles in sediments}

\cite{Odoherty_A71} and \cite{Anstey_D02a} described variations in
well logs by the superposition of different types of deposition,
leading to ``layers inside layers''.  Their classification includes
\begin{enumerate}
\renewcommand{\theenumi}{\arabic{enumi}}
\item A large number of small thickness layers ($\leq 1$~m)
for weakly transitional depositions with small reflection coefficients;
\item Cyclic layers of thicknesses from 1 to 10~m with sharp interfaces,
corresponding to fine layering depositions inside a facies
for short-period sea cycles; and
\item Horizons imaged by seismic reflection, i.e. different facies for a small number of thicker blocky layers associated with low-order cycles.
\end{enumerate}
They suggested that transmission losses could be compensated by
multiple reflections, depending on seismic wavelength.  This
classification is in agreement with the fact that high exponents, $H$,
appear for shorter scales, $b$, in Table~\ref{tbl:sonicgeneral}.  The
estimation performed on the sonic logs indicates fractal properties
for distances shorter than $b\simeq 5$\,m.  Acccording to
\cite{Anstey_D02a}, well log signals are the superposition of several
processes with different scales.  The von K\'arm\'an model captures
part of it.  Parameters extracted by our analysis describe
heterogeneities corresponding to type 2 of the O'Doherty-Anstey
classification, which is a fractal behavior inside major geological
units, at least from 10 down to 1~m, with a correlation length of 5~m.
Previous estimations of the correlation length on well logs were
produced by direct calculation of the spatial autocorrelation
\cite[]{White_SN90,Shiomi_SO97}.  \cite{White_SN90} suggested the
possibility of superposition of two correlation lengths at 5 and 20~m.
The wavelet detection analysis of gamma-ray and resistivity well logs
for a sandstone confirmed the strong evidence of local cyclicity in
the stratigraphic sequences \cite[]{Rivera_RJCA04}.  We think that
direct estimation of correlation distance $b$ using the
autocorrelation function, or our estimation method, captures the
shortest dominant cycle in the sedimentary layers.  This would explain
why the fractal behavior seems to hold for larger scales in
Figures~\ref{fig:rllisignalfA1L} and \ref{fig:rllisignalfC2L}.

Parameter $H$, estimated from well logs, is $0< H\leq0.5$ and 
consistent with an antipersistent fractional Brownian motion characteristic of cyclicity (see Table~\ref{tbl:fractclass}).
The Hurst exponent commonly exhibits some antipersistence  in sediments with values from 0.2
to 0.5 for sandstones \cite[]{Dolan_BR98,Lu_MFC02}.
High values of 0.5 and 0.6 could be interpreted, in a clastic context,
to be caused by a transitional deposition involving persistency, as in natural floods.
Natural flood records exhibit a Hurst exponent, $0.5\leq Hu \leq 1.0$, associated 
with so-called black noise \cite[]{Hurst_51,Mandelbrot_W69}.


\subsubsection{Seismic-scale heterogeneities}


Inside a main facies, we can not recover the short wavelengths of type 1 in the O'Doherty-Anstey 
classification because of well-logging tool limitation.
Type 3 of the classification includes
Milankovitch cycles of about 10 to 20~m
and third-order sea-level cycles from 15 to 300~m \cite[]{Anstey_D02a}.
Seismic reflectors are conventionally identified 
as being chronostratigraphic horizons separating different geological 
units. They correspond to the wavelength of conventional seismic surveys,
up to 100 Hz, and should induce some resonant scattering with ``friendly'' multiples.
Multipathing is observed for this ratio
of seismic wavelength to the size of heterogeneities,
as shown by numerical experiments \cite[]{Frankel_C86}.
This domain should be treated using wave-localization theory and the Rytov method.
These could nevertheless fail to explain the data because 
of quasi-periodicity of the medium at this scale, violating the ergodicity assumption required 
by wave-localization theory. 
Statistical methods using the autocorrelation function seem to be adapted 
to describe quasi-periodic media when the ratio of $b$ over the seismic wavelength is small.
When the wavelength is of the same size, local quasi-cyclicity of the sedimentary sequence should not be ignored \cite[]{Morlet_AFG82,Stovas_U07}.

The non-Gaussian nature and non-stationarity of sedimentary layers
call for more sophisticated methods to be used, especially in order to capture larger scale pseudo-cyclic heterogeneities, 
as, for example, a multifractal analysis \cite[]{Marsan_B03} or a local cyclicity detection by wavelet analysis \cite[]{Rivera_RJCA04}.
The wavelet transform was indeed introduced so that seismic signals in locally cyclic sedimentary layers could be analyzed \cite[]{Morlet_AFG82}.


\section{Scattering attenuation in 3D}


Different scattering regimes exist when waves propagate in heterogeneous media,
according to the ratio of the wavelength, $\lambda$, to the size, $b$, of heterogeneities.
The formalism including the different scattering regimes,
when heterogeneities are modeled by spherical inclusions,
is the Mie scattering theory.
Recent experimental results 
\cite[]{Legonidec_G07} 
on sonic-wave reflectivity in a granular medium,
made up of beads of size $b$ in a water tank, illustrate this classification~:
\begin{enumerate}
\item for low frequencies, when $\lambda>\pi b$, backward scattering is dominant,
the Born approximation can be used, and the regime is Rayleigh scattering;
\item for wavelengths similar in size to heterogeneity, when $\pi b>\lambda>\pi b/2$,
lateral scattering is important, multiples should not be neglected,
and the regime is called resonant scattering; and
\item for high frequencies, when $\pi b/2>\lambda$, waves are scattered mainly forward, and
localization theory and the Rytov formalism are appropriate.
\end{enumerate}
Experiments have demonstrated that wave reflectivity strongly decreases 
when the wavelength of the incident wave is twice the diameter of the beads,
for which lateral scattering starts to be dominant.
Mean wavefield formalism is valid only for a low frequency,
when the wavelength is larger than the size of heterogeneities
because of the assumptions in the derivation \cite[]{Karal_K64}.
The Born approximation can describe the Rayleigh regime and the approach of resonant phase scattering \cite[]{Frankel_C86,Sato_F98}.
For wavelengths shorter than the size of heterogeneities, 
artificial decoherence by phase randomization occurs \cite[]{Wu_82,Sato_F98}.
We intend to describe the 3D attenuation in a stochastic fractal medium,
when $kb\leq1$, which is relevant for seismic survey frequencies.
The limit of validity corresponds to wavelengths approaching the size of heterogeneities.
We assume heterogeneities to be isotropic.
The schemes in Figure~\ref{fig:schemescatter3dgg,schemescatterd,schemescatter3d}
compare a realistic geological structure
with two different end-member models.
Scattering calculations in 1D underestimate the scattering loss by small-scale heterogeneities.

\inputdir{XFig}
\multiplot{3}{schemescatter3dgg,schemescatterd,schemescatter3d}{width=0.25\textwidth}
{Schematic comparison of single scattering effects,
during a vertical wave propagation in sediments,
between a realistic geological structure (a) and two end-member models:
horizontal layers with propagation including 1D scattering (b)
and isotropic heterogeneities with 3D scattering (c).}

\subsection{Low-frequency waves in 3D isotropic heterogeneous media}


A scalar wave $u(\mathbf{x},\omega)$ in a weakly inhomogeneous medium 
\cite[]{Chernov_60,Tatarski_61,Karal_K64}
satisfies the Helmholtz wave equation\,
\begin{eqnarray}
\Delta u(\mathbf{x},\omega) + k_0^2\left[1+f(\mathbf{x})\right]^2u(\mathbf{x},\omega) & = & 0\,, \label{eqn:helmh}
\end{eqnarray}
where $f(\mathbf{x})$ is a small perturbation of the medium from homogeneity, and $k_0=\omega/c_0$.
Phase velocity $c_0$ is the background velocity.
Assuming a second-order stationary statistical distribution 
for fluctuations $f(\mathbf{x})$ and a zero expectation value $\langle f \rangle = 0$, spatial covariance of the velocity variations is
defined by relation \ref{eqn:covarf}.
Expectation $\langle u(\mathbf{x},\omega)\rangle$ of random plane-wave realizations
is calculated \cite[]{Karal_K64} 
using a perturbation theory to the second order in $f(\mathbf{x})$ by
\begin{eqnarray}
\left[\Delta+k_0^2(1+\sigma^2)\right]\langle u(\mathbf{x},\omega)\rangle -
4\,k_0^4\,\sigma^2\int N(\mathbf{x}^{\prime}-\mathbf{x})\,G(\mathbf{x},\mathbf{x}^{\prime},\omega)
\,\langle u(\mathbf{x}^{\prime},\omega)\rangle~d\mathbf{x}^{\prime}= 0\,,
\end{eqnarray}
where $G(\mathbf{x},\mathbf{x}^{\prime},\omega)$ is Green's function of the operator $\left[\Delta+k_0^2\right]$,
and integration is performed over the 3D space.
The dispersion relation for a plane wave propagating in the heterogeneous medium follows as
\begin{eqnarray}
\frac{k^2}{k_0^2} & = & 1 + \sigma^2\left[1 - 4\,k_0^2
\int N(\mathbf{r})\,G(\mathbf{r},\omega)\,e^{i\,\mathbf{k}\cdot\mathbf{r}}d\mathbf{r}\right], \label{eqn:dispersg}\\
G(\mathbf{r},\omega) & = & \frac{e^{ik_0r}}{4\pi r}\,,\label{eqn:green3d}
\end{eqnarray}
where $\mathbf{r} = \mathbf{x}^{\prime}-\mathbf{x}$ is the relative position, with an absolute value $r=|\mathbf{r}|$,
and $G(\mathbf{r},\omega)$ is the 3D isotropic free-space Green's function with the outward radiation condition 
\cite[]{Bleistein_CS01}.
The path of waves should be sufficiently long to significantly sample medium heterogeneities statistically \cite[]{Gist_94}.
The Born approximation is present because of Green's function.
Heterogeneities with the isotropic correlation function $N(r)$
produce an isotropic wave vector $\mathbf{k}$.
Combining Green's function in equation \ref{eqn:green3d} with the isotropic integral in equation \ref{eqn:etf3d} reduces 
the squared dispersion relation of equation \ref{eqn:dispersg} to
\begin{eqnarray}
\frac{k^2}{k_0^2} & = & 1 + \sigma^2\left[1 - \frac{4\,k_0^2}{k}\int_{0}^{\infty}\hspace{-2mm} N(r)\,e^{ik_0r}\sin(kr)\,dr\right].
\end{eqnarray}
Second-order expansion $k/k_0=1+O(\sigma^2)$ in the solution constrains validity to the domain 
$k_0b\ll 1/\sigma$, where $b$ is the characteristic length scale of the heterogeneities.
The second-order approximation for the 3D dispersion relation is finally
\begin{eqnarray}
\frac{k}{k_0} & = & 1 + \frac{\sigma^2}{2} + i\,\sigma^2\,k_0\left[S(0) - S(2k_0)\right]. \label{eqn:dispersc}
\end{eqnarray}
Quantity $S(k)$, introduced above, is related to the real and even function $E^{(1)}(k)$, defined by the isotropic integral of equation \ref{eqn:etf1d}\,:
\begin{eqnarray}
S(k) & = & \int_{0}^{+\infty}\hspace{-3mm}N(r)e^{ikr}dr\,, \label{eqn:defskint} \\
E^{(1)}(k)~\sigma^{-2} & = & \int_{-\infty}^{+\infty}\hspace{-3mm}N(r)e^{ikr}dr = 2~\mbox{Re}[S(k)]\,, \label{eqn:rsint} \\
2\,i\,\mbox{Im}[S(k)] & = & S(k) - S(-k).
\end{eqnarray}
Connection to the O'Doherty-Anstey formula is detailed in Appendix~A.


\subsection{Attenuation in 3D fractal media}


The energy spectrum, $E^{(1)}_{H,b}(k)$, of von K\'arm\'an's autocorrelation 
function $N_{H,b}(r)$ in equation \ref{eqn:vkes} is real and even\,:
\begin{eqnarray}
{E^{(1)}_{H,b}(k)}~{\sigma^{-2}} & = & \int_{-\infty}^{+\infty}\hspace{-3mm}N_{H,b}(r)\,e^{ikr}dr\,,\\
& = & C^{(1)}_{H}~\frac{2b}{\left(1+b^2k^2\right)^{H+\frac{1}{2}}}\,.
\end{eqnarray}
Values of $S(k)$ defined by equation \ref{eqn:defskint} are
\begin{eqnarray}
S(0) & = & C^{(1)}_{H}\,b\,, \\
Re[S(2k_0)] & = & \frac{C^{(1)}_{H}\,b}{(1+4\,b^2k_0^2)^{H+\frac{1}{2}}}\,.
\end{eqnarray}
Coefficient $C^{(1)}_{H}$, defined by equation \ref{eqn:calpha}, is an increasing function of exponent $H$
and has to be calculated numerically, except for some specific values\,:
$$
\begin{array}{lll}
C^{(1)}_{H}\sim{\pi}/{\Gamma(H)}\rightarrow 0 & \mbox{for}\ \ \ H\rightarrow 0\,, & \\
C^{(1)}_{-0.25}=1.3110\ldots\,, & C^{(1)}_{0.25}=0.5991\ldots\,, & C^{(1)}_{0.5}=1\,, \nonumber \\
 & C^{(1)}_{0.75}=1.3317\ldots\,, & C^{(1)}_{1}=\pi/2\,.
\nonumber
\end{array}
$$
The dispersion relation of equation \ref{eqn:dispersc} solves for an explicit solution of attenuation and dispersion\,:
\begin{eqnarray}
Re[k/k_0] & = & 1+\frac{\sigma^2}{2}\left(1+2k_0~Im[S(2k_0)]\right)\,,\\
Im[k/k_0] & = & \sigma^2k_0b~C^{(1)}_{H}
\left[1-\frac{1}{(1+4\,b^2k_0^2)^{H+\frac{1}{2}}} \right].
\end{eqnarray}
When $H=0.5$, the derivation produces simple expressions as detailed in Appendix~B.
The use of $S^{\ast}(k)=S(-k)$ with the Kramers-Kr\"{o}nig relation can be used 
to determine the real part of $k$.
In the context of the second-order approximation,
scattering attenuation in a von K\'arm\'an isotropic medium is
\begin{eqnarray}
\frac{1}{Q} = \frac{2\,Im[k]}{Re[k]} = 2~\sigma^2\,k_0b~C^{(1)}_{H}
\left[1-\frac{1}{(1+4\,b^2k_0^2)^{H+\frac{1}{2}}}\right].
\end{eqnarray}
For $k_0b\,\ll\,1$, the scattering attenuation reduces to the Rayleigh diffusion regime\,:
\begin{eqnarray}
\frac{1}{Q} & \simeq & 8\,\sigma^2\,C^{(1)}_{H}\left(H+\frac{1}{2}\right)\left(k_0b\right)^3.
\end{eqnarray}


\subsubsection{Penetration depth}


Waves propagating in disordered media are exponentially 
attenuated by scattering \cite[]{Odoherty_A71,White_SN90}.
We define penetration depth $d(f)$ to be the skin depth \cite[]{Vanderbaan_WS07}
for low-frequency waves propagating in the heterogeneous medium\,:
\begin{eqnarray}
\frac{1}{d(f)} & = & \frac{k_0}{2Q} = Im[k]\,,\\
d(f) & = & \frac{b~(1+4\,b^2k_0^2)^{H+\frac{1}{2}}}
{\sigma^2\,(k_0b)^2\,C^{(1)}_{H}\left[(1+4\,b^2k_0^2)^{H+\frac{1}{2}}-1\right]}\,, \label{eqn:penedep}
\end{eqnarray}
where $f$ is frequency.
Penetration depth $d(f)$ corresponds to a decrease of wave amplitude by $1/e$.
For two-way traveltime, recorded amplitude is 14~\% of the initial one.
Figure~\ref{fig:depthlfb05M025,depthlfb25025,depthlfb05025,depthlfb10025,depthlfb05075,llqfb05025}
shows the frequency dependence of the penetration depth for different values of parameters $b$ and $H$
and the scattering attenuation for acoustic P- and S- waves.
Seismic background velocities are $V_P=2700$~m/s and $V_S=1230$~m/s.

\inputdir{qcurve}
\multiplot{6}{depthlfb05M025,depthlfb25025,depthlfb05025,depthlfb10025,depthlfb05075,llqfb05025}{width=0.35\textwidth}
{Penetration depth for P (solid) and S (dashed) scalar waves
in heterogeneous media described by the von K\'arm\'an model with $\sigma=30~\%$.
For $b=5$\,m, a higher exponent $H$ decreases the penetration (a,\,c,\,e).
The value of $b$ strongly influences the penetration of waves (b,\,d).
The slope break in the attenuation curve (f) determines the frequency below which our scattering model
is valid.}

Scattering attenuation $1/Q$ is proportional to $1/\lambda^3$
at low frequencies and corresponds to the Rayleigh diffusion regime.
Attenuation increases at higher frequencies, where $1/Q$ is proportional to $1/\lambda$
and wavelength is comparable to the size of the heterogeneities.
Nevertheless, the validity of our scattering theory is constrained to 
the low-frequency bandwidth until the attenuation curves in Figure~\ref{fig:llqfb05025}
reach the change of slope at 45~Hz for S-waves and 70~Hz for P-waves.
For a conventional seismic survey and for correlation length $b=5$\,m, previously estimated,
wavelengths of P- and S-waves and ratio $\lambda/b$ are indicated in Table~\ref{tbl:seisfreq}.


% Table seismic frequencies
\tabl{seisfreq}
{Ratio of the wavelength $\lambda$ over the size of heterogeneities $b=5$\,m for realistic seismic frequencies,
when $V_P=2700$~m/s, $V_S=1230$~m/s.}
{
\begin{center}
\begin{tabular}{|c|c|c|c|}
\hline
Wave & Frequency~(Hz) & $\lambda$~(m) & $\lambda/b$ \\
\hline
P & 10  & 270 & 54 \\
  & 90  & 30  &  6 \\ \hline
S & 10  & 123 & 24 \\
  & 50  & 25  &  5 \\
\hline
\end{tabular}
\end{center}
}

Scattering is more important for seismic wavelengths with a dimension
similar to that of heterogeneities\,: high frequencies and S-waves,
because their wavelengths are shorter than P-waves, are more
attenuated.  Penetration depth is close to infinity at very low
frequencies but decreases drastically in a narrow frequency band,
depending on parameters $H$ and $b$ (see
Figure~\ref{fig:depthlfb05M025,depthlfb25025,depthlfb05025,depthlfb10025,depthlfb05075,llqfb05025}).
This steep descent shifts to higher frequencies when the fractal
exponent decreases, corresponding to a stronger cyclicity of the
layers.  A shorter correlation length of heterogeneities highly
improves penetration of high frequencies for both types of wave (see
Figures~\ref{fig:depthlfb25025}~and~\ref{fig:depthlfb10025}).  For
large-size heterogeneities, i.e. $b>20$~m, the scattering theory we
use is not valid because the seismic frequencies statisfy $k_0b\geq
1$.  Scattering regimes and the suggested description of
heterogeneities are summarized in Figure~\ref{fig:schemeshetb}.  Our
results therefore ignore the effects of large cycles in sediments.  We
refer the reader to \cite{Stovas_U07} for more information on the
effects of cycles on wave progagation.

\inputdir{XFig}
\plot{schemeshetb}{width=0.8\textwidth}
{Schematic representation, including the suggested description of heterogeneities, 
of the different scattering regimes depending on the ratio between
the seismic wavelength $\lambda$ and the size of heterogeneities $b$.
The scale is for indication only and depends on the frequency band of the survey.}

\subsubsection{Dominant frequency versus depth}

If seismic pulse is defined as a Ricker wavelet, a relation can be derived for
modification of the frequency content of P and S acoustic waves by scattering attenuation. 
Dominant frequency $f_{{dom}}(z)$ with depth $z$ and initial spectrum $S_I(f)$ of the source are defined by
\begin{eqnarray}
S_I(f) & = & \left(\frac{f}{f_0}\right)^2e^{-f^2/f_0^2},\\
f_{{dom}}(z) & = & \frac{d}{df}\left(S_I(f)\,e^{-z/d(f)}\right),
\end{eqnarray}
with initial condition $f_{{dom}}(0)=f_0$, and where $d(f)$ is the penetration depth
defined in equation \ref{eqn:penedep}.
Dispersion involves different traveltimes
at different frequencies but does not modify the frequency content or amplitude.
For convenience, we estimate dominant frequency as frequency expectation\,:
\begin{eqnarray}
f_{{dom}}(z) & = & \frac{1}{\langle S_I \rangle_z}\int f\,S_I(f)\,e^{-z/d(f)}\,df\,,\\
\langle S_I \rangle_z & = & \int S_I(f)\,e^{-z/d(f)}\,df\,.
\end{eqnarray}
Figure~\ref{fig:fdomfb25025,fdomfb05M025,fdomfb05025,fdomfb0505,fdomfb10025,fdomfb05075}
shows the evolution of the dominant frequency with depth 
in fractal media, with $V_P=2700$~m/s, $V_S=1230$~m/s, and standard deviation $\sigma=30\,\%$.
The value of correlation length $b$ again has a very high impact,
whereas the fractal exponent moderately influences results.
For a multicomponent seismic survey in a clastic reservoir,
evolution of the peak frequency should show a more 
important decrease with depth for PS data than for PP data.

\inputdir{pdepth}
\multiplot{6}{fdomfb25025,fdomfb05M025,fdomfb05025,fdomfb0505,fdomfb10025,fdomfb05075}{width=0.35\textwidth}
{Evolution of the dominant frequency with depth for P (solid line) and S (dashed line) scalar waves modeled by a Ricker wavelet ($f_0=60$\,Hz)
in heterogeneous media with $\sigma=30~\%$. For a constant exponent $H=0.25$, 
the dominant frequency shifts to lower frequencies faster for larger values of $b$ (a,\,c,\,e).
The exponent $H$ weakly influences the evolution of the dominant frequency (b,\,d,\,f).}

\section{Discussion}


The attenuation caused by anticorrelated 3D small-scale heterogeneities
can be explained by a low-frequency scattering theory.
The length scales that we estimated from sonic logs justify this approach 
for conventional seismic frequencies.
The intensity of scattering attenuation and the value of
frequency cutoff strongly depend on the size of the heterogeneities, and
S-waves are more attenuated than P-waves at the same frequency.
Using low-frequency P-waves provides a better depth of penetration. More reflectors can be detected and imaged,
but, of course, with less resolution. This phenomenon was observed  
in sub-basalt imaging \cite[]{Ziolkowski_HGJDHLL03}.
Our analysis of sonic logs confirms the relevance
of a fractal description for the high-frequency content of quasi-periodic geological layers.
Because sediments are highly stratified, their layered structure
has previously motivated use of 1D models for seismic scattering attenuation,
but a realistic estimate needs to be conducted in 3D.
More generally, description of geological heterogeneities and 
use of scattering theories should be as depicted in Figure~\ref{fig:schemeshetb}.

Several limitations should be pointed out in our study. 
Well logs constitute 1D samples of the geological medium in the near vertical direction.
Use of 3D isotropy is the simplest assumption 
consistent with our limited knowledge.
We consequently ignore the anisotropic effect of layering, which undoubtly affects lateral scattering when $\lambda\sim b$.
Also note that the relation between fractal exponent $\beta$
and parameter $H$ depends on spatial dimension $s$.
We have therefore proposed different values of parameter $H$ for depth of penetration and evolution of dominant frequency
in Figures~\ref{fig:depthlfb05M025,depthlfb25025,depthlfb05025,depthlfb10025,depthlfb05075,llqfb05025}
and~\ref{fig:fdomfb25025,fdomfb05M025,fdomfb05025,fdomfb0505,fdomfb10025,fdomfb05075}
and have attempted to extract some general trends.

Our analysis is limited by the fact that we used variations of the
seismic velocity, but not of density, in order to be consistent with
the mean field theory.  This theory retrieves the Backus limit and the
Rayleigh diffusion regime.  Fortunately, densities commonly exhibit
fewer variations.  Meanwhile, analysis of the logarithm of impedance
$Z$ relates directly to the reflection coefficients
\cite[]{Shtatland_91}.  Backscattering is known to be related to
impedance fluctuations \cite[]{Banik_LS85,Wu_88}.  Modification of the
Helmholtz equation~(\ref{eqn:helmh}) in 1D to incorporate $\ln\,Z$ was
succesfully achieved by \cite{Banik_LS85}, and they provided a proof
of O'Doherty-Anstey formula.  Impedance $Z(z)$, depending on depth
$z$, and reflection coefficient series $R(z)$ are connected by
\begin{eqnarray}
\lim_{dz\rightarrow 0}\,R(z+dz/2) & = & \lim_{dz\rightarrow 0}\,\frac{Z(z+dz)-Z(z)}{Z(z+dz)+Z(z)},\\
R(z) & = & \frac{1}{2}~{d\ln Z(z)}.
\end{eqnarray}
Reflection series can reasonably be considered to be Gaussian and stationary
only inside blocky layers, using the segmentation method \cite[]{Todoeschuck_JL90}.
Incrementation of the fractional Gaussian noise, corresponding to
reflection series $R(z)$, produces the non-stationary and non-Gaussian fBm describing quantity $\ln Z(z)$
\cite[]{Shtatland_91}.
A white spectrum of reflectivity coefficients occurs for $\beta=2$ and generates a Brownian walk describing $\ln Z(z)$
and involving an exponential autocorrelation.
We advocate the use of a non-white reflectivity hypothesis, as previously recommended by 
several authors \cite[]{Todoeschuck_JL90,Lancaster_W00,Anstey_D02b}.

\section{Conclusions}

The high-frequency quasi-cyclic variations of seismic velocities
can be described as an antipersistent fractional Brownian motion
as demonstrated by our sonic-log data from a clastic reservoir.
The correlation length, estimated for von K\'arm\'an's model,
is about 5\,m, but the sonic logs contain
larger local cycles at 10 and 20~m~: our method extracts  
one dominant cycle of deposition.
Conventional seismic surveys contain frequencies as high as 100 Hz,
with typical peak frequencies at 25 Hz.
Our statistical description of geological heterogeneities below 10~m
can thus be used consistently in our low-frequency scattering theory
to estimate the scattering loss caused by small-scale heterogeneities.

Shear waves have shorter wavelengths than compressional waves
and can be more attenuated because they are more sensitive to heterogeneities.
We showed the existence of a high-frequency cutoff for the depth of penetration of waves,
whose position in frequency depends on the maximum size of fractal heterogeneities. 
The dominant frequency of a wavelet decreases faster for
higher fractal exponents and for larger characteristic sizes of heterogeneities.
This loss of high-frequency content
influences resolution in seismic imaging.
Our study recommends using low-frequency P-waves for deep targets 
under a strongly heterogeneous overburden.

Agreement of our results with the Backus limit
and the Rayleigh diffusion regime is due to the use of velocity fluctuations.
Nevertheless, proper connection with the O'Doherty-Anstey formula requires
use of the logarithm of impedance.
Moreover, more complex, multiple scattering occurs when sizes of heterogeneities
are similar to that of the seismic wavelength.
Large-scale local cycles, present in the sonic-log data,
call for incorporation of resonant scattering effects
into high-frequency scattering theories.


\section{ACKNOWLEDGMENTS}


We thank Apache Canada Ltd. for providing the data used in this study and Mark Tomasso 
from the Bureau of Economic Geology for the velocity model.
We acknowledge Apache Corporation, GX Technology Corporation, 
and the Jackson School of Geosciences for financial support.
This publication is authorized by the Director, Bureau of Economic Geology,
The University of Texas at Austin.


\append{O'Doherty-Anstey formula and the mean field theory}
\label{Odohertyformula}


O'Doherty and Anstey \cite[]{Odoherty_A71,Resnick_90} proposed 
that local transmission coefficient $T(\omega)$ for traveltime $\Delta t$ in sedimentary layers 
should be 
\begin{eqnarray}
T(\omega) = \exp\left[-\frac{\omega\,\Delta t}{2\,Q(\omega)}\right] = e^{-R(\omega)\,\Delta t},
\end{eqnarray}
where $R(\omega)$ is the spectrum of reflection coefficients, which is related to the spectrum
of impedance fluctuations \cite[]{Banik_LS85}.
Because we have used velocity fluctuations, attenuation in the dispersion relation of equation \ref{eqn:dispersc}
depends on the velocity spectrum\,:
\begin{eqnarray}
\frac{1}{Q(\omega)} & = & 2~\frac{Im[k]}{Re[k]}, \\
& = & {k_0}\left[E^{(1)}(0) - E^{(1)}(2k_0)\right], \\
T(\omega)  & = & \exp\left\{-\frac{k_0\,\omega}{2}\,\left[E^{(1)}(0) - E^{(1)}(2k_0)\right]\Delta t\right\}. \label{eqn:rspec}
\end{eqnarray}
These formulae have been previously derived and analyzed \cite[]{Lerche_86,Wu_88,Sato_F98}.
The term $E^{(1)}(2k_0)$ can be interpreted as backward scattering
with exchange wavenumber $2k_0$, whereas the term $E^{(1)}(0)$ is forward scattering
with exchange wavenumber $0$.
Both energy terms reduce to 1D expressions because of isotropic integration, 
whereas one symmetry axis is imposed by propagation direction of the wave.
Further interpretation can be found using a dynamic effective model
for multiple scattering \cite[]{Waterman_T61}\,:
the scattered waves interfere with the main wavefield, 
and their relative phase continuously changes in all directions,
except for significant interferences in forward and backward directions.
Previous 1D derivations \cite[]{Sato_F98}
have incorporated a traveltime correction, corresponding to neglecting forward scattering
in order to reproduce the O'Doherty-Anstey formula. This approach extends validity of the analytical expressions
to higher frequencies, but no simple traveltime phase correction exists for the mean field theory in 3D.
Constant $E^{(1)}(0)$ ensures recovery of the Backus effective medium and the Rayleigh diffusion regime for very low frequencies.


\append{Exponentially correlated heterogeneities}
\label{Expohet}


Results have been derived several times \cite[]{Karal_K64,Sato_F98} 
using exponential correlation function
$N(r)=\exp[-r/b]$ and are added here as a specific case
with simple analytical expressions within the general framework of von K\'arm\'an's media.
Values of the integral in equation \ref{eqn:defskint} are
\begin{eqnarray}
S(0) & = & b\,,\\
S(2k_0) & = & \frac{b}{1-2\,i\,k_0b}\,.
\end{eqnarray}
Using these expressions in the dispersion relation of equation \ref{eqn:dispersc}, the new dispersion relation, phase velocity $c(\omega)$, and attenuation are
\begin{eqnarray}
\frac{k}{k_0} & = & \frac{c_0}{c(\omega)} + \frac{i}{2\,Q(\omega)} \; = \; 1 + \frac{\sigma^2}{2}\left[1 + \frac{(2\,k_0b)^2}{1-2\,i\,k_0b}\right] + O(\sigma^4)\,,  \\
\frac{1}{c(\omega)} & = & \frac{1}{c_0}
\left[1+\sigma^2~\frac{1/2+(2\,k_0b)^2}{1+(2\,k_0b)^2}\right]\,, \label{eqn:vdispe1} \\
\frac{1}{Q(\omega)} & = & \sigma^2~\frac{8\,(k_0b)^3}{1+(2\,k_0b)^2}\,. \label{eqn:vqe1}
\end{eqnarray}
In the limit of very long wavelengths, i.e. $k_0\rightarrow 0$,
attenuation and velocity reduce respectively to the Rayleigh diffusion
regime and the effective medium theory of \cite{Backus_62}\,:
\begin{eqnarray}
\frac{1}{c(\omega)} & \rightarrow & \frac{1}{c_0}\left(1+\frac{\sigma^2}{2}\right)\,,\\
\frac{1}{Q(\omega)} & \sim & 8\,\sigma^2\,(k_0b)^3 \rightarrow 0\,.
\end{eqnarray}


\bibliographystyle{seg}  % style file is seg.bst
\bibliography{atten}

%\end{document}
