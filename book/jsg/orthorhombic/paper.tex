\published{Geophysics, 78, no. 4, C33-C40, (2013)}

\title{Modeling of pseudo-acoustic P-waves in orthorhombic media with a lowrank approximation}
\lefthead{Song \& Alkhalifah}
\righthead{Orthorhombic Pseudo-acoustic Modeling}

\renewcommand{\footnotemark}{\fnsymbol{footnote}} 

\author{Xiaolei Song\footnotemark[1] and Tariq Alkhalifah\footnotemark[2]}

\address{
\footnotemark[1]Bureau of Economic Geology \\
Jackson School of Geosciences \\
The University of Texas at Austin \\
University Station, Box X \\
Austin, TX 78713-8924 \\
USA\\
songxl@utexas.edu\\
\footnotemark[2]Physical Sciences and Engineering \\
King Abdullah University of Science and Technology \\
Mail box \# 1280 \\ 
Thuwal 23955-6900 \\
Saudi Arabia\\
tariq.alkhalifah@kaust.edu.sa
}

\maketitle

\begin{abstract}
Wavefield extrapolation in pseudo-acoustic orthorhombic anisotropic media suffers from 
wave-mode coupling and stability limitations in the parameter range.
We use the dispersion relation for scalar wave propagation in pseudo-acoustic orthorhombic media to model
acoustic wavefields. The wavenumber-domain application of the Laplacian operator allows us to propagate the P-waves exclusively,
without imposing any conditions on the parameter range of stability. It also allows us to avoid dispersion artifacts commonly associated
with evaluating the Laplacian operator in space domain using practical finite difference stencils.
To handle the corresponding space-wavenumber mixed-domain operator,
we apply the lowrank approximation approach. Considering the number of parameters necessary to describe orthorhombic anisotropy,
the lowrank approach yields space-wavenumber decomposition of the extrapolator operator that is dependent on space location regardless of the parameters,
a feature necessary for orthorhombic anisotropy.
Numerical experiments show that the proposed wavefield extrapolator is accurate and practically free of dispersion.
Furthermore, there is no coupling of qSv and qP waves,
because we use the analytical dispersion solution corresponding to the $P$-wave. 
\end{abstract}

\section{Introduction}
Nowadays, a growing number of seismic modeling and imaging techniques
are being developed to handle wave propagation in transversely isotropic
media (TI).  Such anisotropic phenomena are typical in sedimentary
rocks, in which the process of lithification usually produces
identifiable layering.  
In anisotropic media, the velocity is no longer described by a single parameter.
Equations for anisotropic wave propagation are more complicated, even for simple cases.
Although exact expressions for phase velocities in VTI media involve four independent parameters,
It has been observed that only three parameters influence wave propagation and are of interest
to surface seismic processing \cite[]{alkhalifah}.
Different approximations have been developed to simplify anisotropic equations, such as the weak-anisotropy
approximation \cite[]{thomsen}, elliptical approximations \cite[]{helbig,dell1}, acoustic approximations \cite[]{alkhalifah1,alkhalifah2} and anelliptic approximations \cite[]{dell2,muir,anelliptic}.
Tectonic movement of the crust may rotate the
rocks and tilt the natural vertical orientation 
of the symmetry axis
(VTI), causing a tilted TI (TTI) anisotropy.  
In addition, tectonic
stresses may also fracture rocks, inducing another TI with a symmetry
axis parallel to the stress direction and usually normal to the
sedimentation-based TI.  The combination of these effects can be
represented by an orthorhombic model with three mutually
orthogonal planes of mirror symmetry; 
the P-waves in each symmetry plane can be described kinematically as an independent TI model.
Realization of the importance of
orthorhombic models mainly comes from observation of azimuthal velocity variations in flat-layered rocks, which may
indicate valuable fracture properties of reservoirs \cite[]{grechka}.\\

Wavefields in anisotropic media are well described by the aniso\-tropic
elastic-wave equation.  However, in practice, we often have little
information about shear waves and prefer to deal with
scalar wavefields, especially for conventional imaging 
of subsurface structure.
\cite{alkhalifah2} derived
an acoustic scalar wave equation for VTI media
by careful reparametrization followed by setting the shear velocity along the symmetry axis to zero,
which provided accurate kinematics for the
conventional elastic wavefield.  
Later on, \cite{alkaor} followed the
same approach and introduced an acoustic wave equation of the sixth order
in axis-aligned orthorhombic media.  
\cite{fowleror} presented coupled
systems of partial differential equations for pseudo-acoustic
wave propagation in orthorhombic media by extending their previous
work in TI media \cite[]{fowler}.  
\cite{zhangor} extended 
self-adjoint differential operators in TTI media \cite[]{duveneck,zhangtti} 
to orthorhombic media.\\

Pseudo-acoustic P-wave modeling with coupled equations may have shear-wave numerical artifacts in the simulated wavefield \cite[]{grechkat,zhang2,duveneckt}.
Those artifacts as well as sharp changes in symmetry axis tilting may introduce severe numerical 
dispersion and instability in modeling. 
\cite{yoon} proposed to reduce the instability by making $\epsilon \,=\, \delta$ in regions 
with rapid tilt changes. 
\cite{fletcher} suggested that including a finite shear-wave velocity enhances the stability 
when solving the coupled equations.
These methods can alleviate the instability problem; 
however, they may alter the wave propagation kinematics or still leave shear-wave components in the P-wave simulation. 
Shear-wave artifacts can be removed from the P-wavefield in the phase-shift extrapolation method because the P- and S-wave solutions lie in a different part of the wavenumber spectrum \cite[]{bale}.
A number of spectral methods are proposed to provide solutions 
which can completely avoid the shear-wave artifacts \cite[]{etgen,songx,lr,songxl,chu,ge,fowlerg}. 


 
In this paper, we adopt a dispersion relation for orthorhombic
anisotropic media \cite[]{alkaor} and introduce a mixed-domain
acoustic wave extrapolator for time marching in orthorhombic media.
We use the lowrank approximation \cite[]{ying,lr} to handle this
mixed-domain operator.
We demonstrate by
numerical examples that our method is kinematically accurate.
Furthermore, there is no coupling of quasi-P and quasi-SV waves in the
wavefield and no constraints on Thomsen's parameters required for
stability.
 

\section{Theory}

\subsection{Acoustic Wave Extrapolation}
The acoustic wave equation is widely used in
forward seismic modeling and reverse-time migration \cite[]{bednar,etgen1}:
\begin{equation}
\label{eq:acoustic} 
\frac{\partial^2p}{\partial t^2} = v(\mathbf{x})^2\,\nabla^2p\;,
\end{equation}
where $p(\mathbf{x},t)$ is the seismic pressure wavefield
and $v(\mathbf{x})$ is the wave propagation velocity.\\

Assuming the model is homogeneous $v(\mathbf{x}) \equiv v_0$,
after a Fourier transform in space,
 we get the following explicit expression in the wavenumber domain:
\begin{equation}
\label{eq:ode} 
\frac{d^2\hat{p}}{dt^2} = -v_0^2|\mathbf{k}|^2\hat{p}\;,
\end{equation}
where
\begin{equation}
\label{eq:p} 
\hat{p}(\mathbf{k},t)=\int^{+\infty}_{-\infty}{p(\mathbf{x},t)e^{i\mathbf{k}\cdot\mathbf{x}}d\mathbf{x}}\;.
\end{equation}
\\
Equation~\ref{eq:ode} has the following analytical solution:
\begin{equation}
\label{eq:fourier} 
\hat{p}(\mathbf{k},t+\Delta t) = e^{\pm i|\mathbf{k}|v_0\Delta t}\hat{p}(\mathbf{k},t)\;,
\end{equation}
\\
which leads to
the well-known second-order time-marching scheme \cite[]{Etgen.sep.60.131,yu}\hyphenation{Sou-ba-ras}:
\begin{eqnarray}
\label{eq:exact} 
\lefteqn{p(\mathbf{x},t+\Delta t)+p(\mathbf{x},t-\Delta t)  = }\nonumber \\
& & 2\int^{+\infty}_{-\infty}{\hat{p}(\mathbf{k},t)\cos(|\mathbf{k}|v_0\Delta t)e^{-i\mathbf{k}\cdot\mathbf{x}}d\mathbf{k}}\;.
\end{eqnarray}

Equation~\ref{eq:exact} provides a very accurate and efficient solution
in the case of a constant-velocity medium with the aid of FFTs.
When the seismic wave velocity varies in the medium,
equation~\ref{eq:exact} turns into a reasonable approximation by replacing $v_0$ with $v(\mathbf{x})$, and taking small time steps, $\Delta t$.
However, FFTs can no longer be applied directly to evaluate
the inverse Fourier transform,
because a space-wavenumber mixed-domain term appears in the integral operation:
\begin{equation}
\label{eq:mix}
W(\mathbf{x},\mathbf{k})=\cos(|\mathbf{k}|v(\mathbf{x})\Delta t).
\end{equation}
As a result, a straightforward numerical implementation of wave extrapolation
in a variable velocity medium with mixed-domain
matrix \ref{eq:mix} will increase the cost from $O(N_xlogN_x)$ to $O(N_x^2)$,
the original cost for the homogeneous case,
in which $N_x$ is the total size of the three-dimensional space grid.
A number of numerical methods \cite[]{etgen,morton,zhang,du,ying,lr,songx,lfd,songxl}
have been proposed to overcome this mixed-domain problem.\\

In the case of orthorhombic acoustic modeling, 
we derive a new phase operator $\phi(\mathbf{x},\mathbf{k})$
to replace $|\mathbf{k}|v(\mathbf{x})$ of the isotropic model.
We describe the details in the next section.


\subsection{Dispersion Relation for Orthorhombic Anisotropic Media}

In transversely isotropic (TI) media, 
the model is fully characterized by five elastic parameters and density.
In orthorhombic media, nine elastic parameters and density are needed to describe the elastic model.
The stiffness tensor $c_{ijkl}$ for an orthorhombic model can be represented, 
using the compressed two-index Voigt notation, as follows:
\begin{equation}
  \label{eq:stiffness}
  \mathbf{C}= \left[\begin{array}{llllll}
           c_{11} & c_{12} & c_{13} & 0 & 0 & 0 \\
           c_{12} & c_{22} & c_{23} & 0 & 0 & 0 \\
           c_{13} & c_{23} & c_{33} & 0 & 0 & 0 \\
           0      & 0      & 0      & c_{44} & 0 & 0  \\
           0      & 0      & 0      & 0  & c_{55} & 0   \\
           0      & 0      & 0      & 0  & 0 & c_{66}    
          \end{array}\right]. 
\end{equation}

%\cite{alkaor} introduced nine parameters determined from the above stiffness tensor:
Instead of strictly adhering to the orthorhombic media used by \cite{tsvank,tsvan}, \cite{alkaor} slightly changed the notations and used the following nine parameters determined from the above stiffness tensor:
\begin{equation}
  \label{eq:p9}
   \begin{array}{l}
           v_v=\sqrt{\frac{c_{33}}{\rho}} \\ 
           v_{s1}=\sqrt{\frac{c_{55}}{\rho}} \\ 
           v_{s2}=\sqrt{\frac{c_{44}}{\rho}} \\
           v_{s3}=\sqrt{\frac{c_{66}}{\rho}} \\ 
           \displaystyle v_{1}=\sqrt{\frac{c_{13}(c_{13}+2c_{55})+c_{33}c_{55}}{\rho(c_{33}-c_{55})}} \\ 
           \displaystyle v_{2}=\sqrt{\frac{c_{23}(c_{23}+2c_{44})+c_{33}c_{44}}{\rho(c_{33}-c_{44})}}\\
           \displaystyle \eta_1=\frac{c_{11}(c_{33}-c_{55})}{2c_{13}(c_{13}+2c_{55})+2c_{33}c_{55}}-\frac{1}{2} \\ 
           \displaystyle \eta_2=\frac{c_{22}(c_{33}-c_{44})}{2c_{23}(c_{23}+2c_{44})+2c_{33}c_{44}}-\frac{1}{2} \\ 
           \displaystyle \delta=\frac{(c_{12}+c_{66})^2-(c_{11}-c_{66})^2}{2c_{11}(c_{11}-c_{66})}, \\
          \end{array} 
\end{equation}
where $v_v$ is P-wave vertical phase velocity, 
$v_{s1}$ and $v_{s2}$ are S-wave vertical phase velocity 
polarized in the $[x_2,x_3]$ and $[x_1,x_3]$ planes, 
$v_{s3}$ is S-wave horizontal phase velocity 
polarized in the $[x_1,x_3]$ but propagating in the $x_1$ direction, 
$v_1$ and $v_2$ are NMO P-wave velocities 
for horizontal reflectors in the $[x_1,x_3]$ and $[x_2,x_3]$ planes, 
and $\eta_1$, $\eta_2$, and $\delta$ are anisotropic parameters 
in the $[x_1,x_3]$, $[x_2,x_3]$, and $[x_1,x_2]$ planes.\\
 
The Christoffel equation in 3D anisotropic media takes the
following general form \cite[]{chapman}:
\begin{equation}
  \label{eq:chris}
\Gamma_{ik}(x_s,\mathbf{p})=a_{ijkl}(x_s)p_jp_l-\delta_{ik},
\end{equation}
with $p_j=\frac{\partial \tau}{\partial x_j}$ and $a_{ijkl}=\frac{c_{ijkl}}{\rho}$,
where $p_j$ are components of the phase vector $\mathbf{p}$, 
$\tau$ is travel-time along the ray, $\rho$ is density,
$x_s, s=1,2,3$ are Cartesian coordinates for position along the ray,
and $\delta_{ik}$ is the Kronecker delta function. \\

\cite{alkhalifah1} pointed out that setting the S-wave velocity to zero
does not compromise accuracy in traveltime computations for TI media.
This conclusion can be applied to orthorhombic media as well \cite[]{tsvank}.
\cite{alkaor} showed that the kinematics of wave propagation
is well described by acoustic approximation. 
\\

In orthorhombic media, the Christoffel equation~\ref{eq:chris} reduces to the following form if $v_{s1}$, $v_{s2}$, and $v_{s3}$ are set to zero:
%\begin{equation}
%  \label{eq:ch}
%   \left[\begin{array}{lll}
%     p_1^2v_1^2(1+2\eta_1)-1 & \gamma p_1p_2v_1^2(1+2\eta_1) & p_1p_3v_1v_v \\
%     \gamma p_1p_2v_1^2(1+2\eta_1) & p_2^2v_2^2(1+2\eta_2)-1 & p_2p_3v_2v_v \\
%     p_1p_3v_1v_v & p_2p_3v_2v_v & p_3^2v_v^2-1
%          \end{array}\right]. 
%\end{equation}
\begin{equation}
  \label{eq:ch}
   \left[\begin{array}{lll}
     p_1^2v_1^2\xi_1-1 & \gamma p_1p_2v_1^2\xi_1 & p_1p_3v_1v_v \\
     \gamma p_1p_2v_1^2\xi_1 & p_2^2v_2^2\xi_2-1 & p_2p_3v_2v_v \\
     p_1p_3v_1v_v & p_2p_3v_2v_v & p_3^2v_v^2-1
          \end{array}\right], 
\end{equation}
where $\gamma=\sqrt{1+2\delta}$, $\xi_1=1+2\eta_1$ and  $\xi_2=1+2\eta_2$.\\

We evaluate the determinant of matrix~\ref{eq:ch} and set it to zero. 
After replacing $p_1$ with $\frac{k_x}{\phi}$,
$p_2$ with $\frac{k_y}{\phi}$, 
and $p_3$ with $\frac{k_z}{\phi}$, 
we obtain a cubic polynomial in $\phi^2$ as follows: 
\begin{equation}
  \label{eq:phi}
   \begin{array}{l}
-\phi ^6+\phi ^4 \left(2 v_1^2 \eta _1 k_x^2+v_1^2 k_x^2+2v_2^2 \eta _2 k_y^2+v_2^2 k_y^2+v_v^2 k_z^2\right)\\
+\phi ^2 (v_1^4 \gamma ^2 \xi_1^2 k_x^2k_y^2-v_2^2 v_1^2 \xi_1 \xi_2
k_x^2 k_y^2-2 v_v^2 v_1^2 \eta _1 k_x^2 k_z^2\\
-2 v_2^2 v_v^2 \eta _2k_y^2 k_z^2)-v_1^4 v_v^2 \gamma ^2 \xi_1^2k_x^2 k_y^2 k_z^2+2 v_1^3 v_2 v_v^2 \gamma \xi_1k_x^2 k_y^2 k_z^2\\
-v_1^2 v_2^2 v_v^2\left(1-4 \eta _1 \eta _2\right) k_x^2 k_y^2 k_z^2=0\;.
\end{array}
\end{equation}

One of the roots of the cubic polynomial corresponds to P-waves in acoustic media and is given by the following expression:
\begin{equation}
  \label{eq:root}
\phi^2=\frac{1}{6}\left|-2^{2/3}d -\frac{2 \sqrt[3]{2} \left(a^2+3 b\right)}{d}+2 a\right|\;,
\end{equation}
where

$$a=2 v_1^2 \eta _1 k_x^2+v_1^2 k_x^2+2 v_2^2 \eta _2 k_y^2+v_2^2k_y^2+v_v^2 k_z^2,$$
$$b=v_1^4 k_x^2 k_y^2 \left(2 \gamma \eta _1+\gamma \right){}^2-v_2^2
v_1^2 \left(2 \eta _1+1\right) \left(2 \eta _2+1\right) k_x^2 k_y^2$$
$$-2
v_v^2 v_1^2 \eta _1 k_x^2 k_z^2-2 v_2^2 v_v^2 \eta _2 k_y^2 k_z^2,$$
$$c=v_v^2 v_1^4 \left(-k_x^2\right) k_y^2 k_z^2 \left(2 \gamma \eta_1+\gamma \right){}^2+2 v_2 v_v^2 v_1^3 \gamma \left(2 \eta_1+1\right) k_x^2 k_y^2 k_z^2$$
$$-v_2^2 v_v^2 v_1^2 \left(1-4 \eta _1 \eta_2\right) k_x^2 k_y^2 k_z^2,$$
$$d=\sqrt[3]{-2 a^3+3 \left(e-9 c\right)-9
a b},$$
$$e=\sqrt{|-3 b^2\left(a^2+4 b\right)+6 a c \left(2 a^2+9 b\right)+81 c^2|}.$$


This root reduces to the isotropic $P$-wave solution 
when we set $v_1=v_2=v_3=v$, $\eta_1=\eta_2=0$, and $\gamma=1$, 
in which $\phi$ in expression~\ref{eq:root} is then given by $|\mathbf{k}|v$, which is the same dispersion relation in isotropic media as that is shown in equation~\ref{eq:mix}. 
In VTI media: 
$v_1=v_2=v$, $\eta_1=\eta_2=\eta$, and $\gamma=1$, 
$\phi$ in expression~\ref{eq:root} reduces to
\begin{equation}
\label{eq:tti} 
\phi(\mathbf{x},\mathbf{k})=\sqrt{\frac{1}{2}(v_h^2\,{k}_h^2+v_v^2\,{k}_z^2)+\frac{1}{2}\sqrt{(v_h^2\,{k}_h^2+v_v^2\,k_z^2)^2-\frac{8\eta}{1+2\eta}v_h^2v_v^2\,k_h^2\,k_z^2}}\;, 
\end{equation} 
where $v_h=v \sqrt{1+2\eta}$ is the P-wave phase velocity in the symmetry plane,
and ${k}_h=\sqrt{{k}_x^2+{k}_y^2}$. 
Expression~\ref{eq:tti} is the same as the dispersion relation for VTI media \cite[]{alkhalifah1,alkhalifah2,anelliptic}.
\\


\subsection{Tilted Orthorhombic Anisotropy}

Tectonic movement of the crust may rotate the
rocks and tilt the plane containing the vertical cracks,
causing a tilted anisotropy.
In the case of tilted orthorhombic media,
$k_x$, $k_y$, and $k_z$ need to be replaced 
by $\hat{k}_x$, $\hat{k}_y$, and $\hat{k}_z$, 
which are spatial wavenumbers evaluated in a rotated coordinate system aligned with the vectors normal to the orthorhombic symmetry planes:

\begin{equation}
\label{eq:wavenumber}
\begin{array}{*{20}c}
\hat{k}_x=k_x\cos{\phi}+k_y\sin{\phi}\;\\ 
\hat{k}_y=-k_x\sin{\phi}\cos{\theta}+k_y\cos{\phi}\cos{\theta}+k_z\sin{\theta}\;\\ 
\hat{k}_z=k_x\sin{\phi}\sin{\theta}-k_y\cos{\phi}\sin{\theta}+k_z\cos{\theta}\;,\\ 
 \end{array}
\end{equation}
where $\theta$ is the dip angle measured with respect to vertical and 
$\phi$ is the azimuth angle, which is the angle between the original X-coordinate 
and the rotated one. 
The original vertical axis has the direction of $\left\{\sin\theta\sin\phi,-\sin\theta\cos\phi,\cos\theta\right\}$.
For a more general rotation, 
one needs three angles to describe the transformation \cite[]{zhangor}.
\\



\section{Lowrank Approximation}

For orthorhombic media, the mixed-domain phase operator, $\phi$, is given by equation~\ref{eq:root}.
Considering inhomogeneous media, we choose lowrank approximation \cite[]{ying,lr} to implement the mixed-domain operator.\\

\cite{ying,lr} showed that mixed-domain matrix 
$W(\mathbf{x},\mathbf{k})=\cos(\phi(\mathbf{x},\mathbf{k})\Delta t)$, which appears in wavefield extrapolation, 
can be decomposed using a separable representation:
\begin{equation}
  \label{eq:lra}
  W(\mathbf{x},\mathbf{k}) \approx \sum\limits_{m=1}^M \sum\limits_{n=1}^N W(\mathbf{x},\mathbf{k}_m) a_{mn} W(\mathbf{x}_n,\mathbf{k}).
\end{equation}
$W(\mathbf{x},\mathbf{k}_m)$ is a submatrix of $W(\mathbf{x},\mathbf{k})$ that consists of a few columns associated with ${\mathbf{k}_m}$,
$W(\mathbf{x}_n,\mathbf{k})$ is another submatrix that contains some rows associated with ${\mathbf{x}_n}$,
and $a_{mn}$ stands for the coefficients.
The construction of the separated form~\ref{eq:lra} follows the method of \cite{eng2009}.
The main observation is that the columns of $W(\mathbf{x},\mathbf{k}_m)$ are able to span the column space of the original matrix and that the rows of $W(\mathbf{x}_n,\mathbf{k})$ can span the row space as well as possible.\\

In the case of smooth models, the mixed-domain operator can be decomposed by a low-rank approximation.  In models with serious roughness and randomness, the time step may be restricted to small values or otherwise; the rank will end up high. As a result, the computational cost maybe high.

To perform a linear-time lowrank decompositon as proposed by \cite{lr},
we first need to restrict the mixed-domain $\mathbf{W}$ to $n$ randomly selected rows.
In practice, $n$ can be scaled as $O(r \log N_x)$ and $r$ is the numerical rank of $\mathbf{W}$. 
Then, we perform pivoted QR algorithm \cite[]{golub} to find the corresponding columns for 
$W(\mathbf{x},\mathbf{k}_m)$.
To find the rows for $W(\mathbf{x}_n,\mathbf{k})$, we apply the pivoted QR algorithm to $\mathbf{W}^*$.\\ 

Representation~\ref{eq:lra} speeds up the computation of
$p(\mathbf{x},t+\Delta t)$ because 
\begin{eqnarray}
\nonumber
  p(\mathbf{x},t+\Delta t) + p(\mathbf{x},t-\Delta t)  =  2 \int e^{-i \mathbf{x} \cdot \mathbf{k}} W(\mathbf{x},\mathbf{k}) \hat{p}(\mathbf{k},t) d \mathbf{k}
\\
  \approx  2 \sum\limits_{m=1}^M W(\mathbf{x},\mathbf{k}_m) \left( \sum\limits_{n=1}^N a_{mn} \left(\int e^{-i \mathbf{x} \cdot \mathbf{k}} W(\mathbf{x}_n,\mathbf{k}) \hat{p}(\mathbf{k},t) d\mathbf{k} \right) \right).
\label{eq:fftwave}
\end{eqnarray}
Evaluation of the last formula requires $N$ inverse FFTs. 
Correspondingly, with lowrank approximation, the cost can be reduced to $O(NN_x\log N_x)$, 
where $N_x$ is the model size and $N$ is a small number, related to the rank of the above decomposition 
and it is automatically calculated for some given error level ( $10^{-5}$ ) with a pre-determined $\Delta t$.  

\inputdir{tiltn}
Figure~\ref{fig:velxfig}-\ref{fig:velzfig} shows an orthorhombic model with smoothly varying velocity -- 
$v_1$: 1500--3088 m/s, $v_2$: 1500--3686 m/s, $v_v$: 1500--3474 m/s, $\eta_1=0.3$, $\eta_2=0.1$, and $\gamma=1.03$.
The time step $\Delta t=4 ms$.
Figure~\ref{fig:errfig1} display error of lowrank decomposition for $\cos(\phi \Delta t)$ at the location (-1.925 km, -1.925 km, -1.925 km) with relatively high velocity values, $v_1=2.257$ km/s,  $v_2=2.534$ km/s, $v_v=2.438$ km/s. 
One can find the error level is around $10^{-5}$.
Figure~\ref{fig:errfig3} display error of lowrank decomposition for $\cos(\phi \Delta t)$ at the location (0.575 km, 0.575 km, 0.575 km) with relatively low velocity values, $v_1=1.544$ km/s,  $v_2=1.561$ km/s, $v_v=1.554$ km/s. 
One can find the error is also well controlled.
\multiplot{3}{velxfig,velyfig,velzfig}{width=0.45\textwidth}{An orthorhombic model with smoothly varying velocity: (a) $v_1$: 1500--3088 m/s; (b) $v_2$: 1500--3686 m/s; (c) $v_v$: 1500--3474 m/s.}
\plot{errfig1}{width=0.8\textwidth}{Error plot for the lowrank approximation for $\cos(\phi \Delta t)$ at the location (-1.925 km, -1.925 km, -1.925 km) with relatively high velocity values, $v_1=2.257$ km/s,  $v_2=2.534$ km/s, $v_v=2.438$ km/s.}
\plot{errfig3}{width=0.8\textwidth}{Error plot for the lowrank approximation for $\cos(\phi \Delta t)$ at the location (0.575 km, 0.575 km, 0.575 km) with relatively low velocity values, $v_1=1.544$ km/s,  $v_2=1.561$ km/s, $v_v=1.554$ km/s.}

We propose using the above lowrank approximation algorithm to handle mixed-domain operator
$\phi$ in equation~\ref{eq:root} for wave extrapolation in orthorhombic media.\\



%\section{Phase Velocity}
\section{Numerical Examples}


Figure~\ref{fig:wavexy}--\ref{fig:wavexz} shows wavefield snapshots (depth, inline, and crossline) in a vertical orthorhombic medium with constant parameters: $v_v=2 km/s$, $v_1=2.1 km/s$, $v_2=2.05 km/s$, $\eta_1=0.3$, $\eta_2=0.1$, and $\gamma=1$.
The time-step size is 1 ms and the space grid sizes in three directions are all 25 m.
As the model is homogeneous, the rank is 1 for the lowrank decomposition.
The depth slice is anelliptical, whereas the inline and crossline display different diamond shapes, indicating different VTI properties.   
In Figures~\ref{fig:wavexy}--\ref{fig:wavexz}, red dashed lines are calculated using ray tracing. 
Note that the red dashed lines match the wavefront from the lowrank method very well.\\ 

\inputdir{orth}
\multiplot{3}{wavexy,waveyz,wavexz}{width=0.4\textwidth}{Three slices of the wavefield snapshot based on the dispersion relation~\ref{eq:root} at 1 second in a vertical orthorhombic medium: (a) Depth Slice; (b) Inline Slice; (c) Crossline Slice. Also plotted are red curves
   representing the wavefront at that time calculated using raytracing.}


\inputdir{test}

To show that the lowrank approximation method can handle rough velocity
models, we use a two-layer velocity model with high velocity contrast.
The first layer has lower velocity parameters: $v_v=1.5 km/s$, $v_1=1.6 km/s$, $v_2=1.7 km/s$, 
while the values in the other layer are much higher: $v_v=3.5 km/s$, $v_1=4.1 km/s$, $v_2=4.2 km/s$. 
And we use the same anisotropic parameters for both layers: $\eta_1=0.3$, $\eta_2=0.1$, and $\gamma=1$.
For this test, we use a time step size of 1 ms and a space grid size of 25 m.
The rank is 2 calculated by the lowrank decomposition within an error level of $10^{-5}$.
Figure~\ref{fig:wavexyt} displays the depth slice above the reflector at 0.6 second. 
Note the snapshot shows the reflection from the velocity contrast. 
Figure~\ref{fig:waveyzt} and \ref{fig:wavexzt} show the inline and crossline slices, which indicate strong anisotropy in the medium.
\multiplot{3}{wavexyt,waveyzt,wavexzt}{width=0.4\textwidth}{Three slices of the wavefield snapshot by the dispersion relation~\ref{eq:root} at 0.6 second in a 2-layer vertical orthorhombic model (high velocity contrast): (a) Depth Slice; (b) Inline Slice; (c) Crossline Slice. }

\inputdir{tiltn}
\multiplot{3}{snapxy4,snapyz4,snapxz4}{width=0.4\textwidth}{Wavefield snapshots based on the dispersion relation~\ref{eq:root} in an rotated and tilted orthorhombic medium ($\theta=\phi=45\,^{\circ}$) with variable velocity shown in Figure~\ref{fig:velxfig}-\ref{fig:velzfig}: (a) Depth Slice; (b) Inline Slice; (c) Crossline Slice}

Our next example is wavefield snapshots in an orthorhombic model with smoothly varying velocity, shown in Figure~\ref{fig:velxfig}-\ref{fig:velzfig}: 
$v_1$: 1500--3088 m/s, $v_2$: 1500--3686 m/s, $v_v$: 1500--3474 m/s, $\eta_1=0.3$, $\eta_2=0.1$, and $\gamma=1.03$.
The time-step size is 4 ms.
We also rotate the model ($\theta=\phi=45\,^{\circ}$).
Figure~\ref{fig:snapxy4}--\ref{fig:snapxz4} shows corresponding wavefield snapshots by the dispersion relation~\ref{eq:root} in depth, inline, and crossline slices through the central source location.
The inline section (Figure~\ref{fig:snapyz4}) displays the strongest anisotropic property, because $\eta_1$ is as large as $0.3$.
Note that the snapshots are free of dispersion and 
that there is no coupling of qSV and qP waves in the middle.
Lowrank parameters were $M=7$ and $N=7$.
Therefore, the cost is 7 FFTs at each time step.\\
\\
Table~\ref{tbl:lrn} displays rank $N$ required for maintaining an error level of $10^{-5}$
with different time step size $\Delta t$.
From table~\ref{tbl:lrn}, one could find for this smooth model, 
$\Delta t=4$ ms and $N=7$ is the optimal choice for cost consideration.
For models with very wide range of parameters and 
rather complicated structures, 
the resulting rank may be high, 
because more space locations and wavenumbers are required to properly represent the original mixed-domain matrix. 
In order to reduce the computational cost,
one may consider Lowrank Finite differences proposed by \cite{songxl}, 
which is a space-domain finite-difference scheme in which the coefficients of the Laplacian finite-difference stencil is derived from the lowrank approximation.
\tabl{lrn}{Rank $N$ calculated from the lowrank approximation of the propagation matrix for a 2D smooth orthorhombic model with different time step size $\Delta t$ for a given error level $10^{-5}$.}
{
    \begin{center}
     \begin{tabular}{c|c|c|c|c|c|c}
      \hline
      \hline
      $\Delta t$ (ms) & 0.5 & 1 & 2 & 3 & 4 & 5 \\
      \hline
      $Rank N$             & 5   & 5 & 7 & 7 & 7 & 12 \\
      \hline
      \hline
     \end{tabular}
    \end{center}
}




\section{Conclusions}

We derive and adopt a dispersion relation 
for acoustic orthorhombic media so as to model seismic wavefields in such media.
To handle the space-wavenumber mixed-domain operator, 
we apply the lowrank approximation to reduce computational cost.
Numerical experiments show that the proposed wavefield extrapolator is accurate.
There is no coupling of qSV and qP in the wavefield snapshots 
because we use the dispersion relation. 
In addition, our approach yields practically 
dispersion-free wavefields, and 
it is also free of stability limitations on media parameters.


\section{Acknowledgments}
We thank Sergey Fomel and Lexing Ying for their help with the lowrank algorithm, and Mirko van der Baan, Richard Bale, Paul J. Fowler, Samuel Gray, and one anonymous reviewer for helpful reviews. 
We also thank KAUST (King Abdullah University of Science and Technology) and TCCS (Texas Consortium for Computational Seismology) for financial support.
This publication is authorized by the Director, Bureau of Economic
Geology, The University of Texas at Austin.
\bibliographystyle{seg}
\bibliography{or,SEP2}

%\end{document}
