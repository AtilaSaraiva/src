\published{Geophysics, 74, no. 3, V43-V48, (2009)}
\title{Stacking seismic data using local correlation}

\renewcommand{\thefootnote}{\fnsymbol{footnote}}

\ms{GEO-2009}

\address{
\footnotemark[1] State Key Laboratory of Petroleum Resources and Prospecting\\
China University of Petroleum\\
Beijing, China \\
\footnotemark[2] Bureau of Economic Geology,\\
John A. and Katherine G. Jackson School of Geosciences \\
The University of Texas at Austin \\
University Station, Box X \\
Austin, TX, USA, 78713-8924 \\
\footnotemark[3] Institute for Geophysics,\\
John A. and Katherine G. Jackson School of Geosciences \\
The University of Texas at Austin \\
Austin, TX, USA, 78713-8924
}

\author{Guochang Liu\footnotemark[1]\footnotemark[2], Sergey Fomel\footnotemark[2], Long Jin\footnotemark[3], Xiaohong Chen\footnotemark[1]}

\footer{GEO-2009}
\lefthead{Liu etc.}
\righthead{Stacking using local correlation}
\maketitle

\begin{abstract}

Stacking plays an important role in improving signal-to-noise ratio and imaging 
quality of seismic data. However, for low-fold-coverage seismic profiles, the 
result of conventional stacking is not always satisfactory. To address this 
problem, we have developed a method of stacking in which we use local 
correlation as a weight for stacking common-midpoint gathers after NMO 
processing or common-image-point gathers after prestack migration. Application 
of the method to synthetic and field data showed that stacking using local
correlation can be more effective in suppressing random noise and artifacts 
than other stacking methods.

\end{abstract}

\section{Introduction}

Stacking as one of the three crucial techniques (deconvolution,
stacking, and migration) plays an important role in improving 
signal-to-noise ratio (S/R) in seismic data processing \cite[]{Yilmaz01}.
Conventional stacking, which is performed by averaging an NMO-corrected
data set or migrated data set, is optimal only when noise
components in all traces are uncorrelated, normally distributed, stationary,
and of equal magnitude \cite[]{Mayne62,Neelamani06}. Therefore, different 
stacking technologies have been proposed,
along with improvements in optimizing weights of seismic
traces.

\cite{Robinson70} proposes using an S/N-based weighted stack to
further minimize noise. Using cross-correlation of seismic traces and
normalized cross-correlation processing, \cite{Chang96} proposes
preserved frequency stacking. \cite{Schoenberger96} proposes optimum
weighted stack for multiple suppression, with weight determined
by solving a set of optimization equations. \cite{Neelamani06} 
propose a stack-and-denoise method called SAD, which exploits
the structure of seismic signals to obtain an enhanced stack.
\cite{Zhang06} present a high-order correlative weighted
stacking technique on the basis of wavelet transformation and high-order
statistics. By estimating the probability distribution of noise,
\cite{Trickett07} applies a maximum-likelihood estimator to stacking.
To eliminate artifacts in angle-domain common-image gathers
(CIGs) caused by sparsely sampled wavefields, \cite{Tang07} presents
a selective stacking approach that applies local smoothing of
envelope function to achieve the weighting function. \cite{Rashed08}
proposes smart stacking, which is based on optimizing seismic amplitudes
of the stacked signal by excluding harmful samples from the
stack and applying larger weight to the central part of the sample
population.

The global correlation coefficient can measure the similarity of
two signals, but it is not a local attribute. Not only does the sliding-window
global-correlation approach need many parameters to be
specified, but this approach cannot smoothly characterize thin layers
well. \cite{Fomel07a} uses shaping regularization, which controls locality
and smoothness to define local correlation \cite[]{Fomel07b}.
Local correlation is applied to multicomponent seismic image registration
\cite[]{Fomel05,Fomel07b} and time-lapse image registration
\cite[]{Fomel07}.

In this paper,we present a new stacking method using local correlation.
This method applies time-dependent smooth weights (which
are taken as local correlation coefficients between reference traces
and prestack traces), stacks the common-midpoint (CMP) gather,
and effectively discards parts of the data that least contribute to
stacked reflection signals. Using synthetic and field data examples,
we show that, compared with other stacking methods, this method
can greatly improve the S/N and suppress artifacts.

\section{Methodology}

 \subsection{Review of local correlation}

The global uncentered correlation coefficient between two discrete
signals $\mathbf{a}_i$ and $\mathbf{b}_i$ can be defined as the functional
      \begin{equation}
          \gamma = \frac{\displaystyle\sum_{i=1}^{N}\mathbf{a}_i \mathbf{b}_i}{\sqrt{\displaystyle\sum_{i=1}^{N}\mathbf{a}^2_i \sum_{i=1}^{N}\mathbf{b}^2_i}}\;,
        \label{eq:eq1}
      \end{equation}
where N is the length of a signal. The global correlation in equation~\ref{eq:eq1}
supplies only one number for the whole signal. For measuring the
similarity between two signals locally, one can define the sliding-window
correlation coefficient
      \begin{equation}
          \gamma_w(t) = \frac{\displaystyle\sum_{i=t-w/2}^{t+w/2} \mathbf{a}_i \mathbf{b}_i}{\sqrt{\displaystyle\sum_{i=t-w/2}^{t+w/2} \mathbf{a}_i^2 \displaystyle\sum_{i=t-w/2}^{t+w/2} \mathbf{b}_i^2 }}\;,
        \label{eq:eq2}
      \end{equation}
where $w$ is window length.

\cite{Fomel07b} proposes the local correlation attribute that identifies
local changes in signal similarity in a more elegant way. In a linear
algebra notation, the correlation coefficient in equation 1 can be
represented as a product of two least-squares inverses $\gamma_1$ and $\gamma_2$:
      \begin{equation}
          \gamma^2 = \gamma_1 \gamma_2\;,
        \label{eq:eq3}
      \end{equation}
      \begin{equation}
          \gamma_1 = \textrm{arg} \min_{\gamma_1}\parallel \mathbf{b}-\gamma_1 \mathbf{a} \parallel^2 = (\mathbf{a}^T\mathbf{a})^{-1}(\mathbf{a}^T\mathbf{b})\;,
        \label{eq:eq4}
      \end{equation}
      \begin{equation}
        \label{eq:eq5}
          \gamma_2 = \textrm{arg} \min_{\gamma_2}\parallel \mathbf{a}-\gamma_1 \mathbf{b} \parallel^2 = (\mathbf{b}^T\mathbf{b})^{-1}(\mathbf{b}^T\mathbf{a})\;,
      \end{equation}
where $\mathbf{a}$ and $\mathbf{b}$ are vector notions for $\mathbf{a}_i$ and 
$\mathbf{b}_i$. Let $\mathbf{A}$ and $\mathbf{B}$ be two diagonal
operators composed of the elements of a and b. Localizing
equations~\ref{eq:eq4} and \ref{eq:eq5} amounts to adding regularization to 
inversion. Using shaping regularization \cite[]{Fomel07a}, scalars $\gamma_1$ 
and $\gamma_2$ turn into vectors $\mathbf{c}_1$ and $\mathbf{c}_2$, defined as
      \begin{equation}
          \mathbf{c}_1 = [\lambda^2 \mathbf{I} + \mathbf{S}(\mathbf{A}^T \mathbf{A} - \lambda^2 \mathbf{I})]^{-1}\mathbf{S}\mathbf{A}^T\mathbf{b}\;,
        \label{eq:eq6}
      \end{equation}
      \begin{equation}
          \mathbf{c}_2 = [\lambda^2 \mathbf{I} + \mathbf{S}(\mathbf{B}^T \mathbf{B} - \lambda^2 \mathbf{I})]^{-1}\mathbf{S}\mathbf{B}^T\mathbf{a}\;,
        \label{eq:eq7}
      \end{equation}
where $\lambda$ scaling controls relative scaling of operators $\mathbf{A}$ 
and $\mathbf{B}$ and
where $\mathbf{S}$ is a shaping operator such as Gaussian smoothing with an
adjustable radius. The component-wise product of vectors $\mathbf{c}_1$ and 
$\mathbf{c}_2$
defines the local correlation measure. Local correlation is a measure
of the similarity between two signals.
An iterative, conjugate-gradient inversion for computing the inverse
operators can be applied in equations~\ref{eq:eq6} and \ref{eq:eq7}. 
Interestingly, the output of the first iteration is equivalent to the 
algorithm of fast local
cross-correlation proposed by \cite{Hale06}.

\subsection{Stacking using local correlation}

The problem of combining a collection of seismic traces into a
single trace is commonly referred to as stacking in seismic data processing.
This process is used to attenuate random noise and simultaneously
amplify the coherent signal in the gather. Typically, the desired
stacked trace is estimated by averaging traces from the CMP
gather \cite[]{Mayne62}:
      \begin{equation}
          \bar{a}_j(t) = \frac{1}{N}\displaystyle\sum_{i=1}^{N} a_{i,j}(t), \qquad j=(1,2,3,\dots,M) \;,
        \label{eq:eq8}
      \end{equation}
where $N$ is the fold of the stack and $a_{i,j}(t)$ is the sample value on 
trace $i$ from the $j$th CMP gather at two-way time $t$. Such a technique 
provides the optimal unbiased estimate of $\bar{a_j}(t)$. \cite{Robinson70} 
proposes weighted stacking of seismic data:
      \begin{equation}
          \bar{a}_j(t) = \frac{1}{\displaystyle\sum_{i=1}^{N} w_{i,j}}\displaystyle\sum_{i=1}^{N} w_{i,j} \cdot a_{i,j}(t), \qquad j=(1,2,3,\dots,M) \;,
        \label{eq:eq9}
      \end{equation}
where $w_{i,j}$ denotes the weight of the $i$th trace from the $j$th CMP gather,
which is determined by noise variances $w_{i,j}=1/\sigma^2_{i,j}$. However, it
is difficult to estimate noise variances reliably in practice. 
\cite{Neelamani06} use an iterative algorithm called ``leave me out'' (LMO) to 
estimate noise variances from data. The desired signal is assumed to be flat 
with constant amplitude across all the traces within a gather in the LMO method.

For using time-dependent smooth weights in the stacking process, we choose the 
local correlation coefficient from the previous section as weights to stack 
seismic data. We find that local correlation better characterizes local 
similarity between prestack and reference traces than the sliding-window 
method.

Consider the two noisy signals with Gaussian random noise but different noise 
levels in Figure~\ref{fig:ricker,ref,signal1,signal2,wcorr,simi}c and 
\ref{fig:ricker,ref,signal1,signal2,wcorr,simi}d. The signals are derived from 
adding noise on convolution of the Ricker wavelet (Figure
\ref{fig:ricker,ref,signal1,signal2,wcorr,simi}a)
with synthetic reflectivity 
(Figure~\ref{fig:ricker,ref,signal1,signal2,wcorr,simi}b). The distribution of 
noise in 
(Figure~\ref{fig:ricker,ref,signal1,signal2,wcorr,simi}c) is 
$N(\mu,\sigma)=N(0,10^{-6})$, where $\mu$ and $\sigma$ 
are mean and variance of noise, respectively. The distribution of noise in 
(Figure~\ref{fig:ricker,ref,signal1,signal2,wcorr,simi}d) is 
$N(\mu,\sigma)=N(0,0.07)$. The sliding-window 
correlation and local correlation
between Figure~\ref{fig:ricker,ref,signal1,signal2,wcorr,simi}c and 
Figure~\ref{fig:ricker,ref,signal1,signal2,wcorr,simi}d are shown in 
Figure~\ref{fig:ricker,ref,signal1,signal2,wcorr,simi}e and 
Figure~\ref{fig:ricker,ref,signal1,signal2,wcorr,simi}f, respectively.
Note that local correlation coefficients 
(Figure~\ref{fig:ricker,ref,signal1,signal2,wcorr,simi}f) are smooth and
better distinguish the thin layer, represented by the first two reflectivities
in Figure~\ref{fig:ricker,ref,signal1,signal2,wcorr,simi}b. In application to 
stacking, the prestack trace is
analogous to Figure~\ref{fig:ricker,ref,signal1,signal2,wcorr,simi}d with 
larger variance noise, and the reference trace is analogous to 
Figure~\ref{fig:ricker,ref,signal1,signal2,wcorr,simi}c with smaller variance 
noise.
\inputdir{similarity}
 \multiplot{6}{ricker,ref,signal1,signal2,wcorr,simi}{width=0.45\textwidth}{(a)
               Zero-phase Ricker wavelet. (b) Reflection coefficient. (c) Noisy
               signal with $N(\mu,\sigma)=N(0,10^{-6})$. (d) Noisy signal with 
               $N(\mu,\sigma)=N(0,0.07)$. (e) Sliding-window correlation. (f) 
               Local correlation.}

To implement seismic data stacking using local correlation, we
first apply the equal-weight stack to the NMO-corrected CMP gather
to obtain the reference trace. Then we compute the local correlation
coefficients between the reference trace and the NMO-corrected
CMP gather and apply soft thresholding \cite[]{Donoho95} to all local
correlation coefficients. Finally, we apply the weighted stack to the
CMP gather using local correlation to get the final stacked trace.
Mathematically, stacking using the local correlation approach
modifies equation~\ref{eq:eq9} to
      \begin{equation}
          \bar{a}_j(t) = \frac{1}{K_j H_j(t)}\displaystyle\sum_{j=1}^{N} w_{i,j} \cdot a_{i,j}(t), \qquad j=(1,2,3,\dots,M) \;,
        \label{eq:eq10}
      \end{equation}
      \begin{equation}
          w_{i,j}(t) = \left \{ \begin{array}{ll}
                    \eta_{i,j}(t)-\varepsilon, & \eta_{i,j}> \varepsilon\\
		    0, & \eta_{i,j}\le \varepsilon
                  \end{array} \right.\;,
        \label{eq:eq11}
      \end{equation}
where $\varepsilon$ is the threshold value, 
$K_j = \displaystyle\sum_{t=0}^{t}\sum_{i=0}^{N}w_{i,j}(t)$
is the sum of weights of the $j$th CMP gather, $H_j(t)$ is the number of 
samples with $w_{i,j}\cdot a_{i,j}(t)\ne0$ for a given two-way time, and 
$\eta_{i,j}(t)$ is the local correlation
between the $i$th prestack trace from $j$th gather and the reference
trace. The local correlation $\eta_{i,j}(t)$ can be computed using equations
\ref{eq:eq6} and \ref{eq:eq7}. The reference trace is derived from averaging 
all the traces of one CMP gather. Here we assume that the equal-weight
stacked trace is close to the desired trace. Because the weights are a 
function of two-way traveltime and offset, recovery scalar $K_j$ has the same 
value for the same CMP gather. Meanwhile, the samples with 
$w_{i,j}(t)\cdot a_{i,j}\ne0$ at a given two-way time are assumed to be full 
noise or null value such as muting parts; we therefore use $H_j(t)$ to
scale the stacked trace.

Changes occurring between equation~\ref{eq:eq9} and equations~\ref{eq:eq10} 
and~\ref{eq:eq11} result from time-dependent smooth weights for the stack and 
application of thresholding to the weights. All local correlation coefficients 
below a specified threshold are discarded, and the rest, with values above the 
threshold, are included. We thus stack only those parts of prestack data whose 
similarity to the reference trace is comparatively large. 
Equations~\ref{eq:eq10} and \ref{eq:eq11} implicitly estimate the noise 
variance by analyzing local cross-correlations between prestack trace and the 
reference trace. This operation can be understood as a nonlinear filter that 
enhances the coherency of events. We perform this operation for all gathers 
using this method to improve the stack profile.

When applied after angle-domain migration, normalization provided by soft 
thresholding is analogous to true-amplitude illumination compensation from the 
so-called Beylkin determinant \cite[]{Albertin99,Audebert05}. Local correlation 
normalizes the image by the number of strongly illuminated angles in 
angle-domain CIGs.

In the following, we discuss the distinctions between seismic
stacking using local correlation and other methods. Our method creates
time-dependent smooth weights without depending on sliding
windows, as compared to other weighted stacking methods such as
statistically optimal stacking \cite[]{Robinson70,Neelamani06} and 
the semblance method \cite[]{Yilmaz01}. In contrast to
smart stacking, proposed by \cite[]{Rashed08} and based on sign difference
between sample point and the alpha-trimmed mean to remove
frequency distortions, our method applies waveform similarity between
prestack trace and mean to make the stacking selective.

 \section{Examples}

To illustrate the proposed method using synthetic and field data,
we apply our approach to three examples. The first example is a simple
case involving a fivefold prestack gather (Figure~\ref{fig:compare}a) with a 
timeshifted-upward trace, which might be distortion by poor static correction.
The peak of the signal in this gather is one.We add Gaussian
random noise with distribution $N(\mu,\sigma)=N(0,0.01)$ on the five
traces. The result of an equal-weight stack is shown in 
Figure~\ref{fig:compare}c. The upside wing in Figure~\ref{fig:compare}c is 
distorted because of the first time-shift
trace. Then we use three weighted stacking methods to stack the five
traces.

Figure~\ref{fig:compare}d and Figure~\ref{fig:compare}e illustrates results 
of smart stacking \cite[]{Rashed08} and LMO-based weighted stacking in which 
the weights are computed by the LMO method \cite[]{Robinson70,Neelamani06}. 
Figure~\ref{fig:compare}f shows the result of stacking using local correlation
with weights (Figure~\ref{fig:compare}b) determined by the similarity between 
the prestack trace (Figure~\ref{fig:compare}a) and the reference 
(Figure~\ref{fig:compare}c). Because the waveform in the first trace in 
Figure~\ref{fig:compare}a is most likely 
noise or artifact, it is reasonable that the weight in the stack procedure is 
lower. Use of local correlation as weights of prestack traces lets us select
those portions, which are more similar to the reference trace to contribute
to the stack.
\inputdir{simple}
  \plot{compare}{width=0.95\textwidth}{Simple stacking test with fivefold 
                 gather. (a) Prestack gather. (b)Weights used in 
                 local-correlation weighted stacking. (c) Conventional 
                 equal-weight stacking method (S/N=8.4). (d) Smart stacking 
                 method (S/N=9.2). (d) LMO-based weighted method (S/N=10.2). 
                 (f) Local-correlation weighted stacking (S/N=13.5).}

Comparing the three methods, one can find that smart stacking
and LMO-based weighted stacking can remove upside wing distortion
cleanly, but stacking using local correlation removes more random
noise than the other two methods and meanwhile corrects upside
wing distortion.

To judge the effect of denoising quantitatively between different
methods, we apply equal-weight stacking on the last four traces
without any noise to get the exact desired stacked trace $d_j(t)$, which
can be regarded as a signal trace. The S/N of the $j$th CMP can therefore
be estimated as
      \begin{equation}
          \textrm{S/N}_j = 10 \log_{10} \left(\frac{\displaystyle\sum_t[d^2_j(t)]}{\displaystyle\sum_t[d_j(t)-\bar{a}_j(t)]^2} \right) \;,
        \label{eq:eq12}
      \end{equation}
where $\bar{a}_j(t)$ is the stacked trace from different
stacking methods. In the first simple example
Figure~\ref{fig:compare}, the S/N of equal-weight stacking is
8.4 dB and the other three weighted methods are,
respectively, 9.2, 10.2, and 13.5 dB. Stacking using
local correlation can improve S/N greatly.

The second example is a 2D synthetic model
that includes four reflectors. Synthetic data are
generated with Kirchhoff modeling. The peak of
the data set is one and Gaussian random noise
with distribution $N(\mu,\sigma)=N(0,0.05)$ is added.
We show the results of stacking one CMP gather
(Figure~\ref{fig:onestack1}a) by three methods in 
Figure~\ref{fig:onestack1}c-e.
Compared to other methods, our method is the
most effective in denoising.
\inputdir{flat4}
  \plot{onestack1}{width=0.95\textwidth}{(a) One CMP gather from synthetic data
                   set. (b) NMO-corrected gather. (c) Result of conventional 
                   equal-weight stacking. (d) Result of LMO-based weighted 
                   stacking. (e) Result of local-correlation weighted 
                   stacking.}

The stacked profile of all CMP gathers is
shown in Figure~\ref{fig:stackss}. We use equation~\ref{eq:eq12} to compute
the S/N of the stacked profile (Figure~\ref{fig:stackss}). The
S/Ns of three methods are 7.1, 9.6, 10.9 dB, respectively.
Noise is attenuated more effectively in
the stacking result using local correlation (Figure~\ref{fig:stackss}c).
  \plot{stackss}{width=0.7\textwidth}{Comparison among three stacking methods 
                 including all synthetic CMP gathers. (a) Conventional 
                 equal-weight stacking (S/N=7.1). (b) LMO-based weighted
                 stacking (S/N=9.6). (c) Local-correlation weighted stacking 
                 (S/N=10.9).}

The third example involves a historic 2D line
from the Gulf of Mexico \cite[]{Claerbout05}. The
stacked sections, using three different methods,
are shown in Figure~\ref{fig:field}. Figure~\ref{fig:weight} shows the local
correlation cube between prestack and reference
traces. Similar cubes have been used in multicomponent
seismic image registration \cite[]{Fomel05,Fomel07b} and time-lapse image
registration \cite[]{Fomel07}. For synthetic
data, the exact desired stacked section can be calculated
by stacking prestack traces without any
noise. But for field data, the S/N is difficult to estimate
using equation~\ref{eq:eq12}. We therefore use singular
value decomposition (SVD) \cite{Andrews76} to evaluate different stacking
methods. The SVD of stacked section matrix gives
      \begin{equation}
          \mathbf{M} = \mathbf{U} \Sigma \mathbf{V}^T\;.
        \label{eq:eq13}
      \end{equation}

The diagonal elements $\sigma_r$ of $\Sigma$ are the singular values of 
$\mathbf{M}$. The S/N can be estimated as \cite[]{Freire88,Peterson88,Grion98}
      \begin{equation}
          \textrm{S/N} = 10 \log_{10} \left(\frac{\sigma^2_1 - \frac{1}{R-1} \displaystyle\sum_{r=2}^{R}\sigma^2_r}{\frac{1}{R-1}\displaystyle\sum_{r=2}^{R}\sigma^2_r} \right) \;,
        \label{eq:eq14}
      \end{equation}

where $R$ is the number of all singular values. The S/Ns of stacked
sections resulting from three stacking methods are, respectively,
27.4, 29.2, and 33.9 dB. Comparing Figure~\ref{fig:field}a-c, we can find also
that random noise is attenuated and coherent reflections are enhanced
better using local correlation (e.g., 0.5--1.5-s range).
\inputdir{bei}
  \plot{field}{width=0.95\textwidth}{Results of (a) conventional equal-weight 
               stacking (S/N=27.4), (b) LMO-based weighted stacking (S/N=29.2) 
               and (c) local-correlation weighted stacking (S/N=33.9).}
  \plot{weight}{width=0.7\textwidth}{Local correlation cube of the field-data 
               example.}

 \section{Conclusion}

We have developed a new method of stacking NMO-corrected or migrated seismic 
data using local correlation. We substitute local correlation for the weight 
value in statistically weighted stacking and then use soft thresholding
to make the stacking selective. Because weights are derived from the input 
data, our method can be regarded as a nonlinear filter. Synthetic and
field data examples confirm that our approach can be significantly more 
effective than other weighted stacking methods in improving S/N and suppressing
distortions resulting from prestack processing. Seismic stacking using local 
correlation can give a poor result if the quality of the reference trace is 
very poor. Because the coherency enhancement from local correlation is not 
based on physics, one should use this approach with caution when aiming to 
preserve physically meaningful amplitudes.


\section{Acknowledgments}

We would like to thank Yang Liu and Jingye Li for inspiring discussions and 
three reviewers for helpful suggestions. G. Liu would like to thank the China 
Scholarship Council (grant 2007U44003) for partial support of this work. X.
Chen's research was supported partially by the National Basic Research Program 
of China (973 program, grant 2007CB209606) and by the National High Technology 
Research and Development Program of China (863 program, grant 2006AA09A102-09).
Publication was authorized by the Director, Bureau of Economic Geology, The 
University of Texas at Austin.

\bibliographystyle{seg}
\bibliography{paper}

