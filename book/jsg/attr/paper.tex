\title{Local seismic attributes}

\author{Sergey Fomel}

\ms{LETTER GEO-2006-0339}

\lefthead{Fomel}
\righthead{Local seismic attributes}

\address{Bureau of Economic Geology, \\
John A. and Katherine G. Jackson School of Geosciences \\
The University of Texas at Austin \\
University Station, Box X \\
Austin, TX 78713-8972}

\maketitle

\begin{abstract}
  Local seismic attributes measure seismic signal characteristics not
  instantaneously at each signal point and not globally across a data
  window but locally in the neighborhood of each point. I define local
  attributes with the help of regularized inversion and demonstrate
  their usefulness for measuring local frequencies of seismic signals
  and local similarity between different datasets\old{, and local
    focusing of a seismic image}. I use shaping regularization for
  controlling the locality and smoothness of local attributes. A
  multicomponent image registration example from a nine-component land
  survey illustrates practical applications of local attributes for
  measuring \old{frequency and phase} differences between registered
  images.
\end{abstract}

\section{Introduction}

\emph{Seismic attribute} is defined by \cite{EDG00-00-03840384} as a
``measurement derived from seismic data''. Such a broad definition
allows for many uses and abuses of the term. Countless attributes have
been introduced in the practice of seismic exploration,
\cite[]{TLE15-10-10901090,TLE16-05-04450456}, 
which led \cite{TLE21-10-09940994} to talk about ``attribute
explosion''.  Many of these attributes play an exceptionally important
role in interpreting and analyzing seismic data \cite[]{marfurt}.

In this paper, I consider \old{three} \new{two} particular attribute applications:
\begin{enumerate}
\item \emph{Measuring local frequency content in a seismic image}
  is important both for studying the phenomenon of seismic wave
  attenuation and for processing of attenuated signals.
\item \emph{Measuring local similarity between two seismic images} is
  useful for seismic monitoring, registration of multicomponent data,
  and analysis of velocities and amplitudes.
\old{\emph{Measuring local focusing of a seismic image} is needed for
  phase correction and diffraction velocity analysis.}
\end{enumerate}

Some of the best known seismic attributes are instantaneous attributes
such as instantaneous phase or instantaneous dip
\cite[]{GEO44-06-10411063,GEO57-11-15201524,GEO58-03-04190428}. 
Such
attributes measure seismic frequency characteristics as being attached
instantaneously to each signal point. This measure is notoriously
noisy and may lead to unphysical values such as negative frequencies
\cite[]{TLE10-07-00260032}.

In this paper, I introduce a concept of \emph{local attributes}. Local
attributes measure signal characteristics not instantaneously at each
data point but in a local neighborhood around the point. According to
the Fourier uncertainty principle, frequency is essentially an
uncertain characteristic when applied to a local region in the time
domain. Therefore, local frequency is more physically meaningful than
instantaneous frequency. The idea of locality extends from local
frequency to other attributes, such as the correlation coefficient
between two different datasets, that are conventionally evaluated in
sliding windows.

The paper starts with reviewing the definition of instantaneous
frequency. I modify this definition to that of local frequency by
recognizing it as a form of regularized inversion and by changing
regularization to constrain the continuity and smoothness of the
output. The same idea is extended next to define local correlation \old{and
local focusing measures}. I illustrate a practical application of local
attributes using an example from multicomponent seismic image
registration in a nine-component land survey.

%I start the paper with a review of the instantenous attribute
%construction.  Next, I show how the mathematical definition and
%numerical implementation of local attributes follow from a
%generalization of the instantaneous formulation. The key idea is to
%define the instantaneous attribute construction as an inverse problem
%and to reformulate it by applying the method of
%regularization. The idea is extended... and applied...

\section{Measuring local \old{frequency content} \new{frequencies}}

\old{I start by reviewing the definition of instantaneous
frequency. This definition will get modified and extended in the later
discussion.}

\subsection{Definition of instantaneous frequency}

Let $f(t)$ represent seismic trace as a function of time $t$. The
corresponding complex trace $c(t)$ is defined as
\begin{equation}
  \label{eq:complex}
  c(t) = f(t) + i\,h(t)\;,
\end{equation}
where $h(t)$ is the Hilbert transform of the real trace $f(t)$. One
can also represent the complex trace in terms of the envelope $A(t)$
and the instantaneous phase $\phi(t)$, as follows:
\begin{equation}
  \label{eq:polar}
  c(t) = A(t)\,e^{i\,\phi(t)}\;.
\end{equation}
By definition, instantaneous frequency is the time derivative of the
instantaneous phase \cite[]{GEO44-06-10411063}
\begin{equation}
  \label{eq:instant}
  \omega(t) = \phi'(t) = \mathit{Im}\left[\frac{c'(t)}{c(t)}\right]
  = \frac{f(t)\,h'(t) - f'(t)\,h(t)}{f^2(t) + h^2(t)}\;.
\end{equation}
\old{In terms of the input seismic trace and its Hilbert transform, the
instantaneous frequency can be defined as \\
\parbox{\textwidth}{
\begin{equation}
  \label{eq:instant2}
  \omega(t) = \frac{f(t)\,h'(t) - f'(t)\,h(t)}{f^2(t) + h^2(t)}\;.
\end{equation}}}
Different numerical realizations of equation~\ref{eq:instant} produce
slightly different algorithms \cite[]{GEO57-11-15201524}.

Note that the definition of instantaneous frequency calls for division
of two signals. In a linear algebra notation,
\begin{equation}
  \label{eq:vecif}
  \mathbf{w} = \mathbf{D}^{-1}\,\mathbf{n}\;,
\end{equation}
where $\mathbf{w}$ represents the vector of instantaneous frequencies
$\omega(t)$, $\mathbf{n}$ represents the numerator in
equation~\ref{eq:instant}, and $\mathbf{D}$ is a diagonal operator
made from the denominator of equation~\ref{eq:instant}. A recipe for
avoiding division by zero is adding a small constant~$\epsilon$ to the
denominator \cite[]{nowack}. Consequently, equation~\ref{eq:vecif}
transforms to
\begin{equation}
  \label{eq:vecife}
  \mathbf{w}_{inst} = \left(\mathbf{D}+\epsilon\,\mathbf{I}\right)^{-1}\,\mathbf{n}\;,
\end{equation}
where $\mathbf{I}$ stands for the identity operator. Stabilization by
$\epsilon$ does not, however, prevent instantaneous frequency from
being a noisy and unstable attribute. \new{The main reason for that is
the extreme locality of the instantaneous frequency measurement, governed
only by the phase shift between the signal and its Hilbert transform.}

Figure~\ref{fig:sign} shows three test signals for comparing frequency
attributes. The first signal is a synthetic chirp function with
linearly varying frequency. Instantaneous frequency shown in
Figure~\ref{fig:inst} correctly estimates the modeled frequency trend.
The second signal is a piece of a synthetic seismic trace obtained by
convolving a 40-Hz Ricker wavelet with synthetic reflectivity. The
instantaneous frequency (Figure~\ref{fig:inst}b) shows many variations
and appears to contain detailed information. However, this information
is useless for characterizing the dominant frequency content of the
data, which remains unchanged due to stationarity of the seismic
wavelet. The last test example (Figure~\ref{fig:sign}c) is a real
trace extracted from a seismic image.  The instantaneous frequency
(Figure~\ref{fig:inst}c) appears noisy and even contains physically
unreasonable negative values. Similar behavior was described by
\cite{TLE10-07-00260032}.

\inputdir{attr}

\plot{sign}{width=0.8\columnwidth}{Test signals for comparing
  frequency attributes. a: Synthetic chirp signal
  with linear frequency change, b: Synthetic seismic trace from
  convolution of a synthetic reflectivity with a Ricker wavelet, c:
  real seismic trace from a marine survey.}
\plot{inst}{width=0.8\columnwidth}{Instantaneous frequency of test
  signals from Figure~\ref{fig:sign}.}
\plot{locl}{width=0.8\columnwidth}{Local frequency of test signals
  from Figure~\ref{fig:sign}.}

\subsection{Definition of local frequency}

The definition of the local frequency attribute starts by recognizing
equation~\ref{eq:vecife} as a regularized form of linear
inversion. Changing regularization from simple identity to a more
general regularization operator~$\mathbf{R}$ provides the definition
for local frequency as follows:
\begin{equation}
  \label{eq:reg}
  \mathbf{w}_{loc} = 
  \left(\mathbf{D}+\epsilon\,\mathbf{R}\right)^{-1}\,\mathbf{n}\;,
\end{equation}
The role of the regularization operator is ensuring continuity and
smoothness of the local frequency measure. A different approach to
regularization \old{is} follows from the shaping method
\cite[]{shape2}. Shaping regularization operates with a smoothing
(shaping) operator $\mathbf{S}$ by incorporating it into the inversion
scheme as follows:
\begin{equation}
  \label{eq:shp}
  \mathbf{w}_{loc} = \left[\lambda^2\,\mathbf{I} + 
    \mathbf{S}\,\left(\mathbf{D} - \lambda^2\,\mathbf{I}\right)\right]^{-1}\,
  \mathbf{S}\,\mathbf{n}\;,
\end{equation}
Scaling by $\lambda$ preserves physical dimensionality and enables
fast convergence when inversion is implemented by an iterative method.
A natural choice for $\lambda$ is the least-squares norm of
$\mathbf{D}$.

Figure~\ref{fig:locl} shows the results of measuring local frequency
in the test signals from Figure~\ref{fig:sign}. I used the shaping
regularization formulation~\ref{eq:shp} with the shaping
operator~$\mathbf{S}$ defined as a triangle smoother. The chirp signal
frequency (Figure~\ref{fig:locl}a) is correctly recovered.  The
dominant frequency of the synthetic signal (Figure~\ref{fig:locl}b) is
correctly estimated to be stationary at 40~Hz. The local frequency of
the real trace (Figure~\ref{fig:locl}c) appears to vary with time
\old{with} \new{according to} the general frequency attenuation trend.

\new{This example highlights some advantages of the local attribute
method in comparison with the sliding window approach:
\begin{itemize}
\item Only one parameter (the smoothing radius) needs to be specified
as opposed to several (window size, overlap, and taper) in the
windowing approach. The smoothing radius directly reflects the
locality of the measurement.
\item The local attribute approach continues the measurement smoothly
through the regions of absent information such as the zero amplitude
regions in the synthetic example, where the signal phase is
undefined. This effect is impossible to achieve in the windowing
approach unless the window size is always larger than the information
gaps in the signal.
\end{itemize}}

\inputdir{vecta}

\multiplot{2}{pp,ss}{width=0.45\columnwidth}{PP (a) and SS (b) images
from a nine-component land survey.}
\multiplot{2}{pi,si}{width=0.45\columnwidth}{Local
  frequency attribute of PP (a) and warped SS (b) images.}

Figure~\ref{fig:pp,ss} shows seismic images from compressional (PP)
and shear (SS) reflections obtained by processing a land
nine-component survey. Figure~\ref{fig:pi,si} shows local frequencies
measured in PP and SS images after warping the SS image into PP time.
The term ``image warping'' comes from medical imaging \cite[]{wolberg}
and refers, in this case, to squeezing the SS image to PP reflection
time to make the two images display in the same coordinate system. We
can observe a general decay of frequency with time caused by seismic
attenuation. After mapping (squeezing) to PP time, the SS image
frequency appears higher in the shallow part of the image because of a
relatively low S-wave velocity but lower in the deeper part of the
image because of the apparently stronger attenuation of shear waves. A
low-frequency anomaly in the PP image might be indicative of gas
presence. Identifying and balancing non-stationary frequency
variations of multicomponent images is an essential part of the
multistep image registration technique
\cite[]{SEG-2003-07810784,warp}.

\section{Measuring local similarity}

\old{Suppose that our task is} \new{Consider the task of} measuring
similarity between two different signals $a(t)$ and $b(t)$. One can
define similarity as a global correlation coefficient and then,
perhaps, measure it in sliding windows across the signal. The local
construction from the previous section suggests approaching this
problem in a more elegant way.

\subsection{Definition of global correlation}

Global correlation coefficient between $a(t)$ and $b(t)$ can be
defined as the functional
\begin{equation}
  \label{eq:glocor}
  \gamma = \frac{<a(t),b(t)>}{\sqrt{<a(t),a(t)>\,<b(t),b(t)>}}\;,
\end{equation}
where $<x(t),y(t)>$ denotes the dot product between two \new{two} signals:
\begin{equation}
  \label{eq:dotprod}
  <x(t),y(t)> = \int x(t)\,y(t)\,d t\;.
\end{equation}
According to \old{the} definition~\ref{eq:glocor}, the correlation
coefficient of two identical signals is equal to one, and the
correlation of two signals with opposite polarity is minus one. In all
the other cases, the correlation will be less then one in magnitude
thanks to the Cauchy-Schwartz inequality. 
\old{A more general Jensen
inequality may lead to other possible similarity measures
\mbox{\cite[]{pvi}}. The correlation coefficient is particularly convenient
in practice because it is insensitive to amplitude fluctuations and
missing data.}

The global measure~\ref{eq:glocor} is inconvenient because it
supplies only one number for the whole signal. The goal of local
analysis is to turn the functional into an operator and to produce
local correlation as a variable function $\gamma(t)$ that identifies
local changes in the signal similarity.

\subsection{Definition of local correlation}

In a linear algebra notation, the squared correlation coefficient
$\gamma$ from equation~\ref{eq:glocor} can be represented as a
product of two least-squares inverses
\begin{eqnarray}
  \label{eq:g}
  \gamma^2 & = & \gamma_1\,\gamma_2\;, \\
  \label{eq:g1}
  \gamma_1 & = & \left(\mathbf{a}^T\,\mathbf{a}\right)^{-1}\,\left(\mathbf{a}^T\,\mathbf{b}\right)\;, \\
  \label{eq:g2}
  \gamma_2 & = & \left(\mathbf{b}^T\,\mathbf{b}\right)^{-1}\,\left(\mathbf{b}^T\,\mathbf{a}\right)\;, 
\end{eqnarray}
where $\mathbf{a}$ is a vector notation for $a(t)$, $\mathbf{b}$ is a
vector notation for $b(t)$, and $\mathbf{x}^T\,\mathbf{y}$ denotes the
dot product operation. Let $\mathbf{A}$ be a diagonal operator
composed from the elements of $\mathbf{a}$ and $\mathbf{B}$ be a
diagonal operator composed from the elements of $\mathbf{b}$.
Localizing equations~\ref{eq:g1}-\ref{eq:g2} amounts to adding
regularization to inversion. Scalars $\gamma_1$ and $\gamma_2$ turn
into vectors $\mathbf{c}_1$ and $\mathbf{c}_2$ defined\old{ as \\
\parbox{\textwidth}{
\begin{eqnarray}
  \label{eq:c1}
  \mathbf{c}_1 & = & \left(\mathbf{A}^T\,\mathbf{A}+\epsilon_1^2\,\mathbf{R}\right)^{-1}\,\mathbf{A}^T\,\mathbf{b}\;, \\
  \label{eq:c2}
  \mathbf{c}_2 & = & \left(\mathbf{B}^T\,\mathbf{B}+\epsilon_2^2\,\mathbf{R}\right)^{-1}\,\mathbf{B}^T\,\mathbf{a}
\end{eqnarray}} 
or}, using shaping regularization \old{with $\mathbf{S}=\mathbf{H}\,\mathbf{H}^T$}, \new{as}
\begin{eqnarray}
  \label{eq:c1s}
  \mathbf{c}_1 & = & 
  \left[\lambda^2\,\mathbf{I} + 
    \mathbf{S}\,\left(\mathbf{A}^T\,\mathbf{A} - \lambda^2\,\mathbf{I}\right)\right]^{-1}\,
  \mathbf{S}\,\mathbf{A}^T\,\mathbf{b}\;, \\
  \label{eq:c2s}
  \mathbf{c}_2 & = & 
    \left[\lambda^2\,\mathbf{I} + 
      \mathbf{S}\,\left(\mathbf{B}^T\,\mathbf{B} - \lambda^2\,\mathbf{I}\right)\right]^{-1}\,
    \mathbf{S}\,\mathbf{B}^T\,\mathbf{a}\;.
\end{eqnarray}
To define a local similarity measure, I apply the component-wise
product of vectors $\mathbf{c}_1$ and $\mathbf{c}_2$. It is
interesting to note that, if one applies an iterative
conjugate-gradient inversion for computing the inverse operators in
equations~\ref{eq:c1s} and~\ref{eq:c2s}, the output of the first
iteration will be the smoothed product of the two signals
$\mathbf{c}_1 = \mathbf{c}_2 = \mathbf{S}\,\mathbf{A}^T\,\mathbf{b}$,
which is equivalent, with an appropriate choice of $\mathbf{S}$\old{ as a
Gaussian filter}, to the algorithm of fast local cross-correlation
proposed by \cite{hale}.

The local similarity attribute is useful for solving the problem of
multicomponent image registration. After an initial registration using
interpreter's ``nails'' \cite[]{TLE23-12-12701281} or velocities from
seismic processing, a useful registration indicator is obtained by
squeezing and stretching the warped shear-wave image while measuring
its local similarity to the compressional image. Such a technique was
named \emph{residual $\gamma$ scan} and proposed by \cite{warp}.
Figure~\ref{fig:vec-sc-0} shows a residual scan for registration of
multicomponent images from Figure~\ref{fig:pp,ss}. Identifying and
picking points of high local similarity enables multicomponent
registration with high-resolution accuracy. The registration result is
visualized in Figure~\ref{fig:vec-in0-0,vec-in1-1}, which shows
interleaved traces from PP and SS images before and after
registration. The alignment of main seismic events is an indication of
successful registration.

\plot{vec-sc-0}{width=0.9\columnwidth}{Residual warping scan for
  multicomponent PP/SS registration computed with the help of the
  local similarity attribute. Picking maximum similarity trends
  enables multicomponent registration.}

\multiplot{2}{vec-in0-0,vec-in1-1}{width=0.45\columnwidth}{Interleaved
  traces from PP and warped SS images before (a) and after (b)
  multicomponent registration. The checkerboard pattern on major
  seismic events in (a) disappears in (b) which is an indication of
  successful registration.}

%I illustrate an application of local correlation by considering the
%problem of multicomponent data
%registration. Figure~\ref{fig:s1cube,s2cube} shows two converted-wave
%images of a fractured reservoir in Colombia obtained by separating
%fast and slow shear-wave components. The time differences between the
%two images are indicative of velocity differences, which are caused by
%azimuthal anisotropy associated with fracturing.

%inputdir{harken}
%\multiplot{2}{s1cube,s2cube}{width=0.9\columnwidth}{Fast (a) and Slow (b) converted-wave images of a fractured reservoir.}
%\plot{scan2}{width=\columnwidth}{here}
%\plot{hinter}{width=\columnwidth}{here}
%\plot{wcube}{width=\columnwidth}{here}
%\plot{frac}{width=\columnwidth}{here}

\old{%\section{Measuring local image focusing}
A focused seismic signal has its energy concentrated at certain
locations instead of being widely spread. How can one define an
attribute for measuring the degree of focusing? It is actually easier
to define the opposite measure. We can take the definition of the
correlation coefficient from equation~\ref{eq:glocor} and measure
the correlation between the squared signal and a constant. Large
correlation means poor focusing. Therefore, the focusing measure
will be the reverse of the correlation with the constant or \\
\parbox{\textwidth}{
\begin{equation}
  \label{eq:gfocus}
  \phi = 
  \left(\frac{\displaystyle \sqrt{<a^2(t),a^2(t)>\,<e(t),e(t)>}}{\displaystyle <a^2(t),e(t)>}\right)^2 =
  \frac{\displaystyle N_t\,\int a^4(t)\, dt}{\displaystyle \left(\int a^2(t)\, dt\right)^2}\;,
\end{equation}}
where $e(t)$ is a unit signal, and $N_t = \int dt$. The global
focusing measure~\ref{eq:gfocus} is the varimax norm of
\mbox{\cite{wiggins}} related to kurtosis of a zero-mean signal.  Varimax was
used by \mbox{\cite{wiggins}} in minimum-entropy deconvolution and by
\mbox{\cite{GEO52-01-00510059}} for automatic zero-phase correction.
While the global focusing attribute~\ref{eq:gfocus} measures total
focusing inside a data window, its localization produces a continuous
measure reflecting local changes in signal focusing. To define a local
focusing attribute, I use equations \ref{eq:c1s}-\ref{eq:c2s} again
substituting the squared signal for $\mathbf{a}$; a constant signal
for $\mathbf{b}$, and taking the inverse of the component-wise product
of $\mathbf{c}_1$ and $\mathbf{c}_2$. The inverted operator in
equation~\ref{eq:c2s} reduces to an identity, and thus only one
inversion is required.
The local focusing attribute defined here finds an application in
migration velocity analysis by diffraction imaging.}

\section{Conclusions}

I have introduced a concept of local seismic attributes and specified
it for such attributes as local frequency and local similarity\old{, and local
focusing}. Local attributes measure signal characteristics not
instantaneously at each signal point and not globally across a data
window but locally in the neighborhood of each point. They find
applications in different steps of multicomponent seismic image
registration. One can extend the idea of local attributes to other
applications.

\section{Acknowledgments}

I would like to thank Bob Hardage, Milo Backus, and Bob Graebner for
introducing me to the problem of multicomponent data registration and
for many useful discussions. The data example was kindly provided by
Vecta. This work was partially supported by the U.S. Department of
Energy through Program DE-PS26-04NT42072 and is authorized for
publication by the Director, Bureau of Economic Geology, The
University of Texas at Austin.

%\plot{vec-gamma2-1}{width=\columnwidth}{Average $V_P/V_S$ ratio from multicomponent registration.} 

\bibliographystyle{seg}
\bibliography{SEG2005,attr}

%%% Local Variables: 
%%% mode: latex
%%% TeX-master: t
%%% End: 
