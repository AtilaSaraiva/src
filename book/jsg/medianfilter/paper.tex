
\title{A 1-D time-varying median filter for seismic random, spike-like noise elimination}

\renewcommand{\thefootnote}{\fnsymbol{footnote}}

\ms{GEO-2007-0338}

\address{
\footnotemark[1] College of Geo-exploration Science and Technology,\\
Jilin University \\
No.6 Xi minzhu street, \\
Changchun, China, 130026 \\
\footnotemark[2] Bureau of Economic Geology,\\
John A. and Katherine G. Jackson School of Geosciences \\
The University of Texas at Austin \\
University Station, Box X \\
Austin, TX, USA, 78713-8924}

\author{Yang Liu\footnotemark[1]\footnotemark[2], Cai Liu\footnotemark[1], Dian Wang\footnotemark[1]}

\footer{GEO-2007-0338}
\lefthead{Liu etc.}
\righthead{1-D time-varying median filter}
\maketitle

\begin{abstract}

Random noise in seismic data affects the signal-to-noise ratio, 
obscures details, and complicates identification of useful information. We 
present a new method for reducing random, spike-like noise in seismic data. 
The method is based on a 1-D stationary median filter (MF) -- the \emph{1-D 
time-varying median filter} (TVMF). We design a threshold value that 
controls the filter window according to characteristics of signal and 
random, spike-like noise. In view of the relationship between seismic data 
and the threshold value, we chose median filters with different time-varying 
filter windows to eliminate random, spike-like noise. When comparing our 
method with other common methods, e.g., the band-pass filter 
and stationary MF, we found that 
the TVMF strikes a balance between eliminating random noise and protecting 
useful information. To demonstrate the feasibility of our method in reducing 
seismic random, spike-like noise, we present results for one synthetic dataset. 
Results of applying the method to seismic land data from Texas 
demonstrate that the TVMF method is effective in practice. 

\end{abstract}

\section{Introduction}

Random noise in prestack seismic data can come from various sources, such as
wind motion, poorly planted geophones, or electrical noise, and some of this 
seismic random noise invariably exhibits spike-like characteristics. Although 
stacking can at least partly suppress random noise in prestack data, residual 
random noise after stacking will decrease the accuracy of final data interpretation. 
In recent years, several authors have developed effective methods of eliminating 
random noise. For example, \cite{Gulunay00} used the noncausal prediction filter 
for random-noise attenuation, \cite{Ristau01} compared several 
adaptive filters, which they applied in an attempt to reduce random noise in 
geophysical data. \cite{Karsli06}
applied complex-trace analysis to seismic data for random-noise suppression,
recommending it for low-fold seismic data, and some transform methods 
were also used to eliminate seismic random noise, e.g., seislet transform \cite[]{Fomel06,Fomel08},
discrete cosine transform \cite[]{Lu07}, and curvelet transform \cite[]{Neelamani08}.

On the other hand, the median filter, a well-known method that can effectively suppress 
spike-like noise, refers to nonlinear signal processing. 
\cite{Bednar83} and \cite{Duncan95} found the method to be both simple and effective for seismic prospecting. 
More recently, new median filters have been 
proposed. \cite{Mi00} incorporated median-filter noise reduction into 
standard Kirchhoff time migration. \cite{Zhang03} used a hyperbolic 
median filter to suppress multiples. \cite{Liu06} advocated random-noise 
attenuation using the 2-D multistage median filter (MLM).

Because the median filter is a nonlinear filter, filter-window length needs 
to be adjusted before its characteristics can be changed. The stationary filter, 
on the other hand, maintains a fixed window length, retaining useful information and 
random noise at the same scale. An unsuitable filter-window choice would therefore 
end up in useful information being destroyed or noise remaining. Here we propose 
a time-varying median filter that adjusts to different filter-window lengths by 
threshold, making value judgments on useful information versus noise throughout 
the process. We show that the time-varying window is more powerful than the 
stationary window. 

In this paper, a new nonlinear filter called the \emph{time-varying median filter} (TVMF) 
is presented, which we designed by defining a threshold value. After adjustment 
of the filter window in the time domain, this filter eliminates 
random noise in seismic data. We compare TVMF with other common methods.
We use numerical examples, along with synthetic and field data, to demonstrate 
the validity of the proposed method in practice.

\section{Theoretical basis}

In contrast to useful information, random, spike-like noise in seismic data is 
neither continuous nor correlative, and a 1-D stationary median filter, having 
a large filter-window length, can easily remove such noise. However,
signal can be damaged by such a filter.

The 1-D TVMF is based on the 1-D stationary median filter. We propose to measure
the local noise content of the data and to adjust the filter length adaptively.
If a threshold value that judges random noise and estimates noise intensity can 
be chosen, a filter having a large filter-window length can eliminate random noise 
while processing useful information by using a small filter-window length. 
Such a filter can thus effectively eliminate random noise while maximizing 
preservation of a detailed structure of useful information. 

 \subsection{1-D stationary median filter (MF)}

The 2-D seismic record can be represented by the following data sequence:
      \begin{equation}
        \label{eq:eq1}
          x_{i,j} \qquad (i=1,\cdots,m,\cdots,N_x;j=1,\cdots,n,\cdots,N_t)\;,
      \end{equation}
where $i$ is the spatial sample index, $j$ is the temporal sample index, and 
$N_x$ and $N_t$ are the numbers of spatial and temporal samples. When filter-window 
length, $C$, of the stationary median filter is defined (normally $C$ is odd), 
the result after filtering at the point on the $m^{th}$ trace and the $n^{th}$ sample 
can be found by

1. Setting the center point at the $m^{th}$ trace and the $n^{th}$ sample, and 
choosing $C$ samples in the $m^{th}$ trace,

2. Sorting the $C$ samples from smaller to larger, and then

3. Picking the center value, after sequencing, as the output at the point on the $m^{th}$ trace 
and the $n^{th}$ sample.

Repetition of the process on all data achieves 
1-D stationary median filtering of the seismic record. The 1-D stationary median 
filter can be expressed as $median[x_{i,j}]$.

\subsection{Signal-to-noise ratio (SNR) estimation using the stack method}

A simple definition of the SNR was introduced by \cite{Liu97}. Window $D$, a part 
of the seismic record, can be chosen for SNR analysis: 
      \begin{equation}
        \label{eq:eq5}
          D = [x_{i,j}]_{M \times N} \qquad (0< M \le N_x, 0< N \le N_t)\;.
      \end{equation}
Further assumptions are: waveform, amplitude, and phase of seismic 
wavelet in window $D$ keep stable in respect to distance ``$i$''; noise is
``zero mean'' randomly distributed, along with survey-line direction being independent 
(decorrelated) of the signal, so that
      \begin{equation}
        \label{eq:eq6}
          x_{i,j} = s_j + n_{i,j}\;
      \end{equation}
      \begin{equation}
        \label{eq:eq7}
          \sum_{i=1}^{M} n_{i,j} = 0\;,
      \end{equation}
where $s_j$ is amplitude of signal, and $n_{i,j}$ is amplitude of noise. These 
assumptions generally imply a limitation to this method, but they 
can be satisfied if the local window is chosen in the stable signal region of the 
seismic section. So if the signal energy in the window is 
      \begin{equation}
        \label{eq:eq8}
          E_S = M\sum_{j=1}^{N} s_{j}^2 = \frac{1}{M} \sum_{j=1}^{N}(\sum_{i=1}^{M}x_{i,j})^2\;,
      \end{equation}
the noise energy can be calculated by
      \begin{equation}
        \label{eq:eq9}
          E_N = \sum_{j=1}^{N}\! \sum_{i=1}^{M}x_{i,j}^2 - E_S\;.
      \end{equation}
Finally, a decibel expression of the SNR is estimated as
      \begin{equation}
        \label{eq:eq10}
          SNR = \frac{E_S}{E_N} = 10log_{10}\left(\frac{\displaystyle\sum_{j=1}^{N}\big(\sum_{i=1}^{M}x_{i,j}\big)^2} 
         {\displaystyle M\sum_{j=1}^{N}\! \sum_{i=1}^{M}x_{i,j}^2-\sum_{j=1}^{N}\big(\sum_{i=1}^{M}x_{i,j}\big)^2}\right)\;.
      \end{equation}

 \subsection{1-D time-varying median filter (TVMF)}

Using the above definitions, a TVMF can be designed. We use the following three steps to determine its parameters:

1. Choose the reference median filter length.

At point $x_{m,n}$, where the filter-window length of the reference 
median filter is chosen as $C$, output can be expressed as
      \begin{equation}
        \label{eq:eq2}
          Y_{m,n}^C = median[x_{i,j}] \qquad (i=m;j=n-(C-1)/2,\cdots,n+(C-1)/2)\;,
      \end{equation}
where filter-window length $C$ is a large odd number 
so that random noise could be eliminated as much as possible. $C$ is determined 
by using the SNR estimation method, which will be discussed later.

2. Choose the threshold value.

Using the reference median filter with its large filter-window length, we processed 
the seismic data first to find $Y_{m,n}^C$. Then we applied the absolute mean value 
to calculate the threshold value, which is shown as
      \begin{equation}
        \label{eq:eq3}
          T = \frac{1}{N_x \times N_t} \sum_{i=1}^{N_x}\! \sum_{j=1}^{N_t}|Y_{i,j}^C|\;,
      \end{equation}
We can evaluate random-noise data versus useful signal data by using the threshold 
value. When $|Y_{i,j}^C|< T$, the point is judged to be random noise, whereas when 
$|Y_{i,j}^C|\ge T$, the point should be signal data. We can therefore use the 
threshold value as a judgment norm -- data in which $|Y_{i,j}^C|\ge T$ should be 
processed by the median filter having windows smaller than $C$ to protect the 
detailed signal structure. Data in which $|Y_{i,j}^C|< T$ should be processed 
by the median filter having windows larger than $C$ to strengthen its ability to 
eliminate random noise.

3. Choose the time-varying filter windows.

Choices involving time-varying windows abound after the threshold value has been 
chosen. We can define four scales of windows. Detailed time-varying 
window length $C_{i,j}$ is defined as 
        \begin{equation}
        \label{eq:eq4}
          C_{i,j} = \left \{ \begin{array}{ll}
                    C+\alpha, & \textrm{$0<|Y_{i,j}^C|<T/2$} \\
                    C+\beta, & \textrm{$T/2<|Y_{i,j}^C|<T$} \\
                    C-\gamma, & \textrm{$T \le |Y_{i,j}^C| <2T$}\\
                    C-\delta, & \textrm{$|Y_{i,j}^C| \ge 2T$}
                  \end{array} \right.\;,
      \end{equation}
where $\alpha$, $\beta$, $\gamma$, and $\delta$ are constant even numbers, and 
$\alpha>\beta$ and $\delta>\gamma$. Specific values for these 
parameters will be discussed in the next section. Using 
the above definition, we distinguish between random noise 
and useful signal, such that we can process 
the seismic data using different filter scales.

 \section{Synthetic data tests}

To choose reference filter windows and time-varying filter windows, we analyzed the 
characteristics of our TVMF formulation using synthetic data. Referring to the 
velocity model (Table~\ref{tbl:table}), note the synthetic common-shot record with 
a dominant frequency of 40 Hz shown in Figure~\ref{fig:model,noise0}a. A 5\%-density 
white spike noise was then added to the model. Three different noise peaks (noise amplitude is half, one, 
and two times the maximum value of the reflections, and the corresponding noise 
intensity is 1/2, 1, and 2) were chosen to separately test the TVMF. Only noise with twice the 
maximum value of the reflected wave is displayed in Figure~\ref{fig:model,noise0}b. Energy 
attenuation has not been taken into account, nor has spherical spreading or AVO. 
Here, the signal has been drowned out by noise.

    \tabl{table}{Velocity model}
    {
      \begin{center}
        \begin{tabular}{ccccc}
          \hline
          Stratum thickness~(m)  & 500  & 1739 & 2069 &  \\
          Stratum velocity~(m/s) & 2500 & 3162 & 4138 & 4500 \\
          \hline
        \end{tabular}
      \end{center}
      }

Definitions of signal energy (equation~\ref{eq:eq8}), noise energy 
(equation~\ref{eq:eq9}), and SNR (equation~\ref{eq:eq10}) were used to analyze 
characteristics of the reference median filter. We chose the stationary MF with 
different filter-window lengths of from 3 to 19 points, comparing various signal 
energies (Figure~\ref{fig:es,en,snr}a), noise energies (Figure~\ref{fig:es,en,snr}b), and the SNR's 
(Figure~\ref{fig:es,en,snr}c) after filtering. To meet the assumptions 
of the SNR model, we chose five traces near the zero offset, where seismic events 
are approximately invariant across the five traces. Note that in the 
figures, curves of the 
three noise intensities (1/2, 1 and 2) appear to have similar tendencies. Both 
signal energy and noise energy decrease as filter-window length increases, but the 
SNR displays a wiggle shape. The energy levels 
of signals are stable after filter-window length reaches 11 points, but the energy levels of noise 
are stable after filter-window length reaches 15 points. And because the curves show a different 
rate of descent, the SNR has a different tendency. It reaches 
a peak at the 5-point filter window and decreases afterward 
because signal energy attenuates faster than noise energy. The SNR then reaches the 
minimum near 11 points, where the signal energy is stable, and next, the SNR improves 
again because the noise energy still decreases. Finally, the SNR reaches stability, 
however, the signal has been barely damaged.

In the noisy model 
(Figure~\ref{fig:model,noise0}b), the SNR is -10.7 dB. After stationary MF filtering, 
even the minimum SNR remains much larger than -10.7 dB, illustrating that all 
stationary median filters can improve the SNR in the noisy model. 
We can select large filter windows of reference median 
filters to keep noise energy to a minimum when threshold $T$ is well able to separate 
signal from noise, but at the same time we can define time-varying filter windows on 
the basis of reference filter windows so that the reference filter window will be 
limited. Three basic principles should be satisfied:

1. The reference filter window can be chosen as a large value to separate signal from 
noise when noise energy is small.

2. Time-varying filter windows at signal positions should be small values to 
protect the signal when removing noise.

3. Time-varying filter windows at noise positions should be large values to attenuate 
noise energy.

In Figure~\ref{fig:es,en,snr}, reference filter window $C$ should be larger than 
the stable signal point (11 points), and time-varying filter windows
$C-\gamma$ and $C-\delta$ at signal positions should be in the range of from 5 to 
7 points in order to preserve signal energy. Time-varying filter windows $C+\alpha$
and $C+\beta$ should be limited in range from 11 to 13 points in order to 
attenuate noise energy and save calculation time. To meet all principles, $C$ can be 
11 points, with $\alpha=2$, $\beta=0$, $\gamma=4$, and $\delta=6$. 
After more tests on synthetic and real data it became clear that these filter 
parameter choices work for most real data.
 \inputdir{model}  
  \multiplot{2}{model,noise0}{width=0.45\textwidth}{Synthetic model (a) and white-noise 
model (b).} 
  \inputdir{window}  
  \multiplot{3}{es,en,snr}{width=0.45\textwidth}{Comparison of different noise levels 
(Intensity (1/2, 1, and 2) is the amplitude ratio between noise and reflections). 
Signal energy (a), noise energy (b), and SNR (c).} 

We used the TVMF, with defined reference and time-varying filter windows, to process the 
noise model (Figure~\ref{fig:model,noise0}b), the result of which is shown in Figure~\ref{fig:w11tvmf,noise1}a. 
When assumptions can be met, the SNR estimation using the stack method can be 
used to compare results. After TVMF processing, the SNR is 19.7 dB. To compare, we also 
used the low-pass filter for preserving the 
signal in the dominant frequency band (Figure~\ref{fig:w11tvmf,noise1}b, the SNR is -7.4 dB). After 
TVMF processing, white spike noise attenuated well, but the result after 
low-pass filtering became a signal with 
band-limited noise, and a great deal of low-frequency noise remained. Next, we 
also used the TVMF to process the band-limited noise model 
(Figure~\ref{fig:w11tvmf,noise1}b). SNR analysis shows that the parameters of TVMF only change a little, so 
the same parameters can be used for processing. The TVMF cannot 
eliminate all band-limited noise, like white noise, but the 
energy of the noise has been degraded. At the same time, the TVMF introduces 
a few high-frequency noise components having low energy, but these can be easily 
removed by using a high-cut filter. Figure~\ref{fig:cb11tvmf,c11mf}a shows the result after
TVMF and high-cut filtering; the SNR is 0.5 dB. The low-pass filter does 
nothing about band-limited noise. The 11-point 
stationary MF (Figure~\ref{fig:cb11tvmf,c11mf}b, the SNR is -8.4 dB) 
can be used to compare with the TVMF. The stationary MF can 
remove most of the noise, but useful information is also destroyed. After comparing 
the result of the TVMF with those of the stationary MF and the band-pass filter, 
we conclude that the TVMF is superior when processing random, spike-like noise. 

We can compare results of using different methods 
by analyzing their spectra as well. We chose spectra of the trace at a distance of 4 km, 
corresponding to the pertinent parts of Figures~\ref{fig:model,noise0}, 
\ref{fig:w11tvmf,noise1}, and \ref{fig:cb11tvmf,c11mf}. Results in 
Figure~\ref{fig:smodel,snoise0,snoise1,scb11tvmf} show that spectral values of random 
noise are larger than those of the signal at every frequency and that reflected waves 
have been masked by random noise (Figure~\ref{fig:smodel,snoise0,snoise1,scb11tvmf}b). The low-pass 
filter can eliminate noise in the high-frequency band, but it does nothing in the 
dominant-frequency band (Figure~\ref{fig:smodel,snoise0,snoise1,scb11tvmf}c). After the 
band-limited model is processed by the TVMF using an 11-point reference filter window, most spectral 
components in the dominant-frequency band can be recovered. An additional high-cut filter was used to remove
high-frequency noise introduced by the TVMF (Figure~\ref{fig:smodel,snoise0,snoise1,scb11tvmf}d). The 
corresponding time profile is shown in Figure~\ref{fig:cb11tvmf,c11mf}a.
\inputdir{model}  
  \multiplot{2}{w11tvmf,noise1}{width=0.45\textwidth}{Denoised result after 11-point TVMF 
filtering (a) and denoised result (band-limited noise model) 
after low-pass filtering (b).}
  \multiplot{2}{cb11tvmf,c11mf}{width=0.45\textwidth}{Denoised band-limited 
noise model using different methods (the result of Figure~\ref{fig:w11tvmf,noise1}b was input 
to these tests); 11-point TVMF followed by a high-cut filter (a) and 11-point 
stationary MF (b).} 
  \multiplot{4}{smodel,snoise0,snoise1,scb11tvmf}{width=0.45\textwidth}{Comparison of amplitude 
spectra (trace at distance of 4 km) before and after processing. Amplitude spectra  
in Figure~\ref{fig:model,noise0}a (a), corresponding spectra in Figure~\ref{fig:model,noise0}b (b),
corresponding spectra in 
Figure~\ref{fig:w11tvmf,noise1}b (c), and corresponding spectra in Figure~\ref{fig:cb11tvmf,c11mf}a (d).}

Given the result of model filtering, the TVMF can easily remove spike-like noise, especially 
noise having a white spectrum. When the TVMF is used to attenuate band-limited, 
spike-like noise, its filter ability decreases, although it can still work better than other 
common methods.

 \section{Processing of field data}

A real-data example involves land prestack data from Texas 
(Figure~\ref{fig:dragon1,band,11mf,11tvmf,diff}a), from which we show one common midpoint 
gather that has 59 traces and an average group interval of 42 m. The sample frequency is 4 ms, and the sample number is 500. In 
Figure~\ref{fig:dragon1,band,11mf,11tvmf,diff}a, noise is mainly random noise caused by the surface condition of this 
area; the 60-Hz electrical 
interference with high amplitude is especially serious. According to the amplitude spectra, 
the dominant frequency of the real data is about 25 Hz; the 60-Hz electrical interference thus exhibits 
high-frequency, spike-like characteristics when compared with the useful signal. In the time domain 
(Figure~\ref{fig:dragon1,band,11mf,11tvmf,diff}a, especially at locations ``A,'' ``B,'' and ``C''), this spike-like 
characteristic of electrical interference can also be observed. To meet the assumptions of SNR 
estimation using the stack method, a rectangle region ``D'' that has 9 traces and 125 samples is chosen, and, after calculating, the SNR 
in the original data is found to be -7.3 dB. First, we apply a low-pass filter to remove 
the random noise (Figure~\ref{fig:dragon1,band,11mf,11tvmf,diff}b); the frequency is limited to 60 Hz. The 
low-pass filter can partly eliminate random noise, and the corresponding 
SNR is improved to 0.4 dB, but it cannot attenuate the noise in the dominant-frequency band. 
We can thus still see lots of random noise after filtering. The 11-point stationary MF is also 
used to compare with the TVMF, and the result is shown in Figure~\ref{fig:dragon1,band,11mf,11tvmf,diff}c. After processing, there is 
still some spike-like noise left, and part of the signal has been attenuated. The SNR is 1.9 dB, and a larger 
filter window will further destroy the useful signal. By 
using the SNR analysis introduced in the theory section, we are choosing the parameters about the TVMF in window 
``D.'' The TVMF with the same parameters as for the synthetic examples, $C=11$, $\alpha=2$, 
$\beta=0$, $\gamma=4$, and $\delta=6$, is applied to the prestack data
(Figure~\ref{fig:dragon1,band,11mf,11tvmf,diff}d). Figure~\ref{fig:dragon1,band,11mf,11tvmf,diff}e shows the difference section, in which the 
processed data using the TVMF have been subtracted from the original data. After TVMF processing, 
continuity of events and information of the reflected waves are all enhanced, and there is little 
noise left. Note that the 60-Hz electrical interference in particular can be completely removed. 
The SNR reaches 4.7 dB. In the difference section it can be seen that the TVMF eliminates 
major spike-like noise and that there is little useful information beyond 
the spike-like noise, showing that the TVMF can effectively eliminate spike-like noise and 
protect useful information at the same time. Because the TVMF is based on the stationary MF that can 
filter noise spikes, the TVMF can effectively eliminate spike-like noise. And because the TVMF 
can separate useful information from spike-like noise, 
it can protect useful information and perform better overall than other potential methods 
in spike-like noise attenuation.
  \inputdir{dragon}
  \multiplot{5}{dragon1,band,11mf,11tvmf,diff}{width=0.45\textwidth}{Real land data (a), denoised 
result after low-pass filtering (b), denoised result after stationary MF filter (c), denoised result after TVMF filtering (d), and 
difference between original data and result after TVMF processing (e).}

We chose a trace from the original data and corresponding traces in processed data and got the 
amplitude spectra of four traces (Figure~\ref{fig:dspec,bspec,mfspec,fspec}). After 
low-pass filtering, high-frequency noise can be removed thoroughly, but 
some noise in the dominant-frequency band can remain; 
electrical interference in particular must be eliminated by other methods (Figure~\ref{fig:dspec,bspec,mfspec,fspec}b). Stationary
MF can eliminate enough noise, but spectra of useful signals have been distorted (Figure~\ref{fig:dspec,bspec,mfspec,fspec}c). 
In Figure~\ref{fig:dspec,bspec,mfspec,fspec}d, the TVMF can attenuate noise in the entire frequency band and protect 
the useful-information frequency components. 

  \multiplot{4}{dspec,bspec,mfspec,fspec}{width=0.45\textwidth}{Comparison of amplitude 
spectra (trace at distance of 1.8 km) before and after processing. Amplitude spectra in 
Figure~\ref{fig:dragon1,band,11mf,11tvmf,diff}a (a), corresponding spectra in 
Figure~\ref{fig:dragon1,band,11mf,11tvmf,diff}b (b), 
corresponding spectra in Figure~\ref{fig:dragon1,band,11mf,11tvmf,diff}c (c), and corresponding 
spectra in Figure~\ref{fig:dragon1,band,11mf,11tvmf,diff}d (d).} 

Next we obtained stacked sections corresponding to Figure~\ref{fig:dragon1,band,11mf,11tvmf,diff}b 
and \ref{fig:dragon1,band,11mf,11tvmf,diff}d; the stacking fold is 59. A rectangular region, ``H,'' was chosen 
to calculate the local SNR. Events of the stacked section using low-pass 
processing were discontinuous (Figure~\ref{fig:stack1,stack2}a), the SNR is -7.8 dB, 
and the continuity of events in the stacked section using TVMF processing was improved, especially 
at locations ``E,'' ``F,'' and ``G'' (Figure~\ref{fig:stack1,stack2}b), and the 
corresponding SNR is -3.1 dB. Because the TVMF can eliminate spike-like noise 
in prestack data, it is advantageous to apply to such data.

  \multiplot{2}{stack1,stack2}{width=0.7\textwidth}{Stacked section of prestack 
data after low-pass filtering (a) and corresponding section after TVMF processing (b).}

 \section{Conclusion}

We have proposed and tested the time-varying median filter 
(TVMF) for attenuating 
prestack random, spike-like noise. We used a signal-to-noise ratio estimation to design 
the reference filter window and time-varying filter windows, in which the filter 
can adapt filter-window length to meet the intensity of the random, spike-like noise.
The ability of this method to eliminate random noise while protecting desired signal 
further attests to the strength of the method.
Our experiments show that in field data the TVMF can eliminate random noise enough to
enhance the continuity of events. Spectral analysis also shows that the TVMF 
can effectively suppress random, spike-like noise in the whole frequency band. Comparison 
of different methods shows that the TVMF is more effective than the stationary 
median filter for eliminating 
spike-like noise, and it can enhance results of band-pass filtering by attenuating 
spike-like noise within the pass band.

\section{Acknowledgments}

The authors gratefully acknowledge Sergey B. Fomel for his suggestions and help. We 
thank Douglas W. McCowan for fruitful discussions. We sincerely thank assistant 
editor Vladimir Grechka, associate editor Eric Verschuur, and three anonymous 
reviewers for their constructive comments and suggestions. We also thank one 
anonymous reviewer for his suggestion about the possibility of our method on marine 
swell-noise attenuation. Thanks also to BGP International for financial support of this work. 
Publication authorized by the Director, Bureau of Economic Geology.

\bibliographystyle{seg}
\bibliography{geo20070338}

