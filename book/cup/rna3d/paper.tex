\published{Journal of Applied Geophysics, 93, 60-66, (2013)}
\title{Noncausal $f$-$x$-$y$ regularized nonstationary prediction filtering for random noise attenuation on 3D seismic data}

\author{Guochang Liu and Xiaohong Chen}

\ms{APPGEO2549}

\address{
China University of Petroleum, \\
State Key Laboratory of Petroleum Resource and Prospecting, \\
102249, Beijing, China. \\
E-mail: guochang.liu@yahoo.com; chenxh@cup.edu.cn}

\lefthead{Liu et al.}
\righthead{f-x-y NRNA for noise attenuation}

\maketitle

\begin{abstract}
Seismic noise attenuation is very important for seismic data analysis and 
interpretation, especially for 3D seismic data. In this paper, we propose 
a novel method for 3D seismic random noise attenuation by applying noncausal 
regularized nonstationary autoregression (NRNA) in $f$-$x$-$y$ domain. The proposed 
method, 3D NRNA (f-x-y domain) is the extended version of 2D NRNA (f-x domain). 
f-x-y NRNA can adaptively estimate seismic events of which slopes vary in 3D space. 
The key idea of this paper is to consider that the central trace can be predicted 
by all around this trace from all directions in 3D seismic cube, while the 2D 
f-x NRNA just considers the middle trace can be predicted by adjacent traces 
along one space direction. 3D $f$-$x$-$y$ NRNA uses more information from circumjacent 
traces than 2D $f$-$x$ NRNA to estimate signals. Shaping regularization technology 
guarantees the nonstationary autoregression problem can be realizable in mathematics 
with high computational efficiency. Synthetic and field data examples demonstrate 
that, compared with $f$-$x$ NRNA method, $f$-$x$-$y$ NRNA can be more effective in suppressing 
random noise and improve trace-by-trace consistency, which are useful in conjunction 
with interactive interpretation and auto-picking tools such as automatic event tracking. 

\end{abstract}

\section{Introduction}

Seismic noise attenuation is very important for seismic data processing and 
interpretation, especially for 3D seismic data. Among the methods of seismic 
noise attenuation, prediction filtering is one of the most effective and most 
commonly used methods, e.g., \cite []{Gulunay1986, Galbraith1991, Gulunay1993, 
Sacchi2001}. Prediction filtering can be implemented in $f$-$x$ 
domain or $t-x$ domain \cite[]{Hornbostel1991, Abma1995}. \cite{Abma1995}
compared $f$-$x$ method and $t-x$ method and gave the advantages 
and disadvantages of both these methods. The proposed method in our paper 
belongs to the category of $f$-$x$ domain methods. The $f$-$x$ prediction technique 
was introduced for random noise attenuation on 2D poststack data by \cite{Canales1984}
and further developed by \cite{Gulunay1986}. \cite{Wang1991} and 
\cite{Hornbostel1991} used noncausal filters for random noise attenuation on 
stacked seismic data and obtain a good result. Linear prediction filtering 
states that the signal can be described by an autoregressive (AR) model, 
which means that a superposition of linear events transforms into a 
superposition of complex sinusoids in the $f$-$x$ domain. \cite{Sacchi2001} 
utilized the autoregressive-moving average (ARMA) structure of 
the signal to estimate a prediction error filter (PEF) and applied ARMA 
model to attenuate random noise. \cite{Liu2009} applied ARMA-based 
noncausal spatial prediction filtering to avoid the model inconsistency.
 
As already noted, these above mentioned $f$-$x$ methods assume seismic section 
is composed of a finite number of linear events with constant dip in $t-x$ 
domain. To cope with the assumption continuous changes dips, short temporal 
and spatial analysis windows are usually used in $f$-$x$ prediction filtering. 
Except using windowing strategy, several nonstationary prediction filters 
are proposed and used in seismic noise attenuation and interpolation. 
\cite{Naghizadeh2009} proposed an adaptive $f$-$x$ prediction filter, 
which was used to interpolate waveforms that have spatially variant dips. 
\cite{Fomel2009} developed a general method of regularized nonstationary 
aoturegression (RNA) with shaping regularization \cite[]{Fomel2007} for 
time domain inverse problems. \cite{Liu2012} propose a method for 
random noise attenuation in seismic data by applying noncausal 
regularized nonstationary autoregression (NRNA) in frequency domain, 
which is implemented for 2D seismic data. These nonstationary methods 
can control algorithm’s adaptability to changes in local dip so that 
they can process curved events.
 
If using $f$-$x$ prediction filter to suppress random noise on 3D seismic 
data, one need to run the 2D algorithm slice by slice (along inline x or 
crossline y). To use more information to predict the effective signal 
in 3D data, several geophysicists extended $f$-$x$ prediction filtering to 
3D case. \cite{Chase1992} designs and applies 2-D prediction filters in the
 plane defined by the inline and crossline directions for each temporal 
frequency slice of the 3-D data volume. \cite{Ozdemir1999} applied 
f-x-y projection filtering to attenuate random noise of seismic data 
with low poor signal to noise ratio (SNR), in which the crucial step 
of 2-D spectral factorization is achieved through the causal helical 
filter. \cite{Gulunay2000} proposed using full-plane noncausal prediction 
filters to process each frequency slice of the 3-D data. \cite{Wang2002} 
applied $f$-$x$-$y$ 3D prediction filter to implement seismic data interpolation 
and gave a good result. \cite{Hodgson2002} presented a novel method of 
noise attenuation for 3D seismic data, which applies a smoothing filter 
to each targeted frequency slice and allows targeted filtering of selected 
parts of the frequency spectrum.

In this paper, we extend $f$-$x$ NRNA method \cite[]{Liu2012} to $f$-$x$-$y$ case 
and use $f$-$x$-$y$ NRNA to attenuate random noise for 3D seismic data. The 
coefficients of 3D NRNA method are smooth along two space coordinates 
(x and y) in $f$-$x$-$y$ domain. This paper is organized as follows: First, 
we provide the theory for random noise on 3D seismic data, paying 
particular attention to establishment of $f$-$x$-$y$ NRNA equations with 
constraints and implementation of it with shaping regularization. 
Then we evaluate and compare the proposed method with $f$-$x$ NRNA using 
synthetic and real data examples and discuss the parameter selection 
problem associated with our algorithm. 

\section{Methodology}

 \subsection{The review of $f$-$x$ NRNA}

Seismic section $S(t,x)$ in $f$-$x$ domain is predictable if it only 
includes linear events in $t-x$ domain. The relationship between the 
n-th trace and (n-i)-th trace can be easily described as
      \begin{equation}
          {{S}_{n}}(f)=\sum\limits_{i=1}^{M}{{{a}_{i}}(f){{S}_{n-i}}(f)},
        \label{eq:eq1}
      \end{equation}
where M is the number of events in 2D seismic section. Eq. (1) describes 
forward prediction equations, namely causal prediction filtering equations 
\cite[]{Gulunay2000}. In the case of both forward and backward prediction equations 
(noncausal prediction filter), Eq. ~\ref{eq:eq1} can be written as \cite[]{Spitz1991,Gulunay2000, Naghizadeh2009, Liu2012}
      \begin{equation}
          {{S}_{n}}(f)\text{=}\sum\limits_{i=1}^{M}{{{a}_{i}}{{S}_{n-i}}}(f)\text{+}\sum\limits_{i=-1}^{-M}{{{a}_{i}}{{S}_{n-i}}(f)},
        \label{eq:eq2}
      \end{equation}
where M is the parameter related to the number of events. Note that Eq. ~\ref{eq:eq2} 
implies the assumption $\sum\nolimits_{i=1}^{M}{{{a}_{i}}{{S}_{n-i}}(f)}\text{=0}.\text{5}{{S}_{n}}(f)$and$\sum\nolimits_{i=-1}^{-M}{{{a}_{i}}{{S}_{n-i}}(f)}\text{=0}.\text{5}{{S}_{n}}(f)$. Theoretically, ${{a}_{i}}$in forward prediction equations 
is the complex conjugation of ${{a}_{-i}}$in backward equations \cite[]{Galbraith1991}. 
\cite{Gulunay2000} pointed that it is possible to reduce the order of the normal equations 
from 2M to M because the coefficients of noncausal prediction filter have conjugate symmetry. 
f-x prediction filtering has the assumption that the events of seismic section are 
linear. If seismic events are not linear, or the amplitudes of wavelet are varying, 
they no longer follow linear or stationary assumptions (Canales, 1984). One needs to 
perform $f$-$x$ prediction filtering over a short sliding window in time and space to 
cope with continuous changes in dips \cite[]{Naghizadeh2009}. \cite{Fomel2009} 
developed a general method of RNA using shaping regularization technology, which 
is implemented for real number. \cite{Liu2012} extended the RNA method to $f$-$x$ 
domain for complex numbers and applied it to seismic random noise attenuation for 
2D seismic data. The $f$-$x$ NRNA is defined as \cite[]{Liu2012}
      \begin{equation}
          {{\varepsilon }_{n}}(f)={{S}_{n}}(f)-\sum\limits_{i=1}^{M}{{{a}_{n,i}}(f){{S}_{n-i}}(f)}-\sum\limits_{i=-1}^{-M}{{{a}_{n,i}}(f){{S}_{n-i}}(f)}.
        \label{eq:eq3}
      \end{equation}

Eq. ~\ref{eq:eq3} indicates that one trace noise-free in $f$-$x$ domain can be predicted by 
adjacent traces with the different weights ${{a}_{n,i}}(f)$. Note that the 
weights ${{a}_{n,i}}(f)$ is varying along the space direction, which 
indicated by subscript i in ${{a}_{n,i}}(f)$. In Eq. ~\ref{eq:eq3}, the coefficients $a$is 
the function of space i, but it is not in Eq. ~\ref{eq:eq2}. When applying $f$-$x$ NRNA to seismic 
noise attenuation, we assume the prediction error${{\varepsilon }_{n}}(f)$ is the 
random noise and the predictable part $\sum\limits_{i=1}^{M}{{{a}_{n,i}}(f){{S}_{n-i}}(f)}+\sum\limits_{i=-1}^{-M}{{{a}_{n,i}}(f){{S}_{n-i}}(f)}$
is the signal. Finding spatial-varying coefficients ${{a}_{n,i}}(f)$ form Eq. ~\ref{eq:eq3} 
is ill-posed problem because there are more unknown variables than constraint equations. To obtain the 
coefficients, we should add constraint equations. Shaping regularization \cite[]{Fomel2009} 
can be used to solve the under-constrained problem \cite[]{Liu2012}. The RNA method 
can also be used for seismic data processing in t-x-y domain, such as seismic data 
interpolation \cite[]{Liu2011}.

 \subsection{f-x-y NRNA for random noise attenuation}

Two dimensional $f$-$x$ NRNA only considers one space coordinate x. If we use $f$-$x$ 
NRNA on 3D seismic cube, we usually apply $f$-$x$ RNA in one space slice. $f$-$x$ NRNA 
reduces the effectiveness because the plane event in 3D cube is predictable along 
different directions rather than only one direction. Therefore, we should develop 
3D $f$-$x$-$y$ NRNA to suppress random noise for 3D seismic data.

\inputdir{.}

\plot{fig1}{width=0.5\columnwidth}{The $f$-$x$-$y$ prediction filter. The trace ${{\text{T}}_{\text{33}}}$ is predicted from circumjacent traces ${{\text{T}}_{11}}\sim {{\text{T}}_{55}}$(except itself ${{\text{T}}_{\text{33}}}$).
}



Next, we use Fig. ~\ref{fig:fig1} to illustrate the idea of $f$-$x$-$y$ NRNA. The middle trace 
${{\text{T}}_{\text{33}}}$is the one we want to predict. Trace ${{\text{T}}_{\text{33}}}$ 
can be predicted from circumjacent traces ${{\text{T}}_{11}}\sim {{\text{T}}_{55}}$(except 
itself ${{\text{T}}_{\text{33}}}$). The prediction process includes all different directions. 
For example, if we use ${{\text{T}}_{21}}$to predict ${{\text{T}}_{\text{33}}}$, we can 
estimate a corresponding coefficient using the described algorithm in the following. $f$-$x$-$y$ 
NRNA uses all around traces to predict the middle trace. Therefore, the prediction uses more 
information than $f$-$x$ NRNA. For all the traces in 3D cube, similar to the trace ${{\text{T}}_{\text{33}}}$, 
we can use circumjacent traces to predict them. Mathematically, we can write the prediction process as
      \begin{equation} 
          {{S}_{x,y}}(f)=\sum\limits_{i=-M,i\ne 0}^{M}{{{a}_{i}}(f){{S}_{x,y,i}}(f)},
        \label{eq:eq4}
      \end{equation}
where M and i are the number and index of circumjacent traces, respectively. In the case of Fig. ~\ref{fig:fig1}, 
M=24 and i is from 1 to 24. Note that ${{S}_{x,y,i}}(f)$indicates the 24 circumjacent traces around
 ${{S}_{x,y}}(f)$. Eq. ~\ref{eq:eq4} is the equations of noncausal regularized stationary autoregression. 
Similarly to $f$-$x$ NRNA, considerng the nonstationary case, we can obtain
      \begin{equation}
          {{\tilde{S}}_{x,y}}(f)=\sum\limits_{i=-M,i\ne 0}^{M}{{{a}_{x,y,i}}(f){{S}_{x,y,i}}(f)},
        \label{eq:eq5}
      \end{equation}
where ${{a}_{x,y,i}}(f)$is the space-varying coefficients, which means they have three free degrees, 
space axis x, space axis y and shift axis i. ${{\tilde{S}}_{x,y}}(f)$can be regarded as the 
estimation of noise-free signal. However, the coefficients ${{a}_{x,y,i}}(f)$ are not 
known. Once we obtain the coefficients, we can estimate the effective signal using 
Eq. ~\ref{eq:eq5} Similar to $f$-$x$ NRNA, we use shaping regularization to solve this ill-posed 
problem. Here, we assume that the coefficients ${{a}_{x,y,i}}(f)$ $f$-$x$-$y$ RNA are 
smooth along two space axes x and y, which is reasonable because the curved surface 
event in 3D seismic data is locally plane. Therefore, we can obtain the following 
least square problem with shaping regularization
      \begin{equation}
          \min_{a_{x,y,i}(f)}||{{S}_{x,y}}(f)-\sum\limits_{i=-M,i\ne 0}^{M}{{{a}_{x,y,i}}(f){{S}_{x,y,i}}(f)}||_{2}^{2}+R[{{a}_{x,y,i}}(f)],
        \label{eq:eq6}
      \end{equation}
where R[.] denotes shaping regularization term which constrains coefficients ${{a}_{x,y,i}}(f)$
to be smooth along space axes. We use one coefficient with a given frequency and a given shift 
(e.g., from ${{\text{T}}_{21}}$ to ${{\text{T}}_{33}}$ indicated by arrow in Fig. ~\ref{fig:fig1}) to 
explain the constraint in Eq. ~\ref{eq:eq6}. This 3D cube of coefficient with a given frequency and a 
given shift can be expressed as ${{a}_{x,y,{{i}_{0}}}}({{f}_{0}})$, which is smooth along 
with variables x and y. The smooth constraint of coefficients is the objective of shaping 
regularization. Finally, we use Eq. ~\ref{eq:eq6} to obtain obtain the complex coefficients of $f$-$x$-$y$ 
RNA, and use Eq. ~\ref{eq:eq5} to achieve the estimation of signal. 

Transform-base methods can also be used for seismic noise attenuation \cite[]{Ma2010}. \cite{Tang2011} 
proposed to total-variation-based curvelet shrinkage for 3D seismic data denoising 
in order to suppress nonsmooth artifacts caused by the curvelet transform. Because the $f$-$x$-$y$ 
NRNA method uses shaping regularization to solve the ill-posed inverse problem and is 
complemented in frequency domain, it has higher computation efficiency than curvelet-based methods.

\section{Synthetic examples}

\inputdir{sin}

\multiplot{2}{sin,cmp}{width=0.4\columnwidth}{Synthetic benchmark 3D cube with one curved 
surface event (a) and noisy data cube (b).}

\multiplot{2}{tsin,flt}{width=0.4\columnwidth}{(a) Travel time of the event in Fig. ~\ref{fig:sin}. 
(b) The imaginary part of $f$-$x$-$y$ NRNA coefficients at a given shift ${{a}_{x,y,{{i}_{0}}}}(f)$}

\multiplot{2}{tpre2d,tpre}{width=0.4\columnwidth}{The results of $f$-$x$ NRNA (a) and $f$-$x$-$y$ NRNA (b).}

We demonstrate the effectiveness of the proposed $f$-$x$-$y$ NRNA using two synthetic 
dataset. The first synthetic example involves only one curved surface. Fig. ~\ref{fig:sin} 
shows the synthetic dataset. Three slices of Fig. ~\ref{fig:sin} illustrate the Y=2.4 km, 
X=2.4 km and Time=1 s, respectively. The following figures in this paper have 
the same way for display. The traveltime of this surface is shifted sine 
function (Fig. ~\ref{fig:tsin}). We can find that the traveltime is not linear varying. 
Therefore, we cannot use stationary $f$-$x$-$y$ prediction filtering to estimate 
the effective signal. Fig. ~\ref{fig:cmp} is the noisy data. This curved surface event 
is greatly contaminated by random noise. We respectively use $f$-$x$ NRNA and 
f-x-y NRNA to attenuate the random noise and compare their results (Fig. ~\ref{fig:tpre2d}-~\ref{fig:tpre}). 
The SNRs of $f$-$x$ NRNA and $f$-$x$-$y$ NRNA are 0.34 and 2.4, respectively. Although 
f-x NRNA has suppressed a lot of random noise, there are still some random 
noises in the result (Fig. ~\ref{fig:tpre2d}). Compared with $f$-$x$ NRNA, $f$-$x$-$y$ NRNA gives a 
better result. The curved surface event is very clear and consistent, which 
may be easier to automatic event tracking for interpretation. Fig. ~\ref{fig:flt} shows 
the imaginary part of $f$-$x$-$y$ NRNA coefficients at a given shift ${{a}_{x,y,{{i}_{0}}}}(f)$, 
which are smooth along with space axes. From the slice with 20.833 Hz (up slice in Fig. ~\ref{fig:flt}), 
one can find that the coefficients reflect the information of traveltime or time shifts 
between circumjacent traces (Fig. ~\ref{fig:tsin}). From the frontal and lateral slices 
in Fig. ~\ref{fig:flt}, one can conclude that the coefficients are related to dips of 
events from the frontal and lateral slices in Fig. ~\ref{fig:sin}. The coefficient is 
zero if the event is horizontal (e.g. position B). The coefficients are 
respectively positive and negative if the events are upgoing (e.g. position A) 
and downgoing (e.g. position C). The estimated rusult of coefficients is 
consistent with theoretical analysis.

\inputdir{shot}

\multiplot{4}{cmp0,cmp,tpre2d,tpre}{width=0.4\columnwidth}{(a) Synthetic 3D shot gather. (b) Noisy data. 
(c) The result of $f$-$x$ RNA. (d) The result of $f$-$x$-$y$ RNA.}

 
The second synthetic example is a synthetic shot gather with four hyperbolic 
events (Fig. ~\ref{fig:cmp0}). Here, we consider anisotropy of the propagating velocity, 
so that there are intersecting events in Y slice but they are not intersecting 
in X slice (the second and third events). Comparing the results of $f$-$x$ NRNA 
(Fig ~\ref{fig:tpre2d}) and $f$-$x$-$y$ NRNA (Fig ~\ref{fig:tpre}), we can find that $f$-$x$-$y$ RNA can remove more 
noise that $f$-$x$ NRNA, especially for poor signals (for example, far offset of 
the events indicated by arrows in Fig. ~\ref{fig:cmp0}-~\ref{fig:tpre}). The SNRs of $f$-$x$ NRNA and $f$-$x$-$y$ 
NRNA are 0.95 and 2.61, respectively. Both of these synthetic examples 
demonstrate the proposed $f$-$x$-$y$ NRNA can be effectively use to attenuation 
random noise for 3D seismic data cube.

\section{Application on field poststack data}

\inputdir{real}

\plot{data}{width=0.9\columnwidth}{The 3D field data cube after time migration.}

\plot{flt-np}{width=0.9\columnwidth}{The imaginary part of $f$-$x$-$y$ NRNA coefficients at 
a given shift ${{a}_{x,y,{{i}_{0}}}}(f)$ for real dataset.}

\multiplot{3}{wi,wi-2,wi-3}{width=0.5\columnwidth}{The slice X of field data cube. (a) Original data; (b) $f$-$x$ NRNA; (c) $f$-$x$-$y$ NRNA.}

\multiplot{3}{wc,wc-2,wc-3}{width=0.5\columnwidth}{The slice Y of field data cube. (a) Original data; (b) $f$-$x$ NRNA; (c) $f$-$x$-$y$ NRNA.}

\multiplot{3}{wt,wt-2,wt-3}{width=0.5\columnwidth}{The time slice of field data cube. (a) Original data; (b) $f$-$x$ NRNA; (c) $f$-$x$-$y$ NRNA.}

The $f$-$x$-$y$ NRNA method is applied to a 3D image after time migration (Fig. ~\ref{fig:data}). 
The shallow structures are simple plane layers (above 1 s) and the deep structures 
are complex curved layers (below 1 s). We respectively apply $f$-$x$ NRNA and $f$-$x$-$y$ 
NRNA to enhance the reflectors of this 3D image cube. Fig. ~\ref{fig:flt-np} shows the imaginary 
part of $f$-$x$-$y$ NRNA coefficients at a given shift ${{a}_{x,y,{{i}_{0}}}}(f)$. 
Similar to synthetic example, the $f$-$x$-$y$ NRNA coefficients are smooth and reflect 
the information of event dips. In this example, we use M=2 for $f$-$x$-$y$ NRNA and 
M=8 for $f$-$x$ NRNA, respectively. Figs. ~\ref{fig:wi}-~\ref{fig:wi-3} and ~\ref{fig:wc}-~\ref{fig:wc-3} respectively shows the X and Y 
slices after $f$-$x$ NRNA noise attenuation and $f$-$x$-$y$ NRNA noise attenuation. We 
can find that $f$-$x$-$y$ NRNA method can give a better result than $f$-$x$ NRNA method. 
The result of $f$-$x$-$y$ NRNA has a much better lateral continuity. These two methods 
not only improve the shallow plane events evidently (e.g. 0s -0.5s), but also 
improve the deep curved surface events (e.g. the area indicated by ellipse). 
This is because these two methods both are nonstationary methods, which is 
suitable for curved events. In addition, comparing $f$-$x$ NRNA and $f$-$x$-$y$ NRNA 
methods from time slices (Fig. ~\ref{fig:wt}-~\ref{fig:wt-3}), one can also see that the $f$-$x$-$y$ NRNA 
gives more consistent result. The lateral continuity and trace-by-trace 
consistency of the reflections are crucial in structural interpretation 
of seismic data by reflection picking especially for the auto-picking 
tools of interactive interpretation systems \cite[]{Fomel2010}.

\section{Conclusions}

We have proposed a novel method for seismic noise attenuation using $f$-$x$-$y$ NRNA for 3D seismic data. 
f-x-y NRNA is the 3D extension of $f$-$x$ NRNA. By using more information to predict the seismic signal, 
the $f$-$x$-$y$ NRNA improves the denoising result for 3D seismic data. The varying coefficients of the 
f-x-y NRNA are smooth along space coordinates for a given direction. The smoothness is controlled 
by shaping regularization, which has the key parameter: the smooth radius. The smooth radius can 
be selected by user according to the smoothness of assumed coefficients. This approach does not 
require breaking the input data into local windows along space axis, although it is conceptually 
analogous to sliding spatial windows with maximum overlap. Execution time of $f$-$x$-$y$ NRNA is reduced 
by iteration inversion and shaping regularization. Synthetic and field data examples both confirm 
that the proposed $f$-$x$-$y$ RNA approach can be significantly more effective in noise attenuation and 
consistency improvement than $f$-$x$ RNA for 3D seismic data. Therefore, it may be useful in conjunction 
with interactive interpretation systems and auto-picking tools such as automatic event tracking. 

\section{Acknowledgments}

We would like to thank Yang Liu and Jingye Li for inspiring discussions. 
This research was partially supported by Science Foundation of China University 
of Petroleum, Beijing (No. YJRC-2013-39) and the National Natural Science 
Foundation of China (no. U1262207).

%\section{Note on reproducibility}

%All computational experiments described in this paper can be
%reproduced using the \emph{Madagascar} open-source software package
%available at \url{http://rsf.sourceforge.net}.


\bibliographystyle{seg}
\bibliography{paper}


