\author{Charles Hewitt Dix} 
%%%%%%%%%%%%%%%%%%%%%%%%%%%
\title{Homework 2}

\begin{abstract}
  This homework has three parts. In the theoretical part, you will
  find the geometrical amplitude of the acoustic displacement, derive
  dynamic ray tracing equations, and extend the hyperbolic
  approximation of reflection moveouts. In the computational part, you
  will experiment with a field dataset from the Gulf of Mexico. In the
  programming part, you will implement exact and approximate
  traveltime computations in a $V(z)$ medium.
\end{abstract}

\section{Prerequisites}

Completing the computational part of this homework assignment requires
\begin{itemize}
\item \texttt{Madagascar} software environment available from \\
  \url{http://www.ahay.org}
\item \LaTeX\ environment with \texttt{SEGTeX} available from \\ 
  \url{http://www.ahay.org/wiki/SEGTeX}
\end{itemize}

You are welcome to do the assignment on your personal computer by
installing the required environments. In this case, you can obtain this
homework assignment from the \texttt{Madagascar} repository by running
\begin{verbatim}
svn co https://rsf.svn.sourceforge.net/svnroot/rsf/trunk/book/geo384w/hw2
\end{verbatim}

The necessary environment is also installed in a computer lab
at the Department of Geological Sciences.

\section{Theoretical part}

\begin{enumerate}
\item In class, we derived the following acoustic wave equation for
  pressure $P(\mathbf{x},t)$:
\begin{equation}
{\frac{1}{V^2(\mathbf{x})}}\,{\frac{\partial^2 P}{\partial t^2}} =\nabla^2 P + 
\rho(\mathbf{x})\,
\nabla\left(\frac{1}{\rho(\mathbf{x})}\right) \cdot \nabla P\;.
\label{eq:pwave}
\end{equation}
\begin{enumerate}
\item Using the connection between the pressure and displacement, derive the
acoustic wave equation for the displacement vector
$\mathbf{u}(\mathbf{x},t)$:
\begin{equation}
  \label{eq:uwave}
  \rho(\mathbf{x})\,\frac{\partial^2 \mathbf{u}}{\partial t^2} =
\end{equation}
\item Consider a geometrical wave representation in the vicinity of a wavefront
\begin{equation}
  {\mathbf{u}(\mathbf{x},t)} = \mathbf{a}(\mathbf{x})\,f\left(t-T(\mathbf{x})\right)
  \label{eq:gwave}
\end{equation}
and derive partial differential equations for the traveltime function
$T(\mathbf{x})$ and the vector amplitude $\mathbf{a}(\mathbf{x})$.
\item Assuming that the geometrical wave propagates in the direction
  of the traveltime gradient
\begin{equation}
  {\mathbf{a}(\mathbf{x})} = A(\mathbf{x})\,V(\mathbf{x})\,\nabla T
  \label{eq:vamp}
\end{equation}
show that the amplitude continuation along a ray is given by equation 
\begin{equation}
  \label{eq:avj}
  \left|\mathbf{a}_1\right| = \left|\mathbf{a}_0\right|\,
  \left(\frac{\rho_0\,V_0\,J_0}{\rho_1\,V_1\,J_1}\right)^{1/2}\;, 
\end{equation}
where $J_0$ and $J_1$ are the corresponding geometrical spreading factors.
\item (EXTRA CREDIT) Consider the elastic wave equation
\begin{equation}
  \label{eq:ewave}
  \rho\,\ddot{u}_i = C_{ijkl,j}\,u_{k,l} + C_{ijkl}\,u_{k,lj}\;
\end{equation}
in the case of an isotropic elasticity
\begin{equation}
  \label{eq:lame}
  C_{ijkl} = \lambda\,\delta_{ij}\,\delta_{kl} + 
  \mu\,(\delta_{ik}\,\delta_{jl} + \delta_{il}\,\delta_{jk})\;.
\end{equation}
Using the geometrical representation~(\ref{eq:gwave}) with the P-wave
polarization given by equation~(\ref{eq:vamp}), show that the
corresponding amplitude equation is similar to
equation~(\ref{eq:avj}).
\end{enumerate}
\item Consider a medium with a constant gradient of slowness squared
  \begin{equation}
    S^2(\mathbf{x}) = S_0^2 + 2\,\mathbf{g} \cdot (\mathbf{x}-\mathbf{x}_0)\;.
  \end{equation}
\begin{enumerate}
\item In the 2-D case, the ray-family coordinate system can be
  specified by $\mathbf{r}=\{\sigma,\theta\}$, where $\sigma$ goes
  along the ray, and $\theta$ is the initial ray angle. A family of
  rays starts from the source point $\mathbf{x}_0$ which each ray
  traveling in the direction $\mathbf{p_0} =
  \{S_0\,\cos{\theta},S_0\,\sin{\theta}\}$. Show that the coordinate
  transformation matrix $\mathbf{P}=\partial \mathbf{p}/\partial
  \mathbf{r}$ changes along the ray as
  \begin{equation}
    \label{eq:p}
    \mathbf{P}(\sigma) = \left[\begin{array}{cc} 
        g_1 & -S_0\,\sin{\theta} \\
        g_2 & S_0\,\cos{\theta} \end{array}\right]\;,
  \end{equation}
  where $g_1$ and $g_2$ are the components of $\mathbf{g}$ and find
  the corresponding transformation of the matrices
  $\mathbf{X}=\partial \mathbf{x}/\partial \mathbf{r}$ and
  $\mathbf{K}=\mathbf{P}\,\mathbf{X}^{-1}$.
  \begin{eqnarray}
    \label{eq:x}
    \mathbf{X}(\sigma)  & = & \\ 
    \mathbf{K}(\sigma)  & = & 
    \label{eq:k}
    \end{eqnarray}
  \item Find the one-point geometrical spreading $J$ from a point
    source in the 2-D case as a function of $S_0$, $\mathbf{g}$, the
    source location $\mathbf{x}_0$, the initial ray
    direction~$\theta$, and the ray coordinate $\sigma$.
  \item Using analytical ray tracing solutions, find the two-point
    geometrical spreading $J$ from a point source in the 2-D case as a
    function of $S_0$, $\mathbf{g}$, the source location
    $\mathbf{x}_0$ and the receiver location $\mathbf{x}_1$.
  \end{enumerate}
\item In class, we discussed the hyperbolic traveltime approximation for normal moveout
  \begin{equation}
  \label{eq:hyper2}
  T(h) \approx \sqrt{T_0^2 + \frac{h^2}{V_0^2}}\;.
\end{equation}
More accurate approximations, involving additional parameters, are possible.
\begin{enumerate}
\item Consider the following three-parameter approximation
  \begin{equation}
    \label{eq:taner}
    T(h) \approx T_0\,\left(1-\frac{1}{S}\right) + 
\frac{1}{S}\,\sqrt{T_0^2+S\,\frac{h^2}{V_0^2}}\;,
  \end{equation}
  where $S$ is the so-called ``heterogeneity'' parameter. 

Evaluate parameter $S$ in terms of the velocity $V(z)$ and the reflector depth $z_0$.
\begin{equation}
  \label{eq:a}
  S =
\end{equation}
by expanding
  equation~(\ref{eq:taner}) in a Taylor series around the zero offset
  $h=0$ and comparing it with the corresponding Taylor series of the exact
  traveltime. The exact traveltime is given by the parametric equations
  \begin{eqnarray}
\label{eq:xofp}
h & = & \int\limits_0^{z_0} \frac{p\,V(z)\,dz}{\sqrt{1-p^2 V^2(z)}}\;, \\
\label{eq:tofp}
T & = & \int\limits_0^{z_0} \frac{dz}{V(z)\,\sqrt{1-p^2 V^2(z)}}\;.
\end{eqnarray}
\item Let $\tau = T-p\,h$. Show that $\tau$ can be approximated to the same accuracy by
\begin{equation}
    \label{eq:taner}
    \tau(p) \approx \tau_0\,\left(1-\frac{1}{S_{\tau}}\right) + 
\frac{\tau_0}{S_{\tau}}\,\sqrt{1-S_{tau}\,{V_{\tau}^2}}\;.
  \end{equation}
Find $\tau_0$, $V_{\tau}$, and $S_{\tau}$.
\end{enumerate}
\end{enumerate}

\lstset{language=python,numbers=left,numberstyle=\tiny,showstringspaces=false}

\newpage


\section{Computational part}
\inputdir{cmp}

In the computational part, we begin working with field data. The left
panel in Figure~\ref{fig:cmp} shows a CMP (common midpoint) gather
from the Gulf of Mexico \cite[]{bei}.

\plot{cmp}{width=\textwidth}{From left to right: (a) CMP (common midpoint) gather 
with overlaid traveltime curves. (b) Interval velocity. (c) RMS
(root-mean-square) velocity overlaid on the semblance scan. (d) CMP
gather after normal moveout.}
 
We will assume a $V(z)$ medium and will use a very simple model of the
interval velocity to explain the geometry of the observed data. The
model involves two parameters: the initial gradient of velocity and
the maximum velocity. The velocity function starts at the water
velocity of~1.5 km/s and grows linearly with vertical time until it
reaches the maximum velocity, after which point it remains flat. The
panels in Figure~\ref{fig:cmp} show the interval velocity, the
corresponding RMS velocity (overlaid on the semblance scan), and the
CMP gather after NMO (normal moveout). 

Your task is to find the best values of the two model parameters for
optimal prediction of the traveltime geometry and for flattening the
CMP gather after NMO.

\begin{enumerate}
\item Change directory 
\begin{verbatim}
cd hw2/cmp
\end{verbatim}
\item Run
\begin{verbatim}
scons cmps.vpl
\end{verbatim}
to generate and display a movie looping through different values of
the maximum velocity. If you are on a computer with multiple CPUs, you
can also try
\begin{verbatim}
pscons cmps.vpl
\end{verbatim}
to generate different movie frames faster by running computations in
parallel.
\item Edit the \texttt{SConstruct} file to modify the velocity gradient. Check your result by running
\begin{verbatim}
pscons cmps.vpl
\end{verbatim}
again. 
\item Edit the \texttt{SConstruct} file to select the 
best frame of the movie (corresponding to the best maximum velocity). Display it by running
\begin{verbatim}
scons view
\end{verbatim}
\end{enumerate}

{\small \lstinputlisting[frame=single]{cmp/SConstruct}}

\newpage


\section{Programming part}

\sideplot{time}{width=\textwidth}{Traveltime in a $V(z)$ medium.}

The program \texttt{cmp/traveltime.c} computes reflection traveltimes
in a $V(z)$ medium by using different methods. 

\begin{enumerate}
\item Modify the program to implement approximation~(\ref{eq:taner}) using your equation~(\ref{eq:a}).
\item Modify the program to implement exact traveltime computation by doing shooting iterations with equations~(\ref{eq:xofp}-\ref{eq:tofp}). 
Using Newton's method, you can find the value of $p$ for a given $h$ by solving the non-linear equation $h(p)=h$ with iterations
\begin{equation}
\label{eq:newton}
p_{n+1} = p_n - \frac{h(p_n)-h}{h'(p_n)}\;.
\end{equation}
\item For the traveltime in Figure~\ref{fig:time}, find the offset, where the absolute error of the hyperbolic approximation~(\ref{eq:hyper2}) exceeds~0.1 s. 
\item For the traveltime in Figure~\ref{fig:time}, find the offset, where the absolute error of the nonhyperbolic approximation~(\ref{eq:taner}) exceeds~0.1 s. 
\end{enumerate}


\lstset{language=c,numbers=left,numberstyle=\tiny,showstringspaces=false}
\lstinputlisting[frame=single]{cmp/traveltime.c}

\newpage

\section{Completing the assignment}

\begin{enumerate}
\item Change directory to \verb#hw2#.
\item Edit the file \texttt{paper.tex} in your favorite editor and change the
  first line to have your name instead of Dix's.
\item Run
\begin{verbatim}
sftour scons lock
\end{verbatim}
to update all figures.
\item Run
\begin{verbatim}
sftour scons -c
\end{verbatim}
to remove intermediate files.
\item Run
 \begin{verbatim} 
  scons pdf
\end{verbatim}
  to create the final document.
\item Submit your result (file \texttt{paper.pdf}) on paper or by
  e-mail. 
\end{enumerate}

\bibliographystyle{seg} 
\bibliography{SEG,hw3}
