\author{Pierre de Fermat} 
%%%%%%%%%%%%%%%%%%%%%%
\title{Homework 2}

\begin{abstract}
  This homework has two parts. In the theoretical part, you will
  derive special analytical solutions to one-point and two-point ray
  tracing problems. In the computational part, you will investigate
  the numerical accuracy of a finite-difference eikonal solver and
  a Runge-Kutta ray tracer.
\end{abstract}

\section{Prerequisites}

Completing the computational part of this homework assignment requires
\begin{itemize}
\item \texttt{Madagascar} software environment available from \\
  \url{http://www.ahay.org}
\item \LaTeX\ environment with \texttt{SEGTeX} available from \\ 
  \url{http://www.ahay.org/wiki/SEGTeX}
\end{itemize}

You are welcome to do the assignment on your personal computer by
installing the required environments. In this case, you can obtain all
homework assignments from the \texttt{Madagascar} repository by running
\begin{verbatim}
svn co https://rsf.svn.sourceforge.net/svnroot/rsf/trunk/book/geo384w 
\end{verbatim}

\section{Theoretical part}
\inputdir{XFig}

You can either write your answers on paper or edit them in the file
\verb#hw2/paper.tex#. Please show all the mathematical
derivations that you perform.

\begin{enumerate}
\item In class, we derived analytical solutions for one-point and
two-point ray tracing problems for the special case of a constant
gradient of slowness squared 
\begin{equation}
  \label{eq:s2}
  S^2(\mathbf{x}) =
  S^2(\mathbf{x_0})+2\,\mathbf{g} \cdot (\mathbf{x}-\mathbf{x_0})\;.
\end{equation}
In this homework, you will consider another special case, that of a
constant gradient of velocity
\begin{equation}
  V(\mathbf{x}) = \frac{1}{S(\mathbf{x})} =
  {V(\mathbf{x_0})+\mathbf{G} \cdot (\mathbf{x}-\mathbf{x_0})}\;.
  \label{eq:v}
\end{equation}
Recall the ray tracing system from Homework 1
\begin{eqnarray}
  \label{eq:plin}
  \frac{d\,\mathbf{p}}{d\,\xi} & = & -\nabla V\;, \\
  \label{eq:xlin}
  \frac{d\,\mathbf{x}}{d\,\xi} & = &
  \mathbf{p}\,V^3(\mathbf{x})\;, \\
  \label{eq:tlin}
  \frac{d\,T}{d\,\xi} & = & V(\mathbf{x})\;.
\end{eqnarray}
and consider the one-point ray tracing problem with the initial conditions
$\mathbf{x}(0)= \mathbf{x}_0$ and $\mathbf{p}(0)=\mathbf{p}_0$.
\begin{enumerate}
\item  Show that the solution of
equation~(\ref{eq:plin}) for the constant gradient of velocity is
\begin{equation}
  \label{eq:pcir}
  \mathbf{p}(\xi) =  \mathbf{p}_0 - \mathbf{G}\,\xi\;.
\end{equation}
and express velocity along the ray as a function of
  $\mathbf{p}_0$, $\mathbf{G}$, and $\xi$:
  \begin{equation}
\label{eq:vsol}
V(\mathbf{x}) = \frac{1}{\sqrt{\mathbf{p} \cdot \mathbf{p}}} = 
\end{equation}
\item Let $a = \mathbf{p} \cdot (\mathbf{x} - \mathbf{x}_0)$. Using
  the chain rule, find the expression for
  \begin{equation}
    \frac{d\,a}{d\,\xi} = 
  \end{equation}
  and solve it to show it that 
  \begin{equation}
    \label{eq:asol}
    a(\xi) = V(\mathbf{x}_0)\,\xi
  \end{equation}
  and
  \begin{equation}
    \label{eq:a0sol}
    a_0(\xi) = \mathbf{p}_0 \cdot (\mathbf{x} - \mathbf{x}_0) = V(\mathbf{x})\,\xi
  \end{equation}
\item One way to seek the solution for the one-point ray tracing
  problem is to look for scalars $\alpha$ and $\beta$ in the representation
\begin{equation}
  \label{eq:xsol}
  \mathbf{x}(\xi) = \mathbf{x}_0 + \alpha(\xi)\,\mathbf{p}_0 + \beta(\xi)\,\mathbf{G}\;.
\end{equation}
Under what condition does the linear system of equations
\begin{eqnarray}
  \label{eq:linsys1}
  V(\mathbf{x})\,\xi = \mathbf{p}_0 \cdot (\mathbf{x} - \mathbf{x}_0) & = & 
  \alpha\,\mathbf{p}_0 \cdot \mathbf{p}_0 + \beta\,\mathbf{p}_0 \cdot \mathbf{G} \\
  \label{eq:linsys2}
  V(\mathbf{x}) - V(\mathbf{x}_0) = \mathbf{G} \cdot (\mathbf{x} - \mathbf{x}_0) & = &
  \alpha\,\mathbf{p}_0 \cdot \mathbf{G} + \beta\,\mathbf{G} \cdot \mathbf{G}
\end{eqnarray}
have a unique solution for $\alpha$ and $\beta$? Solve the system to
find $\alpha$ and $\beta$ and obtain an analytical expression for
the ray trajectory $\mathbf{x}(\xi)$.
\item Express the squared distance between the ray end points
  \begin{eqnarray}
  \nonumber
    (\mathbf{x} - \mathbf{x}_0) \cdot (\mathbf{x} - \mathbf{x}_0) & = &
    \alpha^2\,\mathbf{p}_0 \cdot \mathbf{p}_0 + 2 \alpha\,\beta\,\mathbf{p}_0 \cdot \mathbf{G} +
    \beta^2\,\mathbf{G} \cdot \mathbf{G} \\
    & = & 
    \label{eq:d2} 
  \end{eqnarray}
  in terms of $\mathbf{G}$, $\mathbf{p}_0$, and $\xi$. 
\item In the two-point problem, the unknown
  parameters are $(\mathbf{p}_0 \cdot \mathbf{G})$ and $\xi$.
  Express $(\mathbf{p}_0 \cdot \mathbf{G})$ from your
  equation~(\ref{eq:vsol}) and substitute it into your
  equation~(\ref{eq:d2}). Solve for $\xi$. 
\item Finally, use $\xi$ and $(\mathbf{p}_0 \cdot \mathbf{G})$ 
  expressed in terms of $|\mathbf{x} - \mathbf{x}_0|$,
  $\mathbf{G}$, $V(\mathbf{x}_0)$, and $V(\mathbf{x})$ and
  substitute them into the one-point traveltime solution obtained by
  integrating equation~(\ref{eq:tlin})\footnote{$\mbox{arccosh}(x)$ is
  the inverse hyperbolic cosine function defined as $\mbox{arccosh}(x)
  = \ln\left(x + \sqrt{x^2-1}\right)$.}  
  \begin{equation}
  \label{eq:t1} T(\xi) =
  \frac{1}{|\mathbf{G}|}\,\mbox{arccosh}\left(1 +
  \frac{|\mathbf{G}|^2\,V(\mathbf{x})\,V^2(\mathbf{x}_0)\,\xi^2}
  {V(\mathbf{x})+V(\mathbf{x}_0) - (\mathbf{p}_0 \cdot
  \mathbf{G})\,V(\mathbf{x})\,V^2(\mathbf{x}_0)\,\xi}\right)\;.
  \end{equation} 
  Your result will be the analytical two-point
  traveltime 
  \begin{eqnarray} 
  \label{eq:t2}
    \widehat{T}(\mathbf{x}_0,\mathbf{x}) = & & \hfill \ 
  \end{eqnarray}
\end{enumerate}

\item Consider the source point $s$ and the receiver point $r$ at the
  surface $z=0$ above a 2-D constant-velocity medium and a curved
  reflector defined by the equation $z = z(x)$ with a twice
  differentiable function $z(x)$ (Figure~\ref{fig:curve}).

  \plot{curve}{width=0.7\textwidth}{Geometry of reflection in a
    constant-velocity medium with a curved reflector.}

  Note that the two-point ray trajectory can be parametrized by the
  reflection point $y$ with the following expression for the
  reflection traveltime (using the Pythagoras theorem):
  \begin{equation}
      T(s,r) = {\frac{\sqrt{(s-y)^2+z^2(y)}+\sqrt{(r-y)^2+z^2(y)}}{V}}\;.
      \label{eq:dsr}
  \end{equation}

  \begin{enumerate}
  \item Apply Fermat's principle to specify $F(y)$ in the equation
    \begin{equation}
	\label{eq:fermat}
	    F(y) = 0
      \end{equation}
      required for finding the reflection point $y$.
      \item Newton's method (successive linearization) solves nonlinear
    equations like~(\ref{eq:fermat}) iteratively by starting with some
    $y_0$ and repeating the iteration
      \begin{equation}
      \label{eq:newton}
        y_{n+1} = y_n - \frac{F(y_n)}{F'(y_n)}
      \end{equation}
      for $n=0,1,2,\ldots$ Specify $F'(y)$ for the problem of finding the reflection point.

    \item Consider the special case of a dipping plane boundary $z(x)
    = x\,\tan{\alpha}$, where $\alpha$ is the dip angle. Show that, in
    this case, equation~(\ref{eq:fermat}) reduces to a linear equation
    for $y$ and has the following solution:

    \begin{equation}
      y = {\frac{2\,r\,s\,\cos^2\,\alpha}{r + s}}\;.
      \label{eq:y}
    \end{equation}
    Substitute equation~(\ref{eq:y}) into~(\ref{eq:dsr}) to find the 
    reflection traveltime.
  \end{enumerate}
\end{enumerate}

\section{Computational part}

\begin{enumerate}

\item In the first part of the computational assignment, you will
  investigate the numerical accuracy of a finite-difference eikonal solver.

  \inputdir{eikonal}

\lstset{language=python,numbers=left,numberstyle=\tiny,showstringspaces=false}
\lstinputlisting[frame=single]{eikonal/SConstruct}

  Figure~\ref{fig:exact} shows wavefronts in a medium with a constant
  gradient of slowness squared computed with an analytical two-point
  ray tracing formula from the class.
 
  \plot{exact}{width=\textwidth}{Wavefronts in a constant velocity
    gradient medium computed with an analytical formula.}

  By computing traveltimes numerically at different sampling intervals
  and comparing the numerical result with the analytical one, we can
  get an experimental estimate of the numerical error behavior. The
  error is shown in Figure~\ref{fig:err}. The right plot in
  Figure~\ref{fig:err} displays the error in logarithmic coordinates. The
  slope of the line in these coordinates shows directly the rate of
  convergence of the numerical method. For example, a first-order
  accurate method should have a slope of one.

  \plot{err}{width=\textwidth}{Left: average error of the
    finite-difference eikonal solver as a function of grid spacing.
    Right: the same on a log-log plot. The slope of the curve on the
    log-log plot indicates the order of the numerical accuracy.}

  \begin{enumerate}
  \item Change directory
\begin{verbatim}
> cd ~/geo384w/hw2/eikonal
\end{verbatim}
  \item Run
\begin{verbatim}
> scons view
\end{verbatim}
    to generate figures and display them on your screen.  

  \item In the \texttt{SConstruct} file, find the parameter that
    defines the order of accuracy for the eikonal solver. Change the
    order from $1$ to $2$ and recompute the results. Does the
    numerical accuracy change? What is the experimental order of
    accuracy? 
  \item Instead of an analytical solution for a constant 
    gradient of slowness squared, let us try an analytical solution for a constant
    gradient of velocity. 
    \begin{itemize}
    \item Uncomment the part of the
    \texttt{SConstruct} file that defines a velocity model with the
    constant velocity gradient. 
    \item Modify the program \texttt{analytical.c} to implement your equation~(\ref{eq:t2}).
    \item Recompute the figures and check your results.
    \end{itemize}
  \end{enumerate}

\lstset{language=c,numbers=left,numberstyle=\tiny,showstringspaces=false}
\lstinputlisting[frame=single]{eikonal/analytical.c}


\item In the second-part of the computational assignment, you will investigate the numerical accuracy of one-point ray tracing.

\inputdir{raytracing}

\lstset{language=python,numbers=left,numberstyle=\tiny,showstringspaces=false}
\lstinputlisting[frame=single]{raytracing/SConstruct}

  Figure~\ref{fig:ray} shows the same constant velocity gradient model
  and one ray computed by analytical one-point ray tracing. The ray has the shape of a circular arc.

  \plot{ray}{width=\textwidth}{Analytical ray tracing in a constant
    velocity gradient medium. A single ray is displayed.}

  Figure~\ref{fig:rayerr} displays errors of numerical ray tracing for
  different sizes of the time step. 

  \sideplot{rayerr}{width=\textwidth}{Average error of the
    one-point ray tracer as a function of the time step size.}

    \begin{enumerate}
  \item Change directory 
\begin{verbatim}
> cd ~/geo391/hw2/raytracing
\end{verbatim}
  \item Run
\begin{verbatim}
> scons view
\end{verbatim}
    to generate figures and display them on your screen.  
  \item Edit the \texttt{SConstruct} file to plot the numerical error
    on a log-log scale similarly to Figure~\ref{fig:err}. What is the
    order of accuracy of this numerical method?
  \end{enumerate}
  
\end{enumerate}

\newpage

\section{Completing the assignment}

\begin{enumerate}
\item Change directory to \verb#~/geo384w/hw2#.
\item Edit the file \texttt{paper.tex} in your favorite editor and change the
  first line to have your name instead of Fermat's.
\item Run
\begin{verbatim}
  > sftour scons lock
\end{verbatim}
to update all figures.
\item Run
\begin{verbatim}
  > sftour scons -c
\end{verbatim}
  to remove intermediate files.
\item Run
 \begin{verbatim} 
  > scons pdf
\end{verbatim}
  to create the final document.
\item Submit your result (file \texttt{paper.pdf}) on paper or by
  e-mail. 
\end{enumerate}
