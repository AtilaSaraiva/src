\author{Pierre de Fermat} 
%%%%%%%%%%%%%%%%%%%%%%
\title{Homework 2}

\begin{abstract}
  This homework has two parts. In the theoretical part, you will
  derive special analytical solutions to a point-to-plane and
  two-point ray tracing problems. In the computational part, you will
  investigate the numerical accuracy of a finite-difference eikonal solver.
\end{abstract}

\section{Prerequisites}

Completing the computational part of this homework assignment requires
\begin{itemize}
\item \texttt{Madagascar} software environment available from \\
\url{http://rsf.sourceforge.net}
\item \LaTeX\ environment with \texttt{SEGTeX} available from \\ 
\url{http://segtex.sourceforge.net}
\end{itemize}

You are welcome to do the assignment on your personal computer by
installing the required environments. In this case, you can obtain all
homework assignments from the \texttt{Madagascar} repository by running
\begin{verbatim}
svn co https://rsf.svn.sourceforge.net/svnroot/rsf/trunk/book/geo384w 
\end{verbatim}

The necessary environment is also installed in your class account on
Unix computers in the Department of Geological Sciences.

\section{Theoretical part}
\inputdir{XFig}

You can either write your answers on paper or edit them in the file
\verb#hw2/paper.tex#. Please show all the mathematical
derivations that you perform.

\begin{enumerate}
\item In class, we derived analytical solutions for one-point and
two-point ray tracing problems for the special case of a constant
gradient of slowness squared 
\begin{equation}
  \label{eq:s2}
  S^2(\mathbf{x}) =
  S^2(\mathbf{x_0})+2\,\mathbf{g} \cdot (\mathbf{x}-\mathbf{x_0})\;.
\end{equation}
The one-point solution has the form
\begin{eqnarray}
  \label{eq:xgrad}
  \mathbf{x}(\sigma) & = & \mathbf{x}_0 + \mathbf{p}_0\,\sigma
  + \mathbf{g}\,\frac{\sigma^2}{2} \\
  \label{eq:pgrad}
  \mathbf{p}(\sigma) & = & \mathbf{p}_0 + \mathbf{g}\,\sigma\;. \\
  \label{eq:tgrad}
  T(\sigma) & = & S^2(\mathbf{x_0})\,\sigma +
  \mathbf{p}_0 \cdot \mathbf{g}\,\sigma^2 +  
  \mathbf{g} \cdot \mathbf{g}\,\frac{\sigma^3}{3}\;.
\end{eqnarray}

Using the same model and assuming non-zero gradient $\mathbf{g}$,
consider a \emph{point-to-plane} ray tracing problem. That is, find an
analytical expression for the traveltime from point~$\mathbf{x}_0$ to
the point where the ray arrives with the slowness vector
$\mathbf{p}_1$. Your solution should express the traveltime $T$ in
terms of $S^2(\mathbf{x_0})$, $\mathbf{g}$, and $\mathbf{p}_1$.  How
many different branches does this solution have?

\item Consider the source point $s$ and the receiver point $r$ at the
  surface $z=0$ above a 2-D constant-velocity medium and a plane
  reflector defined by the equation $z(x) = x \tan{\alpha}$, where
  $\alpha$ is the dip angle (Figure~\ref{fig:plane}). 

  \plot{plane}{width=0.7\textwidth}{Geometry of reflection in a
    constant-velocity medium with a plane interface (a scheme).}

  \begin{enumerate}
  \item Apply Fermat's principle to derive the following expression for the
    horizontal coordinate of the reflection point $y$:
    \begin{equation}
      \label{eq:y}
      y = \frac{2\,r\,s\,\cos^2\,\alpha}{r + s}\;.
    \end{equation}
  \item Assuming the velocity above the reflector is $V$, find the
    reflection traveltime.
  \end{enumerate}
\end{enumerate}

\section{Computational part}

\lstset{language=python,numbers=left,numberstyle=\tiny,showstringspaces=false}

\begin{enumerate}

\item In the first part of the computational assignment, you will
  investigate the numerical accuracy of a finite-difference eikonal solver.

  \inputdir{eikonal}
  {\small
    \lstinputlisting[frame=single]{eikonal/SConstruct}
  }

  Figure~\ref{fig:exact} shows wavefronts in a constant velocity
  gradient medium computed with an analytical two-point ray tracing
  formula.
 
  \plot{exact}{width=\textwidth}{Wavefronts in a constant velocity
    gradient medium computed with an analytical formula.}

  By computing traveltimes numerically at different sampling intervals
  and comparing the numerical result with the analytical one, we can
  get an experimental estimate of the numerical error behavior. The
  error is shown in Figure~\ref{fig:err}. The right plot in
  Figure~\ref{fig:err} displays the error in logarithmic coordinates. The
  slope of the line in these coordinates shows directly the rate of
  convergence of the numerical method. For example, a first-order
  accurate method should have a slope of one.

  \plot{err}{width=\textwidth}{Left: average error of the
    finite-difference eikonal solver as a function of grid spacing.
    Right: the same on a log-log plot. The slope of the curve on the
    log-log plot indicates the order of the numerical accuracy.}

  \begin{enumerate}
  \item Change directory
\begin{verbatim}
> cd ~/geo384w/hw2/eikonal
\end{verbatim}
  \item Run
\begin{verbatim}
> scons view
\end{verbatim}
    to generate figures and display them on your screen.  

  \item In the \texttt{SConstruct} file, find the parameter that
    defines the order of accuracy for the eikonal solver. Change the
    order from $1$ to $2$ and recompute the results. Does the
    numerical accuracy change? What is the experimental order of
    accuracy? 
  \item Instead of an analytical solution for a constant velocity
    gradient, let us try an analytical solution for a constant
    gradient of slowness squared. Uncomment the part of the
    \texttt{SConstruct} file that defines a velocity model with the
    constant gradient of slowness squared. Change the analytical
    solution for the two-point traveltime to an appropriate formula.
    Recompute the figures and check your results.
  \end{enumerate}

\item In many cases, no analytical solution exists for measuring the
  accuracy of your numerical scheme. You can still get an estimate of
  the error by computing the solution at different grid sizes and
  comparing them with computations at the finest scale.

  \inputdir{sigsbee}

  {\small
    \lstinputlisting[frame=single]{sigsbee/SConstruct}
  }

  Figure~\ref{fig:eiko} shows numerical wavefronts from a
  finite-difference solution of the eikonal equation in the Sigsbee
  velocity model.

  \plot{eiko}{width=\textwidth}{Wavefronts in the Sigsbee velocity
    model computed with a finite-difference eikonal solver.}

  \begin{enumerate}    
  \item Change directory 
\begin{verbatim}
> cd ~/geo384w/hw2/sigsbee
\end{verbatim}
  \item Run
\begin{verbatim}
> scons view
\end{verbatim}
    to generate figures and to display them on your screen.
  
  \item Edit the \texttt{SConstruct} file to include subsampling of
    the velocity model and to plot numerical error versus sampling
    similar to the previous examples. Can you figure out the
    experimental order of numerical accuracy in this case?
    
  \item Include your figure in \verb#~/geo384w/hw2/paper.tex# following
    an analogy with the previous example.

  \end{enumerate}
  
\end{enumerate}

\section{Completing the assignment}

\begin{enumerate}
\item Change directory to \verb#~/geo384w/hw2#.
\item Edit the file \texttt{paper.tex} in your favorite editor and change the
  first line to have your name instead of Fermat's.
\item Run
\begin{verbatim}
  > sftour scons lock
\end{verbatim}
to update all figures.
\item Run
\begin{verbatim}
  > sftour scons -c
\end{verbatim}
  to remove intermediate files.
\item Run
 \begin{verbatim} 
  > scons pdf
\end{verbatim}
  to create the final document.
\item Submit your result (file \texttt{paper.pdf}) on paper or by
  e-mail. If you do your assignment on one of the computers in the
  Unix lab, you can simply leave the file in your directory.
\end{enumerate}
