\author{Leonhard Euler} 
%%%%%%%%%%%%%%%%%%%%%%
\title{Homework 2}

\begin{abstract}
  This homework has three parts. In the theoretical part, you will
  derive analytical solutions to one-point and two-point ray tracing
  problems. In the computational part, you will experiment with
  traveltime computations with an eikonal solver and a field dataset
  from the Gulf of Mexico. In the programming part, you will implement
  your analytical solutions and investigate their accuracy.
\end{abstract}

Completing the computational part of this homework assignment requires
\begin{itemize}
\item \texttt{Madagascar} software environment available from \\
  \url{http://www.ahay.org}
\item \LaTeX\ environment with \texttt{SEGTeX} available from \\ 
  \url{http://www.ahay.org/wiki/SEGTeX}
\end{itemize}

You are welcome to do the assignment on your personal computer by
installing the required environments. In this case, you can obtain this
homework assignment from the \texttt{Madagascar} repository by running
\begin{verbatim}
svn co http://svn.code.sf.net/p/rsf/code/trunk/book/geo384w/hw2
\end{verbatim}

\section{Theoretical part}

You can either write your answers on paper or edit them in the file
\verb#hw2/paper.tex#. Please show all the mathematical
derivations that you perform.

\begin{enumerate}

\item In class, we derived analytical solutions for one-point and
two-point ray tracing problems for the special case of a constant
gradient of slowness squared 
\begin{equation}
  \label{eq:s2}
  S^2(\mathbf{x}) =
  S^2(\mathbf{x_0})+2\,\mathbf{g} \cdot (\mathbf{x}-\mathbf{x_0})\;.
\end{equation}
In this homework, you will consider another special case, that of a
constant gradient of velocity
\begin{equation}
  V(\mathbf{x}) = {\frac{1}{S(\mathbf{x})}} =
  {V(\mathbf{x_0})+\mathbf{G} \cdot (\mathbf{x}-\mathbf{x_0})}\;.
  \label{eq:v}
\end{equation}
Recall the ray tracing system from Homework 1
\begin{eqnarray}
  \label{eq:plin}
  \frac{d\,\mathbf{p}}{d\,\xi} & = & -\nabla V\;, \\
  \label{eq:xlin}
  \frac{d\,\mathbf{x}}{d\,\xi} & = &
  \mathbf{p}\,V^3(\mathbf{x})\;, \\
  \label{eq:tlin}
  \frac{d\,T}{d\,\xi} & = & V(\mathbf{x})\;.
\end{eqnarray}
and consider the one-point ray tracing problem with the initial conditions
$\mathbf{x}(0)= \mathbf{x}_0$ and $\mathbf{p}(0)=\mathbf{p}_0$.
\begin{enumerate}
\item  Show that the solution of
equation~(\ref{eq:plin}) for the constant gradient of velocity is
\begin{equation}
  \label{eq:pcir}
  \mathbf{p}(\xi) =  \mathbf{p}_0 - \mathbf{G}\,\xi\;.
\end{equation}
and express velocity along the ray as a function of
  $\mathbf{p}_0$, $\mathbf{G}$, and $\xi$:
  \begin{equation}
\label{eq:vsol}
V(\mathbf{x}) = \frac{1}{\sqrt{\mathbf{p} \cdot \mathbf{p}}} = 
\end{equation}
\item Let $a = \mathbf{p} \cdot (\mathbf{x} - \mathbf{x}_0)$. Using
  the chain rule, find the expression for
  \begin{equation}
    \frac{d\,a}{d\,\xi} = 
  \end{equation}
  and solve it to show it that 
  \begin{equation}
    \label{eq:asol}
    a(\xi) = V(\mathbf{x}_0)\,\xi
  \end{equation}
  and
  \begin{equation}
    \label{eq:a0sol}
    a_0(\xi) = \mathbf{p}_0 \cdot (\mathbf{x} - \mathbf{x}_0) = V(\mathbf{x})\,\xi
  \end{equation}
\item One way to seek the solution for the one-point ray tracing
  problem is to look for scalars $\alpha$ and $\beta$ in the representation
\begin{equation}
  \label{eq:xsol}
  \mathbf{x}(\xi) = \mathbf{x}_0 + \alpha(\xi)\,\mathbf{p}_0 + \beta(\xi)\,\mathbf{G}\;.
\end{equation}
Under what condition does the linear system of equations
\begin{eqnarray}
  \label{eq:linsys1}
  V(\mathbf{x})\,\xi = \mathbf{p}_0 \cdot (\mathbf{x} - \mathbf{x}_0) & = & 
  \alpha\,\mathbf{p}_0 \cdot \mathbf{p}_0 + \beta\,\mathbf{p}_0 \cdot \mathbf{G} \\
  \label{eq:linsys2}
  V(\mathbf{x}) - V(\mathbf{x}_0) = \mathbf{G} \cdot (\mathbf{x} - \mathbf{x}_0) & = &
  \alpha\,\mathbf{p}_0 \cdot \mathbf{G} + \beta\,\mathbf{G} \cdot \mathbf{G}
\end{eqnarray}
have a unique solution for $\alpha$ and $\beta$? Solve the system to
find $\alpha$ and $\beta$ and obtain an analytical expression for
the ray trajectory $\mathbf{x}(\xi)$.
\item Express the squared distance between the ray end points
  \begin{eqnarray}
  \nonumber
    (\mathbf{x} - \mathbf{x}_0) \cdot (\mathbf{x} - \mathbf{x}_0) & = &
    \alpha^2\,\mathbf{p}_0 \cdot \mathbf{p}_0 + 2 \alpha\,\beta\,\mathbf{p}_0 \cdot \mathbf{G} +
    \beta^2\,\mathbf{G} \cdot \mathbf{G} \\
    & = & 
    \label{eq:d2} 
  \end{eqnarray}
  in terms of $\mathbf{G}$, $\mathbf{p}_0$, and $\xi$. 
\item In the two-point problem, the unknown
  parameters are $(\mathbf{p}_0 \cdot \mathbf{G})$ and $\xi$.
  Express $(\mathbf{p}_0 \cdot \mathbf{G})$ from your
  equation~(\ref{eq:vsol}) and substitute it into your
  equation~(\ref{eq:d2}). Solve for $\xi$. 
\item Finally, use $\xi$ and $(\mathbf{p}_0 \cdot \mathbf{G})$ 
  expressed in terms of $|\mathbf{x} - \mathbf{x}_0|$,
  $\mathbf{G}$, $V(\mathbf{x}_0)$, and $V(\mathbf{x})$ and
  substitute them into the one-point traveltime solution obtained by
  integrating equation~(\ref{eq:tlin})\footnote{$\mbox{arccosh}(x)$ is
  the inverse hyperbolic cosine function defined as $\mbox{arccosh}(x)
  = \ln\left(x + \sqrt{x^2-1}\right)$.}  
  \begin{equation}
  \label{eq:t1} T(\xi) =
  \frac{1}{|\mathbf{G}|}\,\mbox{arccosh}\left(1 +
  \frac{|\mathbf{G}|^2\,V(\mathbf{x})\,V^2(\mathbf{x}_0)\,\xi^2}
  {V(\mathbf{x})+V(\mathbf{x}_0) - (\mathbf{p}_0 \cdot
  \mathbf{G})\,V(\mathbf{x})\,V^2(\mathbf{x}_0)\,\xi}\right)\;.
  \end{equation} 
  Your result will be the analytical two-point
  traveltime 
  \begin{eqnarray} 
  \label{eq:t2}
    \widehat{T}(\mathbf{x}_0,\mathbf{x}) = & & \hfill \ 
  \end{eqnarray}
\end{enumerate}

\item In class, we discussed the hyperbolic traveltime approximation for normal moveout
  \begin{equation}
  \label{eq:hyper2}
  T(h) \approx \sqrt{T_0^2 + \frac{h^2}{V_0^2}}\;.
\end{equation}
More accurate approximations, involving additional parameters, are possible.
\begin{enumerate}
\item Consider the following three-parameter approximation
  \begin{equation}
    \label{eq:shifted}
    T(h) \approx T_0\,\left(1-\frac{1}{S}\right) + 
\frac{1}{S}\,\sqrt{T_0^2+S\,\frac{h^2}{V_0^2}}\;,
  \end{equation}
  where $S$ is the so-called ``heterogeneity'' parameter. 

Evaluate parameter $S$ in terms of the velocity $V(z)$ and the reflector depth $z_0$.
\begin{equation}
  \label{eq:a}
  S =
\end{equation}
by expanding
  equation~(\ref{eq:shifted}) in a Taylor series around the zero offset
  $h=0$ and comparing it with the corresponding Taylor series of the exact
  traveltime. The exact traveltime is given by the parametric equations
  \begin{eqnarray}
\label{eq:xofp}
h & = & \int\limits_0^{z_0} \frac{p\,V(z)\,dz}{\sqrt{1-p^2 V^2(z)}}\;, \\
\label{eq:tofp}
T & = & \int\limits_0^{z_0} \frac{dz}{V(z)\,\sqrt{1-p^2 V^2(z)}}\;.
\end{eqnarray}
\item Let $\tau = T-p\,h$. Show that $\tau$ can be approximated to the same accuracy by
\begin{equation}
    \label{eq:tauofp}
    \tau(p) \approx \displaystyle \tau_0\,\left(1-\frac{1}{S_{\tau}}\right) + 
\frac{\tau_0}{S_{\tau}}\,\sqrt{1-S_{\tau}\,{V_{\tau}^2}\,p^2}\;.
  \end{equation}
Find $\tau_0$, $V_{\tau}$, and $S_{\tau}$.
\end{enumerate}
\end{enumerate}

\section{Computational part 1}

In the first part of the computational assignment, you will
     investigate the numerical accuracy of a finite-difference eikonal solver.

  \inputdir{eikonal}

\lstset{language=python,numbers=left,numberstyle=\tiny,showstringspaces=false}
\lstinputlisting[frame=single]{eikonal/SConstruct}

  Figure~\ref{fig:exact} shows wavefronts in a medium with a constant
  gradient of slowness squared computed with an analytical two-point
  ray tracing formula from the class.
 
  \plot{exact}{width=\textwidth}{Wavefronts in a constant velocity
    gradient medium computed with an analytical formula.}

  By computing traveltimes numerically at different sampling intervals
  and comparing the numerical result with the analytical one, we can
  get an experimental estimate of the numerical error behavior. The
  error is shown in Figure~\ref{fig:err}. The right plot in
  Figure~\ref{fig:err} displays the error in logarithmic coordinates. The
  slope of the line in these coordinates shows directly the rate of
  convergence of the numerical method. For example, a first-order
  accurate method should have a slope of one.

  \plot{err}{width=\textwidth}{Left: average error of the
    finite-difference eikonal solver as a function of grid spacing.
    Right: the same on a log-log plot. The slope of the curve on the
    log-log plot indicates the order of the numerical accuracy.}

  \begin{enumerate}
  \item Change directory
\begin{verbatim}
> cd hw2/eikonal
\end{verbatim}
  \item Run
\begin{verbatim}
> scons view
\end{verbatim}
    to generate figures and display them on your screen.  

  \item In the \texttt{SConstruct} file, find the parameter that
    defines the order of accuracy for the eikonal solver. Change the
    order from $1$ to $2$ and recompute the results. Does the
    numerical accuracy change? What is the experimental order of
    accuracy? 
  \end{enumerate}


\section{Programming part 1}

Instead of an analytical solution for a constant 
    gradient of slowness squared, let us try an analytical solution for a constant
    gradient of velocity.   

\begin{enumerate}
  \item Change directory
\begin{verbatim}
> cd hw2/eikonal
\end{verbatim}
    \begin{itemize}
    \item Uncomment the part of the
    \texttt{SConstruct} file that defines a velocity model with the
    constant velocity gradient. 
    \item Modify the program \texttt{analytical.c} to implement your equation~(\ref{eq:t2}).
    \item Recompute the figures and check your results.
    \end{itemize}
  \end{enumerate}       

\lstset{language=c,numbers=left,numberstyle=\tiny,showstringspaces=false}
\lstinputlisting[frame=single]{eikonal/analytical.c}

\lstset{language=fortran,numbers=left,numberstyle=\tiny,showstringspaces=false}
\lstinputlisting[frame=single]{eikonal/analytical.f90}

\section{Computational part 2}
\inputdir{cmp}

In the computational part, we begin working with field data. The left
panel in Figure~\ref{fig:cmp} shows a CMP (common midpoint) gather
from the Gulf of Mexico \cite[]{bei}.

\plot{cmp}{width=\textwidth}{From left to right: (a) CMP (common midpoint) gather 
with overlaid traveltime curves. (b) Interval velocity. (c) RMS
(root-mean-square) velocity overlaid on the semblance scan. (d) CMP
gather after normal moveout.}
 
We will assume a $V(z)$ medium and will use a very simple model of the
interval velocity to explain the geometry of the observed data. The
model involves two parameters: the initial gradient of velocity and
the maximum velocity. The velocity function starts at the water
velocity of~1.5 km/s and grows linearly with vertical time until it
reaches the maximum velocity, after which point it remains flat. The
panels in Figure~\ref{fig:cmp} show the interval velocity, the
corresponding RMS velocity (overlaid on the semblance scan), and the
CMP gather after NMO (normal moveout). 

Your task is to find the best values of the two model parameters for
optimal prediction of the traveltime geometry and for flattening the
CMP gather after NMO.

\begin{enumerate}
\item Change directory 
\begin{verbatim}
cd hw2/cmp
\end{verbatim}
\item Run
\begin{verbatim}
scons cmps.vpl
\end{verbatim}
to generate and display a movie looping through different values of
the maximum velocity. If you are on a computer with multiple CPUs, you
can also try
\begin{verbatim}
pscons cmps.vpl
\end{verbatim}
to generate different movie frames faster by running computations in
parallel.
\item Edit the \texttt{SConstruct} file to modify the velocity gradient. Check your result by running
\begin{verbatim}
pscons cmps.vpl
\end{verbatim}
again. 
\item Edit the \texttt{SConstruct} file to select the 
best frame of the movie (corresponding to the best maximum velocity). Display it by running
\begin{verbatim}
scons view
\end{verbatim}
\end{enumerate}

\lstset{language=python,numbers=left,numberstyle=\tiny,showstringspaces=false}
\lstinputlisting[frame=single]{cmp/SConstruct}

\section{Programming part 2}

\sideplot{time}{width=\textwidth}{Traveltime in a $V(z)$ medium.}

The program \texttt{cmp/traveltime.c} computes reflection traveltimes
in a $V(z)$ medium by using different methods. 

\begin{enumerate}
\item Modify the program to implement approximation~(\ref{eq:shifted}) using your equation~(\ref{eq:a}).
\item Modify the program to implement exact traveltime computation by doing shooting iterations with equations~(\ref{eq:xofp}-\ref{eq:tofp}). 
Using Newton's method, you can find the value of $p$ for a given $h$ by solving the non-linear equation $h(p)=h$ with iterations
\begin{equation}
\label{eq:newton2}
p_{n+1} = p_n - \frac{h(p_n)-h}{h'(p_n)}\;.
\end{equation}
\item For the traveltime in Figure~\ref{fig:time}, find the offset, where the absolute error of approximation~(\ref{eq:hyper2}) exceeds~0.05 s. 
\item For the traveltime in Figure~\ref{fig:time}, find the offset, where the absolute error of approximation~(\ref{eq:shifted}) exceeds~0.05 s. 
\end{enumerate}

\lstset{language=c,numbers=left,numberstyle=\tiny,showstringspaces=false}
\lstinputlisting[frame=single]{cmp/traveltime.c}

\lstset{language=fortran,numbers=left,numberstyle=\tiny,showstringspaces=false}
\lstinputlisting[frame=single]{cmp/traveltime.f90}

\newpage

\section{Completing the assignment}

\begin{enumerate}
\item Change directory to \verb#hw2#.
\item Edit the file \texttt{paper.tex} in your favorite editor and change the
  first line to have your name instead of Euler's.
\item Run
\begin{verbatim}
sftour scons lock
\end{verbatim}
to update all figures.
\item Run
\begin{verbatim}
sftour scons -c
\end{verbatim}
to remove intermediate files.
\item Run
\begin{verbatim} 
  scons pdf
\end{verbatim}
  to create the final document.
\item Submit your result (file \texttt{paper.pdf}) on paper or by
  e-mail. 
\end{enumerate}


\bibliographystyle{seg} 
\bibliography{hw2}
