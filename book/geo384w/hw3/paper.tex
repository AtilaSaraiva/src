\author{Christiaan Huygens} 
%%%%%%%%%%%%%%%%%%%%%%
\title{Homework 3}

\begin{abstract}
  This homework has two parts. In the theoretical part, you will find
  the geometrical amplitude of the displacement vector in acoustic and
  elastic media. In the computational part, you will experiment with a
  real marine dataset from offshore Florida.
\end{abstract}

\section{Prerequisites}

Completing the computational part of this homework assignment requires
\begin{itemize}
\item \texttt{Madagascar} software environment available from \\
\url{http://rsf.sourceforge.net}
\item \LaTeX\ environment with \texttt{SEGTeX} available from \\ 
\url{http://segtex.sourceforge.net}
\end{itemize}

You are welcome to do the assignment on your personal computer by
installing the required environments. In this case, you can obtain all
homework assignments from the \texttt{Madagascar} repository by running
\begin{verbatim}
svn co https://rsf.svn.sourceforge.net/svnroot/rsf/trunk/book/geo384w 
\end{verbatim}

The necessary environment is also installed in your class account on
Unix computers in the Department of Geological Sciences.

\section{Theoretical part}

\begin{enumerate}
\item In class, we derived the following acoustic wave equation for
  pressure $P(\mathbf{x},t)$:
\begin{equation}
{\frac{1}{V^2(\mathbf{x})}}\,{\frac{\partial^2 P}{\partial t^2}} =\nabla^2 P + 
\rho(\mathbf{x})\,
\nabla\left(\frac{1}{\rho(\mathbf{x})}\right) \cdot \nabla P\;.
\label{eq:pwave}
\end{equation}
Using the connection between the pressure and displacement, derive the
acoustic wave equation for the displacement vector
$\mathbf{u}(\mathbf{x},t)$:
\begin{equation}
  \label{eq:uwave}
  \rho(\mathbf{x})\,\frac{\partial^2 \mathbf{u}}{\partial t^2} =
\end{equation}
\item Consider a geometrical wave representation in the vicinity of a wavefront
\begin{equation}
  {\mathbf{u}(\mathbf{x},t)} = \mathbf{a}(\mathbf{x})\,f\left(t-T(\mathbf{x}\right)
  \label{eq:gwave}
\end{equation}
and derive partial differential equations for the traveltime function
$T(\mathbf{x})$ and the vector amplitude $\mathbf{a}(\mathbf{x})$.
\item Assuming that the geometrical wave propagates in the direction
  of the traveltime gradient
\begin{equation}
  {\mathbf{a}(\mathbf{x})} = A(\mathbf{x})\,V(\mathbf{x})\,\nabla T
  \label{eq:vamp}
\end{equation}
show that the amplitude equation can take the conservative form
\begin{equation}
  \label{eq:cons}
  \nabla \cdot \left(\rho(\mathbf{x})\,V^2(\mathbf{x})\,A^2(\mathbf{x})\,\nabla T\right) = 0\;.
\end{equation}
\item Show that the displacement amplitude continuation along a ray is given by equation 
\begin{equation}
  \label{eq:avj}
  \left|\mathbf{a}_1\right| = \left|\mathbf{a}_0\right|\,
  \left(\frac{\rho_0\,V_0\,J_0}{\rho_1\,V_1\,J_1}\right)^{1/2}\;, 
\end{equation}
where $J_0$ and $J_1$ are the corresponding geometrical spreading factors.
\item (EXTRA CREDIT) Consider the elastic wave equation
\begin{equation}
  \label{eq:ewave}
  \rho\,\ddot{u}_i = C_{ijkl,j}\,u_{k,l} + C_{ijkl}\,u_{k,lj}\;
\end{equation}
in the case of an isotropic elasticity
\begin{equation}
  \label{eq:lame}
  C_{ijkl} = \lambda\,\delta_{ij}\,\delta_{kl} + 
  \mu\,(\delta_{ik}\,\delta_{jl} + \delta_{il}\,\delta_{jk})\;.
\end{equation}
Using the geometrical representation~(\ref{eq:gwave}) with the P-wave
polarization given by equation~(\ref{eq:vamp}), show that the
corresponding amplitude equations are similar to
equations~(\ref{eq:cons}) and~(\ref{eq:avj}).
\end{enumerate}

\lstset{language=python,numbers=left,numberstyle=\tiny,showstringspaces=false}

\section{Computational part}

In the computational part, we begin working with a real data
set. The dataset is a 2-D line from the Blake Outer Ridge area
offshore Florida. It was collected by USGS in order to study the
occurrence of methane hydrates. The dataset and its analysis for gas
hydrate detection are described by
\cite{GEO63-05-16591669,GEO65-02-05650573}.

Figure~\ref{fig:cmp} shows an example CMP (common midpoint) gather
from the dataset. Note the change in the trace spacing caused by using
a non-linear cable in the acquisition. The presence of gas hydrates is
manifested by a so-called BSR (bottom-simulating reflector). You can
get an idea of the spatial extent of BSR from the near-offset section
in Figure~\ref{fig:noff}.

\inputdir{blake}
\plot{cmp}{width=\textwidth}{CMP (common midpoint) gather from the Blake Outer Ridge dataset. 
The presence of gas hydrates is manifested by BSR (bottom-simulating
reflector).}

\inputdir{image}
\sideplot{noff}{width=\textwidth}{Near offset section of the Blake Outer Ridge data.}

\begin{enumerate}
\item We will use a very simple model to try explaining the geometry of the
observed events in the data. The left plot in Figure~\ref{fig:ray}
shows the model and rays from two-point ray tracing. The model
consists of a constant velocity water layer and a sedimentary layer
that contains BSR with a slowly variable velocity. The right plot in
Figure~\ref{fig:ray} shows the CMP gather and traveltime curves
obtained by ray tracing.

\inputdir{blake}
\plot{ray}{width=\textwidth}{Left: A two-layer model and ray trajectories. 
Right: A CMP gather and traveltime curves.}

Your task is to modify the model so that the predicted traveltime
curves match the geometry of the sea bottom and the BSR reflection
observed in the data.

\begin{enumerate}
\item Change directory 
\begin{verbatim}
> cd ~/geo384w/hw3/blake
\end{verbatim}
\item Run
\begin{verbatim}
> scons view
\end{verbatim}
to generate figures and display them on your screen.  
\item Edit the
top of the \texttt{SConstruct} file to modify the model
parameters. Check your result by running
\begin{verbatim}
> scons ray.view
\end{verbatim}
\end{enumerate}

{\small
  \lstinputlisting[frame=single]{blake/SConstruct}}

\item In the second part of the computational assignment, you will use the water velocity that 
you obtained in the previous part to produce a seismic image of the
near-offset section from Figure~\ref{fig:noff}. We will use the
zero-offset assumption that states that the reflector surface is
obtained by backward extrapolating the recorded zero-offset reflection
time into the subsurface using a velocity that is half of the true
velocity and stopping at zero time. Using the intermediate time steps,
you will create a movie of the backward traveltime extrapolation.

\inputdir{image}
\plot{image}{width=\textwidth}{Seismic image created with the water velocity.}

\begin{enumerate}
\item Change directory 
\begin{verbatim}
> cd ~/geo384w/hw3/image
\end{verbatim}
\item Run
\begin{verbatim}
> scons view
\end{verbatim}
to generate figures and display them on your screen.  
\item Edit the
\texttt{SConstruct} file to change the water velocity to the one you found in the previous part.
Check your result by running
\begin{verbatim}
> scons image.view
\end{verbatim}
\item Modify the last part of the \texttt{SConstruct} file to generate a movie of wavefront extrapolation 
from the surface to the seafloor.
\end{enumerate}

{\small \lstinputlisting[frame=single]{image/SConstruct}}

\end{enumerate}

\section{Completing the assignment}

\begin{enumerate}
\item Change directory to \verb#~/geo384w/hw3#.
\item Edit the file \texttt{paper.tex} in your favorite editor and change the
  first line to have your name instead of Huygens's.
\item Run
\begin{verbatim}
  > sftour scons lock
\end{verbatim}
to update all figures.
\item Run
\begin{verbatim}
  > sftour scons -c
\end{verbatim}
  to remove intermediate files.
\item Run
 \begin{verbatim} 
  > scons pdf
\end{verbatim}
  to create the final document.
\item Submit your result (file \texttt{paper.pdf}) on paper or by
  e-mail. If you do your assignment on one of the computers in the
  Unix lab, you can simply leave the file in your directory.
\end{enumerate}

\bibliographystyle{seg} 
\bibliography{SEG}
