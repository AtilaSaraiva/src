\author{Gustav Kirchhoff} 
%%%%%%%%%%%%%%%%%%%%%%%%%
\title{Homework 4}

\begin{abstract}
  This homework has two parts. In the theoretical part, you will
  derive analytically the reflection traveltime from a hyperbolic
  reflector and the geometrical spreading for the case of a constant
  gradient of slowness squared.  In the computational part, you will
  experiment with imaging synthetic data using traveltimes and waves.
\end{abstract}

\lstset{language=python,numbers=left,numberstyle=\tiny,showstringspaces=false}

\section{Theoretical part}

\begin{enumerate}
\item Consider the source point $s$ and the receiver point $r$ at the
  surface $z=0$ above a 2-D constant-velocity medium and a hyperbolic
  reflector defined by the equation 
  \begin{equation}
    \label{eq:hyper}
    z(x) = \sqrt{h^2 +
      x^2\,\tan^2{\alpha}}\;.
  \end{equation}
  \begin{enumerate}
  \item Apply Fermat's principle to derive the horizontal coordinate of
    the reflection point $y$ and the reflection traveltime $T$ as 
    functions of $s$, $r$, $h$, and $\alpha$.
  \item Test your equations by considering the limiting cases
    \begin{itemize}
    \item $h=0$, 
      when the hyperbolic reflector turns into a dipping 
      plane $z(x)= x\,\tan{\alpha}$ for positive $x$ (recall Homework 2).
    \item $\alpha=0$,
      when the
      hyperbolic reflector turns into a flat plane $z(x)=h$.
    \end{itemize}
  \end{enumerate}
\item Consider a medium with a constant gradient of slowness squared
  \begin{equation}
    S^2(\mathbf{x}) = S_0^2 + 2\,\mathbf{g} \cdot (\mathbf{x}-\mathbf{x}_0)\;.
  \end{equation}
  \begin{enumerate}
  \item Show that, in this kind of medium, the dynamic ray tracing system
    takes the form
  \begin{eqnarray}
    \label{eq:dxds}
    {\frac{d \mathbf{X}}{d \sigma}} & = & \mathbf{P}\;, \\
    {\frac{d \mathbf{P}}{d \sigma}} & = & \mathbf{0}\;.
    \label{eq:dpds}
  \end{eqnarray}
\item Consider a point source in the 2-D case. Show that matrices
  $\mathbf{X}$ and $\mathbf{P}$ can be initialized as
  \begin{eqnarray}
    \label{eq:x0}
    \mathbf{X}(0) & = & \left[\begin{array}{cc} 
        \frac{\cos \theta}{S_0} & \frac{\sin \theta}{S_0} \\
        0 & 0 \end{array}\right]\;, \\
    \mathbf{P}(0) & = & \left[\begin{array}{cc} 
        0 & 0 \\
        -S_0\,\sin \theta & S_0\,\cos \theta \end{array}\right]\;, \\
    \label{eq:p0}
  \end{eqnarray}
  where $\theta$ is the initial ray angle.
\item Solve system~(\ref{eq:dxds}-\ref{eq:dpds}) with initial
  conditions~(\ref{eq:x0}-\ref{eq:p0}) to find $\mathbf{X}(\sigma)$
  and $\mathbf{P}(\sigma)$.
\item Using your derivations, find the two-point geometrical spreading
  $J$ from a point source in the 2-D case as a function of $S_0$,
  $\mathbf{g}$, the source location $\mathbf{x}_0$ and the receiver
  location $\mathbf{x}_1$.
\end{enumerate}

\end{enumerate}

\section{Computational part}

In this part of the homework, we will use a simple synthetic model to
explore common-shot imaging with traveltimes and waves. You will study
the effect of velocities on imaging.

\begin{enumerate}
\item We will start by exploring traveltime imaging using the
  hyperbolic reflector model from equation~(\ref{eq:hyper}). The model
  is shown in Figure~\ref{fig:model}.

  \inputdir{times}
  \plot{model}{width=\textwidth}{Synthetic velocity model with one reflector.}

{\small
  \lstinputlisting[frame=single]{times/SConstruct}}

  Figure~\ref{fig:dtime0-0,dtime1-0,dtime2-0} demonstrates the
  traveltime imaging procedure for three different source locations:
  forward continuation from the source, backward continuation from the
  recorded reflection traveltime, and taking the zero level set of
  their difference as a map of the reflector location.

  \multiplot{3}{dtime0-0,dtime1-0,dtime2-0}{width=\textwidth}{Forward
    traveltime continuation from the source (left), backward
    traveltime continuation from the recorded reflection traveltime
    (center), and their difference (right) for three different source
    locations: (a) $s=1\,\mbox{km}$, (b) $s=2\,\mbox{km}$, 
    (c) $s=3\,\mbox{km}$.}

  Combining images from different sources, we obtain the result shown
  in Figure~\ref{fig:time0}.

  \plot{time0}{width=\textwidth}{Reflector images from traveltime
    continuation overlayed on top of the velocity model.}

  What happens when the velocity used for traveltime continuation is
  incorrect? The results are shown in
  Figures~\ref{fig:dtime0-1,dtime1-1,dtime2-1} and~\ref{fig:time1}.

  \multiplot{3}{dtime0-1,dtime1-1,dtime2-1}{width=\textwidth}{ Imaging
    in velocity that is 10\% lower than the correct one. Forward
    traveltime continuation from the source (left), backward
    traveltime continuation from the recorded reflection traveltime
    (center), and their difference (right) for three different source
    locations: (a) $s=1\,\mbox{km}$, (b) $s=2\,\mbox{km}$, (c)
    $s=3\,\mbox{km}$.}

  \plot{time1}{width=\textwidth}{Imaging in velocity that is 10\%
    lower than the correct one. Reflector images from traveltime
    continuation overlaid on top of the velocity model.}

\begin{enumerate}
\item Change directory 
\begin{verbatim}
> cd ~/geo384w/hw4/times
\end{verbatim}
\item Run
\begin{verbatim}
> scons view
\end{verbatim}
  to generate figures and display them on your screen.  
\item Explain the artifacts. Why, even with the correct velocity, images are not perfectly aligned with the reflector?
\item Edit the \texttt{SConstruct} file to change the reflector shape
  from plane to hyperbolic. Run 
\begin{verbatim}
> scons view
\end{verbatim}
  again to observe the differences.
\item Instead of using the velocity that is 10\% lower, try a velocity that is 10\% higher. 
Make appropriate changes in the \texttt{SConstruct} file and update the figures.
\item Explain how traveltime imaging with multiple sources can be used to estimate the correct velocity. 
\end{enumerate}

\item Now we will turn to imaging with waves. Figure~\ref{fig:shots} shows three synthetic shot gathers generated for the same reflector model.
Figures~\ref{fig:image} shows an image generated by Kirchhoff
migration with the correct velocity.

{\small \lstinputlisting[frame=single]{waves/SConstruct}}

\inputdir{waves}
\plot{shots}{width=\textwidth}{Shot gathers for the hyperbolic reflector model.}
\plot{image}{width=\textwidth}{Kirchhoff image for the hyperbolic reflector model.}

% ADD ANOTHER FIGURE HERE

\begin{enumerate}
\item Change directory 
\begin{verbatim}
> cd ~/geo384w/hw4/waves
\end{verbatim}
\item Run
\begin{verbatim}
> scons view
\end{verbatim}
  to generate figures and display them on your screen.
\item Edit the \texttt{SConstruct} file to change the reflector shape
  from plane to hyperbolic. Run 
\begin{verbatim}
> scons view
\end{verbatim}
  again to observe the differences.
\item Create a loop of different images for velocities that are 10\% lower, correct, 
and 10\% higher. Make appropriate changes in the \texttt{SConstruct} file.
\item Include your new figures in the \texttt{geo384w/hw4/paper.tex} file following the analogy with 
Figure~\ref{fig:image}.
\item Explain how wave imaging with multiple sources can be used to estimate the correct velocity. 
\end{enumerate}

\end{enumerate}

\section{Completing the assignment}

\begin{enumerate}
\item Change directory to \verb#~/geo384w/hw4#.
\item Edit the file \texttt{paper.tex} in your favorite editor and change the
  first line to have your name instead of Kirchhoff's.
\item Run
\begin{verbatim}
  > sftour scons lock
\end{verbatim}
to update all figures.
\item Run
\begin{verbatim}
  > sftour scons -c
\end{verbatim}
  to remove intermediate files.
\item Run
 \begin{verbatim} 
  > scons pdf
\end{verbatim}
  to create the final document.
\item Submit your result (file \texttt{paper.pdf}) on paper or by
  e-mail. If you do your assignment on one of the computers in the
  Unix lab, you can simply leave the file in your directory.
\end{enumerate}

