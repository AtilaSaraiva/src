\author{Gustav Kirchhoff} 
%%%%%%%%%%%%%%%%%%%%%%%%%
\title{Homework 4}

\begin{abstract}
  This homework has two parts. In the theoretical part, you will
  perform several analytical derivations: amplitudes of elastic waves,
  geometric slant stacks, shifted-hyperbola approximations. In the
  computational part, you will experiment with imaging reflections and
  diffractions in two synthetic datasets and a field dataset from the
  Gulf of Mexico.
\end{abstract}

Completing the computational part of this homework assignment requires
\begin{itemize}
\item \texttt{Madagascar} software environment available from \\
  \url{http://www.ahay.org}
\item \LaTeX\ environment with \texttt{SEGTeX} available from \\ 
  \url{http://www.ahay.org/wiki/SEGTeX}
\end{itemize}

You are welcome to do the assignment on your personal computer by
installing the required environments. In this case, you can obtain all
homework assignments from the \texttt{Madagascar} repository by running
\begin{verbatim}
svn co http://svn.code.sf.net/p/rsf/code/trunk/book/geo384w/hw4
\end{verbatim}

\section{Theoretical part}

\begin{enumerate}
\item Consider the elastic wave equation
\begin{equation}
  \label{eq:ewave}
  \rho\,\ddot{u}_i = C_{ijkl,j}\,u_{k,l} + C_{ijkl}\,u_{k,lj}\;
\end{equation}
in the case of an isotropic elasticity
\begin{equation}
  \label{eq:lame}
  C_{ijkl} = \lambda\,\delta_{ij}\,\delta_{kl} + 
  \mu\,(\delta_{ik}\,\delta_{jl} + \delta_{il}\,\delta_{jk})\;.
\end{equation}
Using the geometric representation
\begin{equation}
  u_i(\mathbf{x},t) = a_i(\mathbf{x})\,f\left(t-T(\mathbf{x})\right)
  \label{eq:gwave}
\end{equation}
and assuming the P-wave polarization in the direction of the gradient
of $T$, derive the elastic P-wave amplitude transport equation and
show its similarity to the corresponding equation for the case of
acoustic variable-density wave propagation.
\item Consider a 2-D common-midpoint gather $G(t,x)$, which
  contains a geometric event $A_0\,f\left(t-T(x)\right)$ with a
  constant amplitude $A_0$ along a hyperbolic shape
\begin{equation}
\label{eq:hyper}
T(x) = \sqrt{t_0^2+\frac{x^2}{v^2}}\;.
\end{equation}
The gather gets transformed by the slant-stack (Radon transform) operator
\begin{equation}
\label{eq:radon}
R(\tau,p) = \mathbf{D}_\tau^{1/2} \int G(\tau + p x, x) d x\;.
\end{equation}
where $ \mathbf{D}_\tau^{1/2}$ is a waveform-correcting half-order
derivative operator.

Using the theory of geometric integration, show that $R(\tau,p)$
will contain a geometric event $A_1(p)\,f\left(\tau-T_1(p)\right)$.
Find $T_1(p)$ and $A_1(p)$.

\item Using the hyperbolic traveltime approximation
\begin{equation}
T = \sqrt{\left(T_0-\frac{z}{V_0}\right)^2 + \frac{(x_0-x)^2}{V_0^2}}
\end{equation}
makes the geometric imaging analysis equivalent to analyzing wave
propagation in a constant-velocity medium. In particular, we can easily
verify that the traveltime satisfies the isotropic eikonal equation 
\begin{equation}
\left(\frac{\partial T}{\partial x}\right)^2 + \left(\frac{\partial T}{\partial z}\right)^2 = \frac{1}{V_0^2}\;.
\end{equation}

Suppose that you switch to the more accurate shifted-hyperbola approximation
\begin{equation}
T = \left(T_0-\frac{z}{V_0}\right)\,(1-\frac{1}{S}) + \frac{1}{S}\,\sqrt{\left(T_0-\frac{z}{V_0}\right)^2 + S\,\frac{(x_0-x)^2}{V_0^2}}
\end{equation}

\begin{enumerate}
\item How would you need to modify the eikonal equation?
\item How would you need to modify the following expressions for the escape time and location for use in the angle-domain Kirchhoff time migration?
\begin{eqnarray}
\hat{T} & = & \frac{T_0-z/V_0}{\cos{\theta}} \\
\hat{x} & = & x_0 + V_0\,\left(T_0-\frac{z}{V_0}\right)\,\tan{\theta}
\end{eqnarray}

\end{enumerate}

\end{enumerate}

\newpage

\section{Computational part}

\begin{enumerate}
\item In the first part of the computational assignment, you will experiment with 
  imaging a synthetic seismic reflection dataset from Homework 3 using
  prestack velocity continuation.

  \inputdir{synth}

\plot{data}{width=0.9\textwidth}{2-D synthetic data.}
  \sideplot{model}{width=\textwidth}{(a) Synthetic model: curved
    reflectors in a $V(z)$ velocity.}
  \plot{mig}{width=0.9\textwidth}{Initial constant-velocity migration.}
\sideplot{dmig2}{width=\textwidth}{Time migration converted to depth, with reflectors overlaid.}

Figure~\ref{fig:data} shows a synthetic reflection dataset computed
from a reflector model shown in Figure~\ref{fig:model} and assuming a
velocity model with a constant vertical gradient $V(z) = 1.5 +
0.36\,z$. A small amount of random noise is added to the data.

Figure~\ref{fig:mig} shows an initial prestack common-offset time
migration using a constant velocity of 1.5 km/s. 
Figure~\ref{fig:dmig2} shows the result of prestack time
migration after velocity continuation, extraction of a velocity slice,
and conversion from time to depth.

\begin{enumerate}
\item Change directory 
\begin{verbatim}
cd hw4/synth
\end{verbatim}
\item Run
\begin{verbatim}
scons view
\end{verbatim}
to generate the figures and display them on your screen.
If you are on a computer with multiple CPUs, you
can also try
\begin{verbatim}
pscons view
\end{verbatim}
to run certain computations in parallel.
\item Run 
\begin{verbatim}
pscons velcon.vpl
\end{verbatim}
to display a movie of the velocity continuation process.
\item Run 
\begin{verbatim}
pscons semb.vpl
\end{verbatim}
to display a movie slicing through a semblance cube computed from
velocity continuation.
\item The processing flow in the \texttt{SConstruct} file involves
  some cheating: the exact RMS velocity is used to extract the final
  image. Modify the processing flow so that only properties estimated
  from the data get used. 
\end{enumerate}

\lstset{language=python,numbers=left,numberstyle=\tiny,showstringspaces=false}
\lstinputlisting[frame=single]{synth/SConstruct}

\item In the second part of the computational assignment, we will use velocity continuation again but
  this time on a synthetic zero-offset section containing diffraction
  events.

  Figure~\ref{fig:vel} shows a famous Sigsbee synthetic velocity
  model. We will focus on the left part of the model, which is
  appropriate for time-domain imaging. A synthetically generated
  zero-offset section is shown in Figure~\ref{fig:data}. 
  
  Our processing strategy is to extract diffractions from the data
  (Figure~\ref{fig:dif}) and to image them using zero-offset velocity
  continuation (Figure~\ref{fig:dimage}). In addition, we are going to
  analyze the image by expanding it in dip angles by using dip-angle
  migration (Figure~\ref{fig:anglemig}).

\inputdir{sigsbee}
\sideplot{vel}{width=\textwidth}{Sigsbee velocity model.}
\plot{data}{width=0.7\textwidth}{Zero-offset synthetic data.}
\plot{dif}{width=0.7\textwidth}{Diffractions extracted from the data
  by plane-wave destruction.}
\plot{dimage}{width=0.7\textwidth}{Time-migrated image of
  diffractions.}
\plot{anglemig}{width=0.7\textwidth}{Dip angle gathers from constant-velocity angle-domain migration.}

\begin{enumerate}
\item Change directory 
\begin{verbatim}
cd hw4/sigsbee
\end{verbatim}
\item Run
\begin{verbatim}
scons view
\end{verbatim}
to generate the figures and display them on your screen.
If you are on a computer with multiple CPUs, you
can also try
\begin{verbatim}
pscons view
\end{verbatim}
to run certain computations in parallel.
\item Generate a movie displaying the velocity continuation process. Is it possible to detect velocities from focusing zero-offset diffractions?
\item Modify the program in the \texttt{anglemig.c} file to input a variable migration velocity instead of using a constant velocity. 
Regenerate Figure~\ref{fig:anglemig} using a variable velocity 
\begin{verbatim}
pscons anglemig.view
\end{verbatim}
Do you notice a difference?
\item For EXTRA CREDIT, implement a method for estimating migration
  velocity from the input data.
\end{enumerate}

\lstset{language=python,numbers=left,numberstyle=\tiny,showstringspaces=false}
\lstinputlisting[frame=single]{sigsbee/SConstruct}

\lstset{language=c,numbers=left,numberstyle=\tiny,showstringspaces=false}
\lstinputlisting[frame=single]{sigsbee/anglemig.c}

\item In the final part of the computational assignment, we return to the 2-D field dataset from the Gulf of Mexico.
The zero-offset data after a DMO stack are shown in Figure~\ref{fig:stack}.

\inputdir{gulf}
\plot{stack}{width=0.8\textwidth}{2-D field dataset from the Gulf of Mexico after DMO stack.}

\begin{enumerate}
\item Change directory 
\begin{verbatim}
cd hw4/gulf
\end{verbatim}
\item Run
\begin{verbatim}
scons view
\end{verbatim}
to generate Figure~\ref{fig:stack} and display it on your screen.
\item Edit the \texttt{SConstruct} file to implement a processing flow involving velocity continuation and 
angle-gather migration. Make sure to select appropriate processing parameters.
\end{enumerate}

\newpage

\lstset{language=python,numbers=left,numberstyle=\tiny,showstringspaces=false}
\lstinputlisting[frame=single]{gulf/SConstruct}

\end{enumerate}

\newpage


\section{Completing the assignment}

\begin{enumerate}
\item Change directory to \texttt{hw4}.
\item Edit the file \texttt{paper.tex} in your favorite editor 
      and change the first line to have your name instead of Kirchhoff's.
\item Run
\begin{verbatim}
sftour scons lock
\end{verbatim}
to update all figures.
\item Run
\begin{verbatim}
sftour scons -c
\end{verbatim}
to remove intermediate files.
\item Run
\begin{verbatim} 
  scons pdf
\end{verbatim}
  to create the final document.
\item Submit your result (file \texttt{paper.pdf}) on paper or by
  e-mail. 
\end{enumerate}
