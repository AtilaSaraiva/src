\author{Alexey Malovichko} 
%%%%%%%%%%%%%%%%%%%%%%%%%
\title{Homework 4}

\begin{abstract}
  This homework has two parts. In the theoretical part, you will
  perform analytical derivations related to the shifted hyperbola
  approximation. In the computational part, you will experiment with
  imaging a synthetic dataset and a field dataset from ...
\end{abstract}

Completing the computational part of this homework assignment requires
\begin{itemize}
\item \texttt{Madagascar} software environment available from \\
  \url{http://www.ahay.org}
\item \LaTeX\ environment with \texttt{SEGTeX} available from \\ 
  \url{http://www.ahay.org/wiki/SEGTeX}
\end{itemize}

You are welcome to do the assignment on your personal computer by
installing the required environments. In this case, you can obtain all
homework assignments from the \texttt{Madagascar} repository by running
\begin{verbatim}
svn co https://rsf.svn.sourceforge.net/svnroot/rsf/trunk/book/geo384w/hw4 
\end{verbatim}

\section{Theoretical part}

\begin{enumerate}
\item Consider 
\end{enumerate}

\section{Computational part}

\item In the first part, you will experiment with 
  imaging a synthetic seismic reflection dataset from Homework 3 using
  prestack velocity continuation.

  \inputdir{synth}

\plot{data}{width=0.9\textwidth}{2-D synthetic data.}
  \sideplot{model}{width=\textwidth}{(a) Synthetic model: curved
    reflectors in a $V(z)$ velocity.}
  \plot{mig}{width=0.9\textwidth}{Initial constant-velocity migration.}
  \plot{dmig2}{width=0.9\textwidth}{Time migration converted to depth, with reflectors overlaid.}

Figure~\ref{fig:data} shows a synthetic reflection dataset computed
from a reflector model shown in Figure~\ref{fig:ref} and assuming a
velocity model with a constant vertical gradient $V(z) = 1.5 +
0.25\,z$. A small amount of random noise is added to the data.

Figure~\ref{fig:mig} shows an initial prestack common-offset time
migration using a constant velocity of 1.5 km/s. 
Figure~\ref{fig:dmig2} shows the result of prestack time
migration after velocity continuation, extraction of a velocity slice,
and conversion from time to depth.

\begin{enumerate}
\item Change directory 
\begin{verbatim}
cd hw4/synth
\end{verbatim}
\item Run
\begin{verbatim}
scons view
\end{verbatim}
to generate the figures and display them on your screen.
If you are on a computer with multiple CPUs, you
can also try
\begin{verbatim}
pscons view
\end{verbatim}
to run certain computations in parallel.
\item Run 
\begin{verbatim}
pscons velcon.vpl
\end{verbatim}
to display a movie of the velocity continuation process.
\item Run 
\begin{verbatim}
pscons semb.vpl
\end{verbatim}
to display a movie slicing through a semblance cube computed from
velocity continuation.
\item The processing flow in the \texttt{SConstruct} file involves
  some cheating: the exact RMS velocity is used to extract the final
  image. Modify the processing flow so that only properties estimated
  from the data get used. 
\end{enumerate}

\lstset{language=python,numbers=left,numberstyle=\tiny,showstringspaces=false}
{\small	\lstinputlisting[frame=single]{synth/SConstruct}}

\item In the second part, we will use velocity continuation again but
  this time on a synthetic zero-offset section containing diffraction events.

\sideplot{vel}{width=\textwidth}{Sigsbee velocity model.}
\plot{data}{width=0.9\textwidth}{Zero-offset synthetic data.}
\plot{dif}{width=0.9\textwidth}{Diffractions extracted from the data
  by plane-wave destruction.}
\plot{dimage}{width=0.9\textwidth}{Time-migrated image of
  diffractions.}
\plot{anglemig}{width=0.9\textwidth}{Dip angle gathers.}

\begin{enumerate}
\item Change directory 
\begin{verbatim}
cd hw4/sigsbee
\end{verbatim}
\item Run
\begin{verbatim}
scons view
\end{verbatim}
to generate the figures and display them on your screen.
If you are on a computer with multiple CPUs, you
can also try
\begin{verbatim}
pscons view
\end{verbatim}
to run certain computations in parallel.
\end{enumerate}

\lstset{language=python,numbers=left,numberstyle=\tiny,showstringspaces=false}
{\small	\lstinputlisting[frame=single]{sigsbee/SConstruct}}

\lstset{language=c,numbers=left,numberstyle=\tiny,showstringspaces=false}
\lstinputlisting[frame=single]{sigsbee/anglemig.c}


\item In the second part of the computational assignment, you will use the processing strategy developed for synthetic data, 
to process a 2-D field dataset from ...

\begin{enumerate}
\item Change directory 
\begin{verbatim}
cd hw3/gulf
\end{verbatim}
\item Run
\begin{verbatim}
scons view
\end{verbatim}
to generate the figures and display them on your screen.
\item Edit the \texttt{SConstruct} file to implement your processing strategy. 
Make sure to select appropriate processing parameters.
\end{enumerate}

\lstset{language=python,numbers=left,numberstyle=\tiny,showstringspaces=false}
%\lstinputlisting[frame=single]{gulf/SConstruct}

\end{enumerate}

\section{Completing the assignment}

\begin{enumerate}
\item Change directory to \verb#hw4#.
\item Edit the file \texttt{paper.tex} in your favorite editor and change the
  first line to have your name instead of Malovichko's.
\item Run
\begin{verbatim}
sftour scons lock
\end{verbatim}
to update all figures.
\item Run
\begin{verbatim}
sftour scons -c
\end{verbatim}
to remove intermediate files.
\item Run
 \begin{verbatim} 
 scons pdf
\end{verbatim}
  to create the final document.
\item Submit your result (file \texttt{paper.pdf}) on paper or by
  e-mail. 
\end{enumerate}