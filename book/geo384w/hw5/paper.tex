\author{Jean Baptiste Joseph Fourier} 
%%%%%%%%%%%%%%%%%%%%%%%%%%%%%%%%%%%%%
\title{Homework 5}

\begin{abstract}
  This homework has only a computational part but it will require you
  to make some theoretical developments as well.  The theme of the
  homework is one-way wavefield extrapolation. You will develop
  efficient approximations for one-way extrapolators for a constant
  velocity and a highly variable velocity. Additionally, you will
  experiment with wave-equation migration by least-squares inversion.
\end{abstract}

Completing the computational part of this homework assignment requires
\begin{itemize}
\item \texttt{Madagascar} software environment available from \\
  \url{http://www.ahay.org}
\item \LaTeX\ environment with \texttt{SEGTeX} available from \\ 
  \url{http://www.ahay.org/wiki/SEGTeX}
\end{itemize}

You are welcome to do the assignment on your personal computer by
installing the required environments. In this case, you can obtain all
homework assignments from the \texttt{Madagascar} repository by running
\begin{verbatim}
svn co http://svn.code.sf.net/p/rsf/code/trunk/book/geo384w/hw5
\end{verbatim}

\section{Computational part}

\begin{enumerate}

\item The first example is the model from Homework 1 with a hyperbolic
  reflector under a constant velocity layer.  The model is shown in
  Figure~\ref{fig:model}. Figure~\ref{fig:shot} shows a shot gather
  modeled at the surface and extrapolated to a level of 1 km in depth
  using two extrapolation operators: 
  \begin{enumerate}
  \item The exact phase shift filter
    \begin{equation}
      \label{eq:ps}
      \hat{U}(z,k,\omega) = \hat{U}(0,k,\omega)\,e^{i\,\sqrt{S^2\,\omega^2 - k^2}\,z}\;.
    \end{equation}
  \item Its approximation
    \begin{equation}
      \label{eq:psa}
      \hat{U}(z,k,\omega) \approx \hat{U}(0,k,\omega)\,
      e^{i\,S\,\omega\,z}\,
      \frac
      {\displaystyle S\,\omega + \frac{i\,\left(\cos{(k\,\Delta x)}-1\right)\,z}{2\,(\Delta x)^2}}
      {\displaystyle S\,\omega - \frac{i\,\left(\cos{(k\,\Delta x)}-1\right)\,z}{2\,(\Delta x)^2}}\;.
    \end{equation}
    Approximation~(\ref{eq:psa}) is suitable for an implementation in the space domain
    with a digital recursive filter. However, its accuracy is limited,
    which is evident both from Figure~\ref{fig:shot} and from
    Figure~\ref{fig:phase}, which compares the phases of the exact and
    the approximate extrapolators. We can see that
    approximation~(\ref{eq:psa}) is accurate only for small angles
    from the vertical~$\theta$, defined by 
\[
\theta = \arcsin\left(\frac{k}{S\,\omega}\right)\;.
\] 
  \end{enumerate}

  \textbf{Your task:} Design an approximation that would be more accurate
    than approximation~(\ref{eq:psa}). Your approximation should be
    suitable for a digital filter implementation in the space
    domain. Therefore, it can involve $k$ only through $\cos{(n\,k\,\Delta
      x)}$ functions with integer $n$.
    \begin{enumerate}
    \item Change directory 
\begin{verbatim}
cd hw5/hyper
\end{verbatim}
    \item Run
\begin{verbatim}
scons view
\end{verbatim}
      to generate figures and display them on your screen.  
    \item Edit the \texttt{SConstruct} file to change the approximate extrapolator.
    \item Run
\begin{verbatim}
scons view
\end{verbatim}
      again to observe the differences.
    \end{enumerate}

\inputdir{hyper}

\plot{model}{width=0.85\textwidth}{Synthetic velocity model with a hyperbolic reflector.}

\plot{shot}{width=\textwidth}{Synthetic shot gather. Left: Modeled for receivers at the surface. Middle: Receivers extrapolated to 1 km in depth with an exact phase-shift extrapolation operator. Right: Receivers extrapolated to 1 km in depth with an approximate extrapolation operator.}

\sideplot{phase}{width=\textwidth}{Phase of the extrapolation operator at 5 Hz frequency as a function of the wave propagation angle. Solid line: exact extrapolator. Dashed line: approximate extrapolator.}

\lstset{language=python,numbers=left,numberstyle=\tiny,showstringspaces=false}
\lstinputlisting[frame=single]{hyper/SConstruct}

%\newpage

\item The second example uses the Sigsbee synthetic velocity model which
  you encountered in Homework 4. The model is shown in
  Figure~\ref{fig:vel}. We will select one slice of the model (at 15
  kft depth) to analyze different one-way wavefield
  extrapolators. Figure~\ref{fig:phase2} 
  compares the phases of two one-way wave
  extrapolation operators:
  \begin{enumerate}
  \item The exact non-stationary phase shift filter
    \begin{equation}
      \label{eq:nps}
      E(k,x) = e^{i\,\sqrt{S^2(x)\,\omega^2 - k^2}\,\Delta z}
    \end{equation}
  \item Its split-step approximation
    \begin{equation}
      \label{eq:npsa}
      E(k,x) \approx e^{i\,S(x)\,\omega\,\Delta z}\,e^{i\,\sqrt{S_0^2\,\omega^2 - k^2}\,\Delta z-i\,S_0\,\omega\,\Delta z}
    \end{equation}
    Approximation~(\ref{eq:npsa}) can be implemented efficiently, 
    because it involves only diagonal matrix multiplications in space and wavenumber domains. However, its accuracy is limited.
  \end{enumerate}   
  
  \textbf{Your task:} Design an approximation that would be more accurate
  than approximation~(\ref{eq:npsa}). Your approximation should be
  suitable for an efficient implementation. It can involve products of functions
  of $x$  and functions
  of $k$ but not any functions that mix $x$ and $k$. 
    \begin{enumerate}
    \item Change directory 
\begin{verbatim}
cd hw5/sigsbee
\end{verbatim}
    \item Run
\begin{verbatim}
scons view
\end{verbatim}
      to generate figures and display them on your screen.  
    \item Edit the \texttt{SConstruct} file to change the approximate extrapolator.
    \item Run
\begin{verbatim}
scons view
\end{verbatim}
      again to observe the differences.
    \end{enumerate}

\inputdir{sigsbee}

\plot{vel}{width=\textwidth}{Sigsbee velocity model. A slice of the model at 15 kft depth is selected for analyzing wavefield extrapolation operators.}

\plot{phase2}{width=\textwidth}{Phase of the wavefield extrapolation operators for a depth slice at 15 kft and frequency of 5 Hz as a function of space $x$ and wavenumber $k$. Top: Exact mixed-domain extrapolator. Bottom: Split-step approximation.}

\lstset{language=python,numbers=left,numberstyle=\tiny,showstringspaces=false}
\lstinputlisting[frame=single]{sigsbee/SConstruct}

\item In the third example, we return to the synthetic model shown in Figure~\ref{fig:model} to experiment with  modeling and migration by the phase-shift method in a $V(z)$ medium. 

Figure~\ref{fig:expl} shows an idealized image (band-passed
reflectivity) for the synthetic model. In ``exploding-reflector''
modeling, this image is assumed to be the seismic wavefield frozen at
zero time. The modeling program extrapolates the wavefield to the
surface and is implemented in \texttt{phaseshift.c}. The program
operates in the frequency-wavenumber domain and establishes a linear
relationship between reflectivity as a function of depth and
zero-offset data as a function of frequency for one space wavenumber.
If we think of this linear relationship as a matrix multiplication, we
can associate forward modeling with multiplication
\begin{equation}
\label{eq:mod}
\mathbf{d} = \mathbf{A}\,\mathbf{m}
\end{equation}
and migration with the adjoint multiplication
\begin{equation}
\label{eq:mig}
\widehat{\mathbf{m}} = \mathbf{A}^T\,\mathbf{d}\;,
\end{equation}
where $\mathbf{A}^T$ is the conjugate transpose of $\mathbf{A}$.

Figure~\ref{fig:modl} shows the result of forward modeling
(multiplication by $\mathbf{A}$) after inverse Fourier transform to
time and space. Your task is to implement the corresponding migration
(multiplication by $\mathbf{A}^T$). 

Instead of simply applying the adjoint operator, we can also try to compute
the \emph{least-squares inverse}
\begin{equation}
\label{eq:lsmig}
\widehat{\mathbf{m}} = \left(\mathbf{A}^T\,\mathbf{A}\right)^{-1}\,\mathbf{A}^T\,\mathbf{d}\;,
\end{equation}
which corresponds to the minimum of the least-squares misfit function
$|\mathbf{A}\,\mathbf{m}-\mathbf{d}|^2$. In practice, inversion in
equation~(\ref{eq:lsmig}) is implemented with an
iterative \emph{conjugate-gradient} algorithm, which applies
$\mathbf{A}$ and $\mathbf{A}^T$ (modeling and migration) at each
iteration without the need to form any matrices explicitly.

  \inputdir{lsmig}

  \multiplot{2}{model,expl}{width=0.45\textwidth}{(a) Synthetic model: curved
    reflectors in a $V(z)$ velocity. (b) ``Exploding reflector'', an ideal image of band-passed reflectivity.}
    \plot{modl}{width=0.8\textwidth}{Zero-offset data generated by exploding-reflector phase-shift modeling.}

\begin{enumerate}
\item Change directory 
\begin{verbatim}
cd hw5/lsmig
\end{verbatim}
\item Run
\begin{verbatim}
scons view
\end{verbatim}
to generate the figures and display them on your screen.
\item Modify the program in the \texttt{phaseshift.c} file to fill the missing part and to implement phase-shift migration as the adjoint of phase-shift modeling. 
\item Modify the \texttt{SConstruct} file to uncomment the part related to migration. Check your result by running
\begin{verbatim}
scons migr.view
\end{verbatim}
\item Test if your migration code is truly the adjoint of modeling by running the dot-product test
\begin{verbatim}
sfcdottest ./phaseshift.exe mod=kexpl.rsf dat=kmodl.rsf \
vel=vofz.rsf nw=247 dw=0.16276
\end{verbatim}
On a machine with multiple CPUs, you can also try
\begin{verbatim}
sfcdottest sfomp ./phaseshift.exe split=2 \
mod=kexpl.rsf dat=kmodl.rsf vel=vofz.rsf nw=247 dw=0.16276
\end{verbatim}
If your adjoint is correct, you should see two identical sets of numbers, such as
\begin{verbatim}
sfcdottest:  L[m]*d=(25264,-20273.9)
sfcdottest: L'[d]*m=(25264,-20273.9)
\end{verbatim}
Your actual numbers will be different because of random input vectors but they should be the same between \verb+L[m]*d+ and \verb+L'[d]*m+.
\item Now you are ready for testing least-squares migration. Uncomment the corresponding lines in \texttt{SConstruct} and run
\begin{verbatim}
scons invs.view
\end{verbatim}
Do you notice any difference with the previous result? Increase the
number of iterations from 10 to 100 and repeat the experiment.  
\item Include your figures in this document by 
      editing \texttt{hw5/paper.tex}.
\end{enumerate}

\newpage

\lstset{language=python,numbers=left,numberstyle=\tiny,showstringspaces=false}
\lstinputlisting[frame=single]{lsmig/SConstruct}

\lstset{language=c,numbers=left,numberstyle=\tiny,showstringspaces=false}
\lstinputlisting[frame=single]{lsmig/phaseshift.c}

\end{enumerate}

\newpage

\section{Completing the assignment}

\begin{enumerate}
\item Change directory to \texttt{hw5}.
\item Edit the file \texttt{paper.tex} in your favorite editor and change the first line to have your name instead of Fourier's.
\item Run
\begin{verbatim}
sftour scons lock
\end{verbatim}
to update all figures.
\item Run
\begin{verbatim}
sftour scons -c
\end{verbatim}
to remove intermediate files.
\item Run
\begin{verbatim} 
scons pdf
\end{verbatim}
to create the final document.
\item Submit your result (file \texttt{paper.pdf}) on paper or by
  e-mail. 
\end{enumerate}
