\author{Charles Hewitt Dix} 
%%%%%%%%%%%%%%%%%%%%%%%%%%%
\title{Homework 6}

\begin{abstract}
  This homework has only a computational part, with three different
  tasks. You will experiment with exploding-reflector modeling and
  reverse-time migration and will process a field dataset from the US
  East Coast.
\end{abstract}

Completing the computational part of this homework assignment requires
\begin{itemize}
\item \texttt{Madagascar} software environment available from \\
  \url{http://www.ahay.org}
\item \LaTeX\ environment with \texttt{SEGTeX} available from \\ 
  \url{http://www.ahay.org/wiki/SEGTeX}
\end{itemize}

You are welcome to do the assignment on your personal computer by
installing the required environments. In this case, you can obtain all
homework assignments from the \texttt{Madagascar} repository by running
\begin{verbatim}
svn co http://svn.code.sf.net/p/rsf/code/trunk/book/geo384w/hw6
\end{verbatim}

\section{Computational part}

\begin{enumerate}

\item We start by returning to the model from Homeworks 1 and 5 with a
  hyperbolic reflector under a constant velocity layer.  The model is
  shown in Figure~\ref{fig:model}. Now we will approach the imaging
  task using reverse-time migration with a two-way wave
  extrapolation. Figure~\ref{fig:data} shows synthetic zero-offset
  data generated by Kirchhoff modeling. Figure~\ref{fig:image} shows
  an image generated by zero-offset reverse-time migration using an
  explicit finite-difference wave extrapolation in time.

   \textbf{Your task:} Change the program for reverse-time migration to
  implement forward-time modeling using the ``exploding reflector''
  approach.

    \begin{enumerate}
    \item Change directory 
\begin{verbatim}
cd hw6/hyper
\end{verbatim}
    \item Run
\begin{verbatim}
scons view
\end{verbatim}
      to generate figures and display them on your screen.
  \item Run
\begin{verbatim}
scons wave.vpl
\end{verbatim}
      to observe a movie of reverse-time wave extrapolation.
    \item Edit the program in the \texttt{rtm.c} file to implement a process opposite to migration: starting from the reflectivity image like the one in Figure~\ref{fig:image} and generating zero-offset data like the one in Figure~\ref{fig:data}. 
    \item Run
\begin{verbatim}
scons view
\end{verbatim}
      again to observe the differences.
    \end{enumerate}
 
\inputdir{hyper}

\plot{model}{width=\textwidth}{Synthetic velocity model with a hyperbolic reflector.}

\multiplot{2}{data,image}{width=0.8\textwidth}{(a) Synthetic zero-offset data corresponding to the model in Figure~\ref{fig:model}. (b) Image generated by reverse-time exploding-reflector migration. The location of the exact reflector is indicated by a curve.}

\lstset{language=python,numbers=left,numberstyle=\tiny,showstringspaces=false}
\lstinputlisting[frame=single]{hyper/SConstruct}

\lstset{language=c,numbers=left,numberstyle=\tiny,showstringspaces=false}
\lstinputlisting[frame=single]{hyper/rtm.c}

\item Figure~\ref{fig:zvel,zref} shows the Sigsbee velocity model \cite[]{SEG-2002-21222125} and its an approximate filtered reflectivity. 

\inputdir{sigsbee}

\multiplot{2}{zvel,zref}{width=\textwidth}{(a) Sigsbee velocity model. (b) Approximate reflectivity of the Sigsbee model (an ideal image).}

  \textbf{Your task:} Apply your exploding-reflector modeling and migration program from the previous task to generate zero-offset data for Sigsbee and image it.

    \begin{enumerate}
    \item Change directory 
\begin{verbatim}
cd hw6/sigsbee
\end{verbatim}
    \item Run
\begin{verbatim}
scons view
\end{verbatim}
      to generate figures and display them on your screen.  
    \item Modify the \texttt{SConstruct} file to implement the modeling and migration experiment.
    \item Include your results in the paper by editing the \texttt{hw6/paper.tex} file.
\item For EXTRA CREDIT, make sure that your modeling and migration are adjoint to each other and apply least-squares exploding-reflector reverse-time migration to the Sigsbee model. 
    \end{enumerate}

\lstset{language=python,numbers=left,numberstyle=\tiny,showstringspaces=false}
\lstinputlisting[frame=single]{sigsbee/SConstruct}


\item In the last part of the homework, 
  you will work with a field dataset: a 2-D line
  from the Blake Outer Ridge area offshore Florida and Georgia
  (Figure~\ref{fig:map2}). It was collected by USGS in order to study
  the occurrence of methane hydrates. The presence of gas hydrates is
  manifested by a so-called BSR (bottom-simulating reflector).  The
  dataset and its analysis for gas hydrate detection are described by
  \cite{GEO63-05-16591669,GEO65-02-05650573}.

\inputdir{.}
\sideplot{map2}{width=\textwidth}{Map of the Blake Outer Ridge area. An region of known gas hydrate distribution as mapped from seismic bottom simulating reflectors (BSR) is highlighted. The seismic survey area is marked by a rectangle.}

The following figures show the dataset at different stages of seismic
data processing: from initial data to an image in depth.  

\textbf{Your task:} Modify the data processing sequence to create 
a justifiably better image.

\inputdir{blake}
\plot{nmo}{width=\textwidth}{Common midpoint gather (left),
  velocity analysis panel using normal moveout (middle), and common
  midpoint gather after normal moveout (right). A curve in the middle plot
  indicates an automatically picked velocity trend.}

\multiplot{2}{noff,stack}{width=0.8\textwidth}{(a) Near-offset
  section. (b) Normal moveout stack.}

\multiplot{2}{picks,vel}{width=0.8\textwidth}{(a) Picked stacking
  velocity. (b) Interval velocity estimated by Dix inversion.}

\plot{image}{width=\textwidth}{Seismic image created by converting
  stacked data from time to depth. Can you identify geological
  features that are not properly imaged?}

\lstset{language=python,numbers=left,numberstyle=\tiny,showstringspaces=false}
 \lstinputlisting[frame=single]{blake/SConstruct}

\begin{enumerate}
\item Change directory 
\begin{verbatim}
cd hw6/blake
\end{verbatim}
\item Run
\begin{verbatim}
scons view
\end{verbatim}
to generate figures and display them on your screen.  
\item Modify the \texttt{SConstruct} file.
Check your results by running
\begin{verbatim}
scons view
\end{verbatim}
again.
\end{enumerate}

\end{enumerate}

\section{Completing the assignment}

\begin{enumerate}
\item Change directory to \texttt{hw6}.
\item Edit the file \texttt{paper.tex} in your favorite editor and change the first line to have your name instead of Dix's.
\item Run
\begin{verbatim}
sftour pscons lock
\end{verbatim}
to update all figures.
\item Run
\begin{verbatim}
sftour pscons -c
\end{verbatim}
to remove intermediate files.
\item Run
\begin{verbatim} 
scons pdf
\end{verbatim}
to create the final document.
\item Submit your result (file \texttt{paper.pdf}) on paper or by
  e-mail. 
\end{enumerate}

\bibliographystyle{seg} 
\bibliography{SEG}

