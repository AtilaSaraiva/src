\author{Isaac Newton} 
%%%%%%%%%%%%%%%%%%%%%%
\title{Homework 1}

\begin{abstract}
  This homework has three parts. In the theoretical part, you will
  derive some new forms of ray tracing equations and their solutions.
  In the computational part, you will experiment with wave propagation
  in a simple synthetic model. In the programming part, you will modify
  the wave modeling program.
\end{abstract}

\section{Prerequisites}

Completing the computational part of this homework assignment requires
\begin{itemize}
\item \texttt{Madagascar} software environment available from \\
\url{http://rsf.sourceforge.net}
\item \LaTeX\ environment with \texttt{SEGTeX} available from \\ 
\url{http://segtex.sourceforge.net}
\end{itemize}

You are welcome to do the assignment on your personal computer by
installing the required environments. In this case, you can obtain all
homework assignments from the \texttt{Madagascar} repository by running
\begin{verbatim}
svn co https://rsf.svn.sourceforge.net/svnroot/rsf/trunk/book/geo384w 
\end{verbatim}

%The necessary environment is also installed ion
%all Unix computers in the Department of Geological Sciences.

\section{Theoretical part}

You can either write your answers on paper or edit them in the file
\verb#hw1/paper.tex#. Please show all the mathematical
derivations that you perform.

\begin{enumerate}
\item In class, we used a mysterious parameter $\sigma$ to represent a
  variable continuously increasing along a ray. There are other
  variables that can play a similar role.
  \begin{enumerate}
  \item Transform the isotropic ray tracing system
    \begin{eqnarray}
      \label{eq:psigma}
      \frac{d \mathbf{p}}{d \sigma} & = & S(\mathbf{x})\,\nabla S \\
      \label{eq:xsigma}
      \frac{d \mathbf{x}}{d \sigma} & = & \mathbf{p} \\
      \label{eq:tsigma}
      \frac{d T}{d \sigma} & = & S^2(\mathbf{x})
    \end{eqnarray}
    into an equivalent system that uses $\xi$ instead of
    $\sigma$, where $\xi$ is constrained by equation~(\ref{eq:pxi}):
    \begin{eqnarray} 
      \label{eq:pxi}
      \frac{d \mathbf{p}}{d \xi} & = & \frac{\nabla S}{S^2(\mathbf{x})} \\
      \label{eq:xxi}
      \frac{d \mathbf{x}}{d \xi} & = & \\ 
      \label{eq:txi}
      \frac{d T}{d \xi} & = & 
    \end{eqnarray}
    What are the physical dimensions of $\xi$?
  \item Suppose that you are given $T(\mathbf{x})$ -- the traveltime from
    the source to all points $\mathbf{x}$ in the domain of
    interest. Your task is to find $\xi(\mathbf{x})$ for all
    $\mathbf{x}$. Derive a first-order partial differential equation
    that connects $\nabla \xi$ and $\nabla T$.
  \end{enumerate}

\item The so-called ``parabolic'' or $15^{\circ}$ eikonal
  equation \cite[]{tappert,iei,bamberger} has the form
  \begin{equation}
    {\frac{\partial T}{\partial x_1}} + 
    {\frac{1}{2\,S(\mathbf{x})}\,\left(\frac{\partial T}{\partial x_2}\right)^2} =
    {S(\mathbf{x})}
    \label{eq:pareik}
  \end{equation}
  where $\mathbf{x}=\{x_1,x_2\}$ is a point in space, $T(\mathbf{x})$
  is the traveltime, and $S(\mathbf{x})$ is slowness. 
  \begin{enumerate}
  \item Derive the ray tracing system for equation~(\ref{eq:pareik})
    \begin{eqnarray}
      \label{eq:xt}
      \frac{d x_2}{d x_1} & = & \hspace{5in} \\
      \label{eq:pxt}
      \frac{d p_1}{d x_1} & = & \\
      \label{eq:pzt}
      \frac{d p_2}{d x_1} & = & \\
      \label{eq:tt}
      \frac{d T}{d x_1} & = &
    \end{eqnarray}
    where $p_1$ represents $\partial T/\partial x_1$ and 
    $p_2$ represents $\partial T/\partial x_2$.
  \item Assuming constant slowness $S(\mathbf{x}) \equiv S_0$, solve
    the ray tracing system for a point source at the origin
    $\mathbf{x} = \{0,0\}$.
  \item Using the ray tracing solution, find the shape of the
    wavefronts defined by equation~(\ref{eq:pareik} in the case of a
    constant slowness.
  \item The isotropic eikonal equation
    \begin{equation}
      \label{eq:iso}
      \left(\frac{\partial T}{\partial x_1}\right)^2 +
      \left(\frac{\partial T}{\partial x_2}\right)^2 = S^2(\mathbf{x})
    \end{equation}
    describes wavefronts of the wave equation
    \begin{equation}
      \label{eq:isowave}
      S^2(\mathbf{x})\,\frac{\partial^2 P}{\partial t^2} =
      \nabla^2 P + \cdots
    \end{equation}
    with omitted possible first- and zero-order terms. 
    What wave equation corresponds to equation~(\ref{eq:pareik})?
    \begin{equation}
      S(\mathbf{x})\,{\frac{\partial^2 P}{\partial t^2}} =
      \label{eq:parwave}
    \end{equation}
  \end{enumerate}
\end{enumerate}

\section{Computational part}
\inputdir{wave}

In this part, we will simulate and observe acoustic wave propagation
in a simple velocity model shown in Figure~\ref{fig:model}. A wave
snapshot is shown in Figure~\ref{fig:snap}.

\multiplot{2}{model,snap}{width=0.8\textwidth}{(a) Velocity model for
  simple wave propagation experiments. (b) Wave snapshot with overlaid
  first-arrival wavefront.}

\begin{enumerate}
\item Change directory to \verb#geo384w/hw1/wave#
\item Run
\begin{verbatim}
scons model.view
\end{verbatim}
to generate and view the velocity model from Figure~\ref{fig:model}.
\item Run
\begin{verbatim}
scons wave.vpl
\end{verbatim}
to generate and observe a propagating wave on your screen.
\item Run
\begin{verbatim}
scons fronts.vpl
\end{verbatim}
to generate and observe a propagating first-arrival wavefront on your screen.
\item Run
\begin{verbatim}
scons snap.view
\end{verbatim}
  to generate and view a wave snapshot selected at 1~s as shown in
  Figure~\ref{fig:snap}.
\item Open the file \texttt{SConstruct} in your favorite editor. Your
  task is to make the following modifications in it:
\begin{itemize}
\item Find a parameter responsible for selecting the time frame for
  the snapshot in Figure~\ref{fig:snap}.  Modify it to increase the
  time from 1~s to your favorite point in the movie.  Run
  \texttt{scons snap.view} again to verify your change.
\item Can you observe a geometrical part of the wave that is not
  captured by the first-arrival wavefront?  What is its physical
  meaning?
\item Find a parameter in the \texttt{SConstruct} file responsible for
  the vertical smoothness of the model in Figure~\ref{fig:model}.
  Modify it to increase the smoothness of the model in such a way that
  the first-arrival wavefront follows the wave geometry exactly.  Run
  \texttt{scons snap.view} again to verify your change.
\end{itemize}
\end{enumerate}

\lstset{language=python,numbers=left,numberstyle=\tiny,showstringspaces=false}
\lstinputlisting[frame=single]{wave/SConstruct}

\section{Programming part (extra credit)}
\inputdir{code}

For extra credit, you can modify the wave modeling program to include
anisotropic wave propagation effects. The program below implements
wave modeling with the constant-density acoustic wave equation
\begin{equation}
   {{\frac{1}{V^2(\mathbf{x})}}\,{\frac{\partial^2 P}{\partial t^2}}} =
   {\nabla^2 P} + {F(\mathbf{x},t)}\;,
   \label{eq:cdens}
\end{equation} 
where $F(\mathbf{x},t)$ is the source term. The implementation uses
finite-difference discretization (second-order in time and fourth-order in space). 
Stepping in time involves the following computations:
\begin{equation}
  \mathbf{P}_{t+\Delta t} = \left[ \nabla^2 \mathbf{P}_t + F(\mathbf{x},t)\right] V^2(\mathbf{x}) \Delta t^2 + 2 \mathbf{P}_{t} - \mathbf{P}_{t-\Delta t} \;,
  \label{eq:step}
\end{equation}
where $\mathbf{P}$ represents the propagating wavefield discretized at different time steps.

\lstset{language=c,numbers=left,numberstyle=\tiny,showstringspaces=false}
\lstinputlisting[frame=single]{wave/wave.c}

Your task is to modify the code to implement the variable-density acoustic wave equation
\begin{equation}
  {{\frac{1}{V^2(\mathbf{x})}}\,{\frac{\partial^2 P}{\partial t^2}}} =
  {\rho(\mathbf{x})\, \nabla \cdot 
  \left(\frac{1}{\rho(\mathbf{x})}\,\nabla P\right)}\;.
  \label{eq:vdens}
\end{equation} 

Test your implementation by modifying the \texttt{SConstruct} file to
generate a variable density and the corresponding wavefields. Include
a picture of the variable-density wave snapshot analogous to Figure~\ref{fig:snap}.

\section{Completing the assignment}

\begin{enumerate}
\item Change directory to \verb#geo384w/hw1#.
\item Edit the file \texttt{paper.tex} in your favorite editor and change the
first line to have your name instead of Newton's.
\item Run
\begin{verbatim}
sftour scons lock
\end{verbatim}
to update all figures.
\item Run
\begin{verbatim}
sftour scons -c
\end{verbatim}
  to remove intermediate files.
\item Run
\begin{verbatim} 
scons pdf
\end{verbatim}
to create the final document.
\item Submit your result (file \texttt{paper.pdf}) on paper or by
  e-mail.
\end{enumerate}

\bibliographystyle{seg}
\bibliography{wave}
