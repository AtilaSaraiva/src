\author{Isaac Newton} 
%%%%%%%%%%%%%%%%%%%%%%
\title{Homework 1}


\begin{abstract}
  This homework has three parts. In the theoretical part, you will
  derive some new forms of ray tracing equations and their solutions.
  In the computational part, you will experiment with wave propagation
  in a simple synthetic model.  In the programming part, you will
  modify a finite-difference wave modeling program.
\end{abstract}

\section{Prerequisites}

Completing the computational part of this homework assignment requires
\begin{itemize}
\item \texttt{Madagascar} software environment available from \\
\url{http://www.ahay.org}
\item \LaTeX\ environment with \texttt{SEGTeX} available from \\ 
\url{http://www.ahay.org/wiki/SEGTeX}
\end{itemize}

You are welcome to do the assignment on your personal computer by
installing the required environments. In this case, you can obtain all
homework assignments from the \texttt{Madagascar} repository by running
\begin{verbatim}
svn co http://svn.code.sf.net/p/rsf/code/trunk/book/geo384w/hw1
\end{verbatim}

You can also do this assignment in the computer lab at the Department
of Geological Sciences (JGB 3.216B).

\section{Theoretical part}

You can either write your answers on paper or edit them in the file
\verb#hw1/paper.tex#. Please show all the mathematical
derivations that you perform.

\begin{enumerate}
\item In class, we used a mysterious parameter $\sigma$ to represent a
  variable continuously increasing along a ray. There are other
  variables that can play a similar role.
  \begin{enumerate}
  \item Transform the isotropic ray tracing system
    \begin{eqnarray}
      \label{eq:psigma}
      \frac{d \mathbf{p}}{d \sigma} & = & S(\mathbf{x})\,\nabla S \\
      \label{eq:xsigma}
      \frac{d \mathbf{x}}{d \sigma} & = & \mathbf{p} \\
      \label{eq:tsigma}
      \frac{d T}{d \sigma} & = & S^2(\mathbf{x})
    \end{eqnarray}
    into an equivalent system that uses $\xi$ instead of
    $\sigma$, where $\xi$ is constrained by equation~(\ref{eq:pxi}):
    \begin{eqnarray} 
      \label{eq:pxi}
      \frac{d \mathbf{p}}{d \xi} & = & \frac{\nabla S}{S^2(\mathbf{x})} \\
      \label{eq:xxi}
      \frac{d \mathbf{x}}{d \xi} & = & \\ 
      \label{eq:txi}
      \frac{d T}{d \xi} & = & 
    \end{eqnarray}
What are the physical units of $\xi$?      
  \item Suppose you are given $T(\mathbf{x})$ -- the traveltime from
  the source to all points $\mathbf{x}$ in the domain of
  interest. Your task is to find $\xi(\mathbf{x})$ for all $\mathbf{x}$. Derive a first-order partial
  differential equation that connects $\nabla \xi$ and $\nabla T$.
  \end{enumerate}

\item The so-called ``parabolic'' or $15^{\circ}$ eikonal
  equation \cite[]{tappert,iei,bamberger} has the form
  \begin{equation}
    {\frac{\partial T}{\partial x_1}} + 
    {\frac{1}{2\,S(\mathbf{x})}\,\left(\frac{\partial T}{\partial x_2}\right)^2} =
    {S(\mathbf{x})}
    \label{eq:pareik}
  \end{equation}
  where $\mathbf{x}=\{x_1,x_2\}$ is a point in space, $T(\mathbf{x})$
  is the traveltime, and $S(\mathbf{x})$ is slowness. 
  \begin{enumerate}
  \item Derive the ray tracing system for equation~(\ref{eq:pareik})
    \begin{eqnarray}
      \label{eq:xt}
      \frac{d x_2}{d x_1} & = & \hspace{5in} \\
      \label{eq:pxt}
      \frac{d p_1}{d x_1} & = & \\
      \label{eq:pzt}
      \frac{d p_2}{d x_1} & = & \\
      \label{eq:tt}
      \frac{d T}{d x_1} & = &
    \end{eqnarray}
    where $p_1$ represents $\partial T/\partial x_1$ and 
    $p_2$ represents $\partial T/\partial x_2$.
   
   \item Assuming constant slowness $S(\mathbf{x}) \equiv S_0$, solve
    the ray tracing system for a point source at the origin
    $\{x_1,x_2\} = \{0,0\}$.

   \item Using the ray tracing solution, find the shape of the
    wavefronts defined by equation~(\ref{eq:pareik} in the case of a
    constant slowness.

   \item The isotropic eikonal equation
    \begin{equation}
      \label{eq:iso}
      \left(\frac{\partial T}{\partial x_1}\right)^2 +
      \left(\frac{\partial T}{\partial x_2}\right)^2 = S^2(\mathbf{x})
    \end{equation}
    describes wavefronts of the wave equation
    \begin{equation}
      \label{eq:isowave}
      S^2(\mathbf{x})\,\frac{\partial^2 P}{\partial t^2} =
      \nabla^2 P + \cdots
    \end{equation}
    with omitted possible first- and zero-order terms. 
    What wave equation corresponds to equation~(\ref{eq:pareik})?
    \begin{equation}
      S(\mathbf{x})\,{\frac{\partial^2 P}{\partial t^2}} =
      \label{eq:parwave}
    \end{equation}

  \end{enumerate}
\end{enumerate}

\section{Computational part}
\inputdir{wave}

In this part, we will simulate and observe acoustic wave propagation
in a simple velocity model shown in Figure~\ref{fig:model}. A wave
snapshot is shown in Figure~\ref{fig:snap}.

\multiplot{2}{model,snap}{width=0.8\textwidth}{(a) Velocity model for 
simple wave propagation experiments. (b) Wave snapshot with overlaid
first-arrival wavefront.}

\begin{enumerate}
\item Change directory to \verb#geo384w/hw1/wave#
\item Run
\begin{verbatim}
scons model.view
\end{verbatim}
to generate and view the velocity model from Figure~\ref{fig:model}.
\item Run
\begin{verbatim}
scons wave.vpl
\end{verbatim}
to generate and observe a propagating wave on your screen.
\item Run
\begin{verbatim}
scons fronts.vpl
\end{verbatim}
to generate and observe a propagating first-arrival wavefront on your screen.
\item Run
\begin{verbatim}
scons snap.view
\end{verbatim}
to generate and view a wave snapshot selected at 1~s as shown in
Figure~\ref{fig:snap}.
\item Open the file \texttt{SConstruct} in your favorite editor. Your task is to make the following modifications in it:
\begin{itemize}
\item Find a parameter responsible for selecting the time frame for the snapshot in Figure~\ref{fig:snap}. 
Modify it to increase the time from 1~s to your favorite point in the
movie. Run \texttt{scons snap.view} again to verify your change.
\item Can you observe a geometrical part of the wave that is not captured by the first-arrival wavefront? What is its physical meaning?
\item Find a parameter in the \texttt{SConstruct} file 
responsible for the vertical smoothness of the model in Figure~\ref{fig:model}. Modify it to increase the smoothness of the model in such a way 
that the first-arrival wavefront follows the wave geometry exactly. Run \texttt{scons snap.view} again to verify your change.
\item (EXTRA CREDIT) For extra credit, modify  \texttt{SConstruct} to generate a movie of pictures like Figure~\ref{fig:snap} for a gradually increasing smoothness.
\end{itemize}
\end{enumerate}

\lstset{language=python,numbers=left,numberstyle=\tiny,showstringspaces=false}
\lstinputlisting[frame=single]{wave/SConstruct}

\section{Programming part (extra credit)}
\inputdir{code}

For extra credit, you can modify the wave modeling program to include
anisotropic wave propagation effects. The program below (slightly
modified from the original version by Paul Sava) implements wave
modeling with equation
\begin{equation}
   \label{eq:isowave2}
   {\frac{\partial^2 P}{\partial t^2}} =
   V^2(\mathbf{x})\,{\nabla^2 P} + {F(\mathbf{x},t)} = V^2(\mathbf{x})\,\left(\frac{\partial^2 P}{\partial x_1^2} +\frac{\partial^2 P}{\partial x_2^2}\right) +  {F(\mathbf{x},t)}\;,
\end{equation} 
where $F(\mathbf{x},t)$ is the source term. The implementation uses
finite-difference discretization (second-order in time and fourth-order in space). 
Stepping in time involves the following computations:
\begin{equation}
\mathbf{P}_{t+\Delta t} = \left[ V^2(\mathbf{x})\,\nabla^2 \mathbf{P}_t + F(\mathbf{x},t)\right]\,\Delta t^2 + 2 \mathbf{P}_{t} - \mathbf{P}_{t-\Delta t} \;,
\label{eq:step}
\end{equation}
where $\mathbf{P}$ represents the propagating wavefield discretized at different time steps.

\lstset{language=c,numbers=left,numberstyle=\tiny,showstringspaces=false}
\lstinputlisting[frame=single]{code/wave.c}

Your task is to modify the code to implement elliptically-anisotropic wave propagation according to equation
\begin{equation}
      {\frac{\partial^2 P}{\partial t^2}} = 
      V_1^2(\mathbf{x})\,\frac{\partial^2 P}{\partial x_1^2} +
      V_2^2(\mathbf{x})\,\frac{\partial^2 P}{\partial x_2^2} + F(\mathbf{x},t)
      \label{eq:anisowave2}
\end{equation}
You can test your implementation using a constant velocity example shown in Figure~\ref{fig:wave}.

\sideplot{wave}{width=\textwidth}{Wavefield snapshot for propagation from a point-source in a homogeneous medium. Modify the code to make wave propagation anisotropic.}

\begin{enumerate}
\item Change directory to \verb#geo384w/hw1/code#
\item Run
\begin{verbatim}
scons wave.vpl
\end{verbatim}
to compile and run the program and to observe a propagating wave on your screen.
\item Open the file \texttt{wave.c} in your favorite editor and modify it to implement the wave operator from equation~(\ref{eq:anisowave}).
\item Run
\begin{verbatim}
scons wave.vpl
\end{verbatim}
again to compile and test your program. 
\item (EXTRA EXTRA CREDIT) For an additional test, modify the file \texttt{SConstruct} and run appropriate commands to output a snapshot of wave propagation in the Hess VTI model, shown in Figure~\ref{fig:vp,vx}\footnote{The Hess VTI model was generated at Hess Corporation and released at \url{http://software.seg.org}. For this exercise, we are going to approximate it with an elliptically anisotropic model}. Modify the file \texttt{paper.tex} to include your figure.
\end{enumerate}

\multiplot{2}{vp,vx}{width=0.45\textwidth}{Vertical velocity (a) and horizontal velocity (b) in the Hess VTI model.}

\lstset{language=python,numbers=left,numberstyle=\tiny,showstringspaces=false}
\lstinputlisting[frame=single]{code/SConstruct}

\newpage

\section{Completing the assignment}

\begin{enumerate}
\item Change directory to \verb#geo384w/hw1#.
\item Edit the file \texttt{paper.tex} in your favorite editor and change the
first line to have your name instead of Newton's.
\item Run
\begin{verbatim}
sftour scons lock
\end{verbatim}
to update all figures.
\item Run
\begin{verbatim}
sftour scons -c
\end{verbatim}
  to remove intermediate files.
\item Run
\begin{verbatim} 
scons pdf
\end{verbatim}
to create the final document.
\item Submit your result (file \texttt{paper.pdf}) on paper or by
e-mail.
\end{enumerate}

\bibliographystyle{seg}
\bibliography{wave}
