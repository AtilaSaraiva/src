\lefthead{Thongsang, et al}
\righthead{Marmoisi2}
\footer{Modeling and Processing with Madagascar}

\title{2D Modeling and Basic Processing with Madagascar}   
\author{Pongthep Thongsang, Lian Jiang, Hao Li, and Karl Schleicher}

\email{k\_schleicher@hotmail.com}

\address{
k\_schleicher@hotmail.com \\
John A. and Katherine G. Jackson School of Geosciences \\
The University of Texas at Austin \\
University Station, Box X \\
Austin, TX 78713-8924}

\maketitle

\begin{abstract}
This document demonstrates finite difference modeling and simple processing
using the Madagascar software package.  The paper uses this simple processing 
sequence to teach the basics of Madagascar. 

The segy file of the Marmousi2 model is downloaded from the Internet and used
to create a few synthetic shotpoints.  Geometry is computed and loaded in the
trace headers.   Simple processing including amplitude correction, CMP
sorting, velocity conversions, NMO correction, stack, Kirchhoff migration, and
least squares Kirchhoff migration are applied.  Displays of each processing
stage are created.  The full processing sequence takes less that 6 minutes on
a 2013 MacBook Pro computer. 

This processing sequence was developed at the 2017 Seismic Working Workshop - 
Reproducible Tutorials in Houston.  Processing can be reproduced using an 
SConstruct file.

\end{abstract}

\section{Introduction}
There are several Madagascar tutorials.  This tutorial differs from the other 
tutorials by focusing on the using existing Madagascar programs to create 
a reflection seismic synthetic and processing it using basic, standard 
techniques.  The computation runs quickly which makes it easy to 
modify the SConstruct files and run new experiments.  Processing can be 
reproduced using an SConstruct file listed in the appendix or by using the file:

 \$RSFSRC/book/data/mourmousi2-model-proc/model-proc/SConstruct 

in the Madagascar distribution.

\section{Download, Reformat, Smooth, and Display the Marmousi2 Velocity Model}
We use the p-wave velocity field from the Marmousi2 model described by Irons 
\cite[]{irons}.  Figure \ref{fig:vp} shows the part of the model used in this 
paper.  Figures \ref{fig:vpforward} and \ref{fig:vpmigration} show the model 
smoothed for modeling and processing (NMO and migration). 

These figures can be created with the command:

scons vp.view vpforward.view vpmigration.view

\inputdir{model-proc}
\plot{vp}{width=\textwidth}{A portion of Marmousi2 velocity used in this paper.}
\plot{vpforward}{width=\textwidth}{The velocity smoothed for modeling.}
\plot{vpmigration}{width=\textwidth}{The interval velocity smoothed for NMO and migration. }

\section{Finite Difference Model Shotpoints, Model and subtract the direct arrivals}

The wave equation program sfmodeling2d \cite[]{Yang} is used to create 10 
synthetic shot records using the slightly smoothed Marmousi model.   One 
synthetic shotpoint is show in Figure \ref{fig:shot2500}. Sfmodeling2d was also
run with a constant velocity to model the direct arrivals.  The direct arrivals
were subtracted from the synthetic and shown in Figure 
\ref{fig:mutearr2500} 

These figures and some others can be created with the command:

scons shots.view shot2500.view mutearr.view mutearr2500.view

\plot{shot2500}{width=\textwidth}{Shotpoint at x=2500 computed with acoustic 
finite difference.}
\plot{mutearr2500}{width=\textwidth}{Shotpoint at x=2500 after subtracting the
modelled direct arrivals.}

\section{Compute RMS Velocities, NMO, T-shift, Amplitude Correction, Stack}

The basic trace processing stages of sorting to CMP gathers, time shifting, 
amplitude correction, normal move out, and stack are applied using the trace 
and header (tah) programs.  These programs all start with ‘sftah’.  The 
sftahheadermath program was used to load the geometry into the trace headers.  
The sftahnmo program requires an RMS velocity in m/s at each input CMP and each 
time sample.  This is computed from the original interval velocity model by 
stretching to time, squaring the velocity, integrating, dividing by time, and 
taking the square root using the programs sfdepth2time, sfmul, sfcausint, and 
sfmath.  This creates an RMS velocity every 12.49 meters.  Intermediate traces 
are created using sfinterleave.  The stack is computed by sorting to CMP order,
removing the 250 ms delay in the modeling Ricker wavelet, NMO and stack, 
(Figure \ref{fig:stack}).

This processing can be run with the command:

scons stack.view 

\plot{stack}{width=\textwidth}{CDP stack of 10 shotpoints.}

\section{Post stack Kirchhoff Migration and Least Squares Kirchhoff Migration}

The stack and RMS velocity were input to sfkirchnew to apply Post stack
Kirchhoff migration (Figure \ref{fig:ktmig}).  Least squares Kirchhoff 
migration was computed using the sfkirchinvs program (Figure 
\ref{fig:invsparse}).  

These processing can be reproduced with the command:

scons ktmig.view invsparse.view

\plot{ktmig}{width=\textwidth}{Post stack Kirchhoff migration.}
\plot{invsparse}{width=\textwidth}{Least squares Kirchhoff migration.}

\section{Acknowledgements}
This processing sequence was developed at the 2017 Seismic Working Workshop - 
Reproducible Tutorials in Houston \cite[]{schleicher}.  This working workshop 
was supported by Wavelets (The University of Houston student SEG chapter), 
The Bureau of Economic Geology at the University of Texas, and SEG.

We thank Allied Geophysical Laboratories (AGL) for creating and sharing the 
Marmousi2 model and synthetic data.  http://www.agl.uh.edu/

\newpage
\section{Appendix}

The file \emph{\$RSFSRC/book/data/marmousi2-model-proc/model-proc/SConstruct} 
is an scons script that applies the processing described in this paper.  The 
contents of the file is listed below.

\tiny
\lstinputlisting[frame=single]{model-proc/SConstruct}
\normalsize

\bibliographystyle{seg}
\bibliography{refpaper,SEG}




 
