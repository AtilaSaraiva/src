\title{Sigsbee2 Models}
\author{Trevor Irons}
\shortpaper
\maketitle

\lstset{language=python,numbers=left,numberstyle=\tiny,showstringspaces=false}

\noindent
\textbf	{Data Type:} \emph{2D model and acoustic finite difference synthetic data set with constant density}\\
\textbf	{Source:} \emph{SMAART consortium comprised of BHPBilliton Petroleum, BP, and the ChevronTexaco Exploration 
and Production Technology Company}\\
\textbf {Location:} \emph{http://www.delphi.tudelft.nl/SMAART/sigsbee2a.htm}\\
\textbf	{Format:} \emph{SEGY and Native} \\
\textbf{Date of origin:} \emph{Data were publicly released between September 2001 and November 2002.}\\ 

\section{Introduction}
The Subsalt Multiples Attenuation and Reduction Technology Joint Venture (SMAART JV) publicly released several data sets 
between September 2001 and November 2002.  These synthetic data model the geologic setting found on the Sigsbee escarpment 
in the deep water Gulf of Mexico.  Additional information may be found at: \emph{www.delphi.tudelft.nl/SMAART/}.  
The data sets remain the property of SMAART and are used under the agreement found at the SMAART site listed above.  

The file \emph{sigsbee/FILES} lists all files contained in the Sigsbee2 repository of Madagascar and is reproduced below in
Table \ref{tbl:FILES}.  Any of these files may be downloaded to local machines using ftp protocols.  

The Sigsbee2 data are separated into two distinct categories, A and B.  They share the same general model geometry and structure,
however, the A model has a soft water to seafloor boundary while the B model features a more realistic hard boundary.  As a result 
data produced in the B model features multiple events.  The Sigsbee2B data set was featured in paper SP3.8 "Observations from the 
Sigsbee2B synthetic data set" at the 2002 SEG meeting in Salt Lake City.  

\tabl{FILES}{List of all available files in the Madagascar Sigsbee2 repository} 
{
\tiny
\lstinputlisting[frame=single]{FILES}
\normalsize
}

\section{Sigsbee Models}
This model contains a sedimentary sequence broken up by a number of normal and thrust faults.  Additionally, there is a complex 
salt structure found within the model that results in illumination problems when processing and migrating the data.   

The Sigsbee2A model features an absorbing free surface condition and a weaker than normal water bottom reflection, 
resulting in data do not contain free surface multiples and less than normal internal multiples.  
The Sigsbee 2B model uses the same structural model as Sigsbee2A but the velocity contrast at the water bottom has been 
increased to a normal level thus generating significant internal and free surface multiples. Modeling on the 2B model was 
performed with both free and non free surface boundary conditions.

The Sigsbee2 models found in the Madagascar repository share the same dimensions and sampling rate.  The model is 9.144 km (30 000 ft) in 
depth and 24.384 km (80 000 ft) in length.  All the models contain 7.62 m (25 ft) grid spacing except for the migrated models that have
11.43 m (37.5 ft) lateral grid spacing.  Throughout this article both standard and metric units will be presented in tabular form but all
figures will exclusively utilize metric units.   

Table \ref{tbl:modelHeader} displays the correct values that Sigsbee2 model headers should contain.   

\tabl{modelHeader}{Header information for Sigsbee2 models, note the initial offset in the horizontal direction and the coarse lateral sampling 
of the migrated models.}
{
\begin{tabular}[t]{|llllll|}
        \hline
	\textbf{Stratigraphic Models}	& & & & &                                     \\
	\textit{Standard Units}         & & & & &			              \\ 
        n1=1201  &   d1=25	&  o1=0     &    label1=Depth       &   unit1=ft   &  \\
	n2=3201	 &   d2=25 	&  o2=10000 &    label2=Distance    &   unit2=ft   &  \\
	\textit{Metric Units}           & & & & &                                     \\
	n1=1201  &   d1=.00762  &  o1=0     &    label1=Depth       &   unit1=km   &  \\
	n2=3201  &   d2=.00762  &  o2=3.048 &    label2=Distance    &   unit2=km   &  \\
	\textbf{Migrated Models}        & & & & &                                     \\
	\textit{Standard Units}         & & & & &                                     \\
	n1=1201  &   d1=25      &  o1=0     &    label1=Depth       &   unit1=ft   &  \\
	n2=2133  &   d2=37.5    &  o2=10025 &    label2=Distance    &   unit2=ft   &  \\
	\textit{Metric Units}           & & & & &                                     \\
	n1=1201  &   d1=.00762  &  o1=0     &    label1=Depth       &   unit1=km   &  \\
	n2=2133  &   d2=.0143  &  o2=3.055 &    label2=Distance    &   unit2=km   &  \\ 
        \hline
\end{tabular}
}

\subsection{Sigsbee 2A Models}
The Sigsbee2A velocity and reflection coefficient models are easily viewed using Madagascar.  There are 2 velocity models, a smooth 
migrated model and a true stratigraphic model.  The SCons script \emph{sigsbee/model2A/SConstruct} contains a set of rules that tell Madagascar
to fetch the data append the headers and make several plots.  This script is reproduced in Table \ref{tbl:2ASConstruct}.    

\tabl{2ASConstruct}{Contents of \emph{model2A/SConstruct} script.}
{
\tiny
\lstinputlisting[frame=single]{model2A/SConstruct}
\normalsize
}

Typing Command \ref{eq:SCvel} within the \emph{sigsbee/model2A} directory runs the script.
\begin{equation}\label{eq:SCvel} \texttt{bash-3.1\$\ scons\ view} \end{equation}

The Sigsbee2A migrated and stratigraphic velocity models are shown in Figures \ref{fig:vmig2A} and~\ref{fig:vstr2A} respectively.  
A plot of the reflection coefficients are shown in Figure \ref{fig:reflectionCoefficients}.  

\inputdir{model2A}
\multiplot{3}{vmig2A,vstr2A,reflectionCoefficients}{width=.45\textwidth}{Sigsbee 2A contains a stratigraphic velocity model (a) 
a migrated smoothed model (b) and a reflection coefficient model (c).}

\subsection{Sigsbee 2B Models}
The Sigsbee 2B model contains the same general geometry as the 2A model except for a more realistic water to floor boundary which 
results in multiple generation when shots are modeled on it.  However, dealing with the files is basically identical the headers should 
also be calibrated as shown in Table \ref{tbl:modelHeader}. 

Table \ref{tbl:model2BSConstruct} shows the contents of the \emph{sigsbee/model2b/SConstruct} script.  This file is quite similar to 
the one found in the Sigsbee 2A section and contains a list of rules that fetch the datasets and plot them.  

\inputdir{model2B}
\tabl{model2BSConstruct}{Contents of \emph{model2B/SConstruct} script.}
{
\tiny
\lstinputlisting[frame=single]{model2B/SConstruct}
\normalsize
}

Typing Command \ref{eq:SCvel2} within the \emph{sigsbee/model2B} directory runs the script.
\begin{equation}\label{eq:SCvel2} \texttt{bash-3.1\$\ scons\ view} \end{equation}
 
A plot of the migrated velocity model is shown below Figure \ref{fig:vmig2B} while the stratigraphic model can be seen 
in Figure \ref{fig:vstr2B}.  A plot of the reflection coefficients are shown in Figure \ref{fig:reflectionCoefficients2B}.

\inputdir{model2B}
\multiplot{3}{vstr2B,vmig2B,reflectionCoefficients2B}{width=.45\textwidth}{Sigsbee 2B contains two velocity models, a stratigraphic 
model (a) and a migrated model(b).  The resulting reflection coefficients are shown in (c).}

\section{Shot Records}
Several sets of data were acquired on the Sigsbee models.  The Madagascar repository contains one survey taken on the 
Sigsbee2A model which was performed with an absorbing surface boundary condition.  Two surveys were conducted on the Sigsbee2B  
models one with a free surface boundary and one without.  

The three surveys shared the same acquisition geometry.  Each receiver recorded data every .008 seconds for 1 500 timesteps resulting 
in 12 seconds of data.  A 7 950 m (26 100 ft) long streamer cable was deployed with 348 hydrophones spaced 22.86 m (75 ft) apart.  Shots were fired
every 45.72 m (150 ft) starting at 3 330 m  (10 925 ft).  Table \ref{tbl:shotHeader} shows the values that Sigsbee shot headers should contain.  

\tabl{shotHeader}{Appropriate header values for Sigsbee shot records. The number of shots; n3, varies slightly between the surveys}
{
\begin{tabular}[t]{|llllll|}
        \hline 
	\textbf{Standard}    &  &           &                       &              &  \\
        n1=1500  &   d1=0.008	&  o1=0     &    label1=Time        &   unit1=s    &  \\
	n2=348	 &   d2=75 	&  o2=0     &    label2=Offset      &   unit2=ft   &  \\
	n3=500 or 496   &   d3=150     &  o3=10925 &    label3=Shot-Coord  &   unit3=ft   &  \\
	\textbf{Metric}      &  &           &                       &              &  \\
        n1=1500  &   d1=0.008	&  o1=0     &    label1=Time        &   unit1=s    &  \\
	n2=348	 &   d2=.02286 	&  o2=0     &    label2=Offset      &   unit2=km   &  \\
	n3=500 or 496   &   d3=.04572     &  o3=3.330 &    label3=Shot-Coord  &   unit3=km   &  \\
        \hline
\end{tabular}
}

\subsection{Sigsbee 2A shot records}
The survey performed on the Sigsbee2A model had an infinite surface boundary condition.  The script found at 
\emph{data2A/SConstruct} whose contents are displayed in Table \ref{tbl:data2ASConstruct} generates a Madagascar 
formatted data file \emph{shots.rsf} and also produces several shot images. 
  
\tabl{data2ASConstruct}{Contents of \emph{data2A/SConstruct} script.}
{
\tiny
\lstinputlisting[frame=single]{data2A/SConstruct}
\normalsize
}

Typing Command \ref{eq:shot2A} within the \emph{sigsbee/data2A} directory runs the script.
\begin{equation}\label{eq:shot2A} \texttt{bash-3.1\$\ scons\ view} \end{equation}

A plot of the 70th shot, made 6.5 km in to the model is produced by the \emph{SConstruct} script and is shown below in Figure \ref{fig:shot70}  The zero offset data is presented in Figure \ref{fig:zero}.

\inputdir{data2A}
\plot{shot70}{width=\textwidth}{Snapshot of shot number 70 performed on \emph{sigsbee 2A} the position of the source in km is in the lower left hand corner
of the plot.}
\plot{zero}{width=\textwidth}{Sigsbee2A zero offset data}

\subsection{Sigsbee 2B Shot Records}
The Sigsbee 2B library contains two sets of shot data, \emph{nfs} and \emph{fs}.  These shots were modeled with free and non free surface 
boundary conditions.  

\subsubsection{Free surface model}  
A \emph{SConsctuct} script found at \textit{sigsbee/data2B/fs/} is presented in Table \ref{tbl:fsSConstruct}.  
This script reads the \emph{segy} source file and converts it to Madagascar's \emph{RSF} format and appends the header as 
necessary.  The free surface boundary present within this model allows for the generation of reflections at the model edges.  

\tabl{fsSConstruct}{Contents of \emph{data2B/fs/SConstruct} script.}
{
\tiny
\lstinputlisting[frame=single]{fs2B/SConstruct}
\normalsize
}


Again shot number 70, fired 6.5 km into the model, is plotted in Figure \ref{fig:shot702Bfs}.  The precise coordinates
of the shot are shown in the lower left hand corner of the figure.  The zero offset data is presented in Figure \ref{fig:zero2Bfs}.     

\inputdir{fs2B}
\plot{shot702Bfs}{width=\textwidth}{Shot number 70 performed on \emph{sigsbee 2B FS} model.  The shot location is presented in the lower left hand 
corner of the plot}
\plot{zero2Bfs}{width=\textwidth}{Sigsbee2B free surface boundary zero offset data.}

\subsubsection{Infinite Surface Model}
This data was prepared with the boundaries of the model extending forever; as such multiples are not created as a  result of the model edges.  

A \emph{SConsctuct} script found at \textit{sigsbee/data2B/fs/} is presented in Table \ref{tbl:nfsSConstruct}.  This script 
translates the \emph{segy} source data file and converts it into \emph{rsf} format.  

\tabl{nfsSConstruct}{Contents of \emph{data2B/nfs/SConstruct} script.}
{
\tiny
\lstinputlisting[frame=single]{nfs2B/SConstruct}
\normalsize
}

Similar plots are produced for this model.  Figure \ref{fig:shot702Bnfs} shows an image of shot number 70 taken at 6.5 km into the model.  
Figure \ref{fig:zero2Bnfs} displays the zero offset data acquired on this model.  

\inputdir{nfs2B}
\plot{shot702Bnfs}{width=\textwidth}{Shot 70 performed in \emph{Sigsbee 2B NFS} model.}
\plot{zero2Bnfs}{width=\textwidth}{Sigsbee 2B reflecting surface zero offset data, notice the decreased multiples from the free surface
model}

\section{Finite Difference Modeling}
Finite difference (FD) shot and data modeling can be performed on the Sigsbee models using Madagascar.  This example will use the Sigsbee2A model
but but it could be easily extended to perform modeling on Sigsbee2B.

For the purposes of this example a shot will be fired at 10 km along the horizontal coordinate and at a depth of 10 meters.  Receivers are
spread at a depth of 0 meters every 7.62 m (25 ft) along the entire scope of the model.  This long receiver cable is impractical but useful for these
purposes.  Data is recorded on every receiver at time increments of 1 ms 3000 times resulting in 3 seconds of data.

An \emph{SConstruct} file located within \emph{sigsbee/fdmod2A/} properly formats the model and inputs necessary parameters to perform a shot on
the Sigsbee model.  This file is reproduced below in Table \ref{tbl:fdmodSConstruct}.

\tabl{fdmodSConstruct}{\emph{Scons} script that performs a finite difference synthetic shot on Sigsbee2A.}
{
\tiny
\lstinputlisting[frame=single]{fdmod2A/SConstruct}
\normalsize
}

Typing Command \ref{eq:SCfdmod} within the \emph{sigsbee/fdmod2A/} directory runs the FD modeling script.
\begin{equation}\label{eq:SCfdmod} \texttt{bash-3.1\$\ scons\ view} \end{equation}

This script first constructs the survey acquisition geometry as was previously mentioned.  An image of the survey is created and presented
in Figure \ref{fig:vel}.

\inputdir{fdmod2A}
\plot{vel}{width=\textwidth}{FD model geometry as performed on the Sigsbee 2A velocity model.  The \emph{X} represents the shot while the smaller
\emph{*} symbols represent receivers. The receivers extent along the right hand side although it is not clear in this image.}

Firing the shot results the propagation of a wavefield which can be seen in the movie \emph{wfl.vpl} that is generated.  Typing
Command \ref{eq:viewMov} within the \emph{sigsbee/fdmod2A/Fig} directory displays the wavefield movie.

\begin{equation}\label{eq:viewMov} \texttt{bash-3.1\$\ xtpen\ wfl.vpl} \end{equation}


Four frames from this movie are presented in Figure \ref{fig:time9,time19,time29,time39}  illustrating the
propagation of the wavefield in the model.

\inputdir{fdmod2A}
\multiplot{4}{time9,time19,time29,time39}{width=.45\textwidth}{Images of the propagating wavefield in the Sigsbee model generated by a finite
difference model.}


The resulting data is then presented in the file \emph{dat.vpl}.  This plot is reproduced here in Figure \ref{fig:dat}.  

\plot{dat}{width=\textwidth}{Data gathered by the receivers in the FD model survey.}


FD models can be performed on the Sigsbee2B model in a similar fashion.  The primary change would be in appending line six, the 
model input file, in the \emph{SConstruct} file shown in Table \ref{tbl:fdmodSConstruct}.    
