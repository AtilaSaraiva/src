\title{Marmousi Model}
\author{Trevor Irons}
\lstset{language=python,numbers=left,numberstyle=\tiny,showstringspaces=false}

\maketitle
\noindent
\textbf {Data Type:} \emph{Synthetic 2D acoustic model}\\
\textbf {Source:} \emph{Institut Fran\c{c}ais du P\'{e}trole}\\
\textbf {Location:} \emph{http://www.ifp.fr/IFP/en/aa.htm}\\
\textbf {Format:} \emph{Native, originally Sierra Geophysical Format} \\
\textbf{Date of origin:} \emph{1988}\\

\section{Introduction}
The Marmousi model was created in 1988 by the Institut Fran\c{c}ais du P\'{e}trole (IFP) in 1988.  The geometry of this model is based on a 
profile through the North Quenguela trough in the Cuanza basin. The geometry and velocity model were created to produce complex seismic data 
which require advanced processing techniques to obtain a correct earth image. The Marmousi dataset was used for the workshop on practical 
aspects of seismic data inversion at the 52nd EAEG meeting in 1990.

Since its inception in 1990 Marmousi has come be a sort of industry standard and almost classic dataset. The Madagascar repository contains 
the Marmousi files shown in Table \ref{tbl:FILES}.

\tabl{FILES}{A list of all files contained in the Marmousi repository}
{
\tiny
\lstinputlisting[frame=single]{FILES}
\normalsize
}

\section{Model}
The Marmousi model contains 158 horizontally layered horizons.  A series of normal faults and resulting tilted blocks complicates the model towards its center.  The model sits under approximately 32 m of water and is 9.2 km in length and 3 km in depth.  

The velocity model found in the Madagascar repository, \emph{marmvel.hh} can easily be displayed.  This grid contains 751 data points in the Z direction and 2301 data points in the x direction.  Table \ref{tbl:modelHeader} displays the proper header configuration.  

\tabl{modelHeader}{Header information for Marmousi velocity models}
{
\begin{tabular}{|llllll|}
        \hline        
    n1=751    &     d1=.004   &        o1=0  &        label= Depth     & unit1=km &  \\ 
    n2=2301   &     d2=.004   &        o2=0  &        label2=Position  & unit2=km &  \\
	\hline
\end{tabular}
}

The script found at \emph{marmousi/model/SConstruct} was written to obtain the Marmousi model datasets, append the 
headers as necessary and display the  data.  This file is presented in Table \ref{tbl:velSConstruct} while the 
velocity model image is displayed in Figure \ref{fig:marmvel}.
 
\tabl{velSConstruct}{\emph{SConstruct} script generating the Marmousi velocity model images}
{
\tiny
\lstinputlisting[frame=single]{model/SConstruct}
\normalsize
}


Typing Command \ref{eq:SCmdl} within the \emph{Marmousi\slash model} directory runs the script.
\begin{equation}\label{eq:SCmdl} \texttt{bash-3.1\$\ scons\ view} \end{equation}

\inputdir{model}
\plot{marmvel}{width=\textwidth}{Velocity model}

\section{Shot Records}
The file \emph{marmrefl.hh} contains the shot data collected on the Marmousi model.  The survey was an off end survey with 
receivers to the left of the source being pulled towards the right.  Receiver as well as shot spacing is every 25 meters.  
Near offset is 425 meters from the source.  726 time gates were recorded with 4ms spacing for roughly 3 seconds of data collection.  
Madagascar correctly converts this file according to its header; however, the correct shot header values are reproduced in 
Table \ref{tbl:shotHeader}.   

\tabl{shotHeader}{Shot header information for Marmousi.}
{
	\begin{tabular}{|llllll|}
	\hline  
	    n1=726 &  d1= 0.004   &  o1=0.000  &  label1=Depth       &   unit1=s   & \\
    	    n2=96  &  d2=-0.025   &  o2=2.575  &  label2=Offset      &   unit2=km  & \\
            n3=240 &  d3= 0.025   &  o3=3.000  &  label3=Shot-coord  &   unit3=km  & \\
	\hline
	\end{tabular}
}

The file \emph{marmousi/shots/SConstruct} gathers shot data, appends the header as necessary and produces several plots of the data.  This file is reproduced here in Figure \ref{tbl:shotSConstruct}


\tabl{shotSConstruct}{\emph{SConstruct} script generating the Marmousi shot images}
{
\tiny
\lstinputlisting[frame=single]{shots/SConstruct}
\normalsize
}


To run the script type Command \ref{eq:SCvel} within the \emph{marmousi\slash shots} directory.
\begin{equation}\label{eq:SCvel} \texttt{bash-3.1\$\ scons\ view} \end{equation}

\inputdir{shots}
\plot{shot20}{width=\textwidth}{Shot number 20 of Marmousi synthetic survey dataset.  Shot position in km is shown in the lower left 
hand corner.}
\plot{nearOffset}{width=\textwidth}{Near offset data for Marmousi model}

\section{Finite Difference Modeling}
Madagascar can perform finite difference modeling on the Amoco Velocity model.  This is done using the function fdmod.
The raw velocity model needs to be formatted in a similar fashion to the Model Section of this paper.

For the purposes of this example a shot will be fired at 5 km along the horizontal coordinate and at a depth of 10 meters.  Receivers are
spread at a depth of 25 meters every 12.5 meters along the entire scope of the model.  This long receiver cable is impractical but useful for these
purposes.  Data is recorded on every receiver at time increments of 1 ms 5000 times resulting in 5 seconds of data.  In practice it would be 
necessary to perform longer running models, but this number of time gates is sufficient for this introduction. 

An \emph{SConstruct} file located within \emph{marmousi/fdmod/} properly formats the model and inputs necessary parameters to perform a shot 
on the Marmousi model.  This file is reproduced below in Table \ref{tbl:fdmodSConstruct}.

\tabl{fdmodSConstruct}{\emph{Scons} script that performs a finite difference synthetic shot on the Marmousi model.}
{
\tiny
\lstinputlisting[frame=single]{fdmod/SConstruct}
\normalsize
}

Typing Command \ref{eq:SCfdmod} within the \emph{marmousi/fdmod/} directory runs the FD modeling script.
\begin{equation}\label{eq:SCfdmod} \texttt{bash-3.1\$\ scons\ view} \end{equation}

This script first constructs the survey acquisition geometry as was previously mentioned.  An image of the survey is created and presented
in Figure \ref{fig:vel}.

\inputdir{fdmod}
\plot{vel}{width=\textwidth}{FD model geometry as performed on the Marmousi velocity model.  The \emph{X} represents the shot while the \emph{*} symbols represent receivers.}

Firing the shot results the propagation of a wavefield which can be seen in the movie \emph{wfl.vpl} that is generated.  Typing
Command \ref{eq:viewMov} within the \emph{marmousi/fdmod} directory displays the wavefield movie.

\begin{equation}\label{eq:viewMov} \texttt{bash-3.1\$\ scons\ wfl.vpl} \end{equation}

Four frames from this movie are presented in Figure \ref{fig:time20,time40,time60,time80}  illustrating the
propagation of the wavefield in the model.

\inputdir{fdmod}
\multiplot{4}{time20,time40,time60,time80}{width=.45\textwidth}{Images of the propagating wavefield in the Marmousi model generated by a finite
difference model.}

The resulting data is then presented in the file \emph{dat.vpl}.  This plot is reproduced here in Figure \ref{fig:dat}. 

\plot{dat}{width=\textwidth}{Data gathered by the receivers in the FD model survey.}

