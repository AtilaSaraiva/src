\title{A tutorial for processing a 2D land line using Seismic Unix}                                           % Activate to display a given date or no date
\author{Karl Schleicher}

\maketitle

\begin{abstract}
This paper described how to process a public 2D land line data set
though a very basic processing sequence using Seismic Unix. The data
from the north slopes of Alaska has good signal, although it may be
suitable for testing ground roll attenuation programs.  The detailed
steps download the data from the internet and process it are
described.  You should be able to follow these steps and recreate my
results.  I hope this will accelerate the testing and validation of
programs developed in research groups.
\par
The tutorial starts with background information like data downloading,
region overview, data aquisition, and previous processing.  The
scripts and custom programs to translate to seismic unix, load
headers, qc, apply preprocessing (including velocity filtering and
decon), cdp gather, velocity analysis, and stack are listed and
described.

Header loading is not described in other Seismic Unix documentation.
One approach to header loading is described here.  The velocity
analysis script is better than I found in previous papers.

My effort to process these data is ongoing.  I have not computed and
applied residual statics.  I have not migrated the data.  My current
results are not as good as obtained when the data was originally
processed by GSI in 1981.  I would like to provide my most current
results to anyone who wants to use publicly available software to
process seismic data.

This paper is the beginning of an effort to build a library of
open-access seismic datasets and processing scripts using open-source
geophysical software.
\end{abstract}

\section{Introduction}
Seismic Unix (SU) is one of the best known open software packages for
seismic processing.  The distribution includes some example processing
scripts and there are two books that teach how to use the software,
Geophysical Image Processing with Seismic Unix \cite[]{stockwell} and
Seismic Processing with Seismic Un*x \cite[]{forel} books cover broad
subjects including basic Unix commands, setting up a user environment,
basic seismic processing, and more advanced scripting for seismic
processing.  You can follow the text and recreate some seismic
processing.  These books to not provide much information about
creating trace headers or preprocessing.\nocite{cohen}

This paper assumes you are familiar with Seismic Unix.  I concentrate
on providing example scripts and programs to process a 2D land line
using the processing system. I describe the data acquisition and
previous data processing.  I provide the custom programs I used to
load the trace headers.  The scripts for preprocessing, velocity
analysis, and stacking are provided and explained.  These scripts
provide the detailed parameters used by the Seismic Unix programs.
The parameters for deconvolution, velocity filter, mute, moveout
velocity, and scaling can adapted for use by other seismic processing
systems.  Currently I have not applied residual statics or migration.
The results in the deep section are similar to the results obtained in
1981.  The shallow section is not as good as previously obtained.

This paper provides detailed steps that describe how to download the
data from the internet and process it using seismic unix.  You should
be able to follow these steps and recreate my results.  The scripts,
data, and programs can be modified to allow you to validate your own
ideas about seismic processing.  This should accelerate the testing
and validation of new seismic research.

\section{Background information about the dataset}

President Warren Harding created the Naval Petroleum Reserve Number 4
in 1923.  It was renamed the National Petroleum Reserve, Alaska (NPRA)
in 1976.  Figure 1 is a map of the region and Figure 2 maps the lines
collected and processed between 1974 and 1981.  I selected and
processed line 31-81 because it was a short line from the most recent
acquisition season.

The files I found for line 31-81 are: 
\begin{description}
\item[L23535.SGY] unprocessed seismic data in segy format
\item[L23536.SGY] \ 
\item[L23537.SGY] \
\item[3181O.PDF] the Observer's log
\item[3181.SGY] segy of the 1981 final stack
\item[3181S.PDF] the surveyor's log
\item[3181.SPT] the textfile of shotpoint, latitude, longitude. 
\item[S609.JPG]  A 1981 final stack in jpeg format.
\item[S609.TIF]  A higher resolution stack in tif format
\end{description}
I downloaded these files into my directory,
/home/karl/data/alaska/31-81/clientfiles.  I was able to view the
large JPG and TIF files in OpenOffice.org draw (other programs had
problems).  I zoomed the display of S609.TIF to look at the side label
that describes the acquisition and processing parameters.

The data was acquired 96 trace, 12 fold with a dynamite source.  The
shotpoint interval is 440 ft and the receiver interval is 110 ft.  The
average shot depth is 75 ft.  The elevation plot on the section header
shows the elevation is about 200 ft and there is little variation.

The processing applied by GSI in 1981 was:
\begin{enumerate}
\item Edit and true amplitude recovery
\item spherical divergence and exponentiation
(alpha - 4.5 db/s, initial time - 0 s, final time - 4.5 s)
\item Velocity filtering dip range -12 to 4 ms/trace
\item Designature deconvolution
\item Time-variant scaling (unity)
\item Apply datum statics
\item Velocity estimation
\item Normal moveout correction
\item Automatic Residual Statics application
\item First Break Suppression \\
100 ms at 55 ft\\
380 ms at 2970 ft\\
700 ms at 5225 ft
\item Common depth point stack
\item Time Variant filtering
16-50 Hz at 700 ms\\
14-45 Hz at 1400 ms\\
12-40 Hz at 2000 ms\\
8-40 Hz at 400 ms
\item Time variant scaling
\end{enumerate}


\section{Data Loading and initial data QC}
\inputdir{line31-81}

I found little documentation about loading data and creating trace
headers in precious publications.  This section provides a detailed
description.

I created a script file to load the first segy file,
/home/karl/jobs/alaska/31-81/load/view1.job.  This file is listed
in Appendix 1.

The way I developed this script is by pasting a command or two into a
terminal window and picking information from the print of displays to
write the next few commands.  The segyread command translates data to
SU format.  The surange command prints: 3030 traces:
tracl   1 3030 (1 - 3030)\\
tracr    1 3030 (1 - 3030)\\
fldr     101 130 (101 - 130)\\
tracf   1 101 (1 Ð 101)\\
trid     1\\
ns       3000\\
dt       2000\\
f1       0.000000 0.000000 (0.000000 - 0.000000)\\

The print indicates there is little information in the input headers.
There are 31 shot records (fldr 101 - 130) and each record has 101
traces (tracf 1 - 101).  The suximage produces Figure~\ref{fig:first}
and the zoom plot Figure~\ref{fig:zoomfirst}.
\plot{first}{width=\textwidth}{First 300 traces on the first tape.}
\plot{zoomfirst}{width=\textwidth}{Zoom of figure~\ref{fig:first}.First look at the data. Notice the
ground roll and good signal. There are 101 traces in each shotpoint, 5
auxiliary traces and 96 data traces.}

The first 10 records in the file are test records.  These tests
records are the unuisual traces in the left third of
Figure~\ref{fig:first}. The Observer Log indicates the first shotpoint
is field file identifier (ffid) 110.  There are also 5 auxiliary
traces in each shot record.  These auxiliary traces can be seen o
nfigures 4 and 5.

The suxmovie command allows all the shots record on the line to be
displayed.  The parameter n2 is set to the number of traces in each
shot record, which was based on the tracf range printed in surange (1
- 101).  The movie loops continuously and can be stopped by pressing
ÒsÓ.  Once stopped, you can move forward or backward using the ÒfÓ and
ÒbÓ.  An example screen is shown in Figure 5.  This plot allowed me to
see that traces 1-96 were the data channels and 97-101 were the
auxiliary traces.

The final command in the view1.job file is the sugethw.  This prints
the fldr and tracf header keys.  These are the keys I used to map the
data to the shotpoint and receiver locations.

I wrote a custom java program to load the trace headers.  The code is
not elegant, but it is straightforward to write and it works.  It is
listed in Appendix 2.  This program require files to define the
shotpoint elevation and the ÒFFIDÓ/ÓEPÓ relationship.  The surveyor
log, downloaded as: /home/karl/data/alaska/31-81/clientfiles/3181O.PDF
defines the shotpoint elevations.  The observers log, downloaded as:
/home/karl/data/alaska/31-81/clientfiles/3181O.PDF describes the
relationship between the su header keys ÒffidÓ and ÒepÓ (these are
called ÒREC.Ó and ÒSHOTPOINTÓ by in the observers log).  This
information was typed into two files listed in appendices 3 and 4.

The program to load the headers was run by the script:
/home/karl/jobs/alaska/31-81/load/hdrload1.job This file is listed in
appendix 5.

The other two segy files were loaded by scripts view2.job, view3.job,
hdrload2.job, and hdrload3.job in the same directory,
/home/karl/jobs/alaska/31-81/load.  These scripts differ from the
previous scripts only by the names of the input and output file.

The stack file from the previous process was translated to SU format
and displayed using the viewqcstack.job script listed in appendix 6.
This is a straight forward execution of segyread and suximage.
Display of the checkstack is in Figure 6.


\section{Shot record velocity filter}
I applied Velocity filtering (sudipfilt) to remove the groundroll.
The receivers leading the shotpoint were processed separately from the
trailing receivers.  This allowed an asymmetrical dip filter to be
applied (-15,5 ms/trace).  These parameters are loosely based on the
1981 processing that used (-12,4). Sudipfilter was intended for post
stack processing, so a loop is required in the script to divide the
data into individual shotrecords and seperate the positive and
negative offsets.  The script also includes suxmovie of the velocity
filtered data.  It is listed in appendix 7.  Figure 7 is an example
shot record after velocity filtering,


\section{Shot record edit, mute, and cdp sort}
There is a bad shotpoint I removed.  I also applied a mute and sorted
the data to CDP order in the same script.  The script is listed in
Appendix 8.  Figure 8 is a display of two cdp gathers.


\section{Velocity interpretation}
I used a long script that combined several su program for velocity
interpretation.  This script is listed in Appendix 9.  I combined the
scripts from Forel et. al. 2005 (iva.sh section 7.6.7.3 and velanQC.sh
section 8.2.2.2).  The script is more practical, because you can start
from no velocity function, or review and update and existing velocity
field.  The script hos more capabilities, but it is tidier and a
little shorter than Forel's script.

Figure 9 is an example velocity analysis plot.  There are four
components of the plot, semblance plot, constant velocity stack (CVS)
plot, cmp gather without NMO and cmp gather after NMO.  The semblance
plots and CVS plot have the velocity function overplotted.

The velocities I picked are listed on Appendix 10.


\section{Stack}
The script that stacks the data is in Appendix 11.  This script also
applied decon using the command supef.  The script also applied AGC.
I wanted to start the decon design gate start time to increase with
offset so the shallow data near the mute was not included.  The only
way I knew how to do this us SU was to apply linear moveout to move
the design gate start time to a constant time.  This was accomplished
by removing the sign from the offset header and computing a time
static from the offset.  Supef was bracketed with sustatic to apply
and remove the offset dependent static.  I compared the stack created
in SU to the original processing results created in 1981 (figures 10
and 11).
\section{Comparison of the SU and the 1981 results}
Figures 10 and 11 are the results created using SU and the results
obtained in 1981.  Some of the differences in the processing sequences
are:
\begin{enumerate}
\item The 1981 processing used GSI's designature process.  This source signature removal applies the same operator to all traces in a shotrecord.  The filter is designed differently from conventional deconvolution and the two types of deconvolution will not produce data with the same wavelet. 
\item There are many differences in the application of AGC.  For example the GSI used to apply AGC before velocity filtering 
inverse AGC after velocity filter.
\item GSI used diversity stack to control the high amplitude noise bursts.
\item The original processing includes residual statics.
\end{enumerate}
Considering all these differences, I think the results are
surprisingly similar, especially below 400 ms.  I think residual
statics account for most of the differences.  The shallow result on
the SU processing is worse than the 1981 processing.

In general, I found the SU software hard to use.  Sudipfilter was not
intended for shotrecord velocity filtering, so a collection of program
were required.  Many of my input errors were not trapped by the code.
I was disappointed that the processing applied 30 years ago produced
better results than I obtained using SU.  SU is a useful prototyping
environment, but it falls short of the commercial packages I have
used.

\section{Plans}
Some of the ideas I have to continue this work include:

\begin{enumerate}
\item Adding more processes, especially residual statics and migration.
\item Process using other software (Open source of proprietary. 
\item Working on different datasets (marine, 3D, ...).
\item Make scripts and intermediate results available on the internet.
\item Improve open source software.
\end{enumerate}

\section{Conclusions}
I have provided a set of scripts the process a land line using SU.  These scripts include the processing stages:

\begin{enumerate}
\item data load
\item trace header creation
\item velocity filtering 
\item cdp gather
\item velocity analysis
\item stack
\end{enumerate}

Although this sequence omits important processes like residual statics
and migration, it illustrates processing stages not presented by Forel
or Stockwell.

My processing results can be recreated by others since the processing
scripts are included in appendixes of this paper and the data can be
downloaded from the internet.

The task of loading trace headers is no addressed in other
publications.  The custom java program included in this paper is one
way to complete this processing stage.  The velocity analysis script
(iva.sh in Appendix 9) improves the scripts in Forel's book.  I was
able to pick the velocities on this line using the script.

Some of the ideas I have to continue this work include:
\begin{enumerate}
\item Adding more processes, especially residual statics and migration
\item Process using other software (Open source of proprietary.
\item Working on more data (Marine, 3D, ...)
\item Make scripts and intermediate results available on the internet
\item Improve open source
\end{enumerate}

\bibliographystyle{seg}
\bibliography{refpaper}

\append{\texttt{SConstruct} file}

\lstset{language=python,numbers=left,numberstyle=\tiny,showstringspaces=false}
\lstinputlisting[frame=single]{line31-81/SConstruct}
