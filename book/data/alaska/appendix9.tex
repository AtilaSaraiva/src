This velocity interpretation script combines the capabilities of two scripts in Forel et. al. 2005 (iva.sh  section 7.6.7.3 and velanQC.sh section 8.2.2.2).  The script is much more practical allowing you to start from no velocity function, or review and update and existing velocity field.  The script is tidier and a little shotrer than in Forel's book The script is in the file: \\
 /home/karl/jobs/alaska/31-81/velfilt/iva.sh   \\
 \\
\#!/bin/sh \\
\# File: iva.sh \\
 \\
\# Credits: \\
\# 2004 Hale, Cohen, with Stockwell modifications 2004. \\
\#      In su distribution directory: \\
\#      $~/su/src/demos/Velocity_Analysis/Traditional/Velan$ \\
\# 2005 Seismic Processing with Seismic Un*x  \\
\#      Forel, Benz, Pennington, 2005 \\
\#      script iva.sh  (section 7.6.7.3) and  \\
\#      velanQC.sh (section 8.2.2.2) \\
\# 2011 Schleicher, offerred to David Forel for new edition \\
\#      of "Seismic Processing with Seismic Un*x" \\
 \\
\# Set messages on \\
\#set -x \\
 \\
\#=============================================== \\
\# USER AREA -- SUPPLY VALUES \\
\#------------------------------------------------ \\
\# CMPs for analysis \\
 \\
cmp1=181 cmp2=229 cmp3=277  \\
cmp4=373 cmp5=421 cmp6=469  \\
cmp7=517 cmp8=565 \\
 \\
numCMPs=8 \\
\#------------------------------------------------ \\
\# File names \\
indata=/home/karl/data/alaska/31-81/velfiltcdpsmute.su  \\
outpicks=vpick.txt  \# ASCII file \\
inpicks=outpicks \\
 \\
\#------------------------------------------------ \\
\# display choices \\
 \\
myperc=98       \# perc value for plot \\
$SUXWIGB\_OR\_XIMAGE=suxwigb \\
SUXWIGB\_OR\_XIMAGE=suximage $\\
 \\
\# size of the display windows \\
HBOX=700    \# originally 450 \\
WBOXCVS=300 \# originally 300 \\
WBOXCMP=200 \# originally 300 \\
WBOXVELAN=500 \# originally 300 \\
XBOXVELAN=10 \# kls 1350 puts it on second screen \\
 \\
\# these parameters work nicely to put plots on \\
\# my 2nd screen. I turn them on by changing  \\
\# the next line to:    if [ 1 ]  \\
if [ 0 ]   \\
then \\
    HBOX=1000    \# originally 450 \\
    WBOXCVS=300 \# originally 300 \\
    WBOXCMP=200 \# originally 300 \\
    WBOXVELAN=800 \# originally 300 \\
    XBOXVELAN=1350 \\
fi \\
 \\
XBOXCVS=$`expr \$XBOXVELAN + \$WBOXVELAN`$ \\
XBOXCMP=$`expr \$XBOXCVS + \$WBOXCVS`$ \\
XBOXNMOCMP=$`expr \$XBOXCMP + \$WBOXCMP`$ \\
\#------------------------------------------------ \\
\# Processing variables \\
 \\
\# Semblance variables \\
nvs=100  \# number of velocities \\
dvs=100   \# velocity intervals \\
fvs=7000 \# first velocity \\
 \\
\# CVS variables \\
fc=7000 \# first CVS velocity \\
lc=16000 \# last CVS velocity \\
nc=10   \# number of CVS velocities (panels) \\
SPAN=11   \# ODD number of CMPs to stack into central CVS \\
 \\
\#============================================== \\
 \\
\# HOW SEMBLANCE (VELAN) VELOCITIES ARE COMPUTED \\
\# Last Vel =  fvs + (( nvs-1 ) * dvs ) = lvs \\
\#     5000 =  500 + ((  99-1 ) * 45  ) \\
\#     3900 = 1200 + (( 100-1 ) * 27  ) \\
\# Compute last semblance (velan) velocity \\
lvs=$`echo "\$fvs + (( \$nvs - 1 ) * \$dvs )" | bc -l`$ \\
 \\
\#------------------------------------------------ \\
\# HOW CVS VELOCITIES ARE COMPUTED \\
\# dc = CVS velocity increment \\
\# dc = ( last CVS vel - first CVS vel ) / ( \# CVS - 1 ) \\
\# m = CVS plot trace spacing (m = d2, vel units) \\
\# m = ( last CVS vel - first CVS vel ) / ( ( \# CVS - 1 ) * SPAN ) \\
\# j=1 \\
\# while [ j le nc ] \\
\# do \\
\#   vel =  fc  + \{ [(   lc - fc   ) / ( nc-1 )] * ( j-1) \} \\
\#   j = j + 1 \\
\# done \\
\# EXAMPLE: \\
\#   vel = 1200 + ( (( 3900 - 1200 ) / ( 10-1 )) * ( 1-1) ) \\
\#   vel = 1200 + ( (( 3900 - 1200 ) / ( 10-1 )) * ( 2-1) ) \\
\#   ... \\
\#   vel = 1200 + ( (( 3900 - 1200 ) / ( 10-1 )) * (11-1) ) \\
\#============================================== \\
 \\
 \\
\# FILE DESCRIPTIONS \\
\# spanpanel.\$picknow.su = binary temp file for input CVS gathers \\
\# cvs.\$picknow.su = binary temp file for output CVS traces \\
\# nmopanel.\$picknow.su = binary temp file for NMO (flattened) section \\
\# panel.\$picknow.su = current CMP windowed from line of CMPs \\
\# spanpanel.\$picknow.su = span of CMPs windowed to make cvs \\
\# picks.\$picknow = current CMP picks arranged as "t1 v1" \\
                                                 etc. \\
\# par.\# (\# is a sequential index number; 1, 2, etc.) \\
\#      = current CMP picks arranged as \\
\#        "tnmo=t1,t2,t3,... \\
\#        "vnmo=v1,v2,v3,... \\
\# par.0 = file "par.cmp" re-arranged as \\
\#         "cdp=\#,\#,\#,etc."  NOTE: \# in this line is picked CMP \\
\#         "\#=1,2,3,etc."    NOTE: \# in this line is "\#" \\
\# outpicks = concatenation of par.0 and all par.\# files. \\
 \\
\#=============================================== \\
 \\
echo "  *** INTERACTIVE VELOCITY ANALYSIS ***" \\
 \\
\#------------------------------------------------ \\
\#kls Remove old files.  Open new files \\
\#rm -f panel.*.su picks.* par.* tmp* \\
 \\
echo "save the old outpicks file with date in the name:" \\
suffix=$`date|sed "s/ /_/g"`$ \\
echo  "cp -p \$outpicks \$outpicks.\$suffix" \\
cp -p \$outpicks \$outpicks.\$suffix \\
 \\
\#make a set of picks.* files that will be used to plot on the \\
\#velan and the cvs plots, and apply nmo to gather plot \\
 \\
if [ -s \$inpicks ] \\
then \\
 tvnmoqc mode=2 prefix=picks par=\$inpicks \\
fi \\
 \\
\#------------------------------------------------ \\
\# Get ns, dt, first time from seismic file \\
nt=$`sugethw ns < \$indata | sed 1q | sed 's/.*ns=//'`$ \\
dt=$`sugethw dt < \$indata | sed 1q | sed 's/.*dt=//'`$ \\
delrt=$`sugethw delrt < \$indata | sed 1q | sed 's/.*delrt=//'`$ \\
 \\
\# Convert dt from header value in microseconds \\
\# to seconds for velocity profile plot \\
dt=$`echo "scale=6; \$dt / 1000000 " | bc -l`$ \\
 \\
\# If "delrt", use it; else use zero \\
tstart=$`echo "scale=6; \$\{delrt\} / 1000" | bc -l`$ \\
 \\
\#------------------------------------------------ \\
\# BEGIN IVA LOOP \\
\#------------------------------------------------ \\
 \\
i=1 \\
 \\
while [ \$i -le \$numCMPs ] \\
do \\
 \# set variable \$picknow to current CMP \\
 $eval pickprev=\$cmp`expr i - 1` \\
 eval picknow=\$cmp\$i \\
 eval picknext=\$cmp`expr i + 1` $\\
 \\
 \# make a file so the while loop will run the first time \\
 echo "just some junk" $>$ newpicks.\$picknow \\
 while [ -s "newpicks.\$picknow" ]  \\
 do \\
  \# work this location until the user makes no picks \\
  \# on the velan \\
   \\
  if [ -s picks.\$picknow ] ; then \\
    echo "Location CMP \$picknow has no picks." \\
  fi \\
 \\
  \#------------------------------------------------ \\
  \# Plot CMP (right) \\
  \#------------------------------------------------ \\
 \\
  suwind < \$indata \ \\
           key=cdp min=\$picknow max=\$picknow $>$ panel.\$picknow.su \\
 \\
 $ \$SUXWIGB_OR_XIMAGE < panel.\$picknow.su $ \ \\
      xbox=\$XBOXCMP ybox=0 wbox=\$WBOXCMP hbox=\$HBOX \ \\
      title="CMP gather \$picknow" \ \\
      label1=" Time (s)" label2="Offset (m)" key=offset \ \\
      perc=\$myperc verbose=0 \& \\
 \\
 \#------------------------------------------------ \\
 \# Constant Velocity Stacks (CVS) (middle-left) \\
 \# Make CVS plot for first pick effort. \\
 \# If re-picking t-v values, do not make this plot. \\
 \#------------------------------------------------ \\
 \\
  \# truncate SPAN to odd number less then or equal to SPAN \\
  $HALF_SPAN=`expr \$SPAN / 2`$ \\
  SPAN=$`expr \$HALF_SPAN "*" 2 + 1` $\\
 \\
  \# Select CMPs \$picknow +/- $HALF\_SPAN.$  \\
  \# Write to  spanpanel.\$picknow.su \\
  CMPMIN=$`expr \$picknow - \$HALF\_SPAN`$ \\
  CMPMAX=$`expr \$picknow + \$HALF\_SPAN`$ \\
  suwind < \$indata key=cdp min=\$CMPMIN max=\$CMPMAX \ \\
  $>$ spanpanel.\$picknow.su \\
 \\
  \# Calculate CVS velocity increment \\
  \# dc = ( last CVS vel - first CVS vel ) / ( \# CVS - 1 ) \\
  dc=`echo "( \$lc - \$fc ) / ( \$nc - 1 )" | bc -l` \\
 \\
  \# Calculate trace spacing for CVS plot (m = d2, vel units) \\
  \# m = ( last CVS vel - first CVS vel ) / ( ( \# CVS - 1 ) * SPAN ) \\
  m=$`echo "( \$lc - \$fc ) / ( ( \$nc - 1 ) * \$SPAN )" | bc -l`$ \\
  if [ ! -s cvs.\$picknow.su  ] ; then \\
   \# CVS ve locity loop \\
    rm cvs.\$picknow.su \\
    j=1 \\
    while [ \$j -le \$nc ] \\
    do \\
      vel=$`echo "\$fc + \$dc * ( \$j - 1 )" | bc -l`$ \\
 \\
      \# uncomment to print CVS velocities to screen \\
      \#     echo " vel = \$vel" \\
 \\
      sunmo < spanpanel.\$picknow.su vnmo=\$vel | \\
      sustack $>>$ cvs.\$picknow.su \\
 \\
      $j=`expr \$j + 1`$ \\
    done \\
  fi \\
 \\
  \# Compute lowest velocity for annotating CVS plot \\
  \# $loV = first CVS velocity - $HALF\_SPAN $* vel inc $\\
  $loV=`echo "\$fc - \$HALF_SPAN * \$m" | bc -l`$ \\
 \\
 \#------------------------------------------------ \\
  \# Picking instructions \\
  \#------------------------------------------------ \\
  echo " " \\
  echo "Preparing CMP \$i of \$numCMPs for Picking " \\
  echo "Location is CMP \$picknow. CVS CMPs =\\
 \$CMPMIN,\$CMPMAX" \\
  echo " " \\
  echo "  Use the semblance plot to pick (t,v) pairs." \\
  echo "  Type \char92 "s\char92 " when the mouse pointer is where you want a pick." \\
  echo "  Be sure your picks increase in time." \\
  echo "  To control velocity interpolation, pick a first value" \\
  echo "    near zero time and a last value near the last time." \\
 echo "  Type \char92 " q\char92 " in the semblance plot when you finish picking." \\
  echo " " \\
  echo " If there are no picks (using \char92 "s\char92 ") before you quit (using" \\
  echo " \char92 "q\char92 " in the semblance plot, picking will continue at the" \\
  echo " next CMP."   \\
  \#------------------------------------------------ \\
  \# Plot semblance (velan) (left) \\
  \#------------------------------------------------ \\
 \\
  \# if there is a non-zero length picks.\$picknow file, plotit \\
  \# kls add logic for pickprev, picknow, picknext \\
    if [ -s picks.\$picknow ] \\
    then \\
 \\
      \#---  ---  ---  ---  ---  ---  ---  ---  ---  --- \\
      \# Get the number of picks (number of lines) in picks.\$picknow \\
      \#   Remove blank spaces preceding the line count. \\
      \# Remove file name that was returned from "wc". \\
      \# Store line count in "npair" to guide line on velan. \\
 \\
      npair=`wc -l picks.\$picknow \ \\
            $|$ sed \char94 s/\char94  *$(.*)$/\char92 $1/'$  \ \\
             $|$ sed $'s/picks.\$picknow//' `$ \\
     $ plotline="curve=picks.\$picknow npair=\$npair curvecolor=white"$ \\
    else \\
      plotline=" " \\
    fi \\
    \# echo plotline=\$plotline \\
    \# plot the cvs \\
    \# cat picks.\$picknow \\
    suximage < cvs.\$picknow.su \ \\
	xbox=\$XBOXCVS ybox=0 wbox=\$WBOXCVS hbox=\$HBOX \ \\
        title="CMP \$picknow Constant Velocity Stacks" \ \\
        label1=" Time (s)" label2="Velocity (m/s)" \ \\
        f2=\$loV d2=\$m verbose=0 \ \\
        perc=\$myperc n2tic=5 cmap=rgb0  \$plotline \& \\
 \\
    \# if there is a velocity function, display moved out gather \\
    if [ -s picks.\$picknow ] \\
    then \\
      \# translate picks.\$picknow into tnmo/vnmo for sunmo \\
      sort < picks.\$picknow -n | \\
         mkparfile string1=tnmo string2=vnmo $>$ par.\$i \\
      \# apply nmo and plot the moved out gather \\
      sunmo < panel.\$picknow.su par=par.\$i cdp=\$picknow verbose=0 \ \\
      $|$ $SUXWIGB\_OR\_XIMAGE$  \ \\
         xbox=\$XBOXNMOCMP ybox=0 wbox=\$WBOXCMP hbox=\$HBOX \ \\
         title="CMP \$picknow after NMO" \ \\
         label1=" Time (s)" label2="Offset (m)" \ \\
         verbose=0 perc=\$myperc key=offset \& \\
    fi \\
 \\
    \# compute and plot the semblance/velan \\
    cat picks.\$picknow \\
    suvelan < panel.\$picknow.su nv=\$nvs dv=\$dvs fv=\$fvs \ \\
    | suximage \ \\
	xbox=\$XBOXVELAN ybox=0 wbox=\$WBOXVELAN hbox=\$HBOX perc=99 \ \\
        units="semblance" f2=\$fvs d2=\$dvs n2tic=5 \ \\
        title="Semblance Plot CMP \$picknow" cmap=hsv2 \      \\
       label1=" Time (s)" label2="Velocity (m/s)" \ \\
        legend=1 units=Semblance verbose=0 gridcolor=black \ \\
        grid1=solid grid2=solid \ \\
        mpicks=newpicks.\$picknow \$plotline \\
 \\
    if [ -s "newpicks.\$picknow" ] \\
    then \\
      echo "there is a non-zero length newpicks.\$picknow file" \\
      cat newpicks.\$picknow \\
      cp newpicks.\$picknow picks.\$picknow \\
    fi \\
 \\
    echo " " \\
    echo " t-v PICKS CMP \$picknow" \\
    echo "----------------------" \\
    cat picks.\$picknow \\
    echo "----------------------" \\
   \\
    \#  rm spanpanel.\$picknow.su \\
    zap xwigb $>$/dev/null \\
    zap ximage $>$/dev/null \\
 \\
  done \\
  $=`expr \$i + 1`$\\
done \\
 \\
\#------------------------------------------------ \\
\# Create velocity output file \\
\#------------------------------------------------ \\
 \\
cdplist=\$cmp1 \\
 \\
i=2 \\
while [ \$i -le \$numCMPs ] \\
do \\
    eval picknow=\$cmp\$i \\
    cdplist=\$cdplist,\$picknow \\
   $i=`expr \$i + 1` $\
done \\
echo cdp=\$cdplist \char92 \char92 $>$ \$outpicks \\
 \\
i=1 \\
while [ \$i -le \$numCMPs ] \\
do \\
  $sed < par.\$i 's/\$/ \\/g' > > \$outpicks \\
  i=`expr \$i + 1` $ \\
done \\
 \\
\#------------------------------------------------ \\
\# Remove files and exit \\
\#------------------------------------------------ \\
echo " " \\
echo " The output file of t-v pairs is "\$outpicks: \\
cat \$outpicks \\
$rm -f panel.*.su spanpanel.*.su picks.* par.* newpicks.* cvs.*.su $\\