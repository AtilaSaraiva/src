\title{Pluto Model}
\author{Trevor Irons}

\maketitle
\lstset{language=python,numbers=left,numberstyle=\tiny,showstringspaces=false}
\textbf {Data Type:} \emph{Synthetic}\\
\textbf {Source:} \emph{SMAART Consortium}\\
\textbf {Location:} \emph{http://www.delphi.tudelft.nl/SMAART/pluto15.htm}\\
\textbf {Format:} \emph{SEGY} \\
\textbf{Date of origin:} \emph{Publically released November 2000}\\

\section{Introduction} 
The Subsalt Multiples Attenuation and Reduction Technology Joint Venture (SMAART JV) publicly released several data sets between September 2001 and November 2002.  Additional information may be found at: \emph{http://www.delphi.tudelft.nl/SMAART/}.  The data sets remain the property of SMAART and are used under the agreement found at: \emph{http://www.delphi.tudelft.nl/SMAART/DataReleaseAgreement.pdf}.

The Pluto 1.5 is a 2D elastic dataset released in November 2000, designed to emulate deep water subsalt prospects as found in the Gulf of Mexico. It contains realistic free surface and internal multiples over a structure that is relatively easy to image. Table \ref{tbl:FILES} shows the files contained within the \emph{Pluto} repository of Madagascar.  
 
\tabl{FILES}{A list of all files contained in the \emph{Pluto} repository}
{
\tiny
\lstinputlisting[frame=single]{FILES}
\normalsize
}

\section{Velocity Models}
The\emph{SConstruct} file found within \emph{RSF/book/data/pluto} is shown in \ref{tbl:velSConstruct}.  Entering \texttt{scons view} at the command line will run the script.  

\tabl{velSConstruct}{\emph{SConstruct} script generating the velocity model}
{
\tiny
\lstinputlisting[frame=single]{model/SConstruct}
\normalsize
}        

This script generates three images of the velocity model created with three different input files one of which is reproduced in figure \ref{fig:velo}.

\inputdir{model}
\plot{velo}{width=\textwidth}{Pluto velocity model}

\section{Shot Records}

