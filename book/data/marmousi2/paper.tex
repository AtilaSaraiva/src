\title{Marmousi2 model}
\author{Trevor Irons}
\lstset{language=python,numbers=left,numberstyle=\tiny,showstringspaces=false}
\maketitle

\noindent
\textbf {Data Type:} \emph{Synthetic 2D elastic model}\\
\textbf {Source:} \emph{Allied Geophysical Laboratories (AGL)}\\
\textbf {Location:} \emph{http://www.agl.uh.edu/}\\
\textbf {Format:} \emph{SEGY} \\
\textbf{Date of origin:} \emph{Development occured from 2002-2005}\\

\section{Introduction}
The Marmousi2 dataset is an extension and elastic upgrade of the classic Marmousi model. It was created by Allied Geophysical 
Laboratories (AGL).  The Marmousi2 model has enjoyed widespread use and has been particularily insightful in amplitude versus
offset (AVO) analysis, impedance inversion, multiple attenuation, and multicomponent imaging.  AGL has publically released 
the data for reasearch use around the world.   

Table \ref{tbl:FILES} contains all the Marmousi2 files contained within the Madagascar repository.  

\tabl{FILES}{A list of all files contained in the Marmousi2 repository}
{
\tiny
\lstinputlisting[frame=single]{FILES}
\normalsize
}

\section{model}
The Marmousi2 model completely encapsulates the origional Marmousi model which was based on the Northern Quenguela Trough 
in the Quanza Basin of Angola.  Lithologies include sandstones,shales, limestones and marls.    

In total the Marmousi2 model is 3.5 km in depth and 17 km across.  The model contains 199 horizons which make the model 
stratigraphicaly more complex than its predecessor.  Additionally the water layer was extended to 450 meters.   

As Marmousi2 is an elastic model both shear and primary velocities must be defined across the entire model.  
Additionally a density model is included.  
The files \emph{vp\_marmousi-ii.segy}, \emph{vs\_marmousi-ii.segy}, and \emph{density\_marmousi-ii.segy} contain the velocity and 
density models for Marmousi2.  These three files all share the same data spacing and their headers should be formatted similarily.
This header format is shown in table \ref{tbl:modelHeader}

\tabl{modelHeader}{Header information for Marmousi2 velocity and density models}
{
\begin{center}
\begin{tabular}[t]{|llllll|}
        \hline
        n1=2801    &   o1=0 &   d1=0.001249    &    label1=Depth     &  unit1=km &  \\
        n2=13601   &   o2=0 &   d2=0.001249    &    label2=Position  &  unit2=km &  \\
        \hline
\end{tabular}
\end{center}
}


The file \emph{marmousi2\slash model\slash SConstruct} is a SCons script that fetches the three model files (VP, VS, and Density), 
appends the header information as necessary and produces plots of the models.  This file is reproduced in table 
\ref{tbl:modelSConstruct} and the models themselves are shown in figures \ref{fig:vp}, \ref{fig:vs}, and \ref{fig:density}.   

\tabl{modelSConstruct}{SCons script generating images of the Marmousi2 model}
{
\tiny
\lstinputlisting[frame=single]{model/SConstruct}
\normalsize
}

\inputdir{model}
\plot{vp}{width=\textwidth}{Marmousi2 P-wave velocity model}
\plot{vs}{width=\textwidth}{Marmousi2 S-wave velocity model}
\plot{density}{width=\textwidth}{Marmousi2 Density Model}

\section{Shots}
Three sets of data were collected over this model.  A near surface streamer survey, a vertical souding profile (VSP), and 
an ocean bottom cable (OBC) survey.  Several sets of shot records are included in the Marmousi2 repository; 
multicomponent OBC data found in \emph{obc\_vx\_\#.segy} and\emph{obc\_vz\_\#.segy},  
reduced data from the OBC cable found in \emph{obc\_div\_v\_\#.segy} and \emph{obc\_curl\_v\_\#.segy}, and
streamer cable data found in \emph{surface\_p\#.segy}.  Each of these files was split into components to make them more 
managable.  The  \# symbol above corresponts to either part number 1 or 2. 

In all cases the source was an airgun located on a ship at depth of 10 m.  The source began firing at 3 000 m along the horizontal x 
coordinate and continued firing every 25 m until 14 975 m.  

\subsection{OBC Surveys}
The OBC cable was placed on the ocean floor at a depth of roughly 450 m.  Multicomponent phones were spaced every 12.32 m along the
entire length of the model.  As the model is 2D only the x and z components of the wavefield were measured.  
Marmousi2 OBC survey data should have header information configured as shown in table \ref{tbl:OBCheader}.

\tabl{OBCheader}{Header information for Marmousi2 ocean bottom cable surveys}
{
\begin{tabular}[t]{|llllll|}
        \hline
        n1=2500    &   o1=0     &   d1=0.002    &    label1=Depth\ Z     &  unit1=s  &  \\
        n2=1381    &   o2=0     &   d2=12.32    &    label2=Position\ X  &  unit2=m  &  \\
	n3=480 	   &   o3=3000  &   d3=25	&    label3=Shot-Coord   &  unit2=m  &  \\
        \hline
\end{tabular}
}


\subsubsection{OBC Vz data}
The file \emph{marmousi2\slash vz \slash SConstruct} contains a SCons script that fetches the Vz component
data files from the OBC survey, concatinates the segments, appends the header making a 
three axis file, (time, offset, and shot) and produces several plots of the data.  This file is reproduced in table 
\ref{tbl:VzSConstruct}.  

\tabl{VzSConstruct}{SCons script generating images of the Marmousi2 Vz data}
{
\tiny
\lstinputlisting[frame=single]{vz/SConstruct}
\normalsize
}

\subsubsection{OBC Vx data}
Similar to the Vz data the file \emph{marmousi2 \slash vx \slash SConstruct} contains a list of rules that tell
Madagascar to gather the Vx data files, append the header and produce plots of the data.  This script is reproduced 
in table \ref{tbl:VxSConstruct}
\tabl{VxSConstruct}{SCons script generating images of the Marmousi2 Vx data}
{
\tiny
\lstinputlisting[frame=single]{vz/SConstruct}
\normalsize
}


\subsubsection{OBC div data}
The divergence operator was applied to the multicomponent OBC Pluto dataset.  These files are \emph{obc\_div\_v\_1.segy} and 
\emph{obc\_div\_v\_2.segy}.  Taking the divergence separates out the acoustic component of the data.    

The file \emph{marmousi2 \slash div \slash SConstruct} contains a list of rules that tell Madagascar to gather the 
div data files, append the header and produce plots of the data.  This script is reproduced 
in table \ref{tbl:divSConstruct} and a plot of shot 50 is shown in figure \ref{fig:divShot50}

\tabl{divSConstruct}{SCons script generating images of the Marmousi2 Vx data}
{
\tiny
\lstinputlisting[frame=single]{div/SConstruct}
\normalsize
}

\inputdir{div}
\plot{movie}{width=\textwidth}{Marmousi2 shot 50 of div data.}


\subsubsection{OBC curl data}
Similarily the curl operator was applied to the Pluto OBC data.  These files are \emph{obc\_curl\_v\_1.segy} and \emph{
obc\_curl\_v\_2.segy}  These curl data contain only data generated by the elastic component of the field.  

The file \emph{marmousi2 \slash curl \slash SConstruct} contains a list of rules that tell Madagascar to gather the 
curl data files, append the header and produce plots of the data.  This script is reproduced 
in table \ref{tbl:curlSConstruct}
\tabl{curlSConstruct}{SCons script generating images of the Marmousi2 curl data}
{
\tiny
\lstinputlisting[frame=single]{curl/SConstruct}
\normalsize
}

\subsection{Streamer Surveys}
The streamer survey was not traditional in the sense that it employed a 17 km long static streamer which spanned the entire
model.  In total there were 1 361 single component hydrophones spaced every 12.5 m at a depth of 5 m.  This unrealistic geometry
was chosen both for simplicity and to allow maximum utility of the data.  The table \ref{tbl:streamerHeader} outlines the 
values that streamer data files headers should have.    

\tabl{streamerHeader}{Header information for Marmousi2 streamer surveys}
{
\begin{tabular}[t]{|llllll|}
        \hline
        n1=2500    &   o1=0     &   d1=0.002    &    label1=Depth\ Z     &  unit1=s  &  \\
        n2=1361    &   o2=0     &   d2=12.5     &    label2=Position\ X  &  unit2=m  &  \\
	n3=480 	   &   o3=3000  &   d3=25	&    label3=Shot-Coord   &  unit2=m  &  \\
        \hline
\end{tabular}
}

\section{Finite Difference Modeling}
Madagascar may be used to perform finite difference modeling of the wavefield and receiver data.  The tools to perform these tasks are
found in the fdmod package.    

Note, these processes are somewhat computationally intensive.  I performed the majority of these models on a machine with a 3 GHz 
processor and 1.5 MB of RAM and most of the models took on the order of 3 hours to perform.     


