\section{The $\tau$-$\lowercase{p}$ domain}

The \taup transform is the natural domain for anisotropy parameters
estimation in layered or vertically varying media with horizontal
symmetry planes
\citep{baan:1076,douma:D53,silsen,tsvankin:75A15}. Since the
horizontal slowness is preserved upon propagation, the \taup transform
allows simpler and more accurate traveltime modeling (ray tracing) and
inversion (layer stripping). Moreover, the \taup transform is a
plane-wave decomposition. \old{and} \new{Therefore,} the phase
velocity, rather than the group velocity, is the relevant
velocity. The group velocity controls instead traveltime in the
traditional \tx domain \citep{ilyabook2006}. Unfortunately, the exact
expressions for the group velocities in terms of the group angle are
difficult to obtain and cumbersome for practical use. As a result, it
requires either ray tracing for exact \tx modeling in anisotropic
media or the use of multi-parameter traveltime approximations.  In
this domain, the most straightforward and widely used approximation
for P-waves reflection moveout comes from the Taylor series expansion
of traveltime or squared traveltime \old{near to} \new{around} the
zero offset \citep{taner:859,ursin:D23}:
\begin{equation}
t^{k}(x)=\dsum_{n=0}^N A_{2n}~x^{2n} \mbox{\ \ with \ \ } k=1,2  \label{eq:Taylor}
\end{equation}

%with
%\begin{equation}
%A_{0}=t_{0}^{2},~~~~A_{2}=\left. \frac{d(t^{2})}{d(x^{2})}\right\vert
%_{x=0},~~~~A_{4}=\frac{1}{2}\left. \frac{d^{2}(t^{2})}{d(x^{2})^{2}}%
%\right\vert _{x=0},  \label{eq:TaylorCoeff}
%\end{equation}

%where $t_{0}$ is the two-way zero-offset time, $A_{2}=V_{N}^{-2}$ is the
%reciprocal of the squared NMO velocity which controls short-spread moveout ($%
%x/z<1$), $A_{4}$ and higher terms describe the nonhyperbolic moveout for
%larger offset ($x/z>1$) due to presence of anisotropy and/or vertical
%heterogeneities.


Although it is possible to derive exact formulas for all the series
coefficients \citep{al-dajani:1738,tsvankin:268,ilyabook2006},
equation \ref{eq:Taylor} \old{still remains an approximation that}
loses its accuracy with increasing offset to depth
ratio. \cite{fomel:U9} introduced recently a generalized functional
form for approximating reflection moveout at large offsets. While the
classic \cite{alkhalifah:1550} 4th-order Taylor/Pad\'{e} approximation
uses three parameters, the generalized approximation involves five
parameters, which can be determined from the zero-offset computation
and from tracing one nonzero-offset ray. In a homogeneous
quasi-acoustic VTI medium \citep{alkhalifah:623}, the generalized
approximation of \cite{fomel:U9} reduces to the three-term traveltime
approximation \old{proposed by} \new{of} \cite{FomelVTI}, which is
practical and more accurate than other known \old{analytical}
\new{three-parameter} formulas for non-hyperbolic moveout.

\inputdir{.}
\multiplot{2}{commonray1,commonray2}{height=0.4\textwidth}{Comparison between event geometry in \tx (a) and \taup (b). The \taup domain naturally unveils the position of
equal slope events. If the medium is a stack of horizontal homogeneous
layers, a \taup trace collects the contribution of rays, with ray parameter $p$, that share common ray segments
in each layer. Moreover, local slopes $R=\frac{d\tau }{dp}$ are related to
emerging offset $x=-R$. After the original \tx 
data is \taup transformed, we can measure zero-slope traveltime $\Delta
\tau_0=\frac{\partial \tau_0}{\partial \tau}\Delta \tau$ and offset $\Delta x=-\frac{\partial R}{\partial \tau}\Delta \tau$ differences at common slope $p$ points, by simple differentiation along $\tau$. \label{fig:commonray}}

The \taup domain provides an attractive alternative to computing
P-wave reflection-moveout curves. The \taup transform stacks the data
gathered in \tx domain along straight lines, whose direction
\begin{equation}
t~=~\tau~+~p~X,  \label{eq:tauptransform}
\end{equation}
is parametrized by the horizontal slowness $p$ and the intercept time
$\tau$ (blue lines in figure~\ref{fig:commonray}a). Hence, the \taup
transform maps the data to the slowness domain, where traveltime
depends on the vertical components $q$ of the down and upgoing phase
slowness [equation 20 in \cite{baan:1076}]. Considering an anisotropic
medium with a horizontal symmetry plane (VTI, HTI and orthorhombic with
one of the symmetry axis aligned to the depth direction), the \taup
reflection moveout formula simplifies to
\begin{equation}
\tau (p)=\tau _{0}~V_{P0}~q(p)\;,  \label{eq:taupmoveout}
\end{equation}
where $\tau _{0}=\frac{2z}{V_{P0}}$ is the zero-slope/zero-offset two
way traveltime in a homogeneous layer with thickness $z$ and vertical
velocity $V_{P0}$. According to the Christoffel equation,
$q(p)=\sqrt{1/v^{2}(p)~-~p^{2}}$ and $v=v(p)$ is the phase velocity as
a function of \new{the} ray parameter $p$.

%To circumvent this problem, a plane-wave
%decomposition such as the  can be applied to the data.
%This transformation extracts the phase slowness since the component of the
%group velocity vector $\mathbf{V}$ parallel to the normal of the plane wave
%equals the phase velocity, which \ is
%\begin{equation}
%\mathbf{V}^{T}\mathbf{p=}1,  \label{eq:14}
%\end{equation}
%
%where $\mathbf{p}$ represents the phase slowness vector (see eq. 1.17 in
%\cite{ilyabook2006}). Assuming a single horizontal homogeneous layer, the
%vector ray segment $\mathbf{l}$ is a traveltime $t$ scaled version of the
%group velocity vector,
%
%\begin{equation}
%\mathbf{l=}t\mathbf{V.}  \label{eq:travelGroup}
%\end{equation}
%
%If the investigated plane wave is traveling in plane and it is reflected at
%depth $z$ at the horizontal location $x,$ we have that $\mathbf{l=[}x,z%
%\mathbf{]}^{T}$ . Then, substituting equation \ref{eq:travelGroup} in
%equation \ref{eq:14} we obtain the well known \taup transform relation
%
%\begin{equation}
%t=px+\tau ,  \label{eq:tauptransform}
%\end{equation}
%
%where $p$ is the horizontal slowness, $x$ is the horizontal offset traveled
%by the plane wave in the layer and $\tau $ is the intercept time. In general
%$\tau =z(\grave{q}+\acute{q}),$ where $\grave{q}$ and $\acute{q}$ are the
%vertical slownesses for downgoing and upgoing plane waves propagating in the
%layer. Moreover, in anisotropic media with an horizontal symmetry plane
%(like VTI, HTI or orthorhombic media) $\grave{q}$ $=$ $\acute{q}=q$, thus
%the total intercept time becomes simply

%\begin{equation}
%\tau =2zq.  \label{eq:intercepttime}
%\end{equation}

%Using the fact that zero-slope or zero-offset two way traveltime $\tau _{0}=%
%\dfrac{2z}{v_{0}}$ in combination with equation \ref{eq:intercepttime}, we
%obtain an expression for describing moveout curve in \taup:

%\begin{equation}
%\tau (p)=\tau _{0}v_{0}q(p).  \label{eq:taupmoveout}
%\end{equation}


Equation \ref{eq:taupmoveout} remains exact as long as we use the
exact expression for the phase velocity $v(p)$ (red solid line in
\rfg{taupVS}a). Exact expressions exist for all types of anisotropic
media with a horizontal symmetry plane. Unfortunately, the exact and
the highly accurate \citep{Stovas:SEG2010} expressions for \taup
signatures are not very practical because they depend on multiple
parameters. In practice, one may prefer to employ three-parameters
approximate relations for the phase velocity. Although these
signatures are approximate, they are more reliable then the \taup
transformed version of their dual-pair in the \tx domain (figure \ref{fig:taupVS}b). 

\begin{comment}
Using the symmetric relation that holds between phase and group
velocity \citep{dellingerSEP}, we can prove that the 4th order
\cite{alkhalifah:1550} Taylor/Pad\'{e} formula [lower triangles in
  figure \ref{fig:taup_comparison}] and the \cite{FomelVTI}
shifted-hyperbola approximation [diamonds] comes from
\cite{Thomsen1986} or\cite{byun:1564} weak anisotropy [squares] and
\cite{alkhalifah:1239} quasi-acoustic [upper triangles] phase velocity
approximations respectively.  The Taylor approximation begins to lose
its accuracy even at intermediate offset/slopes. Although
\cite{FomelVTI} formula performs better, the Alkhalifah quasi-acoustic
dispersion relation is still more accurate up to longer offset/slopes.
\end{comment}

\inputdir{.}
\multiplot{2}{taup,error-taup}{width=0.45\textwidth}{
(a) $\tau (p)$ signatures: 
\cite{Thomsen1986} or \cite{byun:1564} weak anisotropy
(squares) and \cite{alkhalifah:1239} quasi-acoustic 
(blue upper triangles) approximation.  (b) percentage error with respect to the
exact formulation (red solid line). These curves are computed in a
layer of 1.24 km thickness with $V_{P0}=3.2$ km/s, $V_{N}=2.9$ km/s,
$V_{H}=3.9$ km/s, $\eta$=0.44 and $ V_{P0}/V_{S0}=2.5.$ The Taylor
curve (lower triangles) represent the \taup transformed quartic
moveout traveltime approximation described in
\cite{alkhalifah:1550}. The Fomel curve (diamonds) is
the \taup mirror of the \tx moveout formula based on the
shifted hyperbola approximation for the group velocity introduced by
\cite{FomelVTI}.
\label{fig:taupVS} 
}



\begin{comment}
Equation \ref{eq:taupmoveout} can also be used for inversion
purposes. Since \taup approximate relations predict reflection moveout
better than their \tx pair, they can provide in principle more refined
estimates for anisotropy parameters characterizing the investigated
subsurface.
\end{comment}

We can extend the result in equation \ref{eq:taupmoveout} to a stack
of $N$ horizontal homogeneous layers with horizontal symmetry
planes. According to Snell's law, the horizontal slowness $p$ is
preserved upon propagation \old{across} \new{through} each layer. Thus, the total
intercept time $\tau $ from the bottom of $N$-th layer is the summation
of each interval intercept time $\Delta \tau _{n}$ in the $N$
contributing layers:
%\begin{equation}
%\tau =\sum\nolimits_{n=1}^{N}\Delta \tau
%_{n}=2\sum\nolimits_{n=1}^{N}z_{n}q_{n}.  \label{eq:intercepttime_layer}
%\end{equation}
\begin{equation}
\tau (p)=\sum\nolimits_{n=1}^{N} \Delta \tau_{n} =  \sum\nolimits_{n=1}^{N}~V_{P0,n}~q_{n}(p)~\Delta \tau _{0,n}~~,
\label{eqn:taup_layers}
\end{equation}
where each single intercept time $\Delta \tau _{n}$ obviously depends
just on interval parameters characterizing the $n$-th layer.
\rEq{taup_layers} states that in the \taup domain both the ray tracing
(forward modeling) and layer stripping (inversion) are linear
processes.  Each trace in a \taup gather collects the contribution of
rays that share common segments of trajectory in the layers
(\rFg{commonray}a).  Moreover, the \taup domain helps us also to
\old{easily} isolate the effect of a single layer, thereby producing
an estimate for its interval properties. Layer stripping is
self-explanatory: by literally subtracting all the unwanted layers,
one can isolate the contribution of a specific layer and access
directly its interval parameters without passing through the effective
parameters as normally happens in \tx domain. In \tx processing, the
traveltime curves are inverted using a two-step procedures. First,
effective parameters are obtained by semblance scans. \old{Then}
\new{Next}, interval parameters are obtained using \new{the} Dix
formula or layer-stripping procedures. The \taup processing offers
instead a direct access to interval parameters.

The summation in \req{taup_layers} can be substituted by a convenient
relation in term of the effective parameters obtained from the Dix
average of interval ones. This result will be used in the
\old{following} \new{next} section to derive a closed-form expression
for P-waves \taup reflection moveout in terms of interval or effective
normal-moveout velocity $V_N$ and horizontal velocity $V_H$ (or,
alternatively, $\eta$): these two parameters control all time-domain
processing steps in VTI media \citep{alkhalifah:1550}.

\begin{comment}
The high accuracy, the ability for unwrapping wavefront cusps
\citep{baan:1076}, and the linearity of traveltime modeling and
inversion enforce that \taup domain is the natural domain to handle
seismic anisotropy in a layered or vertically varying medium.
\end{comment}

\section{Velocity Independent $\tau $-p moveout in VTI}

Under the quasi-acoustic approximation 
\citep{alkhalifah:623,alkhalifah:1239} 
in a VTI homogeneous layer, the
vertical slowness $q$ can be expressed as a function of the horizontal slowness $p$: 

\begin{equation}
q(p)=\frac{1}{\hat{V}_{P0}}\sqrt{\dfrac{1-\hat{V}_{H}^{2}p^{2}}{1-[\hat{V}_{H}^{2}-\hat{V}_{N}^{2}]p^{2}}},
\label{eq:qALKHA}
\end{equation}  

where $\hat{V}_{H}$ and $\hat{V}_{N}$ are the horizontal and normal
moveout velocity in the layer. In the following \new{equations}, the
hat superscript $(\symbol{94})$ indicates layer or interval
parameters. Considering a stack of $N\ $horizontal homogeneous layers
with horizontal symmetry planes, we can insert equation
\ref{eq:qALKHA} into \req{taup_layers} \old{and} \new{to} obtain an
expression for the \taup reflection time from the bottom of the $N$-th
layer, as follows:
\begin{equation}
\tau
(p)=\sum\nolimits_{n=1}^{N}\sqrt{\dfrac{1-\hat{V}_{H,n}^{2}p^{2}}{1-[\hat{V}_{H,n}^{2}-\hat{V}_{N,n}^{2}]p^{2}}}\Delta
\tau _{0,n}
\label{eq:taupALKHAlayer}
\end{equation}
where $\Delta \tau _{0,n}$ is the two way vertical time for the $n$-th
layer.  To simplify the following theoretical derivations, we assume
that, instead of having a layered velocity model, interval parameters
are vertically-varying continuous profiles. Therefore, we replace the
summation in formula \ref{eq:taupALKHAlayer} with an integral along
the vertical time $\tau _{0}$ and \old{obtain} \new{arrive at} the
following \taup moveout formula for a vertically-heterogeneous VTI
medium:
\begin{equation}
\tau (p)=\dint\limits_{0}^{\tau _{0}}\sqrt{ \dfrac{1-\hat{V}_{H}^{2}(\xi
)p^{2} } {1-[\hat{V}_{H}^{2}(\xi )-\hat{V}_{N}^{2}(\xi )]p^{2}}}d\xi ,
\label{eqn:taup_int}
\end{equation}
where \old{$\hat{V}_{N}=\hat{V}_{N}(\tau _{0})$ and
$\hat{V}_{H}=\hat{V}_{H}(\tau _{0})$}
\new{$\hat{V}_{N}=\hat{V}_{N}(\xi)$ and
$\hat{V}_{H}=\hat{V}_{H}(\xi)$} are (smooth) functions for interval
NMO and horizontal velocities, and $\tau _{0}$ is the vertical
time. The vertical heterogeneity is measured as a function of $\tau
_{0}$. The anellipticity parameter $\hat{\eta}=\frac{1}{2}\left(
\frac{\hat{V}^2_{H}}{\hat{V}^2_{N}}-1\right)$ is \new{also} a function
of the vertical time $\tau _{0}$. Using effective parameters, \old{the
expression} \new{equation} \ref{eqn:taup_int} can be \old{further}
approximated by
\begin{equation}
\tau (p)\approx \tau _{0}\sqrt{\dfrac{1-V_{H}^{2}(\tau
_{0})p^{2}}{1-[V_{H}^{2}(\tau _{0})-V_{N}^{2}(\tau _{0})]p^{2}}},
\label{eqn:taup_effective}
\end{equation}
where the effective NMO $V_{N}$ and horizontal $V_{H}$ velocity are
\old{connected} \new{related} to the interval parameters through the second- and
fourth-order average velocities \citep{taner:859,ursin:D23} by the
following direct Dix-type formulas:
\begin{equation}
V_{N}^{2}(\tau _{0})=\dfrac{1}{\tau _{0}}\dint\limits_{0}^{\tau
_{0}}\hat{V}_{N}^{2}(\xi )d\xi , \label{eqn:2ndmoment}
\end{equation}
\begin{equation}
S(\tau _{0})V_{N}^{4}(\tau _{0})=\dfrac{1}{\tau _{0}}\dint\limits_{0}^{\tau
_{0}}{}\hat{S}(\xi )\hat{V}_{N}^{4}(\xi )d\xi ,  \label{eqn:4thmoment}
\end{equation}
where $S$ is the ratio between the fourth- and second-order
moments or the heterogeneity factor \citep{bazelaire:143,alkhalifah:1839,siliqi:2245}. 

\old{The approximation given
in equation \ref{eqn:taup_effective} is valid at small angles (small
values of $p$) because it shares a Taylor series expansion near $p=0$
with equation \ref{eqn:taup_int}.}

\new{ \rEq{taup_effective} is basically the four-parameters rational approximation defined in $\tau$-$p$ domain} \cite{Stovas:SEG2010}
\begin{equation}
\tau (p)\approx \tau _{0}\sqrt{1-V_{N}^{2} p^{2} + \frac{A\,V_{N}^{4}\,p^4}{1-B\,V_{N}^{2}\,p^2 } }  ,
\label{eqn:taup_rational}
\end{equation}
\new{with parameter $A=(1-S)/4$ defined from the Taylor series
expansion of the exact \taup function} \citep{ursin:D23}. \new{Under
the acoustic VTI approximation, $B=-A$ and the \req{taup_rational} now
depends on three parameters only.  Finally, to be consistent with
\req{taup_effective}, the heterogeneity coefficient becomes
$S=4\dfrac{V_H^2}{V_N^2}-3$. In principle, it is possible to use any
other three-parameters approximation in \taup domain apart from the
rational approximation \ref{eqn:taup_rational} like, for example, the
shifted ellipse approximation given by}
\cite{Stovas:SEG2010}. \new{The reason for choosing the approximation
\ref{eqn:taup_effective} is that it accurately describes the \taup
moveout for a single VTI layer (blue line in figure
\ref{fig:taupVS}b). Nevertheless, approximation
\ref{eqn:taup_effective} remains valid for vertically heterogeneous
VTI media with a decrease in accuracy for larger angles (large values
of $p$) because of the Dix averaging}

Letting $R$ represent the slope $\tau ^{\prime }(p)$ and $Q$ the curvature $\tau ^{\prime \prime
}(p)$, we differentiate equation \ref{eqn:taup_effective} once
%\begin{equation}
%R=\frac{{d\tau }}{{dp}}=-\frac{{\tau _{0}^{2}(V_{H}^{2}-A)p}}{{\tau \left( {%
%1-p^{2}A}\right) ^{2}}}=-\frac{{\tau (V_{H}^{2}-A)p}}{{\left( {1-p^{2}A}%
%\right) \left( {1-p^{2}V_{H}^{2}}\right) }},  \label{eqn:slope}
%\end{equation}%
%and twice
%\begin{equation}
%Q=\frac{{dR}}{{dp}}=-\frac{{\tau _{0}^{4}F(p)(V_{H}^{2}-A)}}{{\tau
%^{3}\left( {1-p^{2}A}\right) ^{4}}}=-\frac{{\tau F(p)(V_{H}^{2}-A)}}{{\left(
%{1-p^{2}A}\right) ^{2}\left( {1-p^{2}V_{H}^{2}}\right) ^{2}}},
%\label{eqn:curvature}
\begin{equation}
R(\tau ,p)=-\frac{{\tau (V_{H}^{2}-Y)p}}{{\left( {1-p^{2}Y}\right)
\left( {1-p^{2}V_{H}^{2}}\right) }}, \label{eqn:slope}
\end{equation}
and twice
\begin{equation}
Q(\tau ,p)=-\frac{{\tau F(p)(V_{H}^{2}-Y)}}{{\left(
{1-p^{2}Y}\right) ^{2}\left( {1-p^{2}V_{H}^{2}}\right) ^{2}}},
\label{eqn:curvature}
\end{equation}
where $F(p)=1+2p^{2}Y-3p^{4}V_{H}^{2}Y$ with $Y=V_{H}^{2}-V_{N}^{2}.$
Equations \ref{eqn:slope} and \ref{eqn:curvature} provide an
analytical description of the slope and curvature fields for given
effective values $V_{N}$ and $V_{H}$. Here we have omitted the ${\tau
_{0}}$ dependency for clarity in the notation. Since \taup and \tx
domains are mapped by the linear transformation in equation
\ref{eq:tauptransform}, we observe that
\begin{equation}
\tau ^{\prime }(p)=R=-x.  \label{eqn:emergingoffset}
\end{equation}
Thus, the negative of the slope $R$ has the physical meaning of
emerging offset, as pointed out by \cite{baan:719}. \new{Moreover, when the curvature $Q$ changes sign, there is an inflection point in the $\tau$-$p$ wavefront that is as a condition for caustics in $t$-$X$ domain.} \citep{roganov}.

Given slope $R$
and curvature $Q$ fields in a \taup CMP\ gather, we can eliminate the
velocity $V_{N}$ and the parameter $Y$ in equations \ref{eqn:slope}
and \ref{eqn:curvature}, \new{thus} obtaining \ a \textquotedblleft
velocity-independent\textquotedblright\ \citep{fomel:S139} moveout
equation in the \taup domain:
\begin{equation}
\tau _{0}(\tau ,p)=\tau \sqrt{\frac{{\tau pQ+3\tau R-3pR^{2}}}{{\tau
pQ+3\tau R+pR^{2}}}.}  \label{eqn:tau0mapping}
\end{equation}

Equation \ref{eqn:tau0mapping} describes a direct mapping from events
in the prestack \taup data domain to zero-slope time $\tau _{0}$. This
equation represents the oriented or slope-based moveout correction. As
follows from equations \ref{eqn:slope} and \ref{eqn:curvature}, the
effective parameters, if needed for other tasks, are given by the
following relations as a function of the slope and curvature estimates
(see Table \ref{tbl:velocities}):
\begin{eqnarray}\label{eqn:vNmap} 
V_{N}^{2}(\tau, p) &=&-\frac{1}{p}\frac{{16\tau R^{3}}}{{ND}},\\   
V_{H}^{2}(\tau, p) &=&\frac{1}{{p^{2}}}\dfrac{{N-4\tau R}}{N},  \label{eqn:vHmap}
\end{eqnarray}
%\begin{equation}
%A=\frac{1}{{p^{2}}}\frac{{D-4\tau R}}{D},  \label{eqn:Amap}
%\end{equation}
and
\begin{equation}
\eta(\tau, p)=\frac{1}{p}\frac{{N(4\tau R-D)}}{{32\tau R^{3}}}.  \label{eqn:etamap}
\end{equation}
In the above equations, $N={\tau pQ+3\tau R-3pR^{2}}$ and $
D={\tau pQ+3\tau R+pR^{2}}$ represent the terms in the numerator $N$ and denominator $D$
of the square root in equation \ref{eqn:tau0mapping}. In the isotropic
or elliptically anisotropic case ($V_N=V_H$ or $\eta=0$), equations
\ref{eqn:tau0mapping} to \ref{eqn:etamap} simplify to equations
\begin{equation}
\tau _{0}(\tau ,p) = \sqrt{\tau ^{2}-\tau pR}  \label{eqn:tau0mappingISO} 
\end{equation}
and
\begin{equation}
V_{N}^{2}(\tau ,p) = \frac{R}{{p(pR-\tau )}}\;,  \label{eqn:vNmapISO}
\end{equation}
previously published by \cite{fomel:S139}.

The anisotropic parameters $V_{N}$ and $V_{H}$ (or $\eta $) are no
longer a requirement for the moveout correction, as \old{it is}
\new{in} the case \old{in} \new{of} conventional NMO processing, but
rather they are \old{a} data attributes derived from \old{the} local
slopes and curvatures. Moreover, these parameters are mappable
directly to the appropriate zero-slope time $\tau _{0},$ \old{based
on} \new{according to} \req{tau0mapping}.
