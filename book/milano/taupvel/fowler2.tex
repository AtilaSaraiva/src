\subsection{Fowler's equations}

The second alternative to get VTI interval parameters comes from
the integral formulation of the \taup moveout signature in
\req{taup_int}. \new{The derivation is detailed in Appendix C.} We first compute
\mbox{$\tau_{0,\tau}=\partial\tau_{0}/\partial\tau$}  and
then, applying the chain rule, $R_{\tau}$. Solving for $\hat{V}_{N}$
and $\hat{V}_{H}$, we arrive at the following relations:
\begin{eqnarray}
\hat{V}_{N}^{2}(\tau,p) &=&-\frac{[ \tau_{0,\tau}^{2}-1]^{2}}{%
p^{3}\tau_{0,\tau}^{2}~R_{\tau} }, \label{eqn:fowler1} \\ 
\hat{V}_{H}^{2}(\tau,p) &=& \frac{\tau_{0,\tau}^{2}[1-R_{\tau}]+1}{p^{3}\tau_{0,\tau}^{2}~R_{\tau}~},
\label{eqn:fowler2}
\end{eqnarray}
which are equivalent to those proposed previously by
\cite{fowler:3028}. According to \reqs{fowler1} and \ren{fowler2}, the
gradients of offset $x$ and the zero-slope time $\tau_0$ measured at
common slope locations $p$ on two consecutive seismic event return
the VTI interval parameters for the layer bounded by these two events
(\rFg{commonray}a). \cite{fowler:3028} first pick traveltime curves in
\tx domain, and then differentiate those curves in offset to compute
slopes $p$. Finally, for any given $p$ value on each seismic event,
they determine the corresponding $\Delta x$ and $\Delta \tau_0$ values
(\rFg{commonray}a). The main practical limitation in this
inversion scheme is the difficulty of picking seismic events
accurately.

The processing becomes easier if it is accomplished in \taup with
automatic slope estimation. First, \mbox{\taup} transform unveils the
position of equal slope events.  Second, $\tau_{0,\tau}$ and
$R_{\tau}$ are measured automatically (without event picking) on the
\taup transformed CMP gather. The quantity $\tau_{0,\tau}$ can be
estimated as the $\tau$ finite difference of $\tau _{0}$ values
computed according to velocity-independent moveout equation
\ref{eqn:tau0mapping}. The zero-slope time $\tau_0$ function is still
needed to map the interval parameter estimated using \reqs{fowler1}
and \ren{fowler2} to the correct vertical time (Table
\ref{tbl:velocities}).







%In this section we propose an alternative approach to the oriented Dix
%method discussed above. Although the mathematical derivation is different,
%the aim is still to invert data local slopes estimate to obtain interval
%anisotropic parameters $\hat{V}_{N}$ and $\hat{V}_{H}$ (or $\hat{\eta})$. As
%previously stated, we require that the medium is laterally homogeneous, in
%order that the ray parameter $p$ will be preserved along a given ray. %
%\citet{Claerbout.sep.14.13} shows that traveltime and offset difference
%between equal slope $p$ points on two seismic events, can be inverted for
%seismic velocity in the layer bounded by these events. In layered isotropic
%media, he obtained this simple direct inversion formula
%
%\begin{equation}
%\hat{V}^{2}=\frac{\Delta x}{p\Delta t},  \label{eq:Clarebout}
%\end{equation}
%
%where $\hat{V}$ is the interval velocity, $\Delta x$ is the offset
%difference and $\Delta t$ is the traveltime difference, (see figure \ref%
%{fig:commonray} for a graphical interpretation of this quantities\ ) which
%we measure directly on the $\tau -p$ transformed CMP gather .
%
%\inputdir{extFig}
%
%
%
%We can extend this method to estimate interval parameters in VTI\ lateral
%homogeneous media. It is clear that now we need two independent equations to
%describe offset $\Delta x$ and traveltime $\Delta t$ difference as a
%function of the interval $\hat{V}_{N}$ and $\hat{V}_{H}$ velocities$.$
%
%Since local slopes are related to emerging offset $x$ , we start taking the
%first derivative with respect to the integral moveout formula in equation %
%\ref{eqn:taup_int}, namely
%
%\begin{equation}
%x=-R(\tau ,p)=\dint\limits_{0}^{\tau _{0}}\hat{V}_{P0}(\xi )q^{\prime
%}(p,\xi )d\xi ,  \label{eq:emerging_offset}
%\end{equation}
%
%where $\hat{V}_{P0}=\hat{V}_{P0}(\tau _{0})$ is a vertical profile for the
%vertical phase velocity and $q^{\prime }(p,\tau _{0})=\dfrac{dq(p,\tau _{0})%
%}{dp}$ is the first derivative of vertical slowness $q,$ which is as a
%function of interval parameter $\hat{V}_{N}=\hat{V}_{N}(\tau _{0})$ and $%
%\hat{V}_{H}=\hat{V}_{H}(\tau _{0})$. According to \ref{eq:tauptransform} the
%total traveltime $t$ becomes
%
%\begin{equation}
%t=\dint\limits_{0}^{\tau _{0}}\hat{V}_{P0}(\xi )\left( q^{\prime }(p,\xi
%)-pq^{\prime }(p,\xi )\right) d\xi  \label{eq:totaltraveltime}
%\end{equation}
%
%If we consider an arbitrary thin layer, with vertical interval time $\Delta
%\tau _{0},$ in which the medium parameter are assumed constant, we can
%reduce equation \ref{eq:emerging_offset} and \ref{eq:totaltraveltime} into
%their interval versions:
%
%\begin{eqnarray}
%\Delta x &=&-\Delta R(\Delta \tau _{0},p)=-\hat{V}_{P0}\Delta \tau
%_{0}q^{\prime },  \label{eq:DELTAX} \\
%\Delta t &=&\hat{V}_{P0}\Delta \tau _{0}\left( q-pq^{\prime }\right) ,
%\label{eq:DELTAT}
%\end{eqnarray}
%
%where $\Delta x$ and $\Delta R(\Delta \tau _{0},p)$ are the finite
%difference of offset and slope field at the boundaries of the layer, and $%
%\Delta t$ is the total traveltime that a ray with horizontal slowness $p$
%takes to cross the layer.
%
%
%\plot{commonray}{width=0.8\textwidth}{Common ray geometry considered for the inversion procedure. \label{fig:commonray}}
%
%In order to relate vertical $q$ to horizontal $p$ slowness we use the
%approximate relation in equation \ref{eq:qALKHA}. Substituting the first
%derivatives of vertical slowness function given in \ref{eq:qALKHA} we have
%
%\begin{eqnarray}
%\Delta x &=&\frac{\Delta \tau _{0}}{p~a~b^{3}}(b^{2}-a^{2}), \\
%\Delta t &=&\frac{\Delta \tau _{0}}{a~b^{3}}(b^{2}+a^{2}b^{2}-a^{2}),
%\end{eqnarray}
%
%where $a^{2}=1-p^{2}\hat{V}_{H}^{2}$ and $b^{2}=1-p^{2}(\hat{V}_{H}^{2}-\hat{%
%V}_{N}^{2}).$ We have finally obtained two independent equations for the two
%interval VTI parameters. In the case of isotropy or elliptical anisotropy ($%
%b^{2}=1$) they reduce to equation \ref{eq:Clarebout} previously published by
%\cite{Claerbout.sep.14.13}. After solving this system of equations for the
%interval values $\hat{V}_{N}$ and $\hat{V}_{H},$ we obtain the following
%inversion equations
%\begin{eqnarray}
%\hat{V}_{N}^{2} &=&\frac{(\Delta \tau _{0}^{2}-\Delta \tau ^{2})^{2}}{%
%p^{3}\Delta \tau _{0}^{2}~\Delta x~\Delta \tau },  \label{eqn:VNoriented} \\
%\hat{V}_{H}^{2} &=&\frac{\Delta \tau _{0}^{2}(p\Delta x~-\Delta \tau
%)+\Delta \tau ^{3}}{p^{3}\Delta \tau _{0}^{2}~\Delta x~},
%\label{eqn:VHoriented}
%\end{eqnarray}
%
%where $\Delta \tau =\Delta t-p\Delta x.$ These equations have already been
%derived in \cite{fowler:3028} but they were not applied in the context of
%oriented velocity mapping. \cite{fowler:3028} first pick traveltime curves
%in $t-x$ \ domain, and then they differentiate those curves over offset to
%compute slopes $p$. Finally, for any given $p$ value on each seismic event,
%they determine the corresponding $\Delta x$ and $\Delta t$ values (as
%sketched in figure \ref{fig:commonray}). The main practical limitation in
%this inversion scheme is the possibility to pick seismic events accurately
%and have reliable estimates of the event slopes. It's clear that this
%inversion scheme cannot be fully automated and requires time and manual work
%in the picking procedure.
%
%However according to oriented velocity mapping outlined in previous
%sections, we can measure from $\tau -p$ data the local slope field $R(\tau
%,p)$ using the regularized plane-wave destruction (PWD) algorithm. The $\tau
%-p$ transform naturally aligns seismic events with equal slope (figure \ref%
%{fig:taupscheme}) along the same trace. Moreover, since local slopes are
%related to the emerging offset $x=-\dfrac{d\tau }{dp}$ \citep{baan:719}$,$%
%the offset difference $\Delta x~$can be computed as the finite difference of
%the negative slope field $R(\tau ,p)$ along the $\tau $ direction. The
%quantity $\Delta \tau _{0}$ can be estimated as the finite difference of $%
%\tau _{0}$ values computed according to velocity-independent\ moveout
%equation \ref{eqn:tau0mapping} in ${\tau -p}$ domain.
%
%Finally, the equation \ref{eqn:VNoriented} and \ref{eqn:VHoriented} describe
%the interval parameters $\hat{V}_{N}$ and $\hat{V}_{H}$ as data attributes
%that we can extract directly from the data through local slopes estimate.
%Then we can also map these attributes to the appropriate zero slope-time $%
%\tau _{0}$ according to equation \ref{eqn:tau0mapping}. Figure \ref%
%{fig:dataSynthClar} shows the interval NMO velocity $\hat{V}_{N}$ (a) $,$
%interval horizontal velocity $\hat{V}_{H}$ (b) and interval anellipticity $%
%\hat{\eta}$ (c) oriented mapping according to equation \ref{eqn:VNoriented}
%and \ref{eqn:VHoriented} for the synthetic data example shown in figure \ref%
%{fig:dataSynth} (a).We observe a good match between the exact and the
%recovered \ interval parameters profiles, although $\hat{V}_{H}$ and $\hat{%
%\eta}$ display a loss in resolution mainly due to instabilities of numerical
%differentiation of slopes $R(\tau ,p)$ and $\tau _{0}(\tau ,p)$ field.
%Moreover these results are quite similar to those obtained by oriented Dix
%inversion (figure \ref{fig:dataSynthDix}): this confirms the complete
%equivalence of both methods, even though they are derived in different ways.
%Numerical instabilities affected the field data proessing even more and the
%method produces a reliable estimate of the NMO\ velocity profile only
%(Figure \ref{fig:dataCongoInterval} (b)).


%\begin{figure}[tbph]
%\begin{center}
%\subfigure[][]{\includegraphics[height=7.0
%cm]{fig/figSynthSin/oriented_Vnmo_int.pdf}}
%\subfigure[][]{\includegraphics[height=7.0
%cm]{fig/figSynthSin/oriented_Vhor_int.pdf}}
%\subfigure[][]{\includegraphics[height=7.0
%cm]{fig/figSynthSin/oriented_eta_int.pdf}}
%\end{center}
%\caption{Oriented direct inversion to interval normal moveout (a) and
%horizontal velocity (b) and anellipticity parameter $\eta $ (c).
%Blue dotted lines represent the exact values used for generating the
%synthetic dataset in figure \ref{fig:dataSynth} (a).}
%\label{fig:dataSynthClar}
%\end{figure}
