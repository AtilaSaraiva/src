\section{Field data example}

\inputdir{cmp}

\plot{dataP}{width=\textwidth}{ \taup or Radon transformed data
from a marine acquisition (a). The data are fairly clean even though
there is a slight decrease in SNR for later and steepest
events. The CMP maximum offset-to-depth ratio 
reaches 1.5 for larger value of the horizontal slope $p$.
Dominant local-slope field (b) measured using PWD
algorithm. Using the slopes we estimate the zero-slope time $\tau_0$
mapping fields (c) that predicts reflection curves by which we flatten
the original gather (d).}

\rFg{dataP} presents the results of the proposed \taup processing on a
 field data example from a marine acquisition.  \rFg{dataP}a \old{is
 the} \new{shows a} \taup transformed CMP gather. The data are taken
 from a deep water (3.5 s of sea-depth) dataset with poor offset
 sampling ($\sim 100~m$) that aliases the steepest seismic
 events. Spatial aliasing creates artifacts in \taup domain that bias
 the PWD slope estimate.  In order to mitigate the effect of the
 aliasing, we interpolated the raw data by means of an FX algorithm
 \citep{spitz:785}.  The original trace recording is 7.0~s long but,
 since the SNR\ decreases \new{significantly} after 5.0~s, \old{such
 that even semblance analysis is less reliable,} we window the CMP
 gather and \old{we} process seismic events only between 3.0 and
 6.0~s.  \new{The CMP maximum offset-to-depth ratio reaches 1.5 for
 larger value of the horizontal slope $p$. As for the synthetic case,
 the data should carry enough information to well resolve the
 horizontal velocity and the anellipticity parameter.}  \rFg{dataP}b
 shows the \old{data} dominant local-slope $R$ field \old{we}
 automatically measure from the data using the PWD algorithm. As in
 the synthetic case, we use these slopes to construct the prediction
 operator that allows \new{us} to paint the zero-slope traveltime map
 $\tau_0$ along the reflection events (\rFg{dataP}c).  The $\tau_0$
 values are finally used to unwrap the trace shifts until the gather
 is completely flattened. The good alignment of the NMO corrected
 traces (\rFg{dataP}d) confirms the robustness of predictive painting
 with real data.  \rFg{int-painting-mask} shows spectra of the
 recovered interval parameters using Fowler's equations \ren{fowler1}
 and \ren{fowler2}. The plots are overlaid with the profiles (yellow
 curves) recovered using a layer-based \tx Dix inversion
 \citep{ferla:296} and by the profiles obtained after an automated
 picking of the recovered trends. Our solution (red curves) follows
 the Dix trends (yellow curves) even though it exhibits a slight
 decrease in accuracy.  The poor SNR for later and steeper events and
 the numerical differentiation of the zero-slope traveltime $\tau_0$
 and slope $R$ make the field data results noisier. As expected, the
 high-order moveout parameters appear to be more sensitive to the
 noise. Moreover, the more pronounced enlargement of the $\hat{\eta}$
 trend \old{than} \new{in comparison with the} $\hat{V}_H$ trend
 confirms that the latter parameter is better constrained by the data
 \citep{ilyabook2006}.

\plot{int-painting-mask}{width=\textwidth}{Parameters spectra for
interval normal moveout (a), horizontal velocity (b) and anellipticity
parameter $\eta $ (c). These spectra result from the application of
Fowler's equations using painted $\tau_0$ field. The red lines are the
profiles after an automated picking of the estimated spectra. The
yellow lines are the recovered profiles after a layer-based \tx Dix
inversion procedure \citep{ferla:296}. }


%As in the synthetic example, the CMP gather is properly flattened by the
%application of oriented NMO equation \ref{eqn:tau0mapping} using local event
%slopes. The effective $V_{N},$ $V_{H}$ and $\eta $ mapping after oriented
%NMO is shown in figure \ref{fig:dataCongoOriented}: the superimposed dotted
%blue curves represent the effective parameter trend picked after
%conventional two parameters semblance analysis. Again we can notice a
%significantly higher resolution in parameters estimates.




%\plot{dataNMO1}{width=0.8\textwidth}{(a) Time mapping of each data sample from $\tau -p$ time to
%the zero-slope time $\tau _{0}$ according to relation \ref%
%{eqn:tau0mapping}. (b) The output of oriented velocity independent NMO for
%the field data set investigated: the event are correctly flattened without
%knowing the velocity model. \label{fig:dataCongoNMO}}



%\plot{eff}{width=0.8\textwidth}{Effective normal moveout velocity (a), horizontal velocity (b) and
%anellipticity parameter $\eta $ (c) computed as a data attribute
%through the local estimate of slopes and curvature. These values are mapped
%to the appropriate zero-slope $\tau _{0}$ time by using oriented NMO
%equation given in \ref{eqn:tau0mapping}. The blue dotted lines
%indicate the values obtained from two-parameters anisotropic velocity-scan
%for the field data in figure \ref{fig:dataCongoFX} (a). \label{fig:dataCongoOriented}}
