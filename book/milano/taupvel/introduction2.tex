\section{Introduction}

%1 The estimation of seismic velocities is required for building a
macro model of the subsurface and remains one of the most
labor-intensive and time-consuming procedures in the conventional
approach to seismic data analysis \citep{yilmaz}.  In time-domain
imaging, effective seismic velocities are picked from coherency
scans. Moreover, anisotropic velocity model building requires more
than just a single parameter scan for nonhyperbolic traveltime
approximation \old{, as suggested by} \citep{alkhalifah:1550}. This
means that anisotropic velocity analysis is at least twice more
computationally intensive than its traditional isotropic counterpart.
Conventional human-aided velocity analysis takes up a significant part
of the time needed to process seismic data.  Even with semi-automatic
picking software, this phase alone might take weeks or even months for
modern 3D data sets. Several approaches have been proposed to
automatize and simplify velocity analysis and traveltime picking
procedures \old{, greatly reducing} \new{in order to reduce} the time
and manual work required for handpicked velocities
\citep{lambare:VE25,lambare:2076,Siliqi2007}.  However, these tools
still require significant manual inspection and editing for quality
control.

%2
The idea of using local event slopes estimated from prestack seismic
data goes back to the work of \cite{rieber:97} and
\cite{riabinkin}. Several following papers outline the importance of
local data slopes in seismic data processing, particularly the role
that slope estimates play in the algorithm of stereotomography
\citep{Sword.sepphd.55,lambare:VE25,lambare:2076}.  The concept of
velocity-independent time-domain imaging goes back to
\cite{ottolini}. \cite{wolf:2423} pointed out that it is possible to
perform hyperbolic moveout velocity-analysis by estimating local data
slopes in the prestack data domain using an automated method such as
plane-wave destruction \citep{FomelPWD}. This methodology is
attractive because it can be less time-consuming than the manual work
required to handpick velocities.

%3
By estimating local event slopes in prestack seismic reflection data,
\cite{fomel:S139} demonstrated that it is possible to accomplish all
common time-domain imaging tasks, from normal moveout to prestack time
migration, without the need to estimate seismic velocities or other
attributes. Local slopes contain complete information about the
reflection geometry. Once they are estimated, seismic velocities and
all the other moveout parameters turn into data attributes and
\old{they are} \new{become} directly mappable from the prestack data
domain into the time-migrated image domain. \cite{fomel:S139} focused
on the isotropic prestack time processing and showed several results
of oriented (slope-based) velocity analysis and imaging, both on
synthetic and real data. Although he developed the mathematical
framework for velocity-independent non-hyperbolic processing in the
time-offset \tx domain, he did not provide examples to demonstrate its
use and efficacy. \cite{burnett:WB129,burnett:3710} extended the
method to 3D elliptically anisotropic moveout corrections.

%4
In this paper, we extend the concept of velocity-independent seismic
processing to P-wave VTI data in the \taup or slant-stack domain
obtained by Radon-transforming CMP data.  We account for
VTI\ anisotropy only, but the theory developed here should work for a
general anisotropic horizontally-layered velocity model. We assume
that each layer is laterally homogeneous with a horizontal symmetry
plane and \new{that} the incidence plane represents a symmetry plane
for the model as a whole so that wave propagation is
two-dimensional. The \taup transform is the natural domain for
anisotropic parameter estimation in layered media
\citep{baan:1076,douma:D53,fomel:2046} because it allows for simpler
and more accurate moveout modeling and inversion than the \old{widely
  known} \new{conventional} methods applied in \tx domain. Since the
horizontal slowness is preserved, each trace in \taup CMP gathers sees
the contributions of rays that share the same segments of trajectory
in the layers. \old{Thus} \new{Therefore}, one can simply sum the
contribution of each individual layer and obtain the overall \taup
moveout signature. This makes modelling or ray tracing a linear
procedure. Moreover, by literally subtracting all the unwanted layers,
we can isolate the contribution of a specific layer and access
directly its interval parameters without relying on the
effective-parameter approximations as normally happens in \tx domain.

%5
After \taup transform, \new{seismic} data are mapped to the slowness
domain, where the reflection signature depends on the vertical
component of phase slowness. Phase velocity is the natural parameter
to work with in the case of anisotropic data, because explicit
expressions exist for phase velocities in all the anisotropic media
that display an horizontal symmetry plane. Unfortunately, exact
expressions for \taup signatures are not \new{always}
practical. Nevertheless, approximate expressions provide accurate
traveltime predictions \citep{tsvankin:75A15}.

%6 
After describing the advantages of processing anisotropic data in the
\taup domain, we derive the oriented (slope-based) NMO equation that
describes direct mapping from prestack data to zero-slope time
(analogous to zero-offset time in \tx domain). We obtain the effective
values of anisotropy parameters as data attributes derived from local
slope and curvature estimates and directly mappable to the appropriate
zero-slope time. \old{This procedure is attractive because it is fully
  automated and less time-consuming than searching for the best-fit
  moveout trajectory through simultaneous two-parameter semblance or
  coherency scans.} \new{Similarly to conventional $t$-$X$ processing,
  several procedures applied in \taup domain rely on coherency
  analysis} \citep{baan:1076,silsen} \new{or traveltime picking plus
  inversion} \citep{wang:WB117,fowler:3028} \new{to retrieve the
  anisotropy parameters. We believe that our procedure is more
  attractive because it is fully automated and less time-consuming
  than searching for the best-fit moveout trajectory through
  simultaneous two-parameter inversion or semblance scans.}


\begin{comment}
%7
\blue{\textit{Interval parameters as well as effective parameters can be regarded as
data attributes obtained from local slopes. Unlike \tx domain,
processing data in \taup offers two alternatives to conventional
\citet{dix:68} inversion for retrieving interval values. The first
alternative is a novel set of equations that we call
\textit{stripping equations}. These equations come from a
straightforward use of layer stripping in \taup domain.
differentiating the slopes and curvatures fields along the intercept
time direction to provide the necessary layer stripping for
interval parameters. \textit{Fowler's equations} represent the second
alternative. These equations have been derived by \cite{fowler:3028} 
and were originally applied in the \tx domain. In this paper, we prove
that these equations can be naturally derived from \taup processing
and conveniently applied in \taup domain. Moreover, a slight
modification of these equations allows us to use them in the context
of slope-based processing.  Both \textit{stripping equations} and
\textit{Fowler's equations} can be considered as the VTI extension of
the \quotetation{straightedge determination of interval velocity}
method introduced by \cite{Claerbout.sep.14.13}. }}

%8
\blue{\textit{The main advantage of the Fowler's equations is that they do not require
access to the curvature field that is absorbed by the zero-slope time
function. In fact, in comparison with its isotropic counterpart
\citep{fomel:S139} , slope processing for VTI media suffers from
numerical issues. Unlike slopes, we don't apply a direct method for
estimating the curvature and compute the curvature by numerically
differentiating the slope estimate. Unfortunately, finite differences
act like a high-pass filter that enhances the high frequency noise
especially when we are dealing with low-SNR field data. This makes
curvature values noisy and biased, thus affecting the anisotropy
parameters estimation especially for those parameters that control
long spread/large angle moveout.}}

%9
\blue{\textit{To circumvent this problem, we suggest yet another way to obtain
interval anisotropic parameters, which relies on the predictive
painting algorithm \citep{fomel:A25}. Predictive painting uses the
local-slope field to construct a prediction operator based on plane
wave destruction filters that, spreading a traveltime reference trace
in the image, predicts the reflection surfaces. Since this approach
does not involves any curvature estimation, it represents a more
robust way for obtaining the zero-slope time mapping field we need (1)
to automatically flatten or NMO correct the \taup CMP gathers (2) to
retrieve interval parameters using the curvature independent Fowler's
equations.  This last approach to data processing makes the
anisotropy-parameter estimation closer to an imaging processing task
whose only requirement is the ability to extract the best local slope
field from the data.}}
\end{comment}

%new 7+8
Interval parameters as well as effective parameters can be regarded as
data attributes obtained from local slopes. Unlike \tx domain,
processing data in \taup offers two alternatives to conventional
\citet{dix:68} inversion: \textit{stripping} and \textit{Fowler's
  equations} \citep{fowler:3028}. These relations can be considered as
the VTI extension of the \quotetation{straightedge determination of
  interval velocity} method \new{proposed} by
\cite{Claerbout.sep.14.13}. These three formulations for
interval-parameter inversion require \old{to} \new{an} estimate
\new{of} the \old{data} local-curvature field. \old{However, unlike
  slopes, we don't apply a dedicated method to compute the} \new{To
  estimate} curvature, \old{but} we \old{directly} perform \old{the}
\new{a} numerical differentiation of the slope estimates. This
procedure usually returns noisy and biased curvature values that
affect parameter estimation, especially for those parameters that
control long-spread/large-angle moveout.

%new 9
Fowler's equations offer a solution to this problem. In these
equations, the curvature dependence is absorbed by the zero-slope time
that we can estimate by applying the predictive painting algorithm
\citep{fomel:A25}. This approach does not involve any curvature
estimation and \old{it} represents a more robust way for obtaining the
zero-slope time mapping field \old{we need} \new{required} (1) to
automatically flatten or NMO correct the \taup CMP gathers (2) to
retrieve interval parameters using the curvature-independent Fowler's
equations.  This last approach to data processing makes the
anisotropy-parameter estimation closer to an imaging processing task.
\old{whose} \new{Its} only requirement is the ability to extract the
best local-slope field from the data.


