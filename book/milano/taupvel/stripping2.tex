\subsection{Stripping equations}
The first alternative to the Dix inversion is what we call
\textit{stripping equations} \citep{casasantafomel}. Starting from the integral
\req{taup_int} for \taup reflection moveout and employing the chain
rule (\req{B4}), we first deduce an expression for slope
$R_{\tau}$ (\req{B5}) and curvature $Q_{\tau}$ (\req{B6}) using
$\tau$-derivatives, that now depend on the interval parameters. Then,
solving for $\hat{V}_{N}$ and $\hat{V}_{H}$, we obtain the following
expressions:
\begin{eqnarray}
\hat{V}_{N}^{2}(\tau, p) &=&-\frac{1}{p}\frac{{16R_{{\tau }}^{3}}}{{\hat{N}\hat{D}}},
\label{eqn:strippingVN} \\
\hat{V}_{H}^{2}(\tau, p) &=&\frac{1}{{p^{2}}}\dfrac{{\hat{N}-4R}_{{\tau }}}{{\hat{N}}}%
,  \label{eqn:strippingVH}
\end{eqnarray}
and
\begin{equation}
\hat{\eta}(\tau, p)=\frac{1}{p}\frac{{\hat{N}(4{R}_{{\tau }}-\hat{D})}}{{32\tau {R}_{{%
\tau }}^{3}}},  \label{eqn:strippingETA}
\end{equation}
which provide an estimate for the interval parameters. In the above equations,
$\hat{N}={pQ}_{{\tau }}{+~3R}_{{\tau }}{-3p{R}_{{\tau }}^{2}}$
and 
$\hat{D}=p{Q}_{{\tau }}+3{R}_{{\tau }}+p{{R}_{{\tau }}^{2}},$
which corresponds to the interval values of the numerator $N$ and
denominator $D$ of the square root in equation \ref{eqn:tau0mapping}.
These relations are very similar to those previously derived for the
effective parameters (equations
\ref{eqn:vNmap}--\ref{eqn:etamap}). However, they require the $\tau$
derivative of \new{the} slope and curvature fields (Table
\ref{tbl:velocities}). This result agrees with the discussion above
about layer stripping in $\tau$-$p$. In this domain, layer stripping
\old{is just a matter of} \new{reduces to} computing traveltime differences
(\req{taup_layers}) at each horizontal slowness $p$. Therefore,
differentiating the effective slope $R$ and curvature $Q$ fields in
$\tau$ provides the necessary information to access the interval
parameters directly. This is the power of the \taup domain as opposed
to \tx, where the only practical path to interval parameters is
through Dix inversion that requires the knowledge of effective
parameters. The zero-slope time $\tau_0$ \old{function} is needed to map the
interval parameter estimates to the correct vertical time (Table
\ref{tbl:velocities}).

%Figure \ref{fig:dataSynthStrip} shows the interval NMO velocity $\hat{V}_{N}$
%(a) $,$ interval horizontal velocity $\hat{V}_{H}$ (b) and interval
%anellipticity $\hat{\eta}$ (c) oriented mapping obtained after applying the
%stripping equations to the synthetic data example shown in figure \ref%
%{fig:dataSynth} (a). Again we observe a good match between the exact and the
%recovered \ interval parameters profiles, although $\hat{V}_{H}$ and $\hat{%
%\eta}$ display a loss in resolution mainly due to instabilities of numerical
%differentiation of slopes $R(\tau ,p)$ and curvature $Q(\tau ,p)$ fields.
%Moreover these results confirm the equivalence of all the three inversion
%methods we propose, even though they are derived in different ways.
%Numerical instabilities affected the field data processing even more and the
%method produces a reliable estimate of the NMO\ velocity profile only
%(Figure \ref{fig:dataCongoInterval} (c)).

%\plot{int_Syn}{width=0.8\textwidth}{Oriented inversion for interval parameters %using stripping
%equations: normal moveout (a) and horizontal velocity (b) and anellipticity
%parameter $\eta $ (c). Blue dotted lines represent the exact values
%used for generating the synthetic dataset in figure \ref%
%{fig:dataSynth} (a).\label{fig:dataSynthStrip}}
