\section{Synthetic example \new{of effective-parameter estimation}}

\inputdir{synth}
\renewcommand{\figdir}{Fig}

\plot{dataSynth}{width=\textwidth}{(a) A synthetic \taup CMP gather composed of non
elliptical events modeled using equation \ref{eqn:taup_effective}.
Estimated local slopes (b) and curvatures (c).}

We first test our method on a synthetic example, where the exact
velocity model is known.  The example is introduced in Figure
\ref{fig:dataSynth}. The synthetic data were generated by applying
inverse \taup NMO with time-variable effective velocities. Both the
effective NMO $V_{N}$ and horizontal $V_{H}$ velocity increase
linearly with vertical time and include a sinusoidal change with time,
as described by the following relations
\begin{eqnarray*}
V_{N}^{\text{ \ }}\left( \tau _{0}\right) &=&2.0+0.03\sin (2\pi \frac{\tau
_{0}}{2})+0.08\tau _{0}, \\
V_{H}^{\text{ \ }}\left( \tau _{0}\right) &=&2.2-0.02\sin (2\pi \frac{\tau
_{0}}{3})+0.05\tau _{0}.
\end{eqnarray*}

\new{The CMP maximum offset-to-depth ratio is nearly 2.0 for large
value of the horizontal slope $p$. This should guarantee the necessary
data sensitivity for resolving high-order moveout parameters}
\citep{ilyabook2006}.  Figure \ref{fig:dataSynth}b shows local event
slopes $R$ measured from the data using the plane-wave destruction
(PWD) algorithm (see Appendix A). Plane-wave destruction predicts each
seismic trace from a neighboring one along local slopes. \new{As
explained in appendix A}, local slopes are extracted by minimizing the
prediction error in an iterative regularized least-squares
optimization. Shaping regularization controls the smoothness of the
estimated slope field \citep{FomelShaping}. If the seismic data are
particularly noisy, a more aggressive regularization can help in
getting a more consistent and stable estimate. For cleaner data, less
smoothing yields a better-resolved and detailed slope field.

Unlike slopes, we don't directly estimate the curvature field $Q$.  We
compute the curvature by simply differentiating the slope
estimate. Since slope $R=R(\tau ,p)$ depends on both the current ray
parameter $p$ and the time $\tau =\tau (p)$, which is again a function
of $p$, we compute the derivative of the slope field by a
straightforward application of the chain rule, as follows:
\begin{equation}
Q=\frac{\partial R}{\partial p}+R~\frac{\partial R}{\partial \tau},  \label{eq:curvatureNUM}
\end{equation}
The slope gradient components are easily obtained by numerical
differentiation.  Unfortunately, this procedure suffers from numerical
instability, because finite differences act like a high-pass filter
that enhances the high frequency noise, especially when we are dealing
with real data set with poor SNR (signal-to-noise
ratio). \old{Therefore,} A noisy or biased estimate of \new{the}
curvature field may affect the final result. Figure
\ref{fig:dataSynth}c shows the curvature field $Q$ for the synthetic
data, computed according to equation \ref{eq:curvatureNUM}.

\plot{dataNMO1}{width=\textwidth}{(b) Time mapping of each data
sample from \taup time to the zero-slope time $\tau _{0}$ according to
relation \ref{eqn:tau0mapping}. These time values predict correctly
reflection traveltime trajectory, the red lines in (a), which are then
warped until they are completely flattened (c)}

\rFg{dataNMO1}b represents the zero-slope traveltime $\tau _{0}$
mapped according to \new{the} oriented NMO formula in equation
\ref{eqn:tau0mapping}. These values predict correctly the reflection
trajectories (red lines in Figure \ref{fig:dataNMO1}a) which \old{are}
then \new{get} warped until they are completely flattened (Figure
\ref{fig:dataNMO1}c). Moreover, the oriented NMO\ does not introduce
stretch effects as the traditional NMO\ processing. This is because
the slope-based NMO\ applies a \new{locally} static shift to each data
sample as opposed to to the dynamic one of the conventional NMO\
correction.

%In
%this we observe the same idea of the shifted hyperbola approach
%proposed by \cite{bazelaire:143}.

\plot{mapE}{width=\textwidth}{Effective normal moveout velocity
(a), horizontal velocity (b) and anellipticity parameter $\eta $ (c)
computed as a data attribute through local estimate of slopes and
curvature. These values are here mapped to the appropriate zero-slope
$\tau _{0}$ time using oriented NMO described by the equation
\ref{eqn:tau0mapping}. \label{fig:dataSynthAttribute} }

In conventional NMO processing, one scans a number of velocities,
performs the corresponding moveout corrections, and picks the velocity
trend from velocity spectra maxima. In the oriented processing,
according to equations \ref{eqn:vNmap}--\ref{eqn:etamap}, anisotropy
parameters become data attributes rather than prerequisites for
imaging. Figure \ref{fig:mapE} shows the effective $V_{N}$, $V_{H}$
and $\eta$ values as data attributes mapped to the correct vertical
time $\tau _{0}$ position according to equation
\ref{eqn:tau0mapping}. These parameters have been obtained from the
data through an automatic estimation of the local-slope field. The
computational speed together with the automation are the main
advantages of oriented processing.

Even though this synthetic data is noise-free, the slope and curvature
estimates are not perfect. Nevertheless, Figure \ref{fig:mapE} shows a
nearly constant trend along $p$ direction of the recovered parameters
that confirms the reliability of our method. Despite the large offset-to-depth ratio, we observe that $V_{H}$
and $\eta$ are more sensitive to the slope estimate uncertainty, which agrees with
the observation of \cite{ilyabook2006} that high-order moveout
parameters are in general less constrained than the short-spread
normal moveout velocity ${V}_{N}$. The reduced data sensitivity to
$V_{H}$ and $\eta$ at short offsets can explain the errors in the
upper-right corner in panels (b) and (c) in Figure \ref{fig:mapE}. A
proper filtering procedure of the parameter maps may allow us to recover accurate parameter trends like
those in Figure \ref{fig:eff-Syn}. The panels in Figure
\ref{fig:eff-Syn} represent semblance-like spectra computed by mapping
each data sample to its parameter value at the zero-slope time $\tau
_{0}$.  The yellow lines indicate the exact effective-parameter
profiles used to generate the synthetic gather and confirm that our
estimations follow the exact trends. Compared to conventional
semblance spectra, these plots do not show the elongated ``bull's
eye'' patterns which grow with increasing time.  The improved
resolution comes from the slope estimation accuracy and relates to the
quality and complexity of the input data.
%
\plot{eff-Syn}{width=\textwidth}{Effective normal moveout velocity
(a), horizontal velocity (b) and anellipticity parameter $\eta $ (c)
semblance-like spectra. The yellow profiles indicate the exact
effective values used for generating the synthetic data in Figure
\ref{fig:dataSynth} (a). Compared to conventional semblance spectra,
these plots do not show the elongated bull-eye pattern which enlarges
with increasing time. }






%\rFgs{vel} and \rfn{den} show the example.
