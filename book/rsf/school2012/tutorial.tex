\author{Maurice the Aye-Aye}
%%%%%%%%%%%%%%%%%%%%%%%%%%%%
\title{Madagascar tutorial: Field data processing}

\lefthead{Maurice}
\righthead{Tutorial}
\footer{Madagascar Documentation}

\maketitle

\begin{abstract}
  In this tutorial, you will learn about multiple attenuation using parabolic Radon transform \cite[]{hampson86}. You will first go through an example that explains the process step by step. You will be asked to change some parameters and add missing few lines. In the next part of the tutorial, you will be asked to apply the same workflow to another CMP gather. The CMP gathers used in the tutorial are from the Canterbury data set \cite[]{cant2003}. By the end of this tutorial, you should have learned to:
  \begin{enumerate}
    \item apply NMO and inverse NMO for a CMP gather,
    \item apply forward and inverse parabloic Radon transform,
    \item design a mute function that preserves multiples in the Radon domain,
    \item subtract multiples from the data,
    \item create a semblance scan for a CMP gather.
  \end{enumerate}
\end{abstract}

\section{Prerequisites}

Completing this tutorial requires
\begin{itemize}
\item \textsc{Madagascar} software environment available from \\
\url{http://www.ahay.org}
\item \LaTeX\ environment with \texttt{SEGTeX} available from \\ 
\url{http://www.ahay.org/wiki/SEGTeX}
\end{itemize}
To do the assignment on your personal computer, you need to install
the required environments. An Internet connection is required for
access to the data repository.

The tutorial itself is available from the \textsc{Madagascar} repository
by running
\begin{verbatim}
svn co https://rsf.svn.sourceforge.net/svnroot/rsf/trunk/book/rsf/school2012
\end{verbatim}
\section{Introduction}

In this tutorial, you will be asked to run commands from the Unix
shell (identified by \texttt{bash\$}) and to edit files in a text
editor. Different editors are available in a typical Unix environment
(\texttt{vi}, \texttt{emacs}, \texttt{nedit}, etc.)

Your first assignment:
\begin{enumerate}
\item Open a Unix shell.
\item Change directory to the tutorial directory
\begin{verbatim}
bash$ cd $RSFSRC/book/rsf/school2012
\end{verbatim}
\item Open the \texttt{tutorial.tex} file in your favorite editor, for example by
running
\begin{verbatim}
bash$ nedit tutorial.tex & 
\end{verbatim}
\item Look at the first line in the file and change the author name from Maurice the Aye-Aye to your name (first things first). 
\end{enumerate}

\section{Demo}
\subsection{Part One}
\begin{enumerate}
\item Change directory to \texttt{demo} directory
\begin{verbatim}
bash$ cd demo
\end{verbatim}
\item Run
\begin{verbatim}
bash$ scons cmp.view
\end{verbatim}
in the Unix shell. A number of commands will appear in the shell followed by Figure~\ref{fig:cmp} appearing on your screen. 
\item To understand the commands, examine the script that generated them by opening the \texttt{SConstruct} file in a text editor. Notice that, instead of Shell commands, the script contains rules. 
\begin{itemize}
\item The first rule, \texttt{Fetch}, allows the script to download the input data file \texttt{cmp1.rsf} from the data server. 
\item Other rules have the form \texttt{Flow(target,source,command)} for generating data files or \texttt{Plot} and  \texttt{Result} for 
generating picture files. 
\item \texttt{Fetch}, \texttt{Flow}, \texttt{Plot}, and \texttt{Result} are defined in \textsc{Madagascar}'s \texttt{rsf.proj} package, which extends the functionality of \href{http://www.scons.org}{SCons}
.
\end{itemize}
\item To better understand how rules translate into commands, run 
\begin{verbatim}
bash$ scons -c cmp.rsf
\end{verbatim}
The \texttt{-c} flag tells scons to remove the \texttt{cmp.rsf} file and all its dependencies.
\item Next, run
\begin{verbatim}
bash$ scons -n cmp.rsf
\end{verbatim}
The \texttt{-n} flag tells scons not to run the command but simply to display it on the screen. Identify the lines in the \texttt{SConstruct} file that generate the output you see on the screen.
\item Run
\begin{verbatim}
bash$ scons cmp.rsf
\end{verbatim}
Examine the file \texttt{cmp.rsf} both by opening it in a text editor and by running
\begin{verbatim}
bash$ sfin cmp.rsf
\end{verbatim}

When you view cmp.rsf in the text editor, you see a history of all the programs used to create the file. The \texttt{sfin} program lists basic information about the file including data dimensions and extents of each axis.
\end{enumerate}

\subsection{Part Two}


\inputdir{demo}

\multiplot{4}{cmp,nmo,taup,nmo2}{width=.39\textwidth}{CMP gather from Canterbury dataset before applying NMO (a), after applying NMO (b), after Forward parabolic Radon transfrom (c), after applying inverse parabolic Radon transform (d). The forward and inverse parabolic Radon transforms are applied in sequence to examine the parameters of the process and to ensure that no events are lost during the process}

Figure~\ref{fig:cmp} shows a CMP gather from Canterbury data set Line 12. The multiple energy appears at time around 2.25 s. Figure~\ref{fig:nmo} shows the same gather after applying NMO correction with veloctiy equals to 1500 m/s. The multiple events starting at around 2.25 s and below are flatened while primary events , e.g at 2 s,  are over corrected. The difference in move-out between the primaries and multiples, hence, can be used in Radon domain to attenuate multiple energy. Figure~\ref{fig:taup} is generated by forward parabolic Radon transform while Figure~\ref{fig:nmo2} is generated by inverse parabloic Radon transform. The purpose was to make sure that forward and inverse transforms do not cause any data loss.


\multiplot{2}{taup,taupmult}{width=.39\textwidth}{Forward Radon transform of the gather (a). Mute is applied to preserve multiples (b); so that multiples can be transformed to time-offset domain for subtraction from the CMP gather.}

Figure~\ref{fig:taup} shows the Radon transform of the CMP gather in Figure~\ref{fig:cmp} while Figure~\ref{fig:taupmult} shows in the Radon domain the multiple energy only after mutting the primary energy. The protected multiples can be taken back to the time-offset domain and are subtracted from the data.

\multiplot{4}{cmp,signal2,vscan-cmp,vscan-signal2}{width=.39\textwidth}{CMP gather before multiple attenuation (a). CMP gather after multiple attenuation (b). Gather in (a) is used to generated semblance scan in (c). Gather in (b) is used to generate semblance scan in (d).}

CMP gather before multiple attenuation is shown in Figure~\ref{fig:cmp} and the coresponding semblance scan is shown in Figure~\ref{fig:vscan-cmp}. The CMP gather after multiple attenuation is shown in Figure~\ref{fig:signal2} and the coresponding semblance scan is shown in Figure~\ref{fig:vscan-signal2}. The semblance scans show how multiple energy is reduced for the CMP gather after multiple attenuation.

\begin{enumerate}
\item To examine the forward and inverse Radon transform, Run
\begin{verbatim}
bash$ scons taup-qc.view
\end{verbatim}


\item Edit the \texttt{SConstruct} file and find the line that says \texttt{CHANGE ME}, and modify the reference offset \texttt{x0} for \texttt{sfradon} program. To get more details about \texttt{sfradon} parameters, run 
\begin{verbatim}
bash$ sfradon
\end{verbatim}
Check your result by running
\begin{verbatim}
scons taup-qc.view
\end{verbatim}

\item Edit the \texttt{SConstruct} file and find the second \texttt{CHANGE ME}, and modify the starting time \texttt{t0} for \texttt{sfmutter}. To get more details about \texttt{sfmutter} parameters, run \texttt{sfmutter} in a Unix shell. Check your result by running
\begin{verbatim}
scons taup-mult.view
\end{verbatim}

\item Edit the \texttt{SConstruct} file and find the third \texttt{CHANGE ME}, and modify the parameter \texttt{v0} for \texttt{sfmutter}. Check your result by running
\begin{verbatim}
scons taup-mult.view
\end{verbatim}

\item Edit the \texttt{SConstruct} file and find the line that says ADD CODE to create \texttt{signal2.vpl}. To get more details about \texttt{sfgrey} parameters, run \texttt{sfgrey} in a Unix shell. Add your code and create the vpl file by running
\begin{verbatim}
scons signal2.vpl
\end{verbatim}

Display the figure by running
\begin{verbatim}
sfpen signal2.vpl
\end{verbatim}
\texttt{Hint}: the \texttt{SConstruct} file has similar code for creating the figure

\item Edit the \texttt{SConstruct} file and find the line that says ADD CODE to display \texttt{cmp.vpl} and \texttt{signal2.vpl}. Add your code and view the file by running
\begin{verbatim}
scons cmp-signal2.view
\end{verbatim}

\texttt{Hint}: the \texttt{SConstruct} file has a similar example

\item Edit the \texttt{SConstruct} file and find the line that says ADD CODE to display \texttt{vscan-cmp.vpl} and \texttt{vscan-signal2.vpl}.  Add your code and view the file by running
\begin{verbatim}
scons vcmp-signal2.view
\end{verbatim}

\texttt{Hint}: the \texttt{SConstruct} file has a similar example

\end{enumerate}

\section{Exercise}
\inputdir{ex}

In this part, your task is to apply the workflow explained above to a different CMP gather that requires different parameters. The same workflow should work here, but you need to observe that the CMP gather used for this exercise has shallow events. This means that, after applying NMO correction, amplitudes at far offstes of the shallow events get stretched. Therefore, an additional step is required for this CMP. We need to mute the distorted amplitudes. The mute is already applied in the \texttt{SConstruct}. 
\begin{enumerate}

\item Change directory to \texttt{ex} directory
\begin{verbatim}
bash$ cd ../ex
\end{verbatim}

\item Display the CMP gather after NMO with and without mute applied by running 
\begin{verbatim}
scons nmo1-nmo.view
\end{verbatim}

\item Your task is to add the necessary code to attenuate multiples for this CMP.
The same work flow used in the \texttt{SConstruct} file under \texttt{demo} directory should work here with only changes to 
\begin{itemize}
\item \texttt{x0}
\item \texttt{t0}
\item \texttt{v0}
\newline where it says \texttt{CHANGE ME} in the comments
\end{itemize}

\texttt{Hint}: You will need to copy part of \texttt{../demo/Sconstruct} to this \texttt{SConstruct} file.

\end{enumerate}

\section{Writing a report}

\begin{enumerate}
\item Change directory to the parent directory
\begin{verbatim}
bash$ cd ..
\end{verbatim}
This should be the directory that contains \texttt{tutorial.tex}.
\item Run
\begin{verbatim}
bash$ sftour scons lock
\end{verbatim}
The \texttt{sftour} command visits all subdirectories and runs \texttt{scons lock}, which copies result files to a different location so that they do not get modified until further notice.
\item You can also run
\begin{verbatim}
bash$ sftour scons -c
\end{verbatim}
to clean intermediate results.
\item Edit the file \texttt{paper.tex} to include your additional results. If you have not used \LaTeX\ before, no worries. It is a descriptive language. Study the file, and it should become evident by example how to include figures.
\item Run
\begin{verbatim}
bash$ scons tutorial.pdf
\end{verbatim}
and open \texttt{tutorial.pdf} with a PDF viewing program such as \textbf{Acrobat Reader}. 

\item If you have \LaTeX2HTML installed, you can also generate an HTML version of your paper by running
\begin{verbatim}
bash$ scons tutorial.html
\end{verbatim}
and opening \verb#tutorial_html/index.html# in a web browser.
\end{enumerate}

\lstset{language=python,numbers=left,numberstyle=\tiny,showstringspaces=false}
\lstinputlisting[frame=single]{demo/SConstruct}


%\lstset{language=python,numbers=left,numberstyle=\tiny,showstringspaces=false}
%\lstinputlisting[frame=single]{SConstruct}


\bibliographystyle{seg}
\bibliography{school}



