\title{Madagascar project: reproducibility challenge}

\author{Sergey Fomel}

\begin{abstract}
Challenge...
\end{abstract}

\lefthead{Fomel}
\righthead{Reproducibility challenge}

\address {
sergey.fomel@beg.utexas.edu \\
Jackson School of Geosciences \\
The University of Texas at Austin \\ 
University Station, Box X \\
Austin, TX 78713-8924 \\
USA}

\maketitle

\section{Introduction}
\inputdir{rain}

In 1997, the European Communities organized a Spatial Interpolation
Comparison. Many different organizations participated with the results
published in a special issue of the \emph{Journal of Geographic
Information and Decision Analysis} \cite[]{dubois} and a separate
report \cite[]{rain}.

\sideplot{elev}{width=\textwidth}{Digital elevation map of Switzerland.}

The comparison used a dataset from rainfall measurements in
Switzerland on the 8th of May 1986, the day of the Chernobyl disaster.
Figure~\ref{fig:elev} shows the data area: the Digital Elevation Model
of Switzerland with superimposed country's borders.  A total of 467
rainfall measurements were taken that day. A randomly selected subset
of 100 measurements was used as the input data the 1997 Spatial
Interpolation Comparison in order to interpolate other measurements
using different techniques and to compare the results with the known
data. Figure~\ref{fig:raindata} shows the spatial locations of the
selected data samples and the full dataset.

\plot{raindata}{width=\textwidth}{Left: locations of weather stations
  used as input data in the spatial interpolation contest.  Right: all
  weather stations locations.}

\section{Reproducibility challenge}

\subsection{Madagascar solution}

\subsection{Additional materials}

\bibliographystyle{seg}
\bibliography{rain}
