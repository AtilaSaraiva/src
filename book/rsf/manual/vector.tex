\section{Construction of vectors (vector.c)}\label{sec:vector.c}




\subsection{{det3}}
The determinant of a $3\times 3$ matrix.

\subsubsection*{Call}
\begin{verbatim}d = det3(m);\end{verbatim}

\subsubsection*{Definition}
\begin{verbatim}
double det3(double *m)
{
   ...
}
\end{verbatim}

\subsubsection*{Input parameters}
\begin{desclist}{\tt }{\quad}[\tt ]
   \setlength\itemsep{0pt}
   \item[m] a $3\times3$ matrix (\texttt{double}).  
\end{desclist}

\subsubsection*{Output}
\begin{desclist}{\tt }{\quad}[\tt ]
   \setlength\itemsep{0pt}
   \item[d] the determinant (\texttt{double}).
\end{desclist}



\subsection{{det2}}
The determinant of a $2\times2$ matrix.

\subsubsection*{Call}
\begin{verbatim}d = det2(m);\end{verbatim}

\subsubsection*{Definition}
\begin{verbatim}
double det2(double *m)
{
   ...
}
\end{verbatim}

\subsubsection*{Input parameters}
\begin{desclist}{\tt }{\quad}[\tt ]
   \setlength\itemsep{0pt}
   \item[m] a $2\times2$ matrix (\texttt{double}).  
\end{desclist}

\subsubsection*{Output}
\begin{desclist}{\tt }{\quad}[\tt ]
   \setlength\itemsep{0pt}
   \item[d] the determinant (\texttt{double}).
\end{desclist}



\subsection{{jac3d}}
Returns a 3D jacobian.

\subsubsection*{Call}
\begin{verbatim}r = jac3d(C, T, P, Q);\end{verbatim}

\subsubsection*{Definition}
\begin{verbatim}
double jac3d(pt3d *C, pt3d *T, pt3d *P, pt3d *Q)
/*< 3D jacobian >*/
{
   ...
}
\end{verbatim}

\subsubsection*{Input parameters}
\begin{desclist}{\tt }{\quad}[\tt ]
   \setlength\itemsep{0pt}
   \item[c] a complex number. Must be of type \texttt{sf\_double\_complex}.  
\end{desclist}

\subsubsection*{Output}
\begin{desclist}{\tt }{\quad}[\tt ]
   \setlength\itemsep{0pt}
   \item[c.r] real part of the complex number. It is of type \texttt{double}.
\end{desclist}



\subsection{{vec3d}}
Builds a 3D vector. The components of the vector returned are the difference of the respective components of the two input points (position vectors). The first input vector is the origin.

\subsubsection*{Call}
\begin{verbatim}V = vec3d(O, A);\end{verbatim}

\subsubsection*{Definition}
\begin{verbatim}
vc3d vec3d(pt3d* O, pt3d* A)
/*< build 3D vector >*/
{
   ...
}
\end{verbatim}

\subsubsection*{Input parameters}
\begin{desclist}{\tt }{\quad}[\tt ]
   \setlength\itemsep{0pt}
   \item[O] a 3D point, this serves as the origin (\texttt{pt3d}). 
   \item[A] a 3D point (\texttt{pt3d}).     
\end{desclist}

\subsubsection*{Output}
\begin{desclist}{\tt }{\quad}[\tt ]
   \setlength\itemsep{0pt}
   \item[V] the 3D vector. It is of type \texttt{vc3d}.
\end{desclist}



\subsection{{axa3d}}
Builds a 3D unit vector. The components of the vector returned are zero except for the one indicated in the input argument \texttt{n}, which is equal to one. If \texttt{n}=1 the z axis is 1, if \texttt{n}=2 the x axis is 1 and if \texttt{n}=3 the y axis is 1.

\subsubsection*{Call}
\begin{verbatim}V = axa3d (n);\end{verbatim}

\subsubsection*{Definition}
\begin{verbatim}
vc3d axa3d( int n)
/*< build 3D unit vector >*/
{
   ...
}
\end{verbatim}

\subsubsection*{Input parameters}
\begin{desclist}{\tt }{\quad}[\tt ]
   \setlength\itemsep{0pt}
   \item[n] a number which indicates which axis is to be set equal to 1 (\texttt{int}).     
\end{desclist}

\subsubsection*{Output}
\begin{desclist}{\tt }{\quad}[\tt ]
   \setlength\itemsep{0pt}
   \item[V] the 3D unit vector. It is of type \texttt{vc3d}.
\end{desclist}




\subsection{{scp3d}}
Returns the scalar product of the two 3D vectors.

\subsubsection*{Call}
\begin{verbatim}p = scp3d(U, V);\end{verbatim}

\subsubsection*{Definition}
\begin{verbatim}
double scp3d(vc3d* U, vc3d* V)
/*< scalar product of 3D vectors >*/
{
   ...
}
\end{verbatim}

\subsubsection*{Input parameters}
\begin{desclist}{\tt }{\quad}[\tt ]
   \setlength\itemsep{0pt}
   \item[U] a 3D vector (\texttt{vc3d}). 
   \item[V] a 3D vector (\texttt{vc3d}).     
\end{desclist}

\subsubsection*{Output}
\begin{desclist}{\tt }{\quad}[\tt ]
   \setlength\itemsep{0pt}
   \item[V->dx*V->dx + V->dy*V->dy + V->dz*V->dz] the 3D unit vector.
\end{desclist}




\subsection{{vcp3d}}
Returns the vector product of the two input vectors.

\subsubsection*{Call}
\begin{verbatim}W = vcp3d(U, V);\end{verbatim}

\subsubsection*{Definition}
\begin{verbatim}
vc3d vcp3d(vc3d* U, vc3d* V)
/*< vector product of 3D vectors >*/
{
   ...
}
\end{verbatim}

\subsubsection*{Input parameters}
\begin{desclist}{\tt }{\quad}[\tt ]
   \setlength\itemsep{0pt}
   \item[U] a 3D vector (\texttt{vc3d}). 
   \item[V] a 3D vector (\texttt{vc3d}). 
\end{desclist}

\subsubsection*{Output}
\begin{desclist}{\tt }{\quad}[\tt ]
   \setlength\itemsep{0pt}
   \item[W] the 3D unit vector. It is of type \texttt{vc3d}.
\end{desclist}




\subsection{{len3d}}
Returns the length of a 3D vector.

\subsubsection*{Call}
\begin{verbatim}l = len3d(V);\end{verbatim}

\subsubsection*{Definition}
\begin{verbatim}
double len3d(vc3d* V)
/*< 3D vector length >*/
{
   ...
}
\end{verbatim}

\subsubsection*{Input parameters}
\begin{desclist}{\tt }{\quad}[\tt ]
   \setlength\itemsep{0pt}
   \item[V] a 3D vector (\texttt{vc3d}).     
\end{desclist}

\subsubsection*{Output}
\begin{desclist}{\tt }{\quad}[\tt ]
   \setlength\itemsep{0pt}
   \item[l] the length of the 3D vector. It is of type \texttt{double}.
\end{desclist}




\subsection{{nor3d}}
Normalizes a 3D vector. The components of the 3D input vector are divided by its length. 

\subsubsection*{Call}
\begin{verbatim}W = nor3d(V);\end{verbatim}

\subsubsection*{Definition}
\begin{verbatim}
vc3d nor3d(vc3d* V)
/*< normalize 3D vector >*/
{
   ...
}
\end{verbatim}

\subsubsection*{Input parameters}
\begin{desclist}{\tt }{\quad}[\tt ]
   \setlength\itemsep{0pt}
   \item[V] the input 3D vector. It is of type \texttt{vc3d}.
\end{desclist}

\subsubsection*{Output}
\begin{desclist}{\tt }{\quad}[\tt ]
   \setlength\itemsep{0pt}
   \item[W] the normalized 3D vector. It is of type \texttt{vc3d}.
\end{desclist}




\subsection{{ang3d}}
Returns the angle between the input 3D vectors. 

\subsubsection*{Call}
\begin{verbatim}a = ang3d(U, V);\end{verbatim}

\subsubsection*{Definition}
\begin{verbatim}
double ang3d(vc3d* U, vc3d* V)
/*< angle between 3D vectors >*/
{
   ...   
}
\end{verbatim}

\subsubsection*{Input parameters}
\begin{desclist}{\tt }{\quad}[\tt ]
   \setlength\itemsep{0pt}
   \item[U] the input 3D vector. It is of type \texttt{vc3d}. 
   \item[V] the input 3D vector. It is of type \texttt{vc3d}.
\end{desclist}

\subsubsection*{Output}
\begin{desclist}{\tt }{\quad}[\tt ]
   \setlength\itemsep{0pt}
   \item[a] the angle between the two input 3D vectors. It is of type \texttt{double}.
\end{desclist}





\subsection{{tip3d}}
Returns the tip of a 3D vector. The components of the vector returned are the sum of the respective components of the two input points (position vectors). Unlike the \texttt{sf\_vc3d} where the first input vector was the origin, in \texttt{sf\_tip3d} the origin is zero. This means that the vector returned is a position vector or simply a point, which is of type \texttt{pt3d}.

\subsubsection*{Call}
\begin{verbatim}A = tip3d(O, V);\end{verbatim}

\subsubsection*{Definition}
\begin{verbatim}
pt3d tip3d(pt3d* O, vc3d* V)
/*< tip of a 3D vector >*/
{
   ...    
}
\end{verbatim}

\subsubsection*{Input parameters}
\begin{desclist}{\tt }{\quad}[\tt ]
   \setlength\itemsep{0pt}
   \item[O] a 3D point (\texttt{pt3d}). 
   \item[V] a 3D point (\texttt{pt3d}).     
\end{desclist}

\subsubsection*{Output}
\begin{desclist}{\tt }{\quad}[\tt ]
   \setlength\itemsep{0pt}
   \item[V] the 3D vector. It is of type \texttt{vc3d}.
\end{desclist}




\subsection{{scl3d}}
Scales a 3D vector, that is, it multiplies it by a scalar. The components of the vector returned are the product of the components of the input vector and the input scalar.

\subsubsection*{Call}
\begin{verbatim}W = scl3d(V, s);\end{verbatim}

\subsubsection*{Definition}
\begin{verbatim}
vc3d scl3d(vc3d* V, float s)
/*< scale a 3D vector >*/
{
   ...
}
\end{verbatim}

\subsubsection*{Input parameters}
\begin{desclist}{\tt }{\quad}[\tt ]
   \setlength\itemsep{0pt}
   \item[V] a 3D point (\texttt{pt3d}). 
   \item[s] a scalar which is to be multiplied by every component of the input vector (\texttt{float}).   
\end{desclist}

\subsubsection*{Output}
\begin{desclist}{\tt }{\quad}[\tt ]
   \setlength\itemsep{0pt}
   \item[W] the scaled 3D vector. It is of type \texttt{vc3d}.
\end{desclist}



