\section{Tridiagonal matrix solver (tridiagonal.c)}




\subsection{{sf\_tridiagonal\_init}}
Initializes the object of the abstract data of type \texttt{sf\_tris}, which contains a matrix of size \texttt{n} with separate variables for the diagonal and off-diagonal entries and also for the solution to the matrix equation which it will be used to solve.

\subsubsection*{Call}
\begin{verbatim}slv = sf_tridiagonal_init (n);\end{verbatim}

\subsubsection*{Definition}
\begin{verbatim}
sf_tris sf_tridiagonal_init (int n /* matrix size */)
/*< initialize >*/
{
   ...
}
\end{verbatim}

\subsubsection*{Input parameters}
\begin{desclist}{\tt }{\quad}[\tt ]
   \setlength\itemsep{0pt}
   \item[n] size of the matrix (\texttt{int}).
\end{desclist}

\subsubsection*{Output}
\begin{desclist}{\tt }{\quad}[\tt ]
   \setlength\itemsep{0pt}
   \item[slv] the tridiagonal solver. It is of type \texttt{sf\_tris}.
\end{desclist}




\subsection{{sf\_tridiagonal\_define}}\label{sec:sf_tridiagonal_define}
Fills in the diagonal and off-diagonal entries in the tridiagonal solver based on the input entries \texttt{diag} and \texttt{offd}.

\subsubsection*{Call}
\begin{verbatim}sf_tridiagonal_define (slv, diag, offd);\end{verbatim}

\subsubsection*{Definition}
\begin{verbatim}
void sf_tridiagonal_define (sf_tris slv /* solver object */, 
                            float* diag /* diagonal */, 
                            float* offd /* off-diagonal */)
/*< fill the matrix >*/
{
   ...
}
\end{verbatim}

\subsubsection*{Input parameters}
\begin{desclist}{\tt }{\quad}[\tt offd]
   \setlength\itemsep{0pt}
   \item[slv]  the solver object (\texttt{sf\_tris}). 
   \item[diag] the diagonal (\texttt{float*}).
   \item[offd] the off-diagonal (\texttt{float*}).
\end{desclist}




\subsection{{sf\_tridiagonal\_const\_define}}
Fills in the diagonal and off diagonal entries in the tridiagonal solver based on the input entries \texttt{diag} and \texttt{offd}. It works like \hyperref[sec:sf_tridiagonal_define]{\texttt{sf\_tridiagonal\_define}} but for the special case where the matrix is Toeplitz.

\subsubsection*{Call}
\begin{verbatim}sf_tridiagonal_const_define (slv, diag, offd, damp);\end{verbatim}

\subsubsection*{Definition}
\begin{verbatim}
void sf_tridiagonal_const_define (sf_tris slv /* solver object */, 
                                  float diag  /* diagonal */, 
                                  float offd  /* off-diagonal */,
                                  bool damp   /* damping */)
/*< fill the matrix for the Toeplitz case >*/
{
   ...
}
\end{verbatim}

\subsubsection*{Input parameters}
\begin{desclist}{\tt }{\quad}[\tt pffd]
   \setlength\itemsep{0pt}
   \item[slv]  the solver object. Must be of type \texttt{sf\_tris}. 
   \item[diag] the diagonal (\texttt{float*}).
   \item[offd] the off-diagonal (\texttt{float*}).
   \item[damp] damping (\texttt{bool}).
\end{desclist}




\subsection{{sf\_tridiagonal\_solve}}\label{sec:sf_tridiagonal_solve}
Solves the matrix equation (like $La=b$, where $b$ is the input for the solve function, $a$ is the output and $L$ is the matrix defined by \hyperref[sec:sf_tridiagonal_define]{\texttt{sf\_tridiagonal\_define}}) and stores the solution in the space allocated by the variable \texttt{x} in the \texttt{slv} object.


\subsubsection*{Call}
\begin{verbatim}sf_tridiagonal_solve (sf_tris slv, b);\end{verbatim}

\subsubsection*{Definition}
\begin{verbatim}
void sf_tridiagonal_solve (sf_tris slv /* solver object */, 
                           float* b /* in - right-hand side, out - solution */)
/*< invert the matrix >*/
{
   ...
}
\end{verbatim}

\subsubsection*{Input parameters}
\begin{desclist}{\tt }{\quad}[\tt slv]
   \setlength\itemsep{0pt}
   \item[slv] the solver object. Must be of type \texttt{sf\_tris}. 
   \item[b]   right hand side of the matrix equation $La=b$ (\texttt{float*}).
\end{desclist}

\subsubsection*{Definition}
\begin{verbatim}
void sf_tridiagonal_solve (sf_tris slv /* solver object */, 
                           float* b /* in - right-hand side, out - solution */)
/*< invert the matrix >*/
{
    int k;
    ...
}
\end{verbatim}




\subsection{{sf\_tridiagonal\_close}}
This function frees the allocated space for the \texttt{slv} object.

\subsubsection*{Call}
\begin{verbatim}sf_tridiagonal_close (slv);\end{verbatim}

\subsubsection*{Definition}
\begin{verbatim}
void sf_tridiagonal_close (sf_tris slv)
/*< free allocated storage >*/
{
   ...
}
\end{verbatim}

\subsubsection*{Input parameters}
\begin{desclist}{\tt }{\quad}[\tt ]
   \setlength\itemsep{0pt}
   \item[slv] the solver object. Must be of type \texttt{sf\_tris}.
\end{desclist}





