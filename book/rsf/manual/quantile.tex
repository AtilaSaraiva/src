\section{Computing quantiles by Hoare's algorithm (quantile.c)}




\subsection{{sf\_quantile}}
Returns the quantile - which is specified in the input -  for the input array.

\subsubsection*{Call}
\begin{verbatim}k = sf_quantile(q, n, a);\end{verbatim}

\subsubsection*{Definition}
\begin{verbatim}
float sf_quantile(int q    /* quantile */, 
                  int n    /* array length */, 
                  float* a /* array [n] */) 
/*< find quantile (caution: a is changed) >*/ 
{
   ...
}
\end{verbatim}

\subsubsection*{Input parameters}
\begin{desclist}{\tt }{\quad}[\tt ]
   \setlength\itemsep{0pt}
   \item[q] the required quantile (\texttt{int}). 
   \item[n] length of the input array (\texttt{int}). 
   \item[a] the input array for which the quantile is required (\texttt{float*}).
\end{desclist}

\subsubsection*{Output}
\begin{desclist}{\tt }{\quad}[\tt ]
   \setlength\itemsep{0pt}
   \item[*k] the quantile. It is of type \texttt{float}.
\end{desclist}




