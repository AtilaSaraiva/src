\section{B-spline interpolation (spline.c)}




\subsection{{sf\_spline\_init}}
Initializes and defines a banded matrix for spline interpolation.


\subsubsection*{Call}
\begin{verbatim}slv = sf_spline_init (nw, nd);\end{verbatim}


\subsubsection*{Definition}
\begin{verbatim}
sf_bands sf_spline_init (int nw /* interpolator length */, 
                         int nd /* data length */)
/*< initialize a banded matrix >*/
{
   ...
}
\end{verbatim}


\subsubsection*{Input parameters}
\begin{desclist}{\tt }{\quad}[\tt nw]
   \setlength\itemsep{0pt}
   \item[nw] length of the interpolator (\texttt{int}). 
   \item[nd] length of the data (\texttt{int}).  
\end{desclist}

\subsubsection*{Output}
\begin{desclist}{\tt }{\quad}[\tt ]
   \setlength\itemsep{0pt}
   \item[slv] an object of type \texttt{sf\_band}. It is of type \texttt{sf\_band}.
\end{desclist}



\subsection{{sf\_spline4\_init}}
Initializes and defines a tridiagonal matrix for cubic spline interpolation.

\subsubsection*{Call}
\begin{verbatim}slv = sf_spline4_init(nd);\end{verbatim}

\subsubsection*{Definition}
\begin{verbatim}
sf_tris sf_spline4_init (int nd /* data length */)
/*< initialize a tridiagonal matrix for cubic splines >*/
{
   ...
}
\end{verbatim}

\subsubsection*{Input parameters}
\begin{desclist}{\tt }{\quad}[\tt ]
   \setlength\itemsep{0pt}
   \item[nd] length of the data (\texttt{int}).  
\end{desclist}

\subsubsection*{Output}
\begin{desclist}{\tt }{\quad}[\tt ]
   \setlength\itemsep{0pt}
   \item[slv] an object of type \texttt{sf\_tri}. It is of type \texttt{sf\_tri}.
\end{desclist}




\subsection{{sf\_spline4\_post}}
Performs the cubic spline post filtering.

\subsubsection*{Call}
\begin{verbatim}sf_spline4_post (n, n1, n2, inp, out);\end{verbatim}

\subsubsection*{Definition}
\begin{verbatim}
void sf_spline4_post (int n            /* total trace length */, 
                      int n1           /* start point */, 
                      int n2           /* end point */, 
                      const float* inp /* spline coefficients */, 
                      float* out       /* function values */)
/*< cubic spline post-filtering >*/
{
   ...
}
\end{verbatim}

\subsubsection*{Input parameters}
\begin{desclist}{\tt }{\quad}[\tt out]
   \setlength\itemsep{0pt}
   \item[n]   total length of the trace (\texttt{int}). 
   \item[n1]  start point (\texttt{int}).  
   \item[n2]  end point (\texttt{int}). 
   \item[inp] spline coefficients (\texttt{const float*}).
   \item[out] function values (\texttt{float*}).
\end{desclist}




\subsection{{sf\_spline\_post}}
Performs the post filtering to convert spline coefficients to model.


\subsubsection*{Call}
\begin{verbatim}sf_spline_post(nw, o, d, n, modl, datr);\end{verbatim}

\subsubsection*{Definition}
\begin{verbatim}
void sf_spline_post (int nw, int o, int d, int n, 
                     const float *modl, float *datr)
/*< post-filtering to convert spline coefficients to model >*/
{
   ...
}
\end{verbatim}

\subsubsection*{Input parameters}
\begin{desclist}{\tt }{\quad}[\tt datr]
   \setlength\itemsep{0pt}
   \item[nw]   length of the interpolator (\texttt{int}). 
   \item[o]    start point (\texttt{int}).  
   \item[d]    step size (\texttt{int}). 
   \item[n]    total length of the trace (\texttt{int}). 
   \item[modl] spline coefficients, which have to be converted to model coefficients.  Must be of type \texttt{const float*}.
   \item[datr] model, it is the output (\texttt{float*}).
\end{desclist}




\subsection{{sf\_spline2}}
Performs pre-filtering for spline interpolation for 2D data.

\subsubsection*{call}
\begin{verbatim}sf_spline2 (slv1, slv2, n1, n2, dat, tmp);\end{verbatim}

\subsubsection*{Definition}
\begin{verbatim}
void sf_spline2 (sf_bands slv1, sf_bands slv2, 
                 int n1, int n2, float** dat, float* tmp)
/*< 2-D spline pre-filtering >*/
{
   ...
}
\end{verbatim}

\subsubsection*{Input parameters}
\begin{desclist}{\tt }{\quad}[\tt slv2]
   \setlength\itemsep{0pt}
   \item[slv1] first banded matrix. Must be of type \texttt{sf\_band}. 
   \item[slv2] second banded matrix. Must be of type \texttt{sf\_band}. 
   \item[n1]   data length on the first axis (\texttt{int}).  
   \item[n2]	  data length on the second axis (\texttt{int}). 
   \item[dat]  2D data. Must be of type \texttt{const float*}*.
   \item[tmp]  temporary arrays for calculation (\texttt{float*}).
\end{desclist}





