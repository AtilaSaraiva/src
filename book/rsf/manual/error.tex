\section{Handling warning and error messages (error.c)}




\subsection{{sf\_error}}\label{sec:sf_error}
Outputs an error message to \texttt{stderr}, which is usually the screen. It uses \texttt{sf\_getprog} to get the name of the program which is causing the error and print it on the screen.  Uses \texttt{vfprintf}, which can take a variable number of arguments initialized by \texttt{va\_list}. This gives the user flexibility in choosing the number of arguments.

If there is a '\texttt{:}' at the end of format, information about the system errors is printed, this is done by using \texttt{strerror} to interpret the last error number \texttt{errno} in the system. Also, if there is a '\texttt{;}' at the end of a format the command prompt will not go to the next line.

\subsubsection*{Call}
\begin{verbatim}sf_error (format, ...);\end{verbatim}

\subsubsection*{Definition}
\begin{verbatim}
void sf_error( const char *format, ... )
/*< Outputs an error message to stderr and terminates the program. 
    ---
    Format and variable arguments follow printf convention. Additionally, a ':' at
    the end of format adds system information for system errors. >*/
{
   ...
}
\end{verbatim}

\subsubsection*{Input parameters}
\begin{desclist}{\tt }{\quad}[\tt format]
   \setlength\itemsep{0pt}
   \item[format] a string of \texttt{type \texttt{const char*}} containing the format specifiers for the arguments to be input from the next commands.
   \item[...]    variable number of arguments, which are to replace the format specifiers in the format.    
\end{desclist}

\subsubsection*{Output}
An error message output to \texttt{sterr} (usually printed on screen).




\subsection{{sf\_warning}}\label{sec:sf_warning}
Outputs a warning message to \texttt{stderr} which is usually the screen. It uses \texttt{sf\_getprog} to get the name of the program which is causing the error and print it on the screen.  Uses \texttt{vfprintf}, which can take a variable number of arguments initialized by \texttt{va\_list}. This gives the user flexibility in choosing the number of arguments.
If there is a '\texttt{:}' at the end of format, information about the system errors is printed, this is done by using \texttt{strerror} to interpret the last error number \texttt{errno} in the system. Also, if there is a '\texttt{;}' at the end of a format the command prompt will not go to the next line.

\subsubsection*{Call}
\begin{verbatim}sf_warning (format, ... );\end{verbatim}

\subsubsection*{Definition}
\begin{verbatim}
void sf_warning( const char *format, ... )
/*< Outputs a warning message to stderr. 
---
Format and variable arguments follow printf convention. Additionally, a ':' at
the end of format adds system information for system errors. >*/
{
   ...
}
\end{verbatim}

\subsubsection*{Input parameters}
\begin{desclist}{\tt }{\quad}[\tt format]
   \setlength\itemsep{0pt}
   \item[format] a string of \texttt{type \texttt{const char*}} containing the format specifiers for the arguments to be input from the next commands. 
   \item[...]    variable number of arguments, which are to replace the format specifiers in the format.     
\end{desclist}

\subsubsection*{Output}
A warning message output to \texttt{sterr} ( usually printed on screen).



