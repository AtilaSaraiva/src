\section{Frequency-domain filtering in 2-D (freqfilt.c)}
\index{filtering!frequency domain}




\subsection{{sf\_freqfilt2\_init}}\label{sec:sf_freqfilt2_init}
Initializes the required variables and allocates the required space for frequency filtering for 2D data.


\subsubsection*{Call}
\begin{verbatim}sf_freqfilt2_init (n1, n2, nw1);\end{verbatim}

\subsubsection*{Definition}
\begin{verbatim}
void sf_freqfilt2_init(int n1, int n2 /* data dimensions */, 
                       int nw1        /* number of frequencies */)
/*< initialize >*/
{
   ...
}
\end{verbatim}

\subsubsection*{Input parameters}
\begin{desclist}{\tt }{\quad}[\tt nw1]
   \setlength\itemsep{0pt}
   \item[n1]  number of time samples in first dimension (\texttt{int}).  
   \item[n2]  number of time samples in second dimension (\texttt{int}).  
   \item[nw1] number of frequencies (\texttt{int}).  
\end{desclist}




\subsection{{sf\_freqfilt2\_set}}
Initializes a zero phase filter.


\subsubsection*{Call}
\begin{verbatim}sf_freqfilt2_set (filt);\end{verbatim}

\subsubsection*{Input parameters}
\begin{desclist}{\tt }{\quad}[\tt ]
   \setlength\itemsep{0pt}
   \item[filt] the frequency filter (\texttt{float**}).  
\end{desclist}

\subsubsection*{Definition}
\begin{verbatim}
void sf_freqfilt2_set(float **filt)
/*< set the filter >*/
{
   ...
}
\end{verbatim}




\subsection{{sf\_freqfilt2\_close}}
Frees the space allocated by \hyperref[sec:sf_freqfilt2_init]{\texttt{sf\_freqfilt2\_init}}.

\subsubsection*{Call}
\begin{verbatim}sf_freqfilt2_close();\end{verbatim}

\subsubsection*{Definition}
\begin{verbatim}
void sf_freqfilt2_close(void) 
/*< free allocated storage >*/
{
   ...
}
\end{verbatim}




\subsection{{sf\_freqfilt2\_spec}}
This function 2D spectrum of the input data.

\subsubsection*{Call}
\begin{verbatim}sf_freqfilt2_spec (x, y);\end{verbatim}

\subsubsection*{Definition}
\begin{verbatim}
void sf_freqfilt2_spec (const float* x /* input */, float** y /* spectrum */) 
/*< compute 2-D spectrum >*/
{
   ...
}
\end{verbatim}

\subsubsection*{Input parameters}
\begin{desclist}{\tt }{\quad}[\tt ]
   \setlength\itemsep{0pt}
   \item[x] the data (\texttt{const float*}).  
   \item[y] the data (\texttt{float**}).  
\end{desclist}




\subsection{{sf\_freqfilt2\_lop}}
Applies the frequency filtering to either \texttt{y} or \texttt{x}, depending on whether \texttt{adj} is true of false and then applies the result to \texttt{x} or \texttt{y} as a linear operator.


\subsubsection*{Call}
\begin{verbatim}sf_freqfilt2_lop (adj, add, nx, ny, x, y);\end{verbatim}

\subsubsection*{Definition}
\begin{verbatim}
void sf_freqfilt2_lop (bool adj, bool add, int nx, int ny, float* x, float* y) 
/*< linear filtering operator >*/
{
   ...
}
\end{verbatim}

\subsubsection*{Input parameters}
\begin{desclist}{\tt }{\quad}[\tt add]
   \setlength\itemsep{0pt}
   \item[adj] a parameter to determine whether frequency filter applied to \texttt{y} or \texttt{x} (\texttt{bool}).
   \item[add] a parameter to determine whether the input needs to be zeroed (\texttt{bool}).
   \item[nx]  size of \texttt{x} (\texttt{int}).
   \item[ny]  size of \texttt{y} (\texttt{int}).
   \item[x]   data or operator, depending on whether \texttt{adj} is true or false (\texttt{float*}).
   \item[y]   data or operator, depending on whether \texttt{adj} is true or false (\texttt{float*}).
\end{desclist}


