\section{Dot product test for linear operators (dottest.c)}
Performs the dot product test (see p.~\pageref{sec:dot-product test}), to check whether the adjoint of the operator is coded incorrectly. Coding is incorrect if any of 
\begin{gather*}
   \langle Lm_1,d_2\rangle = \langle m_1, L^*d_2\rangle\quad\textrm{or}\quad
   \langle d_1 + Lm_1,d_2\rangle = \langle m_1, m_2+L^*d_2\rangle
\end{gather*} 
does not hold (within machine precision). $m_1$ and $d_2$ are random vectors.




\subsection{{sf\_dot\_test}}\label{sec:sf_dot_test}
\texttt{dot1[0]} must equal \texttt{dot1[1]} and \texttt{dot2[0]} must equal \texttt{dot2[1]} for the test to pass.

\subsubsection*{Call}
\begin{verbatim}sf_dot_test (oper, nm, nd, dot1, dot2);\end{verbatim}

\subsubsection*{Definition}
\begin{verbatim}
void sf_dot_test(sf_operator oper /* linear operator */, 
                 int nm           /* model size */, 
                 int nd           /* data size */, 
                 float* dot1      /* first output */, 
                 float* dot2      /* second output */) 
{
   ...
}
\end{verbatim}

\subsubsection*{Input parameters}
\begin{desclist}{\tt }{\quad}[\tt oper]
   \setlength\itemsep{0pt}
   \item[oper] the linear operator, whose adjoint is to be tested (\texttt{sf\_operator}). 
   \item[nm]   size of the models (\texttt{int}). 
   \item[nd]   size of the data (\texttt{int}). 
   \item[dot1] first output dot product (\texttt{float*}).
   \item[dot2] second output dot product (\texttt{float*}).
\end{desclist}



