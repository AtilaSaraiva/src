\section{Complex number operations (komplex.c)}




\subsection{{creal}}
Returns the real part of the complex number.

\subsubsection*{Call}
\begin{verbatim}r = sf_creal(c);\end{verbatim}

\subsubsection*{Definition}
\begin{verbatim}
double sf_creal(sf_double_complex c)
/*< real part >*/
{
   ...
}
\end{verbatim}

\subsubsection*{Input parameters}
\begin{desclist}{\tt }{\quad}[\tt ]
   \setlength\itemsep{0pt}
   \item[c] a complex number. Must be of type \texttt{sf\_double\_complex}.  
\end{desclist}

\subsubsection*{Output}
\begin{desclist}{\tt }{\quad}[\tt ]
   \setlength\itemsep{0pt}
   \item[r] real part of the complex number. It is of type \texttt{double}.
\end{desclist}




\subsection{{cimag}}
Returns the imaginary part of the complex number.

\subsubsection*{Call}
\begin{verbatim}im = sf_cimag(c);\end{verbatim}

\subsubsection*{Definition}
\begin{verbatim}
double sf_cimag(sf_double_complex c)
/*< imaginary part >*/
{
   ...
}
\end{verbatim}

\subsubsection*{Input parameters}
\begin{desclist}{\tt }{\quad}[\tt ]
   \setlength\itemsep{0pt}
   \item[c] a complex number. Must be of type \texttt{sf\_double\_complex}.  
\end{desclist}

\subsubsection*{Output}
\begin{desclist}{\tt }{\quad}[\tt ]
   \setlength\itemsep{0pt}
   \item[im] imaginary part of the complex number (\texttt{double}).
\end{desclist}




\subsection{{dcneg}}
Returns the negative complex number.

\subsubsection*{Call}
\begin{verbatim}n = sf_dcneg(a);\end{verbatim}

\subsubsection*{Definition}
\begin{verbatim}
sf_double_complex sf_dcneg(sf_double_complex a)
/*< unary minus >*/
{
   ...
}
\end{verbatim}

\subsubsection*{Input parameters}
\begin{desclist}{\tt }{\quad}[\tt ]
   \setlength\itemsep{0pt}
   \item[a] a complex number. Must be of type \texttt{sf\_double\_complex}.  
\end{desclist}

\subsubsection*{Output}
\begin{desclist}{\tt }{\quad}[\tt ]
   \setlength\itemsep{0pt}
   \item[n] negative of the complex number. It is of type \texttt{sf\_double\_complex}.
\end{desclist}




\subsection{{dcadd}}
Adds two complex numbers.

\subsubsection*{Call}
\begin{verbatim}c = sf_dcadd(a, b);\end{verbatim}

\subsubsection*{Definition}
\begin{verbatim}
sf_double_complex sf_dcadd(sf_double_complex a, sf_double_complex b)
/*< complex addition >*/
{
   ...
}
\end{verbatim}

\subsubsection*{Input parameters}
\begin{desclist}{\tt }{\quad}[\tt ]
   \setlength\itemsep{0pt}
   \item[a] a complex number. Must be of type \texttt{sf\_double\_complex}. 
   \item[b] a complex number. Must be of type \texttt{sf\_double\_complex}.  
\end{desclist}

\subsubsection*{Output}
\begin{desclist}{\tt }{\quad}[\tt ]
   \setlength\itemsep{0pt}
   \item[c] $a+b$. It is of type \texttt{sf\_double\_complex}.
\end{desclist}




\subsection{{dcsub}}
Subtracts two complex numbers.

\subsubsection*{Call}
\begin{verbatim}c = sf_dcsub(a, b);\end{verbatim}

\subsubsection*{Definition}
\begin{verbatim}
sf_double_complex sf_dcsub(sf_double_complex a, sf_double_complex b)
/*< complex subtraction >*/
{
   ...
}
\end{verbatim}

\subsubsection*{Input parameters}
\begin{desclist}{\tt }{\quad}[\tt ]
   \setlength\itemsep{0pt}
   \item[a] a complex number. Must be of type \texttt{sf\_double\_complex}. 
   \item[b] a complex number. Must be of type \texttt{sf\_double\_complex}.  
\end{desclist}

\subsubsection*{Output}
\begin{desclist}{\tt }{\quad}[\tt ]
   \setlength\itemsep{0pt}
   \item[c] $a-b$. It is of type \texttt{sf\_double\_complex}.
\end{desclist}




\subsection{{dcmul}}
Multiplies two complex number.

\subsubsection*{Call}
\begin{verbatim}c = sf_dcmul(a, b);\end{verbatim}

\subsubsection*{Definition}
\begin{verbatim}
sf_double_complex sf_dcmul(sf_double_complex a, sf_double_complex b)
/*< complex multiplication >*/
{
   ...
}
\end{verbatim}

\subsubsection*{Input parameters}
\begin{desclist}{\tt }{\quad}[\tt ]
   \setlength\itemsep{0pt}
   \item[a] a complex number. Must be of type \texttt{sf\_double\_complex}.  
   \item[b] a complex number. Must be of type \texttt{sf\_double\_complex}.  
\end{desclist}

\subsubsection*{Output}
\begin{desclist}{\tt }{\quad}[\tt ]
   \setlength\itemsep{0pt}
   \item[c] the product $ab$. It is of type \texttt{sf\_double\_complex}.
\end{desclist}




\subsection{{dccmul}}
Multiplies two complex number. Its output type and one of the input parameters is of type \texttt{kiss\_fft\_cpx}.

\subsubsection*{Call}
\begin{verbatim}c = sf_dccmul(a, b);
\end{verbatim}

\subsubsection*{Definition}
\begin{verbatim}
kiss_fft_cpx sf_dccmul(sf_double_complex a, kiss_fft_cpx b)
/*< complex multiplication >*/
{
   ...
}
\end{verbatim}

\subsubsection*{Input parameters}
\begin{desclist}{\tt }{\quad}[\tt ]
   \setlength\itemsep{0pt}
   \item[a] a complex number. Must be of type \texttt{sf\_double\_complex}.  
   \item[b] a complex number. Must be of type \texttt{kiss\_fft\_cpx}.
\end{desclist}

\subsubsection*{Output}
\begin{desclist}{\tt }{\quad}[\tt ]
   \setlength\itemsep{0pt}
   \item[c] the product $ab$. It is of type \texttt{sf\_double\_complex}.
\end{desclist}




\subsection{{dcdmul}}
Multiplies two complex number. One of the input parameters is \texttt{kiss\_fft\_cpx}. This means that it should only be used if \texttt{complex.h} header is not used.

\subsubsection*{Call}
\begin{verbatim}c = sf_dcdmul(a, b);\end{verbatim}

\subsubsection*{Definition}
\begin{verbatim}
sf_double_complex sf_dcdmul(sf_double_complex a, kiss_fft_cpx b)
/*< complex multiplication >*/
{
   ...
}
\end{verbatim}

\subsubsection*{Input parameters}
\begin{desclist}{\tt }{\quad}[\tt ]
   \setlength\itemsep{0pt}
   \item[a] a complex number. Must be of type \texttt{sf\_double\_complex}.  
   \item[b] a complex number. Must be of type \texttt{kiss\_fft\_cpx}.
\end{desclist}

\subsubsection*{Output}
\begin{desclist}{\tt }{\quad}[\tt ]
   \setlength\itemsep{0pt}
   \item[c] product $ab$ of the two complex numbers $a$ and $b$. It is of type \texttt{sf\_double\_complex}.
\end{desclist}



\subsection{{dcrmul}}
Multiplies a complex number with a real number of type \texttt{double}. 

\subsubsection*{Call}
\begin{verbatim}c = sf_dcrmul(a, b);\end{verbatim}

\subsubsection*{Definition}
\begin{verbatim}
sf_double_complex sf_dcrmul(sf_double_complex a, double b)
/*< complex by real multiplication >*/
{
   ...
}
\end{verbatim}

\subsubsection*{Input parameters}
\begin{desclist}{\tt }{\quad}[\tt ]
   \setlength\itemsep{0pt}
   \item[a] a complex number. Must be of type \texttt{sf\_double\_complex}.  
   \item[b] a real number (\texttt{double}).  
\end{desclist}

\subsubsection*{Output}
\begin{desclist}{\tt }{\quad}[\tt ]
   \setlength\itemsep{0pt}
   \item[c] product of the complex number $a$ and the real number $b$. It is of type \texttt{sf\_double\_complex}.
\end{desclist}




\subsection{{dcdiv}}
Divides two complex numbers.

\subsubsection*{Call}
\begin{verbatim}c = sf_dcdiv(a, b);\end{verbatim}

\subsubsection*{Definition}
\begin{verbatim}
sf_double_complex sf_dcdiv(sf_double_complex a, sf_double_complex b)
/*< complex division >*/
{
   ...
}
\end{verbatim}

\subsubsection*{Input parameters}
\begin{desclist}{\tt }{\quad}[\tt ]
   \setlength\itemsep{0pt}
   \item[a] a complex number. Must be of type \texttt{sf\_double\_complex}.  
   \item[b] a complex number. Must be of type \texttt{sf\_double\_complex}.  
\end{desclist}

\subsubsection*{Output}
\begin{desclist}{\tt }{\quad}[\tt ]
   \setlength\itemsep{0pt}
   \item[c] $\frac{a}{b}$. It is of type \texttt{sf\_double\_complex}.
\end{desclist}




\subsection{{cabs}}
Returns the absolute value (magnitude) of a complex number. It uses the \texttt{hypot} function from the C library.

\subsubsection*{Call}
\begin{verbatim}a = sf_cabs(z);\end{verbatim}

\subsubsection*{Definition}
\begin{verbatim}
double sf_cabs(sf_double_complex z)
/*< replacement for cabsf >*/
{
   ...
}
\end{verbatim}

\subsubsection*{Input parameters}
\begin{desclist}{\tt }{\quad}[\tt ]
   \setlength\itemsep{0pt}
   \item[z] a complex number. Must be of type \texttt{sf\_double\_complex}.  
\end{desclist}

\subsubsection*{Output}
\begin{desclist}{\tt }{\quad}[\tt ]
   \setlength\itemsep{0pt}
   \item[hypot(z.r,z.i)] absolute value of the complex number. 
\end{desclist}




\subsection{{cabs}}
Returns the argument of a complex number. It uses the \texttt{atan2} function from the C library.

\subsubsection*{Call}
\begin{verbatim}u = sf_carg(z);\end{verbatim}

\subsubsection*{Definition}
\begin{verbatim}
double sf_carg(sf_double_complex z)
/*< replacement for cargf >*/
{
   ...
}
\end{verbatim}

\subsubsection*{Input parameters}
\begin{desclist}{\tt }{\quad}[\tt ]
   \setlength\itemsep{0pt}
   \item[z] a complex number. Must be of type \texttt{sf\_double\_complex}.  
\end{desclist}

\subsubsection*{Output}
\begin{desclist}{\tt }{\quad}[\tt ]
   \setlength\itemsep{0pt}
   \item[atan2(z.r,z.i)] argument of the complex number. 
\end{desclist}




\subsection{{crealf}}
Returns the real part of the complex number.

\subsubsection*{Call}
\begin{verbatim}r = sf_crealf(c);\end{verbatim}

\subsubsection*{Definition}
\begin{verbatim}
float sf_crealf(kiss_fft_cpx c)
/*< real part >*/
{
   ...
}
\end{verbatim}

\subsubsection*{Input parameters}
\begin{desclist}{\tt }{\quad}[\tt ]
   \setlength\itemsep{0pt}
   \item[c] a complex number. Must be of type \texttt{kiss\_fft\_cpx}.
\end{desclist}

\subsubsection*{Output}
\begin{desclist}{\tt }{\quad}[\tt ]
   \setlength\itemsep{0pt}
   \item[r] real part of the complex number. It is of type \texttt{float}.
\end{desclist}




\subsection{{cimagf}}
Returns the imaginary part of the complex number.

\subsubsection*{Call}
\begin{verbatim}im = sf_cimagf(c);\end{verbatim}

\subsubsection*{Definition}
\begin{verbatim}
float sf_cimagf(kiss_fft_cpx c)
/*< imaginary part >*/
{
   ...
}
\end{verbatim}

\subsubsection*{Input parameters}
\begin{desclist}{\tt }{\quad}[\tt ]
   \setlength\itemsep{0pt}
   \item[c]      a complex number. Must be of type \texttt{kiss\_fft\_cpx}.
\end{desclist}

\subsubsection*{Output}
\begin{desclist}{\tt }{\quad}[\tt ]
   \setlength\itemsep{0pt}
   \item[im] imaginary part of the complex number. It is of type \texttt{float}.
\end{desclist}




\subsection{{cprint}}
Prints the complex number on the screen. This is done using the \hyperref[sec:sf_warning]{\texttt{sf\_warning}}.

\subsubsection*{Call}
\begin{verbatim}cprint(c);\end{verbatim}

\subsubsection*{Definition}
\begin{verbatim}
void cprint (sf_complex c)
/*< print a complex number (for debugging purposes) >*/
{
   ...
}
\end{verbatim}

\subsubsection*{Input parameters}
\begin{desclist}{\tt }{\quad}[\tt ]
   \setlength\itemsep{0pt}
   \item[c] a complex number (\texttt{sf\_complex}).  
\end{desclist}




\subsection{{cadd}}
Adds two complex numbers. The output is of type \texttt{kiss\_fft\_cpx}.

\subsubsection*{Call}
\begin{verbatim}c = sf_cadd(a, b);\end{verbatim}

\subsubsection*{Definition}
\begin{verbatim}
kiss_fft_cpx sf_cadd(kiss_fft_cpx a, kiss_fft_cpx b)
/*< complex addition >*/
{
   ...
}
\end{verbatim}

\subsubsection*{Input parameters}
\begin{desclist}{\tt }{\quad}[\tt ]
   \setlength\itemsep{0pt}
   \item[a] a complex number. Must be of type \texttt{kiss\_fft\_cpx}.
   \item[b] a complex number. Must be of type \texttt{kiss\_fft\_cpx}.
\end{desclist}

\subsubsection*{Output}
\begin{desclist}{\tt }{\quad}[\tt ]
   \setlength\itemsep{0pt}
   \item[c] the sum $a+b$ of the two complex numbers $a$, $b$. It is of type \texttt{kiss\_fft\_cpx}.
\end{desclist}




\subsection{{csub}}
Subtracts two complex numbers. The output is of type \texttt{kiss\_fft\_cpx}.

\subsubsection*{Call}
\begin{verbatim}c = sf_csub(a, b);\end{verbatim}

\subsubsection*{Definition}
\begin{verbatim}
kiss_fft_cpx sf_csub(kiss_fft_cpx a, kiss_fft_cpx b)
/*< complex subtraction >*/
{
   ...
}
\end{verbatim}

\subsubsection*{Input parameters}
\begin{desclist}{\tt }{\quad}[\tt ]
   \setlength\itemsep{0pt}
   \item[a] a complex number. Must be of type \texttt{kiss\_fft\_cpx}.
   \item[b] a complex number. Must be of type \texttt{kiss\_fft\_cpx}.
\end{desclist}

\subsubsection*{Output}
\begin{desclist}{\tt }{\quad}[\tt ]
   \setlength\itemsep{0pt}
   \item[c] difference of the two complex numbers $a$, $b$. It is of type \texttt{kiss\_fft\_cpx}.
\end{desclist}




\subsection{{csqrtf}}
Returns the square root of a complex number. The output is of type \texttt{kiss\_fft\_cpx}.

\subsubsection*{Call}
\begin{verbatim}c = sf_csqrtf (c);\end{verbatim}

\subsubsection*{Definition}
\begin{verbatim}
kiss_fft_cpx sf_csqrtf (kiss_fft_cpx c)
/*< complex square root >*/
{
   ...
}
\end{verbatim}

\subsubsection*{Input parameters}
\begin{desclist}{\tt }{\quad}[\tt ]
   \setlength\itemsep{0pt}
   \item[a] a complex number. Must be of type \texttt{kiss\_fft\_cpx}.
   \item[b] a complex number. Must be of type \texttt{kiss\_fft\_cpx}.
\end{desclist}

\subsubsection*{Output}
\begin{desclist}{\tt }{\quad}[\tt ]
   \setlength\itemsep{0pt}
   \item[c] square root of the complex number. It is of type \texttt{kiss\_fft\_cpx}.
\end{desclist}




\subsection{{cdiv}}
Divides two complex numbers. The output is of type \texttt{kiss\_fft\_cpx}.

\subsubsection*{Call}
\begin{verbatim}c = sf_cdiv(a, b);\end{verbatim}

\subsubsection*{Definition}
\begin{verbatim}
kiss_fft_cpx sf_cdiv(kiss_fft_cpx a, kiss_fft_cpx b)
/*< complex division >*/
{
   ...
}
\end{verbatim}

\subsubsection*{Input parameters}
\begin{desclist}{\tt }{\quad}[\tt ]
   \setlength\itemsep{0pt}
   \item[a] a complex number. Must be of type \texttt{sf\_double\_complex}.  
   \item[b] a complex number. Must be of type \texttt{sf\_double\_complex}.  
\end{desclist}

\subsubsection*{Output}
\begin{desclist}{\tt }{\quad}[\tt ]
   \setlength\itemsep{0pt}
   \item[c] $\frac{a}{b}$. It is of type \texttt{kiss\_fft\_cpx}.
\end{desclist}




\subsection{{cmul}}
Multiplies two complex numbers. The output is of type \texttt{kiss\_fft\_cpx}.

\subsubsection*{Call}
\begin{verbatim}c = sf_cmul(a, b);\end{verbatim}

\subsubsection*{Definition}
\begin{verbatim}
kiss_fft_cpx sf_cmul(kiss_fft_cpx a, kiss_fft_cpx b)
/*< complex multiplication >*/
{
   ...
}
\end{verbatim}

\subsubsection*{Input parameters}
\begin{desclist}{\tt }{\quad}[\tt ]
   \setlength\itemsep{0pt}
   \item[a] a complex number. Must be of type \texttt{sf\_double\_complex}.  
   \item[b] a complex number. Must be of type \texttt{sf\_double\_complex}.  
\end{desclist}

\subsubsection*{Output}
\begin{desclist}{\tt }{\quad}[\tt ]
   \setlength\itemsep{0pt}
   \item[c] product of the two complex numbers a and b. It is of type \texttt{kiss\_fft\_cpx}.
\end{desclist}




\subsection{{crmul}}
Multiplies a complex number with a real number. The output is of type \texttt{kiss\_fft\_cpx}.

\subsubsection*{Call}
\begin{verbatim}c = sf_crmul(a, b);\end{verbatim}

\subsubsection*{Definition}
\begin{verbatim}
kiss_fft_cpx sf_crmul(kiss_fft_cpx a, float b)
/*< complex by real multiplication >*/
{
   ...
}
\end{verbatim}

\subsubsection*{Input parameters}
\begin{desclist}{\tt }{\quad}[\tt ]
   \setlength\itemsep{0pt}
   \item[a] a complex number. Must be of type \texttt{sf\_double\_complex}.  
   \item[b] a real number (\texttt{float}).  
\end{desclist}

\subsubsection*{Output}
\begin{desclist}{\tt }{\quad}[\tt ]
   \setlength\itemsep{0pt}
   \item[c] the product $ab$ of a complex number $a$ and real number $b$. It is of type \texttt{kiss\_fft\_cpx}.
\end{desclist}




\subsection{{cneg}}
Returns negative of a complex number. The output is of type \texttt{kiss\_fft\_cpx}.

\subsubsection*{Call}
\begin{verbatim}b = sf_cneg(a);\end{verbatim}

\subsubsection*{Definition}
\begin{verbatim}
kiss_fft_cpx sf_cneg(kiss_fft_cpx a)
/*< unary minus >*/
{
   ...
}
\end{verbatim}

\subsubsection*{Input parameters}
\begin{desclist}{\tt }{\quad}[\tt ]
   \setlength\itemsep{0pt}
   \item[a] a complex number. Must be of type \texttt{sf\_double\_complex}.  
\end{desclist}

\subsubsection*{Output}
\begin{desclist}{\tt }{\quad}[\tt ]
   \setlength\itemsep{0pt}
   \item[a] negative of a complex number. It is of type \texttt{kiss\_fft\_cpx}.
\end{desclist}




\subsection{{conjf}}
Returns complex conjugate of a complex number. The output is of type \texttt{kiss\_fft\_cpx}.

\subsubsection*{Call}
\begin{verbatim}z1 = sf_conjf(z);\end{verbatim}

\subsubsection*{Definition}
\begin{verbatim}
kiss_fft_cpx sf_conjf(kiss_fft_cpx z)
/*< complex conjugate >*/
{
   ...
}
\end{verbatim}

\subsubsection*{Input parameters}
\begin{desclist}{\tt }{\quad}[\tt ]
   \setlength\itemsep{0pt}
   \item[z] a complex number. Must be of type \texttt{sf\_double\_complex}.  
\end{desclist}

\subsubsection*{Output}
\begin{desclist}{\tt }{\quad}[\tt ]
   \setlength\itemsep{0pt}
   \item[z1] complex conjugate of the complex number. It is of type \texttt{kiss\_fft\_cpx}.
\end{desclist}




\subsection{{cabsf}}
Returns the magnitude of a complex number. It uses a function \texttt{hypotf} from \texttt{c99.h}, which calls the \texttt{hypot} function from \texttt{math.h} in the C library.

\subsubsection*{Call}
\begin{verbatim}w = sf_cabsf(kiss_fft_cpx z);\end{verbatim}

\subsubsection*{Definition}
\begin{verbatim}
float sf_cabsf(kiss_fft_cpx z)
/*< replacement for cabsf >*/
{
   ...
}
\end{verbatim}

\subsubsection*{Input parameters}
\begin{desclist}{\tt }{\quad}[\tt ]
   \setlength\itemsep{0pt}
   \item[z] a complex number. Must be of type \texttt{kiss\_fft\_cpx}.
\end{desclist}

\subsubsection*{Output}
\begin{desclist}{\tt }{\quad}[\tt ]
   \setlength\itemsep{0pt}
   \item[w] magnitude of a complex number. It is of type \texttt{kiss\_fft\_cpx}.
\end{desclist}




\subsection{{cargf}}
Returns the argument of a complex number. It uses a function \texttt{atan2f} from \texttt{c99.h}, which calls the \texttt{atan2} function from \texttt{math.h} in the C library.

\subsubsection*{Call}
\begin{verbatim}u = sf_cargf(z);\end{verbatim}

\subsubsection*{Definition}
\begin{verbatim}
float sf_cargf(kiss_fft_cpx z)
/*< replacement for cargf >*/
{
   ...
}
\end{verbatim}

\subsubsection*{Input parameters}
\begin{desclist}{\tt }{\quad}[\tt ]
   \setlength\itemsep{0pt}
   \item[z] a complex number. Must be of type \texttt{kiss\_fft\_cpx}.
\end{desclist}

\subsubsection*{Output}
\begin{desclist}{\tt }{\quad}[\tt ]
   \setlength\itemsep{0pt}
   \item[u] argument of a complex number. It is of type \texttt{kiss\_fft\_cpx}.
\end{desclist}




\subsection{{ctanhf}}
Returns hyperbolic tangent of a complex number. It uses a function
like \texttt{coshf} and \texttt{sinhf} from \texttt{c99.h}, which call \texttt{cosh} and \texttt{sinh} functions from \texttt{math.h} in the C library.

\subsubsection*{Call}
\begin{verbatim}th = sf_ctanhf(z);\end{verbatim}

\subsubsection*{Definition}
\begin{verbatim}
kiss_fft_cpx sf_ctanhf(kiss_fft_cpx z)
/*< complex hyperbolic tangent >*/
{
   ...
}
\end{verbatim}

\subsubsection*{Input parameters}
\begin{desclist}{\tt }{\quad}[\tt ]
   \setlength\itemsep{0pt}
   \item[z] a complex number. Must be of type \texttt{kiss\_fft\_cpx}.
\end{desclist}

\subsubsection*{Output}
\begin{desclist}{\tt }{\quad}[\tt ]
   \setlength\itemsep{0pt}
   \item[th] hyperbolic tangent of a complex number. It is of type \texttt{kiss\_fft\_cpx}.
\end{desclist}




\subsection{{ccosf}}
Returns cosine of a complex number. It uses the functions like \texttt{coshf} and \texttt{sinhf} from \texttt{c99.h}, which call \texttt{cosh} and \texttt{sinh} functions from \texttt{math.h} in the C library.

\subsubsection*{Call}
\begin{verbatim}w =  sf_ccosf (z);\end{verbatim}

\subsubsection*{Definition}
\begin{verbatim}
kiss_fft_cpx sf_ccosf(kiss_fft_cpx z)
/*< complex cosine >*/
{
   ...
}
\end{verbatim}

\subsubsection*{Input parameters}
\begin{desclist}{\tt }{\quad}[\tt ]
   \setlength\itemsep{0pt}
   \item[z] a complex number. Must be of type \texttt{kiss\_fft\_cpx}.
\end{desclist}

\subsubsection*{Output}
\begin{desclist}{\tt }{\quad}[\tt ]
   \setlength\itemsep{0pt}
   \item[w] cosine of a complex number. It is of type \texttt{kiss\_fft\_cpx}.
\end{desclist}




\subsection{{ccoshf}}
Returns hyperbolic cosine of a complex number. It uses the functions like \texttt{coshf} and \texttt{sinhf} from \texttt{c99.h}, which call \texttt{cosh} and \texttt{sinh} functions from \texttt{math.h} in the C library.

\subsubsection*{Call}
\begin{verbatim}w = sf_ccoshf(z);\end{verbatim}

\subsubsection*{Definition}
\begin{verbatim}
kiss_fft_cpx sf_ccoshf(kiss_fft_cpx z)
/*< complex hyperbolic cosine >*/
{
   ...    
}
\end{verbatim}

\subsubsection*{Input parameters}
\begin{desclist}{\tt }{\quad}[\tt ]
   \setlength\itemsep{0pt}
   \item[z] a complex number. Must be of type \texttt{kiss\_fft\_cpx}.
\end{desclist}

\subsubsection*{Output}
\begin{desclist}{\tt }{\quad}[\tt ]
   \setlength\itemsep{0pt}
   \item[w] huperbolic cosine of a complex number. It is of type \texttt{kiss\_fft\_cpx}.
\end{desclist}




\subsection{{ccosf}}
Returns sine of a complex number. It uses the functions like \texttt{coshf} and \texttt{sinhf} from \texttt{c99.h}, which call \texttt{cosh} and \texttt{sinh} functions from \texttt{math.h} in the C library.

\subsubsection*{Call}
\begin{verbatim}w = sf_csinf(z);\end{verbatim}

\subsubsection*{Definition}
\begin{verbatim}
kiss_fft_cpx sf_csinf(kiss_fft_cpx z)
/*< complex sine >*/
{
   ...
}
\end{verbatim}

\subsubsection*{Input parameters}
\begin{desclist}{\tt }{\quad}[\tt ]
   \setlength\itemsep{0pt}
   \item[z] a complex number. Must be of type \texttt{kiss\_fft\_cpx}.
\end{desclist}

\subsubsection*{Output}
\begin{desclist}{\tt }{\quad}[\tt ]
   \setlength\itemsep{0pt}
   \item[w] sine of a complex number. It is of type \texttt{kiss\_fft\_cpx}.
\end{desclist}




\subsection{{csinhf}}
Returns hyperbolic cosine of a complex number. It uses the functions like \texttt{coshf} and \texttt{sinhf} from \texttt{c99.h}, which call \texttt{cosh} and \texttt{sinh} functions from \texttt{math.h} in the C library.

\subsubsection*{Call}
\begin{verbatim}w = sf_csinhf(z);\end{verbatim}

\subsubsection*{Definition}
\begin{verbatim}
kiss_fft_cpx sf_csinhf(kiss_fft_cpx z)
/*< complex hyperbolic sine >*/
{
   ...
}
\end{verbatim}

\subsubsection*{Input parameters}
\begin{desclist}{\tt }{\quad}[\tt ]
   \setlength\itemsep{0pt}
   \item[z] a complex number. Must be of type \texttt{kiss\_fft\_cpx}.
\end{desclist}

\subsubsection*{Output}
\begin{desclist}{\tt }{\quad}[\tt ]
   \setlength\itemsep{0pt}
   \item[w] huperbolic cosine of a complex number. It is of type \texttt{kiss\_fft\_cpx}.
\end{desclist}




\subsection{{clogf}}
Returns natural logarithm of a complex number. It uses the functions like \texttt{logf} and \texttt{hypotf} from \texttt{c99.h}, which call \texttt{log} and \texttt{hypot} functions from \texttt{math.h} in the C library.

\subsubsection*{Call}
\begin{verbatim}w = sf_clogf(z);\end{verbatim}

\subsubsection*{Definition}
\begin{verbatim}
kiss_fft_cpx sf_clogf(kiss_fft_cpx z)
/*< complex natural logarithm >*/
{
   ...    
}
\end{verbatim}

\subsubsection*{Input parameters}
\begin{desclist}{\tt }{\quad}[\tt ]
   \setlength\itemsep{0pt}
   \item[z] a complex number. Must be of type \texttt{kiss\_fft\_cpx}.
\end{desclist}

\subsubsection*{Output}
\begin{desclist}{\tt }{\quad}[\tt ]
   \setlength\itemsep{0pt}
   \item[w] natural logarithm of a complex number. It is of type \texttt{kiss\_fft\_cpx}.
\end{desclist}




\subsection{{cexpf}}
Returns exponential of a complex number. It uses the functions like \texttt{expf} and \texttt{cosf} from \texttt{c99.h}, which call \texttt{exp} and \texttt{cos} functions from \texttt{math.h} in the C library.

\subsubsection*{Call}
\begin{verbatim}w = sf_cexpf(z);\end{verbatim}

\subsubsection*{Definition}
\begin{verbatim}
kiss_fft_cpx sf_cexpf(kiss_fft_cpx z)
/*< complex exponential >*/
{
   ...
}
\end{verbatim}

\subsubsection*{Input parameters}
\begin{desclist}{\tt }{\quad}[\tt ]
   \setlength\itemsep{0pt}
   \item[z] a complex number. Must be of type \texttt{kiss\_fft\_cpx}.
\end{desclist}

\subsubsection*{Output}
\begin{desclist}{\tt }{\quad}[\tt ]
   \setlength\itemsep{0pt}
   \item[z] exponential of a complex number. It is of type \texttt{kiss\_fft\_cpx}.
\end{desclist}




\subsection{{ctanf}}
Returns tangent of a complex number. It uses the functions like \texttt{sinf} and \texttt{cosf} from \texttt{c99.h}, which call \texttt{sin} and \texttt{cos} functions from \texttt{math.h} in the C library.

\subsubsection*{Call}
\begin{verbatim}w = sf_ctanf(z);\end{verbatim}

\subsubsection*{Definition}
\begin{verbatim}
kiss_fft_cpx sf_ctanf(kiss_fft_cpx z)
/*< complex tangent >*/
{
   ...
}
\end{verbatim}

\subsubsection*{Input parameters}
\begin{desclist}{\tt }{\quad}[\tt ]
   \setlength\itemsep{0pt}
   \item[z] a complex number. Must be of type \texttt{kiss\_fft\_cpx}.
\end{desclist}

\subsubsection*{Output}
\begin{desclist}{\tt }{\quad}[\tt ]
   \setlength\itemsep{0pt}
   \item[w] tangent of a complex number. It is of type \texttt{kiss\_fft\_cpx}.
\end{desclist}




\subsection{{casinf}}
Returns hyperbolic arcsine of a complex number. It uses the function \texttt{asinf} from \texttt{c99.h}, which calls the \texttt{asin} function from \texttt{math.h} in the C library.

\subsubsection*{Call}
\begin{verbatim}w = sf_casinf(z);\end{verbatim}

\subsubsection*{Definition}
\begin{verbatim}
kiss_fft_cpx sf_casinf(kiss_fft_cpx z)
/*< complex hyperbolic arcsine >*/
{
   ...  
}
\end{verbatim}

\subsubsection*{Input parameters}
\begin{desclist}{\tt }{\quad}[\tt ]
   \setlength\itemsep{0pt}
   \item[z] a complex number. Must be of type \texttt{kiss\_fft\_cpx}.
\end{desclist}

\subsubsection*{Output}
\begin{desclist}{\tt }{\quad}[\tt ]
   \setlength\itemsep{0pt}
   \item[w] arcsine of a complex number. It is of type \texttt{kiss\_fft\_cpx}.
\end{desclist}




\subsection{{cacosf}}
Returns hyperbolic arccosine of a complex number. It uses \texttt{sf\_cacosf}.

\subsubsection*{Call}
\begin{verbatim}w = sf_cacosf(z);\end{verbatim}

\subsubsection*{Definition}
\begin{verbatim}
kiss_fft_cpx sf_cacosf(kiss_fft_cpx z)
/*< complex hyperbolic arccosine >*/
{
   ...
}
\end{verbatim}

\subsubsection*{Input parameters}
\begin{desclist}{\tt }{\quad}[\tt ]
   \setlength\itemsep{0pt}
   \item[z] a complex number. Must be of type \texttt{kiss\_fft\_cpx}.
\end{desclist}

\subsubsection*{Output}
\begin{desclist}{\tt }{\quad}[\tt ]
   \setlength\itemsep{0pt}
   \item[w] hyperbolic arccosine of a complex number. It is of type \texttt{kiss\_fft\_cpx}.
\end{desclist}




\subsection{{catanf}}
Returns arctangent of a complex number. It uses \texttt{sf\_clogf} and \texttt{sf\_cdiv}.

\subsubsection*{Call}
\begin{verbatim}w = sf_catanf(z);\end{verbatim}

\subsubsection*{Definition}
\begin{verbatim}
kiss_fft_cpx sf_catanf(kiss_fft_cpx z)
/*< complex arctangent >*/
{
   ...   
}     
\end{verbatim}

\subsubsection*{Input parameters}
\begin{desclist}{\tt }{\quad}[\tt ]
   \setlength\itemsep{0pt}
   \item[z] a complex number. Must be of type \texttt{kiss\_fft\_cpx}.
\end{desclist}

\subsubsection*{Output}
\begin{desclist}{\tt }{\quad}[\tt ]
   \setlength\itemsep{0pt}
   \item[w] arctangent of a complex number. It is of type \texttt{kiss\_fft\_cpx}.
\end{desclist}



\subsection{{catanhf}}
Returns hyperbolic arctangent of a complex number. It uses \texttt{sf\_catanf}.

\subsubsection*{Call}
\begin{verbatim}w =_cpx sf_catanhf(z);\end{verbatim}

\subsubsection*{Definition}
\begin{verbatim}
kiss_fft_cpx sf_catanhf(kiss_fft_cpx z)
/*< complex hyperbolic arctangent >*/
{
   ...
}     
\end{verbatim}

\subsubsection*{Input parameters}
\begin{desclist}{\tt }{\quad}[\tt ]
   \setlength\itemsep{0pt}
   \item[z] a complex number. Must be of type \texttt{kiss\_fft\_cpx}.
\end{desclist}

\subsubsection*{Output}
\begin{desclist}{\tt }{\quad}[\tt ]
   \setlength\itemsep{0pt}
   \item[z] hyperbolic arctangent of a complex number. It is of type \texttt{kiss\_fft\_cpx}.
\end{desclist}



\subsection{{casinhf}}
Returns hyperbolic arcsine of a complex number. It uses \texttt{sf\_casinf}.

\subsubsection*{Call}
\begin{verbatim}w = sf_casinhf(z);\end{verbatim}

\subsubsection*{Definition}
\begin{verbatim}
kiss_fft_cpx sf_casinhf(kiss_fft_cpx z)
/*< complex hyperbolic sine >*/
{
   ...
}     
\end{verbatim}

\subsubsection*{Input parameters}
\begin{desclist}{\tt }{\quad}[\tt ]
   \setlength\itemsep{0pt}
   \item[z] a complex number. Must be of type \texttt{kiss\_fft\_cpx}.
\end{desclist}

\subsubsection*{Output}
\begin{desclist}{\tt }{\quad}[\tt ]
   \setlength\itemsep{0pt}
   \item[z] hyperbolic arcsine of a complex number. It is of type \texttt{kiss\_fft\_cpx}.
\end{desclist}



\subsection{{cacoshf}}
Returns hyperbolic arccosine of a complex number. It uses \texttt{sf\_casinf}.

\subsubsection*{Call}
\begin{verbatim}w = sf_cacoshf(z);\end{verbatim}

\subsubsection*{Definition}
\begin{verbatim}
kiss_fft_cpx sf_cacoshf(kiss_fft_cpx z)
/*< complex hyperbolic cosine >*/
{
   ...
}
\end{verbatim}

\subsubsection*{Input parameters}
\begin{desclist}{\tt }{\quad}[\tt ]
   \setlength\itemsep{0pt}
   \item[z] a complex number. Must be of type \texttt{kiss\_fft\_cpx}.
\end{desclist}

\subsubsection*{Output}
\begin{desclist}{\tt }{\quad}[\tt ]
   \setlength\itemsep{0pt}
   \item[w] hyperbolic arccosine of a complex number. It is of type \texttt{kiss\_fft\_cpx}.
\end{desclist}




\subsection{{cpowf}}
Returns the complex base $a$ raised to complex power $b$.

\subsubsection*{Call}
\begin{verbatim}c = sf_cpowf(a, b);\end{verbatim}

\subsubsection*{Definition}
\begin{verbatim}
kiss_fft_cpx sf_cpowf(kiss_fft_cpx a, kiss_fft_cpx b)
/*< complex power >*/
{
   ...
}
\end{verbatim}

\subsubsection*{Input parameters}
\begin{desclist}{\tt }{\quad}[\tt ]
   \setlength\itemsep{0pt}
   \item[a] a complex number. Must be of type \texttt{kiss\_fft\_cpx}.
   \item[b] a complex number. Must be of type \texttt{kiss\_fft\_cpx}.
\end{desclist}

\subsubsection*{Output}
\begin{desclist}{\tt }{\quad}[\tt ]
   \setlength\itemsep{0pt}
   \item[c] $a^b$. It is a complex number of type \texttt{kiss\_fft\_cpx}.
\end{desclist}



