\section{Runge-Kutta ODE solvers (runge.c)}




\subsection{{sf\_runge\_init}}
Initializes the required variables and allocates the required space for the ODE solver for raytracing.

\subsubsection*{Call}
\begin{verbatim}sf_runge_init(dim1, n1, d1);\end{verbatim}

\subsubsection*{Definition}
\begin{verbatim}
void sf_runge_init(int dim1 /* dimensionality */, 
                   int n1   /* number of ray tracing steps */, 
                   float d1 /* step in time */)
/*< initialize >*/
{
   ...
}

\subsubsection*{Input parameters}
\begin{desclist}{\tt }{\quad}[\tt dim1]
   \setlength\itemsep{0pt}
   \item[dim1] dimension of the ODE (\texttt{int}). 
   \item[n1]   number of steps for performing the raytracing (\texttt{int}).  
   \item[d1]   number of time steps for performing the raytracing (\texttt{int}).  
\end{desclist}



\subsection{{sf\_runge\_close\_init}}
Frees all allocated memory.

\subsubsection*{Call}
\begin{verbatim}sf_runge_close();\end{verbatim}

\subsubsection*{Definition}
\begin{verbatim}
void sf_runge_close(void)
/*< free allocated storage >*/
{
   ...
}
\end{verbatim}




\subsection{{sf\_ode23}}
This function solves a first order ODE to calculate the travel time by raytracing.

\subsubsection*{Call}
\begin{verbatim}
f = sf_ode23 (float t, tol, y, par, 
                (*rhs)(void*,float*,float*), (*term)(void*,float*));
\end{verbatim}

\subsubsection*{Definition}
\begin{verbatim}
float sf_ode23 (float t    /* time integration */,
                float* tol /* error tolerance */,
                float* y   /* [dim] solution */, 
                void* par  /* parameters for function evaluation */,
                void (*rhs)(void*,float*,float*) 
                /* RHS function */, 
                int (*term)(void*,float*)
             /* function returning 1 if the ray needs to terminate */)
/*< ODE solver for dy/dt = f where f comes from rhs(par,y,f)
    Note: Value of y is changed inside the function.>*/
{
   ...
}
\end{verbatim}

\subsubsection*{Input parameters}
\begin{desclist}{\tt }{\quad}[\tt (*rhs)(void*,float*,float*)]
   \setlength\itemsep{0pt}
   \item[t]   total time for integration (\texttt{float}). 
   \item[tol] error tolerance (\texttt{float*}).  
   \item[y]   the solution, of dimension \texttt{dim} (\texttt{float*}).  
   \item[par] parameters to evaluate the rhs function (\texttt{void*}). 
   \item[(*rhs)(void*,float*,float*)] function which evaluates the rhs of the ODE. Its inputs the parameters, the solution and the k's in Runge-Kutta scheme. Output is the rhs function $f$ of the ODE (\texttt{void*}).
   \item[(*term)(void*,float*)] a function which returns 1 if the ray is to be terminated (\texttt{int}).
\end{desclist}

\subsubsection*{Output}
\begin{desclist}{\tt }{\quad}[\tt ]
   \setlength\itemsep{0pt}
   \item[t1] total travel time for the ray. It is of type \texttt{float}.
\end{desclist}




\subsection{{sf\_ode23\_step}}
Solves a first order ODE and returns trajectory calculated by raytracing.

\subsubsection*{Call}
\begin{verbatim}
it = sf_ode23_step (y, par, (*rhs)(void*,float*,float*), (*term)(void*,float*), traj);
\end{verbatim}

\subsubsection*{Definition}
\begin{verbatim}
int sf_ode23_step (float* y    /* [dim] solution */, 
                   void* par   /* parameters for function evaluation */,
                   void (*rhs)(void*,float*,float*) 
                   /* RHS function */, 
                   int (*term)(void*,float*)
                   /* function returning 1 if the ray needs to terminate */, 
                   float** traj /* [nt+1][dim] - ray trajectory (output) */) 
/*< ODE solver for dy/dt = f where f comes from rhs(par,y,f)
  Note:
  1. Value of y is changed inside the function.
  2. The output code for it = ode23_step(...)
  it=0 - ray traced to the end without termination
  it>0 - ray terminated
  The total traveltime along the ray is 
  nt*dt if (it = 0); it*dt otherwise 
  >*/
{
   ...  
}
\end{verbatim}

\subsubsection*{Input parameters}
\begin{desclist}{\tt }{\quad}[\tt (*rhs)(void*,float*,float*)]
   \setlength\itemsep{0pt}
   \item[y]   the solution, of dimension \texttt{dim} (\texttt{float*}).  
   \item[par] parameters to evaluate the rhs function (\texttt{void*}).  
   \item[(*rhs)(void*,float*,float*)] function which evaluates the rhs of the ODE. Its inputs the parameters, the solution and the k's in Runge-Kutta scheme. Output is the rhs function $f$ of the ODE (\texttt{void*}).
   \item[(*term)(void*,float*)] a function which returns 1 if the ray is to be terminated (\texttt{int}). 
   \item[traj] location where the output ray trajectory is to be stored (\texttt{float**}).  
\end{desclist}

\subsubsection*{Output}
\begin{desclist}{\tt }{\quad}[\tt ]
   \setlength\itemsep{0pt}
   \item[t1] total travel time for the ray. It is of type \texttt{int}.
\end{desclist}





