\section{Axes (axa.c)}\label{sec:axa.c}




\subsection{{sf\_maxa}}
Creates a simple axis.

\subsubsection*{Call}
\begin{verbatim}AA = sf_maxa(n, o, d);\end{verbatim}

\subsubsection*{Definition}
\begin{verbatim}
sf_axis sf_maxa(int n   /* length */, 
                float o /* origin */, 
                float d /* sampling */)
/*< make a simple axis >*/
{
   ...
}
\end{verbatim}

\subsubsection{Input parameters}
\begin{desclist}{\tt }{\quad}[\tt ]
   \setlength\itemsep{0pt}
   \item[n] length of the axis (\texttt{int}).  
   \item[o] origin of the axis (\texttt{float}).  
   \item[d] sampling of the axis (\texttt{float}).  
\end{desclist}

\subsubsection*{Output}
\begin{desclist}{\tt }{\quad}[\tt ]
   \setlength\itemsep{0pt}  
   \item[AA] the axis. It is of type \texttt{sf\_axis}.
\end{desclist}




\subsection{{sf\_iaxa}}\label{sec:sf_iaxa}
Reads an axis from the file which is given in the input.

\subsubsection*{Call}
\begin{verbatim}AA = sf_iaxa(FF, i);\end{verbatim}

\subsubsection*{Definition}
\begin{verbatim}
sf_axis sf_iaxa(sf_file FF, int i) 
/*< read axis i >*/
{
   ...
}
\end{verbatim}

\subsubsection{Input parameters}
\begin{desclist}{\tt }{\quad}[\tt FF]
   \setlength\itemsep{0pt}
   \item[FF] the file from which the axis is to be read (\texttt{sf\_file}).  
   \item[i]  a number which specified which axis is to be read, for example \texttt{n1}, \texttt{n2}, \texttt{n3} etc (\texttt{int}).  
\end{desclist}

\subsubsection*{Output}
\begin{desclist}{\tt }{\quad}[\tt ]
   \setlength\itemsep{0pt}  
   \item[AA] location where the axis is stored. It is of type \texttt{sf\_axis}.
\end{desclist}




\subsection{{sf\_oaxa}}\label{sec:sf_oaxa}
Writes an axis, from the input location, to the file, which is also given in the input.

\subsubsection*{Call}
\begin{verbatim}sf_oaxa(FF, AA, i);\end{verbatim}

\subsubsection*{Definition}
\begin{verbatim}
void sf_oaxa(sf_file FF, const sf_axis AA, int i) 
/*< write axis i >*/
{
   ...
}
\end{verbatim}

\subsubsection{Input parameters}
\begin{desclist}{\tt }{\quad}[\tt AA]
   \setlength\itemsep{0pt}
   \item[FF] the file in which the axis is to be written (\texttt{sf\_file}).  
   \item[AA] the location from where the axis is to be read. Must be of type \texttt{const sf\_axis}.  
   \item[i]  a number which specified which axis is to be read, for example \texttt{n1}, \texttt{n2}, \texttt{n3} etc (\texttt{int}).  
\end{desclist}




\subsection{{sf\_raxa}}
Prints the information about the axis on the screen.

\subsubsection*{Call}
\begin{verbatim}sf_raxa(AA);\end{verbatim}

\subsubsection*{Definition}
\begin{verbatim}  
void sf_raxa(const sf_axis AA) 
/*< report information on axis AA >*/
{    
   ...
}
\end{verbatim}

\subsubsection{Input parameters}
\begin{desclist}{\tt }{\quad}[\tt ]
   \setlength\itemsep{0pt}
   \item[AA] the axis about which the information is required. Must be of type \texttt{const sf\_axis}.  
\end{desclist}




\subsection{{sf\_n}}\label{sec:sf_n}
Provides access to the length of the axis.

\subsubsection*{Call}
\begin{verbatim}AA->n = sf_n(AA);\end{verbatim}

\subsubsection*{Definition}
\begin{verbatim}  
int sf_n(const sf_axis AA) 
/*< access axis length >*/
{
   ...
}
\end{verbatim}

\subsubsection{Input parameters}
\begin{desclist}{\tt }{\quad}[\tt ]
   \setlength\itemsep{0pt}
   \item[AA] the axis whose length is to be accessed. Must be of type \texttt{const sf\_axis}.  
\end{desclist}

\subsubsection*{Output}
\begin{desclist}{\tt }{\quad}[\tt ]
   \setlength\itemsep{0pt}  
   \item[AA->n] length of the axis. It is of type \texttt{int}.
\end{desclist}




\subsection{{sf\_o}}
Provides access to the origin of the axis.

\subsubsection*{Call}
\begin{verbatim}AA->o = sf_o(AA);\end{verbatim}

\subsubsection*{Definition}
\begin{verbatim}  
float sf_o(const sf_axis AA) 
/*< access axis origin >*/
{
   ...
}
\end{verbatim}

\subsubsection{Input parameters}
\begin{desclist}{\tt }{\quad}[\tt ]
   \setlength\itemsep{0pt}
   \item[AA] the axis whose length is to be accessed. Must be of type \texttt{const sf\_axis}.  
\end{desclist}

\subsubsection*{Output}
\begin{desclist}{\tt }{\quad}[\tt ]
   \setlength\itemsep{0pt}  
   \item[AA->o] length of the axis. It is of type \texttt{float}.
\end{desclist}




\subsection{{sf\_d}}\label{sec:sf_d}
Provides access to the sampling of the axis.

\subsubsection*{Call}
\begin{verbatim}AA->d = sf_d(AA);\end{verbatim}

\subsubsection*{Definition}
\begin{verbatim}  
float sf_d(const sf_axis AA) 
/*< access axis sampling >*/
{
   ...
}
\end{verbatim}

\subsubsection{Input parameters}
\begin{desclist}{\tt }{\quad}[\tt ]
   \setlength\itemsep{0pt}
   \item[AA] the axis whose length is to be accessed. Must be of type \texttt{const sf\_axis}.  
\end{desclist}

\subsubsection*{Output}
\begin{desclist}{\tt }{\quad}[\tt ]
   \setlength\itemsep{0pt}  
   \item[AA->d] length of the axis. It is of type \texttt{float}.
\end{desclist}




\subsection{{sf\_nod}}
Copies the length, origin and sampling of the input axis to another place which is also an object of type \texttt{sf\_axis}.

\subsubsection*{Call}
\begin{verbatim}BB = sf_nod(AA);\end{verbatim}

\subsubsection*{Definition}
\begin{verbatim}  
sf_axa sf_nod(const sf_axis AA) 
/*< access length, origin, and sampling >*/
{
   ...
}
\end{verbatim}

\subsubsection{Input parameters}
\begin{desclist}{\tt }{\quad}[\tt ]
   \setlength\itemsep{0pt}
   \item[AA] the axis whose length, origin and sampling is to be accessed. Must be of type \texttt{const sf\_axis}.  
\end{desclist}

\subsubsection*{Output}
\begin{desclist}{\tt }{\quad}[\tt ]
   \setlength\itemsep{0pt}  
   \item[BB] the location where the length, origin and sampling are copied. It is of type \texttt{sf\_axis}.
\end{desclist}




\subsection{{sf\_setn}}
Changes the length of the axis.

\subsubsection*{Call}
\begin{verbatim}AA->n = sf_setn(AA, n);\end{verbatim}

\subsubsection*{Definition}
\begin{verbatim}  
void sf_setn(sf_axis AA, int n)
/*< change axis length >*/
{ AA->n=n; }
\end{verbatim}

\subsubsection{Input parameters}
\begin{desclist}{\tt }{\quad}[\tt AA]
   \setlength\itemsep{0pt}
   \item[AA] the axis whose length is to be changed (\texttt{sf\_axis}).  
   \item[n]  the new length which is to be set (\texttt{int}).  
\end{desclist}




\subsection{{sf\_seto}}
Changes the origin of the axis.

\subsubsection*{Call}
\begin{verbatim}AA->o = sf_seto(AA, o);\end{verbatim}

\subsubsection*{Definition}
\begin{verbatim}  
void sf_seto(sf_axis AA, float o)
/*< change axis origin >*/
{
   ...
}
\end{verbatim}

\subsubsection{Input parameters}
\begin{desclist}{\tt }{\quad}[\tt AA]
   \setlength\itemsep{0pt}
   \item[AA] the axis whose origin is to be changed (\texttt{sf\_axis}).  
   \item[o]  the new origin which is to be set (\texttt{float}).  
\end{desclist}




\subsection{{sf\_setd}}
Changes the sampling of the axis.

\subsubsection*{Call}
\begin{verbatim}AA->d = sf_setd(AA, d);\end{verbatim}

\subsubsection*{Definition}
\begin{verbatim}  
void sf_setd(sf_axis AA, float d)
/*< change axis sampling >*/
{
   ...
}
\end{verbatim}

\subsubsection{Input parameters}
\begin{desclist}{\tt }{\quad}[\tt ]
   \setlength\itemsep{0pt}
   \item[AA] the axis whose sampling is to be changed (\texttt{sf\_axis}).  
   \item[o]  the new sampling which is to be set (\texttt{float}).  
\end{desclist}




\subsection{{sf\_setlabel}}
Changes the label of the axis.

\subsubsection*{Call}
\begin{verbatim}sf_setlabel(AA, label);\end{verbatim}

\subsubsection*{Definition}
\begin{verbatim}  
void sf_setlabel(sf_axis AA, const char* label)
/*< change axis label >*/
{
   ...
}
\end{verbatim}

\subsubsection{Input parameters}
\begin{desclist}{\tt }{\quad}[\tt label]
   \setlength\itemsep{0pt}
   \item[AA]    the axis whose label is to be changed (\texttt{sf\_axis}).  
   \item[label] the new label which is to be set (\texttt{const char*}).  
\end{desclist}




\subsection{{sf\_setunit}}
Changes the unit of the axis.

\subsubsection*{Call}
\begin{verbatim}sf_setunit(AA, unit);\end{verbatim}

\subsubsection*{Definition}
\begin{verbatim}  
void sf_setunit(sf_axis AA, const char* unit)
/*< change axis unit >*/
{
   ...
}
\end{verbatim}

\subsubsection{Input parameters}
\begin{desclist}{\tt }{\quad}[\tt unit]
   \setlength\itemsep{0pt}
   \item[AA]   the axis whose unit is to be changed (\texttt{sf\_axis}).  
   \item[unit] the new unit which is to be set (\texttt{const char*}).  
\end{desclist}

