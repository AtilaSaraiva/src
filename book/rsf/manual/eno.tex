\section{1-D ENO interpolation (eno.c)}\label{sec:eno.c}




\subsection{{sf\_eno\_init}}\label{sec:sf_eno_init}
Initializes an object of type \texttt{sf\_eno} for interpolation.

\subsubsection*{Call}
\begin{verbatim}ent = sf_eno_init (order, n);\end{verbatim}

\subsubsection*{Definition}
\begin{verbatim}
sf_eno sf_eno_init (int order /* interpolation order */, 
              int n     /* data size */)
/*< Initialize interpolation object. >*/
{
   ...
}
\end{verbatim}

\subsubsection*{Input parameters}
\begin{desclist}{\tt }{\quad}[\tt order]
   \setlength\itemsep{0pt}
   \item[order] order of interpolation (\texttt{int}).  
   \item[n]     size of the data (\texttt{int}).  
\end{desclist}

\subsubsection*{Output}
\begin{desclist}{\tt }{\quad}[\tt ]
   \setlength\itemsep{0pt}  
   \item[ent] an object for interpolation. It is of type \texttt{sf\_eno}.
\end{desclist}




\subsection{{sf\_eno\_close}}
Frees the space allocated for the internal storage by \hyperref[sec:sf_eno_init]{\texttt{sf\_eno\_init}}.

\subsubsection*{Call}
\begin{verbatim}sf_eno_close (ent);\end{verbatim}

\subsubsection*{Definition}
\begin{verbatim}
void sf_eno_close (sf_eno ent)
/*< Free internal storage >*/
{
   ...
}
\end{verbatim}




\subsection{{sf\_eno\_set}}
Creates a table for interpolation.

\subsubsection*{Call}
\begin{verbatim}sf_eno_set (ent, c);\end{verbatim}

\subsubsection*{Definition}
\begin{verbatim}
void sf_eno_set (sf_eno ent, float* c /* data [n] */)
/*< Set the interpolation table. c can be changed or freed afterwords >*/
{
   ...
}
\end{verbatim}

\subsubsection*{Input parameters}
\begin{desclist}{\tt }{\quad}[\tt ent]
   \setlength\itemsep{0pt}
   \item[ent] an object for interpolation. It is of type \texttt{sf\_eno}.
   \item[c]   the data which is to be interpolated (\texttt{float*}).  
\end{desclist}




\subsection{{sf\_eno\_apply}}
Interpolates the data.

\subsubsection*{Call}
\begin{verbatim}sf_eno_apply (ent, i, x, f, f1, what);\end{verbatim}

\subsubsection*{Definition}
\begin{verbatim}
void sf_eno_apply (sf_eno ent, 
                int i     /* grid location */, 
                float x   /* offset from grid */, 
                float *f  /* output data value */, 
                float *f1 /* output derivative */, 
                der what  /* flag of what to compute */) 
/*< Apply interpolation >*/
{
  ...
}
\end{verbatim}

\subsubsection*{Input parameters}
\begin{desclist}{\tt }{\quad}[\tt ]
   \setlength\itemsep{0pt}
   \item[ent]  an object for interpolation. It is of type \texttt{sf\_eno}.
   \item[i]    location of the grid (\texttt{int}).
   \item[x]    offset from the grid (\texttt{float}).  
   \item[f]    output data value (\texttt{float*}).  
   \item[f1]   output derivative (\texttt{float*}).  
   \item[what] whether the function value or the derivative is required. Must be of type \texttt{der}.  
\end{desclist}

