\section{1-D interpolation (int1.c)}
\index{interpolation!1D}




\subsection{{sf\_int1\_init}}\label{sec:sf_int1_init}
Initializes the required variables and allocates the required space for 1D interpolation.

\subsubsection*{Call}
\begin{verbatim}sf_int1_init (coord, o1, d1, n1, interp, nf_in, nd_in);\end{verbatim}

\subsubsection*{Definition}
\begin{verbatim}
void  sf_int1_init (float* coord               /* cooordinates [nd] */, 
                 float o1, float d1, int n1 /* axis */, 
                 sf_interpolator interp     /* interpolation function */, 
                 int nf_in                  /* interpolator length */, 
                 int nd_in                  /* number of data points */)
/*< initialize >*/
{
   ...
}
\end{verbatim}

\subsubsection*{Input parameters}
\begin{desclist}{\tt }{\quad}[\tt coord]
   \setlength\itemsep{0pt}
   \item[coord]  coordinates (\texttt{float*}).  
   \item[o1]     origin of the axis (\texttt{float}).  
   \item[d1]     sampling of the axis (\texttt{float}).  
   \item[n1]     length of the axis (\texttt{float}).  
   \item[interp] interpolation function (\texttt{sf\_interpolator}).  
   \item[nf\_in] interpolator length (\texttt{int}).  
   \item[nd\_in] number of data points (\texttt{int}).  
\end{desclist}




\subsection{{sf\_int1\_lop}}
Applies the linear operator for interpolation.

\subsubsection*{Call}
\begin{verbatim}sf_int1_lop (adj, add, nm, ny, x, ord);\end{verbatim}

\subsubsection*{Definition}
\begin{verbatim}
void  sf_int1_lop (bool adj, bool add, int nm, int ny, float* x, float* ord)
/*< linear operator >*/
{ 
   ...
}
\end{verbatim}

\subsubsection*{Input parameters}
\begin{desclist}{\tt }{\quad}[\tt ord]
   \setlength\itemsep{0pt}
   \item[adj] a parameter to determine whether the output is \texttt{x} or \texttt{ord} (\texttt{bool}).
   \item[add] a parameter to determine whether the input needs to be zeroed (\texttt{bool}).
   \item[nm]  size of \texttt{x} (\texttt{int}).
   \item[ny]  size of \texttt{ord} (\texttt{int}).
   \item[x]   output or operator, depending on whether \texttt{adj} is true or false (\texttt{float*}).
   \item[ord] output or operator, depending on whether \texttt{adj} is true or false (\texttt{float*}).
\end{desclist}




\subsection{{sf\_cint1\_lop}}
Applies the complex \index{interpolation!complex data}linear operator for interpolation of complex data.

\subsubsection*{Call}
\begin{verbatim}sf_cint1_lop (adj, add, nm, ny, x, ord);\end{verbatim}

\subsubsection*{Definition}
\begin{verbatim}
void  sf_cint1_lop (bool adj, bool add, int nm, int ny, sf_complex* x, sf_comple
x* ord)
/*< linear operator for complex numbers >*/
{ 
   ...
}
\end{verbatim}

\subsubsection*{Input parameters}
\begin{desclist}{\tt }{\quad}[\tt ord]
   \setlength\itemsep{0pt}
   \item[adj] a parameter to determine whether the output is \texttt{x} or \texttt{ord} (\texttt{bool}).
   \item[add] a parameter to determine whether the input needs to be zeroed (\texttt{bool}).
   \item[nm]  size of \texttt{x} (\texttt{int}).
   \item[ny]  size of \texttt{ord} (\texttt{int}).
   \item[x]   output or operator, depending on whether \texttt{adj} is true or false (\texttt{sf\_complex*}).
   \item[ord] output or operator, depending on whether \texttt{adj} is true or false (\texttt{sf\_complex*}).
\end{desclist}




\subsection{{sf\_int1\_close}}
Frees the space allocated for 1D interpolation by \hyperref[sec:sf_int1_init]{\texttt{sf\_int1\_init}}.

\subsubsection*{Call}
\begin{verbatim}sf_int1_close ();\end{verbatim}

\subsubsection*{Definition}
\begin{verbatim}
void sf_int1_close (void)
/*< free allocated storage >*/
{
   ...
}
\end{verbatim}


