\section{Anisotropic diffusion, 2-D (impl2.c)}




\subsection{{sf\_impl2\_init}}\label{sec:sf_impl2_init}
Initializes the required variables and allocates the required space for the anisotropic diffusion.

\subsubsection*{Call}
\begin{verbatim}
void sf_impl2_init (r1,r2, n1_in,n2_in, tau, pclip, 
                    up_in, verb_in, dist_in, nsnap_in, snap_in);
\end{verbatim}

\subsubsection*{Definition}
\begin{verbatim}
void sf_impl2_init (float r1, float r2   /* radius */, 
                 int n1_in, int n2_in /* data size */, 
                 float tau            /* duration */, 
                 float pclip          /* percentage clip */, 
                 bool up_in           /* weighting case */,
                 bool verb_in         /* verbosity flag */,
                 float *dist_in       /* optional distance function */,
                 int nsnap_in         /* number of snapshots */,
                 sf_file snap_in      /* snapshot file */)
/*< initialize >*/
{
   ...
}
\end{verbatim}

\subsubsection*{Input parameters}
\begin{desclist}{\tt }{\quad}[\tt ]
   \setlength\itemsep{0pt}
   \item[r1]        radius in the first dimension (\texttt{float}).  
   \item[r2]        radius in the second dimension (\texttt{float}).  
   \item[n1\_in]    length of first dimension in the input data (\texttt{int}).  
   \item[n2\_in]    length of second dimension in the input data (\texttt{int}).  
   \item[tau]       duration (\texttt{float}).  
   \item[pclip]     percentage clip (\texttt{float}).  
   \item[up\_in]    weighting case (\texttt{bool}).  
   \item[verb\_in]  verbosity flag (\texttt{bool}).  
   \item[dist\_in]  optional distance function (\texttt{float*}).  
   \item[nsnap\_in] number of snapshots (\texttt{int}).  
   \item[snap\_in]  snapshot file (\texttt{sf\_file}).  
\end{desclist}




\subsection{{sf\_impl2\_close}}
Frees the space allocated by \hyperref[sec:sf_impl2_init]{\texttt{sf\_impl2\_init}}.

\subsubsection*{Definition}
\begin{verbatim}sf_impl2_close ();\end{verbatim}

\subsubsection*{Definition}
\begin{verbatim}
void sf_impl2_close (void)
/*< free allocated storage >*/
{
   ...
}
\end{verbatim}




\subsection{{sf\_impl2\_set}}
Computes the weighting function for the anisotropic diffusion.

\subsubsection*{Call}
\begin{verbatim}sf_impl2_set(x);\end{verbatim}

\subsubsection*{Definition}
\begin{verbatim}
void sf_impl2_set(float ** x)
/*< compute weighting function >*/
{
   ...
}
\end{verbatim}

\subsubsection*{Input parameters}
\begin{desclist}{\tt }{\quad}[\tt ]
   \setlength\itemsep{0pt}
   \item[x] data (\texttt{float**}).  
\end{desclist}




\subsection{{sf\_impl2\_set}}
Applies the anisotropic diffusion.

\subsubsection*{Call}
\begin{verbatim}sf_impl2_apply (x, set, adj);\end{verbatim}

\subsubsection*{Definition}
\begin{verbatim}
void sf_impl2_apply (float **x, bool set, bool adj)
/*< apply diffusion >*/
{
   ...  
}
\end{verbatim}

\subsubsection*{Input parameters}
\begin{desclist}{\tt }{\quad}[\tt set]
   \setlength\itemsep{0pt}
   \item[x]   data (\texttt{float*}).  
   \item[set] whether the weighting function needs to be computed (\texttt{bool}).  
   \item[adj] whether the weighting function needs to be applied (\texttt{bool}).  
\end{desclist}





\subsection{{sf\_impl2\_lop}}
Applies either \texttt{x} or \texttt{y} as linear operator to \texttt{y} or \texttt{x} and output \texttt{x} or \texttt{y}, depending on whether \texttt{adj} is true or false.

\subsubsection*{Call}
\begin{verbatim}sf_impl2_lop (adj, add, nx, ny, x, y);\end{verbatim}

\subsubsection*{Definition}
\begin{verbatim}
void sf_impl2_lop (bool adj, bool add, int nx, int ny, float* x, float* y)
/*< linear operator >*/
{
   ...
}
\end{verbatim}

\subsubsection*{Input parameters}
\begin{desclist}{\tt }{\quad}[\tt add]
   \setlength\itemsep{0pt}
   \item[adj] a parameter to determine whether the output is \texttt{y} or \texttt{x} (\texttt{bool}).
   \item[add] a parameter to determine whether the input needs to be zeroed (\texttt{bool}).
   \item[nx]  size of \texttt{x} (\texttt{int}).
   \item[ny]  size of \texttt{y} (\texttt{int}).
   \item[x]   data or operator, depending on whether \texttt{adj} is true or false (\texttt{sf\_complex*}).
   \item[y]   data or operator, depending on whether \texttt{adj} is true or false (\texttt{sf\_complex*}).
\end{desclist}



