\section{Pseudo-random numbers: uniform and normally distributed (randn.c)}




\subsection{{sf\_randn1}}
Generates a normally distributed random number using the Box-Muller method.

\subsubsection*{Call}
\begin{verbatim}vset = sf_randn_one_bm ();\end{verbatim}

\subsubsection*{Definition}
\begin{verbatim}
float sf_randn_one_bm (void)
/*< return a random number (normally distributed, Box-Muller method) >*/
{
   ...
}
\end{verbatim}

\subsubsection*{Output}
\begin{desclist}{\tt }{\quad}[\tt ]
   \setlength\itemsep{0pt}
   \item[vset] the random number. It is of type \texttt{float}.
\end{desclist}




\subsection{{sf\_randn}}
Fills an array with normally distributed random numbers.

\subsubsection*{Call}
\begin{verbatim}sf_randn (nr, r);\end{verbatim}

\subsubsection*{Definition}
\begin{verbatim}
void sf_randn (int nr, float *r /* [nr] */)
/*< fill an array with normally distributed numbers >*/
{
   ...
}
\end{verbatim}

\subsubsection*{Input parameters}
\begin{desclist}{\tt }{\quad}[\tt ]
   \setlength\itemsep{0pt}
   \item[nr] size of the array where the random numbers are to be stored (\texttt{int}). 
   \item[r] the array where the random numbers are to be stored (\texttt{float*}).
\end{desclist}




\subsection{{sf\_random}}
Fills an array with uniformly distributed random numbers.

\subsubsection*{Call}
\begin{verbatim}sf_random (nr, r)\end{verbatim}

\subsubsection*{Definition}
\begin{verbatim}
void sf_random (int nr, float *r /* [nr] */)
/*< fill an array with uniformly distributed numbers >*/
{
   ...
}
\end{verbatim}

\subsubsection*{Input parameters}
\begin{desclist}{\tt }{\quad}[\tt ]
   \setlength\itemsep{0pt}
   \item[nr] size of the array where the random numbers are to be stored (\texttt{int}). 
   \item[nr] the array where the random numbers are to be stored (\texttt{float*}).
\end{desclist}



