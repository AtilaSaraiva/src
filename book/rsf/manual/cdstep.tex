\section{Conjugate-direction iteration (cdstep.c)}
\index{conjugate direction method!real data}




\subsection{{sf\_cdstep\_init}}\label{sec:sf_cdstep_init}
Creates a list for internal storage.

\subsubsection*{Call}
\begin{verbatim}sf_cdstep_init();\end{verbatim}

\subsubsection*{Definition}
\begin{verbatim}
void sf_cdstep_init(void) 
/*< initialize internal storage >*/
{
   ...
}
\end{verbatim}




\subsection{{sf\_cdstep\_close}}
Frees the space allocated for internal storage by \hyperref[sec:sf_cdstep_init]{\texttt{sf\_cdstep\_init}}.

\subsubsection*{Call}
\begin{verbatim}sf_cdstep_close();\end{verbatim}

\subsubsection*{Definition}
\begin{verbatim}
void sf_cdstep_close(void) 
/*< free internal storage >*/
{
   ...
}
\end{verbatim}




\subsection{{sf\_cdstep}}\label{sec:sf_cdstep}
Calculates one step for the conjugate direction iteration, that is, it calculates the new conjugate gradient for the new line search direction.

\subsubsection*{Call}
\begin{verbatim}sf_cdstep(forget, nx, ny, x, g, rr, gg);\end{verbatim}

\subsubsection*{Definition}
\begin{verbatim}
void sf_cdstep(bool forget     /* restart flag */, 
               int nx          /* model size */, 
               int ny          /* data size */, 
               float* x        /* current model [nx] */, 
               const float* g  /* gradient [nx] */, 
               float* rr       /* data residual [ny] */, 
               const float* gg /* conjugate gradient [ny] */) 
/*< Step of conjugate-direction iteration. 
  The data residual is rr = A x - dat
>*/
{
   ...
}
\end{verbatim}

\subsubsection*{Input parameters}
\begin{desclist}{\tt }{\quad}[\tt forget]
   \setlength\itemsep{0pt}
   \item[forget] restart flag (\texttt{bool}).  
   \item[nx]     model size (\texttt{int}).  
   \item[ny]     data size (\texttt{int}).  
   \item[x]      current model (\texttt{float*}).  
   \item[g]      gradient (\texttt{const float*}).  
   \item[rr]     data residual (\texttt{float*}).
   \item[gg]     conjugate gradient (\texttt{const float*}).  
\end{desclist}




\subsection{{sf\_cdstep\_diag}}
Calculates the diagonal of the model resolution matrix.

\subsubsection*{Call}
\begin{verbatim}sf_cdstep_diag(nx, res);\end{verbatim}

\subsubsection*{Definition}
\begin{verbatim}
void sf_cdstep_diag(int nx, float *res /* [nx] */)
/*< compute diagonal of the model resolution matrix >*/
{
   ...
}
\end{verbatim}

\subsubsection*{Input parameters}
\begin{desclist}{\tt }{\quad}[\tt res]
   \setlength\itemsep{0pt}
   \item[nx]  model size (\texttt{int}).  
   \item[res] diagonal entries of the model resolution matrix (\texttt{float*}).  
\end{desclist}




\subsection{{sf\_cdstep\_mat}}
Calculates the complete model resolution matrix.

\subsubsection*{Call}
\begin{verbatim}sf_cdstep_mat (nx, res);\end{verbatim}

\subsubsection*{Definition}
\begin{verbatim}
void sf_cdstep_mat (int nx, float **res /* [nx][nx] */)
/*< compute complete model resolution matrix >*/
{
   ...
}
\end{verbatim}

\subsubsection*{Input parameters}
\begin{desclist}{\tt }{\quad}[\tt ]
   \setlength\itemsep{0pt}
   \item[nx]  model size (\texttt{int}).  
   \item[res] diagonal entries of the model resolution matrix (\texttt{float**}).  
\end{desclist}

