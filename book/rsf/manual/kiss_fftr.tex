\section{FFT (kiss\_fftr.c)}

\subsection{{sf\_kiss\_fftr\_alloc}}
Allocates the memory for the FFT and returns an object of type \texttt{kiss\_fftr\_cfg}.

\subsubsection*{Call}
\begin{verbatim}kiss_fftr_alloc(nfft, inverse_fft, mem, lenmem);\end{verbatim}

\subsubsection*{Definition}
\begin{verbatim}
kiss_fftr_cfg kiss_fftr_alloc(int nfft,int inverse_fft,void * mem,size_t * lenmem)
{
   ...
}
\end{verbatim}

\subsubsection*{Input parameters}
\begin{desclist}{\tt}{\quad}[inversewfft]
   \setlength\itemsep{0pt}
   \item[nfft]         length of the forward FFT (\texttt{int}).  
   \item[inverse\_fft] length of the inverse FFT (\texttt{int}).  
   \item[mem]          pointer to the memory allocated for FFT (\texttt{void*}).
   \item[lenmem]       size of the allocated memory (\texttt{size\_t}).
\end{desclist}

\subsubsection*{Input parameters}
\begin{desclist}{\tt}{\quad}[]
   \item[st] an object for FFT (\texttt{kiss\_fftr\_cfg}).
\end{desclist}
 



\subsection{{kiss\_fftr}}
Performs the forward FFT on the input \texttt{timedata} which is real, and stores the transformed complex \texttt{freqdata} in the location specified in the input.

\subsubsection*{Call}
\begin{verbatim}kiss_fftr( st, timedata, freqdata)\end{verbatim}

\subsubsection*{Definition}
\begin{verbatim}
void kiss_fftr(kiss_fftr_cfg st,const kiss_fft_scalar *timedata,kiss_fft_cpx *fr
eqdata)
{
    /* input buffer timedata is stored row-wise */

    /* The real part of the DC element of the frequency spectrum in st->tmpbuf
     * contains the sum of the even-numbered elements of the input time sequence
     * The imag part is the sum of the odd-numbered elements
     *
     * The sum of tdc.r and tdc.i is the sum of the input time sequence. 
     *      yielding DC of input time sequence
     * The difference of tdc.r - tdc.i is the sum of the input (dot product) [1,-1,1,-1... 
     *      yielding Nyquist bin of input time sequence
     */
 
   ...
}
\end{verbatim}

\subsubsection*{Input parameters}
\begin{desclist}{\tt}{\quad}[freqdata]
   \setlength\itemsep{0pt}
   \item[st]       an object for the forward FFT (\texttt{kiss\_fftr\_cfg}).  
   \item[timedata] time data which is to be transformed (\texttt{const kiss\_fft\_scalar*}).  
   \item[freqdata] location where the transformed data is to be stored (\texttt{kiss\_fft\_cpx*}).
\end{desclist}



\subsection{{kiss\_fftri}}
Performs the inverse FFT on the input \texttt{timedata} which is real, and stores the transformed complex \texttt{freqdata} in the location specified in the input.

\subsubsection*{Call}
\begin{verbatim}kiss_fftri(st, freqdata, timedata);\end{verbatim}

\subsubsection*{Definition}
\begin{verbatim}
void kiss_fftri(kiss_fftr_cfg st,const kiss_fft_cpx *freqdata,kiss_fft_scalar *t
imedata)
/* input buffer timedata is stored row-wise */
{
   ...
}
\end{verbatim}

\subsubsection*{Input parameters}
\begin{desclist}{\tt}{\quad}[freqdata]
   \setlength\itemsep{0pt}
   \item[st]       an object for the inverse FFT (\texttt{kiss\_fftr\_cfg}).  
   \item[timedata] location where the inverse time data is to be stored (\texttt{const kiss\_fft\_scalar*}).  
   \item[freqdata] complex frequency data which is to be inverse transformed (\texttt{kiss\_fft\_cpx*}).
\end{desclist}

