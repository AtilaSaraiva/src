\section{2-D triangle smoothing as a linear operator (triangle2.c)}




\subsection{{sf\_triangle2\_init}}
Initializes the triangle filter.

\subsubsection*{Call}
\begin{verbatim}sf_triangle2_init (nbox1,nbox2, ndat1,ndat2, nrep);\end{verbatim}

\subsubsection*{Definition}
\begin{verbatim}
void sf_triangle2_init (int nbox1, int nbox2 /* triangle size */, 
                        int ndat1, int ndat2 /* data size */,
                        int nrep /* repeat smoothing */)
/*< initialize >*/
{
   ...
}
\end{verbatim}

\subsubsection*{Input parameters}
\begin{desclist}{\tt }{\quad}[\tt inbox2]
   \setlength\itemsep{0pt}
   \item[inbox1] size of the triangle filter (\texttt{int}). 
   \item[inbox2] size of the second triangle filter (\texttt{int}). 
   \item[ndat1]  size of the data (\texttt{int}). 
   \item[ndat2]  size of the second data set (\texttt{int}). 
   \item[nrep]   number of times the smoothing is to be repeated (\texttt{int}).
\end{desclist}




\subsection{{sf\_triangle2\_lop}}
Applies the triangle smoothing to one of the input data and applies the smoothed data to the unsmoothed one as a linear operator. This is just like \hyperref[sec:sf_triangle1_lop]{\texttt{sf\_triangle1\_lop}} but with two triangle filters instead of one.

\subsubsection*{Call}
\begin{verbatim}sf_triangle2_lop (adj, add, nx, ny, x, y);\end{verbatim}

\subsubsection*{Definition}
\begin{verbatim}
void sf_triangle2_lop (bool adj, bool add, int nx, int ny, float* x, float* y)
/*< linear operator >*/
{
   ...
}
\end{verbatim}

\subsubsection*{Input parameters}
\begin{desclist}{\tt }{\quad}[\tt add]
   \setlength\itemsep{0pt}
   \item[adj] a parameter to determine whether weights are applied to \texttt{yy} or \texttt{xx} (\texttt{bool}). 
   \item[add] a parameter to determine whether the input needs to be zeroed (\texttt{bool}).
   \item[nx]  size of \texttt{x} (\texttt{int}). 
   \item[ny]  size of \texttt{y} (\texttt{int}).
   \item[x]   data or operator, depending on whether \texttt{adj} is true or false (\texttt{float}).
   \item[y]   data or operator, depending on whether \texttt{adj} is true or false (\texttt{float}).
\end{desclist}

\subsubsection*{Output}
\begin{desclist}{\tt }{\quad}[\tt ]
   \setlength\itemsep{0pt}
   \item[\texttt{x} or \texttt{y}] the output depending on whether \texttt{adj} is true or false (\texttt{float}).
\end{desclist}




\subsection{{sf\_triangle2\_close}}
Frees the space allocated for the triangle smoothing filters.

\subsubsection*{Call}
\begin{verbatim}sf_triangle2_close();\end{verbatim}

\subsubsection*{Definition}
\begin{verbatim}
void sf_triangle2_close(void)
/*< free allocated storage >*/
{
   ...
}
\end{verbatim}





