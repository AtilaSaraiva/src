\section{Construction of points (point.c)}\label{sec:point.c}




\subsection{{printpt2d}}
Prints the value and location of a 2D point (position vector).

\subsubsection*{Call}
\begin{verbatim}printpt2d(pt2d P);\end{verbatim}

\subsubsection*{Definition}
\begin{verbatim}
void printpt2d(pt2d P)
/*< print point2d info  >*/
{
   ...
}
\end{verbatim}

\subsubsection*{Input parameters}
\begin{desclist}{\tt }{\quad}[\tt ]
   \setlength\itemsep{0pt}
   \item[P] a point (position vector) (\texttt{pt3d}).     
\end{desclist}




\subsection{{printpt3d}}
Prints the value and location of a 3D point (position vector).

\subsubsection*{Call}
\begin{verbatim}printpt3d(P);\end{verbatim}

\subsubsection*{Definition}
\begin{verbatim}
void printpt3d(pt3d P)
/*< print point3d info  >*/
{
   ...
}
\end{verbatim}

\subsubsection*{Input parameters}
\begin{desclist}{\tt }{\quad}[\tt ]
   \setlength\itemsep{0pt}
   \item[P] a point (position vector) (\texttt{pt3d}).     
\end{desclist}




\subsection{{pt2dwrite1}}
Outputs a 1D array of 2D points to a file. It can be used to define the source or receiver arrays, for example.

\subsubsection*{Call}
\begin{verbatim}pt2dwrite1(F, v, n1, k);\end{verbatim}

\subsubsection*{Definition}
\begin{verbatim}
void pt2dwrite1(sf_file F, pt2d *v, size_t n1, int k)
/*< output point2d 1-D vector >*/
{
   ...    
}
\end{verbatim}

\subsubsection*{Input parameters}
\begin{desclist}{\tt }{\quad}[\tt File]
   \setlength\itemsep{0pt}
   \item[File] a file to which the 1D array of 2D points is to be output (\texttt{sf\_file}). 
   \item[v]    an array of 2D points which is to be output (\texttt{pt2d}). 
   \item[n1]   size of the array (\texttt{size\_t}). 
   \item[k]    a number, which if equal to 3, indicates that the value of the 2D points must also be included (\texttt{int}).     
\end{desclist}




\subsection{{pt2dwrite2}}
Outputs a 2D array of 2D points to a file. It can be used to define the source or receiver arrays, for example.

\subsubsection*{Call}
\begin{verbatim}pt2dwrite2(F, v, n1, n2, k);\end{verbatim}

\subsubsection*{Definition}
\begin{verbatim}
void pt2dwrite2(sf_file F, pt2d **v, size_t n1, size_t n2, int k)
/*< output point2d 2-D vector >*/
{
   ...
}
\end{verbatim}

\subsubsection*{Input parameters}
\begin{desclist}{\tt }{\quad}[\tt File]
   \setlength\itemsep{0pt}
   \item[File] a file to which the 2D array of 2D points is to be output (\texttt{sf\_file}). 
   \item[v]    an array of 2D points which is to be output (\texttt{pt2d}). 
   \item[n1]   size of one axis of the 2D array (\texttt{size\_t}).
   \item[n2]	   size of the other axis of the 2D array (\texttt{size\_t}). 
   \item[k]    a number, which if equal to 3, indicates that the value of the 2D points must also be included (\texttt{int}).     
\end{desclist}




\subsection{{pt3dwrite1}}
Outputs a 1D array of 3D points to a file. It can be used to define the source or receiver arrays, for example.

\subsubsection*{Call}
\begin{verbatim}pt3dwrite1(F, v, n1, k);\end{verbatim}

\subsubsection*{Definition}
\begin{verbatim}
void pt3dwrite1(sf_file F, pt3d *v, size_t n1, int k)
/*< output point3d 1-D vector >*/
{
   ...
}
\end{verbatim}

\subsubsection*{Input parameters}
\begin{desclist}{\tt }{\quad}[\tt File]
   \setlength\itemsep{0pt}
   \item[File] a file to which the 1D array of 3D points is to be output (\texttt{sf\_file}). 
   \item[v]    an array of 1D points which is to be output (\texttt{pt3d}). 
   \item[n1]   size of one axis of the 3D array (\texttt{size\_t}). 
   \item[k]    a number, which if equal to 4, indicates that the value of the 3D points must also be included (\texttt{int}).     
\end{desclist}




\subsection{{pt3dwrite2}}
Outputs a 2D array of 3D points to a file. It can be used to define the source or receiver arrays, for example.

\subsubsection*{Call}
\begin{verbatim}pt3dwrite2(F, v, n1, n2, k);\end{verbatim}

\subsubsection*{Definition}
\begin{verbatim}
void pt3dwrite2(sf_file F, pt3d *v, size_t n1, size_t n2, int k)
/*< output point3d 2-D vector >*/
{
   ... 
}
\end{verbatim}

\subsubsection*{Input parameters}
\begin{desclist}{\tt }{\quad}[\tt File]
   \setlength\itemsep{0pt}
   \item[File] a file to which the 2D array of 3D points is to be output (\texttt{sf\_file}). 
   \item[v]    an array of 2D points which is to be output (\texttt{pt3d}). 
   \item[n1]   size of one axis of the 2D array (\texttt{size\_t}).
   \item[n2]	   size of the other axis of the 2D array (\texttt{size\_t}). 
   \item[k]    a number, which if equal to 4, indicates that the value of the 3D points must also be included (\texttt{int}).     
\end{desclist}




\subsection{{pt2dread1}}
Reads a 1D array of 2D points from a file. It can be used to define the source or receiver arrays, for example.

\subsubsection*{Call}
\begin{verbatim}pt2dread1(F, v, n1, k);\end{verbatim}

\subsubsection*{Definition}
\begin{verbatim}
void pt2dread1(sf_file F, pt2d *v, size_t n1, int k)
/*< input point2d 1-D vector >*/
{
   ...
}
\end{verbatim}

\subsubsection*{Input parameters}
\begin{desclist}{\tt }{\quad}[\tt File]
   \setlength\itemsep{0pt}
   \item[File] a file from which the 1D array of 2D points is to be read (\texttt{sf\_file}). 
   \item[v]    an array of 2D points which is to be read (\texttt{pt2d}). 
   \item[n1]   size of the array (\texttt{size\_t}). 
   \item[k]    a number, which if equal to 3, indicates that the value of the 2D points must also be included (\texttt{int}).     
\end{desclist}




\subsection{{pt2dread2}}
Reads a 2D array of 2D points from a file. It can be used to define the source or receiver arrays, for example.

\subsubsection*{Call}
\begin{verbatim}pt2dread2(F, v, n1, n2, k);\end{verbatim}

\subsubsection*{Definition}
\begin{verbatim}
void pt2dread1(sf_file F, pt2d *v, size_t n1, int k)
/*< input point2d 1-D vector >*/
{
   ...
}
\end{verbatim}   

\subsubsection*{Input parameters}
\begin{desclist}{\tt }{\quad}[\tt File]
   \setlength\itemsep{0pt}
   \item[File] a file from which the 2D array of 2D points is to be read (\texttt{sf\_file}). 
   \item[v]    an array of 2D points which is to be read (\texttt{pt2d}). 
   \item[n1]   size of one axis of the 2D array (\texttt{size\_t}).
   \item[n2]   size of the other axis of the 2D array (\texttt{size\_t}). 
   \item[k]    a number, which if equal to 3, indicates that the value of the 2D points must also be included (\texttt{int}).     
\end{desclist}




\subsection{{pt3dread1}}
Reads a 1D array of 3D points from a file. It can be used to define the source or receiver arrays, for example.

\subsubsection*{Call}
\begin{verbatim}pt3dread1(F, v, n1, k);\end{verbatim}

\subsubsection*{Definition}
\begin{verbatim}
void pt3dread1(sf_file F, pt3d *v, size_t n1, int k)
/*< input point3d 1-D vector >*/
{
   ...
}
\end{verbatim}

\subsubsection*{Input parameters}
\begin{desclist}{\tt }{\quad}[\tt File]
   \setlength\itemsep{0pt}
   \item[File] a file from which the 1D array of 3D points is to be read (\texttt{sf\_file}). 
   \item[v]    an array of 1D points which is to be read (\texttt{pt3d}). 
   \item[n1]   size of one axis of the 3D array (\texttt{size\_t}). 
   \item[k]    a number, which if equal to 4, indicates that the value of the 3D points must also be included (\texttt{int}).     
\end{desclist}




\subsection{{pt3dread2}}
Reads a 2D array of 3D points from a file. It can be used to define the source or receiver arrays, for example.

\subsubsection*{Call}
\begin{verbatim}pt3dread2(F, v, n1, n2, k);\end{verbatim}

\subsubsection*{Definition}
\begin{verbatim}
void pt3dread2(sf_file F, pt3d **v, size_t n1, size_t n2, int k)
/*< input point3d 2-D vector >*/
{
    ...
}
\end{verbatim}

\subsubsection*{Input parameters}
\begin{desclist}{\tt }{\quad}[\tt File]
   \setlength\itemsep{0pt}
   \item[File] a file from which the 2D array of 3D points is to be read (\texttt{sf\_file}). 
   \item[v]    an array of 2D points which is to be read (\texttt{pt3d}). 
   \item[n1]   size of one axis of the 2D array (\texttt{size\_t}).
   \item[n2]   size of the other axis of the 2D array (\texttt{size\_t}). 
   \item[k]    a number, which if equal to 4, indicates that the value of the 3D points must also be included (\texttt{int}).     
\end{desclist}




\subsection{{pt2dalloc1}}
Allocates memory for 1D array of 2D points.

\subsubsection*{Call}
\begin{verbatim}ptr = pt2dalloc1(n1);\end{verbatim}

\subsubsection*{Definition}
\begin{verbatim}
pt2d* pt2dalloc1( size_t n1)
/*< alloc point2d 1-D vector >*/
{
   ...
}
\end{verbatim}

\subsubsection*{Input parameters}
\begin{desclist}{\tt }{\quad}[\tt ]
   \setlength\itemsep{0pt}
   \item[n1]	size of the array (\texttt{size\_t}).
\end{desclist}

\subsubsection*{Output}
\begin{desclist}{\tt }{\quad}[\tt ]
   \setlength\itemsep{0pt}
   \item[ptr] pointer to memory (\texttt{pt2d*}).
\end{desclist}




\subsection{{pt2dalloc2}}
Allocates memory for 2D array of 2D points.

\subsubsection*{Call}
\begin{verbatim}ptr = pt2dalloc2(n1,n2);\end{verbatim}

\subsubsection*{Definition}
\begin{verbatim}
pt2d** pt2dalloc2( size_t n1, size_t n2)
/*< alloc point2d 2-D vector >*/
{
   ...
}
\end{verbatim}

\subsubsection*{Input parameters}
\begin{desclist}{\tt }{\quad}[\tt n2]
   \setlength\itemsep{0pt}
   \item[n1]	size one axis of the 2D array (\texttt{size\_t}).
   \item[n2]	size of the other axis of the 2D array (\texttt{size\_t}).
\end{desclist}

\subsubsection*{Output}
\begin{desclist}{\tt }{\quad}[\tt ptr]
   \setlength\itemsep{0pt}
   \item[ptr] pointer to memory (\texttt{pt2d**}).
\end{desclist}





\subsection{{pt2dalloc3}}
Allocates memory for 3D array of 2D points.

\subsubsection*{Call}
\begin{verbatim}
pt2d*** pt2dalloc3(n1, n2, n3);\end{verbatim}

\subsubsection*{Definition}
\begin{verbatim}
pt2d*** pt2dalloc3(size_t n1, size_t n2, size_t n3)
/*< alloc point2d 3-D vector >*/
{
    ...
}
\end{verbatim}

\subsubsection*{Input parameters}
\begin{desclist}{\tt }{\quad}[\tt n3]
   \setlength\itemsep{0pt}
   \item[n1] size of first axis of the 3D array (\texttt{size\_t}).
   \item[n2] size of the second axis of the 3D array (\texttt{size\_t}).
   \item[n3] size of the third axis of the 3D array (\texttt{size\_t}).
\end{desclist}

\subsubsection*{Output}
\begin{desclist}{\tt }{\quad}[\tt ptr]
   \setlength\itemsep{0pt}
   \item[ptr] pointer to memory (\texttt{pt2d***}).
\end{desclist}




\subsection{{pt3dalloc1}}
Allocates memory for 1D array of 3D points.

\subsubsection*{Call}
\begin{verbatim}ptr = pt3dalloc1(n1);\end{verbatim}

\subsubsection*{Definition}
\begin{verbatim}
pt3d* pt3dalloc1( size_t n1)
/*< alloc point3d 1-D vector >*/
{
   ...
}
\end{verbatim}

\subsubsection*{Input parameters}
\begin{desclist}{\tt }{\quad}[\tt ]
   \setlength\itemsep{0pt}
   \item[n1]	size of the array (\texttt{size\_t}).
\end{desclist}

\subsubsection*{Output}
\begin{desclist}{\tt }{\quad}[\tt ptr]
   \setlength\itemsep{0pt}
   \item[ptr] pointer to memory (\texttt{pt3d*}).
\end{desclist}




\subsection{{pt3dalloc2}}
Allocates memory for 2D array of 3D points.

\subsubsection*{Call}
\begin{verbatim}
ptr = pt3dalloc2(n1,n2);\end{verbatim}

\subsubsection*{Definition}
\begin{verbatim}
pt3d** pt3dalloc2( size_t n1, size_t n2)
/*< alloc point3d 2-D vector >*/
{
   ...   
}
\end{verbatim}

\subsubsection*{Input parameters}
\begin{desclist}{\tt }{\quad}[\tt n2]
   \setlength\itemsep{0pt}
   \item[n1]	size of one axis of the 2D array (\texttt{size\_t}).
   \item[n2]	size of the other axis of the 2D array (\texttt{size\_t}).
\end{desclist}

\subsubsection*{Output}
\begin{desclist}{\tt }{\quad}[\tt ptr]
   \setlength\itemsep{0pt}
   \item[ptr] pointer to memory (\texttt{pt3d**}).
\end{desclist}





\subsection{{pt3dalloc3}}
Allocates memory for 3D array of 3D points.

\subsubsection*{Call}
\begin{verbatim}ptr =  pt3dalloc3(n1, n2, n3); \end{verbatim}

\subsubsection*{Definition}
\begin{verbatim}
pt3d*** pt3dalloc3(size_t n1, size_t n2, size_t n3)
/*< alloc point3d 3-D vector >*/
{
   ...
}
\end{verbatim}

\subsubsection*{Input parameters}
\begin{desclist}{\tt }{\quad}[\tt n3]
   \setlength\itemsep{0pt}
   \item[n1]	size of first axis of the 3D array (\texttt{size\_t}).
   \item[n2]	size of the second axis of the 3D array (\texttt{size\_t}).
   \item[n3]	size of the third axis of the 3D array (\texttt{size\_t}).
\end{desclist}

\subsubsection*{Output}
\begin{desclist}{\tt }{\quad}[\tt ptr]
   \setlength\itemsep{0pt}
   \item[ptr] pointer to memory (\texttt{pt3d***}).
\end{desclist}



