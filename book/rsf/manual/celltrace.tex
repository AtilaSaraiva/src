\section{Cell ray tracing (celltrace.c)}




\subsection{{sf\_celltrace\_init}}
Initializes the object \texttt{sf\_celltrace} for ray tracing by initializing the required variables and allocating the required space.

\subsubsection*{Call}
\begin{verbatim}ct = sf_celltrace_init (order, nt, nz, nx, dz, dx, z0, x0, slow);\end{verbatim}

\subsubsection*{Definition}
\begin{verbatim}
sf_celltrace sf_celltrace_init (int order   /* interpolation accuracy */, 
                                int nt      /* maximum time steps */,
                                int nz      /* depth samples */, 
                                int nx      /* lateral samples */, 
                                float dz    /* depth sampling */, 
                                float dx    /* lateral sampling */, 
                                float z0    /* depth origin */, 
                                float x0    /* lateral origin */, 
                                float* slow /* slowness [nz*nx] */)
/*< Initialize ray tracing object >*/
{
  ...
} 
\end{verbatim}

\subsubsection*{Input parameters}
\begin{desclist}{\tt }{\quad}[\tt order]
   \setlength\itemsep{0pt}
   \item[order] accuracy of the interpolation (\texttt{int}). 
   \item[nt]    maximum number of time steps (\texttt{int}). 
   \item[nz]    number of depth samples (\texttt{int}). 
   \item[nx]    number of lateral samples (\texttt{int}). 
   \item[dz]    depth sampling interval (\texttt{float}). 
   \item[dx]    lateral sampling interval (\texttt{float}). 
   \item[z0]    depth origin (\texttt{float}). 
   \item[x0]    lateral origin (\texttt{float}). 
   \item[slow]  slowness (\texttt{float*}).  
\end{desclist}

\subsubsection*{Output}
\begin{desclist}{\tt }{\quad}[\tt ]
   \setlength\itemsep{0pt}
   \item[ct] the ray tracing object. It is of type \texttt{sf\_celltrace}.
\end{desclist}




\subsection{{sf\_celltrace\_close}}
Frees the space allocated for the \texttt{sf\_celltrace} object by \texttt{sf\_celltrace\_init}.

\subsubsection*{Call}
\begin{verbatim}sf_celltrace_close (ct);\end{verbatim}

\subsubsection*{Definition}
\begin{verbatim}
void sf_celltrace_close (sf_celltrace ct)
/*< Free allocated storage >*/
{
   ...
}
\end{verbatim}




\subsection{{sf\_cell\_trace}}
Traces the ray with the ray parameter specified in the input.

\subsubsection*{Call}
\begin{verbatim}t = sf_cell_trace (ct, xp, p, it, traj);\end{verbatim}

\subsubsection*{Definition}
\begin{verbatim}
float sf_cell_trace (sf_celltrace ct, 
                     float* xp    /* position */, 
                     float* p     /* ray parameter */, 
                     int* it      /* steps till boundary */, 
                     float** traj /* trajectory */)
/*< ray trace >*/
{
   ...
}
\end{verbatim}

\subsubsection*{Input parameters}
\begin{desclist}{\tt }{\quad}[\tt ]
   \setlength\itemsep{0pt}
   \item[ct]   the ray tracing object. It is of type \texttt{sf\_celltrace}. 
   \item[xp]   position (\texttt{float*}).  
   \item[p]    ray parameters (\texttt{float*}).  
   \item[it]   number steps till the boundary (\texttt{int*}).  
   \item[traj] trajectory of the ray (\texttt{float**}).  
\end{desclist}

\subsubsection*{Output}
\begin{desclist}{\tt }{\quad}[\tt ]
   \setlength\itemsep{0pt}
   \item[t] the travel time obtained by the ray tracing. It is of type \texttt{float}.
\end{desclist}


