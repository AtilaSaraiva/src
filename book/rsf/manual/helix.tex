\section{Helical filter definition and allocation (helix.c)}




\subsection{{sf\_allocatehelix}}\label{sec:sf_allocatehelix}
Initializes the filter.

\subsubsection*{Call}
\begin{verbatim}aa = sf_allocatehelix(nh);\end{verbatim}

\subsubsection*{Definition}
\begin{verbatim}
sf_filter sf_allocatehelix( int nh) 
/*< allocation >*/
{
   ...
}
\end{verbatim}

\subsubsection*{Input parameters}
\begin{desclist}{\tt }{\quad}[\tt ]
   \setlength\itemsep{0pt}
   \item[nh] filter length (\texttt{int}).  
\end{desclist}

\subsubsection*{Output}
\begin{desclist}{\tt }{\quad}[\tt ]
   \setlength\itemsep{0pt}  
   \item[aa] object for helix filter. It is of type \texttt{sf\_filter}.
\end{desclist}




\subsection{{sf\_deallocatehelix}}
Frees the space allocated by \hyperref[sec:sf_allocatehelix]{\texttt{sf\_allocatehelix}} for the filter.

\subsubsection*{Definition}
\begin{verbatim}sf_deallocatehelix (aa);\end{verbatim}

\subsubsection*{Call}
\begin{verbatim}
void sf_deallocatehelix( sf_filter aa) 
/*< deallocation >*/
{
   ...
}
\end{verbatim}

\subsubsection*{Input parameters}
\begin{desclist}{\tt }{\quad}[\tt ]
   \setlength\itemsep{0pt}
   \item[aa] the filter (\texttt{sf\_filter}).  
\end{desclist}




\subsection{{sf\_displayhelix}}
Displays the filter.

\subsubsection*{Call}
\begin{verbatim}sf_displayhelix(aa);\end{verbatim}

\subsubsection*{Definition}
\begin{verbatim}
void sf_displayhelix (sf_filter aa)
/*< display filter >*/
{
   ...
}
\end{verbatim}

\subsubsection*{Input parameters}
\begin{desclist}{\tt }{\quad}[\tt ]
   \setlength\itemsep{0pt}
   \item[aa] the filter (\texttt{sf\_filter}).  
\end{desclist}

