\section{Data types}
This chapter contains the descriptions of the data types used in the RSF API.



\subsection{Complex numbers and FFT}
This section lists the data types for the complex numbers and FFT.

\subsubsection*{kiss\_fft\_scalar}
This is a data type, which defines a scalar real value for the data type \texttt{kiss\_fft\_cpx} for complex numbers. It can be either of type \texttt{short} or \texttt{float}. Default is \texttt{float}.

\subsubsection*{kiss\_fft\_cpx}
This is a data type (a C structure), which defines a complex number. It has the real and imaginary parts of the complex numbers defined to be of type \texttt{kiss\_fft\_scalar}.

\subsubsection*{kiss\_fft\_cfg}
This is an object of type \texttt{kiss\_fft\_state} (which is a C data structure). 

\subsubsection*{kiss\_fft\_state}
The \texttt{kiss\_fft\_state} is a data type which defines the required variables for the Fourier transform and allocates the required space. For example the variable �inverse� of type int indicates whether the transform needs to be an inverse or forward. 

\subsubsection*{kiss\_fftr\_cfg}
This is an object of type \texttt{kiss\_fftr\_state} (which is a C data structure). 

\subsubsection*{kiss\_fftr\_state}
The \texttt{kiss\_fftr\_state} is a data type which defines the required variables for the Fourier transform and allocates the required space. This has the same purpose as \texttt{kiss\_fft\_state} but for the Fourier transform of the real signals. 

\subsubsection*{sf\_complex}
This is an object of type \texttt{kiss\_fft\_cfg} (which is an object of  C data structure). 

\subsubsection*{sf\_double\_complex}
This is a C data structure for complex numbers. It uses the type double for the real and imaginary parts of the complex numbers.



\subsection{Files}
This section lists the data types used to define the \texttt{.rsf} file structure.

\subsubsection*{sf\_file}
This is an object of type \texttt{sf\_File}. \texttt{sf\_File} is a data structure which defines the variables required for creating a \texttt{.rsf} file in Madagascar. It is defined in \hyperref[sec:file.c]{\texttt{file.c}}.  

\subsubsection*{sf\_datatype}
This is a C enumeration, which means that it contains new data types, which are not the fundamental types like int, float, \texttt{sf\_file} etc.  This data type is used in \texttt{sf\_File} data structure to set the type of a \texttt{.rsf} file, for example SF\_CHAR, SF\_INT etc. It is defined in \hyperref[sec:file.c]{\texttt{file.c}}.

\subsubsection*{sf\_dataform}
This is a C enumeration, which means that it contains new data types, which are not the fundamental types like int, float, \texttt{sf\_file} etc.  This data type is used in \texttt{sf\_File} data structure to set the format of an \texttt{.rsf} file, for example \texttt{SF\_ASCII}, \texttt{SF\_XDR} and \texttt{SF\_NATIVE}. It is defined in \hyperref[sec:file.c]{\texttt{file.c}}.



\subsection{Operators}
This section lists the data types used to define linear operators.

\subsubsection*{sf\_triangle}
This is an object of an abstract C datatype type \texttt{sf\_Triangle}. The \texttt{sf\_triangle} data type defines the variables of relevant types to store information about the triangle smoothing filter. It is defined in \hyperref[sec:triangle.c]{\texttt{triangle.c}}.

\subsubsection*{sf\_operator}\label{sec:sf_operator}
This is a C data type of type void. It is also a pointer to a function which takes the input parameters precisely as \texttt{(bool, bool, int, int, float*, float*)}. It is defined in \texttt{\_solver.h}.

\subsubsection*{sf\_solverstep}\label{sec:sf_solverstep}
This is a C data type of type void. It is also a pointer to a function which takes the input parameters precisely as \texttt{(bool, bool, int, int, float*, const float*, float*, const float*)}. It is defined in \texttt{\_solver.h}.

\subsubsection*{sf\_weight}\label{sec:sf_weight}
This is a C data type of type void. It is also a pointer to a function which takes the input parameters precisely as \texttt{(int, int, const sf\_complex*, float*)}. It is defined in \texttt{\_solver.h}.

\subsubsection*{sf\_coperator}
This is a C data type of type void. It is also a pointer to a function which takes the input parameters precisely as \texttt{(bool
 bool, int, int, sf\_complex*, sf\_complex*)}. It works just like \hyperref[sec:sf_operator]{\texttt{sf\_operator}} but does it for complex numbers. It is defined in \texttt{\_solver.h}.

\subsubsection*{sf\_csolverstep}
This is a C data type of type void. It is also a pointer to a function which takes the input parameters precisely as \texttt{(bool, bool, int, int, sf\_complex*, const sf\_complex*, sf\_complex*, const sf\_complex*)}. It works just like \hyperref[sec:sf_solverstep]{\texttt{sf\_solverstep}} but does it for complex numbers. It is defined in \texttt{\_solver.h}.

\subsubsection*{sf\_cweight}
This is a C data type of type void. It is also a pointer to a function which takes the input parameters precisely as \texttt{(int, int, const sf\_complex*, float*)}. It works just like \hyperref[sec:sf_weight]{\texttt{sf\_weight}} but does it for the complex numbers. It is defined in \texttt{\_solver.h}.

\subsubsection*{sf\_eno}
This is a C data structure, which contains the required variables for 1D ENO (Essentially Non Oscillatory) interpolation. It is defined in \hyperref[sec:eno.c]{\texttt{eno.c}}.

\subsubsection*{sf\_eno2}
This is a C data structure, which contains the required variables for 2D ENO (Essentially Non Oscillatory) interpolation. It is defined in \hyperref[sec:eno2.c]{\texttt{eno2.c}}.

\subsubsection*{sf\_bands}
This is a C data structure, which contains the required variables for storing a banded matrix. It is defined in \hyperref[sec:banded.c]{\texttt{banded.c}}.



\subsection{Geometry}
This section lists the data types used to define the geometry of the seismic data.

\subsubsection*{sf\_axa}
This is a C data structure which contains the variables of type int and float to store the length origin and sampling of the axis. It is defined in \hyperref[sec:axa.c]{\texttt{axa.c}}.

\subsubsection*{pt2d}
This is a C data structure which contains the variables of type double and float to store the location and value of a 2D point. It is defined in \hyperref[sec:point.c]{\texttt{point.c}}.

\subsubsection*{pt3d}
This is a C data structure which contains the variables of type double and float to store the location and value of a 3D point. It is defined in \hyperref[sec:point.c]{\texttt{point.c}}.

\subsubsection*{vc2d}
This is a C data structure which contains the variables of type double to store the components of a 2D vector. It is defined in \hyperref[sec:vector.c]{\texttt{vector.c}}.

\subsubsection*{vc3d}
This is a C data structure which contains the variables of type double to store the components of a 3D vector. It is defined in \hyperref[sec:vector.c]{\texttt{vector.c}}.



\subsection{Lists}
This section describes the data types used to create and operate on lists.

\subsubsection*{sf\_list}
This is a C data structure, which contains the required variables for storing the information about the list, for example . It uses another C data structure �Entry�. It is defined in \hyperref[sec:llist.c]{\texttt{llist.c}}.

\subsubsection*{Entry}
This is a C data structure, which contains the required variables for storing the elements and moving the pointer in the list. It is defined in \hyperref[sec:llist.c]{\texttt{llist.c}}.



\subsection{sys/types.h}
This section describes some of the data types used from the C header file \texttt{sys/types.h}.

\subsubsection*{off\_t}
This is a data type defined in the \texttt{sys/types.h} header file (of fundamental type \texttt{unsigned long}) and is used to measure the file offset in bytes from the beginning of the file. It is defined as a signed, 32-bit integer, but if the programming environment enables large files \texttt{off\_t} is defined to be a signed, 64-bit integer.

\subsubsection*{size\_t}
This is a data type defined in the \texttt{sys/types.h} header (of fundamental type \texttt{unsigned int}) and is used to measure the file size in units of character. It is used to hold the result of the sizeof operator in C, for example sizeof(int)=4, sizeof(char)=1, etc.
