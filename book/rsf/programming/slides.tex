% All figures that may be used for EAGE Barcelona 2010 abstract
\def\pcscwp{
Center for Wave Phenomena \\ 
Colorado School of Mines \\ 
psava@mines.edu
}

\def\pcscover{
\author[]{Paul Sava}
\institute{\pcscwp}
\date{}
\logo{WSI}
\large
}

\def\WSI{\textbf{WSI}~}

% ------------------------------------------------------------
% colors
\def\red#1{\textcolor{red}{#1}}
\def\green#1{\textcolor{green}{#1}}
\def\blue#1{\textcolor{blue}{#1}}
\def\yellow#1{\textcolor{yellow}{#1}}
\def\magenta#1{\textcolor{magenta}{#1}}

\def\black#1{\textcolor{black}{#1}}
\def\white#1{\textcolor{white}{#1}}
\def\gray#1{\textcolor{gray}{#1}}

\definecolor{DarkGreen}{rgb}{0,0.5,0}
\definecolor{DarkBlue}{rgb}{0,0,0.5}
\definecolor{DarkRed}{rgb}{0.5,0,0}
\definecolor{DarkYellow}{rgb}{0.5,0,0}
\definecolor{LightRed}{rgb}{1.000,0.752,0.796}
\definecolor{LightGreen}{rgb}{0.792,1.000,0.439}
\definecolor{LightBlue}{rgb}{0.690,0.886,1.000}
\definecolor{LightYellow}{rgb}{1.000,0.925,0.545}
\definecolor{DarkGray}{rgb}{0.45,0.45,0.45}
\definecolor{LightGray}{rgb}{0.90,0.90,0.90}

\def\darkgreen#1{\textcolor{DarkGreen}{#1}}
\def\darkblue#1 {\textcolor{DarkBlue}{#1}}
\def\darkred#1  {\textcolor{DarkRed}{#1}}
\def\lightred#1 {\textcolor{LightRed}{#1}}

\def\lightgray#1{\textcolor{LightGray}{#1}}
\def\darkgrey#1 {\textcolor{DarkGray}{#1}}

% ------------------------------------------------------------
% madagascar
\def\mg{\darkgreen{\sc madagascar~}}
\def\latex{\darkgreen{\sc \LaTeX}}
\def\mex#1{ \normalsize \red{ \texttt{#1} } \large }
\def\mvbt#1{\small{\blue{\begin{semiverbatim}#1\end{semiverbatim}}}}

% ------------------------------------------------------------
% equations
\def\bea{\begin{eqnarray}}
\def\eea{  \end{eqnarray}}

\def\beq{\begin{equation}}
\def\eeq{  \end{equation}}

%\def\req#1{(\ref{#1})}

\def\lp{\left (}
\def\rp{\right)}

\def\lb{\left [}
\def\rb{\right]}

\def\pbox#1{ \fbox {$ \displaystyle #1 $}}

\def\non{\nonumber \\ \nonumber}

% ------------------------------------------------------------
% REFERENCE (equations and figures)
\def\rEq#1{Equation~\ref{eqn:#1}}
\def\req#1{equation~\ref{eqn:#1}}
\def\rEqs#1{Equations~\ref{eqn:#1}}
\def\reqs#1{equations~\ref{eqn:#1}}
\def\ren#1{\ref{eqn:#1}}

\def\rFg#1{Figure~\ref{fig:#1}}
\def\rfg#1{Figure~\ref{fig:#1}}
\def\rFgs#1{Figures~\ref{fig:#1}}
\def\rfgs#1{Figures~\ref{fig:#1}}
\def\rfn#1{\ref{fig:#1}}

% ------------------------------------------------------------
% field operators

% trace
\def\tr{\texttt{tr}\;}

% divergence
\def\DIV#1{\nabla \cdot {#1}}

% curl
\def\CURL#1{\nabla \times {#1}}

% gradient
\def\GRAD#1{\nabla {#1}}

% Laplacian
\def\LAPL#1{\nabla^2 {#1}}

\def\dellin{
\lb
\begin{matrix}
\done{}{x} \; \done{}{y} \; \done{}{z}
\end{matrix}
\rb
}

\def\delcol{
\lb
\begin{matrix}
\done{}{x} \non
\done{}{y} \non
\done{}{z}
\end{matrix}
\rb
}

\def\aveclin{
\lb
\begin{matrix}
a_x \; a_y \; a_z
\end{matrix}
\rb
}


% ------------------------------------------------------------

% elastic tensor
\def\CC{{\bf C}}

% identity tensor
\def\I{\;{\bf I}}

% particle displacement vector
\def\uu{{\bf u}}

% particle velocity vector
\def\vv{{\bf v}}

% particle acceleration vector
\def\aa{{\bf a}}

% force vector
\def\ff{{\bf f}}

% wavenumber vector
\def\kk{{\bf k}}

% ray parameter vector
\def\pp{{\bf p}}

% distance vector
\def\xx{{\bf x}}
\def\kkx{{\kk_\xx}}
\def\ppx{{\pp_\xx}}

\def\yy{{\bf y}}

% normal vector
\def\nn{{\bf n}}
\def\ns{\nn_s}
\def\nr{\nn_r}

% source vector
\def\ss{{\bf s}}
\def\kks{{\kk_\ss}}
\def\pps{{\pp_\ss}}

% receiver vector
\def\rr{{\bf r}}
\def\kkr{{\kk_\rr}}
\def\ppr{{\pp_\rr}}

% midpoint vector
\def\mm{{\bf m}}
\def\kkm{{\kk_\mm}}
\def\ppm{{\pp_\mm}}

% offset vector
\def\ho{{\bf h}}
\def\kkh{{\kk_\ho}}
\def\pph{{\pp_\ho}}

% space-lag vector
\def\hh{ {\boldsymbol{\lambda}} }
\def\kkl{{\kk_\hh}}
\def\ppl{{\pp_\hh}}

% CIP vector
\def\cc{ {\bf c}}

% time-lag scalar
\def\tt{\tau}
\def\tts{\tt_s}
\def\ttr{\tt_r}

% frequency
\def\ww{\omega}

%
\def\dd{{\bf d}}

\def\bb{{\bf b}}
\def\qq{{\bf q}}

\def\ii{{\bf i}} % unit vector
\def\jj{{\bf j}} % unit vector

\def\lo{{\bf l}}

% ------------------------------------------------------------

\def\Fop#1{\mathcal{F}     \lb #1 \rb}
\def\Fin#1{\mathcal{F}^{-1}\lb #1 \rb}

% ------------------------------------------------------------
% partial derivatives

\def\dtwo#1#2{\frac{\partial^2 #1}{\partial #2^2}}
\def\done#1#2{\frac{\partial   #1}{\partial #2  }}
\def\dthr#1#2{\frac{\partial^3 #1}{\partial #2^3}}
\def\mtwo#1#2#3{ \frac{\partial^2#1}{\partial #2 \partial#3} }

\def\larrow#1{\stackrel{#1}{\longleftarrow}}
\def\rarrow#1{\stackrel{#1}{\longrightarrow}}

% ------------------------------------------------------------
% elasticity 

\def\stress{\underline{\textbf{t}}}
\def\strain{\underline{\textbf{e}}}
\def\stiffness{\underline{\underline{\textbf{c}}}}
\def\compliance{\underline{\underline{\textbf{s}}}}

\def\GEOMlaw{
\strain = \frac{1}{2} 
\lb \GRAD{\uu} + \lp \GRAD{\uu} \rp^T \rb
}

\def\HOOKElaw{
\stress = \lambda \; tr \lp \strain \rp {\bf I} + 2 \mu \strain 
}

\def\CONSTITUTIVElaw{
\stress = \stiffness \;\strain 
}


\def\NEWTONlaw{
\rho \ddot{\uu} = \DIV{\stress}
}

\def\NAVIEReq{
\rho \ddot\uu =
\lp \lambda + 2\mu \rp \GRAD{\lp \DIV{\uu} \rp}
             - \mu     \CURL{   \CURL{\uu}}
}

% ------------------------------------------------------------

% potentials
\def\VP{\boldsymbol{\psi}}
\def\SP{\theta}

% stress tensor
\def\ssten{{\bf \sigma}}

\def\ssmat{
\lp \matrix {
 \sigma_{11} &  \sigma_{12}   &  \sigma_{13} \cr
 \sigma_{12} &  \sigma_{22}   &  \sigma_{23} \cr
 \sigma_{13} &  \sigma_{23}   &  \sigma_{33} \cr
} \rp
}

% strain tensor
\def\eeten{{\bf \epsilon}}

\def\eemat{
\lp \matrix {
 \epsilon_{11} &  \epsilon_{12}   &  \epsilon_{13} \cr
 \epsilon_{12} &  \epsilon_{22}   &  \epsilon_{23} \cr
 \epsilon_{13} &  \epsilon_{23}   &  \epsilon_{33} \cr
} \rp
}


% plane wave kernel
\def\pwker{A e^{i k \lp \nn \cdot \xx - v t \rp}}


% ------------------------------------------------------------
% details for expert audience (math, cartoons)
\def\expert{
\colorbox{red}{\textbf{\LARGE \white{!}}}
}

% ------------------------------------------------------------
% image, data, wavefields

\def\RR{R}

\def\UU{W}
\def\US{{\UU_s}}
\def\UR{{\UU_r}}

\def\DD{D}
\def\DS{{\DD_s}}
\def\DR{{\DD_r}}

\def\UUw{\UU}
\def\USw{{\UU_s}}
\def\URw{{\UU_r}}

\def\DDw{\DD}
\def\DSw{{\DD_s}}
\def\DRw{{\DD_r}}

% perturbations

\def\ds{\Delta s}
\def\dl{\Delta l}
\def\di{\Delta \RR}
\def\du{\Delta \UU}

\def\dRR{\Delta \RR}
\def\dUU{\Delta \UU}
\def\dUS{\Delta \US}
\def\dUR{\Delta \UR}

\def\dtt{\Delta \tt}
\def\dhh{\Delta \hh}

% ------------------------------------------------------------
% Green's functions

\def\GG{G}

\def\GS{{\GG_s}}
\def\GR{{\GG_r}}

% ------------------------------------------------------------
% elastic data, wavefields

\def\eRR{\textbf{\RR}}

\def\eDS{{\textbf{\DD}_s}}
\def\eDR{{\textbf{\DD}_r}}
\def\eDD{{\textbf{\DD}}}

\def\eUS{{\textbf{\UU}_s}}
\def\eUR{{\textbf{\UU}_r}}
\def\eUU{{\textbf{\UU}}}

% ------------------------------------------------------------
% sliding bar
\def\tline#1{
\put(95,-3){\small \blue{time}}
\put(-4,-1){\small \blue{0}}
\thicklines
\put( 0,0){\color{blue} \vector(1,0){100}}
\put(#1,0){\color{red}  \circle*{2}}
}

% ------------------------------------------------------------
% arrow on figure
\def\myarrow#1#2#3{
\thicklines
\put(#1,#2){\color{green} \vector(-1,-1){5}}
\put(#1,#2){\color{green} \textbf{#3}}
}

\def\bkarrow#1#2#3{
\thicklines
\put(#1,#2){\color{black} \vector(-1,-1){5}}
\put(#1,#2){\color{black} \textbf{#3}}
}

\def\anarrow#1#2#3#4{
\thicklines
\put(#1,#2){\color{#4} \vector(-1,-1){5}}
\put(#1,#2){\color{#4} \textbf{#3}}
}

\def\myvec#1#2#3#4{
\thicklines
\put(#1,#2){\color{green} \rotatebox{#4}{\vector(4,0){20}}}
\put(#1,#2){\color{green} \textbf{#3}}
}

% ------------------------------------------------------------
% circle on figure
\def\mycircle#1#2#3{
\thicklines
\put(#1,#2){\color{green} \circle{#3}}
}

% ------------------------------------------------------------
% note on figure
\def\mynote#1#2#3{
\put(#1,#2){\color{green} \textbf{#3}}
}

\def\biglabel#1#2#3{
\put(#1,#2){\Large \textbf{#3}}
}

\def\wlabel#1#2#3{ \white{ \biglabel{#1}{#2}{#3} }}
\def\klabel#1#2#3{ \black{ \biglabel{#1}{#2}{#3} }}
\def\rlabel#1#2#3{ \red{   \biglabel{#1}{#2}{#3} }}
\def\glabel#1#2#3{ \green{ \biglabel{#1}{#2}{#3} }}
\def\blabel#1#2#3{ \blue { \biglabel{#1}{#2}{#3} }}
\def\ylabel#1#2#3{ \yellow{\biglabel{#1}{#2}{#3} }}

% ------------------------------------------------------------
% centering
\def\cen#1{ \begin{center} \textbf{#1} \end{center}}
\def\cit#1{ \begin{center} \textit{#1} \end{center}}
\def\ctt#1{ \begin{center} \texttt{#1} \end{center}}

% emphasis (bold+alert)
\def\bld#1{ \textbf{\alert{#1}}}

% huge fonts
\def\big#1{\begin{center} {\LARGE \textbf{#1}} \end{center}}
\def\hug#1{\begin{center} {\Huge  \textbf{#1}} \end{center}}

% ------------------------------------------------------------
% separator
\def\sep{ \vfill \hrule \vfill}
\def\itab{ \hspace{0.5in}}
\def\nsp{\\ \vspace{0.1in}}

% ------------------------------------------------------------
% integrals

\def\tint#1{\!\!\!\int\!\! #1 dt}
\def\xint#1{\!\!\!\int\!\! #1 d\xx}
\def\wint#1{\!\!\!\int\!\! #1 d\ww}
\def\aint#1{\!\!\!\alert{\int}\!\! #1 d\alert{\xx}}

\def\esum#1{\sum\limits_{#1}}
\def\eint#1{\int\limits_{#1}}

% ------------------------------------------------------------
\def\CONJ#1{\overline{#1}}
\def\MOD#1{\left| {#1} \right|}

% ------------------------------------------------------------
% imaging components

\def\IC{\colorbox{yellow}{\textbf{I.C.}}\;}
\def\WR{\colorbox{yellow}{\textbf{W.R.}}\;}
\def\WE{\colorbox{yellow}{\textbf{W.E.}}\;}
\def\SO{\colorbox{yellow}{\textbf{SOURCE}}\;}
\def\WS{\colorbox{yellow}{\textbf{W.S.}}\;}

% ------------------------------------------------------------
% summary/take home message
\def\thm{take home message}

% ------------------------------------------------------------
\def\dx{\Delta x}
\def\dy{\Delta y}
\def\dz{\Delta z}
\def\dt{\Delta t}

\def\dhx{\Delta h_x}
\def\dhy{\Delta h_y}

\def\kz{{k_z}}
\def\kx{{k_x}}
\def\ky{{k_y}}

\def\kmx{k_{m_x}}
\def\kmy{k_{m_y}}
\def\khx{k_{h_x}}
\def\khy{k_{h_y}}

\def\why{ \alert{\widehat{{\khy}}}}
\def\whx{ \alert{\widehat{{\khx}}}}

\def\lx{{\lambda_x}}
\def\ly{{\lambda_y}}
\def\lz{{\lambda_z}}

\def\klx{k_{\lambda_x}}
\def\kly{k_{\lambda_y}}
\def\klz{k_{\lambda_z}}

\def\mx{{m_x}}
\def\my{{m_y}}
\def\mz{{m_z}}
\def\hx{{h_x}}
\def\hy{{h_y}}
\def\hz{{h_z}}

\def\sx{{s_x}}
\def\sy{{s_y}}
\def\rx{{r_x}}
\def\ry{{r_y}}

% ray parameter (absolute value)
\def\modp#1{\left| \pp_{#1} \right|}

% wavenumber
\def\modk#1{\left| \kk_{#1} \right|}

% ------------------------------------------------------------
\def\kzwk{ {\kz^{\kk}}}
\def\kzwx{ {\kz^{\xx}}}
 
\def\PSk#1{e^{\red{#1 i \kzwk \dz}}}
\def\PSx#1{e^{\red{#1 i \kzwx \dz}}}
\def\PS#1{ e^{\red{#1 i k_z   \dz}}}

\def\TT{t}
\def\TS{t_s}
\def\TR{t_r}

\def\oft{\lp t \rp}
\def\ofw{\lp \ww \rp}

\def\ofx{\lp \xx \rp}
\def\ofh{\lp \hh \rp}
\def\ofk{\lp \kk \rp}
\def\ofs{\lp \ss \rp}
\def\ofr{\lp \rr \rp}
\def\ofz{\lp   z \rp}

\def\ofxt{\lp \xx, t  \rp}
\def\ofst{\lp \ss, t  \rp}
\def\ofrt{\lp \rr, t  \rp}

\def\ofxw{\lp \xx, \ww  \rp}
\def\ofsw{\lp \ss, \ww  \rp}
\def\ofrw{\lp \rr, \ww  \rp}

\def\ofxm{\lp \xx,\hh \rp}

\def\ofxmp{\lp \xx+\hh \rp}
\def\ofxmm{\lp \xx-\hh \rp}

\def\ofmm{\lp \mm      \rp}
\def\ofmz{\lp \mm, z   \rp}
\def\ofmw{\lp \mm, \ww \rp}
\def\ofkm{\lp \kkm     \rp}

% ------------------------------------------------------------
% source/receiver data and wavefields

\def\dst{$\DS\ofst$}
\def\drt{$\DR\ofrt$}
\def\ust{$\US\ofxt$}
\def\urt{$\UR\ofxt$}

\def\dsw{$\DS\ofsw$}
\def\drw{$\DR\ofrw$}
\def\usw{$\US\ofxw$}
\def\urw{$\UR\ofxw$}

% ------------------------------------------------------------
\def\Nx{N_x}
\def\Ny{N_y}
\def\Nz{N_z}
\def\Nt{N_t}
\def\Nw{N_{\ww}}
\def\Nm{N_{\mm}}

\def\Nlx{N_{\lambda_x}}
\def\Nly{N_{\lambda_y}}
\def\Nlz{N_{\lambda_z}}
\def\Nlt{N_{\tau}}

\def\wmin{\ww_{min}}
\def\wmax{\ww_{max}}
\def\zmin{z_{min}}
\def\zmax{z_{max}}
\def\tmin{t_{min}}
\def\tmax{t_{max}}
\def\lmin{\hh_{min}}
\def\lmax{\hh_{max}}
\def\xmin{\xx_{min}}
\def\xmax{\xx_{max}}

% ------------------------------------------------------------
% course qualifiers

\def\fun{\hfill \alert{concepts}}
\def\pra{\hfill \alert{applications}}
\def\fro{\hfill \alert{frontiers}}


% ------------------------------------------------------------
% wavefield extrapolation
\def\ws{ {\ww s} }

\def\kows{\lp \frac{\kx}{\ws} \rp}

\def\kmws{\lp \frac{\modk{\mm}}{\ws} \rp}
\def\kzws{\lp \frac{\kz}       {\ws} \rp}

\def\S{\lb\frac{\modk{\mm}}{\ws  }\rb}
\def\C{\lb\frac{\modk{\mm}}{\ws_0}\rb}
\def\K{\lb\frac{\modk{\mm}}{\ww  }\rb}

\def\Cs{\lb\frac{\modk{\mm}^2}{\lp \ws_0 \rp^2}\rb}

\def\SSR#1{  \sqrt{ \lp \ww {#1} \rp^2 - \modk{\mm}^2} }

\def\SQRsum#1{\sum\limits_{n=1}^{\infty} \lp -1 \rp^n
		\displaystyle{\frac{1}{2} \choose n} #1}

\def\TSE#1#2#3#4{\sum\limits_{#4=#3}^{\infty} \lp -1 \rp^#4
		\displaystyle{#2 \choose #4} {#1}^#4}

\def\onefrac#1#2{\frac{#2^2}{a_#1+b_#1 #2^2}}
\def\SQRfrac#1{
	\sum\limits_{n=1}^{\infty}
	\onefrac{n}{#1} }

\def\dkzds { \left. \frac{d {\kz}}  {d s} \right|_{s_b} }
\def\SSX#1#2{\sqrt{ 1 - \lb \frac{\MOD{#2}}{#1} \rb^2} }
\def\SST#1#2{1 + \sum_{j=1}^N c_j \lb \frac{\MOD{#2}}{#1} \rb^{2j} }

% ------------------------------------------------------------
% acknowledgment
\def\ackcwp{\cen{the sponsors of the\\Center for Wave Phenomena\\at\\Colorado School of Mines}}

% ------------------------------------------------------------
% citation in slides
\def\talkcite#1{{\small \sc #1}}

% ------------------------------------------------------------
\def\ise{GPGN302: Introduction to EM and Seismic Exploration}
\def\inv{GPGN409: Inversion}

% ------------------------------------------------------------
\def\model{m}
\def\data {d}

\def\Lop{ {\mathbf{L}}}
\def\Sop{ {\mathbf{S}}}
\def\Eop{ {\mathbf{E}}}
\def\Iop{ {\mathbf{I}}}
\def\Aop{ {\mathbf{A}}}
\def\Pop{ {\mathbf{P}}}
\def\Fop{ {\mathbf{F}}}
\def\Kop{ {\mathbf{K}}}


% ------------------------------------------------------------
\def\mybox#1{
  \begin{center}
    \fcolorbox{black}{yellow}
    {\begin{minipage}{0.8\columnwidth} {#1} \end{minipage}}
  \end{center}
}

\def\hibox#1{
  \begin{center}
    \fcolorbox{black}{LightGreen}
    {\begin{minipage}{0.8\columnwidth} {#1} \end{minipage}}
  \end{center}
}

% ------------------------------------------------------------
% Nota Bene
\def\nbnote#1{
  \vfill
  \begin{center}
    \colorbox{LightGray}
    {\begin{minipage}{\columnwidth} {\textbf{\black{\large N.B.}} #1} \end{minipage}}
  \end{center}
}

\def\notabene#1{
  \begin{leftbar}
    {\sc Nota Bene:~} #1
  \end{leftbar}
}

\def\sidebar#1{
  \begin{leftbar}
    {#1}
  \end{leftbar}
}


\def\highlight#1{
  \begin{center}
    \colorbox{LightRed}
    {\begin{minipage}{0.95\columnwidth} {#1} \end{minipage}}
  \end{center}
}

% ------------------------------------------------------------
\def\pcsshaded#1{
  \definecolor{shadecolor}{rgb}{0.8,0.8,0.8}
  \begin{shaded} {#1} \end{shaded}
  \definecolor{shadecolor}{rgb}{1.0,1.0,1.0}
}

\def\blueshade#1{
  \definecolor{shadecolor}{rgb}{0.690,0.886,1.000}
    \begin{shaded}
      {#1}
    \end{shaded}
  \definecolor{shadecolor}{rgb}{1.0,1.0,1.0}
}

\def\grayshade#1{
  \definecolor{shadecolor}{rgb}{0.8,0.8,0.8}
  \begin{shaded}
    {#1}
  \end{shaded}
  \definecolor{shadecolor}{rgb}{1.0,1.0,1.0}
}

\def\yellowshade#1{
  \definecolor{shadecolor}{rgb}{1.0,1.0,0.0}
  \begin{shaded}
    {#1}
  \end{shaded}
  \definecolor{shadecolor}{rgb}{1.0,1.0,1.0}
}


% ------------------------------------------------------------
\def\postit#1{
  \begin{center}
    \colorbox{yellow}
    {\begin{minipage}{0.80\columnwidth} {#1} \end{minipage}} 
  \end{center}
}

% ------------------------------------------------------------
\def\graybox#1{
  \begin{center}
    \colorbox{LightGray}
    {\begin{minipage}{1.00\columnwidth} {#1} \end{minipage}}
  \end{center}
}

\def\whitebox#1{
  \begin{center}
    \colorbox{white}
    {\begin{minipage}{1.00\columnwidth} {#1} \end{minipage}}
  \end{center}
}

\def\yellowbox#1{
  \begin{center}
    \colorbox{LightYellow}
    {\begin{minipage}{1.00\columnwidth} {#1} \end{minipage}}
  \end{center}
}

\def\greenbox#1{
  \begin{center}
    \colorbox{LightGreen}
    {\begin{minipage}{1.00\columnwidth} {#1} \end{minipage}}
  \end{center}
}

\def\bluebox#1{
  \begin{center}
    \colorbox{LightBlue}
    {\begin{minipage}{1.00\columnwidth} {#1} \end{minipage}}
  \end{center}
}

\def\redbox#1{
  \begin{center}
    \colorbox{LightRed}
    {\begin{minipage}{1.00\columnwidth} {#1} \end{minipage}}
  \end{center}
}

\def\hyellow#1{ \colorbox{yellow} #1 }
\def\hgreen #1{ \colorbox{green}  #1 }

% ------------------------------------------------------------
% boxes for vectors and matrices

\def\pcsbox#1#2#3#4{
  % #1 = hmax
  % #2 = height
  % #3 = width
  % #4 = text
  \begin{picture}(#3,#1)
    \linethickness{0.5mm}
    % 
    \multiput(0,#1)(#3, 0){2}{\line(0,-1){#2}}
    \multiput(0,#1)(0,-#2){2}{\line(+1,0){#3}}
    % 
    \put(1,-10){#4}
  \end{picture}
}

% annotate block equations
\def\pcssym#1#2{
  \begin{picture}(3,#1)
    \put(1,-10){#2}
  \end{picture}
}

% block equation sign
\def\pcsops#1#2#3#4{
  \begin{picture}(#3,#1)
    \put(0,#2){#4}
  \end{picture}
}

% ------------------------------------------------------------
% two color boxes
\def\sidebyside#1#2{
  \begin{center}
    \colorbox{LightBlue}{
      \begin{minipage}{1.0\columnwidth} {#1} \end{minipage}
    }
    \colorbox{LightYellow}{
      \begin{minipage}{1.0\columnwidth} {#2} \end{minipage}
    }
  \end{center}
}

% upward pointing arrow
\def\uparrow#1#2#3{
  \thicklines
  \put(#1,#2){\color{green} \vector(0,+1){5}}
  \put(#1,#2){\color{green} \textbf{#3}}
}

% acknowledge figure source
\def\ackfig#1#2#3{\blabel{#1}{#2}{\normalsize \sc #3}}

\title{Programming in Madagascar}
\author[]{Jeff Godwin$^*$}
\institute{Center for Wave Phenomena \\
Colorado School of Mines \\
godwin.jeffrey@gmail.com
}
\date{}
\large
% ------------------------------------------------------------
\mode<beamer> {\cwpcover}
\cwpnote{}

\lstset{language=Python,showstringspaces=false}

\newcommand{\centered}[1]{\begin{center} #1 \end{center}}
% ------------------------------------------------------------
\begin{frame} \frametitle{Purpose}

``Give a man a fish and you feed him for a day. Teach him how to fish and you feed him for a lifetime."

 - Someone Wise
\end{frame}

\begin{frame} \frametitle{Purpose}
There are:
\begin{itemize}
\item $\approx$ 600 programs in Madagascar
\item both seismic and non-seismic
\item generic data manipulation tools
\end{itemize}
\end{frame}

\begin{frame}
\textbf{Some tasks are not (easily) doable with current tools.}
\end{frame}

\begin{frame} \frametitle{Goals}
\begin{itemize}
\item Madagascar program design
\item Madagascar framework
\item Python API
\begin{itemize}
\item SVD
\end{itemize}
\item Demos:
\begin{itemize}
\item SVD
\item MayaVi
\end{itemize}
\item Python and SAGE
\end{itemize}
\end{frame}

\begin{frame} \frametitle{Disclaimer}
Should you build programs for all of your needs?
\linebreak
\pause
\begin{center}\Huge\textbf{NO!}\end{center}
\end{frame}

\begin{frame}
\centered{\Huge \textbf{Don't reinvent the wheel}}
\end{frame}

\begin{frame} \frametitle{Wheel problems}
\begin{itemize}
\item Multiply two datasets
\pause
\item Concatenate datasets
\pause
\item FFT of a dataset
\pause
\item Apply a bandpass filter
\pause
\item Stolt migrations
\end{itemize}
\end{frame}

\begin{frame} \frametitle{Your friend...}
\begin{center}\Huge\textbf{sfdoc -k .}\end{center}
\end{frame}

\begin{frame} \frametitle{Additional resources}
\begin{itemize}
\item Program examples
\item RSFSRC/book/recipes
\item User mailing list
\item Developer mailing list
\end{itemize}
\end{frame}

\inputdir{XFig}



\begin{frame}
\centered{\Huge Program design}
\end{frame}

\begin{frame} \frametitle{Program architecture}
\begin{itemize}
\item RSF programs are task-centric
\pause
\item ONE task per program
\pause
\item Pass data to another program for next task
\pause
\item Data from standard in 
\item Data to   standard out
\item Options from command line arguments
\end{itemize}
\end{frame}

\begin{frame} \frametitle{Sample problem}
\begin{center} Joe wants to apply the newest XYZ filter in the frequency domain, but his RSF data is in the time domain, how should he design his new RSF program? \end{center}
\end{frame}

\begin{frame} \frametitle{Possible solutions}
\begin{itemize}
\item Write his own code to do the FFT and then apply the filter, and then apply the inverse FFT
\item Write his own code to apply the filter to a dataset that has already had the FFT applied, use a C library for the FFT
\item Write his code to apply the filter to a dataset that has already had the FFT applied, take the inverse FFT using another program
\end{itemize}
\end{frame}

\begin{frame} \frametitle{Possible solutions}
\begin{itemize}
\item Write his own code to do the FFT, apply the filter, and apply the inverse FFT
\item Write his own code to apply the filter, use a C library for the FFT
\item \textbf{Write his code to apply the filter to a dataset that has already had the FFT applied, take the inverse FFT using another program}
\end{itemize}
\end{frame}

\begin{frame}
\centered{\Huge Madagascar framework}
\end{frame}

\begin{frame} \frametitle{Madagascar framework}
\plot{framework}{width=0.8\textwidth}{}
\end{frame}

\begin{frame} \frametitle{API overview}
\plot{c_api}{width=0.8\textwidth}{}
\end{frame}

\begin{frame} \frametitle{Available API}
\begin{itemize}
\item C/C++
\item \textbf{Python}
\item Fortran 77
\item Fortran 90
\item Matlab
\item Java
\item Octave
\end{itemize}
\end{frame}

\begin{frame} \frametitle{API limitations}
\begin{itemize}
\item Do not fully expose all core C functions
\pause
\item Do not expose other RSF programs
\pause
\item Limited communication between APIs
\pause
\item Additional dependencies
\end{itemize}
\end{frame}

\begin{frame}
\centered{\Huge Python API}
\end{frame}

\begin{frame} \frametitle{Why Python?}
\begin{itemize}
\item Simple syntax
\item Easy to maintain and understand
\item Fast function/program prototyping
\item Object Oriented (OOP) \emph{Lite}
\item Good interface to C/C++
\item \textbf{Powerful} libraries and packages
\end{itemize}
\end{frame}

\begin{frame}
\centered{\Huge 80\% results with 20\% of the effort}
\end{frame}

\begin{frame} \frametitle{When NOT to Use Python}
\begin{itemize}
\item Performance
\item Low-level access
\item Significant object overhead 
\end{itemize}
\end{frame}

\begin{frame} \frametitle{Python RSF program outline}
\begin{itemize}
\item Documentation (comments)
\item Import RSF API
\item Initialize RSF command line parser
\item Read command line variables
\item Declare all input and output RSF files
\item Read input data headers
\item Read input data sets
\item ...
\item Create output data headers
\item Write output data
\end{itemize}
\end{frame}

\begin{frame}
\centered{\Huge Python API Demo: Matrix SVD}
\end{frame}

\begin{frame} \frametitle{Comments rules}
\begin{itemize}
\item Shebang execution rule   : $\#$!/usr/bin/env python
\item One line documentation        $'''$My program does SVD on a 2D matrix $'''$
\item Block documentation (comments)
''' line 1
	line 2
	... 
	end comments
'''
\end{itemize}
\end{frame}

\begin{frame} \frametitle{Comments}
\lstinputlisting[firstline=1,lastline=5,frame=shadowbox]{\RSFSRC/user/godwinj/Msvd.py}
\end{frame}

\begin{frame} \frametitle{API import}
\lstinputlisting[firstline=6,lastline=14,frame=shadowbox]{\RSFSRC/user/godwinj/Msvd.py}
\end{frame}

\begin{frame} \frametitle{Initialize command line argument parser}
\lstinputlisting[firstline=17,lastline=18,frame=shadowbox]{\RSFSRC/user/godwinj/Msvd.py}
\end{frame}

\begin{frame} \frametitle{Par(ser) rules}
\begin{itemize}
\item par = Par()                      $\#$ initialize par object
\item par.int('n1',1)                  $\#$ get first dimension              
\item par.float('d1',1.0)              $\#$ get sampling interval        
\item par.bool('verb',False)           $\#$ show verbose output?         
\item par.string('outname','temp.rsf') $\#$ store output where?
\end{itemize}
\end{frame}

\begin{frame} \frametitle{Read command line arguments}
\lstinputlisting[firstline=19,lastline=22,frame=shadowbox]{\RSFSRC/user/godwinj/Msvd.py}
\end{frame}

\begin{frame} \frametitle{RSF input/output classes}
 input = rsf.Input("in.rsf") \\
output = rsf.Output("out.rsf")
\hrule
\begin{itemize}
\item If no name specified, default to stdin or stdout respectively
\item Input.read(numpy.array)
\item Output.write(numpy.array)
\end{itemize}
\end{frame}

\begin{frame} \frametitle{Declare inputs and outputs}
\lstinputlisting[firstline=23,lastline=25,frame=shadowbox]{\RSFSRC/user/godwinj/Msvd.py}
\end{frame}

\begin{frame} \frametitle{Read input headers}
\lstinputlisting[language=Python,firstline=31,lastline=33,frame=shadowbox]{\RSFSRC/user/godwinj/Msvd.py}
\end{frame}

\begin{frame} \frametitle{Read datasets}
\lstinputlisting[language=Python,firstline=35,lastline=38,frame=shadowbox]{\RSFSRC/user/godwinj/Msvd.py}
\end{frame}

\begin{frame} \frametitle{Perform SVD}
\lstinputlisting[firstline=40,lastline=41,frame=shadowbox]{\RSFSRC/user/godwinj/Msvd.py}
\end{frame}

\begin{frame} \frametitle{Setup output headers}
\lstinputlisting[firstline=40,lastline=41,frame=shadowbox]{\RSFSRC/user/godwinj/Msvd.py}
\end{frame}

\begin{frame} \frametitle{Write out data}
\lstinputlisting[firstline=47,lastline=48,frame=shadowbox]{\RSFSRC/user/godwinj/Msvd.py}
\end{frame}

\begin{frame} \frametitle{Close open files}
\lstinputlisting[firstline=60,lastline=62,frame=shadowbox]{\RSFSRC/user/godwinj/Msvd.py}
\end{frame}

\begin{frame}
\centered{\Huge Python demos}
\end{frame}

%\begin{frame} \frametitle{Installation Instructions}
%Source code: samples/python/
%\hrule
%\begin{itemize}
%\item Need scipy, numpy, Python API
%\item Symbolic link from samples/python to RSFSRC/user/pyexample
%\item Rebuild, reinstall: \emph{scons; scons install}
%\end{itemize}
%\end{frame}

\begin{frame} \frametitle{ SVD}
\begin{itemize}
\item Location: book/rsf/programming/samples/svd
\item Program: sfsvd
\item Execution: scons view
\item Demonstrates:  the use of simple numpy program wrapped in the RSF API
\end{itemize}
\end{frame}

\begin{frame}\frametitle{RSF + Python + MayaVi = \emph{Interactive} visualization}
\begin{itemize}
\item Location: book/rsf/programming/samples/cube
\item Program : sfthreedcube
\item Execution: scons or scons view, user must close pop-up window
\item Demonstrates:  How to use more advanced libraries within RSF python programs
\item Requires: MayaVi and VTK, see (http://code.enthought.com/projects/mayavi/)
\end{itemize}
\end{frame}

\begin{frame}
\centered{\Huge Python and SAGE}
\end{frame}

\begin{frame} \frametitle{What is SAGE?}
\begin{itemize}
\item A collection of \emph{many, many} scientific packages and libraries
\item An interactive GUI for developing programs
\item A suite for designing, running and sharing programs
\item ... much more!
\end{itemize}
\hrule
\pause
\centered{\Huge \textbf{Bottom Line: Use SAGE}}
\end{frame}

\begin{frame} \frametitle{Python Scripting}
\centered{rfile = rsf$.<$program name$>$(args)$[$files$]$}
\hrule
\pause
\begin{itemize}
\item rfile = rsf.spike$(n1=150,k1=25)[0]$
\item filt = rsf.bandpass$(fhi=10)[rfile]$
\end{itemize}
\end{frame} 

\begin{frame} \frametitle{RSF Objects}
\begin{itemize}
\item some operations are defined (add, subtract, multiply)
\item can be converted to numpy arrays by slicing: x = rfile[:]
\item can be plotted directly: 
\begin{itemize}
\item sfwiggle: filt.wiggle().show()
\item sfgrey: filt.grey().show()
\end{itemize}
\end{itemize}
\end{frame}

\begin{frame}
\centered{\Huge SAGE Demo}
\end{frame}

\begin{frame} \frametitle{SAGE Demo}
\begin{itemize}
\item Location: samples/sage
\item Program:  None
\item Execution: Run SAGE notebook, and upload rsf\_demo.sws to worksheet
\item Demonstrates: Basic functionality with SAGE
\item Requires: SAGE, see (http://sagemath.org)
\end{itemize}
\end{frame}

\begin{frame} \frametitle{Conclusions}
\begin{itemize}
\item Madagascar framework is (relatively) straightforward
\item Various APIs provide choice
\item Python API is simple, and powerful
\item SAGE $=$ lots of possibilities
\end{itemize}
\end{frame}
