\def \largelabel#1#2#3{
\put(#1,#2){\Large \textbf{#3}}
}
\def\sep{ \vfill \hrule \vfill}
\def\itab{ \hspace{0.5in}}
\def\black#1{\textcolor{black}{#1}}
\def\gray#1{\textcolor{gray}{#1}}
\def\white#1{\textcolor{white}{#1}}

\def\klabellarge#1#2#3{ \black{ \largelabel{#1}{#2}{#3} }}





\title[]{High performance computing and Madagascar}
\author[]{Esteban  D\'{i}az}
\institute{Center for Wave Phenomena \\
Colorado School of Mines 
\date{}
\logo{}
}
\def\big#1{\begin{center} \LARGE \textbf{#1} \end{center}}
\def\cen#1{\begin{center}        \textbf{#1} \end{center}}

% ------------------------------------------------------------
\mode<beamer> { \cwpcover }
% ------------------------------------------------------------

\begin{frame}
    \big{Why should we bother \\ to go parallel?}
\end{frame}

\begin{frame}
    \plot{moore_law}{width=0.95\textwidth}{ \klabellarge{35}{19}{\footnotesize  \url{http://en.wikipedia.org/wiki/Moore\%27s\_law}}}
\end{frame}


\begin{frame}
    \plot{AmdahlsLaw}{width=.8\textwidth}{ \klabellarge{5}{-8}{\small \url{http://en.wikipedia.org/wiki/Amdahl\%27s\_law}}}
\end{frame}



\inputdir{XFig}
\large
\begin{frame} \frametitle{outline}
\itab{{concepts of parallel computing}}
\vfill 
\itab{{approaches: OpenMP and MPI}}
\vfill  
\itab{{do you need to code parallel? Let SCons do it for you!}}
\vfill 
\itab{{demo example}}
\vfill 
\end{frame}




\begin{frame} \frametitle{outline}
\itab{{concepts of parallel computing}}
\vfill 
\itab{\gray{approaches: OpenMP and MPI}}
\vfill  
\itab{\gray{do you need to code parallel? Let SCons do it for you!}}
\vfill 
\itab{\gray{demo example}}
\vfill 
\end{frame}




\begin{frame} 
\plot{shared}{width=0.7\textwidth}{ \klabellarge{20}{+40}{shared memory system}\klabellarge{17}{-35}{\small  modified from Vladimir Bashkardin, 2010}}
\end{frame}
\begin{frame}
\plot{distributed}{width=0.7\textwidth}{ \klabellarge{2}{+65}{distributed memory system}\klabellarge{17}{-20}{\small  modified from Vladimir Bashkardin, 2010}}
\end{frame}

\begin{frame} 
\plot{hybrid}{width=0.7\textwidth}{ \klabellarge{-7}{-10}{hybrid memory system (i.e. clusters)}}
\end{frame} 


\begin{frame} \frametitle{outline}
\itab{\gray{concepts of parallel computing}}
\vfill 
\itab{{approaches: OpenMP and MPI}}
\vfill  
\itab{\gray{do you need to code parallel? Let SCons do it for you!}}
\vfill 
\itab{\gray{demo example}}
\vfill 
\end{frame}


\begin{frame}
\itab{\bf{OpenMP:}}
\vfill
\itab{\itab{addresses shared memory parallelization}}
\vfill
\itab{\itab{supported by most of compilers}}
\vfill
\pause
\itab{\bf{resources:}}
\vfill
\itab{\itab{\url{http://openmp.org/wp/}}}
\vfill
\itab{\itab{\url{http://www.cOMPunity.org/}}}
\end{frame}


\begin{frame}
\itab{\bf{MPI (Message Passing Interface):}}
\vfill
\itab{\itab{addresses distributed memory parallelization}}
\vfill
\itab{\itab{You need MPI libraries}}
\vfill
\pause
\itab{\bf{resources:}}
\vfill
\itab{\itab{\url{http://www.open-mpi.org/software/}}}
\vfill 
\end{frame}








\begin{frame} \frametitle{outline}
\itab{\gray{concepts of parallel computing}}
\vfill 
\itab{\gray{approaches: OpenMP and MPI}}
\vfill  
\itab{{do you need to code parallel? Let SCons do it for you!}}
\vfill 
\itab{\gray{demo example}}
\vfill 
\end{frame}


\begin{frame} \frametitle{parallelization with SCons}

    \itab{instead of scons use \bf{pscons} }
    \vfill
    \itab{it will automatically run in parallel independent Flows}
    \vfill
    \sep
    \pause
    \itab{useful environmental variables to have:}
    \begin{alltt}
    \$ export RSF\_THREADS='8'

    \$ export RSF\_CLUSTER='localhost 8'

    \$ export OMP\_NUM\_THREADS=\$RSF\_THREADS
    \end{alltt}
\end{frame} 


\begin{frame}[fragile]{scons vs pscons}

\begin{python}
from rsf.proj import * 


Flow('a','input','flow a')

Flow('b','input','flow b')

Flow('c','a b','flow c')

\end{python}

\end{frame}

\begin{frame}
\plot{scons}{height=0.7\textheight}{}
\end{frame}

\begin{frame}
\plot{pscons}{height=0.7\textheight}{}
\end{frame}



\begin{frame}[fragile]
\vfill 
we can do better than the sequential flow:
\begin{python}      
Flow('Targets','Sources','flow')
\end{python}
\white{      split=[n,m],reduce='cat'}
\white{{\bf n} = axis to split. \\
{\bf m} = lenght of dimension {\bf n}}
\vfill 
\end{frame}



\begin{frame}[fragile] 
\vfill 
... by setting up some parallel variables in Flow:
\begin{python}      
Flow('Targets','Sources','flow',
      split=[n,m],reduce='cat')
\end{python}
{\bf n} = axis to split. \\
{\bf m} = length of dimension {\bf n}
\vfill 
\end{frame}
 


\begin{frame}[fragile] 
Let's consider a sequential example:
\begin{python}
Flow('a',None,'spike n1=8 n2=4 ')
Flow('b','a','noise rep=y')
\end{python}
\end{frame}

\begin{frame}
\plot{sequential}{height=0.8\textheight}{}
\end{frame}



\begin{frame}[fragile] 
now let's run it in parallel:
\begin{python}
Flow('a',None,'spike n1=8 n2=4 ')
Flow('b','a','noise rep=y',split=[2,4],reduce='cat')
\end{python}
\end{frame}



\begin{frame}
\plot{parallel}{height=0.9\textheight}{}
\end{frame}



\begin{frame}[fragile]
The variable `split` in Flow can take several options:
\begin{python} 
Flow('Targets','Sources','flow',
     split=[n,m],reduce='cat') 
\end{python}
\pause
\begin{python} 
Flow('Targets','Sources','flow',
     split=[n,'omp'],reduce='cat') 
\end{python}
\pause
\begin{python} 
Flow('Targets','Sources','flow',
     split=[n,'mpi'],np=m,reduce='cat') 
\end{python}

other reduce commands: sfrcat, sfadd, sfinterleave ...
\end{frame}




\begin{frame} \frametitle{outline}
\itab{\gray{concepts of parallel computing}}
\vfill 
\itab{\gray{approaches: OpenMP and MPI}}
\vfill  
\itab{\gray{do you need to code parallel? Let SCons do it for you!}}
\vfill 
\itab{{demo example}}
\vfill 
\end{frame}





\begin{frame}[fragile]
\begin{python}
from rsf.proj import * 

# Input data to use in example:
# produces random numbers with uniform 
# distribution [-1000,1000]

Flow('random',None,
    '''spike n1=1024 n2=1024 n3=128|
       noise rep=y range=2000  type=n
      |math output="input-1000"''')

Flow('clip_seq','random','clip clip=500')
\end{python}

Let's see the timing:

   \begin{alltt}
   time pscons clip_seq.rsf
   \end{alltt}
\end{frame}




\begin{frame}[fragile]
\begin{python}
# 1: In this case scons will split the input,
#    process separately, and cat all the 
#    individual outputs into one target.  
Flow('clip_scons','random',' clip clip=500',
      split=[3,128],reduce='cat')
\end{python}

   \begin{alltt}
   time pscons clip_scons.rsf
   \end{alltt}
\end{frame}

\begin{frame}[fragile]
\begin{python}
# 2: Now, I am specifying the type of parallelization
#    using OpenMP, it is parallelizing over the
#    slower axis (the third one) 
Flow('clip_omp_scons','random',' clip clip=500',
    split=[3,'omp'],reduce='cat')
\end{python}

   \begin{alltt}
   time pscons clip_omp_scons.rsf
   \end{alltt}
\end{frame}



\begin{frame}[fragile]
\begin{python}
# 3: In this case I use explicit parallelization 
#    with MPI [n,'mpi'] in this case m is the
#    number of CPU's to use, n is the axis to split

Flow('clip_mpi_scons','random',' clip clip=500',
    split=[3,'mpi'],np=8,reduce='cat')
\end{python}
   \begin{alltt}
   time pscons clip_mpi_scons.rsf
   \end{alltt}
\end{frame}




\begin{frame}[fragile]
\begin{python}
# 4: OMP parallelization in the code. It uses the
#    number of maximum number of cores available if
#    $OMP_NUM_THREADS doesn't exist.
Program('clip_omp','Mclip.c')
Flow('clip_omp_incore','random clip_omp.exe',
     './${SOURCES[1]} clip=500')
\end{python}
   \begin{alltt}
   time pscons clip_omp_incore.rsf
   \end{alltt}
\end{frame}


\begin{frame}[fragile]
\begin{python}
# 5: Hybrid case, OMP parallelization in the code, and 
#    extra parallelization with MPI. In a single 
#    computer doesn't make too much sense but with
#    a cluster it does.
Flow('clip_hybrid','random clip_omp.exe',
    './${SOURCES[1]} clip=500',split=[3,'mpi'],np=4,reduce='cat')
\end{python}

   \begin{alltt}
   time pscons clip_hybrid.rsf
   \end{alltt}
\end{frame}


\begin{frame} 
Issues that affect scalability: \\
\begin{itemize}
\item I/O (read and write data to disk)
\pause
\item network 
\pause
\item combination of both
\end {itemize}
\end{frame}



\begin{frame}\frametitle{conclusions}
\begin {itemize}
    \item parallelize as much as you can (Amdahl's law)
    \item use pscons instead of scons
    \item avoid I/O and network as much as possible
    \item do code parallelization if needed
\end {itemize}
\end{frame}

\begin{frame}\frametitle{resources}
\itab{\url{http://ahay.org/wiki/Parallel_Computing}}
\vfill
\itab{Vladimir Bashkardin's, Houston 2010 school: \small \url{http://ahay.org/wikilocal/docs/Houston2010_HPC.pdf}}
\vfill
\itab{Dave Hale's, Houston 2011 workshop: \small \url{http://www.beg.utexas.edu/pttc/2011/2011_pres_01_03_hale.pdf}}
\vfill
\end{frame}

\begin{frame}
    \big{ Thank you!}
\end{frame}

\begin{frame}
check at the Yang Liu's 2010 school for a real data example with parallelization:\\ 
\$RSFSRC/book/rsf/usp/data/
\end{frame}



