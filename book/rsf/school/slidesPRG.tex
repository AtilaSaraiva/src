\def\pcscwp{
Center for Wave Phenomena \\ 
Colorado School of Mines \\ 
psava@mines.edu
}

\def\pcscover{
\author[]{Paul Sava}
\institute{\pcscwp}
\date{}
\logo{WSI}
\large
}

\def\WSI{\textbf{WSI}~}

% ------------------------------------------------------------
% colors
\def\red#1{\textcolor{red}{#1}}
\def\green#1{\textcolor{green}{#1}}
\def\blue#1{\textcolor{blue}{#1}}
\def\yellow#1{\textcolor{yellow}{#1}}
\def\magenta#1{\textcolor{magenta}{#1}}

\def\black#1{\textcolor{black}{#1}}
\def\white#1{\textcolor{white}{#1}}
\def\gray#1{\textcolor{gray}{#1}}

\definecolor{DarkGreen}{rgb}{0,0.5,0}
\definecolor{DarkBlue}{rgb}{0,0,0.5}
\definecolor{DarkRed}{rgb}{0.5,0,0}
\definecolor{DarkYellow}{rgb}{0.5,0,0}
\definecolor{LightRed}{rgb}{1.000,0.752,0.796}
\definecolor{LightGreen}{rgb}{0.792,1.000,0.439}
\definecolor{LightBlue}{rgb}{0.690,0.886,1.000}
\definecolor{LightYellow}{rgb}{1.000,0.925,0.545}
\definecolor{DarkGray}{rgb}{0.45,0.45,0.45}
\definecolor{LightGray}{rgb}{0.90,0.90,0.90}

\def\darkgreen#1{\textcolor{DarkGreen}{#1}}
\def\darkblue#1 {\textcolor{DarkBlue}{#1}}
\def\darkred#1  {\textcolor{DarkRed}{#1}}
\def\lightred#1 {\textcolor{LightRed}{#1}}

\def\lightgray#1{\textcolor{LightGray}{#1}}
\def\darkgrey#1 {\textcolor{DarkGray}{#1}}

% ------------------------------------------------------------
% madagascar
\def\mg{\darkgreen{\sc madagascar~}}
\def\latex{\darkgreen{\sc \LaTeX}}
\def\mex#1{ \normalsize \red{ \texttt{#1} } \large }
\def\mvbt#1{\small{\blue{\begin{semiverbatim}#1\end{semiverbatim}}}}

% ------------------------------------------------------------
% equations
\def\bea{\begin{eqnarray}}
\def\eea{  \end{eqnarray}}

\def\beq{\begin{equation}}
\def\eeq{  \end{equation}}

%\def\req#1{(\ref{#1})}

\def\lp{\left (}
\def\rp{\right)}

\def\lb{\left [}
\def\rb{\right]}

\def\pbox#1{ \fbox {$ \displaystyle #1 $}}

\def\non{\nonumber \\ \nonumber}

% ------------------------------------------------------------
% REFERENCE (equations and figures)
\def\rEq#1{Equation~\ref{eqn:#1}}
\def\req#1{equation~\ref{eqn:#1}}
\def\rEqs#1{Equations~\ref{eqn:#1}}
\def\reqs#1{equations~\ref{eqn:#1}}
\def\ren#1{\ref{eqn:#1}}

\def\rFg#1{Figure~\ref{fig:#1}}
\def\rfg#1{Figure~\ref{fig:#1}}
\def\rFgs#1{Figures~\ref{fig:#1}}
\def\rfgs#1{Figures~\ref{fig:#1}}
\def\rfn#1{\ref{fig:#1}}

% ------------------------------------------------------------
% field operators

% trace
\def\tr{\texttt{tr}\;}

% divergence
\def\DIV#1{\nabla \cdot {#1}}

% curl
\def\CURL#1{\nabla \times {#1}}

% gradient
\def\GRAD#1{\nabla {#1}}

% Laplacian
\def\LAPL#1{\nabla^2 {#1}}

\def\dellin{
\lb
\begin{matrix}
\done{}{x} \; \done{}{y} \; \done{}{z}
\end{matrix}
\rb
}

\def\delcol{
\lb
\begin{matrix}
\done{}{x} \non
\done{}{y} \non
\done{}{z}
\end{matrix}
\rb
}

\def\aveclin{
\lb
\begin{matrix}
a_x \; a_y \; a_z
\end{matrix}
\rb
}


% ------------------------------------------------------------

% elastic tensor
\def\CC{{\bf C}}

% identity tensor
\def\I{\;{\bf I}}

% particle displacement vector
\def\uu{{\bf u}}

% particle velocity vector
\def\vv{{\bf v}}

% particle acceleration vector
\def\aa{{\bf a}}

% force vector
\def\ff{{\bf f}}

% wavenumber vector
\def\kk{{\bf k}}

% ray parameter vector
\def\pp{{\bf p}}

% distance vector
\def\xx{{\bf x}}
\def\kkx{{\kk_\xx}}
\def\ppx{{\pp_\xx}}

\def\yy{{\bf y}}

% normal vector
\def\nn{{\bf n}}
\def\ns{\nn_s}
\def\nr{\nn_r}

% source vector
\def\ss{{\bf s}}
\def\kks{{\kk_\ss}}
\def\pps{{\pp_\ss}}

% receiver vector
\def\rr{{\bf r}}
\def\kkr{{\kk_\rr}}
\def\ppr{{\pp_\rr}}

% midpoint vector
\def\mm{{\bf m}}
\def\kkm{{\kk_\mm}}
\def\ppm{{\pp_\mm}}

% offset vector
\def\ho{{\bf h}}
\def\kkh{{\kk_\ho}}
\def\pph{{\pp_\ho}}

% space-lag vector
\def\hh{ {\boldsymbol{\lambda}} }
\def\kkl{{\kk_\hh}}
\def\ppl{{\pp_\hh}}

% CIP vector
\def\cc{ {\bf c}}

% time-lag scalar
\def\tt{\tau}
\def\tts{\tt_s}
\def\ttr{\tt_r}

% frequency
\def\ww{\omega}

%
\def\dd{{\bf d}}

\def\bb{{\bf b}}
\def\qq{{\bf q}}

\def\ii{{\bf i}} % unit vector
\def\jj{{\bf j}} % unit vector

\def\lo{{\bf l}}

% ------------------------------------------------------------

\def\Fop#1{\mathcal{F}     \lb #1 \rb}
\def\Fin#1{\mathcal{F}^{-1}\lb #1 \rb}

% ------------------------------------------------------------
% partial derivatives

\def\dtwo#1#2{\frac{\partial^2 #1}{\partial #2^2}}
\def\done#1#2{\frac{\partial   #1}{\partial #2  }}
\def\dthr#1#2{\frac{\partial^3 #1}{\partial #2^3}}
\def\mtwo#1#2#3{ \frac{\partial^2#1}{\partial #2 \partial#3} }

\def\larrow#1{\stackrel{#1}{\longleftarrow}}
\def\rarrow#1{\stackrel{#1}{\longrightarrow}}

% ------------------------------------------------------------
% elasticity 

\def\stress{\underline{\textbf{t}}}
\def\strain{\underline{\textbf{e}}}
\def\stiffness{\underline{\underline{\textbf{c}}}}
\def\compliance{\underline{\underline{\textbf{s}}}}

\def\GEOMlaw{
\strain = \frac{1}{2} 
\lb \GRAD{\uu} + \lp \GRAD{\uu} \rp^T \rb
}

\def\HOOKElaw{
\stress = \lambda \; tr \lp \strain \rp {\bf I} + 2 \mu \strain 
}

\def\CONSTITUTIVElaw{
\stress = \stiffness \;\strain 
}


\def\NEWTONlaw{
\rho \ddot{\uu} = \DIV{\stress}
}

\def\NAVIEReq{
\rho \ddot\uu =
\lp \lambda + 2\mu \rp \GRAD{\lp \DIV{\uu} \rp}
             - \mu     \CURL{   \CURL{\uu}}
}

% ------------------------------------------------------------

% potentials
\def\VP{\boldsymbol{\psi}}
\def\SP{\theta}

% stress tensor
\def\ssten{{\bf \sigma}}

\def\ssmat{
\lp \matrix {
 \sigma_{11} &  \sigma_{12}   &  \sigma_{13} \cr
 \sigma_{12} &  \sigma_{22}   &  \sigma_{23} \cr
 \sigma_{13} &  \sigma_{23}   &  \sigma_{33} \cr
} \rp
}

% strain tensor
\def\eeten{{\bf \epsilon}}

\def\eemat{
\lp \matrix {
 \epsilon_{11} &  \epsilon_{12}   &  \epsilon_{13} \cr
 \epsilon_{12} &  \epsilon_{22}   &  \epsilon_{23} \cr
 \epsilon_{13} &  \epsilon_{23}   &  \epsilon_{33} \cr
} \rp
}


% plane wave kernel
\def\pwker{A e^{i k \lp \nn \cdot \xx - v t \rp}}


% ------------------------------------------------------------
% details for expert audience (math, cartoons)
\def\expert{
\colorbox{red}{\textbf{\LARGE \white{!}}}
}

% ------------------------------------------------------------
% image, data, wavefields

\def\RR{R}

\def\UU{W}
\def\US{{\UU_s}}
\def\UR{{\UU_r}}

\def\DD{D}
\def\DS{{\DD_s}}
\def\DR{{\DD_r}}

\def\UUw{\UU}
\def\USw{{\UU_s}}
\def\URw{{\UU_r}}

\def\DDw{\DD}
\def\DSw{{\DD_s}}
\def\DRw{{\DD_r}}

% perturbations

\def\ds{\Delta s}
\def\dl{\Delta l}
\def\di{\Delta \RR}
\def\du{\Delta \UU}

\def\dRR{\Delta \RR}
\def\dUU{\Delta \UU}
\def\dUS{\Delta \US}
\def\dUR{\Delta \UR}

\def\dtt{\Delta \tt}
\def\dhh{\Delta \hh}

% ------------------------------------------------------------
% Green's functions

\def\GG{G}

\def\GS{{\GG_s}}
\def\GR{{\GG_r}}

% ------------------------------------------------------------
% elastic data, wavefields

\def\eRR{\textbf{\RR}}

\def\eDS{{\textbf{\DD}_s}}
\def\eDR{{\textbf{\DD}_r}}
\def\eDD{{\textbf{\DD}}}

\def\eUS{{\textbf{\UU}_s}}
\def\eUR{{\textbf{\UU}_r}}
\def\eUU{{\textbf{\UU}}}

% ------------------------------------------------------------
% sliding bar
\def\tline#1{
\put(95,-3){\small \blue{time}}
\put(-4,-1){\small \blue{0}}
\thicklines
\put( 0,0){\color{blue} \vector(1,0){100}}
\put(#1,0){\color{red}  \circle*{2}}
}

% ------------------------------------------------------------
% arrow on figure
\def\myarrow#1#2#3{
\thicklines
\put(#1,#2){\color{green} \vector(-1,-1){5}}
\put(#1,#2){\color{green} \textbf{#3}}
}

\def\bkarrow#1#2#3{
\thicklines
\put(#1,#2){\color{black} \vector(-1,-1){5}}
\put(#1,#2){\color{black} \textbf{#3}}
}

\def\anarrow#1#2#3#4{
\thicklines
\put(#1,#2){\color{#4} \vector(-1,-1){5}}
\put(#1,#2){\color{#4} \textbf{#3}}
}

\def\myvec#1#2#3#4{
\thicklines
\put(#1,#2){\color{green} \rotatebox{#4}{\vector(4,0){20}}}
\put(#1,#2){\color{green} \textbf{#3}}
}

% ------------------------------------------------------------
% circle on figure
\def\mycircle#1#2#3{
\thicklines
\put(#1,#2){\color{green} \circle{#3}}
}

% ------------------------------------------------------------
% note on figure
\def\mynote#1#2#3{
\put(#1,#2){\color{green} \textbf{#3}}
}

\def\biglabel#1#2#3{
\put(#1,#2){\Large \textbf{#3}}
}

\def\wlabel#1#2#3{ \white{ \biglabel{#1}{#2}{#3} }}
\def\klabel#1#2#3{ \black{ \biglabel{#1}{#2}{#3} }}
\def\rlabel#1#2#3{ \red{   \biglabel{#1}{#2}{#3} }}
\def\glabel#1#2#3{ \green{ \biglabel{#1}{#2}{#3} }}
\def\blabel#1#2#3{ \blue { \biglabel{#1}{#2}{#3} }}
\def\ylabel#1#2#3{ \yellow{\biglabel{#1}{#2}{#3} }}

% ------------------------------------------------------------
% centering
\def\cen#1{ \begin{center} \textbf{#1} \end{center}}
\def\cit#1{ \begin{center} \textit{#1} \end{center}}
\def\ctt#1{ \begin{center} \texttt{#1} \end{center}}

% emphasis (bold+alert)
\def\bld#1{ \textbf{\alert{#1}}}

% huge fonts
\def\big#1{\begin{center} {\LARGE \textbf{#1}} \end{center}}
\def\hug#1{\begin{center} {\Huge  \textbf{#1}} \end{center}}

% ------------------------------------------------------------
% separator
\def\sep{ \vfill \hrule \vfill}
\def\itab{ \hspace{0.5in}}
\def\nsp{\\ \vspace{0.1in}}

% ------------------------------------------------------------
% integrals

\def\tint#1{\!\!\!\int\!\! #1 dt}
\def\xint#1{\!\!\!\int\!\! #1 d\xx}
\def\wint#1{\!\!\!\int\!\! #1 d\ww}
\def\aint#1{\!\!\!\alert{\int}\!\! #1 d\alert{\xx}}

\def\esum#1{\sum\limits_{#1}}
\def\eint#1{\int\limits_{#1}}

% ------------------------------------------------------------
\def\CONJ#1{\overline{#1}}
\def\MOD#1{\left| {#1} \right|}

% ------------------------------------------------------------
% imaging components

\def\IC{\colorbox{yellow}{\textbf{I.C.}}\;}
\def\WR{\colorbox{yellow}{\textbf{W.R.}}\;}
\def\WE{\colorbox{yellow}{\textbf{W.E.}}\;}
\def\SO{\colorbox{yellow}{\textbf{SOURCE}}\;}
\def\WS{\colorbox{yellow}{\textbf{W.S.}}\;}

% ------------------------------------------------------------
% summary/take home message
\def\thm{take home message}

% ------------------------------------------------------------
\def\dx{\Delta x}
\def\dy{\Delta y}
\def\dz{\Delta z}
\def\dt{\Delta t}

\def\dhx{\Delta h_x}
\def\dhy{\Delta h_y}

\def\kz{{k_z}}
\def\kx{{k_x}}
\def\ky{{k_y}}

\def\kmx{k_{m_x}}
\def\kmy{k_{m_y}}
\def\khx{k_{h_x}}
\def\khy{k_{h_y}}

\def\why{ \alert{\widehat{{\khy}}}}
\def\whx{ \alert{\widehat{{\khx}}}}

\def\lx{{\lambda_x}}
\def\ly{{\lambda_y}}
\def\lz{{\lambda_z}}

\def\klx{k_{\lambda_x}}
\def\kly{k_{\lambda_y}}
\def\klz{k_{\lambda_z}}

\def\mx{{m_x}}
\def\my{{m_y}}
\def\mz{{m_z}}
\def\hx{{h_x}}
\def\hy{{h_y}}
\def\hz{{h_z}}

\def\sx{{s_x}}
\def\sy{{s_y}}
\def\rx{{r_x}}
\def\ry{{r_y}}

% ray parameter (absolute value)
\def\modp#1{\left| \pp_{#1} \right|}

% wavenumber
\def\modk#1{\left| \kk_{#1} \right|}

% ------------------------------------------------------------
\def\kzwk{ {\kz^{\kk}}}
\def\kzwx{ {\kz^{\xx}}}
 
\def\PSk#1{e^{\red{#1 i \kzwk \dz}}}
\def\PSx#1{e^{\red{#1 i \kzwx \dz}}}
\def\PS#1{ e^{\red{#1 i k_z   \dz}}}

\def\TT{t}
\def\TS{t_s}
\def\TR{t_r}

\def\oft{\lp t \rp}
\def\ofw{\lp \ww \rp}

\def\ofx{\lp \xx \rp}
\def\ofh{\lp \hh \rp}
\def\ofk{\lp \kk \rp}
\def\ofs{\lp \ss \rp}
\def\ofr{\lp \rr \rp}
\def\ofz{\lp   z \rp}

\def\ofxt{\lp \xx, t  \rp}
\def\ofst{\lp \ss, t  \rp}
\def\ofrt{\lp \rr, t  \rp}

\def\ofxw{\lp \xx, \ww  \rp}
\def\ofsw{\lp \ss, \ww  \rp}
\def\ofrw{\lp \rr, \ww  \rp}

\def\ofxm{\lp \xx,\hh \rp}

\def\ofxmp{\lp \xx+\hh \rp}
\def\ofxmm{\lp \xx-\hh \rp}

\def\ofmm{\lp \mm      \rp}
\def\ofmz{\lp \mm, z   \rp}
\def\ofmw{\lp \mm, \ww \rp}
\def\ofkm{\lp \kkm     \rp}

% ------------------------------------------------------------
% source/receiver data and wavefields

\def\dst{$\DS\ofst$}
\def\drt{$\DR\ofrt$}
\def\ust{$\US\ofxt$}
\def\urt{$\UR\ofxt$}

\def\dsw{$\DS\ofsw$}
\def\drw{$\DR\ofrw$}
\def\usw{$\US\ofxw$}
\def\urw{$\UR\ofxw$}

% ------------------------------------------------------------
\def\Nx{N_x}
\def\Ny{N_y}
\def\Nz{N_z}
\def\Nt{N_t}
\def\Nw{N_{\ww}}
\def\Nm{N_{\mm}}

\def\Nlx{N_{\lambda_x}}
\def\Nly{N_{\lambda_y}}
\def\Nlz{N_{\lambda_z}}
\def\Nlt{N_{\tau}}

\def\wmin{\ww_{min}}
\def\wmax{\ww_{max}}
\def\zmin{z_{min}}
\def\zmax{z_{max}}
\def\tmin{t_{min}}
\def\tmax{t_{max}}
\def\lmin{\hh_{min}}
\def\lmax{\hh_{max}}
\def\xmin{\xx_{min}}
\def\xmax{\xx_{max}}

% ------------------------------------------------------------
% course qualifiers

\def\fun{\hfill \alert{concepts}}
\def\pra{\hfill \alert{applications}}
\def\fro{\hfill \alert{frontiers}}


% ------------------------------------------------------------
% wavefield extrapolation
\def\ws{ {\ww s} }

\def\kows{\lp \frac{\kx}{\ws} \rp}

\def\kmws{\lp \frac{\modk{\mm}}{\ws} \rp}
\def\kzws{\lp \frac{\kz}       {\ws} \rp}

\def\S{\lb\frac{\modk{\mm}}{\ws  }\rb}
\def\C{\lb\frac{\modk{\mm}}{\ws_0}\rb}
\def\K{\lb\frac{\modk{\mm}}{\ww  }\rb}

\def\Cs{\lb\frac{\modk{\mm}^2}{\lp \ws_0 \rp^2}\rb}

\def\SSR#1{  \sqrt{ \lp \ww {#1} \rp^2 - \modk{\mm}^2} }

\def\SQRsum#1{\sum\limits_{n=1}^{\infty} \lp -1 \rp^n
		\displaystyle{\frac{1}{2} \choose n} #1}

\def\TSE#1#2#3#4{\sum\limits_{#4=#3}^{\infty} \lp -1 \rp^#4
		\displaystyle{#2 \choose #4} {#1}^#4}

\def\onefrac#1#2{\frac{#2^2}{a_#1+b_#1 #2^2}}
\def\SQRfrac#1{
	\sum\limits_{n=1}^{\infty}
	\onefrac{n}{#1} }

\def\dkzds { \left. \frac{d {\kz}}  {d s} \right|_{s_b} }
\def\SSX#1#2{\sqrt{ 1 - \lb \frac{\MOD{#2}}{#1} \rb^2} }
\def\SST#1#2{1 + \sum_{j=1}^N c_j \lb \frac{\MOD{#2}}{#1} \rb^{2j} }

% ------------------------------------------------------------
% acknowledgment
\def\ackcwp{\cen{the sponsors of the\\Center for Wave Phenomena\\at\\Colorado School of Mines}}

% ------------------------------------------------------------
% citation in slides
\def\talkcite#1{{\small \sc #1}}

% ------------------------------------------------------------
\def\ise{GPGN302: Introduction to EM and Seismic Exploration}
\def\inv{GPGN409: Inversion}

% ------------------------------------------------------------
\def\model{m}
\def\data {d}

\def\Lop{ {\mathbf{L}}}
\def\Sop{ {\mathbf{S}}}
\def\Eop{ {\mathbf{E}}}
\def\Iop{ {\mathbf{I}}}
\def\Aop{ {\mathbf{A}}}
\def\Pop{ {\mathbf{P}}}
\def\Fop{ {\mathbf{F}}}
\def\Kop{ {\mathbf{K}}}


% ------------------------------------------------------------
\def\mybox#1{
  \begin{center}
    \fcolorbox{black}{yellow}
    {\begin{minipage}{0.8\columnwidth} {#1} \end{minipage}}
  \end{center}
}

\def\hibox#1{
  \begin{center}
    \fcolorbox{black}{LightGreen}
    {\begin{minipage}{0.8\columnwidth} {#1} \end{minipage}}
  \end{center}
}

% ------------------------------------------------------------
% Nota Bene
\def\nbnote#1{
  \vfill
  \begin{center}
    \colorbox{LightGray}
    {\begin{minipage}{\columnwidth} {\textbf{\black{\large N.B.}} #1} \end{minipage}}
  \end{center}
}

\def\notabene#1{
  \begin{leftbar}
    {\sc Nota Bene:~} #1
  \end{leftbar}
}

\def\sidebar#1{
  \begin{leftbar}
    {#1}
  \end{leftbar}
}


\def\highlight#1{
  \begin{center}
    \colorbox{LightRed}
    {\begin{minipage}{0.95\columnwidth} {#1} \end{minipage}}
  \end{center}
}

% ------------------------------------------------------------
\def\pcsshaded#1{
  \definecolor{shadecolor}{rgb}{0.8,0.8,0.8}
  \begin{shaded} {#1} \end{shaded}
  \definecolor{shadecolor}{rgb}{1.0,1.0,1.0}
}

\def\blueshade#1{
  \definecolor{shadecolor}{rgb}{0.690,0.886,1.000}
    \begin{shaded}
      {#1}
    \end{shaded}
  \definecolor{shadecolor}{rgb}{1.0,1.0,1.0}
}

\def\grayshade#1{
  \definecolor{shadecolor}{rgb}{0.8,0.8,0.8}
  \begin{shaded}
    {#1}
  \end{shaded}
  \definecolor{shadecolor}{rgb}{1.0,1.0,1.0}
}

\def\yellowshade#1{
  \definecolor{shadecolor}{rgb}{1.0,1.0,0.0}
  \begin{shaded}
    {#1}
  \end{shaded}
  \definecolor{shadecolor}{rgb}{1.0,1.0,1.0}
}


% ------------------------------------------------------------
\def\postit#1{
  \begin{center}
    \colorbox{yellow}
    {\begin{minipage}{0.80\columnwidth} {#1} \end{minipage}} 
  \end{center}
}

% ------------------------------------------------------------
\def\graybox#1{
  \begin{center}
    \colorbox{LightGray}
    {\begin{minipage}{1.00\columnwidth} {#1} \end{minipage}}
  \end{center}
}

\def\whitebox#1{
  \begin{center}
    \colorbox{white}
    {\begin{minipage}{1.00\columnwidth} {#1} \end{minipage}}
  \end{center}
}

\def\yellowbox#1{
  \begin{center}
    \colorbox{LightYellow}
    {\begin{minipage}{1.00\columnwidth} {#1} \end{minipage}}
  \end{center}
}

\def\greenbox#1{
  \begin{center}
    \colorbox{LightGreen}
    {\begin{minipage}{1.00\columnwidth} {#1} \end{minipage}}
  \end{center}
}

\def\bluebox#1{
  \begin{center}
    \colorbox{LightBlue}
    {\begin{minipage}{1.00\columnwidth} {#1} \end{minipage}}
  \end{center}
}

\def\redbox#1{
  \begin{center}
    \colorbox{LightRed}
    {\begin{minipage}{1.00\columnwidth} {#1} \end{minipage}}
  \end{center}
}

\def\hyellow#1{ \colorbox{yellow} #1 }
\def\hgreen #1{ \colorbox{green}  #1 }

% ------------------------------------------------------------
% boxes for vectors and matrices

\def\pcsbox#1#2#3#4{
  % #1 = hmax
  % #2 = height
  % #3 = width
  % #4 = text
  \begin{picture}(#3,#1)
    \linethickness{0.5mm}
    % 
    \multiput(0,#1)(#3, 0){2}{\line(0,-1){#2}}
    \multiput(0,#1)(0,-#2){2}{\line(+1,0){#3}}
    % 
    \put(1,-10){#4}
  \end{picture}
}

% annotate block equations
\def\pcssym#1#2{
  \begin{picture}(3,#1)
    \put(1,-10){#2}
  \end{picture}
}

% block equation sign
\def\pcsops#1#2#3#4{
  \begin{picture}(#3,#1)
    \put(0,#2){#4}
  \end{picture}
}

% ------------------------------------------------------------
% two color boxes
\def\sidebyside#1#2{
  \begin{center}
    \colorbox{LightBlue}{
      \begin{minipage}{1.0\columnwidth} {#1} \end{minipage}
    }
    \colorbox{LightYellow}{
      \begin{minipage}{1.0\columnwidth} {#2} \end{minipage}
    }
  \end{center}
}

% upward pointing arrow
\def\uparrow#1#2#3{
  \thicklines
  \put(#1,#2){\color{green} \vector(0,+1){5}}
  \put(#1,#2){\color{green} \textbf{#3}}
}

% acknowledge figure source
\def\ackfig#1#2#3{\blabel{#1}{#2}{\normalsize \sc #3}}

\pgfdeclareimage[height=1.0in]{logo}{Fig/Madagascar}

\title[]{\mg \\ command line usage}
\author[]{Paul Sava}
\institute{
  Center for Wave Phenomena \\
  Colorado School of Mines \\
  psava@mines.edu
}
\date{}
\logo{WSI}
\Large

% ------------------------------------------------------------
\mode<beamer> { \cwpcover }

% ------------------------------------------------------------
\begin{frame} \frametitle{\mg mission}

  \begin{itemize}
  \item \textbf{a research environment} \\
    \uncover<2->{\alert{--open-source software package--}}
    \vfill
  \item \textbf{a technology transfer tool} \\
    \uncover<3->{\alert{--reproducible numeric experiments system--}}
    \vfill
  \end{itemize}

\end{frame}
\cwpnote{}

% ------------------------------------------------------------
\begin{frame} \frametitle{why open source software?}

  \begin{columns}[t]
    \column{.5\textwidth} \cen{freedom}
    \cit{to use}
    \cit{to study}
    \cit{to modify}
    \cit{to redistribute}
    \cit{to improve}

    \column{.5\textwidth} \cen{reproducibility}
  \end{columns}

\end{frame}
\cwpnote{}

% ------------------------------------------------------------
\begin{frame} \frametitle{why numeric reproducibility?}

  \vfill
  \cen{collaborative research}
  \vfill
  \cen{technology transfer}
  \vfill
  \cen{education}
  \vfill
  \cen{peer review}
  \vfill

\end{frame}
\cwpnote{}


% ------------------------------------------------------------
\begin{frame} \frametitle{\mg system}

  \begin{columns}
    \column{.25\textwidth}
    \textbf{software}
    \textbf{documents} \\
    \textbf{blog} \\
    \textbf{merchandise} \\
    \dots
    \column{.75\textwidth}
    \alert{ \texttt{http://www.ahay.org} }
  \end{columns}

  \sep

  \begin{itemize}
  \item standard GPL license
  \item distributed development (academia/industry)
  \item version control (\texttt{git})
  \item installation and flows (\texttt{scons})
  \end{itemize}

\end{frame}
\cwpnote{}


% ------------------------------------------------------------
\begin{frame} \frametitle{\mg developers}

  \begin{itemize}
  \item University of Texas (Austin)
  \item Colorado School of Mines
  \item King Abdullah University of Science and Technology
  \item \dots
  \end{itemize}

\end{frame}
\cwpnote{}


% ------------------------------------------------------------
\begin{frame} \frametitle{\mg heritage}

  \textbf{software system}
  \begin{itemize}
  \item SEPlib (Stanford University)
  \item SU (Colorado School of Mines)
  \item DDS (Amoco/BP)
  \end{itemize}

  \vfill

  \textbf{reproducible documents}
  \begin{itemize}
  \item SEP document system
  \end{itemize}

\end{frame}
\cwpnote{}

% ------------------------------------------------------------
\begin{frame} \frametitle{\mg architecture}

  \begin{columns}
    \column{.25\textwidth}
    documents \\
    \vspace{0.25in}
    flows \\
    \vspace{0.25in}
    programs

    \column{.75\textwidth}
    $3$ overlapping layers

  \end{columns}

\end{frame}
\cwpnote{}

% ------------------------------------------------------------
\begin{frame} \frametitle{\mg architecture}

  \begin{columns}
    \column{.25\textwidth}
    documents \\
    \vspace{0.25in}
    flows \\
    \vspace{0.25in}
    \textbf{programs}

    \column{.75\textwidth}
    \begin{itemize}
    \item provide basic processing modules
    \item C, C++, F90, Python, Matlab \dots
    \item communicate by pipes
    \end{itemize}

  \end{columns}

\end{frame}
\cwpnote{}

% ------------------------------------------------------------
\begin{frame} \frametitle{\mg architecture}

  \begin{columns}
    \column{.25\textwidth}
    documents \\
    \vspace{0.25in}
    \textbf{flows} \\
    \vspace{0.25in}
    programs

    \column{.75\textwidth}
    \begin{itemize}
    \item provide processing history
    \item Python (SCons)
    \item combine processing modules
    \end{itemize}

  \end{columns}

\end{frame}
\cwpnote{}

% ------------------------------------------------------------
\begin{frame} \frametitle{\mg architecture}

  \begin{columns}
    \column{.25\textwidth}
    \textbf{documents} \\
    \vspace{0.25in}
    flows \\
    \vspace{0.25in}
    programs

    \column{.75\textwidth}
    \begin{itemize}
    \item provide reproducible documents
    \item \LaTeX\; and SCons
    \item assemble text and numeric results
    \end{itemize}

  \end{columns}

\end{frame}
\cwpnote{}


% ------------------------------------------------------------
\begin{frame} \frametitle{programs}

  \begin{itemize}
  \item independent processing modules
  \item combined using pipes
  \item ``sf'' prefix
  \item documented by examples (``books'')
  \item implement SMP parallelism
  \item program count: $1757$ on 8/21/2018
  \end{itemize}

\end{frame}
\cwpnote{}

% ------------------------------------------------------------
\begin{frame}[fragile] \frametitle{program list}

  \mex{ {\bf sfdoc} -k .}

  \tiny
  \begin{semiverbatim}
    sfofpwd: Objective function of dip estimation with PWD filters.
    sfinfill: Shot interpolation.
    sfslice: Extract a slice using picked surface (usually from a stack or a semblance).
    sfin: Display basic information about RSF files.
    sfdmo: Kirchhoff DMO with antialiasing by reparameterization.
    sfradstretch: Stretch of the time axis.
    sflpef: Find PEF on aliased traces.
    sfrefer: Subtract a reference from a grid.
    sflevint: Leveler inverse interpolation in 1-D.
    sfnoise: Add random noise to the data.
    ...
  \end{semiverbatim}

\end{frame}
\cwpnote{}

% ------------------------------------------------------------
\begin{frame} \frametitle{self documentation}

  \begin{itemize}
  \item run program without arguments
  \item find program purpose
  \item find execution parameters
  \item find execution examples
  \end{itemize}

\end{frame}
\cwpnote{}

% ------------------------------------------------------------
\begin{frame}[fragile] \frametitle{example}

  \mex{ {\bf sfspike} }

\tiny
\begin{semiverbatim}
NAME
        sfspike
DESCRIPTION
        Generate simple data: spikes, boxes, planes, constants.
SYNOPSIS
        sfspike > spike.rsf mag= nsp=1 k#=[0,...] l#=[k1,k2,...] p#=[0,...] n#=
o#=(0,...) d#=(0.004,0.1,0.1,...) label#=(Time,Distance,Distance,...) unit#=[s,km,km,...] title=
PARAMETERS
        float   d#=(0.004,0.1,0.1,...)  sampling on #-th axis
        ints    k#=[0,...]      spike starting position  [nsp]
        ints    l#=[k1,k2,...]  spike ending position  [nsp]
        string  label#=(Time,Distance,Distance,...)     label on #-th axis
        floats  mag=    spike magnitudes  [nsp]
        int     n#=     dimension of #-th axis
        int     nsp=1   Number of spikes
        float   o#=(0,...)      origin on #-th axis
        floats  p#=[0,...]      spike inclination (in samples)  [nsp]
        string  title=  title for plots
        string  unit#=[s,km,km,...]     unit on #-th axis
USED IN
        bei/conj/causint
        bei/dpmv/matt
        bei/dpmv/yalei
        bei/dwnc/vofz
	...
\end{semiverbatim}
\end{frame}
\cwpnote{}

% ------------------------------------------------------------
\begin{frame} \frametitle{program execution}
  \framesubtitle{single input, single output}

  \mex{
    $<$ i.rsf
    {\bf sfprog}
    arguments
    $>$ o.rsf
  }

  \vfill

  \begin{itemize}
  \item sfprog = \mg program
  \item arguments = program parameters
  \item input from \texttt{stdin} ($<$)
  \item output to \texttt{stdout} ($>$)
  \end{itemize}

\end{frame}
\cwpnote{}

% ------------------------------------------------------------
\begin{frame}[fragile] \frametitle{demo}

  \mex{
    {\bf sfspike}
    n1=100 o1=0 d1=0.01 n2=50 o2=1000 d2=10
    $>$ f1.rsf
  }

  \vfill
  \begin{itemize}
  \item standard in: none
  \item standard out: \mex{f1.rsf}
  \end{itemize}

\end{frame}
\cwpnote{}

% ------------------------------------------------------------
\begin{frame} \frametitle{file format}

  \textbf{header}:
  \begin{itemize}
  \item text file (description of data)
  \item description of regularly-sampled format
  \item small, can be archived
  \end{itemize}

  \textbf{binary}:
  \begin{itemize}
  \item binary file (actual data)
  \item regularly-sampled data (native binary or XDR binary)
  \item large, can be stored on a different file system
  \item path to binary set with environment variable \texttt{DATAPATH}
  \end{itemize}

\end{frame}
\cwpnote{}

% ------------------------------------------------------------
\inputdir{XFig}
% ------------------------------------------------------------

% ------------------------------------------------------------
\begin{frame} \frametitle{file format}
  \plot{rsfdata}{width=\textwidth}{}
\end{frame}
\cwpnote{}

% ------------------------------------------------------------
\begin{frame}[fragile] \frametitle{example}

  \mex{ {\bf sfin} }

\tiny
\begin{semiverbatim}
NAME
        sfin
DESCRIPTION
        Display basic information about RSF files.
SYNOPSIS
        sfin info=true check=2. trail=true file1.rsf file2.rsf ...
COMMENTS
        n1,n2,... are data dimensions
        o1,o2,... are axis origins
        d1,d2,... are axis sampling intervals
        label1,label2,... are axis labels
        unit1,unit2,... are axis units

PARAMETERS
        float   check=2.        Portion of the data (in Mb) to check for zero values.
        bool    info=y [y/n]    If n, only display the name of the data file.
        bool    trail=y [y/n]   If n, skip trailing dimensions of  one
USED IN
        data/sigsbee/fs2B
        data/sigsbee/nfs2B
	...
SOURCE
        filt/main/in.c
\end{semiverbatim}
\end{frame}
\cwpnote{}

% ------------------------------------------------------------
\begin{frame}[fragile] \frametitle{demo}

  \mex{ {\bf sfin} f1.rsf}

\tiny
\begin{semiverbatim}
f1.rsf:
    in="/scratch/f1.rsf@"
    esize=4 type=float form=native
    n1=100         d1=0.01        o1=0          label1="Time" unit1="s"
    n2=50          d2=10          o2=1000       label2="Distance" unit2="km"
        5000 elements 20000 bytes
\end{semiverbatim}
\end{frame}
\cwpnote{}

% ------------------------------------------------------------
\begin{frame}

  \textbf{dataset} described by:
  \begin{itemize}
  \item header: \texttt{f1.rsf}
  \item binary: \texttt{in="/scratch/f1.rsf@"}
  \end{itemize}

  \vfill

  \textbf{axis} described by:
  \begin{itemize}
  \item $n$: number of samples
  \item $o$: sampling origin
  \item $d$: sampling rate (delta)
  \item $label$: axis label
  \item $unit$: axis unit
  \end{itemize}

\end{frame}
\cwpnote{}

% ------------------------------------------------------------
\begin{frame}[fragile] \frametitle{demo}

  \mex{ $<$ f1.rsf {\bf sfattr} }

\tiny
\begin{semiverbatim}
*******************************************
rms = 1
mean value = 1
norm value = 70.7107
variance = 0
standard deviation = 0
maximum value = 1 at 1 1
minimum value = 1 at 1 1
number of nonzero samples = 5000
total number of samples = 5000
*******************************************
\end{semiverbatim}

\end{frame}
\cwpnote{}

% ------------------------------------------------------------
\begin{frame} \frametitle{compatibility}

  \begin{itemize}
  \item SEPlib: identical format
    \\
    \mex{ {\bf In} f1.rsf}
  \item SU: use converters
    \\
    \mex{ {\bf sfsegyread}  tape=f1.su su=y tfile=tfile.rsf endian=0
      \\ $>$ f1.rsf }
    \vspace{0.25in}
    \\
    \mex{ {\bf sfsegywrite} tape=f1.su su=y tfile=tfile.rsf endian=0
      \\ $<$ f1.rsf }
  \end{itemize}

\end{frame}
\cwpnote{}

% ------------------------------------------------------------
\begin{frame}[fragile] \frametitle{demo}

  \mex{ $<$ f1.rsf {\bf sfwindow} n2=25 min2=1200 $>$ f2.rsf}

  \vfill

  \mex{ {\bf sfin} f1.rsf}
\tiny
\begin{semiverbatim}
f1.rsf:
    in="/scratch/f1.rsf@"
    esize=4 type=float form=native
    n1=100         d1=0.01        o1=0          label1="Time" unit1="s"
    n2=50          d2=10          o2=1000       label2="Distance" unit2="km"
        5000 elements 20000 bytes
\end{semiverbatim}
\large

  \vfill

  \mex{ {\bf sfin} f2.rsf}
\tiny
\begin{semiverbatim}
f2.rsf:
    in="/scratch/f2.rsf@"
    esize=4 type=float form=native
    n1=100         d1=0.01        o1=0          label1="Time" unit1="s"
    n2=25          d2=10          o2=1200       label2="Distance" unit2="km"
        2500 elements 10000 bytes
\end{semiverbatim}
\end{frame}
\cwpnote{}


% ------------------------------------------------------------
\begin{frame} \frametitle{program execution}
  \framesubtitle{multiple inputs, multiple outputs}

  \mex{
    $<$ i.rsf
    {\bf sfprog}
    arguments
    l1=f1.rsf
    l2=f2.rsf
    $>$ o.rsf }

  \begin{itemize}
  \item input from \texttt{stdin} ($<$)
  \item output to \texttt{stdout} ($>$)
  \item \texttt{f1.rsf} can be open for input and/or output
  \item \texttt{f2.rsf} can be open for input and/or output
  \end{itemize}

\end{frame}
\cwpnote{}

% ------------------------------------------------------------
\begin{frame}[fragile] \frametitle{example}

  \mex{ {\bf sfbandpass} }

\tiny
\begin{semiverbatim}
NAME
        sfbandpass
DESCRIPTION
        Bandpass filtering.
SYNOPSIS
        sfbandpass < in.rsf > out.rsf flo= fhi= phase=n verb=n nplo=6 nphi=6
PARAMETERS
        float   fhi=    High frequency in band, default is Nyquist
        float   flo=    Low frequency in band, default is 0
        int     nphi=6  number of poles for high cutoff
        int     nplo=6  number of poles for low cutoff
        bool    phase=n [y/n]   y: minimum phase, n: zero phase
        bool    verb=n [y/n]    verbosity flag
SOURCE
        system/generic/Mbandpass.c
\end{semiverbatim}
\end{frame}
\cwpnote{}

% ------------------------------------------------------------
\begin{frame} \frametitle{pipes}

  \begin{itemize}
  \item \mg programs can be piped
  \item \texttt{stdout} from one program becomes \texttt{stdin} for the next
  \item no intrinsic limit for the number of pipes
  \end{itemize}

\end{frame}
\cwpnote{}

% ------------------------------------------------------------
\begin{frame}[fragile] \frametitle{demo}

  \mex{ $<$ f1.rsf
    {\bf sfwindow} n2=25 min2=1200 $|$
    {\bf sftransp}
    $>$ f3.rsf
  }

  \vfill
  \mex{ {\bf sfin} f1.rsf}
\tiny
\begin{semiverbatim}
f1.rsf:
    in="/scratch/f1.rsf@"
    esize=4 type=float form=native
    n1=100         d1=0.01        o1=0          label1="Time" unit1="s"
    n2=50          d2=10          o2=1000       label2="Distance" unit2="km"
        5000 elements 20000 bytes
\end{semiverbatim}
\large

  \vfill
  \mex{ {\bf sfin} f3.rsf}
\tiny
\begin{semiverbatim}
f3.rsf:
    in="/scratch/f3.rsf@"
    esize=4 type=float form=native
    n1=25          d1=10          o1=1200       label1="Distance" unit1="km"
    n2=100         d2=0.01        o2=0          label2="Time" unit2="s"
        2500 elements 10000 bytes
\end{semiverbatim}
\end{frame}
\cwpnote{}

% ------------------------------------------------------------
\begin{frame}
  \big{useful utilities}
\end{frame}
\cwpnote{}


% ------------------------------------------------------------
\begin{frame}[fragile] \frametitle{example}

  \mex{ {\bf sfmath} }
  \tiny
\begin{semiverbatim}
NAME
        sfmath
DESCRIPTION
        Mathematical operations on data files.
SYNOPSIS
        sfmath > out.rsf type= unit= output=
COMMENTS

        Known functions: cos,  sin,  tan,  acos,  asin,  atan,
                         cosh, sinh, tanh, acosh, asinh, atanh,
                         exp,  log,  sqrt, abs, conj (for complex data).

        sfmath will work on float or complex data, but all the input and output
        files must be of the same data type.

        Examples:

        sfmath x=file1.rsf y=file2.rsf power=file3.rsf output='sin((x+2*y)^power)' > out.rsf
        sfmath < file1.rsf tau=file2.rsf output='exp(tau*input)' > out.rsf
        sfmath n1=100 type=complex output="exp(I*x1)"

        See also: sfheadermath.

PARAMETERS
        string  output=         Mathematical description of the output
        string  type=   output data type [float,complex]
        string  unit=
USED IN
        bei/dpmv/matt
        bei/dwnc/sigmoid
        bei/ft1/autocor
	...
\end{semiverbatim}
\end{frame}
\cwpnote{}

% ------------------------------------------------------------
\begin{frame}[fragile] \frametitle{demo}

  \mex{
    {\bf sfmath}
    n1=1000 output='sin(0.5*x1)'
    $>$ s1.rsf
  }

  \vfill

  \mex{ {\bf sfin} s1.rsf}
\tiny
\begin{semiverbatim}
s1.rsf:
    in="/scratch/s1.rsf@"
    esize=4 type=float form=native
    n1=1000        d1=1           o1=0
        1000 elements 4000 bytes
\end{semiverbatim}
\large

\end{frame}
\cwpnote{}

% ------------------------------------------------------------
\begin{frame}[fragile] \frametitle{demo}

  \mex{
    {\bf sfmath}
    n1=300 n2=200 output='sin(0.25*x1+1*x2)'
    $>$ s2.rsf
  }

  \vfill

  \mex{ {\bf sfin} s2.rsf}
\tiny
\begin{semiverbatim}
s2.rsf:
    in="/scratch/s2.rsf@"
    esize=4 type=float form=native
    n1=300         d1=1           o1=0
    n2=200         d2=1           o2=0
        60000 elements 240000 bytes
\end{semiverbatim}
\large

\end{frame}
\cwpnote{}

% ------------------------------------------------------------
\begin{frame} \frametitle{plotting}

\begin{itemize}
   \item {\bf sfgraph}: 1D graphs
   \item {\bf sfgrey}: 2D/3D grayscale graphs
   \item {\bf contour}: contour plots
   \item {\bf sfgrey3}: cube plots
   \item ...
\end{itemize}

\end{frame}
\cwpnote{}

% ------------------------------------------------------------
\begin{frame}[fragile] \frametitle{demo}

  \mex{ {\bf sfin} s1.rsf}

\tiny
\begin{semiverbatim}
s1.rsf:
    in="/scratch/s1.rsf@"
    esize=4 type=float form=native
    n1=1000        d1=1           o1=0
        1000 elements 4000 bytes
\end{semiverbatim}
\large

  \vfill

  \mex{ $<$ s1.rsf {\bf sfgraph} title='' $|$ {\bf xtpen}}

\end{frame}
\cwpnote{}

% ------------------------------------------------------------
\begin{frame}[fragile] \frametitle{demo}

  \mex{ {\bf sfin} s2.rsf}

\tiny
\begin{semiverbatim}
s2.rsf:
    in="/scratch/s2.rsf@"
    esize=4 type=float form=native
    n1=300         d1=1           o1=0
    n2=200         d2=1           o2=0
        60000 elements 240000 bytes
\end{semiverbatim}
\large

  \vfill

  \mex{ $<$ s2.rsf {\bf sfgrey} title='' $|$ {\bf xtpen}}
\end{frame}
\cwpnote{}

% ------------------------------------------------------------
\begin{frame}
  \big{exercise}
\end{frame}
\cwpnote{}

% ------------------------------------------------------------
\inputdir{gauss}
% ------------------------------------------------------------

% ------------------------------------------------------------
\begin{frame}[fragile]
  \frametitle{create 2D Gaussian function}

  \mex{
    {\bf sfmath} output="exp(-(x1*x1+x2*x2)/(2*1.5*1.5))" \\
    n1=200 d1=0.1 o1=-10. n2=200 d2=0.1 o2=-10. $|$ \\
    {\bf sfput} label1=z unit1=km label2=x unit2=km $>$ gg.rsf
  }

  \vfill

  \mex{
    $<$ gg.rsf {\bf sfgrey} pclip=100 screenratio=1 title='' $|$
    {\bf xtpen}
  }

\end{frame}
\cwpnote{}

% ------------------------------------------------------------
\begin{frame}
  \plot{gg}{height=\textheight}{}
\end{frame}
\cwpnote{}

% ------------------------------------------------------------
\begin{frame}[fragile]
  \frametitle{extract 1D subset}

  \mex{
    $<$ gg.rsf {\bf sfwindow} n2=1 f2=100 $>$ gg0.rsf
 }

  \vfill

  \mex{
    $<$ gg0.rsf {\bf sfgraph} title='' $|$
    {\bf xtpen}
  }

\end{frame}
\cwpnote{}

% ------------------------------------------------------------
\begin{frame}
  \plot{gg0}{width=\textwidth}{}
\end{frame}
\cwpnote{}

% ------------------------------------------------------------
\begin{frame}[fragile]
  \frametitle{create a velocity model}

  \mex{
    $<$ gg.rsf {\bf sfmath} output="3-input" $>$ vel.rsf
  }

  \vfill

  \mex{
    $<$ vel.rsf {\bf sfgrey} title='' pclip=100 screenratio=1 $|$
    {\bf xtpen}
  }

\end{frame}
\cwpnote{}

% ------------------------------------------------------------
\begin{frame}
  \plot{vel}{height=\textheight}{}
\end{frame}
\cwpnote{}

% ------------------------------------------------------------
\begin{frame}[fragile]
  \frametitle{compute traveltimes}

  \mex{
    $<$ vel.rsf {\bf sfeikonal} zshot=-10 yshot=0 $>$ fme.rsf
  }

  \vfill

  \mex{
    $<$ fme.rsf {\bf sfcontour} title='' nc=200 screenratio=1 $|$
    {\bf xtpen}
  }

\end{frame}
\cwpnote{}

% ------------------------------------------------------------
\begin{frame}
  \plot{fme}{height=\textheight}{}
\end{frame}
\cwpnote{}

% ------------------------------------------------------------
\begin{frame}[fragile]
  \frametitle{compute rays and wavefronts}

  \mex{
    $<$ vel.rsf {\bf sfhwt2d} xsou=0 zsou=-10 \\
    nt=1000 ot=0 dt=0.01 ng=1801 og=-90 dg=0.1 $>$ hwt.rsf
  }

  \vfill

  \mex{
    $<$ hwt.rsf {\bf sfwindow} j1=20 j2=20 $|$ \\
    {\bf sfgraph} title='' yreverse=y screenratio=1 \\
    min1=-10 max1=+10 min2=-10 max2=+10  $|$
    {\bf xtpen}
  }

\end{frame}
\cwpnote{}

% ------------------------------------------------------------
\begin{frame}
  \plot{hwt}{height=\textheight}{}
\end{frame}
\cwpnote{}

% ------------------------------------------------------------
\begin{frame}

  \big{\url{http://www.ahay.org}}

  \vfill

  \big{\texttt{\$RSF/book/rsf/school}}

  \vfill

  \begin{center}
    \pgfuseimage{logo}
  \end{center}

\end{frame}
\cwpnote{}
