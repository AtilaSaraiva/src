% ------------------------------------------------------------
\begin{abstract}
  This document demonstrates how numeric examples constructed using
  the \mg~software package can be integrated into a reproducible
  document generated using the \latex typesetting program. I use a
  simple modeling/migration exercise based on the exploding reflector
  model to illustrate the main features of this process.
\end{abstract}

% ------------------------------------------------------------
\section{Introduction}
The exploding reflector model, illustrated in \rfg{expref}, allows us
to perform zero offset modeling and migration for models of arbitrary
complexity \cite[]{Claerbout.iei.1985}. Under this model, the image is
described as a collection of points which ``explode'', i.e. become
sources, at the same time arbitrarily set to be the time origin. Data
are obtained at the receivers by forward simulation of acoustic waves
from the exploding reflectors. Likewise, images are obtained by
backward simulation of acoustic waves from the observed data.

% ------------------------------------------------------------
\inputdir{XFig}
\plot*{expref}{width=\textwidth}{The exploding reflector model (after
  \cite{Claerbout.iei.1985}).}
% ------------------------------------------------------------

Acoustic modeling and migration can be implemented using numeric
solutions to an acoustic wave-equation, for example a variable-density
wave-equation:
%
\begin{equation} \label{eqn:AWE}
  \frac{1}{v^2} \dtwo{\UU}{t} - 
  \rho \DIV{\lp \frac{1}{\rho} \GRAD{\UU} \rp} = f \;.
\end{equation}
%
In \rEq{AWE}, $\UU\ofxt$ represents the acoustic wavefield, $v\ofx$
and $\rho\ofx$ represent the velocity and density of the medium,
respectively, and $f\ofxt$ represents a source function.
\begin{itemize}
\item In \textbf{modeling}, we use the distributed source $f\ofxt$ to
  generate the wavefield $\UU\ofxt$ at all positions and all times by
  wave propagation forward in time. The data represent a subset of the
  wavefield observed at receivers distributed in the medium: 
  % 
  \beq
  \DD\lp \rr,t \rp=\UU\lp \xx=\rr,t \rp \;.
  \eeq
  % 
\item In migration, we use the observed data $\DD\lp \rr,t \rp$ to
  generate the wavefield $\UU\ofxt$ at all positions and all times by
  wave propagation backward in time. The image represents a subset of
  the wavefield at time zero: 
  % 
  \beq
  \RR\ofx=\UU\lp \xx=,t=0 \rp \;.
  \eeq
  % 
\end{itemize}
In both cases, we solve \rEq{AWE} with different initial conditions,
but with the same model, $v\ofx$ and $\rho\ofx$ and with the same
boundary conditions.

% ------------------------------------------------------------
\inputdir{sigsbee}
% ------------------------------------------------------------

% ------------------------------------------------------------
\section{Example}
I illustrate the zero-offset modelind and migration methodology using
the Sigsbee 2A synthetic model. This model is based on the Sigsbee
structure in the Gulf of Mexico and the velocity is illustrated in
\rFg{vstr}. The model is characterized by a massive salt body close to
the water bottom and surrounded by sediments. The salt velocity is
$4.5$~km/s and the surrounding sediment velocities range from
approximately $1.5$ to $3.25$~km/s.

\plot{vstr}{width=0.45\textwidth}{Stratigraphic Sigsbee 2A velocity model.}

In this experiment, I consider sources distributed uniformly in the
subsalt region of the model. The data are acquired in a borehole
array, located at $x=8.5$~km and in a horizontal array located at
$z=1.5$~km. In order to avoid multiple scattering in the subsurface, I
simulate waves with a smooth version of the Sigsbee model, illustrated
in \rFg{vsmo}, and with constant density.

\plot{vsmo}{width=0.45\textwidth}{Smooth Sigsbee 2A velocity model.}

Using the \mg program \texttt{sfawefd2d}, we can simulate wavefields
from the distributed sources. \rFgs{wfld-01}-\rfn{wfld-15} show
wavefield snapshots in order of increasing times. We can observe waves
propagating from all subsalt sources, interacting with the variable
velocity medium and arriving at the vertical and horizontal arrays.

\multiplot*{2}{wfld-01,wfld-03,wfld-05,wfld-07,wfld-09,wfld-11,wfld-13,wfld-15}
{width=0.45\textwidth}{Wavefield snapshots at increasing times.}

\rFgs{datH} and \rfn{wigH} show the data observed at the horizontal
array in variable density and wiggle plotting formats,
respectively. Similarly, \rFgs{datV} and \rfn{wigV} show the data
observed in the vertical array using the same plotting formats. The
data are just subsets of the same wavefields at the respective
receiver positions and capture the complications observed in the
wavefield, i.e. triplications due to lateral velocity variation.

\multiplot{1}{datH,wigH}{width=0.45\textwidth}{Acoustic data observed
  in the horizontal array.}

\multiplot{1}{datV,wigV}{width=0.45\textwidth}{Acoustic data observed
  in the vertical array.}

In zero-offset migration, we backprogate the acoustic wavefields using
the acquired data as boundary conditions. The image is the wavefield
at time zero. Since we can acquire data at different locations in
space, the reconstructed wavefields depend on the acquisition
geometry, thus limiting the illumination in the subsurface. Therefore,
the migrated images depend on the acquisition array, as illustrated in
\rFgs{imgH} and \rfn{imgV} for the horizontal and vertical arrays,
respectively.  We can also obtain images by migrating the data
observed in both the horizontal and vertical arrays, as illustrated in
\rFg{imgA}, thus increasing the acquisition aperture and the
subsurface illumination.

\multiplot{1}{imgH,imgV}{width=0.45\textwidth}{Migrated images for data
  acquired in (a) the horizontal array and (b) the vertical array.}

\plot{imgA}{width=0.45\textwidth}{Migrated image for data acquired in 
both the horizontal and vertical arrays.}

% ------------------------------------------------------------
\section{Conclusions}
The combination of \LaTeX\ and \mg allows geoscientists to generate
reproducible documents where the numeric examples can be verified by
any user with the same computer setup. This allows for transparent
peer-review, for recursive development and for technology transfer
between collaborative research groups.

% ------------------------------------------------------------
\section{Acknowledgments}
The reproducible numeric examples in this paper use the \mg~
open-source software package freely available from
http://www.ahay.org.

\bibliographystyle{seg}
\bibliography{BOOKS}
