\title{Madagascar tutorial: Writing technical papers by using Madagascar and LaTeX}

\author{Maurice the Aye-Aye\footnotemark[1]}

\address{
\footnotemark[1] King Julien's Royal Advisor, \\
Island of Madagascar, \\ Africa}

\lefthead{Aye-Aye}
\righthead{Tutorial}
\footer{Madagascar Documentation}

\maketitle

\begin{abstract}
From the viewpoint of science, all technical papers should be
"reproducible" in the sense that someone of reasonable skill ought to
be able to read the paper and then reproduce the results (from Joe
Dellinger). \texttt{SEGTeX} is a \LaTeX\ package for geophysical
publications, which is a important component in \textsc{Madagascar}
Project. \texttt{SEGTeX} consists of class files for Geophysics
papers, SEG expanded abstracts, etc, and cumulative bibliography of
geophysical publications. In this tutorial, you will dive into writing
part in the \textsc{Madagascar} architecture \cite[]{m8r} and learn
how to use the \textsc{Madagascar} and \texttt{SEGTeX} to write a
reproducible paper. By the end of this tutorial, you should have
learned to: 
\begin{enumerate} 
\item design a paper project by following \texttt{SEGTeX} rules,
\item insert figures from \textsc{Madagascar} workflow into papers, 
\item generate a PDF paper with several styles by using different 
\texttt{SEGTeX} classes,
\item revise a paper with notated signs.  
\end{enumerate}
\end{abstract}

\section{Prerequisites}

Completing this tutorial requires
\begin{itemize}
\item \textsc{Madagascar} software environment available from \\
\url{http://www.ahay.org}
\item \LaTeX\ environment with \texttt{SEGTeX} available from \\ 
\url{http://www.ahay.org/wiki/SEGTeX}
\end{itemize}
To do the assignment on your personal computer, you need to install
the required environments. An Internet connection is required for
access to the data repository.

The tutorial itself is available from the \textsc{Madagascar} repository
by running
\begin{verbatim}
svn co https://github.com/ahay/src/trunk/book/rsf/school2020 ~/school2020
\end{verbatim}
\section{Introduction}

In this tutorial, you will be asked to run commands from the Unix
shell (identified by \texttt{bash\$}) and to edit files in a text
editor. Different editors are available in a typical Unix environment
(\texttt{vi}, \texttt{emacs}, \texttt{nedit}, etc.)

Your first assignment:
\begin{enumerate}
\item Open a Unix shell.
\item Change directory to the tutorial directory
\begin{verbatim}
bash$ cd ~/school2020
\end{verbatim}
\item Open the \texttt{school2020.tex} file in your favorite editor,
  for example by running
\begin{verbatim}
bash$ emacs school2020.tex & 
\end{verbatim}
\item Look at the first line in the file and change the author name
  from Maurice the Aye-Aye to your name (first things first).
\end{enumerate}

\section{Examples}
\subsection{Gravity example}
\begin{enumerate}
\item Change directory to \texttt{gravity} directory
\begin{verbatim}
bash$ cd gravity
\end{verbatim}
\item Run
\begin{verbatim}
bash$ scons bin.view
\end{verbatim}

in the Unix shell. A number of commands will appear in the shell
followed by Figure~\ref{fig:bin} appearing on your screen.
\item To understand the commands, examine the script that generated them by 
opening the \texttt{SConstruct} file in a text editor. Notice that,
instead of Shell commands, the script contains rules.
\begin{itemize}
\item The first rule, \texttt{Fetch}, allows the script to download the 
input data file \texttt{gravity.asc} from the data server.
\item Other rules have the form \texttt{Flow(target,source,command)} for
generating data files or \texttt{Plot} and \texttt{Result} for
generating picture files.
\item \texttt{Fetch}, \texttt{Flow}, \texttt{Plot}, and \texttt{Result} 
are defined in \textsc{Madagascar}'s \texttt{rsf.proj} package, which
extends the functionality of \href{http://www.scons.org}{SCons} .
\end{itemize}
\item To better understand how rules translate into commands, run 
\begin{verbatim}
bash$ scons -c bin.rsf
\end{verbatim}
The \texttt{-c} flag tells scons to remove the \texttt{bin.rsf} file
and all its dependencies.
\item Next, run
\begin{verbatim}
bash$ scons -n bin.rsf
\end{verbatim}
The \texttt{-n} flag tells scons not to run the command but simply to
display it on the screen. Identify the lines in
the \texttt{SConstruct} file that generate the output you see on the
screen.
\item Run
\begin{verbatim}
bash$ scons bin.rsf
\end{verbatim}
  Examine the file \texttt{bin.rsf} both by opening it in a text
  editor and by running
\begin{verbatim}
bash$ sfin bin.rsf
\end{verbatim}

  When you view bin.rsf in the text editor, you see a history of all
  the programs used to create the file. The \texttt{sfin} program
  lists basic information about the file including data dimensions and
  extents of each axis.
\end{enumerate}

\inputdir{gravity}

\plot{bin}{width=.39\textwidth}{Relative gravity anomaly}.

\multiplot{4}{mtm,sxder,hxder,cont}{width=.39\textwidth}{Denoised 
              data (a), derivative along X axis (b), derivative along
              Y axis (c), and upward continuation (d).}

Figure~\ref{fig:bin} shows the data about relative gravity anomaly
from China. To improve the \old{SNR}\new{signal-to-noise ratio (SNR)} 
of the data, we used a 2D modified-trimmed-mean (MTM) filter 
\old{\mbox{\cite[]{Wu91}}}\new{\cite[]{Lee85}} to
attenuate random noise, the result is shown in
Figure~\ref{fig:mtm}.

For 1D case, one selects a filter window around each data point $x_i$,
where the filter-window length is $N$, and the median $m_i$ is expressed
as

      \old{
        \parbox{\textwidth}{
        \begin{equation}
        \label{eq:eq1}
          m_k = median[x_i] \qquad (k=i-(N-1)/2,\cdots,i+(N-1)/2)\;,
      \end{equation}
    }}

    \new{\begin{equation}
        \label{eq:eq1}
          m_i = median[x_k] \qquad (k=i-(N-1)/2,\cdots,i+(N-1)/2)\;,
      \end{equation}
    }
where filter-window length $N$ is an odd number. The MTM filter
averages only those samples in the window inside a value range
[$m_k-\delta,m_k+\delta$] as
      \begin{equation}
        \label{eq:eq2}
          mtm_i = mean[x_k|m_i-\delta \le x_k \le m_i+\delta] 
          \qquad (k=i-(N-1)/2,\cdots,i+(N-1)/2)\;,
      \end{equation}
where $mtm$ is MTM filter output and $\delta$ is threshold value. 

One can also calculate derivatives to distinguish the structure along
different directions. Figure~\ref{fig:sxder} and \ref{fig:hxder}
display the spatial derivatives along X and Y directions,
respectively. Finally, upward continuation method is also used to
highlight deep geological objectives (Figure~\ref{fig:cont}).

\subsection{Seismic example}
Add your text and insert the figures from ``seismic'' folder ...

\subsection{CSEM example}
Write your text and insert the figures from ``CSEM'' folder ...



\bibliographystyle{seg}
\bibliography{school2020}



