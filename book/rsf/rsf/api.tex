\title{Guide to RSF API}
\email{sergey.fomel@beg.utexas.edu}
\author{Sergey Fomel}
\lefthead{Fomel}
\righthead{RSF API}

\maketitle

\begin{abstract}

  This guide explains the RSF programming interface.

\end{abstract}

\section{Introduction}

To work with RSF files in your own programs, you may need to use an
appropriate programming interface. We will demonstrate the interface in
different languages using a simple example. The example is a clipping program.
It reads and writes RSF files and accesses parameters both from the input file
and the command line. The input is processed trace by trace. This is not
necessarily the most efficient approach\footnote{Compare with the \href{http://svn.sourceforge.net/viewcvs.cgi/rsf/trunk/filt/proc/Mclip.c?view=markup}{library clip program}.} but it suffices for a simple demonstration.

\section{C interface}

\lstset{language=c,numbers=left,numberstyle=\tiny,showstringspaces=false}

The C clip function is listed below.
\lstinputlisting[frame=single]{\RSF/api/clip.c}
Let us examine it in detail. 

\lstinputlisting[firstline=3,lastline=3,frame=single]{\RSF/api/clip.c}
The include preprocessing directive is required to access the RSF interface. 

\lstinputlisting[firstline=9,lastline=9,frame=single]{\RSF/api/clip.c}
RSF data files are defined with an abstract \texttt{sf\_file} data type. An
abstract data type means that the contents of it are not publicly declared,
and all operations on \texttt{sf\_file} objects should be performed with
library functions. This is analogous to \texttt{FILE *} data type used in
\texttt{stdio.h} and as close as C gets to an object-oriented style of
programming \cite[]{cbook}.

\lstinputlisting[firstline=11,lastline=12,frame=single]{\RSF/api/clip.c}
Before using any of the other functions, you must call
\texttt{sf\_init}. This function parses the command line and
initializes an internally stored table of command-line parameters.

\lstinputlisting[firstline=13,lastline=16,frame=single]{\RSF/api/clip.c}
The input and output RSF file objects are created with \texttt{sf\_input} and
\texttt{sf\_output} constructor functions. Both these functions take a string
argument. The string may refer to a file name or a file tag. For example, if
the command line contains \texttt{vel=velocity.rsf}, then both
\texttt{sf\_input("velocity.rsf")} and \texttt{sf\_input("vel")} are
acceptable. Two tags are special: \texttt{"in"} refers to the file in the
standard input and \texttt{"out"} refers to the file in the standard
output. 

\lstinputlisting[firstline=18,lastline=20,frame=single]{\RSF/api/clip.c} 
RSF files can store data of different types (character, integer,
floating point, complex). We extract the data type of the input file
with the library \texttt{sf\_gettype} function and check if it
represents floating point numbers. If not, the program is aborted with
an error message, using the \texttt{sf\_error} function.  It is
generally a good idea to check the input for user errors and, if they
cannot be corrected, to take a safe exit.

\lstinputlisting[firstline=22,lastline=26,frame=single]{\RSF/api/clip.c}
Conceptually, the RSF data model is a multidimensional hypercube. By
convention, the dimensions of the cube are stored in \texttt{n1=},
\texttt{n2=}, etc. parameters. The \texttt{n1} parameter refers to the
fastest axis. If the input dataset is a collection of traces,
\texttt{n1} refers to the trace length. We extract it using the
\texttt{sf\_histint} function (integer parameter from history) and
abort if no value for \texttt{n1} is found. We could proceed in a
similar fashion, extracting \texttt{n2}, \texttt{n3}, etc. If we are
interested in the total number of traces, like in the clip example, a
shortcut is to use the \texttt{sf\_leftsize} function. Calling
\texttt{sf\_leftsize(in,0)} returns the total number of elements in
the hypercube (the product of \texttt{n1}, \texttt{n2}, etc.), calling
\texttt{sf\_leftsize(in,1)} returns the number of traces (the product
of \texttt{n2}, \texttt{n3}, etc.), calling
\texttt{sf\_leftsize(in,2)} returns the product of \texttt{n3},
\texttt{n4}, etc. By calling \texttt{sf\_leftsize}, we avoid the need
to extract additional parameters for the hypercube dimensions that we
are not interested in.

\lstinputlisting[firstline=28,lastline=29,frame=single]{\RSF/api/clip.c} 
The clip parameter is read from the command line, where it can be
specified, for example, as \texttt{clip=10}. The parameter has the
\texttt{float} type, therefore we read it with the
\texttt{sf\_getfloat} function. If no \texttt{clip=} parameter is
found among the command line arguments, the program is aborted with an
error message using the \texttt{sf\_error} function.

\lstinputlisting[firstline=31,lastline=32,frame=single]{\RSF/api/clip.c} 
Next, we allocate an array of floating-point numbers to store a trace
with the library \texttt{sf\_floatalloc} function. Unlike the standard
\texttt{malloc} the RSF allocation function checks for errors and
either terminates the program or returns a valid pointer.

\lstinputlisting[firstline=34,lastline=48,frame=single]{\RSF/api/clip.c} 
The rest of the program is straightforward. We loop over all available
traces, read each trace, clip it and right the output out. The syntax
of \texttt{sf\_floatread} and \texttt{sf\_floatwrite} functions is
similar to the syntax of the C standard \texttt{fread} and
\texttt{fwrite} function except that the type of the element is
specified explicitly in the function name and that the input and
output files have the RSF type \texttt{sf\_file}.

\subsection{Compiling}

To compile the \texttt{clip} program, run
\begin{verbatim}
cc clip.c -I$RSFROOT/include -L$RSFROOT/lib -lrsf -lm
\end{verbatim}
Change \texttt{cc} to the C compiler appropriate for your system and include
additional compiler flags if necessary. The flags that RSF typically uses are
in \texttt{\$RSFROOT/lib/rsfconfig.py}.

\section{C++ interface}

\lstset{language=c++}

The C++ clip function is listed below.

\lstinputlisting[frame=single]{\RSF/api/clip.cc}
Let us examine it line by line. 

\lstinputlisting[firstline=4,lastline=4,frame=single]{\RSF/api/clip.cc}
Including ``\texttt{rsf.hh}'' is required for accessing the RSF C++
interface.

\lstinputlisting[firstline=8,lastline=8,frame=single]{\RSF/api/clip.cc}
A call to \texttt{sf\_init} is required to initialize the internally stored
table of command-line arguments.

\lstinputlisting[firstline=10,lastline=11,frame=single]{\RSF/api/clip.cc}
Two classes: \texttt{iRSF} and \texttt{oRSF} are used to define input and
output files. For simplicity, the command-line parameters are also handled 
as an \texttt{iRSF} object, initialized with zero.

\lstinputlisting[firstline=16,lastline=17,frame=single]{\RSF/api/clip.cc}
Next, we read the data dimensions from the input RSF file object called
\texttt{in}: the trace length is a parameter called ``\texttt{n1}'' and the
number of traces is the size of \texttt{in} remaining after excluding the
first dimension. It is extracted with the \texttt{size} method.

\lstinputlisting[firstline=19,lastline=19,frame=single]{\RSF/api/clip.cc} 
The clip parameter should be specified on the command line, for
example, as \texttt{clip=10}. It is extracted with the \texttt{get}
method of \texttt{iRSF} class from the \texttt{par} object.

\lstinputlisting[firstline=21,lastline=21,frame=single]{\RSF/api/clip.cc}
The trace object has the single-precision floating-point type and is a
1-D array of length \texttt{n1}. It is declared and allocated using
the \texttt{valarray} template class from the standard C++ library.

\lstinputlisting[firstline=23,lastline=32,frame=single]{\RSF/api/clip.cc}
Next, we loop through the traces, read each trace from \texttt{in}, clip it
and write the output to \texttt{out}.

\subsection{Compiling}

To compile the C++ program, run
\begin{verbatim}
c++ clip.cc -I$RSFROOT/include -L$RSFROOT/lib -lrsf++ -lrsf -lm
\end{verbatim}
Change \texttt{c++} to the C++ compiler appropriate for your system and
include additional compiler flags if necessary. The flags that RSF typically
uses are in \texttt{\$RSFROOT/lib/rsfconfig.py}.

\section{Fortran-77 interface}

\lstset{language=fortran}

The Fortran-77 clip function is listed below.

\lstinputlisting[frame=single]{\RSF/api/clip.f}
Let us examine it in detail.

\lstinputlisting[firstline=8,lastline=8,frame=single]{\RSF/api/clip.f}
The program starts with a call to \texttt{sf\_init}, which initializes the
command-line interface.

\lstinputlisting[firstline=9,lastline=10,frame=single]{\RSF/api/clip.f}
The input and output files are created with calls to
\texttt{sf\_input} and \texttt{sf\_output}. Because of the absence of
derived types in Fortran-77, we use simple integer pointers to
represent RSF files. Both \texttt{sf\_input} and \texttt{sf\_output}
accept a character string, which may refer to a file name or a file
tag. For example, if the command line contains
\texttt{vel=velocity.rsf}, then both
\texttt{sf\_input("velocity.rsf")} and \texttt{sf\_input("vel")} are
acceptable. Two tags are special: \texttt{"in"} refers to the file in
the standard input and \texttt{"out"} refers to the file in the
standard output.

\lstinputlisting[firstline=12,lastline=13,frame=single]{\RSF/api/clip.f}
RSF files can store data of different types (character, integer,
floating point, complex). The function \texttt{sf\_gettype} checks the
type of data stored in the RSF file. We make sure that the type
corresponds to floating-point numbers. If not, the program is aborted
with an error message, using the \texttt{sf\_error} function.  It is
generally a good idea to check the input for user errors and, if they
cannot be corrected, to take a safe exit.

\lstinputlisting[firstline=15,lastline=20,frame=single]{\RSF/api/clip.f}
Conceptually, the RSF data model is a multidimensional hypercube. By
convention, the dimensions of the cube are stored in \texttt{n1=},
\texttt{n2=}, etc. parameters. The \texttt{n1} parameter refers to the
fastest axis. If the input dataset is a collection of traces,
\texttt{n1} refers to the trace length. We extract it using the
\texttt{sf\_histint} function (integer parameter from history) and
abort if no value for \texttt{n1} is found. Since Fortran-77 cannot
easily handle dynamic allocation, we also need to check that
\texttt{n1} is not larger than the size of the statically allocated
array. We could proceed in a similar fashion, extracting \texttt{n2},
\texttt{n3}, etc. If we are interested in the total number of traces,
like in the clip example, a shortcut is to use the
\texttt{sf\_leftsize} function.  Calling \texttt{sf\_leftsize(in,0)}
returns the total number of elements in the hypercube (the product of
\texttt{n1}, \texttt{n2}, etc.), calling \texttt{sf\_leftsize(in,1)}
returns the number of traces (the product of \texttt{n2}, \texttt{n3},
etc.), calling \texttt{sf\_leftsize(in,2)} returns the product of
\texttt{n3}, \texttt{n4}, etc. By calling \texttt{sf\_leftsize}, we
avoid the need to extract additional parameters for the hypercube
dimensions that we are not interested in.

\lstinputlisting[firstline=22,lastline=23,frame=single]{\RSF/api/clip.f}
The clip parameter is read from the command line, where it can be
specified, for example, as \texttt{clip=10}. The parameter has the
\texttt{float} type, therefore we read it with the
\texttt{sf\_getfloat} function. If no \texttt{clip=} parameter is
found among the command line arguments, the program is aborted with an
error message using the \texttt{sf\_error} function.

\lstinputlisting[firstline=25,lastline=37,frame=single]{\RSF/api/clip.f}
Finally, we do the actual work: loop over input traces, reading,
clipping, and writing out each trace.

\subsection{Compiling}

To compile the Fortran-77 program, run
\begin{verbatim}
f77 clip.f -L$RSFROOT/lib -lrsff -lrsf -lm
\end{verbatim}
Change \texttt{f77} to the Fortran compiler appropriate for your system and
include additional compiler flags if necessary. The flags that RSF typically
uses are in \texttt{\$RSFROOT/lib/rsfconfig.py}.

\section{Fortran-90 interface}

\lstset{language=fortran}

The Fortran-90 clip function is listed below.

\lstinputlisting[frame=single]{\RSF/api/clip.f90}

Let us examine it in detail.

\lstinputlisting[firstline=2,lastline=2,frame=single]{\RSF/api/clip.f90}
The program starts with importing the \texttt{rsf} module.

\lstinputlisting[firstline=10,lastline=10,frame=single]{\RSF/api/clip.f90}
A call to \texttt{sf\_init} is needed to initialize the command-line
interface.

\lstinputlisting[firstline=11,lastline=12,frame=single]{\RSF/api/clip.f90}
The standard input and output files are initialized with
\texttt{rsf\_input} and \texttt{rsf\_output} functions. Both functions
accept optional arguments. For example, if the command line contains
\texttt{vel=velocity.rsf}, then both
\texttt{rsf\_input("velocity.rsf")} and \texttt{rsf\_input("vel")} are
acceptable.

\lstinputlisting[firstline=14,lastline=15,frame=single]{\RSF/api/clip.f90}
A call to \texttt{from\_par} extracts the ``\texttt{n1}'' parameter
from the input file. Conceptually, the RSF data model is a
multidimensional hypercube.  The \texttt{n1} parameter refers to the
fastest axis. If the input dataset is a collection of traces,
\texttt{n1} corresponds to the trace length. We could proceed in a
similar fashion, extracting \texttt{n2}, \texttt{n3}, etc. If we are
interested in the total number of traces, like in the clip example, a
shortcut is to use the \texttt{filesize} function.  Calling
\texttt{filesize(in)} returns the total number of elements in the
hypercube (the product of \texttt{n1}, \texttt{n2}, etc.), calling
\texttt{filesize(in,1)} returns the number of traces (the product of
\texttt{n2}, \texttt{n3}, etc.), calling \texttt{filesize(in,2)}
returns the product of \texttt{n3}, \texttt{n4}, etc. By calling
\texttt{filesize}, we avoid the need to extract additional parameters
for the hypercube dimensions that we are not interested in.

\lstinputlisting[firstline=17,lastline=17,frame=single]{\RSF/api/clip.f90}
The clip parameter is read from the command line, where it can be
specified, for example, as \texttt{clip=10}. If we knew a good default
value for \texttt{clip}, we could specify it with an optional
argument, i.e. \texttt{call~from\_par("clip",clip,default)}.

\lstinputlisting[firstline=21,lastline=28,frame=single]{\RSF/api/clip.f90}
Finally, we do the actual work: loop over input traces, reading,
clipping, and writing out each trace.

\subsection{Compiling}
To compile the Fortran-90 program, run
\begin{verbatim}
f90 clip.f90 -I$RSFROOT/include -L$RSFROOT/lib -lrsff90 -lrsf -lm
\end{verbatim}
Change \texttt{f90} to the Fortran-90 compiler appropriate for your system and
include additional compiler flags if necessary. The flags that RSF typically
uses are in \texttt{\$RSFROOT/lib/rsfconfig.py}.

\section{Python interface}

\lstset{language=python}

The Python clip script is listed below.

\lstinputlisting[frame=single]{\RSF/api/clippy.exe}
Let us examine it in detail. 

\lstinputlisting[firstline=3,lastline=4,frame=single]{\RSF/api/clippy.exe}
The script starts with importing the \texttt{numarray} and
\texttt{rsf} modules.

\lstinputlisting[firstline=6,lastline=9,frame=single]{\RSF/api/clippy.exe}
Next, we initialize the command line interface and the standard input and
output files. We also make sure that the input file type is floating point.

\lstinputlisting[firstline=11,lastline=13,frame=single]{\RSF/api/clippy.exe}
We extract the ``\texttt{n1}'' parameter from the input file.
Conceptually, the RSF data model is a multidimensional hypercube.  The
\texttt{n1} parameter refers to the fastest axis. If the input dataset
is a collection of traces, \texttt{n1} corresponds to the trace
length. We could proceed in a similar fashion, extracting \texttt{n2},
\texttt{n3}, etc. If we are interested in the total number of traces,
like in the clip example, a shortcut is to use the \texttt{size}
method of the \texttt{Input} class1.  Calling \texttt{size(0)} returns
the total number of elements in the hypercube (the product of
\texttt{n1}, \texttt{n2}, etc.), calling \texttt{size(1)} returns the
number of traces (the product of \texttt{n2}, \texttt{n3}, etc.),
calling \texttt{size(2)} returns the product of \texttt{n3},
\texttt{n4}, etc.

\lstinputlisting[firstline=15,lastline=16,frame=single]{\RSF/api/clippy.exe}
The clip parameter is read from the command line, where it can be specified,
for example, as \texttt{clip=10}.

\lstinputlisting[firstline=20,lastline=23,frame=single]{\RSF/api/clippy.exe}
Finally, we do the actual work: loop over input traces, reading,
clipping, and writing out each trace.

\subsection{Compiling}

The python script does not require compilation. Simply make sure that
\texttt{\$RSFROOT/lib} is in \texttt{PYTHONPATH}. 

\section{Matlab interface}

\lstset{language=matlab}

The Matlab clip function is listed below.

\lstinputlisting[frame=single]{\RSF/api/clip.m}

Let us examine it in detail. 

\lstinputlisting[firstline=4,lastline=4,frame=single]{\RSF/api/clip.m}
We start by figuring out the input file dimensions.

\lstinputlisting[firstline=5,lastline=6,frame=single]{\RSF/api/clip.m}
The first dimension is the trace length, the product of all other
dimensions correspond to the number of traces.

\lstinputlisting[firstline=7,lastline=8,frame=single]{\RSF/api/clip.m}
Next, we allocate the trace array and create an output file.

\lstinputlisting[firstline=10,lastline=15,frame=single]{\RSF/api/clip.m}
Finally, we do the actual work: loop over input traces, reading,
clipping, and writing out each trace.

\subsection{Compiling}

The Matlab script does not require compilation. Simply make sure that
\texttt{\$RSFROOT/lib} is in \texttt{MATLABPATH}.

\section{Installation} 

To install the interface to a particular language, use \texttt{API=}
parameter in the RSF configuration. For example, to to install C++ and
Fortran-90 API bindings in addition to the basic package, run
\begin{verbatim}
scons API=c++,fortran-90 config
\end{verbatim}
Only the C interface is configured by default. The configuration
parameters are stored in \texttt{\$RSFROOT/lib/rsfconfig.py}.

\bibliographystyle{seg} 
\bibliography{api}

%%% Local Variables: 
%%% mode: latex
%%% TeX-master: t
%%% End: 
