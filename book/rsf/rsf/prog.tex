\title{Guide to RSF programs}
\email{sergey.fomel@beg.utexas.edu}
\author{Sergey Fomel}
\lefthead{Fomel}
\righthead{RSF programs}

\maketitle

\begin{abstract}

This guide introduces some of the most used RSF programs and illustrates their
usage with examples.

\end{abstract}

\section{Main programs}

The source files for these programs can be found under
\href{http://egl.beg.utexas.edu/viewcvs/trunk/filt/main/}{filt/main}
in the RSF distribution.

\noindent\doublebox{\parbox{\textwidth}{
\input{sfadd}
}}

\texttt{sfadd} is useful for combining (adding, dividing, or
multiplying) several datasets. What if you want to subtract two
datasets? Easy. Use the \texttt{scale} parameter as follows:
\begin{verbatim}
bash$ sfadd data1.rsf data2.rsf scale=1,-1 > diff.rsf
\end{verbatim}
or
\begin{verbatim}
bash$ sfadd < data1.rsf data2.rsf scale=1,-1 > diff.rsf
\end{verbatim}
The same task can be accomplished with the more general \texttt{sfmath} program:
\begin{verbatim}
bash$ sfmath one=data1.rsf two=data2.rsf output='one-two' > diff.rsf
\end{verbatim}
or
\begin{verbatim}
bash$ sfmath < data1.rsf two=data2.rsf output='input-two' > diff.rsf
\end{verbatim}
In both cases, the size and shape of \texttt{data1.rsf} and
\texttt{data2.rsf} hypercubes should be the same, and a warning
message is printed out if the the axis sampling parameters (such as
\texttt{o1} or \texttt{d1}) in these files are different.

\noindent\doublebox{\parbox{\textwidth}{
\input{sfattr}
}}

\texttt{sfattr} is a useful diagnostic program. It reports certain statistical
values for an RSF dataset: RMS (root-mean-square) amplitude, mean value, norm
value, maximum and minimum values, number of nonzero samples, and the total
number of samples. If we denote data values as $d_i$ for $i=0,1,2,\ldots,n$,
then the RMS value is $\sqrt{\frac{1}{n}\,\sum\limits_{i=0}^n d_i^2}$, the
mean value is $\frac{1}{n}\,\sum\limits_{i=0}^n d_i$, and the $L_2$-norm value
is $\sum\limits_{i=0}^n d_i^2$. Using \texttt{sfattr} is a quick way to see
the distribution of data values and check it for errors.

\noindent\doublebox{\parbox{\textwidth}{
\input{sfcat}
}}

\texttt{sfcat} and \texttt{sfmerge} concatenate two or more files
together along a particular axis. It is the same program, only
\texttt{sfcat} has the default \texttt{space=n} and \texttt{sfmerge}
has the default \texttt{space=y}.

Example of \texttt{sfcat}:
\begin{verbatim}
bash$ sfspike n1=2 n2=3 > one.rsf
bash$ sfin one.rsf
one.rsf:
    in="/tmp/one.rsf@"
    esize=4 type=float form=native
    n1=2           d1=0.004       o1=0          label1="Time (s)"
    n2=3           d2=0.1         o2=0          label2="Distance (km)"
        6 elements 24 bytes
bash$ sfcat one.rsf one.rsf axis=1 > two.rsf
bash$ sfin two.rsf
two.rsf:
    in="/tmp/two.rsf@"
    esize=4 type=float form=native
    n1=4           d1=0.004       o1=0          label1="Time (s)"
    n2=3           d2=0.1         o2=0          label2="Distance (km)"
        12 elements 48 bytes
\end{verbatim}

Example of \texttt{sfmerge}:
\begin{verbatim}
bash$ sfmerge one.rsf one.rsf axis=2 > two.rsf
bash$ sfin two.rsf
two.rsf:
    in="/tmp/two.rsf@"
    esize=4 type=float form=native
    n1=2           d1=0.004       o1=0          label1="Time (s)"
    n2=7           d2=0.1         o2=0          label2="Distance (km)"
        14 elements 56 bytes
\end{verbatim}
In this case, an extra empty trace is inserted between the two merged files.

The axes that are not being merged are checked for consistency:
\begin{verbatim}
bash$ sfcat one.rsf two.rsf > three.rsf
sfcat: n2 mismatch: need 3
\end{verbatim}

\noindent\doublebox{\parbox{\textwidth}{
\input{sfcmplx}
}}

\texttt{sfcmplx} simply creates a complex dataset from its real and
imaginary parts. The reverse operation can be accomplished with
\texttt{sfreal} and \texttt{sfimag}.

Example of \texttt{sfcmplx}:
\begin{verbatim}
bash$ sfspike n1=2 n2=3 > one.rsf
bash$ sfin one.rsf
one.rsf:
    in="/tmp/one.rsf@"
    esize=4 type=float form=native
    n1=2           d1=0.004       o1=0          label1="Time (s)"
    n2=3           d2=0.1         o2=0          label2="Distance (km)"
        6 elements 24 bytes
bash$ sfcmplx one.rsf one.rsf > cmplx.rsf
bash$ sfin cmplx.rsf
cmplx.rsf:
    in="/tmp/cmplx.rsf@"
    esize=8 type=complex form=native
    n1=2           d1=0.004       o1=0          label1="Time (s)"
    n2=3           d2=0.1         o2=0          label2="Distance (km)"
        6 elements 48 bytes
\end{verbatim}

\noindent\doublebox{\parbox{\textwidth}{
\input{sfcp}
}}

The \texttt{sfcp} and \texttt{sfmv} command imitate the Unix
\texttt{cp} and \texttt{mv} commands and serve for copying and moving
RSF files. Example:
\begin{verbatim}
bash$ sfspike n1=2 n2=3 > one.rsf
bash$ sfin one.rsf
one.rsf:
    in="/tmp/one.rsf@"
    esize=4 type=float form=native
    n1=2           d1=0.004       o1=0          label1="Time (s)"
    n2=3           d2=0.1         o2=0          label2="Distance (km)"
        6 elements 24 bytes
bash$ sfcp one.rsf two.rsf
bash$ sfin two.rsf
two.rsf:
    in="/tmp/two.rsf@"
    esize=4 type=float form=native
    n1=2           d1=0.004       o1=0          label1="Time (s)"
    n2=3           d2=0.1         o2=0          label2="Distance (km)"
        6 elements 24 bytes
\end{verbatim}

\noindent\doublebox{\parbox{\textwidth}{
\input{sfcut}
}}

The \texttt{sfcut} command is related to \texttt{sfwindow} and has the same
set of arguments only instead of extracting the selected window, it fills it
with zeroes. The size of the input data is preserved. 

Examples:
\begin{verbatim}
bash$ sfspike n1=5 n2=5 > in.rsf
bash$ < in.rsf sfdisfil
   0:             1            1            1            1            1
   5:             1            1            1            1            1
  10:             1            1            1            1            1
  15:             1            1            1            1            1
  20:             1            1            1            1            1
bash$ < in.rsf sfcut n1=2 f1=1 n2=3 f2=2 | sfdisfil
   0:             1            1            1            1            1
   5:             1            1            1            1            1
  10:             1            0            0            1            1
  15:             1            0            0            1            1
  20:             1            0            0            1            1
bash$ < in.rsf sfcut j1=2 | sfdisfil
   0:             0            1            0            1            0
   5:             0            1            0            1            0
  10:             0            1            0            1            0
  15:             0            1            0            1            0
  20:             0            1            0            1            0
\end{verbatim}

\noindent\doublebox{\parbox{\textwidth}{
\input{sfdd}
}}

The \texttt{sfdd} program is used to change either the form (\texttt{ascii},
\texttt{xdr}, \texttt{native}) or the type (\texttt{complex}, \texttt{float},
\texttt{int}, \texttt{char}) of the input dataset. 

In the example below, we create a plain text (ASCII) file with numbers and
then use \texttt{sfdd} to generate an RSF file in \texttt{xdr} form with
\texttt{complex} numbers. 
\begin{verbatim}
bash$ cat test.txt
1 2 3 4 5 6
bash$ echo n1=6 data_format=ascii_int in=test.txt > test.rsf
bash$ sfin test.rsf
test.rsf:
    in="test.txt"
    esize=0 type=int form=ascii
    n1=6           d1=?           o1=?
        6 elements
bash$ sfdd < test.rsf form=xdr type=complex > test2.rsf
bash$ sfin test2.rsf
test2.rsf:
    in="/tmp/test2.rsf@"
    esize=8 type=complex form=xdr
    n1=3           d1=?           o1=?
        3 elements 24 bytes
bash$ sfdisfil < test2.rsf
   0:          1,         2i         3,         4i         5,         6i
\end{verbatim}

To learn more about the data format in RSF, consult the
\href{http://egl.beg.utexas.edu/RSF/book/rsf/rsf/format_html/}{guide to RSF
  format}.

\noindent\doublebox{\parbox{\textwidth}{
\input{sfdisfil}
}}

The \texttt{sfdisfil} program simply dumps the data contents to the standard
output in a text form. It is used mostly for debugging purposes to quickly
examine RSF files. Here is an example:
\begin{verbatim}
bash$ sfmath o1=0 d1=2 n1=12 output=x1 > test.rsf
bash$ < test.rsf sfdisfil
   0:             0            2            4            6            8
   5:            10           12           14           16           18
  10:            20           22
\end{verbatim}
The output format is easily configurable.
\begin{verbatim}
bash$ < test.rsf sfdisfil col=6 number=n format="%5.1f"
  0.0  2.0  4.0  6.0  8.0 10.0
 12.0 14.0 16.0 18.0 20.0 22.0
\end{verbatim}
Along with \texttt{sfdd}, \texttt{sfdisfil} provides a simple way to convert
RSF data to an ASCII form.

\noindent\doublebox{\parbox{\textwidth}{
\input{sfget}
}}

The \texttt{sfget} program extracts a parameter value from an RSF file. It is
useful mostly for scripting. Here is, for example, a quick calculation of the
maximum value on the first axis in an RSF dataset (the output of
\texttt{sfspike}) using the standard Unix \texttt{bc} calculator.
\begin{verbatim}
bash$ ( sfspike n1=100 | sfget n1 d1 o1; echo "o1+(n1-1)*d1" ) | bc
.396
\end{verbatim}
See also \texttt{sfput}.

\noindent\doublebox{\parbox{\textwidth}{
\input{sfheaderattr}
}}

The \texttt{sfheaderattr} examines the contents of a trace header file,
typically generated by \texttt{sfsegyread}. In the example below, we examine
trace headers in the output of \texttt{suplane}, a program from Seismic Unix.
\begin{verbatim}
bash$ suplane > plane.su
bash$ sfsegyread tape=plane.su su=y tfile=tfile.rsf > plane.rsf
bash$ sfheaderattr < tfile.rsf
*******************************************
71 headers, 32 traces
key[0]="tracl"      min[0]=1            max[31]=32          mean=16.5
key[1]="tracr"      min[0]=1            max[31]=32          mean=16.5
key[2]="fldr"       min[0]=0            max[31]=0           mean=0
key[3]="tracf"      min[0]=0            max[31]=0           mean=0
key[4]="ep"         min[0]=0            max[31]=0           mean=0
key[5]="cdp"        min[0]=0            max[31]=0           mean=0
key[6]="cdpt"       min[0]=0            max[31]=0           mean=0
key[7]="trid"       min[0]=0            max[31]=0           mean=0
key[8]="nvs"        min[0]=0            max[31]=0           mean=0
key[9]="nhs"        min[0]=0            max[31]=0           mean=0
key[10]="duse"      min[0]=0            max[31]=0           mean=0
key[11]="offset"    min[0]=400          max[31]=400         mean=400
key[12]="gelev"     min[0]=0            max[31]=0           mean=0
key[13]="selev"     min[0]=0            max[31]=0           mean=0
key[14]="sdepth"    min[0]=0            max[31]=0           mean=0
key[15]="gdel"      min[0]=0            max[31]=0           mean=0
key[16]="sdel"      min[0]=0            max[31]=0           mean=0
key[17]="swdep"     min[0]=0            max[31]=0           mean=0
key[18]="gwdep"     min[0]=0            max[31]=0           mean=0
key[19]="scalel"    min[0]=0            max[31]=0           mean=0
key[20]="scalco"    min[0]=0            max[31]=0           mean=0
key[21]="sx"        min[0]=0            max[31]=0           mean=0
key[22]="sy"        min[0]=0            max[31]=0           mean=0
key[23]="gx"        min[0]=0            max[31]=0           mean=0
key[24]="gy"        min[0]=0            max[31]=0           mean=0
key[25]="counit"    min[0]=0            max[31]=0           mean=0
key[26]="wevel"     min[0]=0            max[31]=0           mean=0
key[27]="swevel"    min[0]=0            max[31]=0           mean=0
key[28]="sut"       min[0]=0            max[31]=0           mean=0
key[29]="gut"       min[0]=0            max[31]=0           mean=0
key[30]="sstat"     min[0]=0            max[31]=0           mean=0
key[31]="gstat"     min[0]=0            max[31]=0           mean=0
key[32]="tstat"     min[0]=0            max[31]=0           mean=0
key[33]="laga"      min[0]=0            max[31]=0           mean=0
key[34]="lagb"      min[0]=0            max[31]=0           mean=0
key[35]="delrt"     min[0]=0            max[31]=0           mean=0
key[36]="muts"      min[0]=0            max[31]=0           mean=0
key[37]="mute"      min[0]=0            max[31]=0           mean=0
key[38]="ns"        min[0]=64           max[31]=64          mean=64
key[39]="dt"        min[0]=4000         max[31]=4000        mean=4000
key[40]="gain"      min[0]=0            max[31]=0           mean=0
key[41]="igc"       min[0]=0            max[31]=0           mean=0
key[42]="igi"       min[0]=0            max[31]=0           mean=0
key[43]="corr"      min[0]=0            max[31]=0           mean=0
key[44]="sfs"       min[0]=0            max[31]=0           mean=0
key[45]="sfe"       min[0]=0            max[31]=0           mean=0
key[46]="slen"      min[0]=0            max[31]=0           mean=0
key[47]="styp"      min[0]=0            max[31]=0           mean=0
key[48]="stas"      min[0]=0            max[31]=0           mean=0
key[49]="stae"      min[0]=0            max[31]=0           mean=0
key[50]="tatyp"     min[0]=0            max[31]=0           mean=0
key[51]="afilf"     min[0]=0            max[31]=0           mean=0
key[52]="afils"     min[0]=0            max[31]=0           mean=0
key[53]="nofilf"    min[0]=0            max[31]=0           mean=0
key[54]="nofils"    min[0]=0            max[31]=0           mean=0
key[55]="lcf"       min[0]=0            max[31]=0           mean=0
key[56]="hcf"       min[0]=0            max[31]=0           mean=0
key[57]="lcs"       min[0]=0            max[31]=0           mean=0
key[58]="hcs"       min[0]=0            max[31]=0           mean=0
key[59]="year"      min[0]=0            max[31]=0           mean=0
key[60]="day"       min[0]=0            max[31]=0           mean=0
key[61]="hour"      min[0]=0            max[31]=0           mean=0
key[62]="minute"    min[0]=0            max[31]=0           mean=0
key[63]="sec"       min[0]=0            max[31]=0           mean=0
key[64]="timbas"    min[0]=0            max[31]=0           mean=0
key[65]="trwf"      min[0]=0            max[31]=0           mean=0
key[66]="grnors"    min[0]=0            max[31]=0           mean=0
key[67]="grnofr"    min[0]=0            max[31]=0           mean=0
key[68]="grnlof"    min[0]=0            max[31]=0           mean=0
key[69]="gaps"      min[0]=0            max[31]=0           mean=0
key[70]="otrav"     min[0]=0            max[31]=0           mean=0
*******************************************
\end{verbatim}
For each standard keyword, a minimum, maximum, and mean values are reported.
This quick inspection can help in identifying meaningful keywords set in the
data. The input data type must be \texttt{int}.

\noindent\doublebox{\parbox{\textwidth}{
\input{sfheadermath}
}}

\texttt{sfheadermath} is a versatile program for mathematical
operations on rows of the input file. If the input file is an
\texttt{n1} by \texttt{n2} matrix, the output will be a \texttt{1} by
\texttt{n2} matrix that contains one row made out of mathematical
operations on the other rows. \texttt{sfheadermath} can identify a row
by number or by a standard SEGY keyword. The latter is useful for
processing headers extracted from SEGY or SU files.

Here is an example. First, we create an SU file with \texttt{suplane} and convert it to RSF using \texttt{sfsegyread}.
\begin{verbatim}
bash$ suplane > plane.su
bash$ sfsegyread tape=plane.su su=y tfile=tfile.rsf > plane.rsf
\end{verbatim}
The trace header information is saved in \texttt{tfile.rsf}. It
contains 71 headers for 32 traces in integer format.
\begin{verbatim}
bash$ sfin tfile.rsf
tfile.rsf:
    in="/tmp/tfile.rsf@"
    esize=4 type=int form=native
    n1=71          d1=?           o1=?
    n2=32          d2=?           o2=?
        2272 elements 9088 bytes
\end{verbatim}
Next, we will convert \texttt{tfile.rsf} to a floating-point format
and run \texttt{sfheadermath} to create a new header.
\begin{verbatim}
bash$ < tfile.rsf sfdd type=float | \
sfheadermath myheader=1 output="sqrt(myheader+(2+10*offset^2))" > new.rsf
bash$ sfin new.rsf
new.rsf:
    in="/tmp/new.rsf@"
    esize=4 type=float form=native
    n1=1           d1=?           o1=?
    n2=32          d2=?           o2=?
        32 elements 128 bytes
\end{verbatim}
We defined ``myheader'' as being the row number 1 in the input (note
that numbering starts with 0) and combined it with ``offset'', which
is a standard SEGY keyword that denotes row number 11 (see the output
of \texttt{sfheaderattr} above.) A variety of mathematical expressions
can be defined in the \texttt{output=} string. The expression
processing engine is shared with \texttt{sfmath}.

\noindent\doublebox{\parbox{\textwidth}{
\input{sfheadersort}
}}

\texttt{sfheadersort} is used to sort traces in the input file
according to trace header information. 

Here is an example of using
\texttt{sfheadersort} for randomly shuffling traces in the input
file. First, let us create an input file with seven traces:
\begin{verbatim}
bash$ sfmath n1=5 n2=7 output=x2+1 > input.rsf
bash$ < input.rsf sfdisfil
   0:             1            1            1            1            1
   5:             2            2            2            2            2
  10:             3            3            3            3            3
  15:             4            4            4            4            4
  20:             5            5            5            5            5
  25:             6            6            6            6            6
  30:             7            7            7            7            7 
\end{verbatim}
Next, we can create a random file with seven header values using
\texttt{sfnoise}.
\begin{verbatim}
bash$ sfspike n1=7 | sfnoise rep=y type=n > random.rsf
bash$ < random.rsf sfdisfil
   0:       0.05256      -0.2879       0.1487       0.4097       0.1548
   5:        0.4501       0.2836
\end{verbatim}
If you reproduce this example, your numbers will most likely be different,
because, in the absence of \texttt{seed=} parameter, \texttt{sfnoise}
uses a random seed value to generate pseudo-random numbers. Finally, we
apply \texttt{sfheadersort} to shuffle the input traces.
\begin{verbatim}
bash$ < input.rsf sfheadersort head=random.rsf > output.rsf
bash$ < output.rsf sfdisfil
   0:             2            2            2            2            2
   5:             1            1            1            1            1
  10:             3            3            3            3            3
  15:             5            5            5            5            5
  20:             7            7            7            7            7
  25:             4            4            4            4            4
  30:             6            6            6            6            6
\end{verbatim}
As expected, the order of traces in the output file corresponds to the
order of values in the header. Thanks to the separation between
headers and data, the operation of \texttt{sfheadersort} is optimally
efficient. It first sorts the headers and only then accesses the data,
reading each data trace only once.

\noindent\doublebox{\parbox{\textwidth}{
\input{sfheaderwindow}
}}

\texttt{sfheaderwindow} is used to window traces in the input file
according to trace header information. 

Here is an example of using \texttt{sfheaderwindow} for randomly
selecting part of the traces in the input file. First, let us create
an input file with ten traces:
\begin{verbatim}
bash$ sfmath n1=5 n2=10 output=x2+1 > input.rsf
bash$ < input.rsf sfdisfil
   0:             1            1            1            1            1
   5:             2            2            2            2            2
  10:             3            3            3            3            3
  15:             4            4            4            4            4
  20:             5            5            5            5            5
  25:             6            6            6            6            6
  30:             7            7            7            7            7
  35:             8            8            8            8            8
  40:             9            9            9            9            9
  45:            10           10           10           10           10
\end{verbatim}
Next, we can create a random file with ten header values using
\texttt{sfnoise}.
\begin{verbatim}
bash$ sfspike n1=10 | sfnoise rep=y type=n > random.rsf
bash$ < random.rsf sfdisfil
   0:     -0.005768      0.02258     -0.04331      -0.4129      -0.3909
   5:      -0.03582       0.4595      -0.3326        0.498      -0.3517
\end{verbatim}
If you reproduce this example, your numbers will most likely be different,
because, in the absence of \texttt{seed=} parameter, \texttt{sfnoise}
uses a random seed value to generate pseudo-random numbers. Finally,
we apply \texttt{sfheaderwindow} to window the input traces selecting
only those for which the header is greater than zero.
\begin{verbatim}
bash$ < random.rsf sfmask min=0 > mask.rsf
bash$ < mask.rsf sfdisfil
   0:    0    1    0    0    0    0    1    0    1    0
bash$ < input.rsf sfheaderwindow mask=mask.rsf > output.rsf
bash$ < output.rsf sfdisfil
   0:             2            2            2            2            2
   5:             7            7            7            7            7
  10:             9            9            9            9            9
\end{verbatim}
In this case, only three traces are selected for the output. Thanks to
the separation between headers and data, the operation of
\texttt{sfheaderwindow} is optimally efficient. 


%%% Local Variables: 
%%% mode: latex
%%% TeX-master: t
%%% End: 
