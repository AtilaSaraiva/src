\title{Madagascar Installation}
\email{sergey.fomel@beg.utexas.edu}
\author{Sergey Fomel}
\lefthead{Fomel}
\righthead{Installation}

%\begin{document}

\maketitle

\begin{abstract}
Madagascar has been installed and is periodically tested on
\begin{itemize}
\item Different Linux distributions (Fedora, RedHat, SuSE, Debian, Ubuntu)
\item FreeBSD and OpenBSD
\item Solaris
\item MacOS X
\item Windows under the \href{http://www.cygwin.com/}{Cygwin} environment and 
under Microsoft's \href{http://www.microsoft.com/technet/interopmigration/unix/sfu/default.mspx}{Services for UNIX} environment.
\end{itemize}
\end{abstract}

\section{Precompiled binary packages}

A precompiled binary package of the latest Madagascar stable release
exists for Mac OS X. See the Madagascar
\href{http://rsf.sourceforge.net/Download}{download page} for details.

\section{Installation from source}

\subsection{Prerequisites}
\pdfbookmark[1]{Prerequisites}{prq}

\begin{enumerate}
\item C compiler. ANSI-compliant compiler such as
  \href{http://gcc.gnu.org/}{GCC} should work. GCC usually comes pre-installed
  on Linux machines.
\item Python interpreter. \href{http://www.python.org/}{Python} is an
  interpretable programming language. It is used by in Madagascar
  installation scripts and project management scripts.  Python comes
  pre-installed on some platforms.
\end{enumerate}

\subsection{Environmental variables}

\begin{enumerate}
\item  Set the \texttt{RSFROOT} environmental variable to the directory where you want RSF installed.
\item Set the \texttt{PYTHONPATH} environmental variable to include \texttt{\$RSFROOT/lib}. 
\item Add \texttt{\$RSFROOT/bin} to your \texttt{PATH} environmental variable.
\item Set \texttt{DATAPATH} to the directory for temporary data files. 
\end{enumerate}

Example configuration for \texttt{csh} and \texttt{tcsh}:
\begin{verbatim}
setenv RSFROOT /usr/local/rsf
if ($?PYTHONPATH) then
   setenv PYTHONPATH ${PYTHONPATH}:$RSFROOT/lib
else
   setenv PYTHONPATH $RSFROOT/lib
endif
set path = ($path $RSFROOT/bin)
setenv DATAPATH /var/tmp/
\end{verbatim}

Example configuration for \texttt{bash}:
\begin{verbatim}
export RSFROOT=/usr/local/rsf
if [ -n "$PYTHONPATH" ]; then
   export PYTHONPATH=${PYTHONPATH}:$RSFROOT/lib
else
   export PYTHONPATH=$RSFROOT/lib
fi
export PATH=$PATH:$RSFROOT/bin
export DATAPATH=/var/tmp/
\end{verbatim}

Notice the slash at the end of the \texttt{DATAPATH} variable.

\subsection{Software construction}

\begin{enumerate}

\item Configuration.

Change to the top  source directory and run
\begin{verbatim}
./configure
\end{verbatim}

You can examine the \texttt{config.py} file that this command
generates.  Additional options are available. You can obtain a full
list of customizable variables by running \texttt{scons -h}. For
example, to install C++ and Fortran-90 API bindings in addition to the
basic package, run
\begin{verbatim}
./configure API=c++,fortran-90 
\end{verbatim}

\item Building and installing the package.

Run \texttt{make install} or the following two commands in succession: 
\begin{verbatim}
make
make install
\end{verbatim}
or
\begin{verbatim}
scons
scons install
\end{verbatim}

If you need ``root'' privileges for installing under \texttt{\$RSFROOT}, you may need to run
\begin{verbatim}
scons
su
scons install
\end{verbatim}
or
\begin{verbatim}
scons
sudo scons install
\end{verbatim}

\item Cleaning.

To clean all intermediate files generated by SCons, run
\begin{verbatim}
make clean
\end{verbatim}
or
\begin{verbatim}
scons -c
\end{verbatim}

To clean all intermediate files and all installed files, run
\begin{verbatim}
make distclean
\end{verbatim}
or
\begin{verbatim}
scons -c install
\end{verbatim}

\end{enumerate}

\subsection{Troubleshooting}

Note that SCons does not inherit your environmental variables including
\texttt{PATH}. If the configuration part ends with the message like
\begin{verbatim}
checking if cc works ... failed
\end{verbatim}
the problem may be that your compiler is in unusual place. Try
\begin{verbatim}
./configure CC=/full/path/to/cc
\end{verbatim}
or
\begin{verbatim}
./configure CC=`which cc` 
\end{verbatim}

On Windows under SFU, use the \texttt{gcc} compiler
\begin{verbatim}
./configure CC=/opt/gcc.3.3/bin/gcc
\end{verbatim}

For more information, please consult the \href{http://rsf.sourceforge.net/Advanced_Installation}{Advanced Installation guide}.

\section{Testing and quick start}

Here are a few simple tests and and a brief introduction to Madagascar:

Typing any Madagascar command in a terminal window without parameters
should generate a brief documentation on that command. Try one of the
following:
\begin{verbatim}
sfin
sfattr
sfspike
sfbandpass
sfwiggle
\end{verbatim}

If you get an error like ``Command not found'', you may not have your
\texttt{PATH} environment variable set correctly, or you may need to issue the
\texttt{rehash} command.

Now try making a simple Madagascar data file:
\begin{verbatim}
sfspike n1=1000 k1=300 > spike.rsf
\end{verbatim}
This command generates a one dimensional list of 1000 numbers, all zero
except for a spike equal to one at position 300. If this generates an
error like
\begin{verbatim}
Cannot write to data file /path/spike.rsf@: Bad file descriptor
\end{verbatim}
you may need to create the directory pointed to by your \texttt{DATAPATH}
environment variable.

The file \texttt{spike.rsf} is a text header. The actual data are stored in the
binary file pointed to by the \texttt{in=} parameter in the header. You can look
at the header file directly with \texttt{more}, or better, examine the file
properties with:
\begin{verbatim}
sfin spike.rsf
\end{verbatim}
You can learn more about the contents of \texttt{spike.rsf} with
\begin{verbatim}
sfattr < spike.rsf
\end{verbatim}
The following command applies a bandpass filter to \texttt{spike.rsf} and puts the result
in \texttt{filter.rsf}:
\begin{verbatim}
sfbandpass fhi=2 phase=1 < spike.rsf > filter.rsf
\end{verbatim}
The following command makes a graphics file from \texttt{filter.rsf}:
\begin{verbatim}
sfwiggle title="Welcome to Madagascar" < filter.rsf > filter.vpl
\end{verbatim}
If you have an X-Window display program running, and your \texttt{DISPLAY}
environment variable is set correctly, you can display the graphics file
with
\begin{verbatim}
xtpen < filter.vpl
\end{verbatim}
You can pipe Madagascar commands together and do the whole thing at once
like this:
\begin{verbatim}
sfspike n1=1000 k1=300 | sfbandpass fhi=2 phase=1 | \
sfwiggle title="Welcome to Madagascar" | xtpen
\end{verbatim}
If you have SCons installed, you can use it to automate Madagascar
processing. Here is a simple \texttt{SConstruct} file to make \texttt{filter.rsf} and
\texttt{filter.vpl}:
\definecolor{frame}{rgb}{0.905,0.905,0.905}
\lstset{language=Python,backgroundcolor=\color{frame},showstringspaces=false,numbers=left,numberstyle=\tiny}
\lstinputlisting[frame=single]{test/SConstruct}

Put the file in an empty directory, give it the name \texttt{SConstruct}, cd to
that directory, and issue the command:
\begin{verbatim}
scons
\end{verbatim}
The graphics file is now stored in the Fig subdirectory. You can view it
manually with:
\begin{verbatim}
xtpen Fig/filter.vpl
\end{verbatim}
or you can use
\begin{verbatim}
scons view
\end{verbatim}
When an \texttt{SConstruct} file makes more than one graphics file, the scons
view command will display all of them in sequence.

\inputdir{test}
\plot{filter}{width=\textwidth}{Welcome to Madagascar.}

The result should look like Figure~\ref{fig:filter}.

Now edit the \texttt{SConstruct} file: change the title string on the Result line
to "Hello World!", save the file, and rerun the scons command. You will
see that scons has figured out that the file \texttt{filter.rsf} does not need to
be rebuilt because nothing that affects it has changed. Only the file
\texttt{filter.vpl} is rebuilt.

