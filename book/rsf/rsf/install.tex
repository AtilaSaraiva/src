\title{RSF Installation}
\email{sergey.fomel@beg.utexas.edu}
\author{Sergey Fomel}
\lefthead{Fomel}
\righthead{RSF installation}

%\begin{document}

\maketitle

\begin{abstract}
RSF is currently developed on a Linux platform. It has been installed and is
periodically tested on
\begin{itemize}
\item Different Linux distributions (RedHat 7.3, RedHat 9.0, SuSe 9.1)
\item SGI Irix
\item Solaris
\item MacOs X
\item Windows under the \href{http://www.cygwin.com/}{Cygwin environment} 
\end{itemize}
However, installation on non-standard platform is prone to problems, which
will get fixed as RSF gets further developed and tested.
\end{abstract}

\section{Prerequisites}
\pdfbookmark[1]{Prerequisites}{prq}

\begin{enumerate}
\item C compiler. ANSI-compliant compiler such as
  \href{http://gcc.gnu.org/}{GCC} should work. GCC usually comes pre-installed
  on Linux machines.
\item Python interpreter. \href{http://www.python.org/}{Python} is an
  interpretable programming language. It is used by RSF in installation
  scripts and project management scripts.  Python comes pre-installed on
  RedHat Linux.
\item SCons (Software Construction). SCons is a Python-based tool for project
  management and software construction. It is a modern replacement for the
  Unix ``make'' utility. RSF uses SCons for both software construction and
  processing flow management. The current version of SCons is available at
  \url{http://www.scons.org/}. It is also provided under the \texttt{scons}
  directory in the RSF source distribution.
\end{enumerate}

To display and manipulate figures generated by RSF, you may also need the
vplot-manipulation programs from SEPlib. Refer to the SEPlib documentation at
\\ \url{http://sepwww.stanford.edu/software/seplib/} for installation
instructions. You only need programs specified under \texttt{vplot/filters} in
the SEPlib source tree. RSF will eventually provide analogous functionality.

\section{Downloading the development version}

The best way to obtain the current working version of RSF is with a Subversion
client. \href{http://subversion.tigris.org/}{Subversion} is a version control
system, which is designed as a replacement for CVS.

To obtain the latest RSF source with Subversion, run 
\begin{verbatim}
svn co http://egl.beg.utexas.edu/svn/rsf/trunk RSF
\end{verbatim}
The source will be installed under directory \texttt{RSF}. You can also browse
individual files in the repository at
\url{http://egl.beg.utexas.edu/viewcvs/}

Subversion is a simple and convenient system to use. For example, to update
the source distribution at any time, you can simply run
\begin{verbatim}
svn update
\end{verbatim}
inside the directory that needs updating.

If you cannot get Subversion to work at your location, download the tarball
file by following the link at \url{http://egl.beg.utexas.edu/viewcvs/}.
In this case, you will need to run
\begin{verbatim}
gunzip < root.tar.gz | tar xvf -
cd rsf/trunk
chmod a+x python/sfdoc
\end{verbatim}
to prepare software for installation.

\section{Installation}

\subsection{Environmental variables}

Set the \texttt{RSFROOT} environmental variable to the directory where
you want RSF installed. To use RSF effectively, you may also want to add
\texttt{\$RSFROOT/bin} to your \texttt{PATH} environmental variable, set
\texttt{PYTHONPATH} to \texttt{\$RSFROOT/lib}, add set \texttt{DATAPATH} to
the directory for storing temporary data files. 

Example configuration for \texttt{csh} and \texttt{tcsh}:
\begin{verbatim}
setenv RSFROOT /usr/local/rsf
set path = ($path $RSFROOT/bin)
setenv PYTHONPATH $RSFROOT/lib
setenv DATAPATH /var/tmp/
\end{verbatim}

Example configuration for \texttt{sh} and \texttt{bash}:
\begin{verbatim}
export RSFROOT=/usr/local/rsf
export PATH=$PATH:$RSFROOT/bin
export PYTHONPATH=$RSFROOT/lib
export DATAPATH=/var/tmp/
\end{verbatim}

Notice the slash at the end of the DATAPATH variable.

\subsection{Software construction}

\begin{enumerate}

\item Install SCons.

If you don't have the latest version of SCons already installed, you can do so
by going to \texttt{scons} directory. Install SCons from the enclosed
\texttt{rpm} file or by unpacking the source and running 
\begin{verbatim}
python setup.py install
\end{verbatim}

\item Configuration.

Change to the top RSF source directory and run
\begin{verbatim}
scons config
\end{verbatim}

You can examine the \texttt{config.py} file that this command
generates.  Additional options are available. You can obtain a full
list of customizable variables by running \texttt{scons -h}. For
example, to install C++ and Fortran-90 API bindings in addition to the
basic package, run
\begin{verbatim}
scons API=c++,fortran-90 config
\end{verbatim}

To change configuration at any time, you may need to run 
\begin{verbatim}
scons -c config
\end{verbatim}
first to remove previous settings.

\item Building and installing the package.

Run \texttt{scons install} or the following two commands in succession: 
\begin{verbatim}
scons
scons install
\end{verbatim}

If you need ``root'' privileges for installing under \texttt{\$RSFROOT}, you may need to run
\begin{verbatim}
scons
su
scons install
\end{verbatim}

\item Cleaning.

To clean all intermediate files generated by SCons, run
\begin{verbatim}
scons -c
\end{verbatim}

To clean all intermediate files and all installed files, run
\begin{verbatim}
scons -c install
\end{verbatim}

\end{enumerate}

\subsection{Bugs}

Please report all problems encountered during software construction to \\
\texttt{<sergey.fomel@beg.utexas.edu>}.

\section{Other installations}

There are two other packages that might be useful in conjuction with RSF:

\subsection{RSF reproducible figures}

\begin{itemize}
\item Using Subversion, run
\begin{verbatim}
svn co http://egl.beg.utexas.edu/svn/rsffigs $RSFROOT/figs
\end{verbatim}
\item Without Subversion, download the tarball by following the link at
\url{http://egl.beg.utexas.edu/viewcvs/?root=rsffigs}
\end{itemize}
This installs a wide collection of more than 2,000 reproducible
figures. It may take a long time to download and some space on disk.
The figures are preserved with the purpose to do regression testing
whenever the software or the environment change.

\subsection{\LaTeX\ package}

\begin{itemize}
\item Using Subversion, run
\begin{verbatim}
svn co http://egl.beg.utexas.edu/svn/texmf/trunk texmf
\end{verbatim}
\item Without Subversion, download the tarball by following the link at
  \url{http://egl.beg.utexas.edu/viewcvs/?root=texmf}
\end{itemize}
This installs \LaTeX2e\ and \texttt{latex2html} macros for writing
geophysical papers, books, and reports. The \texttt{texmf} directory should
be placed where \LaTeX\ can find it. Some systems recognize
\texttt{\$HOME/texmf} as one of the default places.

%\end{document}

%%% Local Variables: 
%%% mode: latex
%%% TeX-master: t
%%% End: 
