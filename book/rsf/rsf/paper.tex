\title{\textsc{Madagascar} software package and reproducible research}

\renewcommand{\thefootnote}{\fnsymbol{footnote}} 

\author{Sergey Fomel\footnotemark[1], Paul Sava\footnotemark[2], Ioan Vlad\footnotemark[3], Yang Liu\footnotemark[1], James Jennings\footnotemark[4], 
Jules Browaeys\footnotemark[5], Vladimir Bashkardin\footnotemark[1], Jeff Godwin\footnotemark[2], Xiaolei Song\footnotemark[1], and Gilles Hennenfent\footnotemark[6]}

\ms{GEO-2010}

\address{
\footnotemark[1]Bureau of Economic Geology, \\
John A. and Katherine G. Jackson School of Geosciences \\ The
University of Texas at Austin \\ University Station, Box X \\ Austin,
TX 78713-8972 \\
\footnotemark[2] Department of Geophysics \\
Colorado School of Mines \\
Golden, CO 80401 \\
\footnotemark[3] Statoil \\
\footnotemark[4] Shell \\
\footnotemark[5] Total \\
\footnotemark[6] Chevron}

\lefthead{Fomel et al.}
\righthead{\textsc{Madagascar}}

\maketitle

\begin{abstract}
  \textsc{Madagascar} is a software package...
\end{abstract}

\section{Introduction}

The goal of this paper is to introduce the \textsc{Madagascar}
software package. \textsc{Madagascar} provides a collection of
programs for multidimensional data analysis, including seismic imaging
and seismic data processing, and an integrated environment for
arranging data processing workflows and computational
experiments. \textsc{Madagascar} follows the philosophy of
\emph{reproducible research}, introduced by \cite{SEG-1992-0601}.
Reproducible research implies that the results of computational
experiments are published along with the software code and, if
possible, the input data, so that the reader of an article is able to
reproduce and verify the experiment \cite[]{matt,intro}.

Why is reproducible research important? A scientific communication is
valuable only if the result is communicated to a skeptic, someone who
does not take the author's conclusions for granted unless they are
confirmed by sufficient evidence. The ability to confirm and verify
the published result allows the recepient of a scientific communition
to build on top of previous result and to extend scientific and
technological progress. This is the case... computations...

It is the need for an integrated and open environment for geophysical
research that motivated the development of the
\textsc{Madagascar} software package. reproducible research... ..similarities and
differences with MATLAB... with SU...

To make the presentation more entertaining, we organize the paper as a
story. In this story, a young geophysicist discovers Madagascar, uses
it to perform data processing tasks and to develop new algorithms, and
eventually contributes to the Madagascar development. Through this
story, we introduce different features of the package that the reader
may find useful.

\section{Jordan discovers \textsc{Madagascar}}

This is a story of Jordan, a young geophysicist who works on
geophysical data analysis. One day, Jordan is thinking about...

The design of \textsc{Madagascar} is based on several fundamental
principles. 

One of the principles is \emph{modularity} and
\emph{encapsulation}. Similarly to Seismic Unix \cite[]{TLE16-07-10451049},
the core of \textsc{Madagascar} consists of a collection of separate
modules, connected in data analysis workflows...

\emph{reproducible research} and \emph{test-driven development}.

\emph{file format} 

\section{Jordan writes a \textsc{Madagascar} program}

\section{Jordan conducts numerical experiments}

\section{Jordan writes a reproducible paper}

\section{Jordan contributes to \textsc{Madagascar}}

\section{How to find your way around \textsc{Madagascar}}

An easy way to start exploring \textsc{Madagascar} is checking the
list of reproducible papers at
\url{http://www.ahay.org/wiki/Reproducible_Documents}.  If any of
these papers looks close to your interests, follow the links until you
find a figure with a "wrench" button under it. Click on the wrench,
and it will open a computational recipe used for generating the figure
(the \texttt{SConstruct} file). Here is an simple
example\footnote{\url{http://www.reproducibility.org/RSF/book/rsf/rsf/sfmath.html}}:

\subsection{Madagascar components}

The \textsc{Madagascar} installation consists of several components:
\begin{description}
\item[Main programs.] Hundreds of main programs...
\item[Processing scripts.] Processing scripts (\texttt{SConstruct} files)
\item[Reproducible papers.] Scientific publications...
\item[Reproducible figures.] Figures from reproducible papers are saved...
\item[APIs.] The application programmer interfaces provide...  
\end{description}

\subsection{Installation}

\textsc{Madagascar} has been installed and tested on a variety of platforms...

\section{How to contribute your own programs to \textsc{Madagascar}}

The \textsc{Madagascar} community is open. Developers take a joint
responsibility of maintaining and extending the package. It is easy to
become a developer:
\begin{enumerate}
\item  Follow the Guide on Adding new programs to Madagascar.
\item  If your employer holds the copyright for your software, 
obtain a permission to distribute under a GPL license and place a
copyright and GPL license notice in each file with code.
\item Register at SourceForge.
\item Find out who the current project administrators are and pass your SourceForge
user name to one of them.
\item Upload your directory to the Subversion repository at SourceForge.
\item Make sure to add reproducible examples of using your program. 
As a rule, programs without examples are not included in the release
version.
\end{enumerate}	

\section{Conclusions}

We have introduced \textsc{Madagascar}, a software package for
multidimensional data analysis...

\section{Acknowledgments}

In addition to the authors of this paper, many people contributed to
the development of \textsc{Madagascar}. We would like to thank...


\append{Basic facts about \textsc{Madagascar}}

The work on the \textsc{Madagascar} package (previously named RSF for
\emph{Regularly Sampled Format}) started in 2003.  The package got
publicly released during an EAGE Workshop on open-source E\&P Software
in Vienna in June 2006. Since then, more than 25 people have
contributed to its development, writing cummulatively almost
300,000 lines of code.

\href{http://www.ohloh.net/}{Ohloh}, a website that maps the landscape
of open-source software development, evaluates \textsc{Madagascar} as
\begin{itemize}
\item Mature, well-established codebase;
\item Increasing year-over-year development activity;
\item Large, active development team.
\end{itemize}

Several events helped promote \textsc{Madagascar} in the geophysical
and signal processing communities (Table~\ref{tbl:events}.)

\tabl{events}{\textsc{Madagascar} annual events}{%
\begin{center}
\begin{tabular}{|r|l|l|} \hline
year & location & name \\ \hline \hline
2006 & Vancouver, Canada & School and Workshop on Reproducible Research \\
& &  in Computational Geophysics \\
2007 & Austin, TX & Short Course on Using and Extending \textsc{Madagascar} \\
2008 & Golden, CO & Implementation Workshop \\
2009 & Delft, the Netherlands & School on Reproducible Computational Geophysics \\
2010 & Houston, TX & School on Reproducible Computational Geophysics \\
& & and Hands-On Workshop \\ \hline
\end{tabular}
\end{center}
}


\bibliographystyle{seg}
\bibliography{SEG,intro}




