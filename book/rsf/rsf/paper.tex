\title{Multidimensional data analysis \\ using the \textsc{Madagascar} software package}

\renewcommand{\thefootnote}{\fnsymbol{footnote}} 

\author{Sergey Fomel\footnotemark[1], Paul Sava\footnotemark[2], Jim Jennings, Jules Browaeys, Yang Liu\footnotemark[1], Ioan Vlad, Vladimir Bashkardin\footnotemark[1],  Gilles Hennenfent, Trevor Irons\footnotemark[2], and William Burnett\footnotemark[1]}

\ms{GEO-2010}

\address{
\footnotemark[1]Bureau of Economic Geology, \\
John A. and Katherine G. Jackson School of Geosciences \\
The University of Texas at Austin \\
University Station, Box X \\
Austin, TX 78713-8972 \\
\footnotemark[2] Department of Geophysics \\
Colorado School of Mines \\
Golden, CO 80401}

\lefthead{Fomel et al.}
\righthead{\textsc{Madagascar}}

\maketitle

\begin{abstract}
  \textsc{Madagascar} is a software package...
\end{abstract}

\section{Introduction}

The goal of this paper is to introduce the \textsc{Madagascar}
software package. \textsc{Madagascar} provides a collection of
programs for multidimensional data analysis, including seismic imaging
and seismic data processing, and an integrated environment for
conducting data processing workflows and computational experiments.

It is the need for an integrated and open environment for geophysical
research that motivated the development of
\textsc{Madagascar}... reproducible research... ..similarities and
differences with MATLAB... with SU...

The paper is organized as follows. We start with basic facts about the
\textsc{Madagascar} development status and organizational
structure. Next, we describe the core principles that went into
software design. Finally, we provide guidelines for new users for
exploring the package and for contributing new software to it. The
appendix contains a complete example of a small research project
implemented in \textsc{Madagascar}

\section{Facts}

The work on the \textsc{Madagascar} package (previously named RSF for \emph{Regularly
Sampled Format}) was started in 2003.  The package got
publicly released during an EAGE Workshop on open-source E\&P Software
in Vienna in June 2006. Since then, more than 25 people have
contributed to its development, writing more than 250,000 lines of code.

\href{http://www.ohloh.net/}{Ohloh}, a website that maps the landscape
of open-source software development, evaluates \textsc{Madagascar} as
\begin{itemize}
\item Mature, well-established codebase
\item Increasing year-over-year development activity
\item Large, active development team 
\end{itemize}

Several events helped promote \textsc{Madagascar} in the geophysical
and signal processing communities:
\begin{description}
\item[2006 School and Workshop on Reproducible Research in
  Computational Geophysics] took place in Vancouver, Canada, and
  attracted...
\item[2007 Short Course on Using and Extending Madagascar] took place
  in Austin, Texas, and attracted...
\item[2008 Implementation Workshop] took place in Golden, Colorado, attracted...
\item[2009 School on Reproducible Computational Geophysics] took place
  in Delft, the Netherlands, and attracted...
\item[2009 Madagascar Tutorial] took place in Salvador, Brazil, and attracted...
\item[2010 MadagascarFest] is planning to take place in Houston, Texas.
\end{description}

\section{Design principles}

The design of \textsc{Madagascar} is based on several fundamental principles.

\section{How to find your way around \textsc{Madagascar}}

An easy way to start exploring \textsc{Madagascar} is checking the list of reproducible papers. 

\subsection{Installation}

\section{How to contribute your own programs to \textsc{Madagascar}}

The \textsc{Madagascar} community is open. Developers take a joint
responsibility of maintaining and extending the package. It is easy to
become a developer....

\bibliographystyle{seg}
\bibliography{SEG}

\append{Example}

In this appendix, we provide a simple example of using
\textsc{Madagascar} for a research project.


