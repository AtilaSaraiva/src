\title{Introduction to Madagascar}
\author{Sergey Fomel}
\lefthead{Fomel}
\righthead{Introduction to Madagascar}

\maketitle

\begin{abstract}

Madagascar is an open-source software package for geophysical data
processing and reproducible numerical experiments. Its mission is
to provide
\begin{enumerate}
\item a convenient and powerful environment
\item a convenient technology transfer tool
\end{enumerate}
for researchers working with digital image and data processing. The
package is available in an open-source form to allow effective
collaboration of a wide community of developers. The technology
developed using the Madagascar project management system is
transferred in the form of recorded processing histories, which become
``computational recipes'' to be verified, exchanged, and modified by
the users of the system.

\end{abstract}

\section{What is Madagascar?}
\pdfbookmark[1]{What is Madagascar?}{what}

Madagascar is an open-source software package for geophysical data processing
and reproducible numerical experiments. Its most noticeable features are:

\begin{enumerate}

\item Madagascar is a relatively \emph{new} package. It started in
2003, was developed entirely from scratch, and publicly released in
2006. Being a new package, it follows modern software engineering
practices such as module encapsulation and test-driven development. A
rapid development of a project of this scope (more than 300 main
programs and more than 3,000 tests) would not be possible without
standing on the shoulders of giants and learning from the 30 years of
previous experience in open packages such as SEPlib and Seismic
Unix. We have borrowed and reimplemented functionality and ideas from
these other packages.

\item Madagascar is a \emph{test-driven} package. Test-driven development is
not only an agile software programming practice but also a way of
bringing scientific foundation to geophysical research that involves
numerical experiments. Bringing reproducibility and peer review, the
backbone of any real science, to the field of computational geophysics
is the main motivation for Madagascar development. The package consists of
two levels: low-level main programs (typically developed in the C
programming language and working as data filters) and high-level
processing flows (described with the help of the Python programming
language) that combine main programs and completely document data
processing histories for testing and reproducibility. Experience
shows that high-level programming is easily mastered even by beginning
students that have no previous programming experience.

\item Madagascar is an \emph{open-source} package. It is distributed
under the standard GPL open-source license, which places no
restriction on the usage and modification of the code. Moreover,
access to modifying the source repository is not controlled by one
organization but shared equally among different developers. This
enables an open collaboration among different groups spread all over
the world, in the true spirit of the open source movement.

\item Madagascar uses a \emph{simple, flexible, and universal} data
format that can handle very large datasets but is not tied
specifically to seismic data or data of any other particular
kind. This ``regularly sampled'' format is borrowed from the
traditional SEPlib. A universal data format allows us to share
general-purpose data processing tools with scientists from other
disciplines such as petroleum engineers working on large-scale
reservoir simulations.

\end{enumerate}

\section{What is in Madagascar?}
\pdfbookmark[1]{What is in Madagascar?}{in}

The package consists of
\begin{enumerate}
\item A collection of main programs. Most programs act as filters on input
  data and can be chained in a Unix pipeline. For example:
\begin{verbatim}
< data.rsf sfwindow n1=100 | sfbandpass fhi=60 > data2.rsf
\end{verbatim}
This approach follows the Unix philosophy, as formulated by Doug McIlroy,
the inventor of Unix pipes \cite[]{salus}:
\begin{enumerate}
\item Write programs that do one thing and do it well. 
\item Write programs to work together. 
\item Write programs to handle text streams, because that is a universal
  interface.
\end{enumerate}

Running a command (such as \texttt{sfwindow}) without parameters or the
necessary input and output files shows a brief documentation, explaining the
program purpose and parameters.  Alternatively, brief documentation is
provided by \texttt{sfdoc} program. Main program documentation in HTML format
is available on the web at \url{http://www.reproducibility.org/RSF/}.

Madagascar uses \emph{Regularly Sampled Format} (RSF) for data files,
which is similar to the format used in the SEPlib library developed at
the Stanford Exploration Project (SEP). The file format describes
regularly sampled hypercubes. Up to 9 dimensions are supported. In
accordance with the Unix philosophy, each RSF file (such as
\texttt{data.rsf}) is a simple readable text. It contains a pointer
(\texttt{in=} parameter) to the location of the binary data.

Madagascar provides programs for conversion to and from other formats
such as SEG-Y and SU. Madagascar currently adopts Vplot file format,
also developed at SEP, for generated graphics files.

\item An API (application programmer's interface) for programmers writing
  their own software to manipulate RSF files. The main software language of
  the Madagascar package is C. Interfaces to other languages (C++, Fortran-77,
  Fortran-90, Python, Matlab) are also provided.
  
\item A project management system. The system uses and extends
  \href{http://www.scons.org/}{SCons}, an open-source software
  construction package, to document and maintain data processing
  flows. Documented projects become computational recipes that can be
  easily exchanged among Madagascar users.
  
\item A collection of reproducible documents, organized in living
  books. Each reproducible book contains a collection of Madagascar
  recipes (\texttt{SConstruct} files) used to generate book
  figures. The recipes cover a variety of data processing and imaging
  tasks described in the books. Figures and recipes serve dual purpose
  with respect to Madagascar maintenance. They provide demos for
  introducing new users to the functionality of the package and, at
  the same time, regression tests for assuring the system stability
  under change.

\end{enumerate}

Follow the links at the end of this paper for additional documentation.

\section{Copyright notice}
\pdfbookmark[1]{Copyright notice}{cr}

The Madagascar package is released in an open-source form under the standard GNU GPL
license. In simple words, there are no restrictions on the use of the software
(including copying, modifying, selling, etc.) However, there are restrictions
on the software redistribution intended to prevent the package from losing
its open-source status. Users are encourages to submit their modifications
back to the original distribution to the benefit of the whole user community.

\section{Alternatives} 
\pdfbookmark[1]{Alternatives}{alt}

In the present form, the Madagascar package, while being completely
written from scratch, borrows ideas from the design of
\href{http://sepwww.stanford.edu/software/seplib/}{SEPlib}, a publicly
available software package, maintained by Bob Clapp at the Stanford
Exploration Project. Generations of SEP students and researchers
contributed to SEPlib. Most important contributions came from Rob
Clayton, Jon Claerbout, Dave Hale, Stew Levin, Rick Ottolini, Joe
Dellinger, Steve Cole, Dave Nichols, Martin Karrenbach, Biondo Biondi,
and Bob Clapp.

Madagascar also borrows ideas from
\href{http://timna.mines.edu/cwpcodes/}{Seismic Unix} (SU), a package
maintained by John Stockwell at the Center for Wave Phenomenon at the
Colorado School of Mines \cite[]{TLE16-07-10451049,su}. Main
contributors to SU included Einar Kjartansson, Shuki Ronen, Jack
Cohen, Chris Liner, Dave Hale, and John Stockwell.  SU is open-source
software (distributed with BSD-style license) starting with
\href{ftp://ftp.cwp.mines.edu/pub/cwpcodes/news40}{release 40} (April
10, 2007).

Another option for a seismic processing system is
\href{http://www.freeusp.org/}{Free USP}. USP is a processing package
originally developed by Amoco and released by BP. Another package,
\href{http://www.freeusp.org/DDS/}{DDS} (Data Dictionary System) was
also released by BP.

\begin{comment}
None of these alternative packages can be qualified as free software
according to the
\href{http://www.fsf.org/philosophy/free-sw.html}{Free Software Foundation} or as open-source software according to the
\href{http://www.opensource.org/docs/definition.php}{Open Source Initiative}.  However, they are available for free in the source form
under certain conditions. The Madagascar package is both free and
open-source.
\end{comment}

\section{Other documents}
\pdfbookmark[1]{Other documents}{other}

\begin{itemize}
\item  \href{http://www.reproducibility.org/RSF/book/rsf/rsf/install_html/}{Installation instructions}
\item \href{http://www.reproducibility.org/RSF/}{Self-documentation reference for Madagascar programs}
\item A \href{http://www.reproducibility.org/RSF/book/rsf/rsf/prog_html/}{guide to Madagascar programs}
\item A \href{http://www.reproducibility.org/RSF/book/rsf/rsf/format_html/}{guide to RSF file format}
\item A \href{http://www.reproducibility.org/RSF/book/rsf/rsf/api_html/}{guide to  Madagascar programming interface}
\item A \href{http://www.reproducibility.org/RSF/book/rsf/rsf/demo_html/}{guide to programming with Madagascar}
\item A \href{http://www.reproducibility.org/RSF/book/rsf/rsf/tour_html/}{tour of Madagascar software}
\item A \href{http://www.reproducibility.org/RSF/book/rsf/scons/paper_html/}{guide to SCons interface for reproducible computations}
\end{itemize}

\bibliographystyle{seg}
\bibliography{intro,SEG,SEP2}

%\end{document}


%%% Local Variables: 
%%% mode: latex
%%% TeX-master: t
%%% TeX-master: t
%%% End: 
