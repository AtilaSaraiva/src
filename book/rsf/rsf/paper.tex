\title{Introduction to RSF}
\author{Sergey Fomel}
\lefthead{Fomel}
\righthead{Introduction to RSF}

\maketitle

\begin{abstract}

RSF (from \emph{Regularly Sampled Format}) is an open-source software
package for geophysical data processing and reproducible numerical
experiments. The RSF mission is to provide
\begin{enumerate}
\item a convenient and powerful environment
\item a convenient technology transfer tool
\end{enumerate}
for researchers working with digital image and data processing. The package is
available in an open-source form to allow effective collaboration of a wide
community of developers. The technology developed using the RSF project
management system is transferred in the form of recorded processing histories,
which become ``computational recipes'' to be verified, exchanged, and modified
by users of the system.

\end{abstract}

\section{What is RSF?}

RSF is an open-source software package for geophysical data processing
and reproducible numerical experiments. The four main features of RSF
are:

\begin{enumerate}

\item RSF is a \emph{new} package. It started in 2003 and was
developed entirely from scratch. Being a new package, it follows
modern software engineering practices such as module encapsulation and
test-driven development. A rapid development of a project of this
scope (more than 300 main programs and more than 3000 tests) would not
be possible without standing on the shoulders of giants and learning
from the 30 years of previous experience in open packages such as
SEPlib and Seismic Unix. We have borrowed and reimplemented
functionality and ideas from these packages.

\item RSF is a \emph{test-driven} package. Test-driven development is
not only an agile software programming practice but also a way of
bringing scientific foundation to geophysical research that involves
numerical experiments. Bringing reproducibility and peer review, the
backbone of any real science, to the field of computational geophysics
is the main motivation for RSF development. The package consists of
two levels: low-level main programs (typically developed in the C
programming language and working as data filters) and high-level
processing flows (described with the help of the Python programming
language) that combine main programs and completely document data
processing histories for testing and reproducibility. Experience
shows that high-level programming is easily mastered even by beginning
students that have no previous programming experience.

\item RSF is an \emph{open-source} package. It is distributed under
the standard GPL open-source license, which places no restriction on
the usage and modification of the code. Access to modifying the source
repository is not controlled by one organization but shared equally
among different developers. This enables an open collaboration among
different groups spread all over the world, in the true spirit of the
open source movement.

\item RSF uses a \emph{simple, flexible, and universal} data format
that can handle very large datasets but is not tied specifically to
seismic data or data of any other particular kind. This ``regularly
sampled'' format is borrowed from the traditional SEPlib and is also
related to the DDS format developed by Amoco and BP. A universal data
format allows us to share general-purpose data processing tools with
scientists from other disciplines such as petroleum engineers working
on large-scale reservoir simulations.

\end{enumerate}

\section{What is in RSF?}

RSF consists of
\begin{enumerate}
\item A collection of main programs. Most programs act as filters on input
  data and can be chained in a Unix pipeline. For example:
\begin{verbatim}
< data.rsf sfwindow n1=100 | sfbandpass fhi=60 > data2.rsf
\end{verbatim}
This approach follows the Unix philosophy, as formulated by Doug McIlroy,
the inventor of Unix pipes \cite[]{salus}:
\begin{enumerate}
\item Write programs that do one thing and do it well. 
\item Write programs to work together. 
\item Write programs to handle text streams, because that is a universal
  interface.
\end{enumerate}

Running a command (such as \texttt{sfwindow}) without parameters or the
necessary input and output files shows a brief documentation, explaining the
program purpose and parameters.  Alternatively, brief documentation is
provided by \texttt{sfdoc} program. Main program documentation in HTML format
is available on the web at \\ \url{http://egl.beg.utexas.edu/RSF/}.

RSF uses \emph{Regularly Sampled Format} for data files, which is similar
to the format used in the SEPlib library developed at the Stanford Exploration
Project (SEP). The file format describes regularly sampled hypercubes. Up to 9
dimensions are supported. In accordance with the Unix philosophy, each RSF
file (such as \texttt{data.rsf}) is a simple readable text. It contains a
pointer (\texttt{in=} parameter) to the location of the binary data. 

RSF provides programs for conversion to and from other popular formats such as
SEG-Y and SU. RSF currently adopts Vplot file format, also developed at SEP,
for generated graphics files.

\item An API (application programmer's interface) for programmers writing
  their own software to manipulate RSF files. The main software language of
  the RSF package is C. Interfaces to other languages (C++, Fortran-77,
  Fortran-90, Python, Matlab) are also provided.
  
\item A project management system. The system uses and extends \texttt{SCons},
  an open-source software construction
  package\footnote{\url{http://www.scons.org/}}, to document and maintain data
  processing flows. Documented projects become computational recipes that can
  be easily exchanged among RSF users.
  
\item A collection of reproducible documents, organized in living books. Each
  reproducible book contains a collection of RSF recipes (\texttt{SConstruct}
  files) used to generate book figures. The recipes cover a variety of data
  processing and imaging tasks described in the books. Figures and recipes
  serve dual purpose with respect to RSF maintenance. They provide demos for
  introducing new users to the functionality of the package and regression
  tests for assuring the system stability under change.

\end{enumerate}

Follow the links at the end of this paper for additional documentation.

\section{Copyright notice}

The RSF package is released in an open-source form under the standard GNU GPL
license. In simple words, there are no restrictions on the use of the software
(including copying, modifying, selling, etc.) However, there are restrictions
on the software redistribution intended to prevent the package from loosing
its open-source status. Users are encourages to submit their modifications
back to the original distribution to the benefit of the whole user community.

\section{Alternatives} 

In the present form, the RSF package, while being completely written
from scratch, borrows ideas from the design of SEPlib, a publicly
available software package, maintained by Bob Clapp at the Stanford
Exploration Project
\cite[]{Claerbout.sep.70.413,Dellinger.sep.73.461,Nichols.sep.82.257,Biondi.sep.92.343,Clapp.sep.102.bob1}\footnote{SEPlib
is available at \url{http://sepwww.stanford.edu/software/seplib/}.}.
Generations of SEP students and researchers contributed to
SEPlib. Most important contributions came from Rob Clayton, Jon
Claerbout, Dave Hale, Stew Levin, Rick Ottolini, Joe Dellinger, Steve
Cole, Dave Nichols, Martin Karrenbach, Biondo Biondi, and Bob Clapp.

RSF also borrows ideas from Seismic Unix (SU), a package maintained by John
Stockwell at the Center for Wave Phenomenon at the Colorado School of Mines
\cite[]{TLE16-07-10451049,su}\footnote{SU is available at
  \url{http://timna.mines.edu/cwpcodes/}.}. Main contributors to SU
included Einar Kjartansson, Shuki Ronen, Jack Cohen, Chris Liner, Dave Hale,
and John Stockwell.

Another option for a seismic processing system is Free USP. USP is a
processing package originally developed by Amoco and released by BP
\footnote{Free USP is available at
\url{http://www.freeusp.org/}.}. Another package, DDS (Data Dictionary
System) is also released by BP\footnote{DDS is available at
\url{http://www.freeusp.org/DDS/}}. The DDS design is closer to RSF.

None of these alternative packages can be qualified as free software
according to the
\href{http://www.fsf.org/philosophy/free-sw.html}{Free Software
Foundation} or as open-source software according to the
\href{http://www.opensource.org/docs/definition.php}{Open Source
Initiative}.  However, they are available for free in the source form
under certain conditions. The RSF package is both free and
open-source.

\section{Other documents}

\begin{itemize}
\item
  \href{http://egl.beg.utexas.edu/RSF/book/rsf/rsf/install_html/}{Installation instructions}
\item \href{http://egl.beg.utexas.edu/RSF/}{Self-documentation reference for RSF programs}
\item A \href{http://egl.beg.utexas.edu/RSF/book/rsf/rsf/prog_html/}{guide to RSF programs}
\item A \href{http://egl.beg.utexas.edu/RSF/book/rsf/rsf/format_html/}
  {guide to RSF file format}
\item A \href{http://egl.beg.utexas.edu/RSF/book/rsf/rsf/api_html/}{guide to
    RSF programming interface}
\item A \href{http://egl.beg.utexas.edu/RSF/book/rsf/rsf/demo_html/}{guide to programming with RSF}
\item A \href{http://egl.beg.utexas.edu/RSF/book/rsf/rsf/tour_html/}{tour of RSF software}
\item A
  \href{http://egl.beg.utexas.edu/RSF/book/rsf/scons/paper_html/}{guide
    to SCons interface for reproducible computations}
\end{itemize}

\bibliographystyle{seg}
\bibliography{intro,SEG,SEP2}

%\end{document}


%%% Local Variables: 
%%% mode: latex
%%% TeX-master: t
%%% TeX-master: t
%%% End: 
