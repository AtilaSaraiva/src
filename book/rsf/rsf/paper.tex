\title{Multidimensional data analysis \\ using the \textsc{Madagascar} software package}

\renewcommand{\thefootnote}{\fnsymbol{footnote}} 

\author{Sergey Fomel\footnotemark[1], Paul Sava\footnotemark[2], Jim Jennings, Jules Browaeys, Yang Liu\footnotemark[1], Ioan Vlad, Vladimir Bashkardin\footnotemark[1],  Gilles Hennenfent, Trevor Irons, and William Burnett\footnotemark[1]}

\ms{GEO-2010}

\address{
\footnotemark[1]Bureau of Economic Geology, \\
John A. and Katherine G. Jackson School of Geosciences \\
The University of Texas at Austin \\
University Station, Box X \\
Austin, TX 78713-8972 \\
\footnotemark[2] Center for Wave Phenomena \\
Department of Geophysics \\
Colorado School of Mines \\
Golden, CO 80401}

\lefthead{Fomel et al.}
\righthead{\textsc{Madagascar}}

\maketitle

\begin{abstract}
  \textsc{Madagascar} is a software package...
\end{abstract}

\section{Introduction}

\section{Facts}

The work on the \textsc{Madagascar} package (previously named RSF for \emph{Regularly
Sampled Format}) was started in 2003.  The package got
publicly released during an EAGE Workshop on open-source E\&P Software
in Vienna in June 2006. Since then, more than 20 people have
contributed to its developement, writing more than 250,000 lines of code.

\href{http://www.ohloh.net/}{Ohloh}, a website that maps the landscape
of open-source software development, evaluates \textsc{Madagascar} as
\begin{itemize}
\item Mature, well-established codebase
\item Increasing year-over-year development activity
\item Large, active development team 
\end{itemize}

Several events helped promote \textsc{Madagascar} in the geophysical
and signal processing communities...

\section{Design principles}

The design of \textsc{Madagascar} is based on several fundamental principles.

\section{How to find your way around \textsc{Madagascar}}

An easy way to start exploring \textsc{Madagascar} is checking the list of reproducible papers. 

\section{Example}

\bibliographystyle{seg}
\bibliography{SEG}
