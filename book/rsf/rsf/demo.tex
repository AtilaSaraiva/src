\title{Guide to programming using RSF}
\email{paul.sava@beg.utexas.edu}
\author{Paul Sava}
\lefthead{Sava}
\righthead{RSF DEMO}

\maketitle

\begin{abstract}

  This guide demonstrates a simple time-domain
  finite-differences modeling code in RSF.

\end{abstract}

\section{Introduction}
This section presents time-domain 
finite-difference modeling \footnote{``Hello world'' of seismic imaging.}
written with the RSF library.
The program is demonstrated with the C, C++ and Fortran 90 interfaces.

The acoustic wave-equation
\begin{equation}
\Delta U - \frac{1}{v^2} \frac{\partial^2 U}{\partial t^2} = f(t)
\end{equation}
can be written as
\begin{equation}
\left[ \Delta U - f(t) \right] v^2 =
\frac{\partial^2 U}{\partial t^2} \;.
\end{equation}
$\Delta$ is the Laplacian symbol,
$f(t)$ is the source wavelet,
$v$ is the velocity, and
$U$ is a scalar wavefield.

A discrete time-step involves the following computations:
\begin{equation}
U_{i+1} = \left[ \Delta U -f(t) \right] v^2 \Delta t^2 + 2 U_{i} - U_{i-1} \;,
\end{equation}
where $U_{i-1}$, $U_{i}$ and $U_{i+1}$
represent the propagating wavefield at various time steps.
% ------------------------------------------------------------

\newpage
\section{C program}
\lstset{language=c,numbers=left,numberstyle=\tiny,showstringspaces=false}
\newcommand{\rsfafdm}{\RSF/api/c/test/afdm.c}

\tiny
\lstinputlisting[frame=single]{\rsfafdm}
\normalsize

\begin{itemize}
\item Declare input, output and auxiliary file tags:
\texttt{Fw} for input wavelet, 
\texttt{Fv} for velocity,
\texttt{Fr} for reflectivity, and
\texttt{Fo} for output wavefield.
\tiny
\lstinputlisting[firstline=9,lastline=9,frame=single]{\rsfafdm}
\normalsize

\item Declare RSF cube axes:
\texttt{at} time axis,
\texttt{ax} space axis,
\texttt{az} depth axis.
\tiny
\lstinputlisting[firstline=10,lastline=10,frame=single]{\rsfafdm}
\normalsize

\item Declare multi-dimensional arrays for input, output and computations.
\tiny
\lstinputlisting[firstline=14,lastline=15,frame=single]{\rsfafdm}
\normalsize

\item Open files for input/output.
\tiny
\lstinputlisting[firstline=22,lastline=25,frame=single]{\rsfafdm}
\normalsize

\item Read axes from input files; write axes to output file.
\tiny
\lstinputlisting[firstline=28,lastline=34,frame=single]{\rsfafdm}
\normalsize

\item Allocate arrays and read wavelet, velocity and reflectivity.
\tiny
\lstinputlisting[firstline=41,lastline=43,frame=single]{\rsfafdm}
\normalsize

\item Allocate temporary arrays.
\tiny
\lstinputlisting[firstline=46,lastline=49,frame=single]{\rsfafdm}
\normalsize

\item Loop over time.
\tiny
\lstinputlisting[firstline=62,lastline=62,frame=single]{\rsfafdm}
\normalsize

\item Compute Laplacian: $\Delta U$.
\tiny
\lstinputlisting[firstline=66,lastline=75,frame=single]{\rsfafdm}
\normalsize

\item Inject source wavelet: $\left[ \Delta U - f(t) \right]$
\tiny
\lstinputlisting[firstline=78,lastline=82,frame=single]{\rsfafdm}
\normalsize

\item Scale by velocity: $\left[ \Delta U - f(t) \right] v^2$
\tiny
\lstinputlisting[firstline=85,lastline=89,frame=single]{\rsfafdm}
\normalsize

\item Time step: 
$U_{i+1} = \left[ \Delta U -f(t) \right] v^2 \Delta t^2 + 2 U_{i} - U_{i-1}$
\tiny
\lstinputlisting[firstline=92,lastline=102,frame=single]{\rsfafdm}
\normalsize

\end{itemize}

% ------------------------------------------------------------
\newpage
\section{C++ program}
\lstset{language=c++,numbers=left,numberstyle=\tiny,showstringspaces=false}
\renewcommand{\rsfafdm}{\RSF/api/c++/test/afdm.cc}

\tiny
\lstinputlisting[frame=single]{\rsfafdm}
\normalsize
\newpage

\begin{enumerate}
\item Declare input, output and auxiliary file cubes 
(of type \texttt{CUB}).
\tiny
\lstinputlisting[firstline=19,lastline=22,frame=single]{\rsfafdm}
\normalsize

\item Declare, read and write RSF cube axes:
\texttt{at} time axis,
\texttt{ax} space axis,
\texttt{az} depth axis.
\tiny
\lstinputlisting[firstline=25,lastline=32,frame=single]{\rsfafdm}
\normalsize

\item Declare multi-dimensional 
\texttt{valarrays} for input, output and read data.
\tiny
\lstinputlisting[firstline=39,lastline=41,frame=single]{\rsfafdm}
\normalsize

\item Declare multi-dimensional 
\texttt{valarrays} for temporary storage.
\tiny
\lstinputlisting[firstline=44,lastline=47,frame=single]{\rsfafdm}
\normalsize

\item Initialize multidimensional \texttt{valarray} 
index counter (of type \texttt{VAI}).
\tiny
\lstinputlisting[firstline=50,lastline=50,frame=single]{\rsfafdm}
\normalsize

\item Loop over time.
\tiny
\lstinputlisting[firstline=54,lastline=54,frame=single]{\rsfafdm}
\normalsize

\item Compute Laplacian: $\Delta U$.
\tiny
\lstinputlisting[firstline=58,lastline=67,frame=single]{\rsfafdm}
\normalsize

\item Inject source wavelet: $\left[ \Delta U - f(t) \right]$
\tiny
\lstinputlisting[firstline=70,lastline=70,frame=single]{\rsfafdm}
\normalsize

\item Scale by velocity: $\left[ \Delta U - f(t) \right] v^2$
\tiny
\lstinputlisting[firstline=73,lastline=73,frame=single]{\rsfafdm}
\normalsize

\item Time step: 
$U_{i+1} = \left[ \Delta U -f(t) \right] v^2 \Delta t^2 + 2 U_{i} - U_{i-1}$
\tiny
\lstinputlisting[firstline=76,lastline=78,frame=single]{\rsfafdm}
\normalsize

\end{enumerate}
% ------------------------------------------------------------
\newpage
\section{Fortran 90 program}
\lstset{language=fortran,numbers=left,numberstyle=\tiny,showstringspaces=false}
\renewcommand{\rsfafdm}{\RSF/api/f90/test/afdm.f90}

\tiny
\lstinputlisting[frame=single]{\rsfafdm}
\normalsize
\newpage

\begin{itemize}
\item Declare input, output and auxiliary file tags.
\tiny
\lstinputlisting[firstline=11,lastline=11,frame=single]{\rsfafdm}
\normalsize

\item Declare RSF cube axes:
\texttt{at} time axis,
\texttt{ax} space axis,
\texttt{az} depth axis.
\tiny
\lstinputlisting[firstline=12,lastline=12,frame=single]{\rsfafdm}
\normalsize

\item Declare multi-dimensional arrays for input, output and computations.
\tiny
\lstinputlisting[firstline=16,lastline=17,frame=single]{\rsfafdm}
\normalsize

\item Open files for input/output.
\tiny
\lstinputlisting[firstline=23,lastline=26,frame=single]{\rsfafdm}
\normalsize

\item Read axes from input files; write axes to output file.
\tiny
\lstinputlisting[firstline=29,lastline=30,frame=single]{\rsfafdm}
\normalsize

\item Allocate arrays and read wavelet, velocity and reflectivity.
\tiny
\lstinputlisting[firstline=37,lastline=39,frame=single]{\rsfafdm}
\normalsize

\item Allocate temporary arrays.
\tiny
\lstinputlisting[firstline=42,lastline=45,frame=single]{\rsfafdm}
\normalsize

\item Loop over time.
\tiny
\lstinputlisting[firstline=48,lastline=48,frame=single]{\rsfafdm}
\normalsize

\item Compute Laplacian: $\Delta U$.
\tiny
\lstinputlisting[firstline=52,lastline=61,frame=single]{\rsfafdm}
\normalsize

\item Inject source wavelet: $\left[ \Delta U - f(t) \right]$
\tiny
\lstinputlisting[firstline=64,lastline=64,frame=single]{\rsfafdm}
\normalsize

\item Scale by velocity: $\left[ \Delta U - f(t) \right] v^2$
\tiny
\lstinputlisting[firstline=67,lastline=67,frame=single]{\rsfafdm}
\normalsize

\item Time step: 
$U_{i+1} = \left[ \Delta U -f(t) \right] v^2 \Delta t^2 + 2 U_{i} - U_{i-1}$
\tiny
\lstinputlisting[firstline=70,lastline=72,frame=single]{\rsfafdm}
\normalsize

\end{itemize}

% ------------------------------------------------------------
%
%\bibliographystyle{segnat}
%\bibliography{api}

%%% Local Variables: 
%%% mode: latex
%%% TeX-master: t
%%% End: 
