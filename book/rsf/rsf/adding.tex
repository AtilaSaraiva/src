\title{Adding new programs to madagascar}
\author{Sergey Fomel and Ioan Vlad}
\lefthead{Fomel \& Vlad}
\righthead{Adding programs}
\shortpaper
\maketitle

Whether you want to share your work with the world or keep it for
yourself, madagascar is fully extensible. The following instructions
explain the simplest way to add your own programs to the package:

\begin{enumerate}
\item Create a directory for yourself or your group under
\texttt{MADAGASCAR/user}, where \texttt{MADAGASCAR} is the top source
directory. Put your programs there. 
\item The following conventions are adopted:
\begin{itemize}
\item Files containing main programs start with \texttt{M},
i.e. \texttt{Mmyprog.c}. Each main program file should start with a
short one-line comment describing its purpose.
\item Files containing subroutines start with small letters. A
header/interface file \texttt{myprog.h} is automatically generated
from \texttt{myprog.c}. By using header files, dependencies between
different functions will be figured out automatically. Include a
comment of the form \verb#/*^*/# after any block of text that you
want to be included in the header file. Include a comment of the form
\verb#/*< My function description >*/# after a function definition
if you want its interface included in the header file.  It is a good
habit to comment interfaces to all your functions.
\end{itemize}
\item Create an \texttt{SConstruct} file in your directory by following examples from other \texttt{user} directories. 
Inside the triple quotation marks after \texttt{progs=}, replace the
names of their programs with the names of your programs with spaces
between the names.
\item Note that running \texttt{scons} inside your \texttt{user}
directory compiles programs with debugging flags to make them
suitable for debugging with common debuggers such as
\href{http://www.gnu.org/software/gdb/}{gdb}, dbx, or
\href{http://www.etnus.com/}{TotalView}. Running 
\texttt{scons install} inside the top source directory will compile your programs with optimization flags and install them in \texttt{\$RSFROOT}.
\item If you want to share your code with other madagascar users:
\begin{enumerate}
\item If your employer holds the copyright for your software, obtain a permission to distribute under a \href{http://www.gnu.org/copyleft/gpl.html}{GPL license} and 
place a copyright and GPL license notice in each file with code.
\item \href{http://sourceforge.net/account/newuser_emailverify.php}{Register at SourceForge}
and pass your SourceForge user name to one of the project
administrators.  \item Commit your directory to the repository using
\texttt{svn add} and \texttt{svn commit}.  \item Add examples of using your program
under the \texttt{MADAGASCAR/book} tree. Create new directories if
necessary and commit them to the repository. As a rule, programs
without examples are not included in the release version.
\end{enumerate}
\end{enumerate}
