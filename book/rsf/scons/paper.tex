\title{Reproducible computational experiments using SCons}

\lefthead{Fomel \& Hennenfent}
\righthead{Reproductible research}

\author{Sergey Fomel\/\footnotemark[1] and Gilles Hennenfent\/\footnotemark[2]}

\footnotetext[1]{University of Texas at Austin, E-mail: sergey.fomel@beg.utexas.edu}
\footnotetext[2]{University of British Columbia, E-mail: ghennenfent@eos.ubc.ca}

\maketitle

\begin{abstract}
SCons (from Software Construction) is
\end{abstract}

\section{Introduction}

This paper introduces a computational environment for reproducible
computational experiments developed as part of the ``Madagascar''
software package. To reproduce the example experiments in this paper,
you can download Madagascar from \url{http://rsf.sourceforge.net/}. At
the momemnt, the main Madagascar interface is the Unix shell command
line so that you will need a Unix/POSIX system (Linux, Mac OS X,
Solaris, etc.) or a Unix emulation under Windows (Cygwin, SFU, etc.)

Our focus, however, is not only on particular
tools we use in our research but also on the general philosophy of
reproducible computations.

\subsection{Reproducible research philosophy}

Peer review is the backbone of scientific progress. From the ancient
alchemists, who worked in secret on magic solutions to insolvable
problems, the modern science has come a long way to become a social
enterprise, where hypotheses, theories, and experimental results are
openly published and verified by the community. By reproducing and
verifying previously published research, a researcher can take new
steps to advance the progress of science.

Traditionally, scientific disciplines are divided into theoretical and
experimental studies. Reproduction and verification of theoretical
results usually requires only imagination (apart from pencils and
paper), experimental results are verified in laboratories using
equipment and materials similar to those described in the publication.

During the last century, computational studies emerged as a new
scientific discipline. Computational experiments are carried out on a
computer by applying numerical algorithms to digital data. How
reproducible are such experiments? On one hand, reproducing the result
of a numerical experiment is a difficult undertaking. The reader needs
to have access to precisely the same kind of input data, software and
hardware as the author of the publication in order to reproduce the
published result. It is often difficult or impossible to provide
detailed specifications for these components. On the other hand, basic
components of a computational system such as operating systems and
file formats are getting increasingly standardized, and new components
can be shared in principle because they simply represent digital
information transferable over the Internet.

The practice of software sharing has fueled the miraculously efficient
development of Linux, Apache, and many other open-source software
projects.  Its proponents often refer to this ideology as an analog of
the scientific peer review tradition. Eric Raymond, a well-known
open-source advocate, writes \cite[]{taoup}:
\begin{quote}
  Abandoning the habit of secrecy in favor of process transparency and
  peer review was the crucial step by which alchemy became chemistry.
  In the same way, it is beginning to appear that open-source
  development may signal the long-awaited maturation of software
  development as a discipline.
\end{quote}
While software development is trying to imitate science, computational
science needs to borrow from the open-source model in order to sustain
itself as a fully scientific discipline. In words of Randy LeVeque
\cite[]{randy},
\begin{quote}
  Within the world of science, computation is now rightly seen as a third vertex of a
triangle complementing experiment and theory. However, as it is now often practiced,
one can make a good case that computing is the last refuge of the scientific scoundrel [...]
Where else in science can one get away with publishing observations that are
claimed to prove a theory or illustrate the success of a technique without having to
give a careful description of the methods used, in sufficient detail that others can
attempt to repeat the experiment? [...]
Scientific and mathematical journals are filled with pretty pictures these days of
computational experiments that the reader has no hope of repeating. Even brilliant
and well intentioned computational scientists often do a poor job of presenting their
work in a reproducible manner. The methods are often very vaguely defined, and even
if they are carefully defined theywould normally have to be implemented from scratch
by the reader in order to test them. 
\end{quote}

In computer science, the concept of publishing and explaining computer
programs goes back to the idea of \emph{literate programming} promoted
by \cite{knuth} and expended by many other researchers
\cite[]{thimbleby}. In his 2004 lecture on ``better programming'',
Harold Thimbleby says\footnote{url:http://www.uclic.ucl.ac.uk/harold/}
\begin{quote}
  We want ideas, and in particular programs, that work in one place to
  work elsewhere. One form of objectivity is that published science
  must work elsewhere than just in the author's laboratory or even
  just in the author's imagination; this requirement is called
  \emph{reproducibility}.
\end{quote}

\begin{comment}
The quest for peer review and reproducibility is especially important
for computational geosciences and computational geophysics in
particular. The very first paper published in \emph{Geophysics} was
titled ``Black magic in geophysical prospecting''
\cite[]{GEO01-01-00010008,TLE02-03-00280031} and presented an account
of different ``magical'' methods of oil explorations promoted by
entrepreneurs in the early days of geophysical exploration industry.
Although none of these methods exist today, it is not a secret that
industrial practice is full of nearly magical tricks, often hidden
besides a scientific appearance. Only a scrutiny of peer review and
result verification can help us distinguish magic from science and
advance the latter.
\end{comment}

Nearly ten years ago, the technology of reproducible research in
geophysics was pioneered by Jon Claerbout and his students at the
Stanford Exploration Project (SEP).  SEP's system of reproducible
research requires the author of a publication to document creation of
numerical results from the input data and software sources to let
others test and verify the result reproducibility
\cite[]{SEG-1992-0601,matt}.
%% In Claerbout's words
%% \footnote{http://sepwww.stanford.edu/research/redoc/IRIS.html},
%% \begin{quote}
%%   The purpose of reproducible research is to facilitate someone going a step
%%   further by changing something. The first step that someone will want to make
%%   is to be sure that your work is reproducible before they change and improve
%%   upon it. 
%% \end{quote}
%% \begin{quote}
%%   It takes some effort to organize your research to be reproducible.
%%   We found that although the effort seems to be directed to helping other
%%   people stand up on your shoulders, the principal beneficiary is generally
%%   the author herself. This is because time turns each one of us into another
%%   person, and by making effort to communicate with strangers, we help
%%   ourselves to communicate with our future selves.
%% \end{quote}
The discipline of reproducible research was also adopted and
popularized in the statistics and wavelet theory community by
\cite{donoho}. It is referenced in the popular wavelet theory books
\cite[]{hubbard,mallat}. Pledges for reproducible research appear
nowadays in fields as diverse as bioinformatics \cite[]{bioconductor},
geoinformatics \cite[]{geo}, and computational wave propagation
\cite[]{randy}. However, the adoption or reproducible research
practice by computational scientists has been slow.  Partially, this
is caused by difficult and inadequate tools.

%What kind of documentation is required to make geophysical numerical
%experiments reproducible? Geophysical data analysis involves remote datasets
%that are typically passed through several data processing steps before a final
%image is created and some information about the subsurface geology is
%inferred.  Keeping a precise record of the processing history is necessary for
%reproducing and verifying the result. 

%The closest analog from other
%disciplines is medicine. Both geosciences and medical sciences study unique
%objects (the Earth planet and the human organism respectively). Each
%geophysical dataset is in some sense as unique as each medical patient, and
%keeping a careful log of the data analysis history is essential for scientific
%validation of the methods applied to it.

\subsection{Tools for reproducible research}

The reproducible research system developed at Stanford is based on
``make \cite[]{make}''\footnote{Originally, SEP used ``cake'', a
dialect of ``make''
\cite[]{Nichols.sep.61.341,Claerbout.sep.67.145,Claerbout.sep.73.451,Claerbout.sep.77.427}.
The system was converted to ``GNU make'', a more standard dialect, by
\cite{Schwab.sep.89.217}.}, a Unix software construction utility.  The
``make'' program keeps track of dependencies between different
components of the system and the software construction targets, which,
in the case of a reproducible research system, turn into figures and
manuscripts. The targets and commands for their construction are
specified by the author in ``makefiles'', which serve as databases for
defining source and target dependencies. A dependency-based system
leads to rapid development, because when one of the sources changes,
only parts that depend on this source get recomputed.  \cite{donoho}
based their system on MATLAB, a popular integrated development
environment produced by MathWorks \cite[]{matlab}.  While MATLAB is an
adequate tool for prototyping numerical algorithms, it may not be
sufficient for large-scale computation, typical for many applications
in computational geophysics.

``Make'' is an extremely useful utility employed by thousands of
software development projects. Unfortunately, it is not
well designed from the user experience prospective. ``Make'' employs
an obscure and limited special language (a mixture of Unix shell
commands and special-purpose commands), which often appears confusing
to unexperienced users. According to Peter van der Linden, a software
expert from Sun Microsystems \cite[]{linden},
\begin{quote}
  ``Sendmail'' and ``make'' are two well known programs that are
  pretty widely regarded as originally being debugged into existence.
  That's why their command languages are so poorly thought out and
  difficult to learn. It's not just you -- everyone finds them
  troublesome.
\end{quote}
The inconvenience of ``make'' command language is also in its limited
capabilities.  The reproducible research system developed by
\cite[]{matt} includes not only custom ``make'' rules but also an
obscure and hardly portable agglomeration of shell and Perl scripts
that extend ``make'' \cite[]{Fomel.sep.94.matt3}.

Several alternative systems for dependency-checking software
construction have been developed in recent years. One of the most
promising new tools is SCons, enthusiastically endorsed by
\cite{dubois}. The SCons initial design won the Software Carpentry
competition sponsored by Los Alamos National Laboratory in 2000 in the
category of ``a dependency management tool to replace make''.  Some of
the main advantages of SCons are:
\begin{itemize}
\item SCons configuration files are Python scripts. Python is a modern
  programming language praised for its readability, elegance,
  simplicity, and power \cite[]{python1,python2}.
  \cite{TLE21-03-02600267} recommend Python as the first programming
  language for geophysics students.
\item SCons offers reliable, automatic, and extensible dependency
  analysis and creates a global view of all dependencies -- no more
  ``make depend'', ``make clean'', or multiple build passes of
  touching and reordering targets to get all of the dependencies.
\item SCons has built-in support for many programming languages and
  systems: C, C++, Fortran, Java, LaTeX, CVS, and others.
\item While ``make'' relies on timestamps for detecting file changes
  (creating numerous problems on platforms with different system
  clocks), SCons uses by default a much more reliable detection
  mechanism employing MD5 signatures. It can detect changes not only
  in files but also in commands used to build them.
\item SCons provides integrated support for parallel builds.
\item SCons provides configuration support analogous to the ``autoconf''
  utility for testing the environment on different platforms.
\item SCons is designed from the ground up as a cross-platform tool.
  It is known to work equally well on both POSIX systems (Linux, Mac
  OS X, Solaris, etc.) and Windows.
\item The stability of SCons is assured by an incremental development
  methodology utilizing comprehensive regression tests.
\item SCons is publicly released under a liberal open-source license\footnote{As of
    time of this writing, SCons is in a beta version 0.96 approaching
    the 1.0 official release. See \url{http://www.scons.org/}.}
\end{itemize}

In this paper, we propose to adopt SCons as a new platform for
reproducible research in scientific computing.

\subsection{Paper organization}

To demonstrate our adoption of SCons for reproducible computational
experiments, we will follow simple examples. 

\begin{comment}
\newpage
\section{Madagascar open source code}
%
Madagascar's homepage is \url{http://rsf.sourceforge.net}. Madagascar
source code is proposed in two versions:
\href{https://sourceforge.net/project/showfiles.php?group_id=162909}{stable}
and
\href{http://rsf.sourceforge.net/wiki/index.php/Svn-url}{development}.
The stable version is a snapshot of Madagascar at a given time. It was
installed on different platforms and tested before being released.
Updates are typically done every few months as opposed to the
development version which is updated every few hours or days by a
dynamic team of developers. As such, there is no guarantee that the
development version will be fully functional and stable at any given
point in time. In the remaining of this paper, we assume that you have
successfully installed Madagascar stable version and that you have an
Internet connection\footnote{XXX provide alternate mean to download
  Lena.img if no Internet connection XXX}.
\end{comment}

\section{Example experiments}

%
The main \texttt{SConstruct} commands defined in our reproducible
research environment are collected in Table~\ref{tbl:commands}.

\tabl{commands}{Basic \texttt{rsfproj} methods.}{
\begin{center}
\begin{tabular}{|p{0.9\textwidth}|} \hline
\noindent\texttt{Fetch(data\_file,dir[,ftp\_server\_info])}\\
\indent A rule to download \texttt{$<$data\_file$>$} from a specific
directory \texttt{$<$dir$>$} of an FTP server
\texttt{$<$ftp\_server\_info$>$}. \\
\hline 
\noindent\texttt{Flow(target[s],source[s],command[s][,stdin][,stdout])}\\
\indent A rule to generate \texttt{$<$target[s]$>$} from
\texttt{$<$source[s]$>$} using \texttt{command[s]} \\
\hline 
\noindent\texttt{Plot(intermediate\_plot[,source],plot\_command)} or\\
\texttt{Plot(intermediate\_plot,intermediate\_plots,combination})\\
\indent A rule to generate \texttt{$<$intermediate\_plot$>$} 
in the working directory. \\
\hline 
\noindent\texttt{Result(plot[,source],plot\_command)} or\\ 
\texttt{Result(plot,intermediate\_plots,combination})\\
\indent A rule to generate a final \texttt{$<$plot$>$} in
the special \texttt{Fig} folder of the working directory. \\
\hline 
\noindent\texttt{End()} \\ 
\indent A rule to collect default targets. \\
\hline 
\end{tabular}
\end{center}}

These commands are defined in \texttt{\$RSFROOT/lib/rsfproj.py} where
\texttt{RSFROOT} is the environmental variable to the directory where
Madagascar is installed. The source of this file is in
\href{http://svn.sourceforge.net/viewvc/rsf/trunk/python/rsfproj.py?view=markup}{python/rsfproj.py}.

\subsection{Example 1}

To follow the first example, select a working project directory and
copy the following code
to a file named \texttt{SConstruct}\footnote{The source of this file is also accessible at
  \href{http://svn.sourceforge.net/viewvc/rsf/trunk/book/rsf/scons/easystart/SConstruct?view=markup}{book/rsf/scons/easystart/SConstruct}.}.

%
\definecolor{frame}{rgb}{0.905,0.905,0.905}
\lstset{language=Python,backgroundcolor=\color{frame},showstringspaces=false,numbers=left,numberstyle=\tiny}
\lstinputlisting[firstline=1,lastline=16,frame=single]{easystart/SConstruct}

This is our ``Hello, World'' example that illustrates the basic use of
some of the commands presented in Table~\ref{tbl:commands}. The plan
for this experiment is simply to download data from a public data
server, to convert it to an appropriate file format and to generate a
figure for publication. But let us have a closer look at the
\texttt{SConstruct} script and try to decorticate it.\\

\lstinputlisting[firstline=1,lastline=1,frame=single]{easystart/SConstruct}
%
is a standard Python command that loads the Madagascar project
management module \texttt{rsfproj.py} which provides our extension to
SCons.\\

\lstinputlisting[firstline=4,lastline=4,frame=single]{easystart/SConstruct}
%
instructs SCons to connect to a public data server (the default server
if no FTP server information is provided) and to fetch the data file
\texttt{lena.img} from the \texttt{data/imgs} directory. 
\begin{comment}
Note that
Madagascar expects a \texttt{data} folder on top of the specified
directory (i.e.  \texttt{imgs}). In the directory where you have your
SConstruct, running \texttt{scons lena.img} on the command line will
download the file \texttt{lena.img}.  The equivalent command lines are
\footnote{use login: anonymous, password: anonymous}
\begin{verbatim}
bash$ ftp egl.beg.utexas.edu
ftp> bin
ftp> cd data/imgs
ftp> get lena.img
ftp> bye
\end{verbatim}
\end{comment}
%
Try running ``\texttt{scons lena.img}'' on the command line. The successful output should look like
\begin{verbatim}
bash$ scons lena.img
scons: Reading SConscript files ...
scons: done reading SConscript files.
scons: Building targets ...
retrieve(["lena.img"], [])
scons: done building targets.
\end{verbatim}
with the target file \texttt{lena.img} appearing in your directory.
In the following examples, we will use \texttt{-Q} (quiet) option of
\texttt{scons} to suppress the verbous output.

\lstinputlisting[firstline=7,lastline=9,frame=single]{easystart/SConstruct}
%
prepares the Madagascar header file \texttt{lena.hdr} using the
standard Unix command \texttt{echo}. 
%
\begin{verbatim}
bash$ scons -Q lena.hdr
echo n1=512 n2=513 d1=1 d2=1 in=lena.img data_format=native_uchar > lena.hdr
\end{verbatim}
%
Since \texttt{echo} does not take a standard input, stdin is set to 0
in the Flow command otherwise the first source is the standard input.
Likewise, the first target is the standard output unless otherwise
specified. Note the triple quotes used because the actual command is
broken into several lines in the SConstruct. Note also that
\texttt{lena.img} is refered as \texttt{\$SOURCE} in the command. This
allows us to change the name of the source file without changing the command.

The data format of the \texttt{lena.img} image file is \texttt{uchar}
(unsigned character), the image consists of 513 traces with 512
samples per trace.  Our next step is to convert the image
representation to floating point numbers and to window out the first
trace so that the final image is a 512 by 512 square. The two
transformations are conveniently combined into one with the help of a Unix pipe.

\lstinputlisting[firstline=12,lastline=12,frame=single]{easystart/SConstruct}
%
\begin{verbatim}
  bash$ scons -Q lena
  scons: *** Do not know how to make target `lena'.  Stop.
\end{verbatim}
What happened? In the absence of the file suffix, the \texttt{Flow}
command assumes that the target file suffix is ``\texttt{.rsf}''. Let us try again.
\begin{verbatim}
scons -Q lena.rsf
< lena.hdr /path/to/RSF/bin/sfdd type=float | /path/to/RSF/bin/sfwindow f2=1 > lena.rsf
\end{verbatim}
Notice that Madagascar modules \texttt{sfdd} and \texttt{sfwindow} get
substituted for the corresponding short names in the
\texttt{SConstruct} file. The file \texttt{lena.rsf} is in a regularly
sampled format\footnote{See
\url{http://rsf.sourceforge.net/wiki/index.php/Format}} and can be
examined, for example, with \texttt{sfin lena.rsf}\footnote{See
\url{http://rsf.sourceforge.net/wiki/index.php/Programs\#sfin}.}.
\begin{verbatim}
bash$ sfin lena.rsf
lena.rsf:
    in="/datapath/lena.rsf@"
    esize=4 type=float form=native
    n1=512         d1=1           o1=0
    n2=512         d2=1           o2=1
        262144 elements 1048576 bytes
\end{verbatim}
In the last step, we will create an plot file for displaying the image
on the screen and later including it in the publication.
%
\lstinputlisting[firstline=15,lastline=19,frame=single]{easystart/SConstruct}
%
Notice that we broke the long command string into multiple lines by
using Python's triple quote syntax. All the extra white space will be
ignored when the multiple line string gets translated into the command
line.  The \texttt{Result} command has special targets associated with
it. Try, for example, ``\texttt{scons lena.view}'' to observe the
figure \texttt{Fig/lena.vpl} generated in a specially created
\texttt{Fig} directory and displayed on the screen. The output should
look like Figure~\ref{fig:lena}.
%
\inputdir{easystart}
\sideplot{lena}{height=.25\textheight}{The result of Example 1.}
%
The reproducible script ends with\\
%
\lstinputlisting[firstline=22,lastline=22,frame=single]{easystart/SConstruct}

Ready to experiment? Try some of the following:
\begin{enumerate}
\item clean
\item clip
\item timer 
\item checkpar
\end{enumerate}

The summary of SCons commands is given in Table~\ref{tbl:scons}.

\subsection{Example 2}

Since SConstructs are Python scripts, we can also use all the
flexibility and power of the Python language in our
Madagascar reproducible scripts. A demo script is available in the
\texttt{rsf/scons/rsfpy} directory of Madagascar book section, but,
rather than commenting it line-by-line, we select some parts of
interest.

In a Madagascar reproducible script (i.e. SConstruct) we can declare
Python variables\\

\lstinputlisting[firstline=3,lastline=5,frame=single]{rsfpy/SConstruct}

\noindent that can be used later, for example, to define our customized plot
command as a Python function\\

\lstinputlisting[firstline=7,lastline=11,frame=single]{rsfpy/SConstruct}

This Python function, named ``grey()'', can then be called in Plot or Result
commands, e.g.\\

\lstinputlisting[firstline=52,lastline=52,frame=single]{rsfpy/SConstruct}

We can define a Python dictionary, e.g.\\

\lstinputlisting[firstline=36,lastline=37,frame=single]{rsfpy/SConstruct}

\noindent and loop over its entries, e.g.\\

\lstinputlisting[firstline=38,lastline=43,frame=single]{rsfpy/SConstruct}

Note that the title of the plots is obtained by concatenating Python
strings\\

\lstinputlisting[firstline=40,lastline=40,frame=single]{rsfpy/SConstruct}

Python strings can also be used to define sequences of commands used
in several Flows, e.g.\\

\lstinputlisting[firstline=71,lastline=74,frame=single]{rsfpy/SConstruct}

Finally, in our Madagascar reproducible script, we may want the option
to pass command line arguments when running SCons or use default
values otherwise, e.g.\\

\lstinputlisting[firstline=70,lastline=70,frame=single]{rsfpy/SConstruct}

Running \texttt{scons} only, the default value set for fthr (i.e. 70)
is used whereas running \texttt{scons fthr=68} set fthr to a command
line specified value.

This is by no mean an exhaustive list of options but, hopefully, it
gives you a flavor of the powerful tool you have in hands. Enjoy!

\subsection{Useful SCons commands for reproducible scripts}

On top of SCons standard options (\texttt{scons --help} for more
details), Madagascar has its own SCons options. We already saw
\texttt{scons plot.view} that displays \texttt{plot.vpl} (in the
\texttt{Fig} folder) obtained in a Result command. \texttt{scons view}
displays the result plots one after the other.

It is also possible to check parameters for Madagascar programs in
SCons Flow commands using the CHECKPAR option (\texttt{scons
  CHECKPAR=y target}). Note that CHECKPAR is an experimental option
and will be enhanced in the future to include parameter ranges and
other safety checks.

To time the execution of processing flows in a SConstruct use the
TIMER option (\texttt{scons TIMER=y target}).

\texttt{scons lock} is used to secure result plots and copy them from
the \texttt{Fig} folder of your working directory to your
\texttt{\$RSFFIGS} folder where \texttt{RSFFIGS} is the environmental
variable to the directory where you want Madagascar to put your key
Madagascar result plots. Note that this is a necessary step before
creating a reproducible documentation. \texttt{scons plot.flip} runs
\texttt{xtpen Fig/plot.vpl /locked/figures/plot.vpl} to flip between
the new and locked figure. This is useful when detecting changes.

\section{Creating a reproducible documentation}
%
A Madagascar reproducible paper is a paper written in \LaTeX\ and
whose figures are either generated by Madagascar reproducible scripts
or available for download, e.g.  this paper!  (\texttt{paper.tex}
available in the \texttt{rsf/scons/} directory of Madagascar book
section).

The main SConstruct command set in our reproducible research
environment and related to documentation is

\noindent\texttt{Paper(paper\_name,[,use][,lclass])}\\
\indent A rule to compile \texttt{$<$paper\_name$>$} (Madagascar
\LaTeX\ document) using the \LaTeX\ packages specified in \texttt{use}
and the \LaTeX\ style specified in \texttt{lclass} (default is
\texttt{geophysics.cls}).

This command is defined in \texttt{\$RSFROOT/lib/rsftex.py}.

\subsection{Example}

\lstinputlisting[firstline=1,lastline=3,frame=single]{SConstruct}

This SConstruct generates this paper but it can also compile
\texttt{velan.tex} in the same directory. Note that there is no Paper
command for \texttt{paper.tex} since it is the default documentation
name. Optional \LaTeX\ packages and style used in \texttt{paper.tex}
are passed in the End command.

Let's now have a closer look at \texttt{paper.tex} to understand how
the figures of the documentation are linked to the reproducible
scripts that created them. First of all, note that \texttt{paper.tex}
is not a regular \LaTeX\ document but only its body (no
$\backslash$documentclass, $\backslash$usepackage, etc.). In our
paper, Fig.~\ref{fig:lena} was created in the project folder
\texttt{easystart} (sub-folder of our documentation folder) by the
result plot \texttt{lena.vpl}. In the \LaTeX\ source code, it
translates as\\

\lstset{language=[LaTeX]TeX,backgroundcolor=\color{frame},showstringspaces=false,numbers=left,numberstyle=\tiny}
\lstinputlisting[firstline=377,lastline=378,frame=single]{paper.tex}

The $\backslash$inputdir command points to the project directory and
the $\backslash$plot command calls \texttt{$<$result\_name$>$}. The
\LaTeX\ tag of the figure is \texttt{fig:$<$result\_name$>$}. Only the
first time the paper is compiled, the VPL result file is automatically
converted to PDF file format unless the VPL result file is updated and
locked in which case, the PDF file is also automatically updated at
the next paper compilation.

\subsection{Useful SCons commands for reproducible documentation}

To actually compile this paper you first need to run and lock the
\texttt{easystart} project. Go in the \texttt{easystart} folder and
run \texttt{scons lock}. Go back to the documentation folder and run
\texttt{scons pdf} (alternatively \texttt{scons
  $<$paper\_name$>$.pdf}). Use \texttt{scons read} (alternatively
\texttt{scons $<$paper\_name$>$.read}) or your favorite PDF reader to
read this paper reproduced by yourself...

SERGEY: COULD YOU PLEASE HELP ME WITH scons wiki \& scons html IF YOU
WANT TO INCLUDE THEM?

\bibliographystyle{seg}
\bibliography{SEG,SEP2,scons}

%%% Local Variables: 
%%% mode: latex
%%% TeX-master: t
%%% End: 
