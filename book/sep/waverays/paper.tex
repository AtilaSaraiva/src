
%\noIdocMenu

\title{Wavefront construction using waverays}
\author{Hector Urdaneta}

%
\lefthead{Urdaneta}
\righthead{Waverays and wavefronts}
\footer{SEP--80}

\maketitle
%
%
%\def\CAKEDIR{.}
%\def\figdir{./Fig}
%
\begin{abstract}
A method for computing first arrival traveltimes and amplitudes in
a general two-dimensional (\mbox{2-D}) velocity model is 
presented. The method is the result of merging two
recently published ray tracing methods. The product is a
very robust algorithm that is able to produce broadband wave phenomena,
such as dispersion and wavelength dependent scattering.
Its ability to produce broadband wave phenomena, is achieved
by performing a wavelength-dependent smoothing of the velocity model
across wavefronts. In the limit of high frequency, the method
reduces to geometrical ray theory.
The method is able to illuminate areas of large geometrical
spreading where conventional ray tracing methods may give no
arrivals. The method is tested on synthetic complex
velocity models.
\end{abstract}

\section{Introduction}

Traditionally traveltime and amplitude calculations have been
performed by ray tracing. Different ray tracing algorithms exist that
are well known and well documented.  They include ray
bending \cite[]{Julian}, shooting rays \cite[]{Dines} and paraxial
extrapolation \cite[]{Cervenys}. More recently, several new methods
have appeared and are enjoying an increasing popularity.  They include
finite differences \cite[]{GEO55-05-05210526,Podvin,GEO56-06-08120821}
and shortest path rays \cite[]{GEO56-01-00590067}.

This paper presents a review of two new ray tracing methods and
explores some of the possibilities produced by their fusion.  The
first method is Lomax's waveray method for approximating broadband
wave propagation through complex velocity structures \cite[]{Lomax}.
The second method was developed at the NORSAR institute in Norway
by \cite{Vinje}.  As will be shown later, both methods have their own
advantages and drawbacks, but when they are fused, they interfere
positively.  The combined product produces a very robust method, which
approximates broadband wave phenomena in complex velocity models.

The first two parts of this paper describe the 
basic characteristics of each method and their
implementations. The paper also reviews some of the work
done in the last two references listed above. 
In the last part, I discuss
the combined method. Implementation issues and synthetic
examples are shown.

\section{LOMAX'S WAVERAYS}

Two basic ideas characterize Lomax's method: (1)
it does a wavelength-dependent velocity smoothing and
(2) it uses Huygen's principle to track the motion
of narrow-band wavefronts at a number of center frequencies.
Narrow-band wavefronts are defined as surfaces (lines in \mbox{2-D})
of constant phase or traveltime in a narrow-band ``wavefield''.
As these narrow-band wavefronts propagate with time they define
a wavepath, which is frequency dependent. The wavelength-dependent
smoothing of the velocity is done by averaging with a Gaussian weighting 
curve. The smoothing is done along the wavefronts. The final result; i.e,
the broadband wavefield, is constructed by summing the results of 
independent narrow-band wavefields at many center frequencies.

Three important advantages of using Lomax's waveray over conventional
ray tracing methods are: 
\begin{itemize}
 \item It increases the stability of wavepaths compared
  to the paths produced by high frequency methods, due to the
  wavelength-dependent smoothing.
 \item It provides 
  waverays with a sensitivity that produces
  frequency dependent scattering as a function of the ratio of 
  wavelength to the characteristic size of the scattering region.
  Figure~\ref{fig:lomaxsbs} illustrates this point.
 \item It is capable of handling large to small inhomogeneity sizes,
  since in the former case it is similar to ray theory and in the
  latter it responds to a smooth, averaged velocity structure. 
\end{itemize}

\cite{Lomax} approximates 
the narrow-band wavefronts at any time by a plane wavefront (see
Figure~\ref{fig:lomax1}). This approximation requires that the radius
of curvature of the wavefront be large relative to a wavelength.  A
better approximation to the wavefronts could probably be obtained
using a parabolic approximation.  For the sake of computational time,
plane wavefronts are used.

Before considering the details of the waveray technique, two points
need to be emphasized.  First, \cite{Lomax} points out, ``it is the
wavelength dependent smoothing that makes the waveray method a
broadband wave propagation technique, and distinguishes it from the
high frequency ray methods.''\ Second, there are no equations that
give the waveray method a theoretical basis.  Its support comes from
the fact that it reproduces high-frequency ray propagation and
produces a good approximation of broadband wave phenomena.

\inputdir{XFig}

\sideplot{lomax1}{width=3.in}{Waveray wavepath and
wavelength-dependent velocity smoothing at point ${\bf \vec{x}}_{\nu}$
Adapted from \cite{Lomax}.
}

\subsection{Waveray implementation}

In the waveray method, the wavelength-dependent
averaging of the velocity is done dynamically as a function
of the position and orientation of the plane wavefronts.
The velocity
averaging is done using a Gaussian weight curve, centered at the wavepath
location (see Figure~\ref{fig:lomax1}). Equation~\ref{eqn:velavg} expresses the
wavelength averaged velocity $\overline{v}$ at a point 
($ \vec{\bf x}_{\nu} $) for a wave period $\bf T$ \cite[]{Lomax},
%
\begin{equation}
 \overline{v}\, (\vec{\bf x}_{\nu},{\bf T\/}) = \frac{
  \displaystyle \int_{-\infty}^{\infty}
  \omega(\gamma)\
  c\, [\vec{\bf x}(\gamma, {\bf T\/})]\
  d\gamma}
  {\displaystyle \int_{-\infty}^{\infty}
  \omega(\gamma)\ d\gamma}
\label{eqn:velavg}
\end{equation}
%
where $\gamma$ is the arc length along the wavefront away from 
$\vec{\bf x}_{\nu}$ expressed in wavelengths.
$c\, (\vec{\bf x})$ is the velocity at point $\vec{\bf x}$
and $\omega(\gamma)$ is the Gaussian weight curve:
%
\begin{equation}
 \omega(\gamma) = e^{-4 \ln 2\, \cdot\, (\gamma/\alpha)^{2}}
\label{eqn:gauss}
\end{equation}
%
where $\alpha$ specifies the half width of the Gaussian bell in
wavelengths.

($\vec{\bf x}(\gamma,{\bf T\/})$) is the position along the 
instantaneous straight wavefront given by the recursive relation
\cite[]{Lomax}:
%
\begin{equation}
 \vec{\bf x}(\gamma, {\bf T\/}) = \vec{\bf x}_{\nu} + \frac{{\bf T\/}}{2 \pi}
 \int_{0}^{\gamma} c\, [\vec{\bf x}(\gamma\, ', {\bf T\/})]\
 {\bf \hat{n}}({\bf T\/})\
 d\gamma\, '
\label{eqn:post}
\end{equation}
%
where ${\bf \hat{n}}$ is the unit normal to the wavepath at
point $\vec{\bf x}_{\nu}$.

The discrete representation of equation~\ref{eqn:velavg} is given
by equation~\ref{eqn:dvelavg}:
%
\begin{equation}
 \overline{v}\, (\vec{\bf x}_{\nu},{\bf T\/}) = \frac{
 \displaystyle \sum_{n=-N}^{N}
 \omega_{n}\ c\, [\vec{\bf x}_{\nu}^{\, n}({\bf T\/})] }
 {\displaystyle \sum_{n=-N}^{N}
 \omega_{n}}
\label{eqn:dvelavg}
\end{equation}
%
where the integral has been replaced by a finite sum over $2\,N+1$
{\bf control points\/}. The position of the control points along the
wavefronts
are given by equations~\ref{eqn:dpost0} and \ref{eqn:dpost}.
%
\begin{equation}
 \vec{\bf x}_{\nu} = \vec{\bf x}_{\nu}^{\, 0}
\label{eqn:dpost0}
\end{equation}
%
\begin{equation}
\vec{\bf x}_{\nu}^{\, n} = \left\{ \begin{array}{lll}
        \displaystyle
        \vec{\bf x}_{\nu}^{\, n-1} + \frac{\gamma_{max}}
            {4\pi N}\ \lambda\, (\vec{\bf x}_{\nu}^{\, n-1})\ {\bf \hat{n}} &
        \; &
        n=1,2, \ldots, N. \\
        \, \, \,          \\
        \displaystyle
        \vec{\bf x}_{\nu}^{\, n+1} - \frac{\gamma_{max}}
            {4\pi N}\ \lambda\, (\vec{\bf x}_{\nu}^{\, n+1})\ {\bf \hat{n}} &
        \; &
        n=-1,-2, \ldots, -N. \end{array}
        \right.
%}
\label{eqn:dpost}
\end{equation}

These two equations are the discrete version of equation~\ref{eqn:post}, but,
the dependence on the wavelength has been made explicit.
Notice that the subscript $\nu$ of $\vec{\bf x}_{\nu}^{\, n}$ 
runs along the wavepath and the
superscript $n$ runs along the wavefront.
$\gamma_{max}$ specifies the largest distance in wavelengths along
the wavefront at which smoothing is applied. 

The discrete equivalent of the Gaussian weight function is:
%
\begin{equation}
 \omega_{n} = e^{-4 \ln 2\, \cdot\, (\gamma_{n}/\alpha)^{2}}
\label{eqn:dgauss}
\end{equation}
%
where the distance $\gamma_{n}$ along the wavefront in wavelengths
is expressed as:
%
\begin{equation}
 \gamma_{n} = \frac{n\, \gamma_{max}}{N}
\label{eqn:gamma}
\end{equation}

The motion of the waverays along the direction of propagation 
is expressed by the following equation:
%
\begin{equation}
 \vec{\bf x}_{\nu+1} = \vec{\bf x}_{\nu} + 
 \overline{v}_{\nu}\, \Delta t\; {\bf \hat{s}}
\label{eqn:mvwavpt}
\end{equation}
%
where $\overline{v}_{\nu} = \overline{v}\, (\vec{\bf x}_{\nu},{\bf T\/})$,
$\Delta t$ is the time step and ${\bf \hat{s}}$ is a unit vector
that moves along the direction of propagation.

\sideplot{lomax2}{width=3.in}{Waveray wavepath calculation.
Huygen's principle is used to obtain the bending
$\Delta {\bf \hat{s}}$
of the wavepath from points $\vec{\bf x}_{\nu}^{\, 1}$ 
and $\vec{\bf x}_{\nu}^{\, -1}$.
Adapted from Lomax (1994).
}

The change in direction of the waverays is approximated by the 
difference in movement between the first control point on either
side of the wave location $\vec{\bf x}_{\nu}$ as shown in
Figure~\ref{fig:lomax2}:
%
\begin{equation}
 \Delta {\bf \hat{s}} = - \left( \frac{
    \overline{v}_{\nu}^{\, 1} - \overline{v}_{\nu}^{\, -1}}
    {| \vec{\bf x}_{\nu}^{\, 1} - \vec{\bf x}_{\nu}^{\, -1} |} \right) \,
 \Delta t \; {\bf \hat{n}}
\label{eqn:cdir}
\end{equation}
%
where $\overline{v}_{\nu}^{\, 1}$ and $\overline{v}_{\nu}^{\, -1}$
are the wavelength averaged velocities at the wavefront points
$\vec{\bf x}_{\nu}^{\, 1}$ and $\vec{\bf x}_{\nu}^{\, -1}$
respectively.

Finally, the half width parameter $\alpha$ and the truncation 
parameter $\gamma_{max}$ are set at $\alpha = 2.0$
and $\gamma_{max} = 1.5$, based on \cite{Lomax}
calibration. The number of control points $N$ is set 
proportional to the ratio $T/\Delta t$ of the wave period over the 
time step. 

Figure~\ref{fig:lomaxsbs} shows the significant differences between
the waveray and ray methods. Notice how a high frequency ray
is scattered by the small velocity anomaly,
while the waveray's wavepath is little deflected. Note also, how the
third ray (from right to left) is not perturbed by the low velocity
anomaly, while the waveray wavepath is deflected.
The wavelength-dependent velocity averaging smoothes out
small velocity variations and causes the wavepath to be affected
from velocity variations away from it.

\inputdir{.}

\plot{lomaxsbs}{width=\textwidth}{Left frame shows a fan of high frequency 
ray paths. Right frame shows a fan of 12 Hz waveray wavepaths.
The straight segments perpendicular to the waverays represent the
instantaneous wavefronts. The velocity model is defined by two 
circular anomalies drawn in a homogeneous background. The black dot
(located at 1000 m. by 1250 m. in depth) depicts a low velocity
anomaly. The white circle a high one.
}
%\newpage


\section{NORSAR WAVEFRONT CONSTRUCTION}
\inputdir{XFig}

The main idea behind the ``NORSAR'' method \cite[]{Vinje}
is to compute ray parameters 
along wavefronts instead of computing them from independently traced
rays, as conventional ray methods do. Wavefronts are defined as
isochron traveltime curves (lines in \mbox{2-D}) from the source. 
New wavefronts are constructed from previous ones, by ray tracing over 
a time step. As wavefronts expand out, new rays are interpolated
between rays that go further apart than a predefined distance {\it DSmax\/}.
Figure~\ref{fig:norsar1} illustrates how wavefronts expand out, by ray 
tracing from a time $\tau$ to a time $\tau + \Delta \tau$. The
dashed line on Figure~\ref{fig:norsar1} represents the new
wavefront. The
solid dots represent the end points of the rays. The distance
between contiguous end points is checked against the predefined maximum
distance {\it DSmax\/}. If they are located further than 
{\it DSmax\/}, a new point (empty dots) is interpolated.

The interpolation of these
new points over the wavefront is done using a vectorial
third order polynomial
${\bf \vec{x}}(s) = {\bf \vec{a}} s^{3} + {\bf \vec{b}} s^{2} +
{\bf \vec{c}} s + {\bf \vec{d}}$.
The polynomial is evaluated as a function of the normalized
distance $s$ between points ${\bf \vec{x}}^{\, i}$ and 
${\bf \vec{x}}^{\, i+1}$.
Also a scalar third order polynomial is used to interpolate amplitude values
and the ray's angle of direction [see \cite{Vinje}].

The key property of this procedure is that it
produces a fairly constant density of rays over C1 models
\cite[]{Vinje} (see Figure~\ref{fig:gauss-wv}), illuminating
zones with high geometrical spreading where conventional ray tracing
have shadow zones.

\sideplot{norsar1}{width=\textwidth}{New wavefronts (dashed lines)  are
constructed from the previous wavefront (solid line), by ray
tracing a fix
number of time steps. New rays are interpolated between points
on the wavefront that lay further than a predefined distance.
Adapted from \cite{Vinje}.
}

Rays are eliminated if they go out of the model boundaries.
They may also be eliminated if a wavefront crosses over itself, as shown on
Figure~\ref{fig:norsar2}. The ``self-crossing'' of the wavefronts may
correspond to a caustic or to the intersection of rays from 
different parts of the model. Once again, as in the Lomax algorithm,
for the sake of computational
time and also of memory, a ``first arrival'' mode should be used.
This first arrival mode removes all later arrivals.
When the number of points in a wavefront becomes less than a certain
value (e.g. 4 points), the algorithm stops.

Traveltimes and amplitudes are interpolated into a rectangular
grid. Ray cells, defined as the area enclosed by a pair of contiguous rays
and wavefronts, are checked for the presence of grid points
(see Figure~\ref{fig:norsar3}). Traveltimes at
the receivers are estimated by computing the following quantities:

\begin{enumerate}
 \item the distances $d_{1}$ and $d_{2}$ from the receiver perpendicular
 to the two rays.
 \item the normalized distance $s$ along the wavefront
 $s = d_{1}/(d_{1}+d_{2})$.
 \item the interpolated point ${\bf \vec{x}}(s)$ over the old wavefront.
 \item the distance $l_{r}$ from ${\bf \vec{x}}(s)$ to the receiver
 \item the velocity $v_{mid}$ in the midpoint of the segment $l_{r}$.
\end{enumerate}

The traveltime at the receiver is then estimated to be:
%
\begin{equation}
 t_{rec} = t + \frac{l_{r}}{v_{mid}}
\label{eqn:tt}
\end{equation}
%
where $t$ is the traveltime to the old wavefront.

\sideplot{norsar2}{width=\textwidth}{The new wavefront crosses itself.
If only first arrivals are wanted, the points behind the crossing
(points no. 7, 8, 9) are removed from the wavefront.
Adapted from \cite{Vinje}.
}

In computing amplitudes,
the geometrical spreading factor $\sqrt{(r_{1}+r_{2})/(R_{1}+R_{2})}$,
gives the ratio between the amplitude of one wavefront to the next
one. $R_{1}$, $R_{2}$, $r_{1}$ and $r_{2}$ are shown on
Figure~\ref{fig:norsar4}. The amplitude estimation at
the receivers is also obtained in this way, where the distances
$d_{1}$ and $d_{2}$ are used for $R_{1}$ and $R_{2}$.

Figure~\ref{fig:gauss-wv} shows an example of the Norsar method
run over a highly contrasted velocity model. The velocity model 
is a pair of Gaussian bell curves. The distance between
the peaks is 48 meters with a drop of 4 km/s in velocity.

\sideplot{norsar3}{width=\textwidth}{Traveltimes and amplitudes are found
at receivers by interpolating within each ray cell. The ray cell is
defined by $Ray_{1}$ and $Ray_{2}$, and by the new wavefront and
the previous wavefront.
Adapted from \cite{Vinje}.
}

A final point on Norsar's method is, as said by \cite{Vinje}:
``the way the ray tracing between each wavefront is performed
is irrelevant to the idea of the wavefront construction''.
We notice that all along the discussion on the NORSAR method,
ray tracing was kept as an abstract idea. With this in mind
we proceed to merge the Lomax algorithm, as the ray tracing 
algorithm for the NORSAR method. Another advantage of the
NORSAR method is that the estimation of ray parameters (as
traveltimes, amplitudes, etc.) does not come from a 
posteriori interpolation between single, separate rays,
but instead directly from previously constructed wavefronts.

\sideplot {norsar4}{width=\textwidth}{Amplitudes are computed from
the previous wavefront. The geometrical spreading factor
gives the ratio between the amplitude $A_{i}$ at the previous
wavefront and the new amplitude
value $A_{i+1}$.
}

\inputdir{gauss}
\plot {gauss-wv}{width=6.in}{The ``NORSAR'' or
wavefront construction
method covers the whole model with wavefronts and rays.
The model presents a strong variation
in velocity, given by a Gaussian bell with a maximum amplitude of 5000 m/s
and a second Gaussian bell with a minimum amplitude of 1000 m/s. The
background is 3000 m/s.}

%\newpage

\section{WAVERAYS AND WAVEFRONTS}

The product of merging the two previously discussed ray tracing methods
is sketched in the following pseudo-code:

\begin{tabbing}
Hel \= Hello \= Hello \= Hello \= Hello \kill
\bf 1 \> Allocate memory for wavefronts and output \\
\bf 2 \> for each source \\
\bf 3 \> \> \em initialize  \\
\bf 4 \> \> while number of points in wavefront $> 4$ \\
\bf 5 \> \> \> \em propagate wavefront \\
\bf 6 \> \> \> \em check for self crossing \\
\bf 7 \> \> \> \em check for rays that go out of bounds \\
\bf 8 \> \> \> \em eliminate self crossing and outofbound rays \\
\bf 9 \> \> \> \em calculate amplitudes \\
\bf 10 \> \> \> \em gridding \\
\bf 11 \> \> \> \em interpolate new rays \\
\bf 12 \> \> \> \em movwavf
\end{tabbing}

The algorithm defines the data structure ``{\it cube\/}'' at line {\bf 1}.
In it, the ray parameters are stored as wavefronts
propagate.

\begin{tabbing}
Hello Hello \= Hello \= Hello \= Hello \=       \kill
\> {\bf struct heptagon} \{     \\
\> \>  {\bf struct point} \{    \\
\> \> \>    {\bf float} x;      \\
\> \> \>    {\bf float} z; \} x0, x1;   \\
\> \>  {\bf float} angle;       \\
\> \>  {\bf float} ampl;        \\
\> \>  {\bf char} cf; \} $*$\it cube;
\end{tabbing}

where:
\begin{itemize}
\item {\it cube\/}[ii].x0 contains the starting position of a ray on a
wavefront at a time $\tau$. 
\item {\it cube\/}[ii].x1 contains the arriving position of a ray on a
wavefront at a time step later, $\tau + \Delta \tau$. 
\item {\it cube\/}[ii].angle gives the arriving angle 
at point {\it cube\/}[ii].x1.
\item {\it cube\/}[ii].ampl is the computed amplitude 
at point {\it cube\/}[ii].x1.
\item {\it cube\/}[ii].cf is a flag that defines if point 
{\it cube\/}[ii].x1 is
in bounds, out of bounds, or belongs to the inner section of
a self crossing wavefront.
\end{itemize}

The index ii runs over the points of a wavefront. 
Since there is no a priori way of determining how large can a
wavefront grow over a velocity model, a predefined limit ({\it nrmax})
of {\it cube\/} elements 
is established at the beginning of the algorithm, which defines
the total memory allocated for {\it cube\/}. In other words, 
{\it nrmax} is the maximum number of points that a wavefront can have.
If the wavefront grows bigger than {\it nrmax} points the algorithm is
stopped and an error message is produce indicating that a bigger
value for {\it nrmax} should be used.
This
integer depends directly on the maximum allowed separation {\it DSmax\/}
between
two contiguous points in the wavefront. For the examples shown on
Figure~\ref{fig:marm80-time} through \ref{fig:marm10-ampl}, the program ran 
with a value of {\it nrmax}=1300. {\it DSmax\/} was set to 21 meters
for those examples. 

In the case that the algorithm is run in a ``all arrivals'' mode,
which could be done by eliminating line {\bf 6} out of the algorithm, 
the number of crossing points on a wavefront could become considerably
large. As the wavefronts crosses and crosses many times over itself, for
a velocity model with strong variations (as for example the Marmousi 
model), the number of crossing points
can easily reach the
6 digits figure. This translates directly into a bigger need of computer
resources, in use of memory and time. 

Subroutine {\em initialize} defines the initial wavefront. It assigns
an initial amplitude and take-off angle to the points on the initial
wavefront.

The {\em propagate wavefront} subroutine ray traces using Lomax's
waverays. The waverays are traced starting at {\it cube\/}[ii].x0
with a take-off angle {\it cube\/}[ii].angle during one time step
at a certain frequency. The time step and the frequency are user
predefined.

Subroutine on line {\bf 6} checks for self crossed wavefronts and flags
the points that belong to the inner crossed section of the wavefront.

Line {\bf 7} checks for points that fall out of boundaries, raising
a flag. Notice
from Figure~\ref{fig:gauss-wv} that the rays cross over
the boundaries of the model. This is done in order to obtain arrivals
at the receivers that lie on the boundaries.

Subroutine at line {\bf 8} eliminates the points on a wavefront
that are flagged for laying out of bounds or belong
to self crossed wavefronts.

Subroutines on lines {\bf 9}, {\bf 10} and {\bf 11} are implemented 
as previously explained for the NORSAR method on Figures \ref{fig:norsar4},
\ref{fig:norsar3} and \ref{fig:norsar1} respectively. On line {\bf 11} the
number of rays that may be interpolated between any two contiguous
rays, is given by the number of times the distance between the two rays is
bigger than the maximum allowed distance {\it DSmax\/}.

Subroutine {\it gridding\/} is a very time consuming, due to
the irregular distribution of the data in the model. First, the
subroutine checks for receivers inside the ray cells of two contiguous
wavefronts. If a receiver is found, the ray parameters are interpolated
to it.

Subroutine {\it movwavf} prepares the structure {\it cube\/}
for a new wavefront to be propagated. It takes 
as the new starting point the previous arriving
point ({\it cube\/}[ii].x0 = {\it cube\/}[ii].x1).


\subsection{Travel-times and amplitudes in the Marmousi model}
\inputdir{marm}

\plot {marm80-time}{height=3.5in}{Traveltime contours
computed from the combined method, overlaid on a section of 
the Marmousi model.
A frequency of 80~Hz is used.
}

\plot {marm10-time}{height=3.5in}{As in 
the previous Figure, but for a frequency of 10 Hz.
}

\plot {marm80-ampl}{height=3.5in}{Amplitude maps from the
combined method at a frequency of 80~Hz.
}

\plot {marm10-ampl}{height=3.5in}{As in 
the previous Figure, but for a frequency of 10 Hz.
}

Figures \ref{fig:marm80-time} to \ref{fig:marm10-ampl} display 
the results of a simulation that used the combined 
method. The underlying subsurface structure is the 
Marmousi model \cite[]{Versteeg}.
A source was put at the surface, 5200 meters away from the left edge
of the model, and the wavefronts were
propagated until they crossed the boundaries of the model.
Figure \ref{fig:marm80-time}
shows the first-arrival traveltime contours calculated at a frequency
of 80~Hz. Figure \ref{fig:marm10-time} shows the same experiment at
a frequency of 10~Hz. Not much difference is apparent.

Figure~\ref{fig:marm80-ampl} shows the amplitude estimates for the 80~Hz
shot and Figure~\ref{fig:marm10-ampl} shows the 10~Hz estimates. We see
that more energy gets propagated down in the case of the low
frequency, illuminating part of the high frequency shadow zones.

We have seen that the combined method accomplishes two important tasks,
it can be used to compute first arrival traveltimes and amplitudes 
over any general velocity model and is it able to illuminate high
frequency shadow zones.

For these experiments, the mesh of the model is re-sampled
from the original model at 8 x 8 meters. The traveltime
and amplitude outputs are placed in a mesh of 25 x 12.5 meters.

\section{Conclusions}

I have presented a review of two ray tracing methods. I have 
implemented both of them in a combined version. The method 
computes first arrival travel-times and amplitudes of seismic 
waves in complex \mbox{2-D} velocity structures. 
The method uses a wavefront construction technique that produces
a complete coverage of the medium by a fairly
constant density of wavefronts and rays. Wavefronts are 
propagated using a wavelength-dependent smoothing ray tracing
technique, called the waveray method, which leads to an increased 
stability of the ray paths
relative to high frequency rays. Also, it gives a sensitivity to
the rays to larger velocity anomalies that lay within a fraction
of a wavelength of the ray path. The data (traveltimes and amplitudes)
is computed on an irregular grid. As the wavefronts are constructed 
the data is interpolated into a regular grid.

The result is a very robust ray tracing method that is able to 
illuminate areas of large geometrical spreading zones where
conventional ray tracing methods produce shadow zones. Portions
of the diffracted energy is produced in these shadow zones.

Further work should be done on calibrating and testing the results produced
by the combined method against other methods. Future work
should be done on a formal derivation of the waveray method.
Production of seismograms and a {\mbox 3-D} version are also
sources of future work.

%\newpage
\bibliographystyle{seg}
\bibliography{SEP2,SEG,paper}

%\newpage
