Is it possible there is some other way of separating the
degenerate isotropic shear mode into two well-defined orthonormal shear
wavetypes, one that avoids the artificial singularity the standard ``SV, SH''
definition creates for vertically traveling waves?
\label{Separ3-Question}
The answer is no, as we will discover in the next two sections
in the course of attempting to extend the two-dimensional
wavetype-separation algorithm
from section~\ref{Wavsep2-sec} to three dimensions.

\subsection{Extending 2D to 3D}

The two-dimensional algorithm in section~\ref{W2-algorithm}
seems to extend to three dimensions in a natural way.
As in two dimensions,
we use the solutions of the Christoffel equation
(already presented as equation~(\ref{TI-chris})
on page~\pageref{TI-chris})
to find
the propagation velocity and particle-motion direction
for each plane-wave component in the Fourier domain.
The problem is still a well behaved symmetric eigenvalue-eigenvector
problem; we are still always guaranteed three orthogonal wavetypes
for each $\mbox{\bf k}$.
The only difference should be that there are three solutions now instead
of two.
Following this recipe, we should quickly be able to sketch out
a three-dimensional wavefield-separation algorithm
modeled after the two-dimensional version previously presented
on page~\pageref{W2-algorithm}.

Remember back on page~\pageref{TI-OminousForeboding}
I warned that things would get interesting later?
This time around I will be a little more careful about how I describe
the wavefield-separation algorithm;
the whole point of this section is to show where it goes wrong
in three dimensions.

%\vbox{
First, construct the operator (I'll go over each step in more detail
later):
\begin{list}{}{}
\item[(1.)] Decide which mode this operator will pass.
\item[(2.)] For all $(k_x,k_y,k_z)$:
\item[] \{
\begin{list}{}{}
\item[(A.)] Substitute $(k_x, k_y, k_z)$ into equation~(\ref{TI-chris})
(on page~\pageref{TI-chris}) and find the eigenvector solutions
${\vv}_1$,
${\vv}_2$, and
${\vv}_3$.
\item[(B.)] Decide which solution corresponds to
the desired mode chosen in step (1). Call this solution $\vv$.
\item[(C.)]
Both ${\vv}$ and $-{\vv}$ are equally valid normalized eigenvectors.
Choose the one that is consistent with particle-motion
directions already determined at adjacent $(k_x,k_y,k_z)$ points.
\item[(D.)] Store the result for this $(k_x,k_y,k_z)$ in $\L(k_x,k_y,k_z)$.
\end{list}
\item[] \}
\end{list}
%}

%\vbox{
The vector field $\L$ is then used to calculate the scalar pure-mode
component $M$ of an input elastic wavefield $\U$ by the following:
\begin{list}{}{}
\item[(3.)] Fourier transform $\U(x,y,z;t)$ over $x$, $y$, and $z$,
obtaining $\hat \U(k_x,k_y,k_z;t)$.
\item[(4.)] For all $(k_x,k_y,k_z)$:
\item[] \{
$$ \hat M(k_x,k_y,k_z;t) = \L(k_x,k_y,k_z) \joecdot \hat \U(k_x,k_y,k_z;t) $$
\item[] \}
\item[(5.)] Inverse Fourier transform $\hat M(k_x,k_y,k_z;t)$
over $k_x$, $k_y$, and $k_z$, obtaining $M(x,y,z;t)$.
\end{list}
%}

We will collectively refer to this algorithm
as a ``wavetype-separation operator'':
this operator accepts as input
a vector field $\U$ and outputs a pure-mode scalar component $M$.

\subsection{Possible pitfalls}
Where can things go wrong?
The algorithm attempts to treat each plane-wave component
$(k_x, k_y, k_z)$ separately.
This is misleading.
The values of $\L$ chosen at each vector wavenumber $(k_x,k_y,k_z)$
do not exist in isolation, but must 
together describe a single global wavetype.
Furthermore,
the global wavetype described must agree with the way waves
actually propagate in the given medium.

Mathematically, given that the matrix of elastic constants
$\C$ really does correspond to the medium under consideration,
these requirements boil down to two kinds of Fourier-domain continuity:
First, the vector field $\L$
must be a continuous function of
plane-wave direction $(k_x,k_y,k_z)$; a true Fourier-domain discontinuity
in particle-motion direction would imply there existed an
infinite plane wave in the space domain.
This would violate causality, among other things.\footnote{
Later on in Figure~\protect\ref{Exam3-FrancisAxis} we will see how
the limited sort of Fourier-domain discontinuity that occurs at a singularity
can cause a {\em finite\/} plane wave that is actually physical.
}
Second,
the phase velocity $(\omega / k )$ must also be a
continuous function of plane-wave direction $(k_x,k_y,k_z)$,
since a discontinuous phase velocity would represent
a non-physical discontinuous wavefront.

Our naive algorithm can fail at several steps:
At step~(2A) there is no guarantee of three {\it unique} solutions.
Steps (1)~and~(2B) presuppose we know
what ``the global pure modes'' look like, and that they are
well-defined and continuous.
At step~(2C) there is no guarantee that it is possible
to choose a sign that is continuous with all the
previous adjacent choices.

In section~\ref{W2-example} I showed several
successful applications of the two-dimensional version of this
wavetype-separation algorithm.
No special care was taken to avoid the possible difficulties outlined
above; they simply didn't occur.
Is there any reason to expect the same algorithm that worked in
two dimensions should fail in three?

\subsection{Why two dimensions worked}
\label{Separ3-2DOK}
In two dimensions there are two in-plane wave modes.
(There is also a single uncoupled out of the plane mode,
if ``2 dimensions'' is construed to include particle
motion out of the plane. Here I will use ``2 dimensions'' in the
strict mathematical sense, in which case there is no such thing
as ``out of the plane''.)

Except for certain anomalous cases
(see section~\ref{TI-Anomalous} page~\pageref{TI-Anomalous},
or Helbig and Schoenberg
\referna{Helbig and Schoenberg}
{Anomalous polarization of elastic waves in transversely isotropic media}
{1987}),
the two in-plane modes are clearly analogous to the P and SV modes
in isotropic media.
If this is the case, there are no problems;
the {\qP} wave is faster than
the {\qSV} wave for all propagation directions.
Mathematically, since the two modes never have the same velocity
($\omega / k$),
they must have distinct eigenvalues
($\rho \omega^2$).
Distinct eigenvalues have unique orthogonal eigenvectors,
so there is no difficulty finding the particle-motion directions
in step~(2A).
The phase-velocity inequality also trivially solves the
problem of identifying which mode is which in step~(2B);
the faster one is {\qP}, the slower one is {\qS}.
Finally, in step~(2C), as long as the ``{\qP}'' mode does not have
pure S particle motion for any propagation direction,
defining the ``outward'' particle-motion direction
on the {\qP} mode to be positive will not lead to
an inconsistency.
The right hand rule then
gives a consistent sign convention for the {\qSV} mode.

\Tactiveplot{Separ3-Arrow1}{width=5.75in}{\figdir}
{Choosing signs for particle-motion direction for normal 2D media.}
{
Polar plots of particle-motion direction versus plane wave propagation
direction for ``normal'' two-dimensional media.
The shafts of the arrows indicate particle-motion direction.
Each shaft has an arrowhead on one end,
thus labeling that direction as the ``positive'' one.
This choice is arbitrary, but must be done so that the choice
at each arrow is consistent with the choice at its neighbors.
Left: A {\qP} mode.
Right: A {\qS} mode.
}

This is demonstrated graphically in Figure~\ref{Separ3-Arrow1}.
On the left
is a {\qP} mode with the ``outward'' direction everywhere chosen
to be ``positive''.
On the right is a {\qS} mode with the sign chosen
by a $90^\circ$ clockwise rotation from the {\qP} one.

What if the modes are anomalous, neither {\qP} nor {\qS}?
In section~\ref{TI-Behavior}, we saw that the fast and slow solutions
for transversely isotropic media in general never touched, except for
special values of {\cac} and {\cee} when they could just touch at isolated
points.
So the arguments from the previous paragraphs about
steps 1, 2A, and~2B
still go through (although
we might have to perturb {\cac} or {\cee} to resolve the point degeneracies
if we happen to have elastic constants corresponding to one of the
special cases).

\Tactiveplot{Separ3-Arrow2}{width=5.75in}{\figdir}
{Choosing signs for particle-motion direction for ``exotic'' 2D media.}
{
Polar plots of particle-motion direction versus plane-wave propagation
direction for exotic two-dimensional media.
Left: An ``anomalous'' mode, neither {\qP} nor {\qS}.
Right: An ``impossible'' mode. No consistent choice of sign
for the particle-motion direction is possible.
}

What about the choice of positive particle-motion direction in step~2C?
Figure~\ref{Separ3-Arrow2} demonstrates why this also always works.
On the left is a typical ``anomalous'' mode. The ``outward = positive''
method used in Figure~\protect\ref{Separ3-Arrow1} does not work for this
medium. However, if we simply work around the loop keeping consistent at
each step of the way, the loop still closes properly.
On the right is a mode where this method fails.
If we attempt to proceed as before,
we find the vector has performed a ``half flip''
and the loop will not close.
However, the example on the right is impossible:
The solutions to equation~(\protect\ref{TI-chris})
are unchanged when $\kk$ is replaced by $-\kk$, and so
slowness surfaces for physically realizable media must have
point symmetry about the origin.
For any physically possible medium,
only an even number of ``half flips'' can be performed in one loop
about the origin, and so the loop always closes
consistently.\footnote{
Later in this chapter in Figure~\protect\ref{Separ3-Blowup}
we encounter another kind of loop
where half-flips are the usual case.
}
This ``impossible'' slowness surface lacks
this symmetry.

\Tactiveplot{Separ3-Wavedef}{height=7.55in}{\figdir}
{Typical slowness surfaces for normal and anomalous TI media.}
{
Slowness surfaces for two different transversely isotropic (TI) media.
On the left is ``Greenhorn Shale''
\refer{Jones and Wang}
{Ultrasonic velocities in Cretaceous shales from the Williston basin}
{1981};
this medium displays the standard TI ``\{{\qP},{\qSV},SH\}'' modes.
On the right is an ``anomalous'' transversely isotropic media.
There is still an ``SH'' mode, but the other two modes (both labeled
``A'' for Anomalous) are
{\qP}-like for some propagation directions and {\qSV}-like for others.
\index{wave modes | general}
\index{wave modes | transversely isotropic}
\index{slowness surface | 3D example}
}

\subsection{{\qP} works in 3D}
In two dimensions we found we could always produce two global pure modes;
in three dimensions we intuitively expect there should be three.
The good news is that three-dimensional {\qP} modes are just as
well behaved as two-dimensional ones.

Figure~\ref{Problem-Slowsurf} shows the slowness surfaces
for the familiar isotropic P, SV, and SH wave modes.
Note that as in the two-dimensional isotropic case, the P mode is
faster than the shear modes for all propagation directions.
For media not too far removed from isotropy, there
still ought to be a distinct
``{\qP}'' mode that never intersects the others, and
so as in two dimensions there should be no trouble with the
{\qP} mode at steps (2A) or~(2B).

What about the choice of sign on the particle-motion direction?
First I must precisely define ``{\qP} mode''.
\index{definition | {\qP} mode}
\index{wave modes | general}
A global wave mode is a {\qP} mode if the particle-motion direction
$\L(k_x,k_y,k_z)$ is {\em not perpendicular\/}
to the plane-wave propagation direction
$\kk = (k_x,k_y,k_z)$ {\em for any\/} $(k_x,k_y,k_z) \neq \veczero$.
\index{definition | {\qS} mode}
Similarly,
a global wave mode is a {\qS} mode if the particle-motion direction
$\L(k_x,k_y,k_z)$ is {\em not parallel\/} to the
plane-wave propagation direction
$\kk = (k_x,k_y,k_z)$ {\em for any\/} $(k_x,k_y,k_z) \neq \veczero$.
\index{definition | Anomalous mode}
Modes that are neither {\qP} nor {\qS} are called ``anomalous''.
See Figure~\ref{Separ3-Wavedef} for examples of ``{\qP}'',
``{\qS}'', and ``anomalous'' modes.
(Note my definitions define the various mode types by what they are
{\em not\/}: {\qP} modes are nowhere pure S; {\qS} modes are nowhere pure P;
anomalous modes are neither {\qP} nor {\qS} modes.)

\Tactivesideplot{Separ3-Steve}{height=3.08in}{\figdir}
{Modes with no consistent sign choice possible aren't {\qP} modes.}
{
A slice out of a 3-D slowness surface for an anomalously polarized mode.
Each particle-motion direction vector (the ``sticks''
projecting through the surface) has one end drawn thick to indicate
that direction has been chosen as ``positive''.
At the top and bottom the mode is {\qP}-like, but the ``positive''
direction points away from the origin at the top and towards the origin
at the bottom.
}

For trouble to occur at step~(2C) of our algorithm
there would have to be some closed path on the slowness surface
along which the signed particle-motion direction vector performed
a $180^\circ$ rotation.
However,
the defining property of a {\qP} mode is that
it does not have pure S-direction particle motion anywhere.
Given a {\qP} mode, then, such a flip in direction cannot occur:
somewhere halfway through the P-direction component
of the particle motion would have to pass through zero as it changed
sign. See Figure~\ref{Separ3-Steve} for a graphical demonstration.
(Remember that the particle-motion direction vector $\vv$
always has unit magnitude.)
Thus there should also not be any difficulty at step~(2C) for
{\qP} modes.

We conclude that three-dimensional {\qP} modes present
no new problems over the two-dimensional case.
(Of course there is no guarantee that a {\qP} mode exists;
Figure~\ref{Exam3-FrancisImp} shows an example of a medium for
which all three wave modes are anomalous, neither {\qP} nor {\qS}.)

\subsubsection{3D {\qP} mode-separation example}
Figure~\ref{Separ3-Pfig} shows a three-dimensional {\qP}
mode-separation example. The algorithm works as advertised,
even though the medium is orthorhombic.

\Tactiveplot{Separ3-Pfig}{width=5.75in}{\figdir}
{Wavetype separation works for 3D {\qP} modes.}
{A three-dimensional {\qP} mode-separation example.
The medium is orthorhombic, ``Cracked Greenhorn Shale''
(see Table~\ref{Const-3D}).
The $x$-$z$ symmetry plane containing
the source is shown. (As we might expect from the discussion at the
beginning of this chapter, the symmetry plane slices don't show much
evidence of the cracks.)
The source is a $z$ point force; the $z$
component of displacement is shown.
The model is periodic.
Wavetype separation works for 3D {\qP} modes.
(There is no problem calculating a scalar mode for this example.
It was just convenient to use the same vector algorithm described on
page~\pageref{Separ3-ModAlg} for all the examples in this chapter.)
\index{wavetype separation | example}
}
