\title{Anisotropy in 3 dimensions}
\author{Joseph A. Dellinger}

\begin{quotation}
{\sf
When we try to pick out anything by itself,
we find it hitched to everything else in the universe.
{\small -- John Muir}
}
\end{quotation}
\vspace{.7in}

In the previous chapter I showed how waves in anisotropic
two-dimensional media can display such phenomena as
large amplitude variations,
cusps,
and anomalous polarizations.
Although such things are certainly possible,
we also know that geophysics got along pretty well assuming P
waves were isotropic or at worst elliptically anisotropic for a long time.
So we can hope that the sorts of media we are likely to
encounter in the real Earth
are usually not too far removed from isotropy.
From the examples in the previous chapter,
we might then feel justified in expecting
that we can usually count on having waves that
at least approximately fit the standard isotropic P, SV, and SH
designations.

In this chapter we show that while the two-dimensional story is
complete as far as it goes,
it is not a good preview of what happens in three dimensions.

\section{The problem with only looking at symmetry planes}
We begin with a simple example of how two-dimensional intuition can
be misleading.
Refer to Figure~\ref{Problem-Anisotrick1},
which shows three symmetry-plane slices through a simple
orthorhombic anisotropic slowness surface.
Each slice shows what seem to be pure
P, SV, and SH modes; furthermore, on each slice the P mode and one
of the S modes are circular, while the other S mode is
elliptical. Does this mean this medium supports pure P,
SV, and SH modes?

\inputdir{.}
\plot{problem0}{width=5.75in}{
Symmetry-plane slices through the three-dimensional slowness surfaces
of a simple orthorhombic medium.
The fat bars and dots show particle-motion direction. (Dots represent
motion in and out of the plane.)
Following the numbers from $1$ through
$3$ seems to show the shear surfaces have a mobius-like topology.
To see what's really happening,
look at Figure~\protect\ref{Problem-Anisotrick2}.
\index{Symmetry plane | slices}
\index{slowness surface | 3D slice example}
}

No, as we can see by attempting to follow one shear surface around
through all three plots. Start at ``1'' on the left plot where
the ``SV surface'' intersects the $(k_x / \omega)$ axis. Trace from
there up to ``2''. Now jump to ``2'' on the center plot (remembering
that in three dimensions both ``2''s mark the same point).
Continuing on in the same way, we get to ``3'' and jump to the right plot.
Here there is a surprise:
we are now on the ``SH surface'', and the ``SH surface'' appears not
to reconnect back to ``1'' where we started.
What is going on?

Figure~\ref{Problem-Anisotrick2} shows a more accurate three-dimensional view.
(\cite{musgrave1981elastodynamic}
shows similar pictures for several physical media.)
At ``3'', the outer surface
is ``SV'' in the $k_y$--$k_z$ plane, but ``SH'' in the $k_x$--$k_z$
plane.
This is possible because the designations
``SH'' and ``SV'' are relative concepts that
depend on the orientation of the slice under consideration.
Is the labeled ``crossing'' really a point where a ``{\qSH}'' and
``{\qSV}'' surface intersect? No, as we can see by
closely examining the slices cut
at $10^\circ$ and $45^\circ$ angles to the $k_x$--$k_z$ plane.
Clearly trying to shoehorn shear modes like these into ``{\qSV} and {\qSH}''
designations is not going to work in three dimensions.
(\cite{crampin1981review} also remarks on many of these points.)

\plot{problem1}{width=5.75in}
{
Slices through the three-dimensional slowness surfaces
of a simple orthorhombic medium.
The fat bars show particle-motion direction.
The ``crossing'' is not what it appeared to be in
Figure~\protect\ref{Problem-Anisotrick1}.
For another view of the same surface,
look ahead to Figure~\protect\ref{Separ3-Mobius}.
\index{Symmetry plane | slices}
\index{slowness surface | 3D slice example}
\index{wave modes | orthorhombic}
}

\Tactiveplot{Problem-Slowsurf}{width=5.75in}{\figdir}
{The canonical isotropic ``P, SV, and SH'' slowness surfaces in 3D.}
{
Three-dimensional slowness surfaces for the standard \{P,SV,SH\}
isotropic wave modes. Slowness surfaces are plots of phase slowness
versus plane-wave propagation direction; they can also be considered as
dispersion relation plots.
We show particle-motion direction by the ``sticks'' attached to
the surfaces.
All three surfaces could not be plotted together because the two
shear surfaces are everywhere coincident.
\index{wave modes | isotropic P SV SH}
\index{slowness surface | 3D example}
}

Perhaps this example is atypically perverse.
Do the standard isotropic shear modes harbor any similar misbehaviors?
In Figure~\ref{Problem-Slowsurf},
note that by defining the shear surfaces as SH and SV
we have introduced a nonphysical particle-motion direction
discontinuity for vertically propagating waves.
This is a consequence of the basic ``in-plane'' and ``cross-plane''
definitions of SV and SH;
the same vertically traveling wave with North-South particle motion
is called ``SH'' on an East-West section
but ``SV'' on a North-South one.


\section{Wavetype separation in three dimensions}\index{wavetype separation | 3D}
\label{Separ3-Sec}
Is it possible there is some other way of separating the
degenerate isotropic shear mode into two well-defined orthonormal shear
wavetypes, one that avoids the artificial singularity the standard ``SV, SH''
definition creates for vertically traveling waves?
\label{Separ3-Question}
The answer is no, as we will discover in the next two sections
in the course of attempting to extend the two-dimensional
wavetype-separation algorithm
from section~\ref{Wavsep2-sec} to three dimensions.

\subsection{Extending 2D to 3D}

The two-dimensional algorithm in section~\ref{W2-algorithm}
seems to extend to three dimensions in a natural way.
As in two dimensions,
we use the solutions of the Christoffel equation
(already presented as equation~(\ref{TI-chris})
on page~\pageref{TI-chris})
to find
the propagation velocity and particle-motion direction
for each plane-wave component in the Fourier domain.
The problem is still a well behaved symmetric eigenvalue-eigenvector
problem; we are still always guaranteed three orthogonal wavetypes
for each $\mbox{\bf k}$.
The only difference should be that there are three solutions now instead
of two.
Following this recipe, we should quickly be able to sketch out
a three-dimensional wavefield-separation algorithm
modeled after the two-dimensional version previously presented
on page~\pageref{W2-algorithm}.

Remember back on page~\pageref{TI-OminousForeboding}
I warned that things would get interesting later?
This time around I will be a little more careful about how I describe
the wavefield-separation algorithm;
the whole point of this section is to show where it goes wrong
in three dimensions.

%\vbox{
First, construct the operator (I'll go over each step in more detail
later):
\begin{list}{}{}
\item[(1.)] Decide which mode this operator will pass.
\item[(2.)] For all $(k_x,k_y,k_z)$:
\item[] \{
\begin{list}{}{}
\item[(A.)] Substitute $(k_x, k_y, k_z)$ into equation~(\ref{TI-chris})
(on page~\pageref{TI-chris}) and find the eigenvector solutions
${\vv}_1$,
${\vv}_2$, and
${\vv}_3$.
\item[(B.)] Decide which solution corresponds to
the desired mode chosen in step (1). Call this solution $\vv$.
\item[(C.)]
Both ${\vv}$ and $-{\vv}$ are equally valid normalized eigenvectors.
Choose the one that is consistent with particle-motion
directions already determined at adjacent $(k_x,k_y,k_z)$ points.
\item[(D.)] Store the result for this $(k_x,k_y,k_z)$ in $\L(k_x,k_y,k_z)$.
\end{list}
\item[] \}
\end{list}
%}

%\vbox{
The vector field $\L$ is then used to calculate the scalar pure-mode
component $M$ of an input elastic wavefield $\U$ by the following:
\begin{list}{}{}
\item[(3.)] Fourier transform $\U(x,y,z;t)$ over $x$, $y$, and $z$,
obtaining $\hat \U(k_x,k_y,k_z;t)$.
\item[(4.)] For all $(k_x,k_y,k_z)$:
\item[] \{
$$ \hat M(k_x,k_y,k_z;t) = \L(k_x,k_y,k_z) \joecdot \hat \U(k_x,k_y,k_z;t) $$
\item[] \}
\item[(5.)] Inverse Fourier transform $\hat M(k_x,k_y,k_z;t)$
over $k_x$, $k_y$, and $k_z$, obtaining $M(x,y,z;t)$.
\end{list}
%}

We will collectively refer to this algorithm
as a ``wavetype-separation operator'':
this operator accepts as input
a vector field $\U$ and outputs a pure-mode scalar component $M$.

\subsection{Possible pitfalls}
Where can things go wrong?
The algorithm attempts to treat each plane-wave component
$(k_x, k_y, k_z)$ separately.
This is misleading.
The values of $\L$ chosen at each vector wavenumber $(k_x,k_y,k_z)$
do not exist in isolation, but must 
together describe a single global wavetype.
Furthermore,
the global wavetype described must agree with the way waves
actually propagate in the given medium.

Mathematically, given that the matrix of elastic constants
$\C$ really does correspond to the medium under consideration,
these requirements boil down to two kinds of Fourier-domain continuity:
First, the vector field $\L$
must be a continuous function of
plane-wave direction $(k_x,k_y,k_z)$; a true Fourier-domain discontinuity
in particle-motion direction would imply there existed an
infinite plane wave in the space domain.
This would violate causality, among other things.\footnote{
Later on in Figure~\protect\ref{Exam3-FrancisAxis} we will see how
the limited sort of Fourier-domain discontinuity that occurs at a singularity
can cause a {\em finite\/} plane wave that is actually physical.
}
Second,
the phase velocity $(\omega / k )$ must also be a
continuous function of plane-wave direction $(k_x,k_y,k_z)$,
since a discontinuous phase velocity would represent
a non-physical discontinuous wavefront.

Our naive algorithm can fail at several steps:
At step~(2A) there is no guarantee of three {\it unique} solutions.
Steps (1)~and~(2B) presuppose we know
what ``the global pure modes'' look like, and that they are
well-defined and continuous.
At step~(2C) there is no guarantee that it is possible
to choose a sign that is continuous with all the
previous adjacent choices.

In section~\ref{W2-example} I showed several
successful applications of the two-dimensional version of this
wavetype-separation algorithm.
No special care was taken to avoid the possible difficulties outlined
above; they simply didn't occur.
Is there any reason to expect the same algorithm that worked in
two dimensions should fail in three?

\subsection{Why two dimensions worked}
\label{Separ3-2DOK}
In two dimensions there are two in-plane wave modes.
(There is also a single uncoupled out of the plane mode,
if ``2 dimensions'' is construed to include particle
motion out of the plane. Here I will use ``2 dimensions'' in the
strict mathematical sense, in which case there is no such thing
as ``out of the plane''.)

Except for certain anomalous cases
(see section~\ref{TI-Anomalous} page~\pageref{TI-Anomalous},
or Helbig and Schoenberg
\referna{Helbig and Schoenberg}
{Anomalous polarization of elastic waves in transversely isotropic media}
{1987}),
the two in-plane modes are clearly analogous to the P and SV modes
in isotropic media.
If this is the case, there are no problems;
the {\qP} wave is faster than
the {\qSV} wave for all propagation directions.
Mathematically, since the two modes never have the same velocity
($\omega / k$),
they must have distinct eigenvalues
($\rho \omega^2$).
Distinct eigenvalues have unique orthogonal eigenvectors,
so there is no difficulty finding the particle-motion directions
in step~(2A).
The phase-velocity inequality also trivially solves the
problem of identifying which mode is which in step~(2B);
the faster one is {\qP}, the slower one is {\qS}.
Finally, in step~(2C), as long as the ``{\qP}'' mode does not have
pure S particle motion for any propagation direction,
defining the ``outward'' particle-motion direction
on the {\qP} mode to be positive will not lead to
an inconsistency.
The right hand rule then
gives a consistent sign convention for the {\qSV} mode.

\Tactiveplot{Separ3-Arrow1}{width=5.75in}{\figdir}
{Choosing signs for particle-motion direction for normal 2D media.}
{
Polar plots of particle-motion direction versus plane wave propagation
direction for ``normal'' two-dimensional media.
The shafts of the arrows indicate particle-motion direction.
Each shaft has an arrowhead on one end,
thus labeling that direction as the ``positive'' one.
This choice is arbitrary, but must be done so that the choice
at each arrow is consistent with the choice at its neighbors.
Left: A {\qP} mode.
Right: A {\qS} mode.
}

This is demonstrated graphically in Figure~\ref{Separ3-Arrow1}.
On the left
is a {\qP} mode with the ``outward'' direction everywhere chosen
to be ``positive''.
On the right is a {\qS} mode with the sign chosen
by a $90^\circ$ clockwise rotation from the {\qP} one.

What if the modes are anomalous, neither {\qP} nor {\qS}?
In section~\ref{TI-Behavior}, we saw that the fast and slow solutions
for transversely isotropic media in general never touched, except for
special values of {\cac} and {\cee} when they could just touch at isolated
points.
So the arguments from the previous paragraphs about
steps 1, 2A, and~2B
still go through (although
we might have to perturb {\cac} or {\cee} to resolve the point degeneracies
if we happen to have elastic constants corresponding to one of the
special cases).

\Tactiveplot{Separ3-Arrow2}{width=5.75in}{\figdir}
{Choosing signs for particle-motion direction for ``exotic'' 2D media.}
{
Polar plots of particle-motion direction versus plane-wave propagation
direction for exotic two-dimensional media.
Left: An ``anomalous'' mode, neither {\qP} nor {\qS}.
Right: An ``impossible'' mode. No consistent choice of sign
for the particle-motion direction is possible.
}

What about the choice of positive particle-motion direction in step~2C?
Figure~\ref{Separ3-Arrow2} demonstrates why this also always works.
On the left is a typical ``anomalous'' mode. The ``outward = positive''
method used in Figure~\protect\ref{Separ3-Arrow1} does not work for this
medium. However, if we simply work around the loop keeping consistent at
each step of the way, the loop still closes properly.
On the right is a mode where this method fails.
If we attempt to proceed as before,
we find the vector has performed a ``half flip''
and the loop will not close.
However, the example on the right is impossible:
The solutions to equation~(\protect\ref{TI-chris})
are unchanged when $\kk$ is replaced by $-\kk$, and so
slowness surfaces for physically realizable media must have
point symmetry about the origin.
For any physically possible medium,
only an even number of ``half flips'' can be performed in one loop
about the origin, and so the loop always closes
consistently.\footnote{
Later in this chapter in Figure~\protect\ref{Separ3-Blowup}
we encounter another kind of loop
where half-flips are the usual case.
}
This ``impossible'' slowness surface lacks
this symmetry.

\Tactiveplot{Separ3-Wavedef}{height=7.55in}{\figdir}
{Typical slowness surfaces for normal and anomalous TI media.}
{
Slowness surfaces for two different transversely isotropic (TI) media.
On the left is ``Greenhorn Shale''
\refer{Jones and Wang}
{Ultrasonic velocities in Cretaceous shales from the Williston basin}
{1981};
this medium displays the standard TI ``\{{\qP},{\qSV},SH\}'' modes.
On the right is an ``anomalous'' transversely isotropic media.
There is still an ``SH'' mode, but the other two modes (both labeled
``A'' for Anomalous) are
{\qP}-like for some propagation directions and {\qSV}-like for others.
\index{wave modes | general}
\index{wave modes | transversely isotropic}
\index{slowness surface | 3D example}
}

\subsection{{\qP} works in 3D}
In two dimensions we found we could always produce two global pure modes;
in three dimensions we intuitively expect there should be three.
The good news is that three-dimensional {\qP} modes are just as
well behaved as two-dimensional ones.

Figure~\ref{Problem-Slowsurf} shows the slowness surfaces
for the familiar isotropic P, SV, and SH wave modes.
Note that as in the two-dimensional isotropic case, the P mode is
faster than the shear modes for all propagation directions.
For media not too far removed from isotropy, there
still ought to be a distinct
``{\qP}'' mode that never intersects the others, and
so as in two dimensions there should be no trouble with the
{\qP} mode at steps (2A) or~(2B).

What about the choice of sign on the particle-motion direction?
First I must precisely define ``{\qP} mode''.
\index{definition | {\qP} mode}
\index{wave modes | general}
A global wave mode is a {\qP} mode if the particle-motion direction
$\L(k_x,k_y,k_z)$ is {\em not perpendicular\/}
to the plane-wave propagation direction
$\kk = (k_x,k_y,k_z)$ {\em for any\/} $(k_x,k_y,k_z) \neq \veczero$.
\index{definition | {\qS} mode}
Similarly,
a global wave mode is a {\qS} mode if the particle-motion direction
$\L(k_x,k_y,k_z)$ is {\em not parallel\/} to the
plane-wave propagation direction
$\kk = (k_x,k_y,k_z)$ {\em for any\/} $(k_x,k_y,k_z) \neq \veczero$.
\index{definition | Anomalous mode}
Modes that are neither {\qP} nor {\qS} are called ``anomalous''.
See Figure~\ref{Separ3-Wavedef} for examples of ``{\qP}'',
``{\qS}'', and ``anomalous'' modes.
(Note my definitions define the various mode types by what they are
{\em not\/}: {\qP} modes are nowhere pure S; {\qS} modes are nowhere pure P;
anomalous modes are neither {\qP} nor {\qS} modes.)

\Tactivesideplot{Separ3-Steve}{height=3.08in}{\figdir}
{Modes with no consistent sign choice possible aren't {\qP} modes.}
{
A slice out of a 3-D slowness surface for an anomalously polarized mode.
Each particle-motion direction vector (the ``sticks''
projecting through the surface) has one end drawn thick to indicate
that direction has been chosen as ``positive''.
At the top and bottom the mode is {\qP}-like, but the ``positive''
direction points away from the origin at the top and towards the origin
at the bottom.
}

For trouble to occur at step~(2C) of our algorithm
there would have to be some closed path on the slowness surface
along which the signed particle-motion direction vector performed
a $180^\circ$ rotation.
However,
the defining property of a {\qP} mode is that
it does not have pure S-direction particle motion anywhere.
Given a {\qP} mode, then, such a flip in direction cannot occur:
somewhere halfway through the P-direction component
of the particle motion would have to pass through zero as it changed
sign. See Figure~\ref{Separ3-Steve} for a graphical demonstration.
(Remember that the particle-motion direction vector $\vv$
always has unit magnitude.)
Thus there should also not be any difficulty at step~(2C) for
{\qP} modes.

We conclude that three-dimensional {\qP} modes present
no new problems over the two-dimensional case.
(Of course there is no guarantee that a {\qP} mode exists;
Figure~\ref{Exam3-FrancisImp} shows an example of a medium for
which all three wave modes are anomalous, neither {\qP} nor {\qS}.)

\subsubsection{3D {\qP} mode-separation example}
Figure~\ref{Separ3-Pfig} shows a three-dimensional {\qP}
mode-separation example. The algorithm works as advertised,
even though the medium is orthorhombic.

\Tactiveplot{Separ3-Pfig}{width=5.75in}{\figdir}
{Wavetype separation works for 3D {\qP} modes.}
{A three-dimensional {\qP} mode-separation example.
The medium is orthorhombic, ``Cracked Greenhorn Shale''
(see Table~\ref{Const-3D}).
The $x$-$z$ symmetry plane containing
the source is shown. (As we might expect from the discussion at the
beginning of this chapter, the symmetry plane slices don't show much
evidence of the cracks.)
The source is a $z$ point force; the $z$
component of displacement is shown.
The model is periodic.
Wavetype separation works for 3D {\qP} modes.
(There is no problem calculating a scalar mode for this example.
It was just convenient to use the same vector algorithm described on
page~\pageref{Separ3-ModAlg} for all the examples in this chapter.)
\index{wavetype separation | example}
}


\section{Shear Singularities}
In the previous sections I have shown why mode separation
in two dimensions should always work,
and that mode separation in three dimensions should
always work for {\qP} modes.
Is there any reason to expect mode separation in three dimensions
to fail for {\qS} modes?

\subsection{Defining ``the'' isotropic shear modes}

In isotropic media, it is customary to consider the degenerate
shear wave as a single vector pure mode.
We typically resolve this globally degenerate mode into
``SV'' and ``SH'' scalar modes, but this choice is arbitrary.
Isotropy, with its infinite symmetry,
lies at the intersection of all the various anisotropic
symmetry systems.
Perturb the elastic constants away from isotropy
and the perfect global symmetry is broken; the shear
waves become nondegenerate.

The standard ``\{SV,SH\}'' labels result
if we perturb in the direction of transverse isotropy
with a vertical symmetry axis.
There are many other possibilities,
resulting in other equally valid
definitions of ``the isotropic shear modes''.
If the SV and SH modes as shown in Figure~\ref{Problem-Slowsurf}
are arbitrary, why are they so widely used?
SV and SH are popular because they are easy to
understand in terms of vertical two-dimensional slices,
which is how we usually look at the Earth.
Figure~\ref{Separ3-Funnyshear}
shows an example of the bizarre sorts of modes that usually
result if the elastic constants are perturbed randomly.

\Tactiveplot{Separ3-Funnyshear}{width=5.75in}{\figdir}
{Shear modes resulting from a random perturbation to isotropy.}
{
Shear slowness surfaces for arbitrary isotropic shear modes.
Mathematically these modes are just as valid a decomposition
of the degenerate isotropic shear modes as the
standard method shown in Figure~\protect\ref{Problem-Slowsurf}.
One singularity is labeled, although several are visible.
(The top plot is the slow ({\qS2}) split shear wave;
Figure~\protect\ref{Sing-Funny} shows another view.)
\index{shear singularities | example}
\index{slowness surface | 3D example}
}

Previously we saw that the standard SV and SH modes shown in
Figure~\ref{Problem-Slowsurf} have an artificial
discontinuity for vertically traveling waves.
Such a point on the slowness surface where
the particle-motion direction becomes a discontinuous
function of phase direction is called a ``singularity''.
\index{definition | singularity}
\index{particle motion | singularities in}
They are usually associated with S or {\qS} waves, and so are
often called ``shear singularities''.
\index{shear singularities}
The arbitrary modes shown in Figure~\ref{Separ3-Funnyshear}
likewise display shear singularities in the particle-motion direction
field for waves traveling in several apparently random directions.
Singularities are annoying, because they cause discontinuities
in the vector field $\L$ at the core of our wavetype-separation algorithm.

\subsection{Furry ball theorem}

Finally we are back to our original question
from page~\pageref{Separ3-Question}:
Are singularities avoidable?
Pure S-type motion requires that the particle-motion direction
be perpendicular to the direction of plane-wave propagation.
In Figure~\ref{Problem-Slowsurf}, the particle-motion direction
is represented by ``hairs'' attached to the slowness surface.
For a pure shear wave, each hair must lie flat against
the ball's surface.
Imagine what happens if you try to comb the hairs
on a furry ball flat and regular everywhere.
No matter how you proceed, there must always be
a ``whorl'' or ``cowlick'' somewhere,
where adjacent hairs point in different directions.\footnote{
This is a special case of a well known result from
differentiable manifold theory:
no even-dimensional sphere has any continuous nonvanishing field of
tangent vectors
\refer{Boothby}
{An introduction to differentiable manifolds and Riemannian geometry}
{1975}.
}
We now have the answer: these discontinuous points
are shear singularities, and they cannot be eradicated.
There is no way to define two globally continuous
isotropic shear modes.

Shear-wave singularities must always occur for pure S modes,
but what about {\qS} modes?
Simply apply the previous argument for pure S modes
to the pure S component
of the particle-motion direction at each point (i.e., 
the component of the particle motion perpendicular
to the plane-wave propagation direction).
The only way to escape the singularity is by letting the
pure S component of the particle-motion direction go to zero
where we would have had trouble
(i.e. the ``{\qS} mode'' has a pure P-mode direction there:
the ``hair'' at the center of the whorl sticks straight up).
Since the defining property of a {\qS} mode is that
it does not have pure P-direction particle motion anywhere,
this cannot happen.

\subsection{What do shear singularities represent?}
We know that physical
wavefronts do not support discontinuous particle-motion directions.
How can we reconcile this observation with the ubiquity of singularities?

For any phase direction there are three {\em orthogonal} modes,
so it is impossible for one wavetype to have a discontinuous
particle-motion direction in isolation;
at least one of the other wavetypes
must share in the discontinuous particle-motion directions
at the singular point (although rotated by $90^\circ$).
Precisely at the singular point the particle-motion directions
of these two modes must become indeterminate.
This can only happen if the two modes are degenerate there.

A singularity, then, must correspond to a point where two
otherwise continuous and distinct modes touch and temporarily
lose their unique identities
\refer{Crampin and Yedlin}
{Shear-wave singularities of wave propagation in anisotropic media}
{1981}.
\label{Separ3-Touch}
We can see this illustrated in Figure~\ref{Separ3-Wavedef};
the singularity for vertically traveling S waves (along the $k_z$ axis)
always occurs on two intersecting surfaces. On the axis particle
motions in the $x$ and $y$ directions are symmetrically equivalent.
Figure~\ref{Separ3-Mobius} shows the nature of singular points
even more clearly.
In this orthorhombic example the {\qS} modes are clearly
two separate shells that pinch together at a finite number
of singular points.

\Tactiveplot{Separ3-Mobius}{width=5.75in}{\figdir}
{A canonical orthorhombic slowness surface.}
{
The three-dimensional slowness surfaces
of the same orthorhombic medium shown
in Figure~\protect\ref{Problem-Anisotrick2}.
One octant has been removed to show the true topology:
two nested {\qS} surfaces that just touch at the singularities.
(There are four singularities in all; the other three are symmetrically
equivalent to the labeled one.)
Neither of the surfaces can be adequately
described as ``{\qSV}'' or ``{\qSH}''.
\index{shear singularities | example}
\index{slowness surface | 3D example}
}

A shear singularity does not cause a discontinuous particle motion
in reality because for some range of angles around the
singular phase direction two orthogonal wavetypes
are strongly coupled together.
While individually their particle-motion vectors rapidly twist around,
for any finite frequency their sum remains smooth.
(Singularities can produce other visible effects, however, as
will be demonstrated in section~\ref{Exam3-Chimera}.)

\subsection{{\qS} modes are inseparable}
I have now shown that any {\qS} mode has singularities on it,
and that singularities are places where two modes touch.
If there are two {\qS} modes that are both everywhere slower
than a single {\qP} mode, then the two {\qS} modes must be
touching each other.
Does this automatically mean mode separation of
{\qS} waves is impossible? How does trouble occur?

At step~(2A),
we must expect occasional trouble with nonunique {\qS} solutions
in three dimensions,
because shear-wave singularities represent
degenerate directions.
A typical way around such problems in ray tracing is to perturb
the elastic constants so that the singularity moves off
our current ray direction
\refer{Jech and P\v{s}en\v{c}\'{\i}k}
{First order perturbation method for anisotropic media}
{1989}.
This procedure shouldn't create significant errors since
a tiny perturbation in the elastic constants should only cause a
tiny change in the global wavefield.
(We are still left with the problem of figuring out which
of the barely split modes is the one we want.)

Steps (1)~and~(2B) of our algorithm presuppose we know what the
global {\qS} modes look like. (I haven't even yet proven that there
is such a thing as a global {\qS} mode for general anisotropy!)
The isotropic modes are usually separated into
\{P,SV,SH\}, and the transversely isotropic modes into
\{\qP,\qSV,SH\}, but we have seen these designations are arbitrary
at best.  Is there any sensible way to
separate the connecting {\qS} surfaces into two {\qS} pure modes
for completely general anisotropic media?

The example in Figure~\ref{Separ3-Mobius} suggests one possibility.
We wish to cut the two-sheeted {\qS} surface in this figure
into two pure modes.
We cannot do this arbitrarily.
\index{definition | pure mode}
\index{wave modes | definition}
\index{wave modes | general}
Step~(2B) of our algorithm requires that the three
plane wavetypes at each $(k_x,k_y,k_z)$,
$ \{ {\vv}_1, {\vv}_2, {\vv}_3 \} $,
must be identified as a permutation of
the three global wave modes
(usually \{{\qP},{\qS1},{\qS2}\})
evaluated for that phase direction.
The three resulting pure modes must be
unique single-valued {\em continuous} functions of phase velocity versus
plane-wave propagation direction.
(In fact I would say this must be the definition of a global pure mode.)
As should be clear from Figure~\ref{Separ3-Mobius},
the only possible place to cut is at the shear singularities.

This definition makes the modes very easy to identify
computationally; the modes are simply sorted on their phase velocity
% Reference from chapter 2 where we talk about ordering wavetypes
% by phase velocities.
\label{Separ3-speedorder}
\index{wave modes | ordering}
$(\omega / k )$ into an inner (fastest) {\qP},
\index{definition | {\qS1} and {\qS2} modes}
a middle {\qS1}, and an outer (slowest) {\qS2} mode
\refer{Crampin}
{A review of wave motion in anisotropic and cracked elastic media}
{1981}.
(Of course while this method is the most general, it still may make
sense to use notations such as \{\qP,\qSV,SH\} when appropriate ---
although these symmetric notations are misleading, as will be shown
in section~\ref{Exam3-Chimera}.)

Unfortunately, as can be seen in Figure~\ref{Separ3-Blowup},
sorting the modes by phase velocity
does nothing to ameliorate the particle-motion discontinuity
at the shear singularities. As I will demonstrate in the next section,
this local discontinuity in the Fourier
domain produces a large planar artifact in the space-domain version
of the corresponding wavetype-separation operator, although
it can be suppressed in certain special cases.

We have seen there are troublesome problems at step~(2B).
What about step~(2C)? Figure~\ref{Separ3-Blowup} shows most of the particle-motion
direction vectors as two-sided, since mathematically there is nothing
in equation~(\ref{TI-chris}) to distinguish any eigenvector $\vv$
from its opposite $-\vv$. I have attempted to pick one sign
as ``positive'' for each particle-motion vector along a path
(shown as a dotted line) around the singularity;
these vectors are shown as one-sided rays.
Unfortunately if we pick the vectors to preserve
continuity as we move along the path, we find the signs don't match
when we complete the loop. There is an inescapable branch cut in
step~(2C).\footnote{This is an example of a phenomenon of much
current interest in quantum physics, anholonomy
\refer{Berry}
{Anticipations of the geometric phase}
{1990}.
\index{anholonomy}
The basic idea is that after some parameters of a system have been
perturbed around a closed loop in a continuous way, some other
seemingly unrelated parameters are found to have changed sign or
otherwise shifted in phase.}

\Tactiveplot{Separ3-Blowup}{width=5.75in}{\figdir}
{A close-up view of a slowness surface showing a canonical shear singularity.}
{
A close-up view of the outer surface in Figure~\protect\ref{Separ3-Mobius},
centered on the shear singularity (here indicated by an ``{\bf $S$}'').
Note that the particle-motion directions become discontinuous at
the singularity.
There is a more subtle disturbance at the singularity as well.
Most of the particle-motion direction vectors are drawn two-sided.
For the vectors along the dotted path looping around the singularity,
however, we have attempted to pick a preferred ``positive'' direction and so
these are drawn one-sided.
Once we make a choice of sign at the beginning, we are constrained at each
step to choose the direction consistent with the previous choice.
Unfortunately, the continuity we have carefully maintained going around
the loop is lost at the end; the choices at the beginning and end of
the loop do not agree.
Contrast this situation with the one in two dimensions shown in
Figure~\protect\ref{Separ3-Arrow2}.
\index{shear singularities | example}
\index{slowness surface | 3D example}
}

This branch cut
makes a mess of any attempt to construct a pure-mode scalar field.
If we are satisfied merely to separate the wavetypes, however, we
can avoid the sign-choice problem by using a slightly modified algorithm:
\label{Separ3-ModAlg}
\begin{list}{}{}
\item[($\hbox{\rm 3.}^\prime$)] Fourier transform $\U(x,y,z;t)$ over $x$, $y$, and $z$,
obtaining $\hat \U(k_x,k_y,k_z;t)$.
\item[($\hbox{\rm 4.}^\prime$)] For all $(k_x,k_y,k_z)$:
\item[] \{
$$ \hat \Mode(k_x,k_y,k_z;t) = \L(k_x,k_y,k_z) \Bigl( \L(k_x,k_y,k_z) \joecdot \hat \U(k_x,k_y,k_z;t) \Bigr) $$
\item[] \}
\item[($\hbox{\rm 5.}^\prime$)] Inverse Fourier transform $\hat \Mode(k_x,k_y,k_z;t)$
over $k_x$, $k_y$, and $k_z$, obtaining $\Mode(x,y,z;t)$.
\end{list}
The branch cut does not adversely affect this operator because the
sign of $\vv$ enters the equation twice, and so cancels out.

\Tactiveplot{Separ3-Sfig}{width=5.75in}{\figdir}
{Wavetype separation has trouble for 3D {\qS} modes.}
{Three-dimensional {\qS} mode-separation examples.
The plot and parameters are like those in Figure~\protect\ref{Separ3-Pfig},
except this time the $x$ component of displacement is shown and
a gpow of .8 has been applied to make weak events more easily visible.
Upper left: The original wavefield.
Upper right: The {\qP} mode has been successfully nulled.
Lower left: Trying to pass only the {\qS1} wave causes artifacts.
Lower right: The {\qS2} wave likewise doesn't work by itself.
The sum of the two lower plots gives the upper right plot.
\index{wavetype separation | example}
}
\Tactiveplottop{Separ3-SVfig}{width=5.75in}{\figdir}
{Trying to define and separate a 3D {\qSV} pseudo-mode.}
{Two attempts at defining and separating a three-dimensional {\qSV} mode.
The plot and parameters are like those in Figure~\protect\ref{Separ3-Sfig}.
Left: For each $(k_x,k_y,k_z)$ the {\qS} plane-wave mode with
particle motion more nearly ``SV'' was chosen. This pseudo-mode does
not have a continuous phase-velocity function.
Right: The orthorhombic elastic constants were perturbed to become
transversely isotropic, and the wavetype-separation operator appropriate for
the {\qSV} mode for the perturbed TI medium was applied to the original
orthorhombically anisotropic wavefield. As a result the {\qP} mode is
not completely nulled.
\index{wavetype separation | example}
}

This new wavetype-separation operator does not attempt to
collapse the vector wavefield onto a scalar pure mode $M$. Instead
it nulls the two unwanted wave types while passing the desired
pure-mode component through unchanged.
Since $\Mode$ is a vector wavefield like $\U$, some important
manipulations become easier than they would be with a scalar mode like $M$.
For example, we can construct the difference $\U - \Mode$;
this {\it nulls}\/ a single wavetype instead of passing a single wavetype.

Unfortunately, the fundamental problem of particle-motion discontinuities
at shear singularities remains. There is no solution to this problem;
it is a consequence of the basic inconsistency between
the ``furry ball'' theorem and
the definition of a {\qS} ``pure mode''.

I conclude that in general anisotropic media it is not possible
to cleanly separate the ``{\qS1} and {\qS2} modes''. The particle-motion
discontinuity at the singularities will always cause the two
modes to leak energy into each other.
\index{shear singularities | mode-mode coupling}
\index{wave modes | coupling}
This mode-mode coupling is especially bothersome for elastic ray tracing in
three dimensions.
It is possible to handle the coupling,
but only by propagating the two not-quite-distinct shear modes together
\refer{Chapman and Shearer}
{Ray tracing in azimuthally anisotropic media}{1989}.
While it may often prove useful to differentiate two {\qS} waves
for some set of propagation angles,
globally they are inextricably linked and cannot be cleanly
separated.

\subsection{Examples}
Figure~\ref{Separ3-Sfig} shows how this works in practice.
The upper left plot shows the trial input wavefield.
In the upper right plot the {\qP} wave has been successfully removed.
In the lower two plots I have attempted to pass only the {\qS1} or
{\qS2} pure modes. The Fourier-domain particle-motion discontinuities
at the shear singularities create strong planar artifacts that can
be seen to wrap around several times. The artifacts have opposite
signs on the {\qS1} and {\qS2} plots; when the two {\qS} modes are summed
the artifacts cancel and the ``{\qP} removed'' plot results.

For these models a $z$ source was used.
Since the only particle-motion discontinuity for SV modes is for
vertically traveling shear waves, which are unexcited by a $z$ source,
we might expect an SV mode-separation operator to be more tractable.
There are two possible approaches to use; Figure~\ref{Separ3-SVfig}
shows both of them.

In the left example in Figure~\ref{Separ3-SVfig}
for each $(k_x,k_y,k_z)$ plane-wave direction
I have chosen the mode that most nearly fits the designation ``SV''.
Sure enough, this operator picks out something resembling an SV mode,
but the resulting wavefield is cluttered by the artifacts caused by
both the particle-motion direction {\em and} phase-velocity
discontinuities.

In the right example in Figure~\ref{Separ3-SVfig}
I have perturbed the elastic constants of the
wavetype-separation operator to become transversely isotropic.
(This was easy to do, since the original orthorhombic elastic constants
were generated by numerically ``cracking''
\refer{Schoenberg and Muir}
{A calculus for finely layered anisotropic media}
{1989}
transversely isotropic
Greenhorn Shale
\refer{Jones and Wang}
{Ultrasonic velocities in Cretaceous shales from the Williston basin}
{1981}
in the first place.)
For the perturbed medium there is a true {\qSV} mode, and
so the separation works (given the $z$ source used in this example).
The {\qP} mode does slightly leak through because the elastic constants
don't quite fit the medium, however.


\section{Classifying singularities}
Clearly singularities are one of the more important features of
slowness surfaces; how do they change when the underlying elastic
constants are perturbed?
\label{Sec-Sing}

\Tactiveplot{Sing-Crack}{height=7.50in}{\figdir}
{How the TI kiss singularity splits.}
{
\small
Successive perturbations away from transverse isotropy. (The plots
are ordered from left to right and then top to bottom.)
The first (top left) plot corresponds to the TI medium shown
in the top plot in Figure~\protect\ref{Separ3-Wavedef}.
Only the outer {\qS2} wave is shown here.
The view is from the ``North Pole'': the $+k_z$ axis points
directly out of the page, the $+k_x$ axis points to the bottom of the
page and the $+k_y$ axis points to the right.
For the second (top right) and third (middle left) plot
successively more cracks are added in the $x$-$z$ plane.
Starting with the fourth (middle right) plot the first set of cracks
are held constant and more and more cracks are added
in the $y$-$z$ plane instead. For the fifth plot (lower left)
the amount of $x$-$z$ and $y$-$z$ cracking are the same.
\index{shear singularities | example}
\index{shear singularities | cracks}
\index{slowness surface | 3D example}
}

Figure~\ref{Sing-Crack} shows a succession of {\qS2} slowness surfaces
as the elastic constants are successively perturbed away from
transverse isotropy. The first plot is transversely isotropic.
Cracks are then added in the $x$-$z$ plane
to perturb the elastic
constants to become orthorhombic
\refer{Nichols, Muir, and Schoenberg}
{Elastic properties of rocks with multiple sets of fractures}
{1989}.
The TI kiss singularity at the center of the plot splits into
two point singularities on the $k_y$ axis,
a typical configuration for orthorhombic media
\refer{Crampin and Kirkwood}
{Velocity variations in systems of anisotropic symmetry}
{1981}.
Starting with the fourth plot in the sequence the $x$-$z$ cracks
are held constant and another crack set is
added in the $y$-$z$ plane.
The singularities move back towards the center and converge
back into a single kiss singularity when the amount of cracking
in the two perpendicular directions is equal. As the cracking in
the $y$-$z$ plane becomes dominant in the last plot,
the kiss singularity bifurcates again, this time along the $k_x$ axis.

\Tactivesideplot{Sing-Funny}{width=3.60in}{\figdir}
{Another view of Figure~\protect\ref{Separ3-Funnyshear}.}
{
Another view of the upper plot in Figure~\protect\ref{Separ3-Funnyshear}.
A Saskatchewan shaped portion of the slowness surface
containing the three point singularities visible near the $+k_y$ axis
has been cut out and rotated to the front.
(Ideally, Saskatchewan is bounded by two lines of constant
latitude and two lines of constant longitude.)
\index{shear singularities | example}
\index{slowness surface | 3D example}
}
\Tactiveplot{Sing-Sweep}{width=5.75in}{\figdir}
{How a singularity ``dipole'' appears.}
{
This sequence of four plots (left to right and then top to bottom)
shows how the particle-motion directions change as an orthorhombically
anisotropic medium is perturbed to become generally anisotropic.
Two singularities of opposite sign appear together out of nowhere
and then separate.
Figure~\protect\ref{Sing-Funny} forms the fifth plot in the sequence
and shows the orientation of all of the plots.
The elastic constants are given in Note~\protect\ref{Const3-Funny}
on page~\protect\pageref{Const3-Funny}.
\index{shear singularities | classification of}
\index{shear singularities | example}
\index{slowness surface | 3D example}
}

This behavior suggests to me an alternative notation for singularities
that has some advantages over the ``\{point, kiss, intersection\}''
terminology now in common use.
(This notation was first introduced by
Crampin and Yedlin~\referna{Crampin and Yedlin}
{Shear-wave singularities of wave propagation in anisotropic media}
{1981}, and more recently has been discussed by
Winterstein~\referna{Winterstein}
{Velocity anisotropy terminology for geophysicists}
{1990}.)
I suggest that singularities could be classified
by the number of half-loops the particle-motion direction vector
completes in one loop around the singularity, reminiscent of
integration around residuals in complex analysis.
\index{shear singularities | classification of}
In particular it is possible
for the particle-motion vector to rotate in the same
sense as the loop is traversed ($+$) or in the opposite sense ($-$).
This seems to be a fundamental property of singularities
that is robust against
perturbations in the elastic constants; the sum about a group of
singularities also appears to be conserved when
two of them merge or a double one splits.

Figure~\ref{Separ3-Blowup} shows a canonical example of an order $+1$
singularity; the particle-motion direction vector performs
a half flip in the same direction as the traverse around the singularity.
In Figure~\ref{Sing-Crack} the order $+2$ singularity on the $k_z$ axis
at the center of the plot splits into two order $+1$ singularities.
Although harder to see, the infinity of order $0$ intersection singularities
(the dark circle halfway out from the center in the first plot
in Figure~\ref{Sing-Crack}) splits into 4 order $+1$ singularities
(on the $k_x$ and $k_y$ axes) and 4 order $-1$ singularities (in between).

The labeled ``point'' singularity in Figure~\ref{Separ3-Funnyshear}
shows a clearer example of an order $-1$ singularity (the two adjacent
ones are each of order $+1$).\footnote{A closely related classification
scheme for ``umbilic points'' on surfaces is described by
Berry and Hannay~\referna{Berry and Hannay}
{Umbilic points on Gaussian random surfaces}
{1977}. My order $-1$ singularity they would call a ``Star''
singularity, which has ``index $-1/2$''.
My order $+1$ singularity they would
call a ``Lemon'' singularity, which has ``index $+1/2$''.
They also describe one more possible singularity type, the ``Monstar'',
which also has index $+1/2$.
I have not been able to generate an example of
a ``monstar'' shear singularity.}
Figure~\ref{Sing-Sweep} shows how this singularity spontaneously
appears as part of a singularity-antisingularity pair as originally
orthorhombic elastic constants are perturbed more and more.
The original order $+1$ singularity of the orthorhombic medium moves
slightly but cannot split, since the particle-motion direction vector
must complete an integral number of half-flips.


\section{Some canonical modeling examples}
\section{EXAMPLES}
\subsection{1. Simulating propagation of separated wave modes}

%%%%%%%%%%%%%%%%%%%%%%%%%%%%%%%%%%%%%%%%%%%%%%%%%%%%%%%%%%%%%%%%%%%%%%%%%%%%%%%%%%%%%%%%%%%%%%%%%%%%%
\subsubsection{1.1 Homogeneous VTI model}
\inputdir{homovti.eta0.05}

For comparison, we first appply the original anisotropic elastic wave equation
 to synthesize wavefields in a homogeneous VTI medium with weak anisotropy, in which
$v_{p0}=3000 m/s$, $v_{s0}=1500 m/s$, $\epsilon=0.1$, and $\delta=0.05$. 
Figure 4a and 4b display the horizontal and vertical components of the displacement wavefields at 0.3 s.
 Then we try to simulate propagation of separated wave modes using the pseudo-pure-mode qP-wave equation 
(equation~\ref{eq:pseudoVTIxy} in its 2D form).
 Figure 4c and 4d display the two components of the pseudo-pure-mode qP-wave fields, and Figure 4e
 displays their summation, i.e., the pseudo-pure-mode scalar qP-wave fields with weak residual qSV-wave energy. 
 Compared with the theoretical wavefront curves (see Figure 4f) calculated
on the base of group velocities
 and angles, pseudo-pure-mode scalar qP-wave fields have correct kinematics for both qP- and qSV-waves.
We finally remove residual qSV-waves and get completely separated scalar qP-wave fields by 
 applying the filtering to correct the projection deviation (Figure 4g).

\multiplot{7}{Elasticx,Elasticz,PseudoPurePx,PseudoPurePz,PseudoPureP,WF,PseudoPureSepP}{width=0.2\textwidth}
{
Synthesized wavefields in a VTI medium with weak anisotropy: (a) x- and
(b) z-components synthesized by original elastic wave equation; (c) x- and
 (d) z-components synthesized by pseudo-pure-mode qP-wave equation; (e) pseudo-pure-mode scalar qP-wave fields; 
(f) kinematics of qP- and qSV-waves; and (g) separated scalar qP-wave fields.
}

%%%%%%%%%%%%%%%%%%%%%%%%%%%%%%%%%%%%%%%%%%%%%%%%%%%%%%%%%%%%%%%%%%%%%%%%%%%%%%%%%%%%%%%%%%%%%%%%%%%%%
\inputdir{homovti.eta0.5}

Then we consider wavefield modeling in a homogeneous VTI medium with strong anisotropy,
 in which $v_{p0}=3000 m/s$, $v_{s0}=1500 m/s$, $\epsilon=0.25$, and $\delta=-0.25$.
 Figure 5 displays the wavefield snapshots at 0.3 s synthesized by using original elastic wave equation
and pseudo-pure-mode qP-wave equation respectively. Note that the pseudo-pure-mode qP-wave fields still accurately
represent the qP- and qSV-waves' kinematics. Although the residual qSV-wave energy becomes stronger when
the strength of anisotropy increases, the filtering step still removes these residual qSV-waves effectively.

\multiplot{7}{Elasticx,Elasticz,PseudoPurePx,PseudoPurePz,PseudoPureP,WF,PseudoPureSepP}{width=0.2\textwidth}
{
Synthesized wavefields in a VTI medium with strong anisotropy: (a) x- and
(b) z-components synthesized by original elastic wave equation; (c) x- and
 (d) z-components synthesized by pseudo-pure-mode qP-wave equation; (e) pseudo-pure-mode scalar qP-wave fields; 
(f) kinematics of qP- and qSV-waves; and (g) separated scalar qP-wave fields.
}

%%%%%%%%%%%%%%%%%%%%%%%%%%%%%%%%%%%%%%%%%%%%%%%%%%%%%%%%%%%%%%%%%%%%%%%%%%%%%%%%%%%%%%%%%%%%%%%%%%%%%
\subsubsection{1.2 Two-layer TI model}
\inputdir{twolayer2dti}

This example demonstrates the approach on a two-layer TI model, in which the first layer is a very
strong VTI medium with $v_{p0}=2500 m/s$, $v_{s0}=1200 m/s$, $\epsilon=0.25$, and $\delta=-0.25$, 
and the second layer is a TTI medium with $v_{p0}=3600 m/s$, $v_{s0}=1800 m/s$, 
$\epsilon=0.2$, $\delta=0.1$, and $\theta=30^{\circ}$. The horizontal interface between the two layers 
is positioned at a depth of 1.167 km.
 Figure 6a and 6d display the horizontal and vertical components of the displacement wavefields at 0.3 s.
 Using the pseudo-pure-mode qP-wave equation, we simulate equivalent wavefields on the same model.
Figure 6b and 6e display the two components of the pseudo-pure-mode qP-wave fields at the same time step.
Figure 6c and 6f display pseudo-pure-mode scalar qP-wave fields and separated qP-wave fields respectively.
Obviously, residual qSV-waves (including transmmited, reflected 
and converted qSV-waves) are effectively removed, and all transmitted, reflected as well as converted
qP-waves are accurately separated after the projection deviation correction.

\multiplot{6}{ElasticxInterf,PseudoPurePxInterf,PseudoPurePInterf,ElasticzInterf,PseudoPurePzInterf,PseudoPureSepPInterf}{width=0.25\textwidth}
{
Synthesized wavefields on a two-layer TI model with strong anisotropy and a tilted symmetry axis: (a) x- and 
(d) z-components synthesized by original elastic wave equation; (b) x- and
 (e) z-components synthesized by pseudo-pure-mode qP-wave equation; 
 (c) pseudo-pure-mode scalar qP-wave fields; (f) separated scalar qP-wave fields.
}


%%%%%%%%%%%%%%%%%%%%%%%%%%%%%%%%%%%%%%%%%%%%%%%%%%%%%%%%%%%%%%%%%%%%%%%%%%%%%%%%%%%%%%%%%%%%%%%%%%%%%
\subsubsection{1.3 BP 2007 TTI model}
\inputdir{bptti2007}

Next we test the approach of simulating propagation of the separated qP-wave mode
in a complex TTI model. Figure 7 shows parameters for part of
the BP 2D TTI model. The space grid size is 12.5 m and the time step is 1 ms for 
high-order finite-difference operators. Here the vertical velocities for the qSV-wave are set as
 half of the qP-wave velocities. 
Figure 8 displays snapshots of wavefield components at the time of 1.4s
synthesized by using original elastic wave equation and pseudo-pure-mode qP-wave equation.
The two pictures on
the right side represent the scalar pseudo-pure-mode qP-wave and the separated qP-wave fileds, respectively.
The correction appears to remove residual qSV-waves and accurately separate qP-wave data 
including the converted qS-qP waves from
the pseudo-pure-mode wavefields in this complex model.

\multiplot{4}{vp0,epsi,del,the}{width=0.3\textwidth}
{
Partial region of the 2D BP TTI model: (a) vertical qP-wave velocity, Thomsen coefficients
 (b) $\epsilon$ and (c) $\delta$, and (d) the tilt angle $\theta$. 
}

\multiplot{6}{Elasticx,Elasticz,PseudoPurePx,PseudoPurePz,PseudoPureP,PseudoPureSepP}{width=0.3\textwidth}
{
Synthesized elastic wavefields on BP 2007 TTI model using original elastic wave equation and pseudo-pure-mode 
qP-wave equation respectively: (a) x- and 
(b) z-components synthesized by original elastic wave equation; (c) x- and
 (d) z-components synthesized by pseudo-pure-mode qP-wave equation; 
 (e) pseudo-pure-mode scalar qP-wave fields; (f) separated scalar qP-wave fields.
}


%%%%%%%%%%%%%%%%%%%%%%%%%%%%%%%%%%%%%%%%%%%%%%%%%%%%%%%%%%%%%%%%%%%%%%%%%%%%%%%%%%%%%%%%%%%%%%%%%%%%%
\subsubsection{1.4 Homogeneous 3D ORT model}

\inputdir{ort3dhomo}

Figure 9 shows an example of simulating propagation of separated qP-wave fields in a 3D homogeneous
 vertical ORT model, in which $v_{p0}=3000 m/s$,
$v_{s0}=1500 m/s$, $\delta_{1}=-0.1$, $\delta_{2}=-0.0422$, $\delta_{3}=0.125$, $\epsilon_{1}=0.2$,
$\epsilon_{2}=0.067$, $\gamma_{1}=0.1$, and $\gamma_{2}=0.047$.
The first three pictures display wavefield snapshots at 0.5s synthesized by using
 pseudo-pure-mode qP-wave equation, according to equation~\ref{eq:ort}.
As shown in Figure 9d, qP-waves again appear\old{s} to dominate the wavefields in energy when we sum the 
three wavefield components of the pseudo-pure-mode qP-wave fields.
As for TI media, we obtain completely separated qP-wave fields from the
 pseudo-pure-mode wavefields once the correction of projection deviation is finished (Figure 9e).
By the way, in all above examples, we find that the filtering to remove qSV-waves does not
require the numerical dispersion of the qS-waves to be limited. So there is no additional requirement
of the grid size for qS-wave propagation. The effects of grid dispersion for the separation of low velocity
qS-waves will be further investigated in the second article of this series.
\multiplot{5}{PseudoPurePx,PseudoPurePy,PseudoPurePz,PseudoPureP,PseudoPureSepP}{width=0.3\textwidth}
{
Synthesized wavefield snapshots in a 3D homogeneous vertical ORT medium: (a) x-, (b) y- and (c) z-component
 of the pseudo-pure-mode qP-wave fields, (d) pseudo-pure-mode scalar qP-wave fields, (e) separated scalar qP-wave fields.
}


%%%%%%%%%%%%%%%%%%%%%%%%%%%%%%%%%%%%%%%%%%%%%%%%%%%%%%%%%%%%%%%%%%%%%%%%%%%%%%%%%%%%%%%%%%%%%%%%%%%%%
\subsection{2. Reverse-time migration of Hess VTI model}
\inputdir{hessvti}

Our final example shows application of the pseudo-pure-mode qP-wave equation (i.e., equation~\ref{eq:pseudoVTIxy} in its 2D form) 
 to RTM of conventional seismic data representing mainly qP-wave energy using the synthetic data of
 SEG/Hess VTI model (Figure 10).
In the original data set, there is no vertical velocity model for qSV-wave, namely $v_{s0}$.
 For simplicity, we first get this parameter by setting $\frac{v_{s0}}{v_{p0}}=0.5$ anywhere.
Figures 11a and 11b display the two components of the synthesized pseudo-pure-mode qP-wave fields,
 in which the source is located at the center of the windowed region of the original models. 
 We observe that the summed wavefields (i.e., pseudo-pure-mode scalar qP-wave fields) contain quite weak
 residual qSV-wave energy (Figure 11c).
For seismic imaging of qP-wave data, we try the finite nonzero $v_{s0}$ scheme \cite[]{fletcher:2009}
 to suppress qSV-wave artifacts and enhance computation stability.
Thanks to superposition of multi-shot migrated data, we obtain a good RTM result (Figure 12) 
using the common-shot gathers provided at http://software.seg.org, although spatial filtering
has not been used to remove the residual qSV-wave energy. This example shows that the proposed pseudo-pure-mode
qP-wave equation could be directly used for reverse-time migration of conventional single-component seismic data.

\multiplot{3}{hessvp0,hessepsilon,hessdelta}{width=0.3\textwidth}
{
Part of SEG/Hess VTI model with parameters of (a) vertical qP-wave velocity, Thomsen coefficients 
(b) $\epsilon$ and (c) $\delta$.
}

\multiplot{3}{PseudoPurePx,PseudoPurePz,PseudoPureP}
{width=0.3\textwidth}
{
Synthesized wavefields using the pseudo-pure-mode qP-wave equation in SEG/Hess VTI model:
The three snapshots are synthesized by fixing the ratio of $v_{s0}$ to $v_{p0}$ as 0.5.  
The pseudo-pure-mode qP-wave fields (c) are the sum of the (a) x- and (b) z-components 
of the pseudo-pure-mode wavefields.
}

\plot{hessrtm}{width=0.75\textwidth}
{
RTM of Hess VTI model using the pseudo-pure-mode qP-wave equation with nonzero finite $v_{s0}$.
}


\bibliographystyle{seg}
\bibliography{aniso}
