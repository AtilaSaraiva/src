\published{SEP report, 94, 205-212 (1997)}

\title{Multiple suppression using prediction-error filter}
\righthead{Multiple Suppression}
\lefthead{Sun}
\footer{SEP--94}
\email{yalei@sep.stanford.edu}
%\keywords{ multiple suppression, offset randomization, prediction-error filter }

\author{Yalei Sun}

\maketitle

\begin{abstract}
I present an approach to multiple suppression, that is 
based on the moveout between primary and multiple events in the CMP gather.
After normal moveout correction, primary events will be horizontal, 
whereas multiple events will not be. 
For each NMOed CMP gather, I reorder the 
offset in random order. Ideally, this process has little influence on the 
primaries, but 
it destroys the shape of the multiples. In other words, after randomization 
of the offset order, the multiples appear as random noise. This ``man-made''
random noise can be removed using prediction-error filter (PEF). 
The randomization of the offset order can be regarded as a random process, 
so we can apply it to the CMP gather many times and get many different 
samples. All the samples can be arranged into a 3-D cube, which is further 
divided into many small subcubes. A 3-D PEF can then be estimated from each 
subcube and re-applied 
to it to remove the multiple energy. After that, all the samples are averaged 
back into one CMP gather, which is supposed to be free of multiple events. 
In order to improve the efficiency of the 
algorithm of estimating the PEF for each subcube, except for the first 
subcube which starts with a zero-valued initial guess, all the subsequent 
subcubes take the last estimated PEF as an initial guess. 
Therefore, the iteration 
count can be reduced to one step for all the subsequent subcubes with little 
loss of accuracy. 
Three examples demonstrate the performance of this new approach, especially in 
removing the near-offset multiples.
\end{abstract}


\section{INTRODUCTION}

One of the essential differences between multiples and primaries is moveout. 
On the basis of this difference, a velocity-stacking multiple suppression
approach was developed by Lumley et al \shortcite{Lumley.sep.82.25}. In the 
velocity-stacking domain, the multiple and primary energy are separated and 
a masking function is applied to the data in the velocity-stacking domain to 
remove the multiple energy. Then the data can be
inverse-transformed back to the time domain with only primary energy left.

Since the velocity-stacking operator is time-variant, it does not have
the Toeplitz structure in the frequency domain. Therefore, given an operator
and its adjoint, the inverse transform has to be formulated as an $L_p$ norm
optimization problem, which makes this approach fairly expensive. In addition, 
this approach cannot handle the near-offset multiple events very well. 
Because the near-offset multiple event is parallel to the primary event, 
it is very difficult to separate it from primary energy in the velocity 
stacking domain.

In this paper, I propose a new approach, which also makes use of the moveout
difference between primary and multiple events. Instead of velocity-stacking 
transform, normal-moveout correction is applied to the CMP gather. Ideally,  
the hyperbolic primary events will be flattened after NMO correction, whereas 
the multiple events
will not be. There is a residual moveout for the multiple events. I then
randomize the order of offsets. This process will have little influence over the
primary, but the shape of the multiple event will be destroyed by the residual
moveout. In other words, after this randomization, the primary energy is still
coherent and the multiple energy will look like random noise. Therefore, the
multiple energy can be removed using a prediction-error filter (PEF). 

The randomization of the offset order can be regarded as a {\em  random 
process}, which can be applied to a single CMP gather many times to produce 
an ensemble of samples. The ensemble of samples forms a 3-D cube, which can be 
further 
decomposed into many small 3-D subcubes. A 3-D PEF is estimated from each 
subcube and then convolved with the cube to remove the multiple energy. 
After that, all the samples are averaged back into one CMP gather, 
which is ideally free of multiple. 
Here, the estimation of the PEF for each subcube is posed as an inversion 
problem.

In order to improve the efficiency of the algorithm, I set the initial guess of 
the PEF for one subcube to be the PEF of its preceding subcube. 
In other words, except for the first 
subcube, which starts with a zero-valued initial guess, all the subsequent 
subcubes take last estimated PEF as initial guess. Therefore, the iteration 
steps can be reduced to one step for all the subsequent subcubes with little 
loss of accuracy. Since we are mainly interested in preserving the horizontal 
events, there are some requirements regarding the shape of the PEF. 

Two synthetic and one real example are presented to show that this multiple 
suppression approach can remove the multiples from near to far offset.  
Especially in the near offset, its performance is very promising. Since this 
approach is based on PEF, other kinds of random noise will be removed as well 
as multiples. Therefore, the signal-to-noise ratio will be increased. 

\section{REORDERING THE OFFSET ORDER AND SIMULATING A RANDOM PROCESS}
\inputdir{syn}

Starting with a NMOed CMP gather, we can rearrange the offset order in a 
random manner.
Figure \ref{fig:syn-in} illustrates a synthetic NMOed CMP gather which 
contains one primary and one multiple event. After offset randomization, 
the primary event remains horizontal and the multiple event turns into 
``random noise''.

\plot{syn-in}{width=5in,height=3.2in}{A synthetic NMOed CMP gather 
contains one primary event and one multiple event.}

If we apply the offset randomization multiple times, all the different samples 
of this random 
process can generate a 3-D data cube. As shown in Figure \ref{fig:syn-cube}, 
the primary energy is coherent along the cube, whereas the multiple part is 
incoherent. Therefore, not only can we estimate a 2-D PEF in the CMP gather, 
but also estimate a 3-D PEF in the data cube. It is very natural to expect 
that the 3-D result will be better than the 2-D, since the 3-D approach 
exploits the difference between the primary and multiple more thoroughly. 

\plot{syn-cube}{width=5in,height=3.2in}{A NMOed CMP gather is 
transformed into a 3-D data cube by conducting the offset randomization 
for multiple times.} 


\section{{\tt T-X-Y} DOMAIN PREDICTION FILTERING}

The theory of {\tt T-X-Y} prediction filtering can be found in 
Claerbout \shortcite{Claerbout.tdf.82}. Abma \shortcite{Abma.sepphd.88} 
applies the {\tt T-X-Y} prediction-error filter to signal/noise separation.  
The algorithm used in this paper is the same as the one in Abma's thesis. 
Here, I show three examples of the application of {\tt T-X-Y} 
prediction filtering to two synthetic CMP gathers and one real CMP gather.

\subsection{A simple synthetic CMP gather}

The synthetic CMP gather consists of a horizontal event which represents 
the primary and a curved event representing the multiple. After offset 
randomization, the CMP gather is shown in Figure \ref{fig:syn-in}. Figure 
\ref{fig:syn-tx} illustrates the result of PEF-based multiple suppression in 
2-D case. In this simple example, although there are strong aliasing effects 
in the CMP gather, the 2-D prediction-error filter separates the horizontal 
and curved events very well.

\plot{syn-tx}{width=5in,height=3.2in}{A simple synthetic CMP gather composed of one primary event and one multiple event. This simple example demonstrates the applicability of the PEF based multiple suppression scheme.}

\subsection{A more complicated synthetic CMP gather}
\inputdir{haskell}

This synthetic gather is generated by the Thompson-Haskell method, which 
has been used as an example in Lumley et al \shortcite{Lumley.sep.82.25}.
The gather includes both primary and multiple events. Velocity stacking 
approach can remove the far offset multiples successfully, but it is not 
so successful in near offset, since the multiple event is nearly horizontal 
in the near offset. I show that PEF based approach can handle the near-offset 
multiples effectively.

Figure \ref{fig:hm-nmo-tx} illustrates that most of the 
multiple energy has been well-separated from primary energy, from near-offset 
to far-offset. I also extract three traces from the near offset and compare 
the difference between the two approaches, as shown in Figure 
\ref{fig:hm-near-offset-tx}. 
It is very clear that the PEF scheme has removed most of the multiple energy 
in the near offset. The signal-to-noise ratio of PEF based approach higher 
than the other two figures.

\plot{hm-nmo-tx}{width=5in,height=3.2in}{A synthetic CMP gather generated by Haskell-Thompson method. After applying the 3-D PEF, both multiple events and random noise have been removed out of the input CMP gather.}

\plot{hm-near-offset-tx}{width=5in,height=3.2in}{Comparison of the PEF scheme and velocity-stacking scheme in the near offset. Three traces are extracted from the near offset. The left one is the original input, the middle one is the PEF result, and the right one is the result of velocity stacking approach. It is very easy to identify the primary events in the middle one.}

\subsection{A real CMP gather}
\inputdir{mobil}

This real CMP gather is extracted from Mobil AVO dataset. The multiple energy 
is very strong in this dataset. It is not very difficult to locate several 
high amplitude horizontal events in the original events, as shown in Figure 
\ref{fig:mo-nmo-tx}. After convolving with the 3-D PEF, these events have been 
preserved. Most of the multiple energy has been removed. As for the 
near-offset traces, the signal-to-noise ratio of the PEF approach is also 
higher than that of the velocity-stacking approach. 

Obviously this result is not as convincing as last one. The reason is 
actually the assumption of the PEF based approach. The PEF based 
multiple suppression scheme has an assumption over the amplitude of the 
primary events, that is, the change of amplitude versus offset should be as 
small as possible. The first example is generated following this assumption 
strictly. Since this real data has strong AVO effect, the PEF based approach 
will remove this effect by forcing the amplitude to be invariant along offset. 
Therefore, some of the primary energy will be lost for this reason.
Even for this, the improvement of signal-to-noise ratio in the near offset 
is still obvious in the near offset, as shown in Figure 
\ref{fig:mo-near-offset-tx}.

\plot{mo-nmo-tx}{width=5in,height=3.2in}{A real CMP gather. This is from Mobil's AVO dataset. The high amplitude events in the input section are primary energy. The middle section is the result of the 3-D PEF scheme. Although most of the multiple events have been removed, the AVO effect is also damaged by the filtering. This shows that the PEF approach has an assumption of constant amplitude along the offset.}

\plot{mo-near-offset-tx}{width=5in,height=3.2in}{Comparison of the PEF scheme and velocity-stacking scheme in the near offset. Although the AVO effect has been damaged by the filtering, the signal-to-noise ratio is improved in the near offset.}

\section{CONCLUSION and discussion}

Two synthetic and one real CMP gathers show that this multiple 
suppression scheme can remove both near-offset and far-offset multiples. 
This multiple suppression approach has three features. The first 
one is offset randomization, which destroys the shape of multiple events. The 
idea of converting coherent noise like multiples into random noise is very 
novel, it may find applications in other fields. 
The second is the assumption of primary events being horizontal after normal 
moveout correction. This assumption makes it difficult in dealing with 
nonhyperbolic moveout. The third is that this approach cannot handle the 
amplitude variation along offset. Therefore, it will lose some primary energy 
caused by AVO or NMO stretch. Obviously, the second and third features 
will limit the application range of this approach. 

I used a trick to improve the efficiency in using {\tt T-X-Y} PEF. That is, 
taking the preceding subcube's PEF as an initial guess when estimating a new 
PEF. In my application, this trick can make the algorithm at least ten times 
faster than the algorithm without using this trick. The error is less than 
one percent. 

\section{Acknowledgements}

I thank Dr. Ray Abma of Shell and Dr. Jiuying Guo of Tsinghua University for 
stimulating discussions during the 66th SEG Denver convention.

\bibliographystyle{seg}
\bibliography{SEP2,pef}
