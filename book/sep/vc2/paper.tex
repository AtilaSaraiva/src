\lefthead{Fomel}
\righthead{Velocity continuation}
\footer{SEP--92}

\published{Geophysics, 68, 1662-1672 (2003)}
\title{Time migration velocity analysis by velocity continuation}

\email{sergey@sep.stanford.edu}
\author{Sergey Fomel}
\maketitle

\begin{abstract}
Time migration velocity analysis can be performed by velocity
  continuation, an incremental process that transforms
  migrated seismic sections according to changes in the migration velocity.
  Velocity continuation enhances residual \new{normal moveout} correction by
  properly taking into account both vertical and lateral movements of
  \new{events on seismic images}. Finite-difference and spectral algorithms
  provide efficient practical implementations for velocity continuation.
  Synthetic and field data examples demonstrate the performance of the method
  and confirm theoretical expectations.
\end{abstract}

\section{Introduction}
%%%%%%%%%%%%%%%%%%
Migration velocity analysis is a routine part of prestack time
migration applications. It serves both as a tool for velocity
estimation \cite[]{FBR08-06-02240234} and as a tool for optimal
stacking of migrated seismic sections prior to modeling zero-offset
data for depth migration \cite[]{GEO62-02-05680576}. In the most
common form, migration velocity analysis amounts to residual moveout
correction on CIP (common image point) gathers. However, in the case
of dipping reflectors, this correction does not provide optimal
focusing of reflection energy, since it does not account for lateral
movement of reflectors caused by the change in migration velocity. In
other words, different points on a stacking hyperbola in a CIP gather
can correspond to different reflection points at the actual reflector.
The situation is similar to that of the conventional \new{normal
  moveout} (NMO) velocity analysis, where the reflection point
dispersal problem is usually overcome with the help of \new{dip
  moveout} \cite[]{FBR04-07-00070024,dmo}. An analogous correction is
required for optimal focusing in the post-migration domain. In this
paper, I propose and test velocity continuation as a method of
migration velocity analysis. The method enhances the conventional
residual moveout correction by taking into account lateral movements
of migrated reflection events.

Velocity continuation is a process of transforming time migrated images
according to the changes in migration velocity. This process has wave-like
properties, which have been described in earlier papers
\cite[]{me,SEG-1997-1762,first}.  \cite{hubral} and
\cite{GEO62-02-05890597} use the term \emph{image waves} to describe a
similar concept. \cite{adlerSEG,adler} generalizes the velocity
continuation approach for the case of variable background velocities, using
the term \emph{Kirchhoff image propagation}. \new{Although the velocity
continuation concept is tailored for time migration, it finds important
applications in depth migration velocity analysis by recursive methods} 
\cite[]{SEG-1999-17231726,SEG-2000-08740877}.

\par
Applying velocity continuation to migration velocity analysis involves
the following steps: 
\begin{enumerate}
\item prestack common-offset (and common-azimuth) migration - to
  generate the initial data for continuation,
\item velocity continuation with stacking and semblance analysis across
  different offsets - to transform the offset data dimension into the velocity
  dimension,
\item picking the optimal velocity and slicing through the migrated
  data volume - to generate an optimally focused image.
\end{enumerate}
The first step transforms the data to the image space. The regularity of this
space can be exploited for devising efficient algorithms for the next two
steps. The idea of slicing through the velocity space goes back to the work of
\cite{shurtleff}, \cite{SEG-1984-S1.8,Fowler.sepphd.58}, and
\cite{GEO57-01-00510059}. While the previous slicing methods constructed
the velocity space by repeated migration with different velocities, velocity
continuation navigates directly in the migration velocity space without
returning to the original data. This leads to both more efficient algorithms
and a better understanding of the theoretical continuation properties
\cite[]{first}.

In this paper, I demonstrate all three steps, using both synthetic data and a
North Sea dataset. I introduce and exemplify two methods for the efficient
practical implementation of velocity continuation: the finite-difference
method and the Fourier spectral method. The Fourier method is recommended as
optimal in terms of the accuracy versus efficiency trade-off. Although all the
examples in this paper are two-dimensional, the method easily extends to 3-D
under the assumption of common-azimuth geometry (one oriented offset). More
investigation may be required to extend the method to the multi-azimuth case.

It is also important to note that although the velocity continuation result
could be achieved in principle by using prestack residual migration in
Kirchhoff \cite[]{Etgen.sepphd.68} or frequency-wavenumber
\cite[]{GEO61-02-06050607} formulation, the first is inferior in efficiency, and
the second is not convenient for the conventional velocity analysis across
different offsets, because it mixes them in the Fourier domain
\cite[]{SEG-2000-09920995}.  \new{Fourier-domain angle-gather analysis}
\cite[]{SEG-2001-02960299,sandf} \new{opens new possibilities for the future
development of the Fourier-domain velocity continuation. New insights into the
possibility of extending the method to depth migration can follow from the
work of} \cite{adler}.

\section{Numerical velocity continuation in the post-stack domain}
\par
The post-stack velocity continuation process is governed by a partial
differential equation in the domain, composed by the seismic image
coordinates (midpoint $x$ and vertical time $t$) and the additional
velocity coordinate $v$. Neglecting some amplitude-correcting terms
\cite[]{first}, the equation takes the form
\cite[]{Claerbout.sep.48.79}
\begin{equation}
  {\frac{\partial^2 P}{\partial v\, \partial t}} +
  {v\,t\,\frac{\partial^2 P}{\partial x^2}} = 0\;.
\label{PDE}
\end{equation}
Equation (\ref{PDE}) is linear and belongs to the hyperbolic type. It
describes a wave-type process with the velocity $v$ acting as a
propagation variable. Each constant-$v$ slice of the function
$P(x,t,v)$ corresponds to an image with the corresponding constant
velocity. The necessary boundary and initial conditions are 
\begin{equation}
  \label{BC}
  \left.P\right|_{t=T} = 0\;\quad \left.P\right|_{v=v_0} = P_0 (x,t)\;,
\end{equation}
where $v_0$ is the starting velocity, $T=0$ for continuation to a smaller
velocity and $T$ is the largest time on the image (completely attenuated
reflection energy) for continuation to a larger velocity.  The first case
corresponds to ``modeling'' (demigration); the latter case, to seismic
migration.
\par
Mathematically, equations (\ref{PDE}) and (\ref{BC}) define a
Goursat-type problem \cite[]{kurant2}. Its analytical solution can be
constructed by a variation of the Riemann method in the form of an
integral operator \cite[]{me,first}:
\begin{equation}
  P(t,x,v) = {\frac{1}{(2\,\pi)^{m/2}}}\,\int\, 
  {\frac{1}{(\sqrt{v^2-v_0^2}\,\rho)^{m/2}}}\, 
  \left(- \frac{\partial}{\partial t_0}\right)^{m/2}
  P_0\left(\frac{\rho}{\sqrt{v^2-v_0^2}},x_0\right)\,dx_0\;,
\label{kirch}
\end{equation}
where $\rho = \sqrt{(v^2-v_0^2)\,t^2 + (x - x_0)^2}$, $m=1$ in the 2-D
case, and $m=2$ in the 3-D case. In the case of continuation from zero
velocity $v_0=0$, operator (\ref{kirch}) is equivalent (up to the
amplitude weighting) to conventional Kirchoff time migration
\cite[]{GEO43-01-00490076}.  Similarly, in the frequency-wavenumber
domain, velocity continuation takes the form
\begin{equation}
  \label{stolt}
  \hat{P} (\omega,k,v) = \hat{P}_0 (\sqrt{\omega^2+k^2 (v^2-v_0^2)},k)\;,
\end{equation}
which is equivalent (up to scaling coefficients) to Stolt migration
\cite[]{GEO43-01-00230048}, regarded as the most efficient constant-velocity
migration method.
\par
If our task is to create many constant-velocity slices, there are
other ways to construct the solution of problem (\ref{PDE}-\ref{BC}).
Two alternative approaches are discussed in the next two
subsections.
\subsection{Finite-difference approach}
%%%%%%%%%%%%%%%%%%%%%%%%%%%%%%%%%%%%%%%%%%%%%%%%%%%%%%%%%%%%
The differential equation~(\ref{PDE}) has a mathematical form
analogous to that of the 15-degree wave extrapolation equation
\cite[]{Claerbout.blackwell.76}. Its finite-difference implementation,
first described by \cite{Claerbout.sep.48.79} and \cite{Li.sep.48.85},
is also analogous to that of the 15-degree equation, except for the
variable coefficients. One can write the implicit unconditionally
stable finite-difference scheme for the velocity continuation equation
in the form
\begin{equation}
({\bf I} + a_{j+1}^{i+1}\,{\bf T})\,{\bf P}_{j+1}^{i+1} - 
({\bf I} - a_{j}^{i+1}\,{\bf T})\,{\bf P}_{j}^{i+1} -
({\bf I} - a_{j+1}^{i}\,{\bf T})\,{\bf P}_{j+1}^{i} + 
({\bf I} + a_{j}^{i}\,{\bf T})\,{\bf P}_{j}^{i} = 0\;, 
\label{eqn:fds} 
\end{equation}
where index $i$ corresponds to the time dimension, index $j$
corresponds to the velocity dimension, ${\bf P}$ is a vector along the
midpoint direction, ${\bf I}$ is the identity matrix, ${\bf T}$
represents the finite-difference approximation to the 
second-derivative operator in midpoint,
and $a_{j}^{i} = v_j\,t_i\,{\Delta v\,\Delta t}$.

\inputdir{XFig}

In the two-dimensional case, equation \ref{eqn:fds} reduces to a
tridiagonal system of linear equations, which can be easily inverted.
In 3-D, a straightforward extension can be obtained by using either
directional splitting or helical schemes \cite[]{SEG-1998-1124}. The
direction of stable propagation is either forward in velocity and
backward in time or backward in velocity and forward in time as shown
in Figure \ref{fig:vlcfds}.

\sideplot{vlcfds}{height=2in}{Finite-difference scheme for the
velocity continuation equation. A stable propagation is either forward
in velocity and backward in time (a) or backward in velocity
and forward in time (b).}

\inputdir{vcmod}

In order to test the performance of the finite-difference velocity
continuation method, I use a simple synthetic model from
\cite{Claerbout.bei.95}. The reflectivity model is shown in
Figure \ref{fig:mod}. It contains several features that challenge the
migration performance: dipping beds, unconformity, syncline,
anticline, and fault. \new{The velocity is taken to be constant 
$v=1.5\,\mbox{km/s}$.}

\sideplot{mod}{height=2.5in}{Synthetic model for testing
finite-difference migration by velocity continuation.}

\new{Figures~\ref{fig:vlckaa}--\ref{fig:vlcstv}} compare invertability of different
migration methods. In all cases, constant-velocity modeling (demigration) 
%by the adjoint operator \cite[]{Claerbout.blackwell.92}
was followed by migration with the correct velocity.  Figures
\ref{fig:vlckaa} and \ref{fig:vlcsto} show the results of modeling and
migration with the Kirchhoff \cite[]{GEO43-01-00490076} and $f$-$k$
\cite[]{GEO43-01-00230048} methods, respectively. These figures should
be compared with Figure \ref{fig:vlcvel}, showing the analogous result
of the finite-difference velocity continuation.  The comparison
reveals a remarkable invertability of velocity continuation, which
reconstructs accurately the main features and frequency content of the
model. Since the forward operators were different for different
migrations, this comparison did not test the migration properties
themselves. For such a test, I compare the results of the Kirchhoff
and velocity-continuation migrations after Stolt modeling.  The result
of velocity continuation, shown in Figure \ref{fig:vlcstk,vlcstv}, is
noticeably more accurate than that of the Kirchhoff method.

%\activeplot{vlckir}{width=6in,height=2.5in}{ER}{Result of modeling and
%migration with the fast Kirchhoff method. Left plot shows the
%reconstructed model. Right plot compares the average amplitude
%spectrum of the true model with that of the reconstructed image.}

\plot{vlckaa}{width=6in}{Result of modeling and migration with the
Kirchhoff method. Top left plot shows the reconstructed model. Top
right plot compares the average amplitude spectrum of the true model
with that of the reconstructed image.  Bottom left is the
reconstruction error. Bottom right is the absolute error in the
spectrum.}

%\activeplot{vlcpha}{width=6in,height=2.5in}{ER}{Result of modeling and
%migration with the phase-shift method. Left plot shows the
%reconstructed model. Right plot compares the average amplitude
%spectrum of the true model with that of the reconstructed image.}

\plot{vlcsto}{width=6in}{Result of modeling and migration with the
  Stolt method. Top left plot shows the reconstructed model.  Top
  right plot compares the average amplitude spectrum of the true model
  with that of the reconstructed image. Bottom left is the
  reconstruction error. Bottom right is the absolute error in the
  spectrum.}

\plot{vlcvel}{width=6in}{Result of modeling and
  migration with the finite-difference velocity continuation. Top left plot
  shows the reconstructed model. Top right plot compares the average amplitude
  spectrum of the true model with that of the reconstructed image. Bottom left
  is the reconstruction error. Bottom right is the absolute error in the
  spectrum.  }

\multiplot{2}{vlcstk,vlcstv}{width=4in}{(a) Modeling with Stolt method, migration with the Kirchhoff
  method. (b) Modeling with Stolt method, migration with the
finite-difference velocity continuation. Left plots show the
reconstructed models. Right plots show the reconstruction errors.}

These tests confirm that finite-difference velocity continuation is an
attractive migration method. It possesses remarkable invertability properties,
which may be useful in applications that \new{require inversion. While the
  traditional migration methods transform the data between two completely
  different domains (data-space and image-space), velocity continuation
  accomplishes the same trans\-for\-ma\-ti\-on by propagating the data in the
  extended domain along the velocity direction. Inverse propagation restores
  the original data.} According to \cite{Li.sep.48.85}, the computational
speed of this method compares favorably with that of Stolt migration. The
advantage is apparent for cascaded migration or migration with multiple
velocity models. In these cases, the cost of Stolt migration increases in
direct proportion to the number of velocity models, while the cost of velocity
continuation stays the same.

\subsection{Fourier approach}

\inputdir{spec}

The change of variable $\sigma = t^2$ transforms
equation~(\ref{PDE}) to the form
\begin{equation}
  {2\,\frac{\partial^2 P}{\partial v\, \partial \sigma}} +
  {v\,\frac{\partial^2 P}{\partial x^2}} = 0\;,
\label{PDE2}
\end{equation}
whose coefficients do not depend on the time variables.  Double Fourier
transform in $\sigma$ and $x$ further simplifies equation (\ref{PDE2})
to the ordinary differential equation
\begin{equation}
  \label{ODE}
  2\,i\Omega\,{{d \hat{P}} \over {d v}} -
  v\,k^2\,\hat{P} = 0\;,
\end{equation}
where the ``frequency'' variable $\Omega$ corresponds to the stretched time
coordinate $\sigma$, and $k$ is the wavenumber in $x$:
\begin{equation}
\label{FFT}
\hat{P}(k,\Omega,v) = \int\!\!\int P(x,t,v)\,
e^{-i\,\Omega\,t^2 + i\,k\,x} d x\,dt
\end{equation}
Equation (\ref{ODE}) has
an explicit analytical solution
\begin{equation}
  \label{ODEsol}
  \hat{P} (k,\Omega,v) = \hat{P}_0 (k,\Omega)\,
  e^{\frac{i k^2(v_0^2-v^2)}{4\Omega}}\;,
\end{equation}
which leads to a very simple algorithm for the numerical velocity
continuation. The algorithm consists of the following steps:
\begin{enumerate}
\item \new{Input the zero-offset (post-stack) data migrated with velocity $v_0$ (or
  unmigrated if $v_0=0$).}
\item Transform the input from a regular grid in $t$ to a regular grid
  in $\sigma$.
\item Apply Fast Fourier Transform (FFT) in $x$ and $\sigma$.
\item Multiply by the all-pass phase-shift filter $e^{\frac{i
      k^2(v_0^2-v^2)}{4\Omega}}$.
\item Inverse FFT in $x$ and $\sigma$.
\item Inverse transform to a regular grid in $t$.
\end{enumerate}

\plot{t2}{width=6in}{Synthetic seismic data before (left) and after
  (right) transformation to the $\sigma$ grid.}

Figure \ref{fig:t2} shows a simple synthetic model of seismic
reflection data generated from the model in Figure~\ref{fig:mod}
before and after transforming the grid, regularly spaced in $t$, to a
grid, regular in $\sigma$. The left plot of Figure \ref{fig:t2-fft}
shows the Fourier transform of the data. Except for the nearly
vertical event, which corresponds to a stack of parallel layers in the
shallow part of the data, the data frequency range is contained near
the origin in the $\Omega-k$ space.  The right plot of Figure
\ref{fig:t2-fft} shows the phase-shift filter for continuation from
zero imaging velocity (which corresponds to unprocessed data) to the
velocity of 1 km/sec. The rapidly oscillating part (small frequencies
and large wavenumbers) is exactly in the region, where the data
spectrum is zero. It corresponds to physically impossible reflection
events.

\plot{t2-fft}{width=6in}{Left: the real part of the data Fourier
  transform.  Right: the real part of the velocity continuation
  operator (continuation from 0 to 1 km/s) in the Fourier domain.}

\new{The described algorithm}
 is very attractive from the practical point of view
because of its efficiency (based on the FFT algorithm). The operation count is
roughly the same as in the Stolt migration implemented with equation
(\ref{stolt}): two forward and inverse FFTs and forward and inverse grid
transform with interpolation (one complex-number transform in the case of
Stolt migration). The velocity continuation algorithm can be more efficient
than the Stolt method because of the simpler structure of the innermost loop
\new{(step 4 in the algorithm)}. 

\section{Numerical velocity continuation in the prestack domain}

To generalize the algorithm of the previous section to the prestack case, it is
first necessary to include the residual NMO term \cite[]{first}.  Residual normal
moveout can be formulated with the help of the differential equation:
\begin{equation}
{{\partial P} \over {\partial v}} + 
{{h^2} \over {v^3\,t}}\,{{\partial P} \over {\partial t}} = 0\;,
\label{eqn:ResNMOdyn} 
\end{equation}
where $h$ stands for the half-offset.  The analytical solution of
equation (\ref{eqn:ResNMOdyn}) has the form of the residual NMO
operator:
\begin{equation}
  P(t,h,v) = P_0\left(\sqrt{t^2 + h^2\,
    \left(\frac{1}{v_0^2} - \frac{1}{v^2}\right)},h\right)\;.
\label{eqn:ResNMOsol} 
\end{equation}
After transforming to the squared time $\sigma = t^2$ and the
corresponding Fourier frequency $\Omega$, equation
(\ref{eqn:ResNMOdyn}) takes the form of the ordinary differential
equation
\begin{equation}
  \frac{d \hat{P}}{d v} + 
  i \Omega\,\frac{2\,h^2}{v^3}\,\hat{P} = 0
\label{eqn:FFTResNMO} 
\end{equation}
with the analytical frequency-domain phase-shift solution
\begin{equation}
  \hat{P} (\Omega, h, v) = \hat{P_0} (\Omega,h) e^{i\,\Omega\,h^2\,
    \left(\frac{1}{v_0^2} - \frac{1}{v^2}\right)}\;.
\label{eqn:ResNMOshift} 
\end{equation}
To obtain a Fourier-domain prestack velocity continuation algorithm, one just
needs to combine the phase-shift operators in equations (\ref{ODEsol}) and
(\ref{eqn:ResNMOshift}) and to include stacking across different offsets. 
\new{The
exact velocity continuation theory also includes the residual DMO term}
\cite[]{first}, \new{which has a second-order effect, pronounced only at small
depths. It is neglected here for simplicity.}  The algorithm takes the
following form:
\begin{enumerate}
  \item Input a set of common-offset images, migrated with velocity $v_0$.
  \item Transform the time axis $t$ to the squared time coordinate:
    $\sigma=t^2$.
  \item Apply a fast Fourier transform (FFT) on both the squared time
    and the midpoint axis. The squared time $\sigma$ transforms to the
    frequency $\Omega$, and the midpoint coordinate $x$ transforms to
    the wavenumber $k$. 
  \item Apply a phase-shift operator to transform to different velocities $v$:
    \begin{equation}
      \label{eqn:zophase2}
      \hat{P}(\Omega,k,v) = \sum_{h} \hat{P}_0 (\Omega,k,h)\,
      e^{i\,\frac{k^2\left(v_0^2 - v^2\right)}{4\,\Omega} +
        i\,\Omega\,h^2\, \left(\frac{1}{v_0^2} -
          \frac{1}{v^2}\right)}\;.
    \end{equation}
    To save memory, the continuation step is immediately followed
    by stacking. \new{For velocity analysis purposes, a semblance measure}
    \cite[]{GEO36-03-04820497} \new{is computed in addition to the simple stack
    analogously to the standard practice of stacking velocity analysis.}
    
  \new{Implementing the residual moveout correction in the Fourier
    domain allows one to package it conveniently with the phase-shift
    operator without the need to transform the continuation result
    back to the time domain. The offset dimension in
    equation~(\ref{eqn:zophase2}) is replaced by the velocity
    dimension similarly to the velocity transform of the conventional
    stacking velocity analysis} \cite[]{yilmaz}.

  \item Apply an inverse FFT to transform from $\Omega$ and $k$ to
    $\sigma$ and $x$.
  \item Apply an inverse time stretch to transform from $\sigma$ to $t$.
  \end{enumerate}  
  One can design similar algorithms by using the finite difference method.
  \new{Although the finite-difference approach offers a faster continuation speed,
  the spectral algorithm has a higher accuracy while maintaining an acceptable
  cost.}

\inputdir{vcspk}

%The complete theory of prestack velocity continuation also requires a
%residual DMO operator
%\cite[]{Etgen.sepphd.68,Fomel.sep.92.159,Fomel.segab.97}. However, the
%difficulty of implementing this operator is not fully compensated by
%its contribution to the full velocity continuation. For simplicity, I
%decided not to include residual DMO in the current implementation.
\par  
Figure~\ref{fig:velimp} shows impulse responses of prestack velocity
continuation. The input for producing this figure was a time-migrated
constant-offset section, corresponding to an offset of 1 km and a constant
migration velocity of 1 km/s. In full accordance with the theory \cite[]{first},
three spikes in the input section transformed into shifted ellipsoids after
continuation to a higher velocity and into shifted hyperbolas after
continuation to a smaller velocity. \new{
Padding of the time axis helps to avoid
the wrap-around artifacts of the Fourier method. Alternatively, one could use
the artifact-free but more expensive Chebyshev spectral method}
\cite[]{Fomel.sep.97.sergey2}.

\plot{velimp}{width=6in}{Impulse responses of prestack velocity
  continuation. Left plot: continuation from 1 km/s to 1.5 km/s.
  Right plot: continuation from 1 km/s to 0.7 km/s. Both plots
  correspond to the offset of 1 km.}
\par

Velocity continuation creates a time-midpoint-velocity cube (four-dimensional
for 3-D data), which is convenient for picking imaging velocities in the same
way as the result of common-midpoint or common-reflection-point velocity
analysis. The important difference is that velocity continuation provides an
optimal focusing of the reflection energy by properly taking into account both
vertical and lateral movements of reflector images with changing migration
velocity. \new{An experimental evidence for this conclusion is provided in
  the examples section of this paper.}

\begin{comment}
Figure \ref{fig:consmb} compares velocity spectra
(semblance panels) at a CIP location of about 11.5 km after residual
NMO and after prestack velocity continuation. Although the overall
difference between the two panels is small, the velocity continuation
panel shows a noticeably better focusing, especially in the region of
conflicting dips between 1 and 2 seconds. 
\end{comment}
The next subsection discusses
the velocity picking step in more detail.

%\plot{consmb}{width=6in}{Velocity spectra around 11.5 km CRP after
%  residual NMO (left) and after prestack velocity continuation
%  (right). The right plot shows improved focusing in the region between
%  1 and 2 seconds.}

\subsection{Velocity picking and slicing}

After the velocity continuation process has created a time-midpoint-velocity
cube, one can pick the best focusing velocity from that cube and create an
optimally focused image by slicing through the cube. \new{This step is common in
other methods that involve velocity slicing}
\cite[]{shurtleff,SEG-1984-S1.8,GEO57-01-00510059}. \new{The algorithm described
below has been also adopted by} \cite{SEG-2000-09920995} \new{for velocity
analysis in wave-equation migration.}

A simple automatic velocity picking algorithm follows from solving the
following regularized least-squares system:
\begin{equation}
  \label{eqn:pick}
  \left\{\begin{array}{rcl}
      \mathbf{W}\,\mathbf{x} & \approx & \mathbf{W}\,\mathbf{p} \\
      \epsilon \mathbf{D} \mathbf{x} & \approx & \mathbf{0}
    \end{array}\right.\;.
\end{equation}
\new{In the more standard notation, the solution $\mathbf{x}$ minimizes the
least-squares objective function}
\begin{equation}
  \label{eqn:ofpick}
(\mathbf{x}-\mathbf{p})^{T}\, \mathbf{W}^2\,(\mathbf{x}-\mathbf{p}) +
\epsilon^2\,\mathbf{x}^T\,\mathbf{D}^T\,\mathbf{D}\,\mathbf{x}
\end{equation}
Here $\mathbf{p}$ is the vector of blind maximum-semblance picks (possibly in a
predefined fairway), $\mathbf{x}$ is the estimated velocity picks, $\mathbf{W}$ is
the weighting operator with the weight corresponding to the semblance values
at $\mathbf{p}$, $\epsilon$ is the scalar regularization parameter, $\mathbf{D}$
is a roughening operator, and $\mathbf{D}^T$ is the adjoint operator.  The first
least-squares fitting goal in (\ref{eqn:pick}) states that the estimated
velocity picks should match the measured picks where the semblance is high
enough\footnote{Of course, this goal might be dangerous, if the original picks
  $\mathbf{p}$ include regular noise (such as multiple reflections) with high
  semblance value \cite[]{Toldi.sepphd.43}. For simplicity, and to preserve the
  linearity of the problem, I assume that this is not the case.}.  The second
fitting goal tries to find the smoothest velocity function possible.  The
least-squares solution of problem (\ref{eqn:pick}) takes the form
\begin{equation}
  \label{eqn:LS}
  \mathbf{x} = 
  \left(\mathbf{W}^2 + 
    \epsilon^2\,\mathbf{D}^T\,\mathbf{D}\right)^{-1}\,\mathbf{W}^2\,\mathbf{p}\;.
\end{equation}

\inputdir{beivc}

In the case of picking a one-dimensional velocity function from a single
semblance panel, one can simplify the algorithm by choosing $\mathbf{D}$ to be a
convolution with the derivative filter $(1,-1)$. It is easy to see that in
this case the inverted matrix in formula (\ref{eqn:LS}) has a tridiagonal
structure and therefore can be easily inverted with a linear-time algorithm.
The regularization parameter $\epsilon$ controls the amount of smoothing of
the estimated velocity function.  Figure \ref{fig:velpick} shows an example
velocity spectrum and two automatic picks for different values of $\epsilon$.

\plot{velpick}{width=6in}{Semblance panel (left) and automatic
  velocity picks for different values of the regularization parameter.
%  Center: $\epsilon=0.01$, right: $\epsilon=0.1$. 
Higher values of
  $\epsilon$ lead to smoother velocities.}

\par
In the case of picking two- or three-dimensional velocity functions,
one could generalize problem (\ref{eqn:pick}) by defining $\mathbf{D}$
as a 2-D or 3-D roughening operator. I chose to use a more simplistic
approach, which retains the one-dimensional structure of the algorithm. 
I transform system (\ref{eqn:pick}) to the form
\begin{equation}
  \label{eqn:pick2}
  \left\{\begin{array}{rcl}
      \mathbf{W}\,\mathbf{x} & \approx & \mathbf{W}\,\mathbf{p} \\
      \epsilon \mathbf{D} \mathbf{x} & \approx & \mathbf{0} \\
      \lambda \mathbf{x} & \approx & \lambda \mathbf{x_0}
    \end{array}\right.\;,
\end{equation}
where $\mathbf{x}$ is still one-dimensional, and $\mathbf{x}_0$ is the
estimate from the previous midpoint location. The scalar parameter
$\lambda$ controls the amount of lateral continuity in the estimated
velocity function. The least-squares solution to system
(\ref{eqn:pick2}) takes the form
\begin{equation}
  \label{eqn:LS2}
  \mathbf{x} = 
  \left(\mathbf{W}^2 + \epsilon^2\,\mathbf{D}^T\,\mathbf{D}
    + \lambda^2\,\mathbf{I}\right)^{-1}\,
  \left(\mathbf{W}^2\,\mathbf{p} + \lambda^2 \mathbf{x_0}\right)\;,
\end{equation}
where $\mathbf{I}$ denotes the identity matrix. Formula (\ref{eqn:LS2})
also reduces to an efficient tridiagonal matrix inversion.

\new{After the velocity has been picked, an optimally focused image is constructed
by slicing in the time-midpoint-velocity cube. I used simple linear
interpolation for slicing between the velocity grid values. A more accurate
interpolation technique can be easily adopted.}

\section{Examples}

I demonstrate the performance of the method using a simple 2-D synthetic
test and a field data example from the North Sea.

\subsection{Synthetic Test}
\inputdir{sigvc}

The synthetic test uses constant-velocity prestack modeling and migration to
check the validity of the method when all the theoretical requirements are
satisfied. The data were generated from the synthetic reflectivity model
(Figure~\ref{fig:mod}) and included 60~offsets ranging from 0 to 0.5~km.
The exact velocity in the model is~1.5~km/s, and the initial velocity for
starting the continuation process was chosen at 2~km/s.

%\plot{sig-mva2}{width=6in}{Semblance panels for migration
%  velocity analysis at the common image point~0.5~km. Left: after
%  velocity continuation.  Right: after conventional (NMO) velocity
%  analysis. In this structurally simple region, the difference between
%  two methods is small. The correct velocity is 1.5~km/s.}

Figure~\ref{fig:sig-mva1} compares the semblance panels for migration
velocity analysis using velocity continuation and using the
conventional (NMO) analysis. In the top part of the image, both panels
show maximum picks at the correct velocity (1.5~km/s).  The advantage
of velocity continuation is immediately obvious in the deeper part of
the image, where the events are noticeably better focused.

\plot{sig-mva1}{width=6in}{Semblance panels for migration
  velocity analysis at the common image point~1~km. Left: after
  velocity continuation.  Right: after conventional (NMO) velocity
  analysis. In a structurally complex region, velocity continuation
  clearly provides better focusing. The correct velocity is 1.5~km/s.}

The final result of velocity continuation (after picking maximum semblance and
slicing in the velocity cube) is shown in the bottom left plot of
Figure~\ref{fig:sig-all}. For comparison, Figure~\ref{fig:sig-all} also shows
the result of migration with the correct velocity (the top left plot), initial
velocity (the top right plot), and the result of velocity slicing after the
simple NMO correction, corresponding to the conventional MVA (the bottom right
plot). \new{The same velocity picking and slicing program was used in both cases.}
The comparison clearly shows that, in this simple example, velocity
continuation is able to accurately reproduce the correct image without using
any prior information about the migration velocity and without any need for
repeating the prestack migration procedure. \new{Velocity continuation correctly
images events with conflicting dips by properly taking into account both
vertical and lateral shifts in the image position.}

\plot{sig-all}{width=6in,height=6in}{Velocity continuation tested 
  on the synthetic example. Top left: prestack migration with the
  correct velocity of 1.5~km/s. Top right: prestack migration
  with the velocity of 2~km/s. Bottom left: the result of velocity
  continuation. Bottom right: the result from picking migration
  image after only conventional NMO correction.}

\subsection{\new{Field} Data Example}
\inputdir{elfvc}

Figure \ref{fig:elf-migr} compares the result of a constant-velocity prestack
migration with the velocity of 2 km/s, applied to a dataset from the North Sea
(courtesy of Elf Aquitaine) and the result of velocity continuation to the
same velocity from a migration with a smaller velocity of 1.4 km/s (Figure
\ref{fig:elf-migr}a). The two images (Figures \ref{fig:elf-migr}b and
\ref{fig:elf-migr}c) look remarkably similar, in full accordance with the
theory.

\plot{elf-migr}{width=6in}{Constant-offset section of the North Sea
  dataset after migration with the velocity of 1.4 km/s (a), migration
  with the velocity of 2 km/s (b), migration with the velocity of 1.4
  km/s and velocity continuation to 2 km/s (c).}

Figure~\ref{fig:elf-npk} shows a result of two-dimensional velocity
picking after velocity continuation. I used values of $\epsilon=0.1$
and $\lambda=0.1$. The first parameter controls the vertical smoothing
of velocities, while the second parameter controls the amount of
lateral continuity.

%\ref{fig:beivpk} shows a result of two-dimensional velocity picking
%after residual NMO. Figure \ref{fig:beifpk} shows an analogous result

%The difference between residual NMO velocities and
%velocities picked after velocity continuation is small, but clearly
%visible.

%\plot{beivpk}{width=6in,height=3.5in}{Automatic picks of 2-D RMS velocity
%  after residual NMO. The contour spacing is 0.1 km/s, starting from 1.5 km/s.}

\plot{elf-npk}{angle=90,totalheight=8.5in,width=6in}{Automatically
  picked migration velocity after velocity continuation.}

Figure \ref{fig:elf-fmg} shows the final result of velocity
continuation: an image, obtained by slicing through the velocity cube
with the picked imaging velocities. The edges of the salt body in the middle
of the section have been sharply focused by the velocity
continuation process. To transform the already well focused image into the
depth domain, one may proceed in a way similar to
\emph{hybrid migration}: demigration to zero-offset, followed by
post-stack depth migration \cite[]{GEO62-02-05680576}. This step
would require constructing an interval velocity model from the picked imaging
velocities.

\plot{elf-fmg}{angle=90,totalheight=8.5in,width=6in}{Final result of
  velocity continuation: seismic image, obtained by slicing through
  the velocity cube.}

\begin{comment}
Without repeating the details of the procedure, Figures~\ref{fig:pck}
and \ref{fig:img} show picked imaging velocities and the velocity
continuation image for the Blake Outer Ridge data, shown in many other
papers in this report.

%\plot{pck}{width=6in,height=3.5in}{Blake Outer Ridge data. Automatic
%  picks of 2-D imaging velocity after velocity continuation. The contour
%  spacing is 0.01 km/s, starting from 1.5 km/s.}

%\plot{img}{angle=90,totalheight=8.5in,width=6in}{Blake Outer Ridge
%  data. Final result of velocity continuation: seismic image, obtained
%  by slicing through the velocity cube.}
\end{comment}

\section{Conclusions}

Velocity continuation is a powerful method for time migration velocity
analysis.  The strength of this method follows from its ability to take into
account both vertical and lateral movement of the reflection events in seismic
images with the changes of migration velocity.

Efficient practical algorithms for velocity continuation can be constructed
using either finite-difference or spectral methods. When applied in the
post-stack (zero-offset) setting, velocity continuation can be used as a
computationally attractive method of time migration. Both finite-difference
and spectral approaches possess remarkable invertability properties:
continuation to a lower velocity reverses continuation to a higher velocity.
\new{For the finite-difference algorithm, this property is confirmed by synthetic
tests. For the spectral algorithm, it follows from the fact that velocity
continuation reduces to a simple phase-shift unitary operator.}

\new{Including velocity continuation in the practice of migration velocity analysis
can improve the focusing power of time migration and reduce the production
time by avoiding the need for iterative velocity refinement. No prior velocity
model is required for this type of velocity analysis. This conclusion is
confirmed by synthetic and field data examples.}

\section{Acknowledgments}

I thank Jon Claerbout, Biondo Biondi, and Bill Symes for useful and
stimulating discussions, the sponsors of the Stanford Exploration Project for
their financial support, and Elf Aquitaine for providing the data used in this
work. \new{I am also grateful to Paul Fowler, Samuel Gray, Hugh Geiger, and one
anonymous reviewer for thorough and helpful reviews that improved the quality
of the paper.}

\newpage
\bibliographystyle{seg}
\bibliography{SEP2,SEG,paper,spec,velcon}

%%% Local Variables: 
%%% mode: latex
%%% TeX-master: t
%%% TeX-master: t
%%% TeX-master: t
%%% TeX-master: t
%%% TeX-master: t
%%% TeX-master: t
%%% End: 
