\published{SEP report, 70, 361-366 (1991)}

\footer{SEP--70}
\lefthead{Dellinger \& Muir}
\righthead{double-elliptic approximation}
\title{The double-elliptic approximation \\ in the group and phase domains}
%\keywords{approximation, velocity, dispersion }
\author{Joe Dellinger and Francis Muir}

\maketitle

\section{Introduction}
Elliptical anisotropy has found wide use as a simple approximation
to transverse isotropy because of a unique symmetry property
(an elliptical dispersion relation corresponds to an elliptical
impulse response) and a simple relationship to standard geophysical
techniques (hyperbolic moveout corresponds to elliptical wavefronts;
NMO measures horizontal velocity, and time-to-depth conversion
depends on vertical velocity).
However, elliptical anisotropy is only useful as an approximation
in certain restricted cases, such as when the underlying true anisotropy
does not depart too far from ellipticity
or the observed angular aperture is small.
This limitation is fundamental, because there are only two parameters
needed to define an ellipse: the horizontal and vertical velocities.
(Sometimes the orientation of the principle axes is also included as a
free parameter, but usually not.)
% get Levin reference from Pat Berge

In a previous SEP report \cite{Muir.sep.67.41} showed how to extend the standard
elliptical approximation to a so-called {\em double\/}-elliptic form. (The 
relation between the elastic constants of a TI medium 
and the coefficients of the corresponding
double-elliptic approximation is
developed in a companion paper, \cite[]{Muir.sep.70.367}.)
The aim of this new approximation is to preserve the useful properties
of elliptical anisotropy while doubling the number of free parameters,
thus allowing a much wider range of transversely isotropic media to
be adequately fit.
At first glance this goal seems unattainable: elliptical anisotropy 
is the most complex form of anisotropy possible with a simple analytical form 
in both the dispersion relation and impulse response domains.
Muir's approximation is useful because it {\em nearly\/} satisfies
both incompatible goals at once:
it has a simple relationship to NMO and
true vertical and horizontal velocity, and to a good approximation
it has the same simple analytical form in both domains of interest.

The purpose of this short note is to test by example how well
the double-elliptic approximation comes to meeting these goals:
\begin{enumerate}
\item Simple relationships to NMO and true velocities on principle axes.
\item Simple analytical form for both the dispersion relation and impulse 
response.
\item Approximates general transversely isotropic media well.
\end{enumerate}
The results indicate that the method should work well in practice.

\section{REVIEW OF SYMMETRIC PROPERTIES OF ELLIPSES}
\inputdir{XFig}
Figure~\ref{fig:group-four} shows a transversely isotropic medium
and its vertical paraxial elliptic approximation represented
four different ways. Note that the dashed elliptical-approximation curve
plots as an ellipse in both the group-velocity and phase-slowness domains.
The vertical and horizontal velocities of the elliptical impulse response
define its principle axes; the corresponding vertical and horizontal
slownesses give the principle axes of the elliptical dispersion relation.

\plot{group-four}{width=4.5in}{Group velocity (= impulse response), group slowness, phase velocity,
and phase slowness (= dispersion relation) plots for the {\sl q}SV
mode of Greenhorn Shale (thick solid line) and an elliptically anisotropic
paraxial approximation to it (thin dashed line). The ``$\hbox{\rm R}$''
and ``$1 / \hbox{\rm r}$'' symbols indicate the recurring transformations
linking the four representations.}

\inputdir{vels}

These vertical and horizontal velocities have simple geophysical
interpretations. The vertical velocity
of the paraxial elliptic approximation
(equals the true vertical velocity)
is what we need to do time-to-depth
conversion.
The horizontal velocity
of the paraxial elliptic approximation
({\em not\/} the same thing as the true horizontal velocity)
is what is required for NMO.
(In a surface survey the sources and receivers are laid out along a
horizontal line; if the vertical velocity were changed,
it would effectively change the vertical scale of the survey, but
the traveltime field recorded along the horizontal surface would
remain unchanged.
So it is horizontal velocity that matters. But it isn't the true
horizontal velocity, because we're only considering NMO for near-vertical
propagation. It is the horizontal velocity of the vertical paraxial
elliptic approximation.)

\plot{verthoriz}{width=5.9in}
{
Three different approximations (dashed curves) to the
{\sl q}SV impulse-response surface of Greenhorn Shale (bold curves).
On the left is the standard vertical paraxial elliptic approximation. In the
center is the horizontal paraxial elliptic approximation.
On the right is Muir's double-elliptic approximation.
}
\plot{verthoriz2}{width=5.9in}
{
Three different approximations (dashed curves) to the
{\sl q}SV dispersion-relation surface of Greenhorn Shale (bold curves).
On the left is the standard vertical paraxial elliptic approximation. In the
center is the horizontal paraxial elliptic approximation.
On the right is Muir's double-elliptic approximation;
the fit is so close the dashed curve is hard to see.
}

\section{THE DOUBLE ELLIPTIC APPROXIMATION...}
\subsection{in the impulse-response domain}
Muir's trick is to turn the problem on its side and also look at the
{\em horizontal\/} paraxial approximating ellipse. This has all the
useful properties of the vertical ellipse, with the difference
that it fits the true horizontal velocity and the vertical NMO velocity
(not the same thing as the true vertical velocity).
Figure~\ref{fig:verthoriz} shows the vertical and horizontal
paraxial approximating ellipses, along with
Muir's double-elliptic approximation. Muir's approximation fits
four parameters, the vertical and horizontal true and NMO velocities.
Since it is single-valued, it cannot follow the {\sl q}SV triplication,
but fits well elsewhere.

\subsection{in the dispersion-relation domain}
There is no such problem in the dispersion-relation domain, since
triplications are impossible there.
Figure~\ref{fig:verthoriz2} shows how the three approximations shown
in Figure~\ref{fig:verthoriz} work in the dispersion-relation domain.
Just like elliptical anisotropy,
Muir's four-parameter approximation is calculated
in the dispersion-relation domain
by simply using slownesses instead of velocities
in the same approximation formula.
Note in Figure~\ref{fig:verthoriz2} that the
double-elliptic approximation in the dispersion-relation domain
is nearly exact for this example.

\subsection{but are they consistent?}
Of course there is one technicality: the elliptical approximations in
the previous two figures were perfectly consistent, but the double-elliptic
approximations were not. Muir's approximation did a much better job
in the dispersion-relation domain than it did in the impulse-response domain.
Figure~\ref{fig:compare} shows how the double-elliptic approximations
fit in the two different domains compare in the impulse-response domain.
Should we be concerned by this discrepancy?
Probably not, because this is a rather extreme example.
If there are no triplications the
approximations as fit in the two domains are much more nearly consistent,
as is demonstrated in Figure~\ref{fig:compare2}.

\section{Conclusions}
The double-elliptic approximation introduced by \cite{Muir.sep.67.41}
can approximate the kinematics of TI media quite accurately, although
it is forced to cut off triplications when fit in the impulse-response
domain.  In the dispersion-relation domain it remains accurate even
for triplicating media.

Our analysis in this paper was limited to simple kinematics.
Before pronouncing the double-elliptic approximation
a success, we need to also demonstrate how it
works when used as the basis for an imaging technique;
i.e, how accurately does it model the {\em dynamics\/} of the wave 
equation? \cite{Karrenbach.sep.70.123} examines this question in a companion paper.

\plot{compare}{width=5.6in}
{
Two different double-elliptic approximations (dashed curves) fit
to the {\sl q}SV mode of Greenhorn Shale (solid curves).
Left: the approximation is fit in the impulse-response domain, and
so the dashed curve has a simple analytic form.
Right: the approximation is fit in the dispersion-relation domain, and
so is able to closely follow the triplication. This approximating
curve can only be calculated parametrically, however, and so is less useful.
}
\plot{compare2}{width=5.6in}
{
Two different double-elliptic approximations (dashed curves) corresponding
to those in Figure~\protect\ref{fig:compare}, but this time fit
to the {\sl q}P mode of Greenhorn Shale (solid curves). (The size of the
``$*$'' in the middle shows the relative scales.)
The discrepancy is much less since there are no troublesome triplications.
}

\bibliographystyle{seg}
\bibliography{SEP2}

