%\shortnote
%\keywords{random lines}

\lefthead{Claerbout}
\righthead{Random lines in a plane}
\footer{SEP--103}

\title{Random lines in a plane}

\email{claerbout@stanford.edu}
\author{Jon Claerbout}
\def\eq{\quad =\quad}

%\ABS{ We throw random lines onto a plane.  }

\maketitle

\section{INTRODUCTION}

Locally, seismic data is a superposition of plane waves.
The statistical properties of such superpositions
are relevant to geophysical estimation
and they are not entirely obvious.

\par
Clearly, a planar wave can be constructed from a planar
distribution of point sources.
Contrariwise, a point source can be constructed from
a superposition of plane waves going in all directions.
We can represent a random wave source either as a superposition
of points or as a superposition of plane waves.
Here is the question:\\

\boxit{
        Given a superposition of infinitely many impulsive plane waves
        of random amplitudes and orientations,
        what is their spectrum?
        }

\inputdir{randline}

If you said the spectrum is white, you guessed wrong.
Figure \ref{fig:lines100,lines10000,randpt} shows that it does not
even look white.
\multiplot{3}{lines100,lines10000,randpt}{width=0.45\textwidth}{
        Top shows a superposition of 100 randomly positioned lines.
        Middle shows a superposition of 10,000 such lines.
        Bottom shows a superposition of 10,000 random point values.
        The bottom panel shows the most high frequency.
        Its spectrum is theoretically white.
        This paper claims that the middle panel
        is more representative of natural noises.
        }
It is lower frequency than white for good reason.
Mathematically independent variables
are not necessarily statistically independent variables.

\section{RESOLUTION OF THE PARADOX}

If we throw impulse functions randomly onto a plane,
the power spectrum of the plane
is the power spectrum of impulse functions, namely white.
\par
Think of a 2-D Gaussian whose contour of half-amplitude describes
an ellipse of great eccentricity.
In the limit of large eccentricity,
this Gaussian could be one of the lines that we sprinkle on the plane
with random amplitudes and orientations.
The spatial spectrum of such an eccentric Gaussian
must be lower than that of a symmetrical point Gaussian
because the spectrum along the long axis of the ellipsoid
is concentrated at very low frequency.

\par
Consider a single delta function along a line with
an arbitrary slope and location in a plane.
The autocorrelation of this dipping line
is another dipping line with the same slope,
but passing through the origin at zero lag.
The polarity of the impulse function is lost in the autocorrelation;
in the autocorrelation space,
the amplitude of the dipping line is positive.

\par
Now consider a superposition of many dipping lines on the plane.
Its autocorrelation is the sum of the autocorrelations of individual lines.
The autocorrelation of any individual line is a line of 
the same slope that is translated to pass through the origin.
[The 2-D autocorrelation is not shown in the graphics here.
You'll need to understand it from the words here.  Sorry.]
The autocorrelation is a superposition of lines of various slopes
all passing through the origin, all having positive amplitude.
This function would resemble a positive impulse function at the origin
(and hence suggest a white spectrum).
The function is actually not an impulse function,
but, as we'll see, it is the pole $1/r$.

\par
Consider an integral on a circular path around the origin.
The circle crosses each line exactly twice.
Thus the integral on this circular path
is independent of the radius of the circle.
Hence the average amplitude on the circumference
is inverse with the circumference to keep the integral constant.
Thus the autocorrelation function is the pole $1/r$.

\section{FOURIER TRANSFORM OF 1/r}
We would like to know the 2-D Fourier transform
of $1/r$.
Everywhere I found tables of 1-D Fourier transforms but only
one place did I find a table that included this 2-D Fourier transform.
It was at
        \url{http://www.ph.tn.tudelft.nl/Courses/FIP/noframes/fip-Statisti.html}
\par
Sergey Fomel showed me how to work it out:
Express the FT in radial coordinates:
\def\FT{{\rm FT}\left({1\over r}\right)}
\begin{eqnarray}
\FT &=& \int \int \exp[i k_x r \cos\theta + i k_y r \sin\theta]\
        {1\over r}\ r \ dr \ d\theta   \\
\FT &=& \int \delta[k_x \cos\theta + k_y \sin\theta]\ d\theta
\end{eqnarray}
        To evaluate the integral, we use the fact that
        $\int\delta(f(x))dx = 1/|f'(x_0)|$ where $x_0$ is defined by $f(x)=0$
        and the definition
        $ \theta_0 =\arctan (-k_x/k_y)$.
\begin{eqnarray}
\FT &=&
{1\over |-k_x \sin\theta_0 + k_y \cos\theta_0|}
\\
\FT &=&
{1\over\sqrt{k_x^2 + k_y^2}} \ =\  {1\over k_r} 
\end{eqnarray}





\section{UTILITY OF THIS RESULT}
At present we are accustomed to estimating
statistical properties of seismic data
by computing a 2-D prediction-error filter.
This filter is needed to interpolate and extrapolate missing values.

\par
Knowing that the prior spectral estimate is not a constant
but instead is $1/k_r$ suggests a procedure
that is
more efficient
statistically:
By more efficient,
I mean that a simpler model should fit the data,
a model with fewer adjustable parameters.

\section{INTERPOLATION}
\par
We'll need to know a wavelet in the time and space domain
whose amplitude spectrum is
$\sqrt{k_r}$
(so its power spectrum is $k_r$).
Do not mistake this for the the helix derivative \cite[]{geofizhelix}
whose power spectrum is $k_r^2$.
What we need to use here is the square root of the helix derivative.
Let the (unknown) wavelet with 
amplitude spectrum
$\sqrt{k_r}$
be known as $G$.


\begin{enumerate}
\item
        Apply $G$ to the data.
        The prior spectrum of the modified data is now white.
\item Estimate the PEF of the modified data.
\item The interpolation filter for the original data
        is now $G$ times the PEF of the modified data.
\end{enumerate}

Why is this more efficient?
The important point is that the PEF should estimate
the minimal practical number of freely adjustable parameters.
If $G$ is a function that is lengthy in time or space,
then the PEF does not need to be.

\par
How important is this extra statistical efficiency?
I don't know.

\section{WHAT ARE THE NEXT STEPS?}

\begin{itemize}
\item Compute $\sqrt{k_r}$ in physical space and look at it.
How best to do this?
How best to package the software?
\item Invent a synthetic data test.
\item Think about how it might impact Sean (or Sergey).
        Which example of Sean's would be worth redoing?
\item Extend this idea to 3-D (lettuce versus noodles).
\item Matt Schwab and I were frequently disappointed in the performance
of local PEFs for the task of visualizing data.
This might explain it.
Which example of his might be worth redoing?
\end{itemize}

\bibliographystyle{seg}
\bibliography{jon}

