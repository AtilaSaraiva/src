\def\figdir{./Fig} 
\lefthead{Rickett}
\righthead{Traveltime sensitivity kernels}
\footer{SEP--103} 

\title{Traveltime sensitivity kernels: Banana-doughnuts \\ or just
plain bananas?}

\email{james@sep.stanford.edu}
\author{James Rickett}

\begin{abstract}
To model the first-order effects of finite-frequency in
traveltime tomography, I compute three-dimensional traveltime
sensitivity functions using the first Born and first Rytov
approximations. 
As previous authors have noted, numerical sensitivity kernels computed
with the first Born approximation have the appearance of a {\em hollow
banana}: that is appearing as a banana if visualized in the plane of
propagation, but as a doughnut on a cross-section perpendicular to the
ray.  Somewhat counter-intuitively, this suggests that traveltimes
have zero sensitivity to small velocity perturbations along the
geometric raypath.  
\end{abstract}

\section{Introduction}
Estimating an accurate velocity function is one of the most critical
steps in building an accurate seismic depth image of the subsurface.
In areas with significant structural complexity, one-dimensional
updating schemes become unstable, and more robust algorithms are
needed.
Reflection tomography both in the premigrated \cite[]{bishop85} and
postmigrated domains \cite[]{stork92,kosloff96} bring the powerful
technologies of geophysical inversion theory
to bear on the problem. 

\par
Unfortunately, however, inversion methods can be limited by the
accuracy of their forward modeling operators, and most practical 
implementations of traveltime tomography are based on ray-theory,
which assumes a high frequency wave, propagating through a smoothly
varying velocity field, perhaps interrupted with a few discrete
interfaces. 
Real world wave-propagation is much more complicated than this, and 
the failure of ray-based methods to adequately model wave propagation
through complex media is fueling interest in ``wave-equation''
migration algorithms that both accurately model finite-frequency
effects, and are practical for large 3-D datasets.
As a direct consequence, finite-frequency velocity analysis and
tomography algorithms are also becoming an important area of research 
\cite[]{woodward92,biondi.segab99}.

\par
Recent work in the global seismology community
\cite[]{marquering98,marquering99} is drawing attention to a
non-intuitive observation first made by \cite{woodward92}, that in
the weak-scattering limit, finite-frequency traveltimes have
zero-sensitivity to velocity perturbations along the geometric
ray-path. 
This short-note aims to explore and explain this non-intuitive
observation. 

\section{Theory}
A generic discrete linear inverse problem may be written as
\begin{equation}
{\bf d} = {\bf A \; m}
\end{equation}
where ${\bf d}=(d_1 \; d_2 \;...)^T$ is the known data vector, 
${\bf m}=(m_1 \; m_2 \;...)^T$
is the unknown model vector, and ${\bf A}$ represents the linear
relationship between them.
A natural question to ask is: which parts of the
model influence a given observed data-point?
The answer is that the row of matrix, ${\bf A}$,
corresponding to the data-point of interest will be non-zero
where that point in model space influences the data-value.
Rows of ${\bf A}$ may therefore be thought of as sensitivity kernels,
describing which points in model space are sensed by a given
data-point. 

\par
For a generic linearized traveltime tomography problem, traveltime
perturbations, ${\bf \delta T}$, are related to slowness
perturbations, ${\bf \delta S}$, through a linear system,
\begin{equation}
{\bf \delta T} = {\bf A \; \delta S}.
\end{equation}
The form of the sensitivity kernels depend on the the modeling
operator, ${\bf A}$.

\par
Under the ray-approximation, traveltime for a given ray, $T$, is
calculated by integrating slowness along the ray-path,
\begin{equation}\label{eqn:generictomo}
T=\int_{\rm ray} s({\bf x}) \;dl.
\end{equation}
Assuming that the ray-path is insensitive to a small slowness
perturbation, the perturbation in traveltime is
given by the path integral of the slowness perturbation along the ray, 
\begin{equation}\label{eqn:tomoray}
\delta T=\int_{\rm ray} \delta s({\bf x}) \;dl.
\end{equation}

\par
Since traveltime perturbations given by equation~(\ref{eqn:tomoray})
are insensitive to slowness perturbations anywhere off the geometric
ray-path, the sensitivity kernel is identically zero everywhere in
space, except along the ray-path where it is constant.
The implication for ray-based traveltime tomography is that traveltime
perturbations should be back-projected purely along the ray-path.

\par
We are interested in more accurate tomographic systems of the form 
of equation~(\ref{eqn:generictomo}), that model the effects of
finite-frequency wave-propagation more accurately than simple
ray-theory. 
Once we have such an operator, the first question to ask is: what
do the rows look like?

\subsection{Born traveltime sensitivity}
One approach to building a linear finite-frequency traveltime operator
is to apply the first-order Born approximation, to obtain a linear
relationship between slowness perturbation, $\delta S$, and wavefield
perturbation, $\delta U$,
\begin{equation}
{\bf \delta U} = {\bf B \; \delta S}.
\end{equation}
The Born operator, ${\bf B}$, is a discrete implementation of
equation~(\ref{eqn:born}), which is described in the Appendix. 

\par
Traveltime perturbations may then be calculated from the wavefield
perturbation through a (linear) picking operator, ${\bf C}$, such that 
\begin{equation}
{\bf \delta T} = {\bf C \; \delta U} = {\bf C B \; \delta S}
\end{equation}
where ${\bf C}$ is a (linearized) picking operator, and a function
of the background wavefield,~$U_0$. 

\par
Cross-correlating the total wavefield, $U(t)$, with $U_0(t)$, provides
a way of measuring their relative 
time-shift, $\delta T$.  \cite{marquering99} uses
this to provide the following explicit linear relationship between
$\delta T$ and $\delta U(t)$,
\begin{equation} \label{eqn:timeshift}
\delta T = 
\frac{\int_{t_1}^{t_2} {\dot U}(t) \; \delta U(t) \; dt}
{\int_{t_1}^{t_2} {\ddot U}(t) \; U(t) \; dt},
\end{equation}
where dots denote differentiation with respect to $t$, and $t_1$ and
$t_2$ define a temporal window around the event of interest. 
Equation~(\ref{eqn:timeshift}) is only valid for small time-shifts,
$\delta T \ll \lambda s_0$.

\subsection{Rytov traveltime sensitivity}
The first Rytov approximation (or the phase-field linearization
method, as it is also known) provides a linear relationship between
the slowness and complex phase perturbations.
\begin{equation}
{\bf \delta \Psi} = {\bf R \; \delta S},
\end{equation}
where ${\bf \Psi}=\exp({\bf U})$, and the Rytov operator, ${\bf R}$,
is a discrete implementation of equation~(\ref{eqn:rytov}), which is
also described in Appendix~A.  

\par
Traveltime is related to the complex phase by the equation,
$\Im (\delta \psi)=\omega \, \delta t$.
For a band-limited arrival with amplitude spectrum, $F(\omega)$,
traveltime perturbation can be calculated simply by summing over
frequency \cite[]{woodward92}, 
\begin{equation}
{\bf \delta T} = \sum_\omega \frac{F(\omega)}{\omega} \;
\Im \left({\bf \delta \Psi}\right) = 
\sum_\omega \frac{F(\omega)}{\omega} 
\;\Im \left( {\bf R \; \delta S}\right).
\end{equation}

\par
Of the two approximations, several authors
\cite[]{beydoun88,Woodward.sep.60.203} note that the Born approximation 
is the better choice for modeling reflected waves, while the Rytov
approximation is better for transmitted waves.  Differences tend to
zero, however, as the scattering becomes small.

\section{Kernels compared}
\inputdir{kernel}

This section contains images of traveltime kernels computed
numerically for a simple model that may be encountered in a reflection
tomography problem.  The source is situated at the surface, and the
receiver (known reflection point) is located at a depth of 1.8~km in
the subsurface.  The background velocity model, $v_0(z)=1/s_0(z)$, is
a linear function of depth with $v_0(0)=1.5 \;{\rm km\, s^{-1}}$, and
$\frac{dv_0}{dz}= 0.8 \; s^{-1}$.  I chose a linear velocity function
since Green's functions can be computed on-the-fly with rapid two-point
ray-tracing.  

\par
Figure~\ref{fig:RayKernel} shows the ray-theoretical traveltime
sensitivity kernel: zero except along the geometric ray-path.  
\plot{RayKernel}{width=5in}{Traveltime sensitivity kernel for
ray-based tomography in a linear $v(z)$ model.  
The kernel is zero everywhere {\em except} along geometric ray-path. 
Right panel shows a cross-section at $X=1$~km.}

\par
Figures~\ref{fig:BananaPancake8} and~\ref{fig:BananaPancake2} show
first Rytov traveltime sensitivity kernels for 30~Hz and 120~Hz
wavelets respectively.  The important features of these kernels are
identical to the features of kernels that \cite{marquering99}
obtained for teleseismic $S-H$ wave scattering, and to Woodward's
band-limited wave-paths \cite[]{woodward92}.  They have the
appearance of a  hollow banana: that is appearing as a banana if
visualized in the plane of propagation, but as a doughnut on a
cross-section perpendicular to the ray.  
Somewhat counter-intuitively, this suggests that traveltimes
have zero sensitivity to small velocity perturbations along the
geometric raypath.  
Fortunately, however, as the frequency of the seismic wavelet increases,
the bananas become thinner, and approach the ray-theoretical kernels
in the high-frequency limit.
Parenthetically, it is also worth noticing that the width of the
bananas increases with depth as the velocity (and seismic wavelength)
increases. 
\plot{BananaPancake8}{width=5in}{Rytov traveltime sensitivity
kernel for 30~Hz wavelet in a linear $v(z)$ model.  
The kernel is zero along geometric ray-path.
Right panel shows a cross-section at $X=1$~km.}
\plot{BananaPancake2}{width=5in}{Rytov traveltime sensitivity
kernel for 120~Hz wavelet in a linear $v(z)$ model.  
The kernel is zero along geometric ray-path.
Right panel shows a cross-section at $X=1$~km.} 

\par
I do not show the first-Born kernels here, since, in appearance, they 
are identical to the Rytov kernels shown in 
Figures~\ref{fig:BananaPancake8} and~\ref{fig:BananaPancake2}.

\section{The Banana-doughnut paradox}
The important paradox is not the apparent contradiction between
ray-theoretical and finite-frequency sensitivity kernels, 
since they are compatible in the high-frequency limit. 
Instead, the paradox is how do you reconcile the zero-sensitivity
along the ray-path with your intuitive understanding of wave
propagation? 

\par
A first potential resolution to the paradox is that the wavefront
healing removes any effects of a slowness perturbation.  
This alone is a somewhat unsatisfactory explanation since it does not
explain why traveltimes are sensitive to slowness perturbations just
off the geometric ray-path.

\par
A second potential resolution is that the hollowness of the banana is
simply an artifact of modeling procedure.  This is partially true.
Both Born and Rytov are single scattering approximations, and a single
scatterer located on the geometric ray-path may only contribute energy
in-phase with the direct arrival.  In contrast, if there are two
scatterers on the geometric ray-path traveltimes may be affected.
However, just because the paradox may appear to be an artifact of the 
modeling procedure does not mean it is not a real phenomenom. 
In the weak scattering limit, traveltimes will indeed be insensitive
to a slowness perturbation situated on the geometric ray-path.

\section{Conclusions: does it matter?}
Practitioners of traveltime tomography typically understand the
shortcomings of ray-theory; although they realize using ``fat-rays''
would be better, they smooth the slowness model both explicitly and by 
regularizing during the inversion procedure. 
In practice, any shortcomings of traveltime tomography are unlikely to
be caused by whether or not the fat-rays are hollow. 

\par
However, the null space of seismic tomography problems is typically
huge.  Smoothing and regularization are often done with very ad 
hoc procedures.  Understanding the effects of finite-frequency through
sensitivity kernels may lead to incorporating more physics during the
regularization and improve tomography results.

\bibliographystyle{seg}  
\bibliography{SEP2,banana}

\append{Born/Rytov review}
Modeling with the first-order Born (and Rytov) approximations 
[e.g. \cite{beydoun88}] can be
justified by the assumption that slowness heterogeneity in the earth
exists on two separate scales: a smoothly-varying background,
$s_0$, within which the ray-approximation is valid, 
and weak higher-frequency perturbations, $\delta s$, that
act to scatter the wavefield.  
The total slowness is given by the sum,
\begin{equation}
s({\bf x})=s_0({\bf x})+\delta s({\bf x}).
\end{equation}

\par
Similarly, the total wavefield, $U$, can be considered 
as the sum of a background wavefield, $U_0$, and a
scattered field, $\delta U$, so that 
\begin{equation}
U({\bf x},\omega)=U_0({\bf x},\omega)+\delta U({\bf x},\omega),
\end{equation}
where $U_0$ satisfies the Helmholtz equation in the background
medium,
\begin{equation}
\left[ \nabla^2 + \omega^2 \, s_0^2({\bf x})\right]
U_0({\bf x},\omega) = 0,
\end{equation}
and the scattered wavefield is given by the (exact) non-linear
integral equation \cite[]{morsefeshbach}, 
\begin{equation}\label{eqn:morse}
\delta U({\bf x},\omega)=\frac{\omega^2}{4 \pi}
\int_V G_0({\bf x},\omega; {\bf x}')\,U({\bf x},\omega; {\bf x}')
\, \delta s({\bf x}') \; dV({\bf x}').
\end{equation}
In equation~(\ref{eqn:morse}), $G_0$ is the Green's function for
the Helmholtz equation in the background medium:  
i.e. it is a solution of the equation
\begin{equation}
\left[ \nabla^2 + \omega^2 \, s_0^2({\bf x})
\right] G_0({\bf x},\omega ; {\bf x}_s) = -4 \pi \delta({\bf x} - 
{\bf x}_s). 
\end{equation}
Since the background medium is smooth, in this paper I use Green's
functions of the form,
\begin{equation}
G_0({\bf x},\omega ; {\bf x}_s) = A_0({\bf x},{\bf x}_s)
e^{i \omega T_0({\bf x},{\bf x}_s)}.
\end{equation}
where $A_0$ and $T_0$ are ray-traced traveltimes and amplitudes
respectively.

\par
A Taylor series expansion of $U$ about $U_0$ for small $\delta s$,
results in the infinite Born series, which is a Neumann series
solution \cite[]{arfken} to equation~(\ref{eqn:morse}).
The first term in the expansion is given below: it corresponds
to the component of wavefield that interacts with scatters only once.
\begin{equation}\label{eqn:born}
\delta U_{\rm Born}({\bf x},\omega)=\frac{\omega^2}{4 \pi}
\int_V G_0({\bf x},\omega; {\bf x}')\,U_0({\bf x},\omega; {\bf x}')
\delta s({\bf x}') dV({\bf x}').
\end{equation}
The approximation implied by equation~(\ref{eqn:born}) is known as the
first-order Born approximation. It provides a linear relationship
between $\delta U$ and $\delta s$, and it can be computed more easily
than the full solution to equation~(\ref{eqn:morse}). 

\par
The Rytov formalism starts by assuming the heterogeneity perturbs the 
phase of the scattered wavefield.  The total field, $U=\exp (\psi)$,
is therefore given by
\begin{equation}
U({\bf x},\omega)=U_0({\bf x},\omega) \exp(\delta \psi) =
\exp(\psi_0+\delta \psi).
\end{equation}
The linearization based on small $\delta \psi / \psi$ leads to the
infinite Rytov series, on which the first term is given by
\begin{eqnarray} 
\delta \psi_{\rm Rytov} ({\bf x},\omega) & = &
\frac{\delta U_{\rm Born}({\bf x},\omega)}{U_0({\bf x},\omega)} \\ 
& = & \frac{\omega^2}{4 \pi \, U_0({\bf x},\omega)}
\int_V G_0({\bf x},\omega; {\bf x}')\,U_0({\bf x},\omega; {\bf x}')
\delta s({\bf x}') dV({\bf x}'). \label{eqn:rytov} 
\end{eqnarray}

\par
The approximation implied by equation~(\ref{eqn:rytov}) is known as
the first-order Rytov approximation. It provides a linear relationship
between $\delta \psi$ and $\delta s$.
