\lefthead{ Brown }
\righthead{ Imaging with Multiples }
\footer{SEP--111}

\title{Least-squares joint imaging of primaries and multiples}
\email{morgan@sep.stanford.edu}
\author{Morgan Brown}

\ABS{ Multiple reflections provide redundant, and sometimes additional, information
	about the corresponding primary reflections.  I implement a least-squares inversion 
	scheme to jointly image (by normal moveout) primaries and multiples, with the goal of 
	enforcing consistency between the images and the input data.  Furthermore, to effect 
	noise (``crosstalk'') suppression, I introduce a novel form of
	model regularization which exploits kinematic similarities between imaged primaries
	and multiples, and which also preserves the amplitude-versus-offset (AVO) response of
	the data.  In tests on synthetic data, my approach exhibits good noise
	suppression and signal preservation characteristics.  Real data tests highlight
	the need for careful data preprocessing.  Future work points toward use
	of migration as the imaging operators, to exploit cases where multiples actually
	exhibit better angular coverage than primaries, and thus add new information to the
	inversion.
}

\section{introduction}
Multiple reflections have long been treated as noise in the seismic imaging process.
In contrast to many other types of ``noise'', like surface waves, multiply reflected 
body waves may still penetrate deep into the earth, and thus have a potential to 
aid in imaging the prospect zone.  I refer generically to {\em joint imaging with multiples} 
as any process which creates a ``pseudo-primary'' image from multiples by removing the 
propogation effects of body waves through arbitrary multiple layer (generator + free 
surface), and which then seeks to integrate the information provided by the primary 
and pseudo-primary images.
\par
\longcite{GEO56-07-10811086} present an early example of imaging multiples directly 
using a prestack Kirchhoff scheme.  \longcite{SEG-2001-04560459} describe a 
cross-correlation method for imaging multiples.  \longcite{SEG-1994-1497} and
\longcite{Guitton.sep.111.antoine3} apply shot-profile migration for multiples.
The aforementioned approaches produce separate-but-complementary pseudo-primary and primary 
images, yet they either do not attempt to, or employ simplistic methods to integrate the information
contained in the two images; either add \cite{GEO56-07-10811086} or multiply \cite{SEG-2001-04560459} 
them together.
%\longcite{GEO66-01-02460255} present a novel migration scheme which solves the two-way elastic 
%wave equation and simultaneously images all primaries and multiples, though by their own 
%admission the computer memory requirements of their scheme are staggering, even by today's
%standards.
\par
In this paper, I introduce a new methodology for jointly imaging primaries and multiples.
In addition to a desire to correctly image the multiples, my approach is driven by three 
primary motivations:
\begin{enumerate}
	\item {\bf Data Consistency} - The primary and pseudo-primary images both should be
		maximally consistent with the input data.
	\item {\bf Self-consistency} - The primary and pseudo-primary images should be consistent
		with one another, both kinematically and in terms of amplitudes. 
	\item {\bf Noise Suppression} - In the primary image, all orders of multiples should be
		suppressed.  In the pseudo-primary image created from, say first-order water-bottom 
		multiples, contributions from primaries and secord-order or greater multiples should be 
		suppressed.
\end{enumerate}
Least squares optimization provides an excellent, and perhaps the only viable approach to 
address all three requirements.  
I adopt an approach similar to \longcite{GEO64-01-02080221}, which used a least-squares scheme 
to jointly image compressional and surface waves, for improved wavefield separation.  
Data consistency is effected by minimization of a data residual; self-consistency and noise 
suppression through the use of regularization terms which penalize 1) differences between 
primary and pseudo-primary images, and 2) attributes which are not characteristic to 
true primaries or pseudo-primaries.
\par
In my approach, I use the simplest possible imaging operation, Normal Moveout (NMO).  I 
derive an NMO equation for water-bottom multiple reflections, which maps these multiples
to the same zero-offset traveltime as their associated primaries, creating a ``pseudo-primary''
section.  To account for the amplitude differences between the primary and pseudo-primary
sections, I assume constant seafloor AVO behavior and estimate a single water-bottom 
reflection coefficient from the data.  To address the AVO differences between primary
and pseudo-primary, I derive an expression -- valid only for constant velocity -- for the
AVO of the pseudo-primary as a function of the AVO of the primary, and then enforce
this constraint in the inversion via an offset- and time-dependent regularization term.

\section{methodology}
\subsection{NMO for Multiple Reflections}
\inputdir{XFig}

In a horizontally-stratified, $v(z)$ medium, multiple reflections can be treated as 
kinematically-equivalent primaries with the same source-receiver spacing but additional 
zero-offset traveltime $\tau^*$, as illustrated in Figure \ref{fig:schem}.  We can write 
an extension to the NMO equation which flattens multiples to the zero-offset traveltime 
of the reflector of interest.
\begin{equation}
	t^2 = \sqrt{ (\tau+j\tau^{*})^2 + \frac{x^2}{V_{eff}^2} }  \label{eq:nmomult}
\end{equation}
$j\tau^{*}$ is the two-way traveltime of a $j^{\mbox{th}}$-order multiple in the
top layer.  $V_{eff}(\tau)$ is the effective RMS velocity of the 
equivalent primary shown in the figure.  For the simple case of constant velocity $v^*$ 
in the multiple-generating layer,
\begin{equation}
	V_{eff}(\tau) = \frac{ \tau^{*} v^{*} + \tau V(\tau) }{ \tau^{*} + \tau } \label{eq:veff}
\end{equation}
So for the common case of relatively flat reflectors, $v(z)$, and short offsets, equation 
(\ref{eq:nmomult}) should do a reasonable job of flattening water-bottom multiples of any 
order to the $\tau$ of interest, assuming that we pick the water bottom ($\tau^{*}$) and 
that we know the seismic velocity of water.

\sideplot{schem}{width=4.0in,height=2.0in}{ Schematic for NMO of multiples.  From the
	standpoint of NMO, multiples can be treated as pseudo-primaries with the same source-receiver
	spacing, but with extra zero-offset traveltime $\tau^*$, assuming that the velocity and 
	time-thickness of the multiple layer are known.
 }

\subsection{AVO of Multiple Reflections}
Even after application of the water-bottom reflection coefficient, the AVO response of the
pseudo-primary section created by equation (\ref{eq:nmomult}) does not match that of the
corresponding NMO-corrected primary section.  Refer to Figure \ref{fig:avo} and note that 
for constant-AVO water-bottom reflection (and a free surface reflection coefficient of
-1), the amplitude of the water-bottom multiple at 
offset $h_p+h_m$ is simply the amplitude of the primary at offset $h_p$, scaled by the 
negative water-bottom reflection coefficient.  Still, the question remains: {\em what are $h_m$
and $h_p$}?  For the case of constant velocity, we can use trigonometry to derive $h_m$
and $h_p$ as a function of the zero offset traveltimes of the primary reflection and
water bottom ($\tau$ and $\tau^*$, respectively), and the source-receiver offset $x$.
In constant velocity, the multiple and primary legs of the raypath are similar triangles:
\begin{equation}
	\frac{\tau v}{h_p} = \frac{\tau^* v}{h_m}. \label{eq:triangles}
\end{equation}
Also, for a first-order water-bottom multiple,
\[ h_p + h_m = x. \]
These two independent equations can be solved and simplified to give expressions for 
$h_p$ and $h_m$:
\begin{equation}
	h_p = \frac{\tau}{\tau+\tau^*}x \hspace{0.2in} \mbox{ and } \hspace{0.2in}
	h_m = \frac{\tau^*}{\tau+\tau^*}x. \label{eq:avo}
\end{equation}
I omit the general form of the expression for orders of multiple higher than one, although 
it is straightforward to derive.

\sideplot{avo}{width=3.0in,height=2.5in}{ Assuming a constant AVO water-bottom reflection
	and constant velocity, we can write the AVO of water-bottom multiples with offset $h_p+h_m$ 
	as a function of the AVO of the primary recorded at a shorter offset, $h_p$.  }
\par
To obtain an estimate of the water-bottom reflection coefficient, I solve a simple least squares
problem to estimate a function of location, $\bold a(x)$, which when applied to a small window of 
dimension $nt \times nx$
around the NMO-corrected water-bottom reflection, $\bold p(t,x)$, optimally resembles the NMO-corrected 
[equation (\ref{eq:nmomult})] first-order water-bottom multiple reflection, $\bold m(t,x)$.  To achieve 
this, $\bold a(x)$ is perturbed to minimize the following quadratic functional.
\begin{equation}
	\mbox{min} \; \left( \sum_{j=1}^{nx} \sum_{i=1}^{nt} 
		a(j)*p(i,j) - m(i,j) \right)^2 \label{eq:refl-coeff}
\end{equation}
$\bold a(x)$ may not be reliable at far offsets, due to either NMO stretch or non-hyperbolicity,
so in practice, an estimate of the single best-fitting water-bottom reflection coefficient 
is made using the $\bold a(x)$ from ``useful'' offsets only.

\subsection{Least-squares imaging of multiples}
Applied to a common-midpoint gather, equation (\ref{eq:nmomult}) produces an approximate unstacked
zero-offset image of pseudo-primaries from water bottom multiple reflections.  
In this section, I introduce a least squares scheme to compute self-consistent images of 
primaries and pseudo-primaries which are in turn consistent with the data.  First I define 
some terms:
\begin{eqnarray*}
	\bold d   & \leftrightarrow & \mbox{ One CMP gather. } \\
	\bold m_j & \leftrightarrow & \mbox{ Prestack model vector for multiple order $j$.
		Produced by applying equation (\ref{eq:nmomult}) to $\bf d$.  } \\
	\bold N_j & \leftrightarrow & \mbox{ {\em Adjoint} of NMO for multiple of order $j$
		(primaries: $j=0$). $\bold N_j^T$ applies equation (\ref{eq:nmomult}) to $\bf d$. } \\
	\bold R_j & \leftrightarrow & \mbox{ Given a single water-bottom reflection coefficient, $r$,
		[estimated via equation (\ref{eq:refl-coeff})],} \\ 
	          &                 & \mbox{ this operator scales $\bold m_j$
		by $1/r^j$ to make the amplitudes of all the $\bold m_j$ comparable.  } \\
\end{eqnarray*}
With these definitions in hand, we can now write the forward modeling operator for joint 
NMO of primaries and multiples of order 1 to $p$.
\begin{equation}
	\left[ \bold N_0 \ \ \bold N_1 \bold R_1 \ \cdots \ \bold N_p \bold R_p \right]
	\left[
	\begin{array}{c}
		\bold m_0 \\
		\bold m_1 \\
		\vdots    \\
		\bold m_p 
	\end{array}
	\right] = \bf Lm          \label{eq:jointnmo0}
\end{equation}
In words, equation (\ref{eq:jointnmo0}) takes a collection of psuedo-primary panels, divides
each by the appropriate reflection coefficient, applies inverse (adjoint) NMO to each, and
then sums them together to create something that should resemble ``data''.
We define the data residual as the difference between the input data and the 
forward-modeled data:
\begin{equation}
	\bold r_d = \bf d - Lm       \label{eq:data-resid}
\end{equation}
Viewed as a standard least-squares inversion problem, minimization of $L_2$ norm of the 
data residual by solution of the normal equations is underdetermined.  Additional regularization
terms, defined in later sections, force the problem to be overdetermined.

\subsection{Consistency of the Data and the Crosstalk Problem}
Figure \ref{fig:schem.hask} shows the result of applying the adjoint of equation 
(\ref{eq:jointnmo0}) to a synthetic CMP gather which was constructed by an elastic modeling
scheme.  Imagine for a moment that the CMP gather consists {\em only} of primaries and
first- and second-order water-bottom multiples.   The ``NMO for Primaries'' panel would
contain flattened primaries (signal) and downward-curving first- and second-order 
multiples (noise).  Likewise, the ``NMO for multiple 1'' and ``NMO for multiple 2''
panels contain flattened signal and curving noise.  Why do I call these
components ``signal'' and ``noise''?  If each of the three panels contained all signal and 
no noise, then we could 1) perfectly reconstruct the data from the model by applying 
equation (\ref{eq:jointnmo0}), and 2) be in the enviable position of having a perfect 
estimate of the primaries.  
\plot{schem.hask}{width=5.5in,height=3.5in}{From left to right: Raw synthetic CMP gather;
	Conventional NMO applied to data;  NMO for first-order water-bottom multiple; NMO for 
	second-order water-bottom multiples.
}
\par
Unfortunately, the curved events -- so-called ``crosstalk'' -- in all three model panels 
spoil this idealized situation \cite{Claerbout.blackwell.92}.  Because the crosstalk events 
map back to actual events in the data, they are difficult to suppress in a least-squares 
minimization of the data residual [equation (\ref{eq:data-resid})].  \longcite{GEO64-01-02080221}
shows that crosstalk relates directly to non-invertibility of the Hessian ($\bold L^T \bold L$), 
and that data-space or model-space regularization may partially overcome the difficulty.
In the following section, I introduce a novel form of model-space regularization which 
promotes discrimination of signal from crosstalk.

\subsection{Regularization of the Least-Squares Problem}
Visual inspection of Figure \ref{fig:schem.hask} motivates the two forms of 
regularization utilized in this paper.  Find any first-order multiple on the section marked 
``NMO for Primaries''.  Notice that the corresponding event on the first- and second-order 
pseudo-primary panels, originally second- and third-order multiples, respectively, all have 
a different moveout.  In fact, the only events which are kinematically consistent across all 
offsets are the flattened primary and pseudo-primaries.  The other events, all crosstalk, are
inconsistent between panels.  Therefore, the first regularization 
operator seeks to penalize the difference between the $\bold m_i$, at fixed $\tau$.  To account
for the dissimilarity of the AVO of primaries and multiples, this difference is taken at 
different offsets, as defined in equation (\ref{eq:avo}).  Written in the form of a model 
residual vector, this difference is:
\begin{equation}
	\bold r_m^{[1]}(\tau,x,i) = m_i(\tau,h_p) - m_{i+1}(\tau,x). \label{eq:order-reg}
\end{equation}
\par
The third index in equation (\ref{eq:offset-reg}), $i$, ranges from 0 to $n_p-1$, where
$n_p$ is the highest order multiple modeled in the inversion [see equation (\ref{eq:jointnmo0})].
\par
The second form of regularization used in this paper is the more obvious of the two:
a difference operator along offset.  This difference exploits the fact that all 
non-primaries are not flat after NMO.  Again, we can write this difference in the form 
of a model residual vector:
\begin{equation}
	\bold r_m^{[2]}(\tau,x,i) = m_i(\tau,x) - m_i(\tau,x+\Delta x). \label{eq:offset-reg}
\end{equation}
The second regularization is applied to all the $\bold m_i$.  A similar approach is used 
by \longcite{Prucha.sep.108.marie1} to regularize prestack depth migration in the angle 
domain.

\subsection{Combined Data and Model Residuals}
To compute the optimal set of $\bold m_i$, a quadratic objective function, $Q(\bf m)$, 
consisting of sum of the weighted norms of a data residual [equation (\ref{eq:data-resid})] and 
of two model residuals [equations (\ref{eq:order-reg}) and (\ref{eq:offset-reg})], is 
minimized via a conjugate gradient scheme:  
\begin{equation}
	\mbox{min} Q(\bold m) \; = \; \| \bold r_d \|^2 \; + \; \epsilon_m^2 \| \bold r_m^{[1]} \|^2
	                       \; + \; \epsilon_x^2 \| \bold r_m^{[2]} \|^2  \label{eq:all-resid}
\end{equation}
$\epsilon_m$ and $\epsilon_x$ are scalars which balance the relative weight of the two model 
residuals with the data residual.

\section{Results}
\subsection{Testing the Raw Algorithm}
We begin with the results of testing the proposed algorithm on a single synthetic CMP gather,
which was shown previously in Figure \ref{fig:schem.hask}.  This gather, generated using 
Haskell-Thompson elastic modeling, with earth properties drawn from a well log provided with
the ``Mobil AVO'' dataset (see \cite{Lumley.sep.82.25} for a description of data details), 
has traditionally found use at SEP as a multiple suppression benchmark 
\cite{Lumley.sep.82.25,Nichols.sep.82.1,Clapp.sep.102.bob2,Guitton.sep.103.antoine1,Clapp.sep.103.bob2}.  
The data contain all surface-related and internal multiples, as well as P-to-S-to-P converted waves.
\par
Figure \ref{fig:cmps.lsrow.hask} illustrates the application of the proposed algorithm to the
so-called ``Haskell'' data.  Comparing the raw data and the estimated primary panels 
[$\bold m_0$ in equation (\ref{eq:jointnmo0})], we see that my algorithm does a decent job of 
suppressing the strongest multiples, especially at far offsets, though some residual multiple 
energy remains at the near offsets.  We expect poorer performance at near offsets; recall that 
the first regularization operator [equation (\ref{eq:order-reg})] penalizes dissimilarity of 
events across orders of multiple, yet all orders of multiple align at near offsets.  Moreover, 
the second regularization operator [equation (\ref{eq:offset-reg})] penalizes residual curvature, 
yet all events in the section, both primaries and the residual multiples, are flat at near offsets. 
\par
The difference panel shows little residual primary energy, which illustrates the favorable signal
preservation capability of my approach.  The bulk of the residual primary energy exists at far 
offsets and small times, where NMO stretch makes the primaries nonflat, and hence, vulnerable to
smoothing across offset by equation (\ref{eq:offset-reg}).
\par
The bottom three panels in Figure \ref{fig:cmps.lsrow.hask} show the data residual 
[equation (\ref{eq:data-resid})], and the first two panels of the two model residuals
[equations (\ref{eq:offset-reg}) and (\ref{eq:order-reg}), respectively].  Put simply, the data 
residual consists of events which are not modeled by equation (\ref{eq:jointnmo0}) -- hopefully,
 the multiples only.  The model residuals consist roughly of the portions of the model which 
were removed by the two regularization terms (again, hopefully multiples only): high-wavenumber 
events and events which are inconsistent from one panel to the next.  
\plot{cmps.lsrow.hask}{width=5.0in,height=8.0in}{ 
	Test of equation (\ref{eq:all-resid}) on Haskell synthetic CMP gather. 
	Top row, left to right: Raw Haskell data, NMO applied; Estimated primary panel; difference panel.  
	Bottom row, left to right: Data residual; first panel of model residuals, equations 
	(\ref{eq:offset-reg}) and (\ref{eq:order-reg}), respectively.
}

\subsection{Better Understanding the Regularization}
Figure \ref{fig:cmps.nograd.hask} illustrates the effect of setting $\epsilon_x=0$ in equation
(\ref{eq:all-resid}), which removes the influence of the regularization term which roughens
the model across offset [equation (\ref{eq:offset-reg})].  The results are intriguing.  Most
noticeably, leftover multiple reflections in the ``Estimated Primaries'' panel appear to be 
scrambled over offset, while primaries appear mostly intact.  Signal-to-noise ratio has increased
considerably.  The fact that the roughener across offset decorrelates the residual multiples should 
be further exploited.
\par
Notice that the model residual is zero at long offsets and small times.  This is due to the fact that
the difference is not taken across the same offsets, to account for the AVO multiples, according to
equation (\ref{eq:avo}).  When $h_p + h_m > h_{max}$, no difference is taken. 
\plot{cmps.nograd.hask}{width=5.0in,height=8.0in}{ 
	Only the regularization which roughens across orders of pseudo-primary, equation (\ref{eq:order-reg}),
	is used in the inversion.  
	Top row, left to right: Raw Haskell data, NMO applied; Estimated primary panel; difference panel.  
	Bottom row, left to right: Data residual; first panel of model residuals, equations 
	(\ref{eq:offset-reg}) and (\ref{eq:order-reg}), respectively.
}

\subsection{Devil's Advocate: What do the Multiples Add?}
Figure \ref{fig:cmps.devils.hask} illustrates application of the algorithm without the use of
multiples.  Only the regularization across offset, equation (\ref{eq:offset-reg}), is in operation.  
Though we see some suppression of multiples, the results are not nearly as good as those in Figure 
\ref{fig:cmps.lsrow.hask}.  More insidiously, note the presence of considerable of primary energy
in the difference panel.  When exploited as a constraint against crosstalk, the
multiple reflections add considerable information.  My approach integrates this information 
in a systematic framework.
\plot{cmps.devils.hask}{width=5.0in,height=8.0in}{ 
	Only the regularization which roughens across offset, equation (\ref{eq:offset-reg}) is used in the inversion.  
	Only one order of pseudo-primary is used, so no information is added by the multiples.
	Top row, left to right: Raw Haskell data, NMO applied; Estimated primary panel; difference panel.  
	Bottom row, left to right: Data residual; first panel of model residuals, equations 
	(\ref{eq:offset-reg}) and (\ref{eq:order-reg}), respectively.
}

\subsection{A Real Data Example}
I test the proposed algorithm on a single CMP gather from the Mobil AVO dataset, described
above.  The results are shown in Figure \ref{fig:cmps.lsrow.haskreal}.  Relative to the results 
seen on the Haskell synthetic, they are fairly poor.  On the bright side, notice decent preservation
of signal amplitude.  The earliest water-bottom multiples are suppressed quite effectively, although
the later reverberations are left almost untouched.
\par
The reasons for the less-than-perfect are likely numerous.  First, and most important, the multiple
reflections quickly become incoherent with an increasing number of bounces.  They match well with the 
primaries only for the strongest reflections.  I estimated a relatively small water-bottom reflection
coefficient, 0.1, so the multiples are relatively weak in amplitude.  I did not perform any preprocessing
on the data, and I believe they were donated to SEP as raw gathers.  \longcite{Berlioux.sep.80.349}
applied cable balancing to the Mobil AVO dataset.  High-wavenumber, offset-variant amplitude variations 
along events spoil the ability regularization equation (\ref{eq:order-reg}) to discriminate against
crosstalk.
\plot{cmps.lsrow.haskreal}{width=6.0in,height=8.0in}{ 
	Application of equation (\ref{eq:all-resid}) to CMP from Mobil AVO data.
	Top row, left to right: Raw CMP gather, NMO applied; Estimated primary panel; difference panel.  
}

\section{Discussion}
I presented a new approach for the joint imaging of primary and multiple reflections.  
My approach goes further than the separate imaging of multiples and primaries.  I integrate information
from the multiples and primaries in a least-squares inversion, via a new regularization term which
exploits the kinematic similarity of primaries and pseudo-primaries, and the kinematic dissimilarity
of crosstalk terms to obtain a noise-free image of the primaries.
\par
The proposed algorithm demonstrates good noise suppression and signal preservation characteristics in 
the synthetic tests of Figure \ref{fig:cmps.lsrow.hask}.  Comparison of Figures \ref{fig:cmps.nograd.hask}
and \ref{fig:cmps.devils.hask} proves the validity of the new regularization term, equation 
(\ref{eq:order-reg}), and more importantly, that the multiples provide valuable information in the inversion.
\par
The results of testing a real data gather were mixed.  I believe the single largest problem in this case is
poor coherency of the water-bottom multiples.  As the water bottom and most shallow reflectors on the
Mobil AVO dataset are nearly perfectly flat, geologic complexity is surely not to blame.  More likely, the
solution(s) to the trouble is(are) more mundane; things like source/cable balancing and spherical 
divergence.  An accurate RMS velocity function is important to success, but errors can be tolerated.   
Velocity errors lead to curvature in NMO'ed primaries and pseudo-primaries, but as I have dealt here only 
with water-layer multiples, the real danger, a large phase shift between primaries and pseudo-primaries,
is somewhat unlikely.
\par
In all tests, the removal of multiples at near offsets was incomplete.  Since the near offsets contribute
most to residual multiple energy in the stack, it is of crucial importance to improve performance.  

\section{Future Directions}
The obvious direction in which to move this project is migration.  By using migration, rather than NMO, as
the imaging operator, the limiting assumptions of NMO ($v(z)$, flat reflectors) can be abandoned.  
Furthermore, the limitations of operating in the offset domain can be overcome by moving to the more
intuitive angle domain.  In some cases, multiples provide better angular coverage over a recorded cable
length.  Systematic integration of this extra information could prove revolutionary in regions of poor
illumination.  But the fruits of this transition are not without challenges.  Because NMO is a vertical mapping,
each CMP can be processed independently, making the memory requirements of the current implementation 
reasonable, and parallelization quite simple.  Correctly handling the transformation of amplitudes between
primary and pseudo-primary may prove even more of a challenge.  Also, regardless of whether NMO or migration
is used, the move to 3-D is never a forgiving one from the computational standpoint. 

\section{Acknowledgement}
Thanks go to ExxonMobil for providing SEP with the Mobil AVO dataset.

\bibliographystyle{sep}
\bibliography{SEG,SEP,MISC}
