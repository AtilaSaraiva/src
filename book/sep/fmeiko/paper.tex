\lefthead{Fomel}
\righthead{Fast marching}
\footer{SEP--95}

\title{A variational formulation \\ of the fast marching eikonal solver}

\email{sergey@sep.stanford.edu}

\author{Sergey Fomel}

\maketitle

\begin{abstract}
  I exploit the theoretical link between the eikonal equation and
  Fermat's principle to derive a variational interpretation of the
  recently developed method for fast traveltime computations. This
  method, known as fast marching, possesses remarkable computational
  properties. Based originally on the eikonal equation, it can be
  derived equally well from Fermat's principle.  The new variational
  formulation has two important applications: First, the method can be
  extended naturally for traveltime computation on unstructured
  (triangulated) grids. Second, it can be generalized to handle other
  Hamilton-type equations through their correspondence with
  variational principles.
\end{abstract}

\footnotesize
 \begin{quote}
   Now we are in the rarefied atmosphere of theories of excessive
   beauty and we are nearing a high plateau on which geometry, optics,
   mechanics, and wave mechanics meet on a common ground. Only
   concentrated thinking, and a considerable amount of re-creation,
   will reveal the full beauty of our subject in which the last word
   has not been spoken yet.--Cornelius Lanczos, \emph{The variational
     principles of mechanics}
 \end{quote}
\normalsize

\section{Introduction}

Traveltime computation is one of the most important tasks in seismic
processing (Kirchhoff depth migration and related methods) and modeling.  The
traveltime field of a fixed source in a heterogeneous medium is governed by
the eikonal equation, derived about 150 years ago by Sir William Rowan
Hamilton. A direct numerical solution of the eikonal equation has become a
popular method of computing traveltimes on regular grids, commonly used in
seismic imaging \cite[]{GEO55-05-05210526,GEO56-06-08120821,podvin.gji.91}. A
recent contribution to this field is the \emph{fast marching} level set
method, developed by \cite{paper2} in the general context of level set methods
for propagating interfaces \cite[]{osher,book}. \cite{mihai} report a
successful application of this method in three-dimensional seismic
computations. The fast marching method belongs to the family of upwind
finite-difference schemes aimed at providing the \emph{viscosity} solution
\cite[]{lions}, which corresponds to the first-arrival branch of the
traveltime field. The remarkable stability of the method results from a
specifically chosen order of finite-difference evaluation.  The order
selection scheme resembles the \emph{expanding wavefronts} method of
\cite{GEO57-03-04780487}.  The fast speed of the method is provided by the
heap sorting algorithm, commonly used in Dijkstra's shortest path computation
methods \cite[]{mit}. A similar idea has been used previously in a slightly
different context, in the \emph{wavefront tracking} algorithm of
\cite{GEO59-04-06320643}.
\par
In this paper, I address the question of evaluating the fast marching
method's applicability to more general situations. I describe a simple
interpretation of the algorithm in terms of variational principles
(namely, Fermat's principle in the case of eikonal solvers.) Such an
interpretation immediately yields a useful extension of the method for
unstructured grids: triangulations in two dimensions and tetrahedron
tesselations in three dimensions.  It also provides a constructive way
of applying similar algorithms to solving other eikonal-like
equations: anisotropic eikonal \cite[]{SEG-1991-1530}, ``focusing''
eikonal \cite[]{Biondi.sep.95.biondo1}, kinematic offset continuation
\cite[]{Fomel.sep.84.179}, and kinematic velocity continuation
\cite[]{Fomel.sep.92.159}. Additionally, the variational formulation can
give us hints about higher-order enhancements to the original
first-order scheme.

\section{A brief description of the fast marching method}

For a detailed description of level set methods, the reader is
referred to Sethian's recently published book \cite[]{book}. More
details on the fast marching method appear in articles by
\cite{paper2} and \cite{mihai}. This section serves as a brief
introduction to the main bulk of the algorithm.
\par
The key feature of the algorithm is a carefully selected order of
traveltime evaluation. At each step of the algorithm, every grid point
is marked as either \emph{Alive} (already computed), \emph{NarrowBand}
(at the wavefront, pending evaluation), or \emph{FarAway} (not touched
yet). Initially, the source points are marked as \emph{Alive}, and the
traveltime at these points is set to zero. A continuous band of points
around the source are marked as \emph{NarrowBand}, and their
traveltime values are computed analytically. All other points in the
grid are marked as \emph{FarAway} and have an ``infinitely large''
traveltime value.
\par
An elementary step of the algorithm consists of the following
moves:
 \begin{enumerate}
 \item Among all the \emph{NarrowBand} points, extract the point with
 the minimum traveltime.
 \item Mark this point as \emph{Alive}.
 \item Check all the immediate neighbors of the minimum point and
 update them if necessary.
 \item Repeat.
 \end{enumerate}
\par
An update procedure is based on an upwind first-order approximation to
the eikonal equation. In simple terms, the procedure starts with
selecting one or more (up to three) neighboring points around the
updated point. The traveltime values at the selected neighboring
points need to be smaller than the current value. After the selection,
one solves the quadratic equation
\begin{equation}
\label{eqn:update}
\sum_{j} \left(\frac{t_i-t_j}{\triangle x_{ij}}\right)^2 = s_i^2
\end{equation}
for $t_i$. Here $t_i$ is the updated value, $t_j$ are traveltime
values at the neighboring points, $s_i$ is the slowness at the point
$i$, and $\triangle x_{ij}$ is the grid size in the $ij$ direction.
As the result of the updating, either a \emph{FarAway} point is marked
as \emph{NarrowBand} or a \emph{NarrowBand} point gets assigned a new
value.
\par
Except for the updating scheme (\ref{eqn:update}), the fast marching
algorithm bears a very close resemblance to the famous shortest path
algorithm of \cite{dijkstra}.  It is important to point out that
unlike Moser's method, which uses Dijkstra's algorithm directly
\cite[]{GEO56-01-00590067}, the fast marching approach does not
construct the ray paths from predefined pieces, but dynamically
updates traveltimes according to the first-order difference operator
(\ref{eqn:update}). As a result, the computational error of this
method goes to zero with the decrease in the grid size in a linear
fashion.  The proof of validity of the method (omitted here) is also
analogous to that of Dijkstra's algorithm \cite[]{paper2,book}. As in
most of the shortest-path implementations, the computational cost of
extracting the minimum point at each step of the algorithm is greatly
reduced [from $O (N)$ to $O (\log N)$ operations] by maintaining a
priority-queue structure (heap) for the \emph{NarrowBand} points
\cite[]{mit}.

\inputdir{fmarch}

Figure \ref{fig:salt} shows an example application of the fast marching
eikonal solver on the three-dimensional SEG/EAGE salt model.  The computation
is stable despite the large velocity contrasts in the model. The current
implementation takes about 10 seconds for computing a 100x100x100 grid on one
node of SGI Origin 200. \cite{Alkhalifah.sep.95.tariq5} discuss the
differences between Cartesian and polar coordinate implementations.

\sideplot{salt}{width=3.5in}{Constant-traveltime con\-tours
  of the first-arrival travel\-time, computed in the SEG/EAGE salt
  model. A point source is positioned inside the salt body. The top
  plot is a diagonal slice; the bottom plot, a depth slice.}
\par
The difference equation (\ref{eqn:update}) is a finite-difference approximation
to the continuous eikonal equation
\begin{equation}
\label{eqn:eikonal}
\left(\frac{\partial t}{\partial x}\right)^2 +
\left(\frac{\partial t}{\partial y}\right)^2 +
\left(\frac{\partial t}{\partial z}\right)^2 = s^2 (x,y,z)\;,
\end{equation}
where $x$, $y$, and $z$ represent the spatial Cartesian
coordinates. In the next two sections, I show how the updating
procedure can be derived without referring to the eikonal
equation, but with the direct use of Fermat's principle.

\section{The theoretic grounds of variational principles}

\inputdir{XFig}

This section serves as a brief reminder of the well-known theoretical
connection between Fermat's principle and the eikonal equation.  The
reader, familiar with this theory, can skip safely to the next
section.

\sideplot{fermat}{width=2in}{Illustration of the connection
  between Fermat's principle and the eikonal equation. The shortest
  distance between a wavefront and a neighboring point $M$ is along
  the wavefront normal.}
\par
Both Fermat's principle and the eikonal equation can serve as the
foundation of traveltime calculations. In fact, either one can be
rigorously derived from the other. A simplified derivation of this
fact is illustrated in Figure \ref{fig:fermat}. Following the notation
of this figure, let us consider a point $M$ in the immediate
neighborhood of a wavefront $t (N) = t_N$.  Assuming that the source
is on the other side of the wavefront, we can express the traveltime
at the point $M$ as the sum
\begin{equation}\label{eqn:tm}
  t_M = t_N + l (N,M) s_M\;,
\end{equation}
where $N$ is a point on the front, $l (N,M)$ is the length of the
ray segment between $N$ and $M$, and $s_M$ is the local slowness.
As follows directly from equation (\ref{eqn:tm}),
\begin{equation}\label{eqn:link}
  \left|\nabla t\right| \cos{\theta} = \frac{\partial t}{\partial l}
  = \lim_{M \rightarrow N} \frac{t_M - t_N}{l (N,M)} = s_N\;.
\end{equation}
Here $\theta$ denotes the angle between the traveltime gradient (normal
to the wavefront surface) and the line from $N$ to $M$, and
$\frac{\partial t}{\partial l}$ is the directional traveltime derivative along
that line.
\par
If we accept the local Fermat's principle, which says that the ray
from the source to $M$ corresponds to the minimum-arrival time, then,
as we can see geometrically from Figure \ref{fig:fermat}, the angle
$\theta$ in formula (\ref{eqn:link}) should be set to zero to achieve
the minimum. This conclusion leads directly to the eikonal equation
(\ref{eqn:eikonal}). On the other hand, if we start from the eikonal
equation, then it also follows that $\theta=0$, which corresponds to
the minimum traveltime and constitutes the local Fermat's principle.
The idea of that simplified proof is taken from \cite{lanc},
though it has obviously appeared in many other publications. The
situations in which the wavefront surface has a discontinuous normal
(given raise to multiple-arrival traveltimes) require a more elaborate
argument, but the above proof does work for first-arrival traveltimes
and the corresponding viscosity solutions of the eikonal equation
\cite[]{lions}.
\par
The connection between variational principles and first-order
partial-differential equations has a very general meaning, explained
by the classic Hamilton-Jacobi theory. One generalization of the
eikonal equation is 
\begin{equation}\label{eqn:elips}
  \sum_{i,j} a_{ij} (\mathbf{x})\,
  \frac{\partial \tau}{\partial x_i}\,
  \frac{\partial \tau}{\partial x_j} = 1\;,
\end{equation}
where $\mathbf{x} = \{x_1, x_2, \ldots\}$ represents the vector of space
coordinates, and the coefficients $a_{ij}$ form a positive-definite
matrix $A$.  Equation (\ref{eqn:elips}) defines the characteristic
surfaces $t = \tau (\mathbf{x})$ for a linear hyperbolic second-order
differential equation of the form
\begin{equation}\label{eqn:wave}
  \sum_{i,j} a_{ij} (\mathbf{x})
  \frac{\partial^2 u}{\partial x_i \partial x_j} +
  F (\mathbf{x}, u, \frac{\partial u}{\partial x_i}) =
  \frac{\partial^2 u}{\partial t^2}\;,
\end{equation}
where F is an arbitrary function.
\par
A known theorem \cite[]{smirnov} states that the propagation rays
[characteristics of equation (\ref{eqn:elips}) and, correspondingly,
bi-characteristics of equation (\ref{eqn:wave})] are geodesic
(extreme-length) curves in the Riemannian metric
\begin{equation}\label{eqn:riman}
  d \tau = \sqrt{\sum_{i,j} b_{ij} (\mathbf{x})\, dx_i\, dx_j}\;,
\end{equation}
where $b_{ij}$ are the components of the matrix $B = A^{-1}$. This
means that a ray path between two points $\mathbf{x}_1$ and $\mathbf{x}_2$
has to correspond to the extreme value of the curvilinear integral
\[
\int_{\mathbf{x}_1}^{\mathbf{x}_2}\,\sqrt{\sum_{i,j} b_{ij} (\mathbf{x})\,
  dx_i\, dx_j}\;.
\]
For the isotropic eikonal equation (\ref{eqn:eikonal}), $a_{ij} =
\delta_{ij}/s^2 (\mathbf{x})$, and metric (\ref{eqn:riman}) reduces to
the familiar traveltime measure
\begin{equation}\label{eqn:simple}
  d \tau = s (\mathbf{x})\, d\sigma\;,
\end{equation}
where $d\sigma = \sqrt{\sum_{i} dx_i^2}$ is the usual Euclidean
distance metric.  In this case, the geodesic curves are exactly
Fermat's extreme-time rays.
\par
From equation (\ref{eqn:riman}), we see that Fermat's principle in the
general variational formulation applies to a much wider class of
situations if we interpret it with the help of non-Euclidean
geometries.

\section{Variational principles on a grid}

In this section, I derive a discrete traveltime computation procedure,
based solely on Fermat's principle, and show that on a Cartesian
rectangular grid it is precisely equivalent to the update formula
(\ref{eqn:update}) of the first-order eikonal solver.

\sideplot{triangle}{width=2in}{A geometrical scheme for the
  traveltime updating procedure in two dimensions.}
\par
For simplicity, let us focus on the two-dimensional case\footnote{A
  very similar analysis applies in three dimensions, but requires a
  slightly more tedious algebra. It is left as an exercise for the
  reader.}.  Consider a line segment with the end points $A$ and $B$,
as shown in Figure \ref{fig:triangle}. Let $t_A$ and $t_B$ denote the
traveltimes from a fixed distant source to points $A$ and $B$,
respectively. Define a parameter $\xi$ such that $\xi=0$ at $A$,
$\xi=1$ at $B$, and $\xi$ changes continuously on the line segment
between $A$ and $B$. Then for each point of the segment, we can
approximate the traveltime by the linear interpolation formula
\begin{equation}\label{eqn:linear}
  t (\xi) = (1-\xi) t_A + \xi t_B\;.
\end{equation}
Now let us consider an arbitrary point $C$ in the vicinity of $AB$. If
we know that the ray from the source to $C$ passes through some point
$\xi$ of the segment $AB$, then the total traveltime at $C$ is
approximately
\begin{equation}\label{eqn:tc}
  t_C = t (\xi) + s_C\,\sqrt{|AB|^2 (\xi-\xi_0)^2 +
  \rho_0^2}\;,
\end{equation}
where $s_C$ is the local slowness, $\xi_0$ corresponds to the
projection of $C$ to the line $AB$ (normalized by the length $|AB|$),
and $\rho_0$ is the length of the normal from $C$ to $\xi_0$.
\par
Fermat's principle states that the actual ray to $C$ corresponds to a
local minimum of the traveltime with respect to raypath perturbations.
According to our parameterization, it is sufficient to find a local
extreme of $t_C$ with respect to the parameter $\xi$.  Equating the
$\xi$ derivative to zero, we arrive at the equation
\begin{equation}\label{eqn:fermat}
 t_B - t_A + \frac{s_C\,|AB|^2\,(\xi-\xi_0)}
 {\sqrt{|AB|^2 (\xi-\xi_0)^2 + \rho_0^2}} = 0\;,
\end{equation}
which has (as a quadratic equation) the explicit solution for $\xi$:
\begin{equation}
  \label{eqn:xi}
  \xi = \xi_0 \pm \frac{\rho_0\,(t_A - t_B)}
  {|AB|\,\sqrt{s_C^2\,|AB|^2 - (t_A - t_B)^2}}\;.
\end{equation}
Finally, substituting the value of $\xi$ from (\ref{eqn:xi}) into
equation (\ref{eqn:tc}) and selecting the appropriate branch of the
square root, we obtain the formula
\begin{eqnarray}\label{eqn:formula}
  c\,t_C & = & \rho_0\,\sqrt{s_C^2 c^2 - (t_A - t_B)^2} +
  c\,t_A\,(1-\xi_0) + c\,t_B\,\xi_0 = \nonumber \\
  & & \rho_0\,\sqrt{s_C^2 c^2 - (t_A - t_B)^2} +
  a\,t_A\,\cos{\beta} + b\,t_B\,\cos{\alpha}\;,
\end{eqnarray}
where $c = |AB|$, $a = |BC|$, $b = |AC|$, angle $\alpha$
corresponds to $\widehat{BAC}$, and angle $\beta$ corresponds to
$\widehat{ABC}$ in the triangle $ABC$ (Figure \ref{fig:triangle}).

\sideplot{square}{width=1.5in}{NR}{A geometrical scheme for
  traveltime updating on a rectangular grid.}
\par
To see the connection of formula (\ref{eqn:formula}) with the eikonal
difference equation (\ref{eqn:update}), we need to consider the case
of a rectangular computation cell with the edge $AB$ being a diagonal
segment, as illustrated in Figure \ref{fig:square}. In this case,
$\cos{\alpha} = \frac{a}{c}$, $\cos{\beta} = \frac{b}{c}$, $\rho_0 =
\frac{ab}{c}$, and formula (\ref{eqn:formula}) reduces to 
\begin{equation}\label{eqn:square}
  t_C = \frac{ab\,\sqrt{s_C^2 (a^2 + b^2) - (t_A - t_B)^2} + a^2\,t_A + b^2\,t_B}
{a^2+b^2}\;.
\end{equation}
We can notice that (\ref{eqn:square}) is precisely equivalent to the
solution of the quadratic equation (\ref{eqn:formula}), which in our
new notation takes the form
\begin{equation}\label{eqn:supdate}
  \left(\frac{t_C - t_A}{b}\right)^2 +
  \left(\frac{t_C - t_B}{a}\right)^2 = s_C^2\;.
\end{equation}
\par
What have we accomplished by this analysis? First, we have derived a
local traveltime computation formula for an arbitrary grid.  The
derivation is based solely on Fermat's principle and a local linear
interpolation, which provides the first-order accuracy.  Combined with
the fast marching evaluation order, which is also based on Fermat's
principle, this procedure defines a complete algorithm of
first-arrival traveltime calculation. On a rectangular grid, this
algorithm is exactly equivalent to the fast marching method of
\cite{paper2} and \cite{mihai}.  Second, the derivation
provides a general principle, which can be applied to derive analogous
algorithms for other eikonal-type (Hamilton-Jacobi) equations and
their corresponding variational principles.

%A very similar analysis applies in 3-D (Appendix A).

\section{Solving the eikonal equation on a triangulated grid}

Unstructured (triangulated) grids have computational advantages over
rectangular ones in three common situations:
 \begin{itemize}
 \item When the number of grid points can be substantially reduced by
   putting them on an irregular grid. This situation corresponds to
   irregular distribution of details in the propagation medium.
 \item When the computational domain has irregular boundaries. One
   possible kind of boundary corresponds to geological interfaces and
   seismic reflector surfaces \cite[]{SEG-1993-0170}. Another type of irregular
   boundary, in application to traveltime computations, is that of
   seismic rays. The method of bounding the numerical eikonal solution
   by ray envelopes has been introduced recently by \cite{SEG-1996-1208}.
 \item When the grid itself needs to be dynamically updated to maintain
   a certain level of accuracy in the computation.
 \end{itemize}
With its computational speed and unconditional stability, the fast
marching method provides considerable savings in comparison with 
alternative, more accurate methods, such as semi-analytical ray
tracing \cite[]{SEG-1991-1497,SEG-1995-1247} or the general Hamilton-Jacobi
solver of \cite{abgrall}.

%\plot{test}{width=6in}{Traveltime contours, computed in the rough
%  Marmousi model (left), the smoothed Marmousi (middle), and the
%  smoothed triangulated Marmousi (right).}
%\par
%Figure \ref{fig:test} shows a comparison between first-arrival
%traveltime computations in regularly gridded and triangulated Marmousi
%models. The two results match each other within the first-order
%accuracy of the fast marching method. However, the cost of the
%triangulated computation has been greatly reduced by constraining the
%number of nodes.
\par
Computational aspects of triangular grid generation are outlined in
Appendix A. A three-dimensional application would follow the same
algorithmic patterns.

\section{Conclusions}

Variational principles have played an exceptionally important role in
the foundations of mathematical physics. Their potential in numerical
algorithms should not be underestimated.
\par
In this paper, I interpret the fast marching eikonal solver with the
help of Fermat's principle. Two important generalizations follow
immediately from that interpretation. First, it allows us to obtain a
fast method of first-arrival traveltime computation on triangulated
grids.  Furthermore, we can obtain a general principle, which extends
the fast marching algorithm to other Hamilton-type equations and their
variational principles. More research is required to confirm these
promises. 
\par
In addition, future research should focus on 3-D implementations and
on increasing the approximation order of the method.

\section{Acknowledgments}
I thank Mihai Popovici and Biondo Biondi for drawing my attention to
the fast marching level set method. Discussions with Bill Symes, Dan
Kosloff, and Tariq Alkhalifah were crucial for developing a general
understanding of the method. Jamie Sethian kindly responded to more
specific questions. The conforming triangulation program was developed
at the suggestion of Leonidas Guibas as a research project for his
Geometrical Algorithms class at Stanford.

\bibliographystyle{seg}
\bibliography{SEP2,SEG,paper}

%\APPENDIX{A}
%\section{Fermat's principle on a three-dimensional grid}
%\begin{equation}\label{eqn:box}
%  t_D = \frac{\sqrt{3\,\left( s_C^2 (\triangle x)^2 - t_A^2 - t_B^2 - t_C^2\right) +
%      (t_A + t_B + t_C)^2} + (t_A + t_B + t_C)}{3}\;.
%\end{equation}
%\begin{equation}\label{eqn:bupdate}
%  (t_D - t_A)^2 + (t_D - t_B)^2 + (t_D - t_C)^2 = s_D^2 (\triangle x)^2\;.
%\end{equation}
%and so on.

\newpage

%\APPENDIX{A}
\append{Incremental DELAUNAY TRIANGULATION and related problems}

Delaunay triangulation \cite[]{delaunay2,sibson,stolfi} is a fundamental
geometric construction, which has numerous applications in different
computational problems. For a given set of nodes (points on the
plane), Delaunay triangulation constructs a triangle tessellation of
the plane with the initial nodes as vertices. Among all possible
triangulations, the Delaunay triangulation possesses optimal
properties, which make it very attractive for practical applications,
such as computational mesh generation. One of the most well-known
properties is maximizing the minimum triangulation angle. In three
dimensions, Delaunay triangulation generalizes naturally to a
tetrahedron tessellation.
\par
Several optimal-time algorithms of Delaunay triangulation (and its
counterpart--Voronoi diagram) have been proposed in the literature.
The divide-and-conquer algorithm \cite[]{shamos,stolfi} and the
sweep-line algorithm \cite[]{fortune} both achieve the optimal $O (N
\log N)$ worst-case time complexity. Alternatively, a family of
incremental algorithms has been used in practice because of their
simplicity and robustness. Though the incremental algorithm can take
$O (N^2)$ time in the worst case, the expectation time can still be $O
(N \log N)$, provided that the nodes are inserted in a random order
\cite[]{knuth}.
\par
The incremental algorithm consists of two main parts:
 \begin{enumerate}
 \item Locate a triangle (or an edge), containing the inserted point.
 \item Insert the point into the current triangulation, making the
   necessary adjustments.
 \end{enumerate}
\par
The Delaunay criterion can be reduced in the second step to a simple
\emph{InCircle} test \cite[]{stolfi}: if a circumcircle of a triangle
contains another triangulation vertex in its circumcenter, then the
edge between those two triangles should be ``flipped'' so that two new
triangles are produced. The testing is done in a recursive fashion
consistent with the incremental nature of the algorithm. When a new
node is inserted inside a triangle, three new triangles are created,
and three edges need to be tested. When the node falls on an edge,
four triangles are created, and four edges are tested. In the case of
test failure, a pair of triangles is replaced by the flip operation
with another pair, producing two more edges to test. Under the
randomization assumption, the expected total time of point insertion
is $O (N)$.  Randomization can be considered as an external part of
the algorithm, provided by preprocessing.
\par
\cite{knuth} reduce the point location step to an efficient $O (N
\log N)$ procedure by maintaining a hierarchical tree structure: all
triangles, occurring in the incremental triangulation process, are
kept in memory, associated with their ``parents.'' One or two point
location tests (\emph{CCW} tests) are sufficient to move to a lower
level of the tree. The search terminates with a current Delaunay
triangle.
\par
To test the algorithmic performance of the incremental construction, I
have profiled the execution time of my incremental triangulation
program with the Unix \texttt{pixie} utility. The profiling result,
shown in Figures~\ref{fig:itime} and~\ref{fig:ctime}, complies
remarkably with the theory: $O (N \log N)$ operations for the point
location step, and $O (N)$ operations for the point insertion step.
The experimental constant for the insertion step time is about $8.6$.
The experimental constant for the point location step is $4$.  The CPU
time, depicted in Figure~\ref{fig:time}, also shows the expected $O (N
\log N)$ behavior.

\inputdir{.}

\sideplot{itime}{width=2.5in}{The number of point insertion operations
(\emph{InCircle} test) plotted against the number of points.}

\sideplot{ctime}{width=2.5in}{Number of point location operations
  (\emph{CCW} test) plotted against the number of points.}

\sideplot{time}{width=2.5in}{CPU time (in seconds per point) plotted
  against the number of points.}
\par
A straightforward implementation of Delaunay triangulation would
provide an optimal triangulation for any given set of nodes. However,
the quality of the result for unfortunate geometrical
distributions of the nodes can be unsatisfactory. In the rest of this
appendix, I describe three problems, aimed at improving the
triangulation quality: conforming triangulation, triangulation of
height fields, and mesh refinement.  Each of these problems can be
solved with a variation of the incremental algorithm.

\subsection{Conforming Triangulation}

\inputdir{tri}

In the practice of mesh generation, the input nodes are often
supplemented by boundary edges: geologic interfaces, seismic rays, and
so on. It is often desirable to preserve the edges so that they appear
as edges of the triangulation \cite[]{SEG-1994-0502}. One possible approach is
\emph{constrained} triangulation, which preserves the edges, but only
approximately satisfies the Delaunay criterion \cite[]{lee,chew}. An
alternative, less investigated, approach is \emph{conforming}
triangulation, which preserves the ``Delaunayhood'' of the
triangulation by adding additional nodes \cite[]{hansen} (Figure
\ref{fig:conform}).  Conforming Delaunay triangulations are difficult
to analyze because of the variable number of additional nodes. This
problem was attacked by \cite{edels}, who suggested an algorithm
with a defined upper bound on added points. Unfortunately,
Edelsbrunner's algorithm is slow in practice because the number of
added points is largely overestimated.  I chose to implement a
modification of the simple incremental algorithm of Hansen and Levin.
Although Hansen's algorithm has only a heuristic justification and
sets no upper bound on the number of inserted nodes, its simplicity is
attractive for practical implementations, where it can be easily
linked with the incremental algorithm of Delaunay triangulation.
\par
The incremental solution to the problem of conforming triangulation
can be described by the following scheme:
 \begin{itemize}
 \item First, the boundary nodes are triangulated.
 \item Boundary edges are inserted incrementally.
 \item If a boundary edge is not present in the triangulations, it is
   split in half, and the middle node is inserted into the triangulation. This
   operation is repeated for the two parts of the original boundary
   edge and continues recursively until all the edge parts 
   conform.
 \item If at some point during the incremental process, a boundary edge
   violates the Delaunay criterion (the \emph{InCircle} test), it is
   split to assure the conformity.
 \end{itemize}

\plot{conform}{width=4in,height=2in}{An illustration of conforming triangulation.
  The left plot shows a triangulation of 500 random points; the
  triangulation in the right plot is conforming to the embedded
  boundary.  Conforming triangulation is a genuine Delaunay
  triangulation, created by adding additional nodes to the original
  distribution.}
\par
To insert an edge $AB$ into the current triangulation, I use the
following recursive algorithm:
 \begin{quote}
 Function \textbf{InsertEdge} ($AB$)
 \begin{enumerate}
 \item Define $C$ to be the midpoint of $AB$.
 \item Using the triangle tree structure, locate triangle $\mathcal{T} = DEF$
   that contains $C$ in the current triangulation.
 \item \textbf{If} $AB$ is an edge of $\mathcal{T}$ \textbf{then return}.
 \item \textbf{If} $A$ (or $B$) is a vertex of $\mathcal{T}$ (for example, $A = D$)
   {\bf then} define $C$ as an intersection of $AB$ and $EF$.
 \item {\bf Else} define $C$ as an intersection of $AB$ and an
   arbitrary edge of $\mathcal{T}$ (if such an intersection exists).
 \item Insert $C$ into the triangulation.
 \item {\bf InsertEdge} ($CA$).
 \item {\bf InsertEdge} ($CB$).
 \end{enumerate}
 \end{quote}
\par
The intersection point  of edges $AB$ and $EF$ is given by the formula
\begin{equation}
  C = A + \lambda (B-A)\;,
\end{equation}
where
\begin{equation}
  \lambda = \frac{(F_y - E_y)\,(E_x - A_x) - (F_x - E_x)\,(E_y-A_y)}{
    \det \left|\begin{array}{cc}
        B_x - A_x & B_y - A_y \\
        F_x - E_x & F_y - E_y
    \end{array}\right|}\;.
\end{equation}
The value of $\lambda$ should range between $0$ and $1$.
\par
If, at some stage of the incremental construction, a boundary edge
$AB$ fails the Delaunay \emph{InCircle} test for the circle $CABD$,
then I simply split it into two edges by adding the point of
intersection into the triangulation.  The rest of the process is very
much like the process of edge validation in the original incremental
algorithm.

\subsection{Triangulation of Height Fields}

Often, a velocity field (or other object that we want to triangulate)
is defined on a regular Cartesian grid. One way to perform a
triangulation in this case is to select a smaller subset of the
initial grid points, using them as the input to a triangulation
program. We need to select the points in a way that preserves the main
features of the original image, while removing some unnecessary
redundancy in the regular grid description.

\plot{sphere}{width=6in,height=2in}{Illustration of Garland's
  algorithm for triangulation of height fields. The left plot shows
  the input image of a sphere, containing 100 by 100 pixels. The
  middle plot shows 500 pixels, selected by the algorithm and
  triangulated. The right plot is the result of local plane
  interpolation of the triangulated surface.}
\par
\cite{height} surveyed different approaches
to this problem and proposed a fast version of the incremental
\emph{greedy insertion} algorithm. Their algorithm adds points
incrementally, selecting at each step the point with the maximum
interpolation error with respect to the current triangulation. Though
a straightforward implementation of this idea would lead to an
unacceptably slow algorithm, Garland and Heckbert have discovered
several sources for speeding it up. First, we can take advantage of
the fact that only a small area of the current triangulation gets
updated at each step. Therefore, it is sufficient to recompute the
interpolation error only inside this area. Second, the maximum
extraction procedure can be implemented very efficiently with a
priority queue data structure.

\inputdir{.}

\sideplot{opengl}{width=2in,height=2in}{An image from the previous
  example, as rendered by the OpenGL library. The shades on polygonal
  (triangulated) sides are exaggerated by a simulation of the direct
  light source.}
\par
Figure \ref{fig:sphere} illustrates this algorithm with a simple
example. The original image (the left plot) contained 10,000 points,
laid out on a regular rectangular grid. The algorithm selects a
smaller number of points and immediately triangulates them (the middle
plot).  The image can be reconstructed by local plane interpolation
(the right plot.) The reconstruction quality can be further improved
by increasing the number of triangles. Figure \ref{fig:opengl} shows
the same image as rendered by the OpenGL graphics library
\cite[]{opengl}.

\inputdir{tri}

\plot{marmousi}{width=6in}{Applying the height triangulation algorithm
  to the Marmousi model. The left plot shows a smoothed and windowed
  version of the Marmousi model. The middle plot is a result of
  10,000-point triangulation, followed by linear interpolation. The
  right plot shows the difference between the two images.}
\par
Figure \ref{fig:marmousi} shows an application of the height
triangulation algorithm to the famous Marmousi model. The left plot
shows a smoothed and windowed version of the Marmousi, plotted on a
501 by 376 computational grid. In the middle plot, 10,000 points from
the original 188,376 were selected for triangulation and interpolated
back to the rectangular grid. The right plot demonstates the small
difference between the two images.

\subsection{Mesh Refinement}

One the main properties of Delaunay triangulation is that, for a
given set of nodes, it provides the maximum smallest angle among
all possible triangulations. It is this property that supports the wide
usage of Delaunay algorithms in the mesh generation problems.
However, it doesn't guarantee that the smallest angle will always be small.
In fact, for some point distributions, it is impossible to avoid
skinny small-angle triangles. The remedy is to add additional
nodes to the triangulation so that the quality of the triangles is
globally improved. This problem has become known as
\emph{mesh refinement} \cite[]{ruppert}.

\plot{hole}{width=6in,height=1.5in}{An illustration of Rivara's
  refinement algorithm. The left plot shows an input to the algorithm:
  a valid Delaunay triangulation with ``skinny'' triangles. Two
  other plots show successive applications of the refinement
  algorithm.}
\par
One of the recently proposed mesh refinement algorithms is Rivara's
\emph{backward longest-side refinement} technique \cite[]{rivara}.  The
main idea of the algorithm is to trace the LSPP (longest-side
propagation path) for each refined triangle. The LSPP is an ordered
list of triangles, connected by a common edge, such that the longest
triangle edge is strictly increasing. After tracing the LSPP, we
bisect the longest edge and insert its midpoint into the
triangulation. Rivara's algorithm is remarkably efficient and easy to
implement. In comparison with alternative methods, it has the
additional advantage of being applicable in three dimensions.
\par
Figure \ref{fig:cerveny} demonstrates an application of different
triangulation techniques to a simple layered model, borrowed from the
Seismic Unix demos (where it is attributed it to V.\v{C}erven\'{y}.)
Another model from \cite{hale} is used in Figure \ref{fig:susalt}.

\plot{cerveny}{width=6in,height=4in}{A comparison of different
  triangulation techniques on a simple layered model. The top left
  plot shows the original model; the top right plot, the result of
  noncomforming triangulation; the two bottom plots, conforming
  triangulation and an additional mesh refinement.}

\plot{susalt}{width=6in,height=4in}{A comparison of different
  triangulation techniques on a simple salt-type model. The top left
  plot shows the original model; the top right plot, the result of
  noncomforming triangulation; the two bottom plots, conforming
  triangulation and an additional two-step mesh refinement.}

\subsection{Implementation Details}
Edge operations form the basis of the incremental algorithm.
Therefore, it is convenient to describe triangulation with
edge-oriented data structures \cite[]{stolfi}. I have followed the idea
of \cite{hansen} of associating with each edge two pointers to the
end points and two pointers to the adjacent triangles.  The triangle
structure is defined by three pointers to the edges of a triangle.
Additionally, as required by the point location algorithm, each
triangle has a pointer to its ``children.'' This pointer is NULL when
the triangle belongs to the current Delaunay triangulation.
\par
Many researchers have pointed out that the geometric
primitives used in triangulation must be robust with respect to
round-off errors of finite-precision calculation. To assure the
robustness of the code, I used the adaptive-precision predicates of
\cite{shewchuk}, available as a separate package from the
\texttt{netlib} public-domain archive.

%%% Local Variables: 
%%% mode: latex
%%% TeX-master: t
%%% End: 


%%% Local Variables: 
%%% mode: latex
%%% TeX-master: t
%%% End: 
