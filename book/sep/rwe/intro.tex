\section{Introduction}

\inputdir{XFig}

Imaging complex geology is one of the main challenges of 
\sout{today's}
seismic processing
\uline{today}. 
Of the many seismic imaging methods available,
downward continuation \cite[]{Claerbout.blackwell.85}
is accurate, robust, and capable of handling
models with large and sharp velocity variations.
This method naturally handles the multipathing
that occurs in complex geology and provides a band-limited
solution to the seismic imaging problem.
Furthermore, as computational power increases, such methods
are gradually moving into the mainstream of seismic
processing.
This explains why one-way wave extrapolation has been a 
subject of extensive theoretical research in the recent years
\cite[]{ristow.geo.94,dehoop.jmathphys.96,SEG-1996-0415,thomson1999,GEO67-03-08720882}.
\par
However, migration by downward 
continuation imposes strong limitations on the dip of reflectors 
that can be imaged since, by design, it favors 
downward propagating energy.
%% which is 
%%propagating mainly in the downward direction. 
Upward propagating energy, for example overturning waves,
can be imaged in principle using downward continuation
methods \cite[]{GEO57-11-14531462}, although the procedure 
is difficult, particularly for prestack data.
In contrast, Kirchhoff-type methods based on ray-traced
traveltimes can image steep dips and handle 
overturning waves, although those methods are far less 
reliable in complex velocity models given their
asymptotic assumption \cite[]{GEO66-05-16221640}.
\par
The steep-dip limitation of downward continuation techniques has
been addressed in several ways: 
\bit
\itt A first option is to increase the
angular accuracy of the extrapolation operator, for example by 
employing methods from the Fourier finite-difference (FFD)
family \cite[]{ristow.geo.94,GEO67-03-08720882}, or the
Generalized Screen Propagator (GSP) family 
\cite[]{dehoop.jmathphys.96,SEG-1996-0415}. 
The enhancements brought about by these methods
come at a price, since they increase the cost of 
extrapolation without guaranteeing unconditional stability.
%%
\itt A second option is to perform the wavefield extrapolation
in tilted coordinate systems \cite[]{Etgen-SEG02}, 
or by designing sources that favor illumination of
particular regions of the image 
\cite[]{GEO59-05-08010809,Wu-SEG02}.
We can thus increase angular accuracy, although these methods 
are best suited for only a subset of the model 
(a salt flank, for example),
and potentially decrease the accuracy in other regions of the model. 
In complex geology, 
%%it is also not obvious what is 
defining an optimal
tilt angle for the extrapolation grid is not obvious. 
%%
\itt A third possibility is hybridization of 
wavefield and ray-based techniques, either in the form 
of Gaussian beams 
\cite[]{cerveny,GEO.55.11.14161428,GEO66-04-12401250,Gray-SEG02},
coherent states 
\cite[]{Albertin-SEG01,Albertin-SEG02},
or beam-waves 
\cite[]{BDEtgen-EAGE03}.
Such techniques are quite powerful, since they
couple wavefield methods with
multipathing and band-limited properties, with
ray methods, which deliver arbitrary directions 
of propagation, even overturning.
Beams can be understood as localized wave packets propagating in the 
preferential direction
\cite[]{SEG-2000-10081011}.
Numerically, they are usually implemented using localized 
extrapolation paths.
Extrapolation
beams may leave shadow zones in various parts 
of the model, which hamper their imaging abilities.
Furthermore, extrapolation beams have limited width, 
and do not allow diffractions from sharp features in the
velocity model to develop completely without being
attenuated at the boundaries \uline{of the beam}.
%%limits 
%%the extent of the diffractions created by sharp features 
%%in the model to 
%%that of any particular beam, no matter how accurate the 
%%extrapolator within each beam is. 
In addition, the narrow 
extrapolation domain \uline{can} generate\sout{s} beam superposition
artifacts, such as beam boundary effects.
\eit
\par
In our method, we recognize that Cartesian coordinates for 
downward, tilted continuation or along beams of limited spatial extent,
are just a mathematical convenience that do not reflect a 
physical reality.
The main idea of our paper is to reformulate wavefield 
extrapolation in general Riemannian coordinates that conform
with the general direction of wave propagation,
thus the name {\it Riemannian wavefield extrapolation} (RWE)
for our method.
We formulate the wavefield extrapolation theory in arbitrary {3-D} 
semi-orthogonal Riemannian spaces, where the extrapolation direction
is orthogonal to all other directions.
Examples of such coordinate systems include, but are not limited
to, fans of rays emerging from a source point, 
or bundles of rays initiated by plane waves of arbitrary
initial dips at the source.
For constant background velocity,
our method reduces to 
extrapolation in polar/spherical coordinates
\cite[]{Nichols.sepphd.81,GEO61-01-02530263},
or extrapolation in tilted coordinates
\cite[]{Etgen-SEG02}.
Our method is also closely related to Huygens wavefront tracing
\cite[]{GEO66-03-08830889}, 
which represents a finite-difference
solution to the eikonal equation in ray coordinates.
\par
The main strength of our method is that the coordinate system
can follow the waves, which may overturn, such that 
we can use one-way extrapolators to image 
diving waves (Figure~\ref{fig:overturned}).
\sideplot{overturned}{width=3.0in}
{
Ray coordinate systems are superior to tilted coordinate 
systems for imaging overturning waves using one-way 
wavefield extrapolators.
Overturning reflected energy may become evanescent 
in tilted coordinate systems (a), 
but stays non-evanescent in ray coordinate systems (b).
}
We can also use inexpensive
extrapolators with limited angle accuracy (e.g. $15^\circ$),
since, in principle, we are never too far from the 
wave propagation direction.
We are also not confined to the extent
of any individual extrapolation 
beam, therefore we can track diffractions
for their entire spatial extent (Figure~\ref{fig:beams}).
%%
\sideplot{beams}{width=3.0in}
{Extrapolated energy is attenuated
at beam boundaries (a), but is propagated in a Riemannian 
coordinate system (b).}
%%





















