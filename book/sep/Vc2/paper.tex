\normalem

\renewcommand{\uline}[1]{#1}

\def\figdir{./Fig}
\def\cakedir{.}

\lefthead{Fomel}
\righthead{Velocity continuation}
\footer{SEP--92}

\title{\uline{Time} migration velocity analysis by velocity continuation}

\email{sergey@sep.stanford.edu}
\author{Sergey Fomel}

\ABS{\uline{Time} migration velocity analysis can be performed by velocity
  continuation, an \uline{incremental} process that \uline{transforms}
  migrated seismic sections according to changes in the migration velocity.
  Velocity continuation enhances residual \uline{normal moveout} correction by
  properly taking into account both vertical and lateral movements of
  \uline{events on seismic images}. Finite-difference and spectral algorithms
  provide efficient practical implementations for velocity continuation.
  Synthetic and field data examples demonstrate the performance of the method
  and confirm theoretical expectations.}

\begin{comment}
{Residual and cascaded migration can be described as a continuous
  process of velocity continuation in the post-migration domain. This
  process moves reflection events on migrated seismic sections
  according to changes in the migration velocity. Understanding the
  laws of velocity continuation is crucially important for a
  successful application of migration velocity analysis.  In this
  paper, I derive kinematic and dynamic laws for the case of prestack
  residual migration from simple trigonometric principles. The main
  theoretical result is a decomposition of prestack velocity
  continuation into three different components corresponding to
  residual normal moveout, residual dip moveout, and residual
  zero-offset migration. I analyze the contribution and properties of
  each of the three components separately and develop efficient
  finite-difference and spectral algorithms of velocity continuation.
  The algorithms are applied for migration velocity analysis on a real
  data example.}
\end{comment}

\newpage

\section{Introduction}
%%%%%%%%%%%%%%%%%%
Migration velocity analysis is a routine part of prestack time
migration applications. It serves both as a tool for velocity
estimation \cite{FBR08.06.02240234} and as a tool for optimal stacking
of migrated seismic sections prior to modeling zero-offset data for depth
migration \cite{GEO62-02-05680576}. In the most common form, migration
velocity analysis amounts to residual moveout correction on CIP
(common image point) gathers. However, in the case of dipping
reflectors, this correction does not provide optimal focusing of
reflection energy, since it does not account for lateral movement of
reflectors caused by the change in migration velocity. In other words,
different points on a stacking hyperbola in a CIP gather can
correspond to different reflection points at the actual reflector. The
situation is similar to that of the conventional \uline{normal moveout} (NMO)
velocity analysis, where the reflection point dispersal problem is
usually overcome with the help of \uline{dip moveout} \cite{FBR04.07.00070024,dmo}. An
analogous correction is required for optimal focusing in the
post-migration domain. In this paper, I propose and test velocity
continuation as a method of migration velocity analysis. The method
enhances the conventional residual moveout correction by taking into
account lateral movements of migrated reflection events.

Velocity continuation is a process of transforming time migrated images
according to the changes in migration velocity. This process has wave-like
properties, which have been described in earlier papers
\cite{me,Fomel.segab.97,first}.  \longcite{hubral} and
\longcite{GEO62-02-05890597} use the term \emph{image waves} to describe a
similar concept. \longcite{adlerSEG,adler} generalizes the velocity
continuation approach for the case of variable background velocities, using
the term \emph{Kirchhoff image propagation}. \uline{Although the velocity
continuation concept is tailored for time migration, it finds important
applications in depth migration velocity analysis by recursive methods} 
\cite{SEG-1999-17231726,SEG-2000-08740877}.

\begin{comment}
Velocity
continuation extends the theory of residual and cascaded migrations
\cite{GEO50.01.01100126, GEO52.05.06180643}. In practice, the
continuation process can be modeled by finite-difference or spectral
methods \cite{Fomel.sep.95.sergey2, Fomel.sep.97.sergey2}.

In a recent work \cite{me,Fomel.sep.92.159}, I
introduced the process of \emph{velocity continuation} to describe a
continuous transformation of seismic time-migrated images with a
change of the migration velocity. Velocity continuation generalizes
the ideas of residual migration
\cite{GEO50.01.01100126,Etgen.sepphd.68} and cascaded migrations
\cite{GEO52.05.06180643}. In the zero-offset (post-stack) case, the
velocity continuation process is governed by a partial differential
equation in midpoint, time, and velocity coordinates, first discovered
by \longcite{Claerbout.sep.48.79}. \longcite{hubral1} and
\longcite{hubral2} describe this process in a broader context of
``image waves''. Generalizations are possible for the non-zero offset
(prestack) case \cite{Fomel.sep.92.159,Fomel.segab.97}.
\par
A numerical implementation of velocity continuation process provides
an efficient method of scanning the velocity dimension in the search
of an optimally focused image. The first implementations
\cite{Li.sep.48.85,Fomel.sep.92.159} used an analogy with Claerbout's
15-degree depth extrapolation equation to construct a
finite-difference scheme with an implicit unconditionally stable
advancement in velocity. \longcite{Fomel.sep.95.sergey2} presented an
efficient three-dimensional generalization, applying the helix
transform \cite{Claerbout.sep.95.jon1}.
\par
\plot{fd-imp}{width=6in}{Impulse responses (Green's functions) of
  velocity continuation, computed by a second-order finite-difference
  method.  The left plots corresponds to continuation to a larger
  velocity ($+1$ km/sec); the right plot, smaller velocity, ($-1$
  km/sec).}
\par
A low-order finite-difference method is probably the most efficient
numerical approach to this method, requiring the least work per
velocity step. However, its accuracy is not optimal because of the
well-known numerical dispersion effect. Figure \ref{fig:fd-imp} shows
impulse responses of post-stack velocity continuation for three
impulses, computed by the second-order finite-difference method
\cite{Fomel.sep.92.159}. As expected from the residual migration
theory \cite{GEO50.01.01100126}, continuation to a higher velocity
(left plot) corresponds to migration with a residual velocity, and its
impulse responses have an elliptical shape. Continuation to a smaller
velocity (right plot in Figure \ref{fig:fd-imp}) corresponds to
demigration (modeling), and its impulse responses have a hyperbolic
shape. The dispersion artifacts are clearly visible in the figure.
\par
In this paper, I explore the possibility of implementing a numerical
velocity continuation by spectral methods. I adopted two different
methods, comparable in efficiency with finite differences.  The first
method is a direct application of the Fast Fourier Transform (FFT)
technique.  The second method transforms the time grid to Chebyshev
collocation points, which leads to an application of the
Chebyshev-$\tau$ method \cite{lanc,orszag,boyd}, combined with an
unconditionally stable implicit advancement in velocity.  Both methods
employ a transformation of the grid from time $t$ to the squared time
$\sigma = t^2$, which removes the dependence on $t$ from the
coefficients of the velocity continuation equation. Additionally, the
Fourier transform in the space (midpoint) variable $x$ takes care of
the spatial dependencies. This transform is a major source of
efficiency, because different wavenumber slices can be processed
independently on a parallel computer before transforming them back to
the physical space.
\end{comment}
\par
Applying velocity continuation to migration velocity analysis involves
the following steps: 
\begin{enumerate}
\item prestack common-offset (and common-azimuth) migration - to
  generate the initial data for continuation,
\item velocity continuation with stacking and semblance analysis across
  different offsets - to transform the offset data dimension into the velocity
  dimension,
\item picking the optimal velocity and slicing through the migrated
  data volume - to generate an optimally focused image.
\end{enumerate}
The first step transforms the data to the image space. The regularity of this
space can be exploited for devising efficient algorithms for the next two
steps. The idea of slicing through the velocity space goes back to the work of
\longcite{shurtleff}, \longcite{SEG-1984-S1.8,Fowler.sepphd.58}, and
\longcite{GEO57-01-00510059}. While the previous slicing methods constructed
the velocity space by repeated migration with different velocities, velocity
continuation navigates directly in the migration velocity space without
returning to the original data. This leads to both more efficient algorithms
and a better understanding of the theoretical continuation properties
\cite{first}.

In this paper, I demonstrate all three steps, using both synthetic data and a
North Sea dataset. I introduce and exemplify two methods for the efficient
practical implementation of velocity continuation: the finite-difference
method and the Fourier spectral method. The Fourier method is recommended as
optimal in terms of the accuracy versus efficiency trade-off. Although all the
examples in this paper are two-dimensional, the method easily extends to 3-D
under the assumption of common-azimuth geometry (one oriented offset). More
investigation may be required to extend the method to the multi-azimuth case.

It is also important to note that although the velocity continuation result
could be achieved in principle by using prestack residual migration in
Kirchhoff \cite{Etgen.sepphd.68} or frequency-wavenumber
\cite{GEO61-02-06050607} formulation, the first is inferior in efficiency, and
the second is not convenient for the conventional velocity analysis across
different offsets, because it mixes them in the Fourier domain
\cite{SEG-2000-09920995}.  \uline{Fourier-domain angle-gather analysis}
\cite{SEG-2001-02960299,sandf} \uline{opens new possibilities for the future
development of the Fourier-domain velocity continuation. New insights into the
possibility of extending the method to depth migration can follow from the
work of} \longcite{adler}.

\begin{comment}
\par
In order to understand the concept of velocity continuation, we need
to look at the fundamentals of seismic time migration. 
When post-stack migration is performed in the time domain, it
possesses peculiar properties, which can lead to a different viewpoint
on migration.  One of the most interesting properties is an ability to
decompose the time migration procedure into a cascade of two or more
migrations with smaller migration velocities. This remarkable property
is described by 
\longcite{GEO50.01.01100126} as {\em residual migration}. 
\longcite{GEO52.05.06180643} have generalized the method of
residual migration to one of {\em cascaded migration.} Cascading
finite-difference migrations overcomes the dip limitations of
conventional finite-difference algorithms \cite{GEO52.05.06180643};
cascading Stolt-type {\em f-k} migrations expands their range of
validity to the case of a vertically varying velocity
\cite{GEO53.07.08810893}. Further theoretical generalization sets the
number of migrations in a cascade to infinity, making each step in the
velocity space infinitely small. This leads to the partial
differential equation in the time-midpoint-velocity space, discovered
by Claerbout \shortcite{Claerbout.sep.48.79}. Claerbout's equation
describes the process of {\em velocity continuation,} which fills the
velocity space in the same manner as a set of constant-velocity
migrations. Slicing in the migration velocity space can serve as a
method of velocity analysis for migration with non-constant velocity
\cite{Fowler.sepphd.58}.

In this paper, I generalize the velocity continuation concept to the
case of prestack migration, connecting it with the theory of prestack
residual migration \cite{Etgen.sepphd.68}. By exploiting the
mathematical theory of characteristics, a simplified kinematic
derivation of the velocity continuation equation leads to differential
equations with reasonable dynamic properties. In practice, one can
accomplish dynamic velocity continuation by integral,
finite-difference, or Fourier-domain methods. For practical
applications, I chose the Fourier spectral method. The method has its
limitations \cite{Fomel.sep.97.sergey2}, but looks optimal in terms of
the accuracy versus efficiency trade-off.
 
Applying velocity continuation to migration velocity analysis involves
the following steps: 
\begin{enumerate}
\item prestack common-offset (and common-azimuth) migration - to
  generate the initial data for continuation,
\item velocity continuation with stacking across different offsets -
  to transform the offset data dimension into the velocity dimension,
\item picking the optimal velocity and slicing through the migrated
  data volume - to generate an optimally focused image.
\end{enumerate}
The final part of this paper includes a demonstration of all three
steps on a simple two-dimensional dataset.

It is important to note that although the velocity continuation result
could be achieved in principle by using prestack residual migration in
Kirchhoff \cite{Etgen.sepphd.68} or Stolt \cite{GEO61-02-06050607}
formulation, the first is evidently inferior in efficiency, and the
second is not convenient for velocity analysis across different
offsets, because it mixes them in the Fourier domain.

\mhead{KINEMATICS OF VELOCITY CONTINUATION}

From the kinematic point of view, it is convenient to describe the
reflector as a locally smooth surface $z = z(x)$, where $z$ is the
depth, and $x$ is the point on the surface ($x$ is a two-dimensional
vector in the 3-D problem). The image of the reflector obtained after
a common-offset prestack migration with a half-offset $h$ and a
constant velocity $v$ is the surface $z = z(x;h,v)$. Appendix A
provides the derivations of the partial differential equation
describing the image surface in the depth-midpoint-offset-velocity
space. The purpose of this section is to discuss the laws of kinematic
transformations implied by the velocity continuation equation. Later
in this paper, I obtain dynamic analogs of the kinematic
relationships in order to describe the continuation of migrated
sections in the velocity space.

The kinematic equation for prestack velocity continuation, derived in
Appendix A, takes the following form:
\begin{equation}
{{\partial \tau} \over {\partial v}} = 
v\,\tau\,\left({{\partial \tau} \over {\partial x}}\right)^2 +
{{h^2} \over {v^3\,\tau}}\,\left(1-v^4\,
\left({{\partial \tau} \over {\partial x}}\right)^2\,
\left({{\partial \tau} \over {\partial h}}\right)^2\right)\;.
\EQNLABEL{eikonal} 
\end{equation}
Here $\tau$ denotes the one-way vertical traveltime $\left(\tau = {z
\over v}\right)$. The right-hand side of equation \EQN{eikonal}
consists of three distinctive terms. Each has its own geophysical meaning. The first term is the only one remaining
when the offset $h$ equals zero. It corresponds to the procedure of
{\em zero-offset residual migration}.  Setting the reflector dip to
zero eliminates the first and third terms, leaving the second,
dip-independent one. We can associate the second term with the process of
{\em residual normal moveout}. The third term is both dip- and offset-
dependent. The process that it describes is {\em residual dip
moveout}. It is convenient to analyze each of the three processes
separately, evaluating their contributions to the cumulative process
of prestack velocity continuation.

\shead{Kinematics of Zero-Offset Velocity Continuation}
%%%%%%%%%%%%%%%%%%%%%%%%%%%%%%%%%%%%%%%%%%%%%%%%%%%%%%
The kinematic equation for zero-offset velocity continuation is
\begin{equation}
{{\partial \tau} \over {\partial v}} = 
v\,\tau\,\left({{\partial \tau} \over {\partial x}}\right)^2\;.
\EQNLABEL{POMeikonal} 
\end{equation}
The typical boundary problem associated with it is to find the
traveltime surface $\tau(x)$ for a constant velocity $v$, given the
traveltime surface $\tau_1(x_1)$ at some other velocity $v_1$. Both
surfaces correspond to the reflector images obtained by time
migrations with the specified velocities.  When the migration velocity
approaches zero, post-stack time migration approaches the identity
operator. Therefore, the case of $v_1 = 0$ corresponds kinematically
to the zero-offset (post-stack) migration, and the case of $v = 0$
corresponds to the zero-offset modeling (demigration).

The appropriate mathematical method of solving the kinematic
problem posed above is the method of characteristics \cite{kurant2}. The
characteristics of equation \EQN{POMeikonal} are the trajectories
followed by individual points of the reflector image in the velocity
continuation process, which I have called {\em velocity rays} 
\cite{me}. Velocity rays are defined by the system of ordinary
differential equations derived from \EQN{POMeikonal} according to the
classic rules of mathematical physics:
\begin{eqnarray}
{{{dx} \over {dv}} = - 2\,v\,\tau\,\tau_x} & , &
{{{d\tau} \over {dv}} = - \tau_v}\;,
\EQNLABEL{velray1} \\
{{{d\tau_x} \over {dv}} = v\,\tau_x^3} & , &
{{{d\tau_v} \over {dv}} = \left(\tau + v\,\tau_v\right)\,\tau_x^2}\;.
\EQNLABEL{velray2} 
\end{eqnarray}
An additional constraint for the quantities $\tau_x$ and $\tau_v$
follows from equation \EQN{POMeikonal}, rewritten in the form
\begin{equation}
\tau_v = v\,\tau\,\tau_x^2\;. 
\EQNLABEL{equiveikonal} 
\end{equation}
One can easily solve the system of equations \EQN{velray1} and
\EQN{velray2} by the classic mathematical methods for ordinary
differential equations. The general solution of the system takes the
parametric form
\begin{eqnarray}
x(v) & = A - C v^2\;,\;
\tau^2(v) & = B - C^2\,v^2\;,
\EQNLABEL{velrayg1} \\ 
\tau_x(v) & = {C \over {\tau(v)}}\;,\;
\tau_v(v) & = {{C^2\,v} \over {\tau(v)}}\;,
\EQNLABEL{velrayg2} 
\end{eqnarray}
where $A$, $B$, and $C$ are constant along each individual velocity
ray. These three constants are determined from the boundary conditions
as
\begin{equation}
A = x_1 + v_1^2\,\tau_1\,{{\partial \tau_1} \over {\partial x_1}} = x_0\;,
\EQNLABEL{a} 
\end{equation}
\begin{equation}
B = \tau_1^2\,\left(1 + v^2\,
\left({{\partial \tau_1} \over {\partial x_1}}\right)^2\right) = \tau_0^2\;,
\EQNLABEL{b} 
\end{equation}
\begin{equation}
C = \tau_1\,{{\partial \tau_1} \over {\partial x_1}} = 
\tau_0\,{{\partial \tau_0} \over {\partial x_0}}\;,
\EQNLABEL{c} 
\end{equation}
where $\tau_0$ and $x_0$ correspond to the zero velocity (unmigrated
section). Equations \EQN{a}, \EQN{b}, and \EQN{c} have a clear
geometric meaning illustrated in Figure \FIG{vlczor}. Noting the
simple relationship between the midpoint derivative of the vertical
traveltime and the local dip angle $\alpha$ (appendix A),
\begin{equation}
{{\partial \tau} \over {\partial x}} = 
{{\tan{\alpha}} \over v}\;,
\EQNLABEL{dtaudx} 
\end{equation}
we can see that equations \EQN{a} and \EQN{b} are precisely equivalent
to the evident geometric relationships
\begin{equation}
x + v\,\tau\,\tan{\alpha} = x_0\;,
\;{\tau \over {\cos{\alpha}}} = \tau_0\;.
\EQNLABEL{evident}
\end{equation}
Equation \EQN{c} states that the points on a velocity ray correspond
to a single reflection point, constrained by the values of $\tau$,
$v$, and $\alpha$.  As follows from equations \EQN{velrayg1}, the
projection of a velocity ray to the time-midpoint plane has the
parabolic shape $x(\tau) = A + (\tau^2 - B) / C$, which has been
noticed by Chun and Jacewitz \shortcite{GEO46.05.07170733}. On the
depth-midpoint plane, the velocity rays have the circular shape
$z^2(x) = (A - x)\,B / C - (A - x)^2$, described by Liptow and Hubral
\shortcite{them} as ``Thales circles.''

\activesideplot{vlczor}{height=2.5in}{NR}{Zero-offset reflection in a constant
velocity medium (a scheme).}

For an example of kinematic continuation by velocity rays, let us
consider the case of a point diffractor. If the diffractor location in
the subsurface is the point ${x_d,z_d}$, then the reflection traveltime at
zero offset is defined from Pythagoras's theorem as the hyperbolic
curve
\begin{equation}
\tau_0(x_0) = {{\sqrt{z_d^2 + (x_0 - x_d)^2}} \over v_d}\;,
\EQNLABEL{dift0}
\end{equation}
where $v$ is half of the actual velocity. Applying formulas
\EQN{velrayg1}, we can deduce that the velocity rays in this case have the
following mathematical expressions:
\begin{eqnarray}
x(v) & = & x_d\,{v^2 \over v_d^2} + 
x_0\,\left(1 -  {v^2 \over v_d^2}\right)\;,
\EQNLABEL{difrayg1} \\ 
\tau^2(v) & = & \tau_d^2 + {{(x_0 - x_d)^2} \over v_d^2}\,
\left(1 -  {v^2 \over v_d^2}\right)\;,
\EQNLABEL{difrayg2} 
\end{eqnarray}
where $\tau_d = {z_d \over v_d}$.
Eliminating $x_0$ from the system of equations \EQN{difrayg1} and
\EQN{difrayg2} leads to the expression for the velocity continuation
``wavefront'': 
\begin{equation}
\tau(x)=\sqrt{\tau_d^2 + {{(x - x_d)^2} \over {v_d^2 - v^2}}}\;.
\EQNLABEL{diffront}
\end{equation}
For the case of a point diffractor, the wavefront corresponds precisely
to the summation path of the residual migration operator
\cite{GEO50.01.01100126}. It has a hyperbolic shape when $v_d > v$
(undermigration) and an elliptic shape when $v_d < v$
(overmigration). The wavefront collapses to a point when the velocity
$v$ coincides with the actual effective velocity $v_d$. At zero
velocity, $v=0$, the wavefront takes the familiar form of the post-stack migration
hyperbolic summation path. The form of the velocity rays and wavefronts
is illustrated in the left plot of Figure \FIG{vlcvrs}.

\activeplot{vlcvrs}{width=5.5in,height=2.25in}{CR}{Kinematic
velocity continuation in the post-stack migration domain. Solid lines
denote wavefronts: reflector images for different migration
velocities; dashed lines denote velocity rays. Left: the case of a
point diffractor. Right: the case of a dipping plane reflector.}

Another important example is the case of a dipping plane
reflector. For simplicity, let us put the origin of the midpoint
coordinate $x$ at the point of the plane intersection with the surface
of observations. In this case, the plane reflector has the simple
expression
\begin{equation}
z_p(x) = x\,\tan{\alpha}\;,
\EQNLABEL{plane}
\end{equation}
where $\alpha$ is the dip angle. The zero-offset reflection traveltime
is the plane with a changed angle. It can be expressed as
\begin{equation}
\tau_0(x_0) = p\,x_0\;,
\EQNLABEL{plt0}
\end{equation}
where $p = {{\sin{\alpha}}\over v_p}$, and $v_p$ is half of the actual
velocity. Applying formulas \EQN{velrayg1} leads to the following
parametric expression for the velocity rays:
\begin{eqnarray}
x(v) & = & x_0\,(1 -  p^2\,v^2)\;,
\EQNLABEL{plrayg1} \\ 
\tau(v) & = & p\,x_0\,\sqrt{1 -  p^2\,v^2}\;.
\EQNLABEL{plrayg2} 
\end{eqnarray}
Eliminating $x_0$ from the system of equations \EQN{plrayg1} and \EQN{plrayg2}
shows that the velocity continuation wavefronts are planes with a
modified angle:
\begin{equation}
\tau(x)={{p\,x} \over {\sqrt{1 -  p^2\,v^2}}}\;.
\EQNLABEL{plfront}
\end{equation}
The right plot of Figure \FIG{vlcvrs} shows the geometry of the
kinematic velocity continuation for the case of a plane reflector.

\shead{Kinematics of Residual NMO}
%%%%%%%%%%%%%%%%%%%%%%%%%%%%%%%%%%
The residual NMO differential equation is the second term in
\EQN{eikonal}:
\begin{equation}
{{\partial \tau} \over {\partial v}} = 
{{h^2} \over {v^3\,\tau}}\;.
\EQNLABEL{ResNMOeikonal} 
\end{equation}
Equation \EQN{ResNMOeikonal} is independent from the midpoint
$x$. This fact indicates the one-dimensional nature of normal
moveout. The general solution of equation \EQN{ResNMOeikonal} is
obtained by simple integration. It takes the form
\begin{equation}
\tau^2(v) = C - {h^2 \over v^2} = \tau_1^2 + 
h^2\,\left({1 \over v_1^2} - {1 \over v^2}\right)\;,
\EQNLABEL{ResNMO} 
\end{equation}
where $C$ is an arbitrary velocity-independent constant, and I have
chosen the constants $\tau_1$ and $v_1$ so that $\tau(v_1) = \tau_1$.

For the case of a point diffractor, solution \EQN{ResNMO} easily
combines with the zero-offset solution \EQN{diffront}. The result is a
simplified version of the prestack residual migration summation path:
\begin{equation}
\tau(x)=\sqrt{\tau_d^2 + 
{{(x - x_d)^2} \over {v_d^2 - v^2}} +
h^2\,\left({1 \over v_d^2} - {1 \over v^2}\right)}\;.
\EQNLABEL{PSRMfront}
\end{equation}
Summation paths of the form \EQN{PSRMfront} for a set of diffractors
with different depths are plotted in Figures \FIG{vlcve1} and
\FIG{vlcve2}. The parameters chosen in these plots allow a direct
comparison with Etgen's Figures 2.4 and 2.5 \cite{Etgen.sepphd.68},
based on the exact solution and reproduced in Figures \FIG{vlcve3} and
\FIG{vlcve4}. The comparison shows that the approximate
solution \EQN{PSRMfront} captures the main features of the prestack
residual migration operator, except for the residual DMO cusps
appearing in the exact solution when the diffractor depth is smaller
than the offset.

\activesideplot{vlcve1}{height=2in}{CR}{Summation paths of the
simplified prestack residual migration for a series of depth
diffractors. Residual slowness $v/v_d$ is 1.2; offset $h$ is 1
km. This figure is to be compared with Etgen's Figure 2.4.}

\activesideplot{vlcve2}{height=2in}{CR}{Summation paths of the simplified
prestack residual migration for a series of depth diffractors. Residual
slowness $v/v_d$ is 0.8; offset $h$ is 1 km. This figure is to be
compared with Etgen's Figure 2.5.}

Neglecting the residual DMO term in residual migration is
approximately equivalent in accuracy to neglecting the DMO step in
conventional processing. Indeed, as follows from the geometric analog
of equation \EQN{eikonal} derived in Appendix A, dropping the residual
DMO term corresponds to the condition
 \begin{equation}
\tan^2{\alpha}\,\tan^2{\gamma} \ll 1\;,
\EQNLABEL{tatg}
\end{equation}
where $\alpha$ is the dip angle, and $\gamma$ is the reflection angle.
As shown by Yilmaz and Claerbout \shortcite{GEO45.12.17531779}, the
conventional processing sequence without the DMO step corresponds to
the separable approximation of the double-square-root equation \EQN{DSR}:
\begin{equation}
\sqrt{1 -  v^2\,\left({{\partial t} \over {\partial s}}\right)^2} +
\sqrt{1 -  v^2\,\left({{\partial t} \over {\partial r}}\right)^2}
\approx
2\,\sqrt{1 -  v^2\,\left({{\partial t} \over {\partial x}}\right)^2} +
2\,\sqrt{1 -  v^2\,\left({{\partial t} \over {\partial h}}\right)^2} -
2\;.
\EQNLABEL{Sep}
\end{equation}
In geometric terms, approximation \EQN{Sep} transforms to
\begin{equation}
\cos{\alpha}\,\cos{\gamma}
\approx
\sqrt{1 -  \sin^2{\alpha}\,\cos^2{\gamma}} +
\sqrt{1 -  \sin^2{\gamma}\,\cos^2{\alpha}} - 1\;.
\EQNLABEL{Sepgeom}
\end{equation}
Estimating the accuracy of the separable approximation by the
first term of the Taylor series for small $\alpha$ and $\gamma$ yields
the estimate of ${3 \over 4}\,\tan^2{\alpha}\,\tan^2{\gamma}$
\cite{GEO45.12.17531779}, which agrees qualitatively with
\EQN{tatg}. Though approximation \EQN{PSRMfront} fails in 
situations where the dip moveout correction is necessary, it is 
significantly more accurate than the 15-degree approximation of the
double-square-root equation, implied in the migration velocity
analysis method of Yilmaz and Chambers \shortcite{GEO49.10.16641674}
and MacKay and Abma \shortcite{abma}. The 15-degree approximation
\begin{equation}
\sqrt{1 -  v^2\,\left({{\partial t} \over {\partial s}}\right)^2} +
\sqrt{1 -  v^2\,\left({{\partial t} \over {\partial r}}\right)^2}
\approx
2 - {v^2 \over 2}\,
\left(  \left({{\partial t} \over {\partial s}}\right)^2 +
        \left({{\partial t} \over {\partial s}}\right)^2\right)
\EQNLABEL{FDeg}
\end{equation}
corresponds geometrically to the equation 
\begin{equation}
2\,\cos{\alpha}\,\cos{\gamma}
\approx
{{3 + \cos{2\alpha}\,\cos{2\gamma}} \over 2}\;.
\EQNLABEL{FDgeom}
\end{equation} 
Its estimated accuracy is ${1 \over 8}\,\tan^2{\alpha} + {1 \over
8}\,\tan^2{\gamma}$.  Unlike the separable approximation, which is
accurate separately for zero offset and zero dip, the 15-degree
approximation fails at zero offset in the case of a steep dip and at zero
dip in the case of a large offset.

\shead{Kinematics of Residual DMO}   
%%%%%%%%%%%%%%%%%%%%%%%%%%%%%%%%%%
The partial differential equation for kinematic residual DMO is the
third term in \EQN{eikonal}:
\begin{equation}
{{\partial \tau} \over {\partial v}} = 
- {{h^2 v} \over {\tau}}\,
\left({{\partial \tau} \over {\partial x}}\right)^2\,
\left({{\partial \tau} \over {\partial h}}\right)^2\;.
\EQNLABEL{ResDMOeikonal} 
\end{equation}
It is more convenient to consider the residual dip-moveout process
coupled with residual normal moveout. Etgen
\shortcite{Etgen.sepphd.68} describes this procedure as the cascade of
inverse DMO with the initial velocity $v_0$, residual NMO, and DMO
with the updated velocity $v_1$. The kinematic equation for residual
NMO+DMO is the sum of the two terms in \EQN{eikonal}:
\begin{equation}
{{\partial \tau} \over {\partial v}} = 
{{h^2} \over {v^3\,\tau}}\,\left(1-v^4\,
\left({{\partial \tau} \over {\partial x}}\right)^2\,
\left({{\partial \tau} \over {\partial h}}\right)^2\right)\;.
\EQNLABEL{DMONMOeikonal} 
\end{equation}
If the boundary data for equation \EQN{DMONMOeikonal} are on a
common-offset gather, it is appropriate to rewrite this equation
purely in terms of the midpoint derivative ${{\partial \tau} \over
{\partial x}}$, eliminating the offset-derivative term ${{\partial
\tau} \over {\partial h}}$. The resultant expression, derived in
Appendix A, has the form
\begin{equation}
v^3\,{{\partial \tau} \over {\partial v}} = 
{{2\,h^2} \over
{\sqrt{\tau^2 + 4\,h^2\,
Q\left(v,{{\partial \tau} \over {\partial x}}\right)} + \tau}}\;,
\EQNLABEL{noth} 
\end{equation}
where 
\begin{equation}
Q(v,\tau_x) = {{\tau_x^2} \over 
{\left(1 + v^2\,\tau_x^2\right)^2}}\;.   
\EQNLABEL{qtx} 
\end{equation}
The direct solution of equation \EQN{noth} is nontrivial. A simpler
way to obtain this solution is to decompose residual NMO+DMO into
three steps and to evaluate their contributions separately. Let the
initial data be the zero-offset reflection event $\tau_0(x_0)$. The
first step of the residual NMO+DMO is the inverse DMO operator. One
can evaluate the effect of this operator by means of the offset
continuation concept \cite{Fomel.sep.84.179}. According to this
concept, each point of the input traveltime curve $\tau_0(x_0)$
travels with the change of the offset from zero to $h$ along a special
trajectory, which I call a {\em time ray}. Time rays are parabolic
curves of the form
\begin{equation}
x\left(\tau\right) =  x_0+{{\tau^2-\tau_0^2\left(x_0\right)} \over
{\tau_0\left(x_0\right)\,\tau_0'\left(x_0\right)}}\;,
\EQNLABEL{xoftau}
\end{equation}
with the final points constrained by the equation
\begin{equation}
h^2 = \tau^2\,{{\tau^2-\tau_0^2\left(x_0\right)} \over
{\left(\tau_0\left(x_0\right)\,\tau_0'\left(x_0\right)\right)^2}}\;.
\EQNLABEL{hoftau}
\end{equation}
The second step of the cumulative residual NMO+DMO process is the
residual normal moveout. According to equation \EQN{ResNMO}, residual
NMO is a one-trace operation transforming the traveltime $\tau$ to
$\tau_1$ as follows:
\begin{equation}
\tau_1^2 = \tau^2 + h^2\,s\;,
\EQNLABEL{ResNMO2} 
\end{equation}
where 
\begin{equation}
s = {1 \over v_0^2} - {1 \over v_1^2}\;.
\EQNLABEL{s} 
\end{equation}
The third step is dip moveout corresponding to the new velocity
$v_1$. DMO is the offset continuation from $h$ to zero
offset along the redefined time rays \cite{Fomel.sep.84.179}
\begin{equation}
x_2\left(\tau_2\right) =  
x + {{h\,X} \over {\tau_1^2\,H}}\,\left(\tau_1^2-\tau_2^2\right)\;,
\EQNLABEL{xoftau2}
\end{equation}
where $H = {{\partial \tau_1} \over {\partial h}}$, and $X =
{{\partial \tau_1} \over {\partial x}}$. 
The end points of the time rays \EQN{xoftau2} are defined by the
equation
\begin{equation}
\tau_2^2 = - \tau_1^2\,{{\tau_1\,H} \over {h\,X^2}}\;.
\EQNLABEL{tauofh2}
\end{equation}
The partial derivatives of the common-offset traveltimes are
constrained by the offset continuation kinematic equation
\begin{equation}
h\,(H^2 - Y^2) = \tau_1\,H\;,
\EQNLABEL{OCequation}
\end{equation}
which is equivalent to equation \EQN{OCeikonal} in Appendix
A. Additionally, as follows from equations \EQN{ResNMO2} and the ray
invariant equations from \cite{Fomel.sep.84.179},
\begin{equation}
\tau_1\,X = \tau\,{{\partial \tau} \over {\partial x}} = 
{{\tau^2\,\tau_0'\left(x_0\right)} \over {\tau_0\left(x_0\right)}}\;.
\EQNLABEL{invariant}
\end{equation}
Substituting \EQN{xoftau}, \EQN{hoftau}, \EQN{ResNMO2},
\EQN{OCequation}, and \EQN{invariant} into equations \EQN{xoftau2} and
\EQN{tauofh2} and performing the algebraic simplifications, we arrive
at the parametric expressions for velocity rays of the residual
NMO+DMO process:
\begin{equation}
\left\{
\begin{array}{rcl}
x_2(s) & = & \displaystyle{x_0 + {{h^2\,\tau_0'(x_0)} \over T}\,
\left(1 - {T^2 \over T_2^2(s)}\right)}\;,
\\ 
\\
\tau(s) & = & \displaystyle{{{\tau_1^2(s)} \over {T_2(s)}}}\;,
\end{array}
\right.
\EQNLABEL{ResDMO2}
\end{equation}
where the function
$T\left(h,\tau_0(x_0),\tau_0'\left(x_0\right)\right)$ is defined by
\begin{equation}
T\left(h,\tau,\tau_x\right) = 
{{\tau + \sqrt{\tau^2 + 4\,h^2\,\tau_x^2}} \over 2}\;,
\EQNLABEL{CapT}
\end{equation}
\begin{equation}
T_2(s) = \sqrt{T\left(h,\tau_1^2(s),\tau_0'\left(x_0\right)\,
T\left(h,\tau_0(x_0),\tau_0'\left(x_0\right)\right)\right)}\;,
\EQNLABEL{CapT2}
\end{equation}
and
\begin{equation}
\tau_1^2(s) = \tau_0\,T + s\,h^2\;.
\EQNLABEL{tauofs}
\end{equation}

The last step of the cascade of inverse DMO, residual NMO, and DMO is
illustrated in Figure \FIG{vlcvoc}. The three plots in the figure show the
offset continuation to zero offset of the inverse DMO impulse response
shifted by the residual NMO operator. The middle plot corresponds to zero NMO shift, for which the DMO step collapses the wavefront back to a point.
Both positive (top plot) and negative (bottom plot) NMO shifts
 result in the formation of the specific triangular impulse
response of the residual NMO+DMO operator. As noticed by Etgen
\shortcite{Etgen.sepphd.68}, the size of the ``triangle'' operators
dramatically decreases with the time increase. For large times
(pseudo-depths) of the initial impulses, the operator collapses to a
point corresponding to the pure NMO shift. This fact agrees with the
conclusions of the preceding subsection about the comparative
importance
of
the
residual
DMO
term.
It
is
illustrated
in
Figure
\FIG{vlcvcp}
with
the
theoretical
impulse response curves, and in Figure \FIG{vlccps} with the result of
an actual cascade of the inverse DMO, residual NMO, and DMO operators.

\activesideplot{vlcvoc}{height=5.5in}{CR}{Kinematic residual NMO+DMO
operators constructed by the cascade of inverse DMO, residual NMO, and
DMO. The impulse response of inverse DMO is shifted by the residual
DMO procedure. Offset continuation back to zero offset forms the
impulse response of the residual NMO+DMO operator. Solid lines denote
traveltime curves; dashed lines denote the offset continuation
trajectories (time rays). Top plot: $v_1/v_0 = 1.2$. Middle plot:
$v_1/v_0 = 1$; the inverse DMO impulse response collapses back to the
initial impulse. Bottom plot: $v_1/v_0 = 0.8$. The half-offset $h$ in
all three plots is 1 km.}

\activeplot{vlcvcp}{width=6in,height=3.5in}{CR}{Theoretical
kinematics of the residual NMO+DMO impulse responses for three
impulses. Left plot: the velocity ratio $v_1/v_0$ is $1.333$. Right
plot: the velocity ratio $v_1/v_0$ is $0.833$. In both cases the
half-offset $h$ is 1 km.}

\activeplot{vlccps}{width=6in,height=3.5in}{ER}{The result of
residual NMO+DMO (cascading inverse DMO, residual NMO, and DMO) for
three impulses. Left plot: the velocity ratio $v_1/v_0$ is
$1.333$. Right plot: the velocity ratio $v_1/v_0$ is $0.833$. In both
cases the half-offset $h$ is 1 km.}

Figure \FIG{vlcvrd} illustrates the residual NMO+DMO velocity
continuation for two particularly interesting cases. The left plot
shows the continuation for a point diffractor. One can see that when
the velocity error is large, focusing of the velocity rays forms a
specific loop on the zero-offset hyperbola. The right plot illustrates
the case of a plane dipping reflector. The image of the reflector
shifts both vertically and laterally with the change in NMO
velocity.

\activeplot{vlcvrd}{width=5.5in,height=2.25in}{CR}{Kinematic
velocity continuation for residual NMO+DMO. Solid lines denote
wavefronts: zero-offset traveltime curves; dashed lines denote
velocity rays. Left plot: the case of a point diffractor; the velocity
ratio $v_1/v_0$ changes from $0.9$ to $1.1$. Right plot:
the case of a dipping plane reflector; the velocity
ratio $v_1/v_0$ changes from $0.8$ to $1.2$. In both cases, the
half-offset $h$ is 2 km.}

The full residual migration operator is the result of cascading
residual zero-offset migration and residual NMO+DMO. I illustrate the
kinematics of this operator in Figures \FIG{vlcve3} and \FIG{vlcve4},
which are designed to match Etgen's Figures 2.4 and 2.5
\cite{Etgen.sepphd.68}. A comparison with Figures \FIG{vlcve1} and
\FIG{vlcve2} shows that including the residual DMO term affects
the images of shallow objects (with the depth smaller than the offset
$h$) and complicates the residual migration operator with cusps.

\activesideplot{vlcve3}{height=2in}{CR}{Summation paths of prestack
residual migration for a series of depth diffractors. Residual
slowness $v/v_d$ is 1.2; offset $h$ is 1 km. This figure reproduces
Etgen's Figure 2.4.}

\activesideplot{vlcve4}{height=2in}{CR}{Summation paths of prestack
residual migration for a series of depth diffractors. Residual
slowness $v/v_d$ is 0.8; offset $h$ is 1 km. This figure reproduces
Etgen's Figure 2.5.}

\mhead{FROM KINEMATICS TO DYNAMICS}
%%%%%%%%%%%%%%%%%%%%%%%%%%%%%%%%%%%
The theory of characteristics \cite{kurant2} states that if a partial
differential equation has the form
\begin{equation}
\sum_{i,j=1}^{n}\,\Lambda_{ij}(\xi_1,\ldots,\xi_n)\,
{{\partial^2 P} \over {\partial \xi_i\,\partial \xi_j}} +
F\left(\xi_1,\ldots,\xi_n,P,
{{\partial P} \over {\partial \xi_1}},\ldots,
{{\partial P} \over {\partial \xi_n}}\right) = 0\;,
\EQNLABEL{gequation}
\end{equation}
where F is some arbitrary function, and if the eigenvalues of the matrix
$\Lambda$ are nonzero, and one of them is different in sign from the
others, then equation \EQN{gequation} describes a wave-type process, and its
kinematic counterpart is the characteristic equation
\begin{equation}
\sum_{i,j=1}^{n}\,\Lambda_{ij}(\xi_1,\ldots,\xi_n)\,
{{\partial \psi} \over {\partial \xi_i}}\, 
{{\partial \psi} \over {\partial \xi_j}} = 0
\EQNLABEL{charequation}
\end{equation}
with the characteristic surface 
\begin{equation}
\psi(\xi_1,\ldots,\xi_n) = 0
\EQNLABEL{charsurface}
\end{equation}
corresponding to the wavefront. In velocity continuation problems,
it is appropriate to choose the variable $\xi_1$ to denote the time
$t$, $\xi_2$ to denote the velocity $v$, and the rest of the
$\xi$-variables to denote one or two lateral coordinates $x$. Without
loss of generality, we can set the characteristic surface to be
\begin{equation}
\psi = t - \tau(x;v) = 0\;,
\EQNLABEL{chartau}
\end{equation}
and use the theory of characteristics to reconstruct the main
(second-order) part of the dynamic differential equation from the
corresponding kinematic equations. As in the preceding section, it is
convenient to consider separately the three different components of the prestack
velocity continuation process.

\shead{Dynamics of Zero-Offset Velocity Continuation}
%%%%%%%%%%%%%%%%%%%%%%%%%%%%%%%%%%%%%%%%%%%%%%%%%%%%%
In the case of zero-offset velocity continuation, the characteristic
equation is reconstructed from equation \EQN{POMeikonal} to have the form
\begin{equation}
{{\partial \psi} \over {\partial v}}\,
{{\partial \psi} \over {\partial t}} +
v\,t\,\left({{\partial \psi} \over {\partial x}}\right)^2 = 0\;.
\EQNLABEL{POMchar} 
\end{equation}
According to formula \EQN{gequation}, the corresponding dynamic equation is
\begin{equation}
{{\partial^2 P} \over {\partial v\, \partial t}} +
v\,t\,{{\partial^2 P} \over {\partial x^2}} +
F\left(x,t,v,P,
{{\partial P} \over {\partial t}},
{{\partial P} \over {\partial v}},
{{\partial P} \over {\partial x}}
\right) = 0\;,
\EQNLABEL{POMequation} 
\end{equation}
where the function $F$ remains to be defined. The simplest case of $F$
equal to zero corresponds to Claerbout's velocity continuation
equation \cite{Claerbout.sep.48.79}, derived in a different way. Levin
\shortcite{Levin.sep.48.101} provides the dispersion-relation
derivation, conceptually analogous to applying the method of
characteristics.

In high-frequency asymptotics, the wavefield $P$ can be
represented by the ray-theoretical (WKBJ) approximation,
\begin{equation}
P(t,x,v) \approx A(x,v)\,f\left(t - \tau(x,v)\right)\;, 
\EQNLABEL{WKB} 
\end{equation}
where $A$ is the amplitude, $f$ is the short (high-frequency) wavelet,
and the function $\tau$ satisfies the kinematic equation
\EQN{POMeikonal}. Substituting approximation \EQN{WKB} into the dynamic
velocity continuation equation \EQN{POMequation}, collecting the
leading-order terms, and neglecting the $F$ function, we arrive at the
partial differential equation for amplitude transport:
\begin{equation}
{\partial A \over \partial v} = v\,\tau\,\left(2\,
{\partial A \over \partial x}\,
{\partial \tau \over \partial x} + A\,
{\partial^2 \tau \over \partial x^2}\right)\;.
\EQNLABEL{PAMPequation} 
\end{equation}
The general solution of equation \EQN{PAMPequation} follows from the
theory of characteristics. It takes the form
\begin{equation}
A(x,v) = A_0(x_0)\,\exp{\left(\int_0^{v}\,u\,\tau(x,u)\,
{\partial^2 \tau(x,u) \over \partial x^2}\,du\right)}\;,
\EQNLABEL{PAMPsolution} 
\end{equation}
where $A_0(x_0) = A(x,0)$, and the integral corresponds to the
curvilinear integration along the corresponding velocity ray.
In the case of a plane dipping reflector, the image of the reflector remains
plane in the velocity continuation process. Therefore, the second
traveltime derivative ${\partial^2 \tau(x,u) \over \partial x^2}$ in
\EQN{PAMPsolution} equals zero, and the exponential is equal to
one. This means that the amplitude of the image does not change with
the velocity along the velocity rays. This fact does not agree with the
theory of conventional post-stack migration, which suggests
downscaling the image by the ``cosine'' factor $\tau_0 \over
\tau$ \cite{GEO46.05.07170733,Levin.sep.48.147}. The simplest way to
include the cosine factor in the velocity continuation equation is to
set the function $F$ to be ${1 \over t}\,{\partial P \over \partial
v}$. The resulting differential equation
\begin{equation}
{{\partial^2 P} \over {\partial v\, \partial t}} +
v\,t\,{{\partial^2 P} \over {\partial x^2}} +
{1 \over t}\,{\partial P \over \partial v} = 0
\EQNLABEL{POMequation2} 
\end{equation}
has the amplitude transport
\begin{equation}
A(x,v) = {\tau_0 \over \tau}\,A_0(x_0)\,
\exp{\left(\int_0^{v}\,u\,\tau(x,u)\,
{\partial^2 \tau(x,u) \over \partial x^2}\,du\right)}\;,
\EQNLABEL{PAMPsolution2} 
\end{equation}
corresponding to the differential equation
\begin{equation}
{\partial A \over \partial v} = v\,\tau\,\left(2\,
{\partial A \over \partial x}\,
{\partial \tau \over \partial x} + A\,
{\partial^2 \tau \over \partial x^2}\right) - 
A\,{1 \over \tau}\,{\partial \tau \over \partial v}\;.
\EQNLABEL{PAMPequation2} 
\end{equation}
Appendix B proves that the time-and-space solution of the dynamic
velocity continuation equation \EQN{POMequation2} coincides with the
conventional Kirchhoff migration operator.

The finite-difference implementation of zero-offset velocity
continuation resembles the implementation of Claerbout's
15-degree equation in a retarded coordinate system
\cite{Claerbout.blackwell.76}. This implementation is discussed in
more detail in Appendix C. 

\shead{Dynamics of Residual NMO}
%%%%%%%%%%%%%%%%%%%%%%%%%%%%%%%%
According to the theory of characteristics, described in the beginning
of this section, the kinematic residual NMO equation
\EQN{ResNMOeikonal} corresponds to the dynamic equation of the form
\begin{equation}
{{\partial P} \over {\partial v}} + 
{{h^2} \over {v^3\,t}}\,{{\partial P} \over {\partial t}}\;,
\EQNLABEL{ResNMOdyn} 
\end{equation}
whose general solution is easily found to be
\begin{equation}
P(t,x,v) = \phi\left(t^2 + {h^2 \over v^2}\right)\;,
\EQNLABEL{ResNMOsol} 
\end{equation}
where $\phi$ is an arbitrary smooth function.
The combination of dynamic equations \EQN{ResNMOdyn} and
\EQN{POMequation2} leads to an approximate prestack velocity
continuation with the residual DMO effect neglected. To accomplish the
combination, we can simply add the term ${{h^2} \over
{v^3\,t}}\,{{\partial^2 P} \over {\partial t^2}}$ to the left-hand
side of equation \EQN{POMequation2}. This addition changes the
kinematics of velocity continuation, but does not change the amplitude
properties embedded in the transport equation \EQN{PAMPsolution2}.

\shead{Dynamics of Residual DMO}
%%%%%%%%%%%%%%%%%%%%%%%%%%%%%%%%
The case of residual DMO complicates the building of a dynamic equation
because of the essential nonlinearity of the kinematic equation
\EQN{noth}. One possible way to linearize the problem is to increase
the order of the equation. In this case, the resultant dynamic
equation would include a term that has the second-order derivative with
respect to velocity $v$. Such an equation describes two different
modes of wave propagation and requires additional initial conditions
to separate them. Another possible way to linearize equation
\EQN{noth} is to approximate it at small dip angles. For example, one
can obtain a recursively accurate approximation by a continuous
fraction expansion of the square root in equation \EQN{noth},
analogously to Muir's method in conventional finite-difference
migration \cite{Claerbout.blackwell.85}. In this case, the dynamic
equation would contain only the first-order derivative with respect to
the velocity and high-order derivatives with respect to the other
parameters. The third, and probably the most attractive, method is to
change the domain of consideration. For example, we could switch from
the common-offset domain to the domain of common offset dip. This
method implies a transformation similar to slant stacking of
common-midpoint gathers in the post-migration domain in order to
obtain the local offset dip information. Equation \EQN{noth}
transforms, with the help of the results from Appendix A, to the form
\begin{equation}
v^3\,{{\partial \tau} \over {\partial v}} = 
{{\tau\,\sin^2{\gamma}} \over
{\cos^2{\alpha} - \sin^2{\gamma}}}\;,
\EQNLABEL{noh} 
\end{equation}
with
\begin{equation}
\cos^2{\alpha} = \left(1 + v^2 \,
\left({{\partial \tau} \over {\partial x}}\right)^2\right)^{-1}\;,
\EQNLABEL{cos2a} 
\end{equation}
and
\begin{equation}
\sin^2{\alpha} = v^2\,
\left({{\partial \tau} \over {\partial h}}\right)^2\,
\left(1 + v^2 \,
\left({{\partial \tau} \over {\partial h}}\right)^2\right)^{-1}\;.
\EQNLABEL{sin2g} 
\end{equation}
For a constant offset dip $\tan{\gamma} = v\,{{\partial \tau} \over
{\partial h}}$, the dynamic analog of equation \EQN{noh} is the
third-order partial differential equation
\begin{equation}
v\,  \cot^2{\gamma}\,{{\partial^3 P} \over {\partial t^2\, \partial v}} -
v^3\,{{\partial^3 P} \over {\partial x^2\, \partial v}}
+ t\,{{\partial^3 P} \over {\partial t^2\, \partial v}} +
v^2\,t\,{{\partial^3 P} \over {\partial x^2\, \partial t}} = 0\;.
\EQNLABEL{ResDMOdyn} 
\end{equation}
Equation \EQN{ResDMOdyn} does not strictly comply with the theory of
second-order linear differential equations. Its properties and
practical applicability require further research.

\section{Conclusions}
%%%%%%%%%%%%%%%%%%%%%
I have derived kinematic and dynamic equations for residual time migration
in the form of a continuous velocity continuation process. This
derivation explicitly decomposes prestack
velocity continuation into three parts corresponding to
zero-offset continuation, residual NMO, and residual DMO. These three
parts can be treated separately both for simplicity of theoretical
analysis and for practical purposes. It is important to note that in
the case of a three-dimensional migration, all three components of
velocity continuation have different dimensionality. Zero-offset
continuation is fully 3-D. It can be split into two 2-D continuations
in the in- and cross-line directions. Residual DMO is a
two-dimensional common-azimuth process. Residual NMO is a 1-D
single-trace procedure.

The dynamic properties of zero-offset velocity continuation are
precisely equivalent to those of conventional post-stack migration
methods such as Kirchhoff migration. Moreover, the Kirchhoff migration
operator coincides with the integral solution of the velocity
continuation differential equation for continuation from the zero
velocity plane.

This rigorous theory of velocity continuation can give us new insights
into the methods of prestack migration velocity analysis. However, its
practical applicability faces several important problems. One of them
concerns the comparative value of the residual DMO term. Another
problem is the choice of the implementation method for velocity
continuation. Both these problems require further research efforts.

I have applied two spectral methods for a numerical solution of the
velocity continuation problem.
\par
The Fourier method is attractive because of its numerical efficiency.
However, it requires additional computational effort to suppress
numerical artifacts: the inaccuracy of the grid transform and
the artificial periodicity in the physical space.
\par
The Chebyshev-$\tau$ method is free of most of these difficulties,
although its overall efficiency can be slightly inferior to that of
the Fourier method.
\par
Both methods possess a ``spectral'' accuracy, which is highly desired
if accuracy is a concern.
\par
\plot{mig-impl}{width=6in}{Top left: synthetic model (the ideal image). 
  Top right: synthetic data. 
  Bottom left: the result of velocity continuation 
  with the Fourier method. Bottom right: the result of 
  velocity continuation with the Chebyshev method.}
\par
Figure \ref{fig:mig-impl} compares the results of velocity
continuation with different methods. The top left plot shows an
implied subsurface model (an ``ideal image''). The top right plot is
the corresponding synthetic data. The bottom left plot is the output
of the Fourier method, and the bottom right plot is the output of the
Chebyshev method. The Fourier result shows a poor quality in the
shallow part (caused by subsampling in the $t^2$ grid). The wraparound
artifacts were suppressed by a zero-padding correction. The quality of
the Chebyshev result is noticeably higher. It is close to the best
possible accuracy, under the natural limitations of seismic resolution.

I have demonstrated an application of velocity continuation to
migration velocity analysis on simple data sets. 


\section{Acknowledgments}
%%%%%%%%%%%%%%%%%%%%%%%%%
I thank Bee Bednar, Biondo Biondi, Jon Claerbout, Sergey Goldin, Bill
Harlan, David Lumley, and Bill Symes for useful and stimulating
discussions. 

\APPENDIX{A}
%%%%%%%%%%%%
\mhead{DERIVING THE KINEMATIC EQUATIONS}
%%%%%%%%%%%%%%%%%%%%%%%%%%%%%%%%%%%%%%%%
The main goal of this appendix is to derive the partial differential
equation describing the image surface in a
depth-midpoint-offset-velocity space.

\activesideplot{vlcray}{height=2.5in}{NR}{Reflection rays in a constant
velocity medium (a scheme).}

The derivation starts with observing a simple geometry of reflection
in a constant-velocity medium, shown in Figure \FIG{vlcray}. The
well-known equations for the apparent slowness
\begin{equation}
{{\partial t} \over {\partial s}} \,=\,
{ {\sin{\alpha_1}} \over {v}}\;,
\EQNLABEL{snell1}
\end{equation}
\begin{equation}
{{\partial t} \over {\partial r}} \,=\, 
{{\sin{\alpha_2}} \over {v}}  
\EQNLABEL{snell2}
\end{equation} 
relate the first-order traveltime derivatives for the reflected waves
to the emergency angles of the incident and reflected rays. Here $s$
stands for the source location at the surface, $r$ is the receiver
location, $t$ is the reflection traveltime, $v$ is the constant
velocity, and $\alpha_1$ and $\alpha_2$ are the angles shown in Figure
\FIG{vlcray}. Considering the traveltime derivative with respect to
the depth of the observation surface $z$, we can see that the
contributions of the two branches of the reflected ray, added
together, form the equation
\begin{equation}
- {{\partial t} \over {\partial z}} \,=\,
{{\cos{\alpha_1}} \over {v}} +
{{\cos{\alpha_2}} \over {v}}\;.
\EQNLABEL{snell3}
\end{equation}
It is worth mentioning that the elimination of angles from equations
\EQN{snell1}, \EQN{snell2}, and \EQN{snell3} leads to the famous {\em
double-square-root equation,}
\begin{equation}
- v\,{{\partial t} \over {\partial z}} \,=\,
\sqrt{1 -  v^2\,\left({{\partial t} \over {\partial s}}\right)^2} +
\sqrt{1 -  v^2\,\left({{\partial t} \over {\partial r}}\right)^2}\;,
\EQNLABEL{DSR}
\end{equation}
published in the Russian literature by Belonosova and Alekseev
\shortcite{alekseev} and commonly used in the form of a
pseudo-differential dispersion relation
\cite{Clayton.sep.14.21,Claerbout.blackwell.85} for prestack migration
\cite{Yilmaz.sepphd.18,Popovici.sep.84.53}. Considered locally,
equation \EQN{DSR} is independent of the constant velocity assumption
and enables prestack downward continuation of reflected waves in
heterogeneous media.

Introducing midpoint coordinate $x = {{s+ r} \over 2}$ and half-offset
$h = {{r - s} \over 2}$, we can apply the chain rule and elementary
trigonometric equalities to formulas \EQN{snell1} and
\EQN{snell2} and transform these formulas to 
\begin{equation}
{{\partial t} \over {\partial x}} \,=\, 
{{\partial t} \over {\partial s}} + 
{{\partial t} \over {\partial r}} \,=\, 
{ {2 \sin{\alpha}\,\cos{\gamma}} \over {v}}\;,
\EQNLABEL{snells1}
\end{equation}
\begin{equation}
{{\partial t} \over {\partial h}} \,=\,
{{\partial t} \over {\partial r}} - 
{{\partial t} \over {\partial s}} \,=\, 
{ {2 \cos{\alpha}\,\sin{\gamma}} \over {v}} \;,
\EQNLABEL{snells2}
\end{equation}
where $\alpha = {{\alpha_1 + \alpha_2} \over 2}$ is the dip angle, and
$\gamma = {{\alpha_2 - \alpha_1} \over 2}$ is the reflection angle
\cite{Clayton.sep.14.21,Claerbout.blackwell.85}. Equation
\EQN{snell3} transforms analogously to
\begin{equation}
- {{\partial t} \over {\partial z}} \,=\,
{{2 \cos{\alpha} \cos{\gamma}} \over {v}}\;. 
\EQNLABEL{snells3}
\end{equation}
This form of equation \EQN{snell3} is used to describe the stretching
factor of the waveform distortion in depth migration \cite{Tygel}.

Dividing \EQN{snells1} and \EQN{snells2} by \EQN{snells3}, we obtain
\begin{equation}
{{\partial z} \over {\partial x}} \,=\,
- \tan{\alpha}\;, 
\EQNLABEL{snellz1}
\end{equation}
\begin{equation}
{{\partial z} \over {\partial h}} \,=\,
- \tan{\gamma}\;.
\EQNLABEL{snellz2}
\end{equation}
Substituting formulas \EQN{snellz1} and \EQN{snellz2} into equation
\EQN{snells3} yields yet another form of the double-square-root equation:
\begin{equation}
- {{\partial t} \over {\partial z}} \,=\, {2 \over {v}}\,
\sqrt{1 + \left({\partial z} \over {\partial x}\right)^2}\,
\sqrt{1 + \left({\partial z} \over {\partial h}\right)^2}\;, 
\EQNLABEL{snellz3}
\end{equation}
which is analogous to the dispersion relationship of Stolt prestack
migration \cite{GEO43.01.00230048}.
 
The law of sines in the triangle formed by the incident and reflected
ray leads to the explicit relationship between the traveltime and the
offset:
\begin{equation}
v\,t = 2\,h\,  {{\cos{\alpha_1}+ \cos{\alpha_2}} \over
\sin{\left(\alpha_2-\alpha_1\right)}} = 2\,h\,{\cos{\alpha} \over
\sin{\gamma}} \;.
\EQNLABEL{length} 
\end{equation}
The combination of formulas \EQN{length}, \EQN{snells1}, and
\EQN{snells2} forms the basic kinematic equation of the offset
continuation theory \cite{Fomel.sep.84.179}:
\begin{equation}
{{\partial t} \over {\partial h}} \,
\left(t^2 + {{4\,h^2} \over {v^2}}\right)\,=\,
h\,t\,\left({4 \over {v^2}} + 
\left({{\partial t} \over {\partial h}}\right)^2\,-
\left({{\partial t} \over {\partial x}}\right)^2\right)\;.
\EQNLABEL{OCeikonal}
\end{equation}

Differentiating \EQN{length} with respect to the velocity $v$ yields
\begin{equation}
- v^2\,{{\partial t} \over {\partial v}} = 
2\,h\,{\cos{\alpha} \over \sin{\gamma}}\;.
\EQNLABEL{dtdv} 
\end{equation}
Finally, dividing \EQN{dtdv} by \EQN{snells3}, we get
\begin{equation}
v\,{{\partial z} \over {\partial v}} = 
{h \over {\cos{\gamma}\,\sin{\gamma}}}\;.
\EQNLABEL{dzdv} 
\end{equation}
Equation \EQN{dzdv} can be written in a variety of ways with the help
of an explicit geometric relationship between the half-offset $h$ and
the depth $z$, 
\begin{equation}
h = z\,
{{\sin{\gamma}\,\cos{\gamma}} \over
{\cos^2{\alpha}-\sin^2{\gamma}}}\;, 
\EQNLABEL{z2h} 
\end{equation}
which follows directly from the trigonometry of the triangle in
Figure \FIG{vlcray} \cite{Fomel.sep.84.179}. For example, equation
\EQN{dzdv} can be transformed to the form obtained recently by Liu and
Bleistein \shortcite{Liu}:
\begin{equation}
v\,{{\partial z} \over {\partial v}} = 
{z \over{\cos^2{\alpha}-\sin^2{\gamma}}} =
{z \over{\cos{\alpha_1}\,\cos{\alpha_2}}}\;.
\EQNLABEL{liu} 
\end{equation}
In order to separate different factors contributing to the velocity
continuation process, we can transform this equation to the form
\begin{eqnarray*}
v\,{{\partial z} \over {\partial v}} = 
{z \over {\cos^2{\alpha}}} +
{{h^2} \over z}\,\left(1-\tan^2{\alpha}\,\tan^2{\gamma}\right) =
\end{eqnarray*}
\begin{equation}
= z\,\left(1 + \left({{\partial z} \over {\partial x}}\right)^2\right) +
{{h^2} \over z}\,\left(1-\left({{\partial z} \over {\partial x}}\right)^2\,
\left({{\partial z} \over {\partial h}}\right)^2\right)\;.
\EQNLABEL{zeikonal} 
\end{equation}
Rewritten in terms of the vertical traveltime, it further transforms
to equation \EQN{eikonal} in the main text. Yet another form of the
kinematic velocity continuation equation follows from eliminating the
reflection angle $\gamma$ from equations \EQN{dzdv} and \EQN{z2h}. The
resultant expression takes the following form:
\begin{equation}
v\,{{\partial z} \over {\partial v}} = 
{{2\,(z^2 + h^2)} \over
{\sqrt{z^2 + h^2 \sin^2{2\,\alpha}} + z\,\cos{2\,\alpha}}} =
{z \over {\cos^2{\alpha}}} + {{2\,h^2} \over
{\sqrt{z^2 + h^2 \sin^2{2\,\alpha}} + z}}\;.
\EQNLABEL{nogamma} 
\end{equation}

\APPENDIX{B}
%%%%%%%%%%%%
\mhead{INTEGRAL VELOCITY CONTINUATION AND KIRCHHOFF MIGRATION}
%%%%%%%%%%%%%%%%%%%%%%%%%%%%%%%%%%%%%%%%%%%%%%%%%%%%%%%%%%%%%%
The goal of this appendix is to prove the equivalence between the
result of zero-offset velocity continuation from zero velocity and
conventional post-stack migration. After solving the velocity
continuation problem in the frequency domain, I transform the solution
back to the time-and-space domain and compare it with the famous
Kirchhoff migration operator.
 
Zero-offset migration based on velocity continuation is the solution
of the boundary problem for equation \EQN{POMequation2} with the
boundary condition
\begin{equation}
\left.P\right|_{v=0} = P_0\;,
\EQNLABEL{POMbound} 
\end{equation}
where $P_0(t_0,x_0)$ is the zero-offset seismic section, and
$P(t,x,v)$ is the continued wavefield. In order to find the solution
of the boundary problem composed of \EQN{POMequation2} and \EQN{POMbound}, it is
convenient to apply the function transformation $R(t,x,v) =
t\,P(t,x,v)$, the time coordinate transformation $\sigma = t^2/2$, and,
finally, the double Fourier transform over the squared time coordinate
$\sigma$ and the spatial coordinate $x$:
\begin{equation}
\widehat{R}(v) = \int \int\,P(t,x,v)\,
\exp(i \Omega \sigma - i k x )\,t^2\,dt\,dx\;.
\EQNLABEL{FTK} 
\end{equation}
With the change of domain, equation \EQN{POMequation2} transforms
to the ordinary differential equation
\begin{equation}
{{d\,\widehat{R}} \over {d\,v}} = 
i\,{k^2 \over \Omega}\,v\,\widehat{R}\;,
\EQNLABEL{ODE} 
\end{equation}
and the boundary condition \EQN{POMbound} transforms to the initial
value condition
\begin{equation}
\widehat{R}(0) = \widehat{R}_0\;, 
\EQNLABEL{ODEbound} 
\end{equation}
where 
\begin{equation}
\widehat{R}_0 = \int \int\,P_0(t_0,x_0)\,
\exp(i \Omega \sigma_0 - i k x_0 )\,t_0^2\,dt_0\,dx_0\;,
\EQNLABEL{FTK0}
\end{equation}
and $\sigma_0 = t_0^2/2$.  The unique solution of the initial value
(Cauchy) problem \EQN{ODE} - \EQN{ODEbound} is easily found to be
\begin{equation}
\widehat{R}(v) = \widehat{R}_0\,
\exp\left( i\,{{k^2} \over {2\,\Omega}}\,v^2\right)\;.
\EQNLABEL{ODEsolution} 
\end{equation}

Observe that, in the transformed domain, velocity continuation is a
unitary phase-shift operator. An immediate consequence of this
remarkable fact is the cascaded migration decomposition of post-stack
migration \cite{GEO52.05.06180643}:
\begin{equation}
\exp\left( i\,{{k^2} \over {2\,\Omega}}\,
(v_1^2 +  \cdots + v_n^2)\right) =
\exp\left( i\,{{k^2} \over {2\,\Omega}}\,v_1^2\right)\,\cdots\,
\exp\left( i\,{{k^2} \over {2\,\Omega}}\,v_n^2\right)\;.
\EQNLABEL{cascaded} 
\end{equation}
Analogously, three-dimensional post-stack migration is decomposed
into the two-pass procedure \cite{GPR31.01.00340056}:
\begin{equation}
\exp\left( i\,{{k_1^2+k_2^2} \over {2\,\Omega}}\,v^2\right) =
\exp\left( i\,{{k_1^2} \over {2\,\Omega}}\,v^2\right)\,
\exp\left( i\,{{k_2^2} \over {2\,\Omega}}\,v^2\right)\;.
\EQNLABEL{two-pass}
\end{equation}

The inverse double Fourier transform of both sides of equality
\EQN{ODEsolution} yields the integral (convolution) operator
\begin{equation}
P(t,x,v) = \int\int\,P_0(t_0,x_0)\,K(t_0,x_0;t,x,v)\,dt_0\,dx_0\;,
\EQNLABEL{convolution}
\end{equation}
with the kernel $K$ defined by
\begin{equation}
K = {{t_0^2/t} \over {(2\,\pi)^{m+1}}}\,
\int\int\,\exp\left(
i\,{{k^2} \over {2\,\Omega}}\,v^2 + ik\,(x - x_0) - 
{{i\Omega} \over 2}\,(t^2 - t_0^2)
\right)\,dk\,d\Omega\;,
\EQNLABEL{kernel}
\end{equation}
where $m$ is the number of dimensions in $x$ and $k$ ($m$ equals $1$
or $2$). The inner integral on the wavenumber axis $k$ in formula
\EQN{kernel} is a known table integral \cite{grad}. Evaluating this
integral simplifies equation \EQN{kernel} to the form
\begin{equation}
K = {{t_0^2/t} \over {(2\,\pi)^{m/2+1}\,v^m}}\,
\int\,(i\Omega)^{m/2}\,\exp\left[
{{i\Omega} \over 2}\,
\left(t_0^2 - t^2 - {{(x - x_0)^2} \over v^2}\right)\right]\,
d\Omega\;.
\EQNLABEL{skernel}
\end{equation}
The term $(i\Omega)^{m/2}$ is the spectrum of the anti-causal
derivative operator ${d \over {d\sigma}}$ of the order $m/2$. Noting
the equivalence
\begin{equation}
\left({\partial \over {\partial \sigma}}\right)^{m/2} =
\left({1 \over t}\,{\partial \over {\partial t}}\right)^{m/2} =
\left({1 \over t}\right)^{m/2}\,
\left({\partial \over {\partial t}}\right)^{m/2}\;,
\EQNLABEL{halfdif}
\end{equation}
which is exact in the 3-D case ($m=2$) and asymptotically correct in
the 2-D case ($m=1$), and applying the convolution theorem, we can
transform operator \EQN{convolution} to the form
\begin{equation}
P(t,x,v) = {1 \over {(2\,\pi)^{m/2}}}\,\int\,
{{\cos{\alpha}} \over {(v\,\rho)^{m/2}}}\,
\left(- {\partial \over {\partial t_0}}\right)^{m/2}
P_0\left({\rho \over v},x_0\right)\,dx_0\;,
\EQNLABEL{Kirchhoff}
\end{equation}
where $\rho = \sqrt{v^2\,t^2 + (x - x_0)^2}$, and $\cos{\alpha} =
t_0/t$. Operator \EQN{Kirchhoff} coincides with the Kirchhoff operator
of conventional post-stack time migration \cite{GEO43.01.00490076}.

%\APPENDIX{C}
%%%%%%%%%%%%
\section{Introduction}
\par
In a recent work \cite{me,Fomel.sep.92.159}, I
introduced the process of \emph{velocity continuation} to describe a
continuous transformation of seismic time-migrated images with a
change of the migration velocity. Velocity continuation generalizes
the ideas of residual migration
\cite{GEO50.01.01100126,Etgen.sepphd.68} and cascaded migrations
\cite{GEO52.05.06180643}. In the zero-offset (post-stack) case, the
velocity continuation process is governed by a partial differential
equation in midpoint, time, and velocity coordinates, first discovered
by \longcite{Claerbout.sep.48.79}. \longcite{hubral1} and
\longcite{hubral2} describe this process in a broader context of
``image waves''. Generalizations are possible for the non-zero offset
(prestack) case \cite{Fomel.sep.92.159,Fomel.segab.97}.
\par
A numerical implementation of velocity continuation process provides
an efficient method of scanning the velocity dimension in the search
of an optimally focused image. The first implementations
\cite{Li.sep.48.85,Fomel.sep.92.159} used an analogy with Claerbout's
15-degree depth extrapolation equation to construct a
finite-difference scheme with an implicit unconditionally stable
advancement in velocity. \longcite{Fomel.sep.95.sergey2} presented an
efficient three-dimensional generalization, applying the helix
transform \cite{Claerbout.sep.95.jon1}.
\par
\plot{fd-imp}{width=6in}{Impulse responses (Green's functions) of
  velocity continuation, computed by a second-order finite-difference
  method.  The left plots corresponds to continuation to a larger
  velocity ($+1$ km/sec); the right plot, smaller velocity, ($-1$
  km/sec).}
\par
A low-order finite-difference method is probably the most efficient
numerical approach to this method, requiring the least work per
velocity step. However, its accuracy is not optimal because of the
well-known numerical dispersion effect. Figure \ref{fig:fd-imp} shows
impulse responses of post-stack velocity continuation for three
impulses, computed by the second-order finite-difference method
\cite{Fomel.sep.92.159}. As expected from the residual migration
theory \cite{GEO50.01.01100126}, continuation to a higher velocity
(left plot) corresponds to migration with a residual velocity, and its
impulse responses have an elliptical shape. Continuation to a smaller
velocity (right plot in Figure \ref{fig:fd-imp}) corresponds to
demigration (modeling), and its impulse responses have a hyperbolic
shape. The dispersion artifacts are clearly visible in the figure.
\par
In this paper, I explore the possibility of implementing a numerical
velocity continuation by spectral methods. I adopted two different
methods, comparable in efficiency with finite differences.  The first
method is a direct application of the Fast Fourier Transform (FFT)
technique.  The second method transforms the time grid to Chebyshev
collocation points, which leads to an application of the
Chebyshev-$\tau$ method \cite{lanc,orszag,boyd}, combined with an
unconditionally stable implicit advancement in velocity.  Both methods
employ a transformation of the grid from time $t$ to the squared time
$\sigma = t^2$, which removes the dependence on $t$ from the
coefficients of the velocity continuation equation. Additionally, the
Fourier transform in the space (midpoint) variable $x$ takes care of
the spatial dependencies. This transform is a major source of
efficiency, because different wavenumber slices can be processed
independently on a parallel computer before transforming them back to
the physical space.
\end{comment}
\par
\section{Numerical velocity continuation in the post-stack domain}
\par
The post-stack velocity continuation process is governed by a partial
differential equation in the domain, composed by the seismic image
coordinates (midpoint $x$ and vertical time $t$) and the additional
velocity coordinate $v$. Neglecting some amplitude-correcting terms
\cite{first}, the equation takes the form
\cite{Claerbout.sep.48.79}
\begin{equation}
  \label{PDE}
  {{\partial^2 P} \over {\partial v\, \partial t}} +
  v\,t\,{{\partial^2 P} \over {\partial x^2}} = 0\;.
\end{equation}
Equation (\ref{PDE}) is linear and belongs to the hyperbolic type. It
describes a wave-type process with the velocity $v$ acting as a
propagation variable. Each constant-$v$ slice of the function
$P(x,t,v)$ corresponds to an image with the corresponding constant
velocity. The necessary boundary and initial conditions are 
\begin{equation}
  \label{BC}
  \left.P\right|_{t=T} = 0\;\quad \left.P\right|_{v=v_0} = P_0 (x,t)\;,
\end{equation}
where $v_0$ is the starting velocity, $T=0$ for continuation to a smaller
velocity and $T$ is the largest time on the image (completely attenuated
reflection energy) for continuation to a larger velocity.  The first case
corresponds to ``modeling'' (demigration); the latter case, to seismic
migration.
\par
Mathematically, equations (\ref{PDE}) and (\ref{BC}) define a
Goursat-type problem \cite{kurant2}. Its analytical solution can be
constructed by a variation of the Riemann method in the form of an
integral operator \cite{me,first}:
\begin{equation}
  \label{kirch}
  P(t,x,v) = \frac{1}{(2\,\pi)^{m/2}}\,\int\, 
  \frac{1}{(\sqrt{v^2-v_0^2}\,\rho)^{m/2}}\, 
  \left(- \frac{\partial}{\partial t_0}\right)^{m/2}
  P_0\left(\frac{\rho}{\sqrt{v^2-v_0^2}},x_0\right)\,dx_0\;,
\end{equation}
where $\rho = \sqrt{(v^2-v_0^2)\,t^2 + (x - x_0)^2}$, $m=1$ in the 2-D
case, and $m=2$ in the 3-D case. In the case of continuation from zero
velocity $v_0=0$, operator (\ref{kirch}) is equivalent (up to the
amplitude weighting) to conventional Kirchoff time migration
\cite{GEO43.01.00490076}.  Similarly, in the frequency-wavenumber
domain, velocity continuation takes the form
\begin{equation}
  \label{stolt}
  \hat{P} (\omega,k,v) = \hat{P}_0 (\sqrt{\omega^2+k^2 (v^2-v_0^2)},k)\;,
\end{equation}
which is equivalent (up to scaling coefficients) to Stolt migration
\cite{GEO43.01.00230048}, regarded as the most efficient constant-velocity
migration method.
\par
If our task is to create many constant-velocity slices, there are
other ways to construct the solution of problem (\ref{PDE}-\ref{BC}).
Two alternative approaches are discussed in the next two
subsections.
\subsection{Finite-difference approach}
%%%%%%%%%%%%%%%%%%%%%%%%%%%%%%%%%%%%%%%%%%%%%%%%%%%%%%%%%%%%
The differential equation~(\ref{PDE}) has a mathematical form analogous to
that of the 15-degree wave extrapolation equation
\cite{Claerbout.blackwell.76}. Its finite-difference implementation, first
described by Claerbout \shortcite{Claerbout.sep.48.79} and Li
\shortcite{Li.sep.48.85}, is also analogous to that of the 15-degree equation,
except for the variable coefficients. One can write the implicit
unconditionally stable finite-difference scheme for the velocity continuation
equation in the form
\begin{equation}
({\bf I} + a_{j+1}^{i+1}\,{\bf T})\,{\bf P}_{j+1}^{i+1} - 
({\bf I} - a_{j}^{i+1}\,{\bf T})\,{\bf P}_{j}^{i+1} -
({\bf I} - a_{j+1}^{i}\,{\bf T})\,{\bf P}_{j+1}^{i} + 
({\bf I} + a_{j}^{i}\,{\bf T})\,{\bf P}_{j}^{i} = 0\;, 
\EQNLABEL{fds} 
\end{equation}
where index $i$ corresponds to the time dimension, index $j$
corresponds to the velocity dimension, ${\bf P}$ is a vector along the
midpoint direction, ${\bf I}$ is the identity matrix, ${\bf T}$
represents the finite-difference approximation to the 
second-derivative operator in midpoint,
and $a_{j}^{i} = v_j\,t_i\,{\Delta v\,\Delta t}$.

In the two-dimensional case, equation \EQN{fds} reduces to a
tridiagonal system of linear equations, which can be easily inverted.
In 3-D, a straightforward extension can be obtained by using either
directional splitting or helical schemes \cite{SEG-1998-1124}. The
direction of stable propagation is either forward in velocity and
backward in time or backward in velocity and forward in time as shown
in Figure \FIG{vlcfds}.

\activesideplot{vlcfds}{height=2in}{NR}{Finite-difference scheme for the
velocity continuation equation. A stable propagation is either forward
in velocity and backward in time (a) or backward in velocity
and forward in time (b).}

\begin{comment}
The following simple ratfor program implements the finite-difference
velocity continuation. It is slightly modified from Jon Claerbout's
original version and has a more straightforward loop structure than
Zhiming Li's {\tt Caso15} program.
\listing{velcon.rt}
The parameter {\tt adj} controls the direction of propagation. It is
equal to zero for backward propagation, which corresponds to the
modeling (demigration) operator. The parameter {\tt inv} controls the
amplitude behavior by introducing time-dependent divisors to equation
\EQN{fds}. For {\tt inv=1}, the program implements
Claerbout's velocity continuation equation. For {\tt inv=2}, it
implements the modified equation \EQN{POMequation2}. The value of {\tt
inv=0} corresponds to the intermediate case. It leads to the
{\em pseudounitary} velocity continuation, for which the reverse
continuation is the exact adjoint operator
\cite{Fomel.sep.92.267}. We can easily test different types of
amplitude behavior with the dot-product test and its modifications.
The parameter {\tt b} is required for the ``1/6 trick'' introduced by
Claerbout \shortcite{Claerbout.blackwell.85} to increase the accuracy
of the second-derivative operator ${\bf T}$ in \EQN{fds}. The
second-order difference in subroutine {\tt diffxx} implies simple
zero-slope boundary conditions on the midpoint coordinate. The call to
{\tt rtris} solves the tridiagonal system.
\end{comment}

In order to test the performance of the finite-difference velocity
continuation method, I use a simple synthetic model from
\longcite{Claerbout.bei.95}. The reflectivity model is shown in
Figure \FIG{vlcmod}. It contains several features that challenge the
migration performance: dipping beds, unconformity, syncline,
anticline, and fault. \uline{The velocity is taken to be constant 
$v=1.5\,\mbox{km/s}$.}

\activesideplot{vlcmod}{height=2.5in}{ER}{Synthetic model for testing
finite-difference migration by velocity continuation.}

\uline{Figures~\ref{fig:vlckaa}--\ref{fig:vlcspv}} compare invertability of different
migration methods. In all cases, constant-velocity modeling (demigration) 
%by the adjoint operator \cite{Claerbout.blackwell.92}
was followed by migration
with the correct velocity.  Figures \FIG{vlckaa} and \FIG{vlcsto} show the
results of modeling and migration with the Kirchhoff \cite{GEO43.01.00490076}
and $f$-$k$ \cite{GEO43.01.00230048} methods, respectively. These figures
should be compared with Figure \FIG{vlcvel}, showing the analogous result of
the finite-difference velocity continuation.  The comparison reveals a
remarkable invertability of velocity continuation, which reconstructs
accurately the main features and frequency content of the model. Since the
forward operators were different for different migrations, this comparison did
not test the migration properties themselves. For such a test, I compare the
results of the Kirchhoff and velocity-continuation migrations after Stolt
modeling.  The result of velocity continuation, shown in Figure \FIG{vlcspv},
is noticeably more accurate than that of the Kirchhoff method.

%\activeplot{vlckir}{width=6in,height=2.5in}{ER}{Result of modeling and
%migration with the fast Kirchhoff method. Left plot shows the
%reconstructed model. Right plot compares the average amplitude
%spectrum of the true model with that of the reconstructed image.}

\activeplot{vlckaa}{width=6in}{ER}{Result of modeling and
migration with the Kirchhoff method. Top left plot shows the
reconstructed model. Top right plot compares the average amplitude
spectrum of the true model with that of the reconstructed image.
Bottom left is the reconstruction error. Bottom right is the absolute error in
the spectrum.
}

%\activeplot{vlcpha}{width=6in,height=2.5in}{ER}{Result of modeling and
%migration with the phase-shift method. Left plot shows the
%reconstructed model. Right plot compares the average amplitude
%spectrum of the true model with that of the reconstructed image.}

\activeplot{vlcsto}{width=6in}{ER}{Result of modeling and
  migration with the Stolt method. Top left plot shows the reconstructed model.
  Top right plot compares the average amplitude spectrum of the true model
  with that of the reconstructed image. Bottom left is the reconstruction
  error. Bottom right is the absolute error in the spectrum.}

\activeplot{vlcvel}{width=6in}{ER}{Result of modeling and
  migration with the finite-difference velocity continuation. Top left plot
  shows the reconstructed model. Top right plot compares the average amplitude
  spectrum of the true model with that of the reconstructed image. Bottom left
  is the reconstruction error. Bottom right is the absolute error in the
  spectrum.  }

\activeplot{vlcspv}{width=6in,height=5in}{ER}{Top:
modeling with Stolt method, migration with the Kirchhoff
method. Bottom: modeling with Stolt method, migration with the
finite-difference velocity continuation. Left plots show the
reconstructed models. Right plots show the reconstruction errors.}

These tests confirm that finite-difference velocity continuation is an
attractive migration method. It possesses remarkable invertability properties,
which may be useful in applications that \uline{require inversion. While the
  traditional migration methods transform the data between two completely
  different domains (data-space and image-space), velocity continuation
  accomplishes the same trans\-for\-ma\-ti\-on by propagating the data in the
  extended domain along the velocity direction. Inverse propagation restores
  the original data.} According to \longcite{Li.sep.48.85}, the computational
speed of this method compares favorably with that of Stolt migration. The
advantage is apparent for cascaded migration or migration with multiple
velocity models. In these cases, the cost of Stolt migration increases in
direct proportion to the number of velocity models, while the cost of velocity
continuation stays the same.

\par
\subsection{Fourier approach}
\par
The change of variable $\sigma = t^2$ transforms
equation~(\ref{PDE}) to the form
\begin{equation}
  \label{PDE2}
  2\,{{\partial^2 P} \over {\partial v\, \partial \sigma}} +
  v\,{{\partial^2 P} \over {\partial x^2}} = 0\;,
\end{equation}
whose coefficients do not depend on the time variables.  Double Fourier
transform in $\sigma$ and $x$ further simplifies equation (\ref{PDE2})
to the ordinary differential equation
\begin{equation}
  \label{ODE}
  2\,i\Omega\,{{d \hat{P}} \over {d v}} -
  v\,k^2\,\hat{P} = 0\;,
\end{equation}
where the ``frequency'' variable $\Omega$ corresponds to the stretched time
coordinate $\sigma$, and $k$ is the wavenumber in $x$:
\begin{equation}
\label{FFT}
\hat{P}(k,\Omega,v) = \int\!\!\int P(x,t,v)\,
e^{-i\,\Omega\,t^2 + i\,k\,x} d x\,dt
\end{equation}
Equation (\ref{ODE}) has
an explicit analytical solution
\begin{equation}
  \label{ODEsol}
  \hat{P} (k,\Omega,v) = \hat{P}_0 (k,\Omega)\,
  e^{\frac{i k^2(v_0^2-v^2)}{4\Omega}}\;,
\end{equation}
which leads to a very simple algorithm for the numerical velocity
continuation. The algorithm consists of the following steps:
\begin{enumerate}
\item \uline{Input the zero-offset (post-stack) data migrated with velocity $v_0$ (or
  unmigrated if $v_0=0$).}
\item Transform the input from a regular grid in $t$ to a regular grid
  in $\sigma$.
\item Apply Fast Fourier Transform (FFT) in $x$ and $\sigma$.
\item Multiply by the all-pass phase-shift filter $e^{\frac{i
      k^2(v_0^2-v^2)}{4\Omega}}$.
\item Inverse FFT in $x$ and $\sigma$.
\item Inverse transform to a regular grid in $t$.
\end{enumerate}

\plot{t2}{width=6in}{Synthetic seismic data before (left) and after
  (right) transformation to the $\sigma$ grid.}

Figure \ref{fig:t2} shows a simple synthetic model of seismic
reflection data generated from the model in Figure~\ref{fig:vlcmod}
before and after transforming the grid, regularly spaced in $t$, to a
grid, regular in $\sigma$. The left plot of Figure \ref{fig:t2-fft}
shows the Fourier transform of the data. Except for the nearly
vertical event, which corresponds to a stack of parallel layers in the
shallow part of the data, the data frequency range is contained near
the origin in the $\Omega-k$ space.  The right plot of Figure
\ref{fig:t2-fft} shows the phase-shift filter for continuation from
zero imaging velocity (which corresponds to unprocessed data) to the
velocity of 1 km/sec. The rapidly oscillating part (small frequencies
and large wavenumbers) is exactly in the region, where the data
spectrum is zero. It corresponds to physically impossible reflection
events.

\plot{t2-fft}{width=6in}{Left: the real part of the data Fourier
  transform.  Right: the real part of the velocity continuation
  operator (continuation from 0 to 1 km/s) in the Fourier domain.}

\uline{The described algorithm}
 is very attractive from the practical point of view
because of its efficiency (based on the FFT algorithm). The operation count is
roughly the same as in the Stolt migration implemented with equation
(\ref{stolt}): two forward and inverse FFTs and forward and inverse grid
transform with interpolation (one complex-number transform in the case of
Stolt migration). The velocity continuation algorithm can be more efficient
than the Stolt method because of the simpler structure of the innermost loop
\uline{(step 4 in the algorithm)}. 

\begin{comment}
However, its practical implementation faces two
difficult problems: artifacts of the $t^2$ grid transform and
wraparound artifacts
\par
\subsection{Improving the accuracy of the $t^2$ grid transform}
\par
The first problem is the loss of information in the transform to the
$t^2$ grid. As illustrated in Figure \ref{fig:t2}, the shallow part of
the data gets severely compressed in the $t^2$ grid. The amount of
compression can lead to inadequate sampling, and as a result, aliasing
artifacts in the frequency domain. Moreover, it can be difficult to
recover from the loss of information in the transformed domain when
transforming back into the original grid. A partial remedy for this
problem is to increase the grid size in the $t^2$ domain. The top plots
in Figure \ref{fig:fft-inv} show the result of back transformation to
the $t$ grid and the difference between this result and the original
model (plotted on the same scale). We can see a noticeable loss of
information in the upper (shallow) part of the data, caused by
undersampling. The bottom plots in Figure \ref{fig:fft-inv} correspond
to increasing the grid size by a factor of three. Some of the
artifacts have been suppressed, at the expense of dealing with a
larger grid.

\plot{fft-inv}{width=6in}{The left plots show the reconstruction of the original data 
  after transforming back from the $t^2$ grid to the original $t$
  grid.  The right plots show the difference with the original model.
  Top: using the original grid size ($N_t = 200$). Bottom: increasing
  the grid size by a factor of three.}

To perform an accurate transform of the grid, I adopted the following
method, inspired by \cite{Claerbout.sep.48.347}. Let $d_{\mbox{\tiny
    new}}$ denote the data on the new grid and $d_{\mbox{\tiny old}}$
be the data on the old grid. If $L$ is the interpolation operator,
defined on the new grid, then the optimal least-squares transformation
is
\begin{equation}
  \label{interp}
  d_{\mbox{\tiny new}} = (L^T L)^{-1}\,L\,d_{\mbox{\tiny old}}\;,
\end{equation}
where $L^T$ denotes the adjoint interpolation operator. The operator
$(L^T L)^{-1}$ provides a proper scaling of the result. If we use
simple linear interpolation for the $L$ operator, then $L^T L$ is a
tridiagonal matrix, which can be easily inverted (in $8 N$
operations). If some parts in $d_{\mbox{\tiny new}}$ are not fully
constrained, then the tridiagonal matrix is not invertible. To obtain
a solution in this case, we can include a regularization operator $D$
in (\ref{interp}), as follows:
\begin{equation}
  \label{regul}
  d_{\mbox{\tiny new}} = (L^T L + \epsilon^2 D)^{-1}\,L\,d_{\mbox{\tiny
      old}}\;,
\end{equation}
A convenient choice for $D$ is a second derivative operator,
represented with the second-order finite-difference approximation.
This operator allows the selection of the smoothest possible function
$d_{\mbox{new}}$ while preserving the efficient tridiagonal structure
of $L^T L + \epsilon^2 D$. In this problem, the parameter $\epsilon$
can be chosen as small as possible, as long as it prevents the
inversion from getting unstable.
\par
\subsection{Suppressing wraparound artifacts of the Fourier method}
\par
The periodic boundary conditions both in the squared time $\sigma$ and
the spatial coordinate $x$, implied by the Fourier approach, are
artificial in the problem of velocity continuation. The artificial
periodicity is convenient from the computational point of view.
However, false periodic events (wraparound artifacts) should be
suppressed in the final output. A natural method for attacking this
problem is to apply zero padding in the physical space prior to
Fourier transform. Of course, this method involves an additional
expense of the grid size increase.
\par
\plot{fft-imp}{width=6in}{Impulse responses (Green's functions) of velocity continuation, 
  computed by the Fourier method. Top: without zero padding, bottom:
  with zero padding. The left plots correspond to continuation to a
  larger velocity ($+1$ km/sec); the right plots, smaller velocity,
  ($-1$ km/sec).}
\par
The top plots in Figure \ref{fig:fft-imp} show the numerical impulse
responses of velocity continuation, computed by the Fourier method.
The initial data contained three spikes, passed through a narrow-band
filter. Theoretically, continuation to larger velocity (the left plot)
should create three elliptical wavefronts, and continuation to smaller
velocity (right plot) should create three hyperbolic wavefronts
\cite[]{GEO50.01.01100126}. We can see that the results are largely
contaminated with wraparound artifacts. The result of applying zero
padding (the bottom plots in Figure \ref{fig:fft-imp}) shows most of
the artifacts suppressed.
\par
Chebyshev spectral method, discussed in the next section, provides a
spectral accuracy while dealing correctly with non-periodic data.
\par 
\section{Chebyshev Approach}
For an alternative spectral approach, I adopted the Chebyshev-$\tau$
method \cite{lanc,orszag}. The Chebyshev-$\tau$ velocity continuation
algorithm consists of the following steps:
\begin{enumerate}
\item Transform the regular grid in $t$ to Gauss-Lobato collocation
  points, required for the fast Chebyshev transform.  First, a new
  variable $\xi$ is introduced by the shift transform:
  \begin{equation}
    \label{t2x}
    \xi = 1 - \frac{2\,t^2}{T^2}
  \end{equation}
  so that the domain $0 \leq t \leq T$ is mapped into the domain $1
  \geq \xi \geq -1$. Second, the $\xi$ grid points are distributed
  regularly in the cosine projection: $\xi_j = \cos(\frac{\pi j}{N}), j
  = 0,1,2,\ldots,N$.
\item Transform the initial image $P_0 (x,t)$ into the Chebyshev space
  in $\xi$ and Fourier transform in $x$, using the FFT algorithm. The
  Chebyshev-Fourier representation of $P_0 (x,t)$ is
  \begin{equation}
    \label{cheb}
    P_0 (x,t) = \sum_{k=-N_x/2}^{N_x/2-1}\sum_{j=0}^{N_t} 
    \hat{P}_{kj} T_j (\xi) e^{i k x}\;, 
  \end{equation}
  where $T_j$ denotes the Chebyshev polynomial of degree $j$.
\item Apply equation (\ref{PDE}) to advance the image in velocity $v$.
  It is convenient to rewrite this equation in the form
  \begin{equation}
    \label{chebPDE}
  {{\partial P} \over {\partial v}} =
  \frac{v T^2}{4}\,\int d\xi\,
  {{\partial^2 P} \over {\partial x^2}}\;. 
  \end{equation}
  In the Chebyshev-$\tau$ domain, the double differentiation in $x$ is
  performed by multiplying the Fourier transform of $P$ by $-k^2$, and
  integration in $\xi$ is performed
  as a direct operations on the Chebyshev coefficients. In
  particular, if $\sum_{j=0}^{N} a_j T_j (\xi)$ is the Chebyshev
  representation of the function $f (\xi)$, then the coefficients
  $b_j$ of $\int f (\xi) d\xi$ are defined by the relation
  \begin{equation}
    \label{int}
    2 j\,b_j = c_{j-1} a_{j-1} - a_{j+1}
  \end{equation}
  where $c_0 = 2$, $c_j = 0$ for $j < 0$, and $c_j = 1$ for $j > 0$.
  The constant of integration (and, correspondingly, the coefficient
  $b_0$) can be found at each velocity step from the boundary
  conditions (\ref{BC}), which are transformed to the form
  \begin{equation}
    \label{chebBC}
    \left.P\right|_{\xi=-1} = \sum_{j=0}^{N_t} \hat{P}_{kj} (-1)^j = 0\;.
  \end{equation}
  
  For the velocity advancement I used an implicit Crank-Nicholson scheme,
  which is unconditionally stable independent of the velocity step size.
  By writing equation (\ref{chebPDE}) in the matrix form
  \begin{equation}
    \label{matrix}
    \frac{\partial \bold{P}}{\partial v} = \bold{A} \bold{P}\;,
  \end{equation}
  the Crank-Nicolson advancement is represented by the equation
    \begin{equation}
    \label{CN}
    \bold{P}_{v+dv} = \left(\bold{I} - \bold{A}\,\frac{dv}{2}\right)^{-1}
    \left(\bold{I} + \bold{A}\,\frac{dv}{2}\right) \bold{P}_v\;,
  \end{equation}
  where $\bold{I}$ is the identity matrix. The inverted matrix
  $\left(\bold{I} - \bold{A}\,\frac{dv}{2}\right)$ has a tridiagonal
  structure, except for the first row, implied by the boundary
  condition (\ref{chebBC}). A careful treatment of the boundary
  condition by the matrix-bordering method \cite{faddeev,boyd} allows
  for an efficient inversion at a tridiagonal solver speed.
\item Transform the result of the velocity advancement back to the
  physical domain.
\item Transform the grid back to being regularly space in $t$.
\end{enumerate}
\par
\plot{cheb1}{width=6in}{Synthetic seismic data before (left) and 
  after (right) transformation to the Chebyshev grid in squared time.}
\par
\plot{cheb1-inv}{width=6in}{The left plots show the reconstruction of the original data 
  after transforming back from the Chebyshev grid to the original $t$
  grid.  The right plots show the difference with the original model.
  Top: using the original grid size ($N_t = 200$). Bottom: increasing
  the grid size by a factor of three.}
\par
The first advantage of the Chebyshev approach comes from the better
conditioning of the grid transform.  Figure \ref{fig:cheb1} shows the
synthetic data before and after the grid transform.  Figure
\ref{fig:cheb1-inv} shows a reconstruction of the original data after
transforming back from the Chebyshev grid (Gauss-Lobato collocation
points). The difference with the original image is negligibly small.
\par
\plot{cheb1-fft}{width=6in}{Left: Synthetic data after Chebyshev
  transform.  Right: the real part of the Fourier transform in the
  space coordinate.}
\par
The second advantage is the compactness of the Chebyshev
representation.  Figure \ref{fig:cheb1-fft} shows the data after the
decomposition into Chebyshev polynomials in $\xi$ and Fourier
transform in $x$. We observe a very rapid convergence of the Chebyshev
representation: a relatively small number of polynomials suffices to
represent the data.
\par
\plot{cheb-impl}{width=6in}{Impulse responses (Green's functions) of
  velocity continuation, computed by the Chebyshev-$\tau$ method. Top:
  without zero padding, bottom: with zero padding on the $x$ axis. The
  left plots correspond to continuation to a larger velocity ($+1$
  km/sec); the right plots, smaller velocity, ($-1$ km/sec).}
\par
The third advantage is the proper handling of the non-periodic boundary
conditions. Figure \ref{fig:cheb-impl} shows the velocity continuation
impulse responses, computed by the Chebyshev method. As expected, no
wraparound artifacts occur on the time axis, and the accuracy of the
result is noticeably higher than in the case of finite differences
(Figure \ref{fig:fd-imp}).

\section{Data tests}
\end{comment}

\section{\uline{Numerical velocity continuation in the prestack domain}}

\begin{comment}
Velocity continuation in the zero-offset (post-stack case) can be
performed with a simple Fourier-domain algorithm
\cite{Fomel.sep.97.sergey2}:
\begin{enumerate}
  \item Input an image, migrated with velocity $v_0$.
  \item Transform the time axis $t$ to the squared time coordinate:
    $\sigma=t^2$.
  \item Apply a fast Fourier transform (FFT) on both the squared time
    and the midpoint axis. The squared time $\sigma$ transforms to the
    frequency $\Omega$, and the midpoint coordinate $x$ transforms to
    the wavenumber $k$. It is safe to assume that in the
    post-migration domain seismic images are uniformly sampled in $x$,
    which allows us to use the FFT technique. In the case of 3-D data,
    FFT should be applied in both midpoint coordinates.
  \item Apply a phase-shift operator to transform to different velocities $v$:
    \begin{equation}
      \label{eqn:zophase}
      \hat{P}(\Omega,k,v) = \hat{P}_0 (\Omega,k)\,
      e^{\frac{i\,k^2\left(v_0^2 - v^2\right)}{4\,\Omega}}\;.
    \end{equation}
  \item Apply an inverse FFT to transform from $\Omega$ and $k$ to
    $\sigma$ and $x$.
  \item Apply an inverse time stretch to transform from $\sigma$ to $t$.
\end{enumerate}    
The computational complexity of this algorithm has the same order as
that of the Stolt migration \cite{GEO43.01.00230048}, but in practice
it can be even faster because of the very simple inner computation. 
\par
\end{comment}

To generalize the algorithm of the previous section to the prestack case, it is
first necessary to include the residual NMO term \cite{first}.  Residual normal
moveout can be formulated with the help of the differential equation:
\begin{equation}
{{\partial P} \over {\partial v}} + 
{{h^2} \over {v^3\,t}}\,{{\partial P} \over {\partial t}} = 0\;,
\label{eqn:ResNMOdyn} 
\end{equation}
where $h$ stands for the half-offset.  The analytical solution of
equation (\ref{eqn:ResNMOdyn}) has the form of the residual NMO
operator:
\begin{equation}
  P(t,h,v) = P_0\left(\sqrt{t^2 + h^2\,
    \left(\frac{1}{v_0^2} - \frac{1}{v^2}\right)},h\right)\;.
\label{eqn:ResNMOsol} 
\end{equation}
After transforming to the squared time $\sigma = t^2$ and the
corresponding Fourier frequency $\Omega$, equation
(\ref{eqn:ResNMOdyn}) takes the form of the ordinary differential
equation
\begin{equation}
  \frac{d \hat{P}}{d v} + 
  i \Omega\,\frac{2\,h^2}{v^3}\,\hat{P} = 0
\label{eqn:FFTResNMO} 
\end{equation}
with the analytical frequency-domain phase-shift solution
\begin{equation}
  \hat{P} (\Omega, h, v) = \hat{P_0} (\Omega,h) e^{i\,\Omega\,h^2\,
    \left(\frac{1}{v_0^2} - \frac{1}{v^2}\right)}\;.
\label{eqn:ResNMOshift} 
\end{equation}
To obtain a Fourier-domain prestack velocity continuation algorithm, one just
needs to combine the phase-shift operators in equations (\ref{ODEsol}) and
(\ref{eqn:ResNMOshift}) and to include stacking across different offsets. 
\uline{The
exact velocity continuation theory also includes the residual DMO term}
\cite{first}, \uline{which has a second-order effect, pronounced only at small
depths. It is neglected here for simplicity.}  The algorithm takes the
following form:
\begin{enumerate}
  \item Input a set of common-offset images, migrated with velocity $v_0$.
  \item Transform the time axis $t$ to the squared time coordinate:
    $\sigma=t^2$.
  \item Apply a fast Fourier transform (FFT) on both the squared time
    and the midpoint axis. The squared time $\sigma$ transforms to the
    frequency $\Omega$, and the midpoint coordinate $x$ transforms to
    the wavenumber $k$. 
  \item Apply a phase-shift operator to transform to different velocities $v$:
    \begin{equation}
      \label{eqn:zophase2}
      \hat{P}(\Omega,k,v) = \sum_{h} \hat{P}_0 (\Omega,k,h)\,
      e^{i\,\frac{k^2\left(v_0^2 - v^2\right)}{4\,\Omega} +
        i\,\Omega\,h^2\, \left(\frac{1}{v_0^2} -
          \frac{1}{v^2}\right)}\;.
    \end{equation}
    To save memory, the continuation step is immediately followed
    by stacking. \uline{For velocity analysis purposes, a semblance measure}
    \cite{GEO36-03-04820497} \uline{is computed in addition to the simple stack
    analogously to the standard practice of stacking velocity analysis.}
    
    \uline{Implementing the residual moveout correction in the Fourier domain allows
    one to package it conveniently with the phase-shift operator without the
    need to transform the continuation result back to the time domain. The
    offset dimension in equation~(\ref{eqn:zophase2}) is replaced by the
    velocity dimension similarly to the velocity transform of the conventional
    stacking velocity analysis} \cite{IG202-00-10012027}.

  \item Apply an inverse FFT to transform from $\Omega$ and $k$ to
    $\sigma$ and $x$.
  \item Apply an inverse time stretch to transform from $\sigma$ to $t$.
  \end{enumerate}  
  One can design similar algorithms by using the finite difference method.
  \uline{Although the finite-difference approach offers a faster continuation speed,
  the spectral algorithm has a higher accuracy while maintaining an acceptable
  cost.}

%The complete theory of prestack velocity continuation also requires a
%residual DMO operator
%\cite{Etgen.sepphd.68,Fomel.sep.92.159,Fomel.segab.97}. However, the
%difficulty of implementing this operator is not fully compensated by
%its contribution to the full velocity continuation. For simplicity, I
%decided not to include residual DMO in the current implementation.
\par  
Figure~\ref{fig:velimp} shows impulse responses of prestack velocity
continuation. The input for producing this figure was a time-migrated
constant-offset section, corresponding to an offset of 1 km and a constant
migration velocity of 1 km/s. In full accordance with the theory \cite{first},
three spikes in the input section transformed into shifted ellipsoids after
continuation to a higher velocity and into shifted hyperbolas after
continuation to a smaller velocity. \uline{
Padding of the time axis helps to avoid
the wrap-around artifacts of the Fourier method. Alternatively, one could use
the artifact-free but more expensive Chebyshev spectral method}
\cite{Fomel.sep.97.sergey2}.

\plot{velimp}{width=6in}{Impulse responses of prestack velocity
  continuation. Left plot: continuation from 1 km/s to 1.5 km/s.
  Right plot: continuation from 1 km/s to 0.7 km/s. Both plots
  correspond to the offset of 1 km.}
\par

Velocity continuation creates a time-midpoint-velocity cube (four-dimensional
for 3-D data), which is convenient for picking imaging velocities in the same
way as the result of common-midpoint or common-reflection-point velocity
analysis. The important difference is that velocity continuation provides an
optimal focusing of the reflection energy by properly taking into account both
vertical and lateral movements of reflector images with changing migration
velocity. \uline{An experimental evidence for this conclusion is provided in
  the examples section of this paper.}

\begin{comment}
Figure \ref{fig:consmb} compares velocity spectra
(semblance panels) at a CIP location of about 11.5 km after residual
NMO and after prestack velocity continuation. Although the overall
difference between the two panels is small, the velocity continuation
panel shows a noticeably better focusing, especially in the region of
conflicting dips between 1 and 2 seconds. 
\end{comment}
The next subsection discusses
the velocity picking step in more detail.

%\plot{consmb}{width=6in}{Velocity spectra around 11.5 km CRP after
%  residual NMO (left) and after prestack velocity continuation
%  (right). The right plot shows improved focusing in the region between
%  1 and 2 seconds.}

\subsection{Velocity picking and slicing}

After the velocity continuation process has created a time-midpoint-velocity
cube, one can pick the best focusing velocity from that cube and create an
optimally focused image by slicing through the cube. \uline{This step is common in
other methods that involve velocity slicing}
\cite{shurtleff,SEG-1984-S1.8,GEO57-01-00510059}. \uline{The algorithm described
below has been also adopted by} \longcite{SEG-2000-09920995} \uline{for velocity
analysis in wave-equation migration.}

A simple automatic velocity picking algorithm follows from solving the
following regularized least-squares system:
\begin{equation}
  \label{eqn:pick}
  \left\{\begin{array}{rcl}
      \bold{W}\,\bold{x} & \approx & \bold{W}\,\bold{p} \\
      \epsilon \bold{D} \bold{x} & \approx & \bold{0}
    \end{array}\right.\;.
\end{equation}
\uline{In the more standard notation, the solution $\bold{x}$ minimizes the
least-squares objective function}
\begin{equation}
  \label{eqn:ofpick}
(\bold{x}-\bold{p})^{T}\, \bold{W}^2\,(\bold{x}-\bold{p}) +
\epsilon^2\,\bold{x}^T\,\bold{D}^T\,\bold{D}\,\bold{x}
\end{equation}
Here $\bold{p}$ is the vector of blind maximum-semblance picks (possibly in a
predefined fairway), $\bold{x}$ is the estimated velocity picks, $\bold{W}$ is
the weighting operator with the weight corresponding to the semblance values
at $\bold{p}$, $\epsilon$ is the scalar regularization parameter, $\bold{D}$
is a roughening operator, and $\bold{D}^T$ is the adjoint operator.  The first
least-squares fitting goal in (\ref{eqn:pick}) states that the estimated
velocity picks should match the measured picks where the semblance is high
enough\footnote{Of course, this goal might be dangerous, if the original picks
  $\bold{p}$ include regular noise (such as multiple reflections) with high
  semblance value \cite{Toldi.sepphd.43}. For simplicity, and to preserve the
  linearity of the problem, I assume that this is not the case.}.  The second
fitting goal tries to find the smoothest velocity function possible.  The
least-squares solution of problem (\ref{eqn:pick}) takes the form
\begin{equation}
  \label{eqn:LS}
  \bold{x} = 
  \left(\bold{W}^2 + 
    \epsilon^2\,\bold{D}^T\,\bold{D}\right)^{-1}\,\bold{W}^2\,\bold{p}\;.
\end{equation}
In the case of picking a one-dimensional velocity function from a single
semblance panel, one can simplify the algorithm by choosing $\bold{D}$ to be a
convolution with the derivative filter $(1,-1)$. It is easy to see that in
this case the inverted matrix in formula (\ref{eqn:LS}) has a tridiagonal
structure and therefore can be easily inverted with a linear-time algorithm.
The regularization parameter $\epsilon$ controls the amount of smoothing of
the estimated velocity function.  Figure \ref{fig:velpick} shows an example
velocity spectrum and two automatic picks for different values of $\epsilon$.

\plot{velpick}{width=6in}{Semblance panel (left) and automatic
  velocity picks for different values of the regularization parameter.
  Center: $\epsilon=0.01$, right: $\epsilon=0.1$. Higher values of
  $\epsilon$ lead to smoother velocities.}

\par
In the case of picking two- or three-dimensional velocity functions,
one could generalize problem (\ref{eqn:pick}) by defining $\bold{D}$
as a 2-D or 3-D roughening operator. I chose to use a more simplistic
approach, which retains the one-dimensional structure of the algorithm. 
I transform system (\ref{eqn:pick}) to the form
\begin{equation}
  \label{eqn:pick2}
  \left\{\begin{array}{rcl}
      \bold{W}\,\bold{x} & \approx & \bold{W}\,\bold{p} \\
      \epsilon \bold{D} \bold{x} & \approx & \bold{0} \\
      \lambda \bold{x} & \approx & \lambda \bold{x_0}
    \end{array}\right.\;,
\end{equation}
where $\bold{x}$ is still one-dimensional, and $\bold{x}_0$ is the
estimate from the previous midpoint location. The scalar parameter
$\lambda$ controls the amount of lateral continuity in the estimated
velocity function. The least-squares solution to system
(\ref{eqn:pick2}) takes the form
\begin{equation}
  \label{eqn:LS2}
  \bold{x} = 
  \left(\bold{W}^2 + \epsilon^2\,\bold{D}^T\,\bold{D}
    + \lambda^2\,\bold{I}\right)^{-1}\,
  \left(\bold{W}^2\,\bold{p} + \lambda^2 \bold{x_0}\right)\;,
\end{equation}
where $\bold{I}$ denotes the identity matrix. Formula (\ref{eqn:LS2})
also reduces to an efficient tridiagonal matrix inversion.

\uline{After the velocity has been picked, an optimally focused image is constructed
by slicing in the time-midpoint-velocity cube. I used simple linear
interpolation for slicing between the velocity grid values. A more accurate
interpolation technique can be easily adopted.}

\section{Examples}

I demonstrate the performance of the method using a simple 2-D synthetic
test and a field data example from the North Sea.

\subsection{Synthetic Test}

The synthetic test uses constant-velocity prestack modeling and migration to
check the validity of the method when all the theoretical requirements are
satisfied. The data were generated from the synthetic reflectivity model
(Figure~\ref{fig:vlcmod}) and included 60~offsets ranging from 0 to 0.5~km.
The exact velocity in the model is~1.5~km/s, and the initial velocity for
starting the continuation process was chosen at 2~km/s.

%\plot{sig-mva2}{width=6in}{Semblance panels for migration
%  velocity analysis at the common image point~0.5~km. Left: after
%  velocity continuation.  Right: after conventional (NMO) velocity
%  analysis. In this structurally simple region, the difference between
%  two methods is small. The correct velocity is 1.5~km/s.}

Figure~\ref{fig:sig-mva1} compares the semblance panels for migration
velocity analysis using velocity continuation and using the
conventional (NMO) analysis. In the top part of the image, both panels
show maximum picks at the correct velocity (1.5~km/s).  The advantage
of velocity continuation is immediately obvious in the deeper part of
the image, where the events are noticeably better focused.

\plot{sig-mva1}{width=6in}{Semblance panels for migration
  velocity analysis at the common image point~1~km. Left: after
  velocity continuation.  Right: after conventional (NMO) velocity
  analysis. In a structurally complex region, velocity continuation
  clearly provides better focusing. The correct velocity is 1.5~km/s.}

The final result of velocity continuation (after picking maximum semblance and
slicing in the velocity cube) is shown in the bottom left plot of
Figure~\ref{fig:sig-all}. For comparison, Figure~\ref{fig:sig-all} also shows
the result of migration with the correct velocity (the top left plot), initial
velocity (the top right plot), and the result of velocity slicing after the
simple NMO correction, corresponding to the conventional MVA (the bottom right
plot). \uline{The same velocity picking and slicing program was used in both cases.}
The comparison clearly shows that, in this simple example, velocity
continuation is able to accurately reproduce the correct image without using
any prior information about the migration velocity and without any need for
repeating the prestack migration procedure. \uline{Velocity continuation correctly
images events with conflicting dips by properly taking into account both
vertical and lateral shifts in the image position.}

\plot{sig-all}{width=6in,height=6in}{Velocity continuation tested 
  on the synthetic example. Top left: prestack migration with the
  correct velocity of 1.5~km/s. Top right: prestack migration
  with the velocity of 2~km/s. Bottom left: the result of velocity
  continuation. Bottom right: the result from picking migration
  image after only conventional NMO correction.}

\subsection{\uline{Field} Data Example}

Figure \ref{fig:elfmigr} compares the result of a constant-velocity prestack
migration with the velocity of 2 km/s, applied to a dataset from the North Sea
(courtesy of Elf Aquitaine) and the result of velocity continuation to the
same velocity from a migration with a smaller velocity of 1.4 km/s (Figure
\ref{fig:elfmigr}a). The two images (Figures \ref{fig:elfmigr}b and
\ref{fig:elfmigr}c) look remarkably similar, in full accordance with the
theory.

\plot{elfmigr}{width=6in,height=9in}{Constant-offset section of the
  North Sea dataset after migration with the velocity of 1.4 km/s (a),
  migration with the velocity of 2 km/s (b), migration with the
  velocity of 1.4 km/s and velocity continuation to 2 km/s (c).}

Figure~\ref{fig:elffpk} shows a result of two-dimensional velocity
picking after velocity continuation. I used values of $\epsilon=0.1$
and $\lambda=0.1$. The first parameter controls the vertical smoothing
of velocities, while the second parameter controls the amount of
lateral continuity.

%\ref{fig:beivpk} shows a result of two-dimensional velocity picking
%after residual NMO. Figure \ref{fig:beifpk} shows an analogous result

%The difference between residual NMO velocities and
%velocities picked after velocity continuation is small, but clearly
%visible.

%\plot{beivpk}{width=6in,height=3.5in}{Automatic picks of 2-D RMS velocity
%  after residual NMO. The contour spacing is 0.1 km/s, starting from 1.5 km/s.}

\plot{elffpk}{angle=90,totalheight=8.5in,width=6in}{Automatically
  picked migration velocity after velocity continuation.}

Figure \ref{fig:elffmg} shows the final result of velocity
continuation: an image, obtained by slicing through the velocity cube
with the picked imaging velocities. The edges of the salt body in the middle
of the section have been sharply focused by the velocity
continuation process. To transform the already well focused image into the
depth domain, one may proceed in a way similar to
\emph{hybrid migration}: demigration to zero-offset, followed by
post-stack depth migration \cite{GEO62-02-05680576}. This step
would require constructing an interval velocity model from the picked imaging
velocities.

\plot{elffmg}{angle=90,totalheight=8.5in,width=6in}{Final result of
  velocity continuation: seismic image, obtained by slicing through
  the velocity cube.}

\begin{comment}
Without repeating the details of the procedure, Figures~\ref{fig:pck}
and \ref{fig:img} show picked imaging velocities and the velocity
continuation image for the Blake Outer Ridge data, shown in many other
papers in this report.

\plot{pck}{width=6in,height=3.5in}{Blake Outer Ridge data. Automatic
  picks of 2-D imaging velocity after velocity continuation. The contour
  spacing is 0.01 km/s, starting from 1.5 km/s.}

\plot{img}{angle=90,totalheight=8.5in,width=6in}{Blake Outer Ridge
  data. Final result of velocity continuation: seismic image, obtained
  by slicing through the velocity cube.}
\end{comment}

\section{Conclusions}

Velocity continuation is a powerful method for time migration velocity
analysis.  The strength of this method follows from its ability to take into
account both vertical and lateral movement of the reflection events in seismic
images with the changes of migration velocity.

Efficient practical algorithms for velocity continuation can be constructed
using either finite-difference or spectral methods. When applied in the
post-stack (zero-offset) setting, velocity continuation can be used as a
computationally attractive method of time migration. Both finite-difference
and spectral approaches possess remarkable invertability properties:
continuation to a lower velocity reverses continuation to a higher velocity.
\uline{For the finite-difference algorithm, this property is confirmed by synthetic
tests. For the spectral algorithm, it follows from the fact that velocity
continuation reduces to a simple phase-shift unitary operator.}

\uline{Including velocity continuation in the practice of migration velocity analysis
can improve the focusing power of time migration and reduce the production
time by avoiding the need for iterative velocity refinement. No prior velocity
model is required for this type of velocity analysis. This conclusion is
confirmed by synthetic and field data examples.}

\section{Acknowledgments}

I thank Jon Claerbout, Biondo Biondi, and Bill Symes for useful and
stimulating discussions, the sponsors of the Stanford Exploration Project for
their financial support, and Elf Aquitaine for providing the data used in this
work. \uline{I am also grateful to Paul Fowler, Samuel Gray, Hugh Geiger, and one
anonymous reviewer for thorough and helpful reviews that improved the quality
of the paper.}

\newpage
\bibliographystyle{sep}
\bibliography{SEP,MISC,GEOTLE,EAEG,SEG,paper,spec,velcon}

%%% Local Variables: 
%%% mode: latex
%%% TeX-master: t
%%% TeX-master: t
%%% TeX-master: t
%%% TeX-master: t
%%% TeX-master: t
%%% TeX-master: t
%%% End: 
