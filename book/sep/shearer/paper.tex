\lefthead{Claerbout}
\righthead{Earthquake stacks}
\footer{SEP--77}


\title{Earthquake stacks at constant offset}
%\keywords{earthquake, traveltime, graphics }

\author{ Jon F. Claerbout }

\begin{abstract}
	I show Shearer's earthquake stacks
	over all source-receiver locations at constant offset
	and compare them to exploration seismic data.
	This electronic document simply reads the stacks and plots them.
\end{abstract}

\section{INTRODUCTION}

The figures in this paper show Professor Peter Shearer's stacks
over many earthquakes
at constant epicentral distance
(offset angle).
Notice that the time scale is minutes and the offset is degrees of angle
on the earth's surface.
Otherwise, there are remarkable similarities
to conventional exploration seismic data.
For details of how the stacks were made, see the references.

\section{SHEARER's EARTHQUAKE STACKS}
Professor Shearer has found a way to choose the polarity for each seismogram
so that the SH waves do not randomly cancel in the stack.
Since SH waves have no conversions, the SH-wave stack in Figure~\ref{fig:reftran}
is easier to understand than the P-SV stack shown later.
The first big event is
direct S (shear wave).
It comes tangent to ScS (reflection from the earth's core)
hitting grazing angle around 90 degrees
at about 23 minutes of travel time.
The first
{\it multiple}
reflection of ScS at grazing angle
is at double the epicentral distance and double the travel time.
It is easily seen at the opposite side of the earth
near 180 degrees offset.
Figure~\ref{fig:reftran} also shows six bounces
of direct S bouncing from the earth's surface
before the plot ends at one hour travel time.
Incidentally, many fewer earthquakes are observed near 180 degrees
than near 90 degrees for the simple geometrical reason that ten
degrees surrounding the equator is a much bigger area than ten degrees
surrounding the pole.
Thus the quality of the stacks degrades rapidly toward the poles.

\activeplot{reftran}{width=6.00in,height=8.00in}{\CAKEDIR}{
	Transverse component.  SH waves.
	}

\par
Waves analogous to familiar water-bottom pegleg multiples
are readily seen tracking S about 4 minutes later than S.
The data shows these two events,
one being a reflection from a layer at 410 km
and the other from 660 km.
As with familiar water-bottom peglegs,
each event is a superposition of two events,
one with a short path near the source
and the other with a short path near the receiver.

\par
Of particular interest are waves identified by Professor Shearer
that track SS, the first surface multiple of S,
but arrive about 4 minutes {\it earlier.}
Were it not for the confusing similarity between ``shear'' and ``Shearer'',
I would call these ``Shearer'' waves.
They arrive {\it earlier} than the first multiple of direct S because
instead of an earth-surface bounce at the middle of the path,
they bounce a little sooner
from the {\it underside} of the layers at 660 km and 410 km.
The nice thing about Shearer's waves is that they are uniquely
associated with the middle of the path,
rather than having two bounce locations, as do familiar first-order peglegs.
We might like to search our files of exploration data
to see if we can find analogous events.

\par
In the stack in Figure~\ref{fig:refvert}, Professor Shearer
has chosen the polarities of each seismogram
so that P-SV waves stack in phase.
Because of conversions between P and SV,
there are a huge number of new events which I leave to you to pick out.
Waves that propagate further than 180 degrees fold back
(like discrete Fourier transforms).
I cannot explain why such folding was not visible on the SH section.



\activeplot{refvert}{width=6.00in,height=8.00in}{\CAKEDIR}{
	Vertical component showing P and SV waves.
	Press button for movie blinking vertical with transverse.
	}

Figure~\ref{fig:anis} is a plot of signal amplitudes.
These were the first stacks that Professor Shearer made
because they require no knowledge of polarity.

\activeplot{anis}{width=6.00in,height=7.50in}{\CAKEDIR}{
	Transverse and vertical component of signal magnitude.
	Press button for movie blinking between them
	to see if you believe in whole-earth anisotropy.
	}

A challenging goal is to redo these stacks from the original data
trying to boost the spectral bandwidth of the final result.
This involves statics corrections and corrections for source depth.


\section{ACKNOWLEDGMENT}
I received this data from Peter Shearer
and I received his permission to redistribute it to friends and colleagues
provided I request you to reference its origin,
should you have occasion to copy it.
Examples of earlier versions of these stacks are found in the references.
(The stacks here were done before February 1993.)
Peter may be willing to supply newer and better stacks.
His electronic mail address is {\tt shearer@mahi.ucsd.edu}.


%\mhead{REFERENCES}
%Shearer, PM,
%Imaging Global Body Wave Phases by Stacking long-period seismograms,
%JGR,  v 96, n B12, paaaages 20,535-20,324 Nove 10, 1991
%
%Shearer, P.M.,
%Constraints on upper mantle discontinuities from observations
%of long period reflected and converted phases,
%JGR, vol 96, no.b11, pages 18,147-18,182 Oct 10, 1991

\section{REFERENCES}
\reference{
	Shearer, P.M., 1991,
	Imaging global body wave phases by stacking long-period seismograms.
	J. Geophy. Res.,  v.~96, n.~B12,
	pp. 20,535--20,324.
	% November 10, 1991
	}

\reference{
	Shearer, P.M., 1991,
	Constraints on upper mantle discontinuities from observations
	of long period reflected and converted phases.
	J. Geophy. Res.,  v.~96, n.~B11,
	pp. 18,147--18,182.
	% October 10, 1991
	}

