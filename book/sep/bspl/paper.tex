% Started 06/28/00

%\shortnote

\lefthead{Fomel}
\righthead{Inverse interpolation}
\footer{SEP--105}

\title{Inverse B-spline interpolation}

\email{sergey@sep.stanford.edu}
%\keywords{}

\author{Sergey Fomel}

\maketitle

\begin{abstract}
B-splines provide an accurate and efficient method for interpolating
  regularly spaced data. In this paper, I study the applicability of
  B-spline interpolation in the context of the inverse interpolation
  method for regularizing irregular data.  Numerical tests show that,
  in comparison with lower-order linear interpolation, B-splines lead
  to a faster iterative conversion in under-determined problems and a
  more accurate result in over-determined problems. In addition, they
  provide a constructive method for creating discrete regularization
  operators from continuous differential equations.
\end{abstract}

\section{Introduction}

The problem of interpolating irregularly sampled data to regular grid
(data regularization) can be recast as the inverse process with
respect to interpolating regularly sampled data to irregular
locations.  \longcite{gee} describes an iterative least-squares
optimization approach to data regularization. The optimization is
centered around two goals. The first goal is to minimize the power of
the residual difference between the observed and predicted data. The
second goal is to style the solution according to some predefined
regularization criterion.
\par
The ability of inverse interpolation to reach the data fitting goal
depends on the accuracy of the forward interpolation operator. Forward
interpolation is one of the classic problems in numerical analysis and
has been studied extensively by generations of theoreticians and
practitioners \cite{Fomel.sep.94.sergey2}. The two simplest and most
widely used methods are the nearest neighbor interpolation and linear
interpolation.  There are several approaches for constructing more
accurate (albeit more expensive) linear forward interpolation
operators: cubic convolution \cite{keys}, local Lagrange, tapered sinc
\cite{Harlan.sep.30.103}, etc. \longcite{wolberg} presents a detailed
review of different conventional approaches.
\par
Spline interpolation, based on representing the interpolated function
by smooth piece-wise polynomials, has been in use for a long time
\cite{deBoor}, but only recently \longcite{unser1,unser2} have
discovered a way of implementing forward B-spline interpolation with
an arbitrary order of accuracy in an efficient signal-processing
fashion. The key idea is to implement the B-spline transform with
recursive filtering. First, an efficient recursive filtering
transforms regularly spaced data into spline coefficients, then the
spline coefficients are interpolated onto irregular locations.
B-spline interpolants exhibit a superior performance for any given
order of accuracy in comparison with other methods of similar
efficiency \cite{handbook}.
\par
In this paper, I study the applicability of B-spline interpolation in
the context of the inverse interpolation method. In the first section,
I review the forward interpolation problem and confirm the
observations of \longcite{handbook} about the superior performance of
B-splines. The second section introduces a constructive method of
creating discrete regularization operators from B-splines and helical
filtering \cite{Claerbout.geo.98}. The method performance is evaluated
with a simple numerical test. In conclusion, I summarize the benefits
of using B-splines for data regularization.

\section{Forward Interpolation}

Forward interpolation plays only a supplementary role in this paper,
but it has many applications of its own in the seismic processing
practice. It is sufficient to mention such applications as trace
resampling, NMO, Kirchoff and Stolt migrations, log-stretch, radial
transform, etc. Two simple examples appear at the end of this section.
\par
The general form of a linear forward interpolation operator is 
\begin{equation}\label{eqn:linear}
  f (x) = \sum_{n \in N} W (x, n) f (n)\;,
\end{equation}
where $n$ is a point on a given regular grid $N$, $x$ is a point in
the continuum, $f(x)$ is the reconstructed continuous function, and
$W(x,n)$ is a linear weight. Although in the discussion that follows, I refer
to only the one-dimensional theory, a generalization to many
dimensions is straightforward.

\subsection{Nearest neighbor and beyond}
\inputdir{Math}

The two simplest forms of the forward interpolation operators are the
1-point nearest neighbor interpolation with the weight
\begin{equation}\label{eqn:bin}
  W (x, n) = \left\{\begin{array}{lcr}
1, & \mbox{for} & n - 1/2 \leq x < n + 1/2 \\
0, & \mbox{otherwise} &
\end{array}\right.
\end{equation}
and the 2-point linear interpolation with the weight
\begin{equation}\label{eqn:lin}
  W (x, n) = \left\{\begin{array}{lcr}
1 - |x-n|, & \mbox{for} & n - 1 \leq x < n + 1 \\
0, & \mbox{otherwise} &
\end{array}\right.
\end{equation}
Because of their simplicity, the nearest neighbor and linear
interpolation methods are very practical and easy to apply. Their
accuracy is, however, limited and may be inadequate for
interpolating high-frequency signals. The shapes of
interpolants~(\ref{eqn:bin}) and~(\ref{eqn:lin}) and their spectra are
plotted in Figures~\ref{fig:nnint} and~\ref{fig:linint}. The spectra
plots show that both interpolants act as low-pass filters, preventing
the high-frequency energy from being correctly interpolated.

\plot{nnint}{width=6in}{Nearest neighbor interpolant (left) and its spectrum
  (right).}
\plot{linint}{width=6in}{Linear interpolant (left) and its spectrum
  (right).}
\par
On the other side of the accuracy scale, there is the infinitely long
sinc interpolant:
\begin{equation}\label{eqn:sinc}
  W (x, n) = \frac{\sin \left[\pi (x - n) \right]}{\pi (x - n)} \;.
\end{equation}
According to the sampling theorem \cite{kotelnikov,shannon},
equation~(\ref{eqn:sinc}) provides an optimal interpolation for any
band-limited signal. In practice, it is not directly applicable
because of a prohibitively expensive computation. The shape of the sinc
function and its spectrum are shown in Figure~\ref{fig:sincint}. The
spectrum is identically equal to one in the Nyquist frequency band.

\plot{sincint}{width=6in}{Sinc interpolant (left) and its spectrum
  (right).}
\par
Several approaches exist for extending the nearest neighbor and linear
interpolation to more accurate (albeit more expensive) methods. One
example is the 4-point cubic convolution suggested by \longcite{keys}.
The cubic convolution interpolant is a local piece-wise cubic
function, which approximates the ideal sinc equation~(\ref{eqn:sinc}).
Another popular approach is to taper the ideal sinc function in a
local window. For example, one can use the Kaiser window \cite{kaiser}
\begin{equation}
  \label{eqn:kais}
  W (x, n) = \left\{\begin{array}{lcr} \displaystyle
      \frac{\sin \left[\pi (x - n) \right]}{\pi (x - n)}\,
      \frac{I_0\left(\alpha\,\sqrt{1-\left(\frac{x-n}{N}\right)^2}\right)}
      {I_0(\alpha)} & \mbox{for} & n - N < x < n + N \\
      0, & \mbox{otherwise} &
\end{array}\right.
\end{equation}
where $I_0$ is the zero-order modified Bessel function of the first
kind.  The Kaiser-windowed sinc interpolant~(\ref{eqn:kais}) has the
adjustable parameter $\alpha$, which controls the behavior of its
spectrum. I have found empirically the value of $\alpha=4$ to provide
a spectrum that deviates from 1 by no more than 1\% in a relatively
wide band.
\par
I compare the accuracy of different forward interpolation methods on a
one-dimensional signal shown in Figure~\ref{fig:chirp}. The ideal
signal has an exponential amplitude decay and a quadratic frequency
increase from the center towards the edges. It is sampled at a regular
50-point grid and interpolated to 500 regularly sampled locations. The
interpolation result is compared with the ideal one. Observing
Figures~\ref{fig:binlin}, \ref{fig:lincub}, and \ref{fig:cubkai}, we
can see the interpolation error steadily decreasing as we go
subsequently from 1-point nearest neighbor to 2-point linear, 4-point
cubic convolution, and 8-point windowed sinc interpolation. At the
same time, the cost of interpolation grows proportionally to the
interpolant length.

\inputdir{chirp}
\sideplot{chirp}{height=2.5in}{One-dimensional test signal. Top:
  ideal.  Bottom: sampled at 50 regularly spaced points. The bottom
  plot is the input in a forward interpolation test.}

\sideplot{binlin}{height=2.5in}{Interpolation error of the nearest
  neighbor interpolant (dashed line) compared to that of the linear
  interpolant (solid line).}

\sideplot{lincub}{height=2.5in}{Interpolation error of the linear
  interpolant (dashed line) compared to that of the cubic convolution
  interpolant (solid line).}

\sideplot{cubkai}{height=2.5in}{Interpolation error of the cubic
  convolution interpolant (dashed line) compared to that of the
  8-point windowed sinc interpolant (solid line).}
\par
The differences among different methods are also clearly visible from
the discrete spectra of the corresponding interpolants.  The left plots
in figures~\ref{fig:speclincub} and~\ref{fig:speccubkai} show discrete
interpolation responses: the function $W(x,n)$ for a fixed value of
$x=0.7$. The right plots compare the corresponding discrete spectra. We
can see that the spectrum gets flatter and wider as the accuracy of
the method increases.

\sideplot{speclincub}{height=2.5in}{Discrete interpolation responses of
  linear and cubic convolution interpolants (left) and their
  discrete spectra (right) for $x=0.7$.}

\sideplot{speccubkai}{height=2.5in}{Discrete interpolation responses
  of cubic convolution and 8-point windowed sinc interpolants (left)
  and their discrete spectra (right) for $x=0.7$.}

\subsection{Interpolation and convolution}

As I discussed in an earlier paper \cite{Fomel.sep.94.sergey2}, a
general approach for constructing the interpolant function $W(x,n)$ in
equation~(\ref{eqn:linear}) is to select an appropriate function basis
for representing the function $f(x)$. The functional basis
representation has the general form
\begin{equation}\label{eqn:basis}
  f (x) = \sum_{k \in K} c_k \psi_k (x)\;,
\end{equation}
where $\psi_k(x)$ are basis function, and $c_k$ are the corresponding
coefficients. Once an appropriate basis is selected, one can define
the $W(x,n)$ function by means of the least squares method.
\par
\longcite{unser1} noticed that the function basis idea has an
especially simple implementation if the basis is
convolutional and satisfies the equation
\begin{equation}
  \label{eqn:conv}
  \psi_k (x) = \beta (x-k)\;.
\end{equation}
In other words, the basis is constructed by integer shifts
of a single function $\beta(x)$. Substituting
formula~(\ref{eqn:conv}) into equation~(\ref{eqn:basis}) yields
\begin{equation}\label{eqn:beta}
  f (x) = \sum_{k \in K} c_k \beta (x - k)\;.
\end{equation}
Evaluating the function $f(x)$ in equation~(\ref{eqn:beta}) at an
integer value $n$, we obtain the equation
\begin{equation}\label{eqn:dig}
  f (n) = \sum_{k \in K} c_k \beta (n-k)\;,
\end{equation}
which has the exact form of a discrete convolution. The basis function
$\beta(x)$, evaluated at integer values, is digitally convolved with
the vector of basis coefficients to produce the sampled values of the
function $f(x)$. We can invert equation~(\ref{eqn:dig}) to obtain the
coefficients $c_k$ from $f(n)$ by inverse recursive filtering
(deconvolution). In the case of a non-causal filter $\beta(n)$, an 
appropriate spectral factorization will be
needed prior to applying the recursive filtering.
\par
According to the convolutional basis idea, forward interpolation
becomes a two-step procedure. The first step is the direct inversion
of equation~(\ref{eqn:dig}): the basis coefficients $c_k$ are found by
deconvolving the sampled function $f(n)$ with the factorized filter
$\beta(n)$. The second step reconstructs the continuous (or arbitrarily
sampled) function $f(x)$ according to formula~(\ref{eqn:beta}). The
two steps could be combined into one, but usually it is more
convenient to apply them separately. I show a schematic relationship
among different variables in Figure~\ref{fig:scheme}.

\inputdir{XFig}
\sideplot{scheme}{height=1.5in}{Schematic relationship among
  different variables for interpolation with a convolutional basis.}

\subsection{B-splines}

B-splines represent a particular example of a convolutional basis.
Because of their compact support and other attractive numerical
properties, B-splines are a good basis choice for the forward
interpolation problem and related signal processing problems \cite{unser}.
\par
B-splines of the order 0 and 1 coincide with the nearest neighbor and
linear interpolants~(\ref{eqn:bin}) and~(\ref{eqn:lin}) respectively.
B-splines $\beta^n(x)$ of a higher order $n$ can be defined by a
repetitive convolution of the zeroth-order spline $\beta^0(x)$ (the
box function) with itself:
\begin{equation}
\label{eqn:bdef}
\beta^n(x) = 
\underbrace{\beta^0(x) \ast \cdots \ast \beta^0(x)}_{(n+1)\quad 
\mbox{times}}\;.
\end{equation}
There is also the explicit expression
\begin{equation}
\label{eqn:expl}
\beta^n(x) = 
\frac{1}{n!}\,\sum_{k=0}^{n+1} C_k^{n+1} (-1)^k 
(x + \frac{n+1}{2} - k)_{+}^n\;,
\end{equation}
which can be proved by induction. Here $C_k^{n+1}$ are the binomial
coefficients, and the function $x_{+}$ is defined as follows:
\begin{equation}
  \label{eqn:xp}
  x_{+} = \left\{\begin{array}{lcr}
x, & \mbox{for} & x > 0 \\
0, & \mbox{otherwise} &
\end{array}\right.
\end{equation}
As follows from formula~(\ref{eqn:expl}), the most commonly used cubic
B-spline $\beta^3(x)$ has the expression
\begin{equation}
  \label{eqn:beta3}
  \beta^3(x) = \left\{\begin{array}{lcr}
\displaystyle \left(4-6 |x|^2+3 |x|^3\right)/6, & 
\mbox{for} & 1 > |x| \geq 0 \\
\displaystyle (2-|x|)^3/6, & \mbox{for} & 2 > |x| \geq 1 \\
0, & \mbox{elsewhere} &
\end{array}\right.
\end{equation}
The corresponding discrete filter $\beta^3(n)$ is a centered 3-point
filter with coefficients 1/6, 2/3, and 1/6. According to the
traditional method, a deconvolution with this filter is performed as a
tridiagonal matrix inversion \cite{deBoor}. One can accomplish it more
efficiently by spectral factorization and recursive filtering
\cite{unser1}. The recursive filtering approach generalizes
straightforwardly to B-splines of higher orders.
\par
Both the support length and the smoothness of B-splines increase with
the order. In the limit, B-slines converge to the Gaussian function.
Figures~\ref{fig:splint3} and~\ref{fig:splint7} show the third- and
seventh-order splines $\beta^3(x)$ and $\beta^7(x)$ and their
continuous spectra.

\inputdir{Math}

\plot{splint3}{width=6in}{Third-order B-spline $\beta^3(x)$ (left)
  and its spectrum (right).}

\plot{splint7}{width=6in}{Seventh-order B-spline $\beta^7(x)$ (left)
  and its spectrum (right).}
\par
It is important to realize the difference between B-splines and the
corresponding interpolants $W(x,n)$, which are sometimes called
\emph{cardinal splines}.  An explicit computation of the cardinal
splines is impractical, because they have infinitely long support.
Typically, they are constructed implicitly by the two-step
interpolation method, outlined in the previous subsection. The
cardinal splines of orders 3 and 7 and their spectra are shown in
Figures~\ref{fig:crdint3} and~\ref{fig:crdint7}. As B-splines converge
to the Gaussian function, the corresponding interpolants rapidly
converge to the sinc function~(\ref{eqn:sinc}). A good convergence
is achieved with the help of the infinitely long support, which
results from recursive filtering at the first step of the
interpolation procedure.

\plot{crdint3}{width=6in}{Effective third-order B-spline interpolant
  (left) and its spectrum (right).}

\plot{crdint7}{width=6in}{Effective seventh-order B-spline interpolant
  (left) and its spectrum (right).}
\par
\inputdir{chirp}
In practice, the recursive filtering step adds only marginally to the
total interpolation cost. Therefore, an $n$-th order B-spline
interpolation is comparable in cost with any other method with an
$(n+1)$-point interpolant. The comparison in accuracy usually turns
out in favor of B-splines. Figures~\ref{fig:cubspl}
and~\ref{fig:kaispl} compare interpolation errors of B-splines and
other similar-cost methods on the example from Figure~\ref{fig:chirp}.

\sideplot{cubspl4}{height=2.5in}{Interpolation error of the cubic
  convolution interpolant (dashed line) compared to that of the
  third-order B-spline (solid line).}

\sideplot{kaispl8}{height=2.5in}{Interpolation error of the 8-point
  windowed sinc interpolant (dashed line) compared to that of the
  seventh-order B-spline (solid line).}
\par
Similarly to Figures~\ref{fig:speclincub} and~\ref{fig:speccubkai}, we
can also compare the discrete responses of B-spline interpolation with
those of other methods. The right plots in
Figures~\ref{fig:speccubspl} and~\ref{fig:speckaispl} show that the
discrete spectra of the effective B-spline interpolants are genuinely
flat at low frequencies and wider than those of the competitive
methods. Although the B-spline responses are infinitely long because
of the recursive filtering step, they exhibit a fast amplitude decay.

\sideplot{speccubspl4}{height=2.5in}{Discrete interpolation responses
  of cubic convolution and third-order B-spline interpolants (left)
  and their discrete spectra (right) for $x=0.7$.}

\sideplot{speckaispl8}{height=2.5in}{Discrete interpolation responses of
  8-point windowed sinc and seventh-order B-spline interpolants (left)
  and their discrete spectra (right) for $x=0.7$.}

\subsection{2-D example}
\inputdir{chirp2}

For completeness, I include a 2-D forward interpolation example.
Figure~\ref{fig:chirp2} shows a 2-D analog of function in
Figure~\ref{fig:chirp} and its coarsely-sampled version.

\plot{chirp2}{width=6in,height=3in}{Two-dimensional test function
  (left) and its coarsely sampled version (right).}
\par
Figure~\ref{fig:plcbinlin} compares the errors of the 2-D nearest
neighbor and 2-D linear (bi-linear) interpolation. Switching to
bi-linear interpolation shows a significant improvement, but the error
level is still relatively high. As shown in
Figures~\ref{fig:plccubspl} and~\ref{fig:plckaispl}, B-spline
interpolation again outperforms other methods with comparable
cost complexity. In all cases, I constructed 2-D interpolants by orthogonal
splitting. Although the splitting method reduces computational
overhead, the main cost factor is the total interpolant size, which
squares when going from 1-D to 2-D.

\plot{plcbinlin}{width=6in,height=3.5in}{2-D Interpolation errors of
  nearest neighbor interpolation (left) and linear interpolation
  (right).  Top graphs show 1-D slices through the center of the
  image.}

\plot{plccubspl}{width=6in,height=3.5in}{2-D Interpolation errors of
  cubic convolution interpolation (left) and third-order B-spline
  interpolation (right).  Top graphs show 1-D slices through the
  center of the image.}

\plot{plckaispl}{width=6in,height=3.5in}{2-D Interpolation errors of
  8-point windowed sinc interpolation (left) and seventh-order
  B-spline interpolation (right).  Top graphs show 1-D slices through
  the center of the images.}

\subsection{Beyond B-splines}

It is not too difficult to construct a convolutional basis with better
interpolation properties than those of B-splines, for example by
sacrificing their smoothness. The following piece-wise cubic function
has a lower smoothness than $\beta^3(x)$ in equation~(\ref{eqn:beta3})
but slightly better interpolation behavior:
\begin{equation}
  \label{eqn:mu3}
  \mu^3(x) = \left\{\begin{array}{lcr}
\displaystyle \left(10-13 |x|^2+6 |x|^3\right)/16, & 
\mbox{for} & 1 > |x| \geq 0 \\
\displaystyle (2-|x|)^2 (5-2 |x|)/16, & \mbox{for} & 2 > |x| \geq 1 \\
0, & \mbox{elsewhere} &
\end{array}\right.
\end{equation}
\par
Figures \ref{fig:splmom4} and \ref{fig:specsplmom4} compare the test
interpolation errors and discrete responses of methods based on the
B-spline function $\beta^3(x)$ and the lower smoothness function
$\mu^3(x)$. The latter method has a slight but visible performance
advantage and a slightly wider discrete spectrum.

\sideplot{splmom4}{height=2.5in}{Interpolation error of the
  third-order B-spline interpolant (dashed line) compared to that of
  the lower smoothness spline interpolant (solid line).}
\sideplot{specsplmom4}{height=2.5in}{Discrete interpolation responses
  of third-order B-spline and lower smoothness spline interpolants
  (left) and their discrete spectra (right) for $x=0.7$.}
\par
\longcite{mom} have developed a general approach for constructing
non-smooth piece-wise functions with optimal interpolation properties.
However, the gain in accuracy is often negligible in practice. In the
rest of this paper, I use the classic B-spline method.

\subsection{Seismic applications of forward interpolation}

For completeness, I conclude this section with two simple examples of
forward interpolation in seismic data processing.
Figure~\ref{fig:stolt} shows a 3-D impulse response of Stolt migration
\cite{GEO43.01.00230048}, computed by using 2-point linear
interpolation and 8-point B-spline interpolation. As noted by
\longcite{Ronen.sep.30.95} and \longcite{Harlan.sep.30.103},
inaccurate interpolation may lead to spurious artifact events in
Stolt-migrated image. Indeed, we see several artifacts for the image
with linear interpolation (the left plots in Figure~\ref{fig:stolt}.)
The artifacts are removed by a more accurate interpolation method (the
right plots in Figure~\ref{fig:stolt}.)

\plot{stolt}{width=5.5in}{Stolt migration impulse response. Left: using
  linear interpolation. Right: using seventh-order B-spline
  interpolation. Migration artifacts are removed by a more accurate
  forward interpolation method.}

\par
Another simple example is radial trace transform
\cite{Ottolini.sepphd.33} Figure~\ref{fig:radialdat} shows a land shot
gather contaminated by nearly radial ground-roll. As discussed by
\longcite{Claerbout.sep.35.43}, \longcite{SEG-1999-0601}, and
\longcite{Brown.sep.103.morgan1}, one can effectively eliminate
ground-roll noise by applying radial trace transform, followed by
high-pass filtering and the inverse radial transform.
Figure~\ref{fig:radial} shows the result of the forward radial
transform of the shot gather in Figure~\ref{fig:radialdat} in the
radial band of the ground-roll noise and the transform error after
going back to the original domain. Comparing the results of using
linear and third-order B-spline interpolation, we see once again that
the transform artifacts are removed with a more accurate interpolation
scheme.

\sideplot{radialdat}{height=2.5in}{Ground-roll-contaminated shot
  gather used in a radial transform test}

\plot{radial}{width=5.5in}{Radial trace transform results. Top: radial
  trace domain. Bottom: residual error after the inverse transform.
  The error should be zero in a radial band from 0 to 0.65 km/s radial
  velocity. Left: using linear interpolation. Right: using third-order
  B-spline interpolation.}

\section{Inverse Interpolation and Data Regularization}

In the notation of \longcite{gee}, inverse interpolation amounts to a
least-squares solution of the system
\begin{eqnarray}
  \label{eqn:fit}
  \bold{L m} & \approx & \bold{d}\;; \\
  \label{eqn:reg}
  \epsilon \bold{A m} & \approx & \bold{0}\;, 
\end{eqnarray}
where $\bold{d}$ is a vector of known data $f(x_i)$ at irregular
locations $x_i$, $\bold{m}$ is a vector of unknown function values
$f(n)$ at a regular grid $n$, $\bold{L}$ is a linear interpolation
operator of the general form~(\ref{eqn:linear}), $\bold{A}$ is an
appropriate regularization (model styling) operator, and $\epsilon$ is
a scaling parameter. In the case of B-spline interpolation, the
forward interpolation operator $\bold{L}$ becomes a cascade of two
operators: recursive deconvolution $\bold{B}^{-1}$, which converts the
model vector $\bold{m}$ to the vector of spline coefficients
$\bold{c}$, and a spline basis construction operator $\bold{F}$.
System~(\ref{eqn:fit}-\ref{eqn:reg}) transforms to
\begin{eqnarray}
  \label{eqn:bfit}
  \bold{F B^{-1} m} & \approx & \bold{d}\;; \\
  \label{eqn:breg}
  \epsilon \bold{A m} & \approx & \bold{0}\;. 
\end{eqnarray}
We can rewrite (\ref{eqn:bfit}-\ref{eqn:breg}) in the form that
involves only spline coefficients:
\begin{eqnarray}
  \label{eqn:cfit}
  \bold{W c} & \approx & \bold{d}\;; \\
  \label{eqn:creg}
  \epsilon \bold{A B c} & \approx & \bold{0}\;. 
\end{eqnarray}
After we find a solution of system~(\ref{eqn:cfit}-\ref{eqn:creg}),
the model $\bold{m}$ will be reconstructed by the simple convolution
\begin{equation}
  \label{eqn:post}
  \bold{m = B c}\;.
\end{equation}
This approach resembles a more general method of model preconditioning
\cite{Fomel.sep.94.sergey1}.
\par
The inconvenient part of system~(\ref{eqn:cfit}-\ref{eqn:creg}) is the
complex regularization operator $\bold{A B}$. Is it possible to avoid
the cascade of $\bold{B}$ and $\bold{A}$ and to construct a
regularization operator directly applicable to the spline coefficients
$\bold{c}$? In the following subsection, I develop a method for
constructing spline regularization operators from differential
equations.

\subsection{Spline regularization}

In many cases, the regularization (styling) condition originates in a
continuous differential operator. For example, one can think of the
gradient or Laplacian operator for regularizing smooth functions
\cite{Fomel.sep.103.sergey1}, plane-wave destructor for regularizing
local plane waves \cite{Fomel.sep.105.sergey1}, or the offset
continuation equation for regularizing seismic reflection data
\cite{Fomel.sep.103.sergey2}. 
\par
Let us denote the continuous regularization operator by $D$.
Regularization implies seeking a function $f(x)$ such that the
least-squares norm of $D\left[f(x)\right]$ is minimum. Using the usual
expression for the least-squares norm of continuous functions and
substituting the basis decomposition~(\ref{eqn:beta}), we obtain 
the expression
\begin{equation}
  \label{eqn:goal}
  \left\|D\left[f(x)\right]\right\| = 
  \int \left(D\left[f(x)\right]\right)^2\,dx =
  \int \left(\sum_{k \in K} c_k D\left[ \beta (x-k)\right]\right)^2\,dx\;.
\end{equation}
The problem of finding function $f(x)$ reduces to the problem of
finding the corresponding set of basis coefficients $c_k$. We can
obtain the solution to the least-squares optimization by
differentiating the quadratic objective function~(\ref{eqn:goal}) with
respect to the basis coefficients $c_k$. This leads to the system of
linear equations
\begin{equation}
  \label{eqn:norm}
  \sum_{k \in K} c_k \int D \left[\beta (x-k)\right] 
  D\left[\beta (x-j)\right] \,dx = 
  \sum_{k \in K} c_k d_{j-k} = 0\;,
\end{equation}
where
\begin{equation}
\label{eqn:dc}
d_j = \int D\left[\beta (x)\right] D\left[\beta (x-j)\right]\,dx\;.
\end{equation}
Equation~(\ref{eqn:norm}) is clearly a discrete convolution of the
spline coefficients~$c_k$ with the filter $d_j$ defined in
equation~(\ref{eqn:dc}). To transform the system~(\ref{eqn:norm}) to a
regularization condition of the form
\begin{equation}
  \label{eqn:dtd}
  \bold{D c} \approx \bold{0}\;,
\end{equation}
we need to treat the digital filter $d_j$ as an autocorrelation and
find its minimum-phase factor. Equation~(\ref{eqn:dtd}) replaces
equation~(\ref{eqn:creg}) in the inverse interpolation problem setting.
\par
We have found a constructive way of creating B-spline regularization
operators from continuous differential equations.
\par
A simple regularization example is shown in Figure~\ref{fig:sthree}.
The continuous operator $D$ in this case comes from the theoretical
plane-wave differential equation.  I constructed the auto-correlation
filter $d_j$ according to formula~(\ref{eqn:dc}) and factorized it
with the efficient Wilson-Burg method on a helix
\cite{Sava.sep.97.paul1}. The figure shows three plane waves
constructed from three distant spikes by applying an inverse recursive
filtering with two different plane-wave regularizers. The left plot
corresponds to using first-order B-splines (equivalent to linear
interpolation). This type of regularizer is identical to Clapp's
steering filters \cite{Clapp.sep.95.bob1} and suffers from numerical
dispersion effects. The right plot was obtained with third-order
splines. Most of the dispersion is suppressed by using a more accurate
interpolation.

\plot{sthree}{width=6in,height=3in}{B-spline regularization. Three
  plane waves constructed by 2-D recursive filtering with the B-spline
  plane-wave regularizer. Left: using first-order B-splines (linear
  interpolation). Right: using third-order B-splines.}

\subsection{Test example}

Now that we have all the problem pieces together, we can test the
performance gain in the inverse interpolation
problem~(\ref{eqn:cfit})-(\ref{eqn:dtd}) from the application of B-splines.
\par
For a simple 1-D test, I chose the function shown in
Figure~\ref{fig:chirp}, but sampled at irregular locations. To create two
different regimes for the inverse interpolation problem, I chose 50
and 500 random locations. The two sets of points were interpolated to
500 and 50 regular samples respectively. The first test corresponds to
an under-determined situation, while the second test is clearly
over-determined. Figures~\ref{fig:bin500} and~\ref{fig:bin50} show the
input data for the two test after normalized binning to the selected
regular bins.

\sideplot{bin500}{height=2.5in}{50 random points binned to 500 regular
  grid points.  The random data are used for testing inverse
  interpolation in an under-determined situation.}
\sideplot{bin50}{height=2.5in}{500 random points binned to 50 regular
  grid points.  The random data are used for testing inverse
  interpolation in an over-determined situation.}

\par
I solved system~(\ref{eqn:cfit})-(\ref{eqn:dtd}) by the iterative
conjugate-gradient method, utilizing a recursive filter
preconditioning \cite{Fomel.sep.94.sergey1} for faster convergence.
The regularization operator $\bold{D}$ was constructed by using the
method of the previous subsection with the tension-spline differential
equation \cite{GEO55.03.02930305,Fomel.sep.103.sergey1} and the
tension parameter of $0.01$.
\par
The least-squares differences between the true and the estimated model
are plotted in Figures~\ref{fig:norm500} and \ref{fig:norm50}.
Observing the behavior of the model misfit versus the number of
iterations and comparing simple linear interpolation with the
third-order B-spline interpolation, we discover that
\begin{itemize}
\item In the under-determined case, both methods converge to the same
  final estimate, but B-spline inverse interpolation does it faster at
  earlier iterations. The total computational gain is not significant,
  because each B-spline iteration is more expensive than the
  corresponding linear interpolation iteration.
\item In the over-determined case, both methods converge similarly at
  early iterations, but B-spline inverse interpolation results in a
  more accurate final estimate.
\end{itemize}
From the results of this simple experiment, it is apparent that the
main advantage of using more accurate interpolation in the data
regularization context occurs in the over-determined situation, when
the estimated model is well constrained by the available data.

\sideplot{norm500}{height=2.5in}{Model convergence in the
  under-determined case. Dashed line: using linear
  interpolation. Solid line: using third-order B-spline.}
\sideplot{norm50}{height=2.5in}{Model convergence in the
  over-determined case. Dashed line: using linear
  interpolation. Solid line: using third-order B-spline.}

\subsection{Application to 3-D seismic data regularization}

In this subsection, I demonstrate an application of B-spline inverse
interpolation for regularizing three-dimensional seismic reflection
data. The dataset of this example comes from the North Sea and was
used before for testing AMO \cite{Biondi.geo.amo98} and common-azimuth
migration \cite{Biondi.sep.93.1}. Figure~\ref{fig:cmpfold} shows the
midpoint geometry and the corresponding bin fold for a selected range
of offsets and azimuths. The goal of data regularization is to create
a regular data cube at the specified bins from the irregular input
data, preprocessed by NMO. As typical of marine acquisition, the fold
distribution is fairly regular but has occasional gaps caused by the
cable feathering effect.

\plot{cmpfold}{width=6in,height=3.5in}{Midpoint geometry (left) and
  fold distribution (right) for the 3-D data test}

The data cube after normalized binning (inverse nearest neighbor
interpolation) is shown in Figure~\ref{fig:bin1}.  Binning works
reasonably well in the areas of large fold but fails to fill the zero
fold gaps and has an overall limited accuracy.

\plot{bin1}{width=6in}{3-D data after normalized binning}

Inverse interpolation using bi-linear interpolants significantly
improves the result (Figure~\ref{fig:int2}), and inverse B-spline
interpolation improves the accuracy even further
(Figure~\ref{fig:int4}). In both cases, I regularized the data in
constant time slices, using recursive filter preconditioning with
plane-wave destructor filters analogous to those in
Figure~\ref{fig:sthree}. The plane wave slope was estimated from the
binned data with the method of \longcite{Fomel.sep.105.sergey1}.  The
inverse interpolation results preserve both flat reflection events in
the data and steeply-dipping diffractions. When data regularization is
used as a preprocessing step for common-azimuth migration
\cite{Biondi.geo.comaz96}, preserving diffractions is important for
correct imaging of sharp edges in the subsurface structure.

\plot{int2}{width=6in}{3-D data after inverse interpolation with
  bi-linear interpolants}

\plot{int4}{width=6in}{3-D data after inverse interpolation with
  third-order B-spline interpolants}

\section{Conclusions}

I have reviewed the B-spline forward interpolation method and
confirmed the observation of \longcite{handbook} about its superior
performance in comparison with other methods of similar cost. Whenever
an accurate forward interpolation scheme is desired, B-splines can be
an extremely valuable tool. B-spline forward interpolation involves
two steps. The first step is recursive filtering, which results in a
set of spline coefficients. The second step is a linear spline
interpolation operator.
\par
Analyzing the role of B-spline interpolation in data regularization, I
have introduced a method of constructing B-spline discrete
regularization operators from continuous differential equations.
\par
Simple numerical experiments with B-spline inverse interpolation show
that the main advantage of using a more accurate interpolation scheme
occurs in an over-determined setting, where B-splines lead to a more
accurate model estimates. In an under-determined setting, the B-spline
inverse interpolation scheme converges faster at early iterations, but
the total computational gain may be insignificant.
\par
I have shown on a simple real data example that inverse B-spline
interpolation can be used as an accurate method of data regularization
for processing 3-D seismic reflection data.

\section{Acknowledgments}

A short conversation with Dave Hale brought me to a better
understanding of different forward interpolation methods.
\par
The 3-D North Sea dataset was released to SEP by Conoco and its
partners, BP and Mobil.

\bibliographystyle{sep}
\bibliography{MISC,SEP,GEOTLE,spline}

%\APPENDIX{A}

%\plot{name}{width=6in,height=}{caption}
%\sideplot{name}{height=1.5in,width=}{caption}
%
%\begin{equation}
%\label{eqn:}
%\end{equation}
%
%\ref{fig:}
%(\ref{eqn:})
