\lefthead{Fomel}
\righthead{Spectral velocity continuation}
\footer{SEP--97}

\title{Velocity continuation by spectral methods}

\email{sergey@sep.stanford.edu} 
%\keywords{imaging, velocity continuation, spectral methods, Fourier,
%  Chebyshev}

\author{Sergey Fomel}

\maketitle

\begin{abstract}
I apply Fourier and Chebyshev spectral methods to derive accurate
  and efficient algorithms for velocity continuation.  As expected,
  the accuracy of the spectral methods is noticeably superior to that
  of the finite-difference approach. Both methods apply a
  transformation of the time axis to squared time. The Chebyshev
  method is slightly less efficient than the Fourier method, but has
  less problems with the time transformation and also handles
  accurately the non-periodic boundary conditions.
\end{abstract}

\section{Introduction}
\par
In a recent work \cite[]{me,Fomel.sep.92.159}, I
introduced the process of \emph{velocity continuation} to describe a
continuous transformation of seismic time-migrated images with a
change of the migration velocity. Velocity continuation generalizes
the ideas of residual migration
\cite[]{GEO50-01-01100126,Etgen.sepphd.68} and cascaded migrations
\cite[]{GEO52-05-06180643}. In the zero-offset (post-stack) case, the
velocity continuation process is governed by a partial differential
equation in midpoint, time, and velocity coordinates, first discovered
by \cite{Claerbout.sep.48.79}. \cite{hubral1} and
\cite{hubral2} describe this process in a broader context of
``image waves''. Generalizations are possible for the non-zero offset
(prestack) case \cite[]{Fomel.sep.92.159,SEG-1997-1762}.
\par
A numerical implementation of velocity continuation process provides
an efficient method of scanning the velocity dimension in the search
of an optimally focused image. The first implementations
\cite[]{Li.sep.48.85,Fomel.sep.92.159} used an analogy with Claerbout's
15-degree depth extrapolation equation to construct a
finite-difference scheme with an implicit unconditionally stable
advancement in velocity. \cite{Fomel.sep.95.sergey2} presented an
efficient three-dimensional generalization, applying the helix
transform \cite[]{Claerbout.sep.95.jon1}.
\par
%\plot{fd-imp}{width=6in}{Impulse responses (Green's functions) of
%  velocity continuation, computed by a second-order finite-difference
%  method.  The left plots corresponds to continuation to a larger
%  velocity ($+1$ km/sec); the right plot, smaller velocity, ($-1$
%  km/sec).}
\par
A low-order finite-difference method is probably the most efficient
numerical approach to this method, requiring the least work per
velocity step. However, its accuracy is not optimal because of the
well-known numerical dispersion effect. Figure \ref{fig:fd-imp} shows
impulse responses of post-stack velocity continuation for three
impulses, computed by the second-order finite-difference method
\cite[]{Fomel.sep.92.159}. As expected from the residual migration
theory \cite[]{GEO50-01-01100126}, continuation to a higher velocity
(left plot) corresponds to migration with a residual velocity, and its
impulse responses have an elliptical shape. Continuation to a smaller
velocity (right plot in Figure \ref{fig:fd-imp}) corresponds to
demigration (modeling), and its impulse responses have a hyperbolic
shape. The dispersion artifacts are clearly visible in the figure.
\par
In this paper, I explore the possibility of implementing a numerical
velocity continuation by spectral methods. I adopted two different
methods, comparable in efficiency with finite differences.  The first
method is a direct application of the Fast Fourier Transform (FFT)
technique.  The second method transforms the time grid to Chebyshev
collocation points, which leads to an application of the
Chebyshev-$\tau$ method \cite[]{lanc,orszag,boyd}, combined with an
unconditionally stable implicit advancement in velocity.  Both methods
employ a transformation of the grid from time $t$ to the squared time
$\sigma = t^2$, which removes the dependence on $t$ from the
coefficients of the velocity continuation equation. Additionally, the
Fourier transform in the space (midpoint) variable $x$ takes care of
the spatial dependencies. This transform is a major source of
efficiency, because different wavenumber slices can be processed
independently on a parallel computer before transforming them back to
the physical space.
\par
\section{Problem formulation}
\par
The post-stack velocity continuation process is governed by a partial
differential equation in the domain, composed by the seismic image
coordinates (midpoint $x$ and vertical time $t$) and the additional
velocity coordinate $v$. Neglecting some amplitude-correcting terms
\cite[]{Fomel.sep.92.159}, the equation takes the form
\cite[]{Claerbout.sep.48.79}
\begin{equation}
  \label{PDE}
  {{\partial^2 P} \over {\partial v\, \partial t}} +
  v\,t\,{{\partial^2 P} \over {\partial x^2}} = 0\;.
\end{equation}
Equation (\ref{PDE}) is linear and belongs to the hyperbolic type. It
describes a wave-type process with the velocity $v$ acting as a
``time-like'' variable. Each constant-$v$ slice of the function
$P(x,t,v)$ corresponds to an image with the corresponding constant
velocity. The necessary boundary and initial conditions are 
\begin{equation}
  \label{BC}
  \left.P\right|_{t=T} = 0\;\quad \left.P\right|_{v=v_0} = P_0 (x,t)\;,
\end{equation}
where $v_0$ is the starting velocity, $T=0$ for continuation to a
smaller velocity and $T$ is the largest time on the image (completely
attenuated reflection energy) for continuation to a larger velocity.
The first case corresponds to ``modeling''; the latter case, to
seismic migration.
\par
Mathematically, equations (\ref{PDE}) and (\ref{BC}) define a
Goursat-type problem \cite[]{kurant}. Its analytical solution can be
constructed by a variation of the Riemann method in the form of an
integral operator \cite[]{me,Fomel.sep.92.159}:
\begin{equation}
  \label{kirch}
  P(t,x,v) = \frac{1}{(2\,\pi)^{m/2}}\,\int\, 
  \frac{1}{(\sqrt{v^2-v_0^2}\,\rho)^{m/2}}\, 
  \left(- \frac{\partial}{\partial t_0}\right)^{m/2}
  P_0\left(\frac{\rho}{\sqrt{v^2-v_0^2}},x_0\right)\,dx_0\;,
\end{equation}
where $\rho = \sqrt{(v^2-v_0^2)\,t^2 + (x - x_0)^2}$, $m=1$ in the 2-D
case, and $m=2$ in the 3-D case. In the case of continuation from zero
velocity $v_0=0$, operator (\ref{kirch}) is equivalent (up to the
amplitude weighting) to conventional Kirchoff time migration
\cite[]{GEO43-01-00490076}.  Similarly, in the frequency-wavenumber
domain, velocity continuation takes the form
\begin{equation}
  \label{stolt}
  \hat{P} (\omega,k,v) = \hat{P}_0 (\sqrt{\omega^2+k^2 (v^2-v_0^2)},k)\;,
\end{equation}
which is equivalent (up to scaling coefficients) to Stolt migration
\cite[]{GEO50-11-22192244}, regarded as the most efficient migration
method.
\par
If our task is to create many constant-velocity slices, there are
other ways to construct the solution of problem (\ref{PDE}-\ref{BC}).
Two alternative spectral approaches are discussed in the next two
sections.
\par
\section{Fourier approach}
\par
Introducing the change of variable $\sigma = t^2$, we can transform
equation (\ref{PDE}) to the form
\begin{equation}
  \label{PDE2}
  2\,{{\partial^2 P} \over {\partial v\, \partial \sigma}} +
  v\,{{\partial^2 P} \over {\partial x^2}} = 0\;,
\end{equation}
whose coefficients don't depend on the time variables.  Double Fourier
transform in $\sigma$ and $x$ further simplifies equation (\ref{PDE2})
to the ordinary differential equation
\begin{equation}
  \label{ODE}
  2\,i\Omega\,{{d^2 \hat{P}} \over {d v}} -
  v\,k^2\,\hat{P} = 0\;,
\end{equation}
where the frequency $\Omega$ corresponds to the time coordinate
$\sigma$, and $k$ is the wavenumber in $x$. Equation (\ref{ODE}) has
an explicit analytical solution
\begin{equation}
  \label{ODEsol}
  \hat{P} (k,\Omega,v) = \hat{P}_0 (k,\Omega)\,
  e^{\frac{i k^2(v_0^2-v^2)}{4\Omega}}\;,
\end{equation}
which defines a very simple algorithm for the numerical velocity
continuation. The algorithms consists of the following steps:
\begin{enumerate}
\item Transform the input from a regular grid in $t$ to a regular grid
  in $\sigma$.
\item Apply FFT in $x$ and $\sigma$.
\item Multiply by the all-pass phase-shift filter $e^{\frac{i
      k^2(v_0^2-v^2)}{4\Omega}}$.
\item Inverse FFT in $x$ and $\sigma$.
\item Inverse transform to a regular grid in $t$.
\end{enumerate}

%\plot{t2}{width=6in}{Synthetic seismic data before (left) and after
%  (right) transformation to the $\sigma$ grid.}

Figure \ref{fig:t2} shows a simple synthetic model of seismic
reflection data from \cite[]{Claerbout.bei.95} before and after
transforming the grid, regularly spaced in $t$, to a grid, regular in
$\sigma$. The left plot of Figure \ref{fig:t2-fft} shows the Fourier
transform of the data. Except for the nearly vertical event, which
corresponds to a stack of parallel layers in the shallow part of the
data, the data frequency range is contained near the origin in the
$\Omega-k$ space.  The right plot of Figure \ref{fig:t2-fft} shows the
phase-shift filter for continuation from zero imaging velocity (which
corresponds to unprocessed data) to the velocity of 1 km/sec. The
rapidly oscillating part (small frequencies and large wavenumbers) is
exactly in the place, where the data spectrum is zero and corresponds
to physically impossible reflection events.

%\plot{t2-fft}{width=6in}{Left: the real part of the data Fourier
%  transform.  Right: the real part of the velocity continuation
%  operator (continuation from 0 to 1 km/s) in the Fourier domain.}

Algorithm (\ref{ODEsol}) is very attractive from the practical point
of view because of its efficiency (based on the FFT algorithm). The
operations count is roughly the same as in Stolt migration
(\ref{stolt}): two forward and inverse FFTs and forward and inverse
grid transform with interpolation (one complex-number transform in the
case of Stolt migration). Algorithm (\ref{ODEsol}) can be even more
efficient than Stolt method because of the simpler structure of the
innermost loop.  However, its practical implementation faces two
difficult problems: artifacts of the $t^2$ grid transform and
wraparound artifacts
\par
\subsection{Improving the accuracy of the $t^2$ grid transform}
\par
The first problem is the loss of information in the transform to the
$t^2$ grid. As illustrated in Figure \ref{fig:t2}, the shallow part of
the data gets severely compressed in the $t^2$ grid. The amount of
compression can lead to inadequate sampling, and as a result, aliasing
artifacts in the frequency domain. Moreover, it can be difficult to
recover from the loss of information in the transformed domain when
transforming back into the original grid. A partial remedy for this
problem is to increase the grid size in the $t^2$ domain. The top plots
in Figure \ref{fig:fft-inv} show the result of back transformation to
the $t$ grid and the difference between this result and the original
model (plotted on the same scale). We can see a noticeable loss of
information in the upper (shallow) part of the data, caused by
undersampling. The bottom plots in Figure \ref{fig:fft-inv} correspond
to increasing the grid size by a factor of three. Some of the
artifacts have been suppressed, at the expense of dealing with a
larger grid.

%\plot{fft-inv}{width=6in}{The left plots show the reconstruction of the original data 
%  after transforming back from the $t^2$ grid to the original $t$
%  grid.  The right plots show the difference with the original model.
%  Top: using the original grid size ($N_t = 200$). Bottom: increasing
%  the grid size by a factor of three.}

To perform an accurate transform of the grid, I adopted the following
method, inspired by \cite[]{Claerbout.sep.48.347}. Let $d_{\mbox{\tiny
    new}}$ denote the data on the new grid and $d_{\mbox{\tiny old}}$
be the data on the old grid. If $L$ is the interpolation operator,
defined on the new grid, then the optimal least-square transformation
is
\begin{equation}
  \label{interp}
  d_{\mbox{\tiny new}} = (L^T L)^{-1}\,L\,d_{\mbox{\tiny old}}\;,
\end{equation}
where $L^T$ denotes the adjoint interpolation operator. The operator
$(L^T L)^{-1}$ provides a proper scaling of the result. If we use
simple linear interpolation for the $L$ operator, then $L^T L$ is a
tridiagonal matrix, which can be easily inverted (in $8 N$
operations). If some parts in $d_{\mbox{\tiny new}}$ are not fully
constrained, then the tridiagonal matrix is not invertible. To obtain
a solution in this case, we can include a regularization operator $D$
in (\ref{interp}), as follows:
\begin{equation}
  \label{regul}
  d_{\mbox{\tiny new}} = (L^T L + \epsilon^2 D)^{-1}\,L\,d_{\mbox{\tiny
      old}}\;,
\end{equation}
A convenient choice for $D$ is a second derivative operator,
represented with the second-order finite-difference approximation.
This operator allows the selection of the smoothest possible function
$d_{\mbox{new}}$ while preserving the efficient tridiagonal structure
of $L^T L + \epsilon^2 D$. In this problem, the parameter $\epsilon$
can be chosen as small as possible, as long as it prevents the
inversion from getting unstable.
\par
\subsection{Suppressing wraparound artifacts of the Fourier method}
\par
The periodic boundary conditions both in the squared time $\sigma$ and
the spatial coordinate $x$, implied by the Fourier approach, are
artificial in the problem of velocity continuation. The artificial
periodicity is convenient from the computational point of view.
However, false periodic events (wraparound artifacts) should be
suppressed in the final output. A natural method for attacking this
problem is to apply zero padding in the physical space prior to
Fourier transform. Of course, this method involves an additional
expense of the grid size increase.
\par
%\plot{fft-imp}{width=6in}{Impulse responses (Green's functions) of velocity continuation, 
%  computed by the Fourier method. Top: without zero padding, bottom:
%  with zero padding. The left plots correspond to continuation to a
%  larger velocity ($+1$ km/sec); the right plots, smaller velocity,
%  ($-1$ km/sec).}
\par
The top plots in Figure \ref{fig:fft-imp} show the numerical impulse
responses of velocity continuation, computed by the Fourier method.
The initial data contained three spikes, passed through a narrow-band
filter. Theoretically, continuation to larger velocity (the left plot)
should create three elliptical wavefronts, and continuation to smaller
velocity (right plot) should create three hyperbolic wavefronts
\cite[]{GEO50-01-01100126}. We can see that the results are largely
contaminated with wraparound artifacts. The result of applying zero
padding (the bottom plots in Figure \ref{fig:fft-imp}) shows most of
the artifacts suppressed.
\par
Chebyshev spectral method, discussed in the next section, provides a
spectral accuracy while dealing correctly with non-periodic data.
\par 
\section{Chebyshev Approach}
For an alternative spectral approach, I adopted the Chebyshev-$\tau$
method \cite[]{lanc,orszag}. The Chebyshev-$\tau$ velocity continuation
algorithm consists of the following steps:
\begin{enumerate}
\item Transform the regular grid in $t$ to Gauss-Lobato collocation
  points, required for the fast Chebyshev transform.  First, a new
  variable $\xi$ is introduced by the shift transform:
  \begin{equation}
    \label{t2x}
    \xi = 1 - \frac{2\,t^2}{T^2}
  \end{equation}
  so that the domain $0 \leq t \leq T$ is mapped into the domain $1
  \geq \xi \geq -1$. Second, the $\xi$ grid points are distributed
  regularly in the cosine projection: $\xi_j = \cos(\frac{\pi j}{N}), j
  = 0,1,2,\ldots,N$.
\item Transform the initial image $P_0 (x,t)$ into the Chebyshev space
  in $\xi$ and Fourier transform in $x$, using the FFT algorithm. The
  Chebyshev-Fourier representation of $P_0 (x,t)$ is
  \begin{equation}
    \label{cheb}
    P_0 (x,t) = \sum_{k=-N_x/2}^{N_x/2-1}\sum_{j=0}^{N_t} 
    \hat{P}_{kj} T_j (\xi) e^{i k x}\;, 
  \end{equation}
  where $T_j$ denotes the Chebyshev polynomial of degree $j$.
\item Apply equation (\ref{PDE}) to advance the image in velocity $v$.
  It is convenient to rewrite this equation in the form
  \begin{equation}
    \label{chebPDE}
  {{\partial P} \over {\partial v}} =
  \frac{v T^2}{4}\,\int d\xi\,
  {{\partial^2 P} \over {\partial x^2}}\;. 
  \end{equation}
  In the Chebyshev-$\tau$ domain, the double differentiation in $x$ is
  performed by multiplying the Fourier transform of $P$ by $-k^2$, and
  integration in $\xi$ is performed
  as a direct operations on the Chebyshev coefficients. In
  particular, if $\sum_{j=0}^{N} a_j T_j (\xi)$ is the Chebyshev
  representation of the function $f (\xi)$, then the coefficients
  $b_j$ of $\int f (\xi) d\xi$ are defined by the relation
  \begin{equation}
    \label{int}
    2 j\,b_j = c_{j-1} a_{j-1} - a_{j+1}
  \end{equation}
  where $c_0 = 2$, $c_j = 0$ for $j < 0$, and $c_j = 1$ for $j > 0$.
  The constant of integration (and, correspondingly, the coefficient
  $b_0$) can be found at each velocity step from the boundary
  conditions (\ref{BC}), which are transformed to the form
  \begin{equation}
    \label{chebBC}
    \left.P\right|_{\xi=-1} = \sum_{j=0}^{N_t} \hat{P}_{kj} (-1)^j = 0\;.
  \end{equation}
  
  For the velocity advancement I used an implicit Crank-Nicolson scheme,
  which is unconditionally stable independent of the velocity step size.
  By writing equation (\ref{chebPDE}) in the matrix form
  \begin{equation}
    \label{matrix}
    \frac{\partial \mathbf{P}}{\partial v} = \mathbf{A} \mathbf{P}\;,
  \end{equation}
  the Crank-Nicolson advancement is represented by the equation
    \begin{equation}
    \label{CN}
    \mathbf{P}_{v+dv} = \left(\mathbf{I} - \mathbf{A}\,\frac{dv}{2}\right)^{-1}
    \left(\mathbf{I} + \mathbf{A}\,\frac{dv}{2}\right) \mathbf{P}_v\;,
  \end{equation}
  where $\mathbf{I}$ is the identity matrix. The inverted matrix
  $\left(\mathbf{I} - \mathbf{A}\,\frac{dv}{2}\right)$ has a tridiagonal
  structure, except for the first row, implied by the boundary
  condition (\ref{chebBC}). A careful treatment of the boundary
  condition by the matrix-bordering method \cite[]{faddeev,boyd} allows
  for an efficient inversion at a tridiagonal solver speed.
\item Transform the result of the velocity advancement back to the
  physical domain.
\item Transform the grid back to being regularly space in $t$.
\end{enumerate}
\par
%\plot{cheb1}{width=6in}{Synthetic seismic data before (left) and 
%  after (right) transformation to the Chebyshev grid in squared time.}
\par
%\plot{cheb1-inv}{width=6in}{The left plots show the reconstruction of the original data 
%  after transforming back from the Chebyshev grid to the original $t$
%  grid.  The right plots show the difference with the original model.
%  Top: using the original grid size ($N_t = 200$). Bottom: increasing
%  the grid size by a factor of three.}
\par
The first advantage of the Chebyshev approach comes from the better
conditioning of the grid transform.  Figure \ref{fig:cheb1} shows the
synthetic data before and after the grid transform.  Figure
\ref{fig:cheb1-inv} shows a reconstruction of the original data after
transforming back from the Chebyshev grid (Gauss-Lobato collocation
points). The difference with the original image is negligibly small.
\par
%\plot{cheb1-fft}{width=6in}{Left: Synthetic data after Chebyshev
%  transform.  Right: the real part of the Fourier transform in the
%  space coordinate.}
\par
The second advantage is the compactness of the Chebyshev
representation.  Figure \ref{fig:cheb1-fft} shows the data after the
decomposition into Chebyshev polynomials in $\xi$ and Fourier
transform in $x$. We observe a very rapid convergence of the Chebyshev
representation: a relatively small number of polynomials suffices to
represent the data.
\par
%\plot{cheb-impl}{width=6in}{Impulse responses (Green's functions) of
%  velocity continuation, computed by the Chebyshev-$\tau$ method. Top:
%  without zero padding, bottom: with zero padding on the $x$ axis. The
%  left plots correspond to continuation to a larger velocity ($+1$
%  km/sec); the right plots, smaller velocity, ($-1$ km/sec).}
\par
The third advantage is the proper handling of the non-periodic boundary
conditions. Figure \ref{fig:cheb-impl} shows the velocity continuation
impulse responses, computed by the Chebyshev method. As expected, no
wraparound artifacts occur on the time axis, and the accuracy of the
result is noticeably higher than in the case of finite differences
(Figure \ref{fig:fd-imp}).
\par 
\section{Conclusions}
\par
I have applied two spectral methods for a numerical solution of the
velocity continuation problem.
\par
The Fourier method is attractive because of its numerical efficiency.
However, it requires additional computational effort to suppress
numerical artifacts: the inaccuracy of the grid transform and
the artificial periodicity in the physical space.
\par
The Chebyshev-$\tau$ method is free of most of these difficulties,
although its overall efficiency can be slightly inferior to that of
the Fourier method.
\par
Both methods possess a ``spectral'' accuracy, which is highly desired
if accuracy is a concern.
\par
%\plot{mig-impl}{width=6in}{Top left: synthetic model (the ideal image). 
%  Top right: synthetic data. 
%  Bottom left: the result of velocity continuation 
%  with the Fourier method. Bottom right: the result of 
%  velocity continuation with the Chebyshev method.}
\par
Figure \ref{fig:mig-impl} compares the results of velocity
continuation with different methods. The top left plot shows an
implied subsurface model (an ``ideal image''). The top right plot is
the corresponding synthetic data. The bottom left plot is the output
of the Fourier method, and the bottom right plot is the output of the
Chebyshev method. The Fourier result shows a poor quality in the
shallow part (caused by subsampling in the $t^2$ grid). The wraparound
artifacts were suppressed by a zero-padding correction. The quality of
the Chebyshev result is noticeably higher. It is close to the best
possible accuracy, under the natural limitations of seismic resolution.

\bibliographystyle{seglike} \bibliography{SEG,SEP2,spec}

%%% Local Variables: 
%%% mode: latex
%%% TeX-master: t
%%% End: 
