\lefthead{Rickett \& Fomel}
\righthead{Second-order fast marching}
\footer{SEP--100}
%\keywords{traveltimes, eikonal}

%\shortnote
\published{SEP Report, 100, 287-292 (1999)}
\title{A second-order fast marching eikonal solver}

\email{james@sep.stanford.edu, sergey@sep.stanford.edu}
\author{James Rickett and Sergey Fomel}

\shortpaper
\maketitle

\section{Introduction}
The fast marching method~\cite[]{sethian} is widely used for solving the
eikonal equation in Cartesian coordinates.  
The method's principal advantages are: stability,
computational efficiency, and algorithmic simplicity. 
Within geophysics, fast marching traveltime
calculations~\cite[]{SEG-1997-1778} may be used 
for 3-D depth migration or velocity analysis.

\par
Unfortunately, first-order implementations lead to inaccuracies in
computed traveltimes, which may lead to poor image focusing for
migration applications. In addition, first-order traveltimes are not
accurate enough for reliable amplitude calculations.
This has lead to the development of the fast marching method
on non-Cartesian~\cite[]{Alkhalifah.sep.95.tariq5,Sun.sep.97.yalei2},
and even unstructured~\cite[]{Fomel.sep.95.sergey3} grids. 
These non-Cartesian formulations reduce inaccuracies, while
retaining the fast marching method's characteristic stability and
efficiency.  Unfortunately, the cost is the loss of algorithmic
simplicity. 

\par
We implement a second-order fast marching eikonal solver, which
reduces inaccuracies while retaining stability, efficiency {\em and}
simplicity. 

\section{Fast marching and the eikonal equation}
Under a high frequency approximation, propagating wavefronts may be 
described by the eikonal equation,
\begin{equation} \label{eqn:eikonal}
\left( \frac{\partial t}{\partial x} \right)^2 +
\left( \frac{\partial t}{\partial y} \right)^2 +
\left( \frac{\partial t}{\partial z} \right)^2 =s^2(x,y,z),
\end{equation}
where $t$ is the traveltime, $s$ is the slowness, and $x$, $y$ and $z$
represent the spatial Cartesian coordinates.

\par
The fast marching method solves equation~(\ref{eqn:eikonal}) by
directly mimicking the advancing wavefront.  
Every point on the computational 
grid is classified into three groups: 
points behind the wavefront, whose traveltimes are known and fixed; 
points on the wavefront, whose traveltimes have been calculated, but
are not yet fixed; and points ahead of the wavefront.  
The algorithm then proceeds as follows: 
\begin{enumerate}
\item Choose the point on the wavefront with the smallest traveltime. 
\item Fix this traveltime.
\item Advance the wavefront, so that this point is behind it, and
adjacent points are either on the wavefront or behind it. 
\item Update traveltimes for adjacent points on the wavefront by
solving equation~(\ref{eqn:eikonal}) numerically.
\item Repeat until every point is behind the wavefront. 
\end{enumerate}


\par
The update procedure (step 4.) requires the solution of the following
quadratic equation for $t$, 
\begin{eqnarray} \label{eqn:numeikonal}
\max(D_{ijk}^{-x} t, 0)^2+
\min(D_{ijk}^{+x} t, 0)^2 & + & \nonumber \\
\max(D_{ijk}^{-y} t, 0)^2+
\min(D_{ijk}^{+y} t, 0)^2 & + & \nonumber \\
\max(D_{ijk}^{-z} t, 0)^2+
\min(D_{ijk}^{+z} t, 0)^2 & = & s_{ijk}
\end{eqnarray}
where $D_{ijk}^{-x}$ is a backward $x$ difference operator at
grid point, $ijk$, $D_{ijk}^{+x}$ is a forward $x$ operator, and
finite-difference operators in $y$ and $z$ are defined similarly. 
The roots of the quadratic equation, $a t^2 + b t + c = 0$,
can be calculated explicitly as
\begin{equation} \label{eqn:quadratic}
t = \frac{ -b \pm \sqrt{b^2 -4ac}}{2a}.
\end{equation}
Solving equation~(\ref{eqn:numeikonal}) amounts to accumulating
coefficients $a$, $b$ and $c$ from its non-zero terms, and evaluating
$t$ with equation~(\ref{eqn:quadratic}).

\par
If we choose a two-point finite-difference operator, such as
\begin{eqnarray}
& D_{ijk}^{-x} t & = \;
\frac{t_{ijk}-t_{(i-1)jk}}{\Delta x} \\ 
{\rm then} \;  &
(D_{ijk}^{-x} t)^2 & = \;
\alpha t_{ijk}^2 + \beta t_{ijk} + \gamma
\end{eqnarray} 
where 
$\alpha = \frac{1}{\Delta x^2}$, 
$\beta  = -2 t_{(i-1)jk} \alpha$ and  
$\gamma = t_{(i-1)jk}^2 \alpha$.
%%$\alpha = \frac{1}{\Delta x^2}$, 
%%$\beta  = \frac{-2 t_{(i-1)jk}}{\Delta x^2}$ and 
%%$\gamma = \frac{t_{(i-1)jk}^2 }{\Delta x^2}$.
Coefficients $a$, $b$ and $c$ can now be calculated from 
$a = \Sigma_l \alpha_l$,
$b = \Sigma_l \beta_l$, and 
$c = \Sigma_l \gamma_l - s^2$,
where the summation index, $l$, refers to the six terms in
equation~(\ref{eqn:numeikonal}) subject to the various min/max 
conditions. 

\par
This two-point stencil, however, is only accurate to first-order.  
If instead we choose a suitable three-point finite-difference
stencil, we may expect the method to have second-order accuracy. For
example, the second-order upwind stencil,
\begin{eqnarray}
& D_{ijk}^{-x} t & = \; \frac{3t_{ijk}-4t_{(i-1)jk} +
t_{(i-2)jk}}{2 \Delta x} \\ {\rm gives} \;
& (D_{ijk}^{-x} t)^2 & = \;
\alpha' t_{ijk}^2 + \beta' t_{ijk} + \gamma'
\end{eqnarray}
\begin{eqnarray*}
{\rm where \; this \; time \;} &
\alpha' & = \; \frac{9}{4 \Delta x^2}, \\ [.1in] & 
\beta'  & = \; \frac{-3 (4 t_{(i-1)jk}-t_{(i-2)jk})}{2 \Delta x^2}
= -2 \alpha' t'_{(i-1)jk}, \\ &
\gamma' & = \; \frac{(4 t_{(i-1)jk}-t_{(i-2)jk})^2}{4 \Delta x^2} 
= \alpha' {t'}_{(i-1)jk}^2,
\end{eqnarray*}
\begin{eqnarray*}
{\rm and} \; \; &
{t'}_{(i-1)jk} & = \; \frac{1}{3}(4 t_{(i-1)jk}-t_{(i-2)jk}).
\end{eqnarray*}
Coefficients, $a$, $b$ and $c$ can be accumulated from $\alpha'$,
$\beta'$ and $\gamma'$ as before, and if the traveltime, $t_{(i-2)jk}$ 
is not available, first-order values may be substituted.

%%Similarly, the third-order stencil,
%%\begin{equation}
%%D_{ijk}^{-x} t = \; \frac{11t_{ijk}-18t_{(i-1)jk} +
%%9 t_{(i-2)jk} - 2 t_{(i-3)jk}}{6 \Delta x}
%%\end{equation}
%%\begin{eqnarray*}
%%{\rm produces \;} &
%%\alpha'' & = \; \left( \frac{11}{6 \Delta x}\right)^2, \\ [.1in] & 
%%\beta''  & = -2 \alpha'' t''_{(i-1)jk}, \\ {\rm and \;} &
%%\gamma'' & = \alpha'' {t''}_{(i-1)jk}^2,
%%\end{eqnarray*}
%%\begin{eqnarray*}
%%{\rm where} \; \; &
%%{t''}_{(i-1)jk} & = \; \frac{1}{11}(18 t_{(i-1)jk}-9t_{(i-2)jk}
%%+2 t_{(i-3)jk}).
%%\end{eqnarray*}

\section{Accuracy}

\inputdir{cvel}

Figure~\ref{fig:circles} shows traveltime
contour maps computed with the first and second-order fast marching
methods on a sparse ($20 \times 20$) grid. 
The large errors for waves propagating at 45$^\circ$ to the grid are
visibly reduced by the second-order formulation.
\plot{circles}{width=6in}{Traveltime contours in a constant
velocity medium. The solid line shows the exact result. The dashed
line shows the first-order (left panel) and second-order (right panel)
fast marching results, calculated on a $20 \times 20$ grid.}

\par
Figure~\ref{fig:error} shows the average error as a function of grid
spacing for the first and second-order solvers.  Not only is the
second-order formulation more accurate at large grid spacing, but its
accuracy increases more rapidly as grid spacing decreases. 
Theory predicts the $\log-\log$ plots of average error against
grid spacing to be a linear function with gradient of one for
first-order methods, and two for second order methods. 
In practice, the fast marching results come very close to these
criteria up to the limits of machine precision.
Figure~\ref{fig:error} demonstrates the superiority of the  
second-order fast marching formulation.

\par
It is worth noting, at this point, that special treatment is required 
at the source location, since the singularity in wavefront curvature
will cause numerical errors to propagate into the traveltime
solution.  We surround the source with a constant
velocity box, within which we calculate traveltimes by ray-tracing.
Errors are inversely proportional to the radius of this
box. Therefore, if the radius of the box decrease with grid spacing, 
errors will increase linearly, 
reducing the accuracy of the method to first-order. 
For full second-order accuracy, the box size should be independent of
grid spacing.
\plot{error}{width=6in}{Average error against grid spacing for a
constant velocity model.  The solid line corresponds to the
first-order eikonal solver, and the dashed line corresponds to the
second-order solver.  The left panel has linear axes, whereas the
right panel is a $\log-\log$ plot.}

\inputdir{marm}

\plot{marmousi}{width=6in}{Traveltime contours calculated through the
Marmousi velocity model sampled at 4~m.  
Solid line shows first-order results, and
dashed line shows second-order results.}

\section{Computational cost}

\inputdir{.}

The leading term in the computational cost of the fast marching
algorithm comes from the first step: 
choosing the point on the wavefront with the smallest traveltime. 
Consequently, the cost should not depend strongly on the order of the 
finite-difference stencil, but rather the sort algorithm used. 
Heap sorting has a cost of $O(\log N)$, and so in principle, with this 
algorithm, the fast marching method has a cost of $O( N \log N)$.

\par
The left panel of Figure~\ref{fig:times} shows a plot of CPU time
against $N$ for the same models as Figure~\ref{fig:error}. 
The time shown is elapsed (wall clock) time on a 300~MHz Pentium~II.  
For the largest model computed here, the second-order code takes 11\%
longer to run than the first-order code, and this percentage decreases
as $N$ increases. 

\par
Because $\log N$ grows slowly compared to $N$, the plot of CPU time
against $N$ is dominated by the linear term.  The right panel in
Figure~\ref{fig:times} addresses this issue by showing CPU time
divided by $N$ versus $N$.  On this graph, the $\log N$ behaviour is
clearly visible.
\plot{times}{width=6in}{Elapsed CPU time vs. the number of grid
points, $N$, for first-order (solid line) and second order (dashed
line) eikonal solvers. Left panel shows CPU time vs $N$. Right
panel shows CPU time$/N$ vs $N$.}

\section{Conclusions}
We have shown that a second-order implementation of the fast marching
eikonal solver produces traveltimes with a much higher accuracy than
the first-order implementation.  What is more, the additional accuracy
is acheived at only a marginal increase in cost.

\par
This second-order implementation should become the standard method for
computing first-arrival traveltimes within SEP.

\bibliographystyle{seg}  
\bibliography{SEP2,SEG,fastmarch}



