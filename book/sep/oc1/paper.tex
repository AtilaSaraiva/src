\title{Theory of differential offset continuation}
\author{Sergey Fomel}

\maketitle

\begin{abstract}
 I introduce a partial differential equation to describe the
  process of prestack reflection data transformation in the offset,
  midpoint, and time coordinates. The equation is proved theoretically
  to provide correct kinematics and amplitudes on the transformed
  constant-offset sections. Solving an initial-value problem with the
  proposed equation leads to integral and frequency-domain offset
  continuation operators, which reduce to the known forms of dip
  moveout operators in the case of continuation to zero offset.  
\end{abstract}

\section{Introduction}
  
The Earth subsurface is three-dimensional, while seismic reflection
data from a multi-coverage acquisition belong to a five-dimensional
space (time, 2-D offset, and 2-D midpoint coordinates). This fact
alone indicates the additional connection that exists
in the data space. I show in this paper that it is possible, under
certain assumptions, to express this connection in a concise
mathematical form of a partial differential equation. The theoretical
analysis of this equation allows us to explain and predict the data
transformation between different offsets.
  
The partial differential equation, introduced in this
paper\footnote{To my knowledge, the first derivation of the revised
offset continuation equation was accomplished by Joseph Higginbotham
of Texaco in 1989.  Unfortunately, Higginbotham's derivation never
appeared in the open literature.}, describes the process of
\emph{offset continuation}, which is a transformation of common-offset
seismic gathers from one constant offset to another
\cite[]{GPR30-06-08130828}. \cite{GEO61-06-18461858} identified offset
continuation (OC) with a whole family of prestack continuation
operators, such as shot continuation \cite[]{SEG-1993-0673}, dip
moveout as a continuation to zero offset \cite[]{DMObook}, and
three-dimensional azimuth moveout \cite[]{GEO63-02-05740588}. An
intuitive introduction to the concept of offset continuation is
presented by \cite{TLE20-01-02100213}. A general data mapping
prospective is developed by \cite{GPR48-01-01350162}.
   
As early as in 1982, Bolondi et al.  came up with the idea of
describing offset continuation and dip moveout (DMO) as a continuous
process by means of a partial differential equation
\cite[]{GPR30-06-08130828}.  However, their approximate differential
operator, built on the results of Deregowski and Rocca's classic paper
\cite[]{GPR29-03-03740406}, failed in the cases of steep reflector dips
or large offsets.  \cite{Hale.sepphd.36} writes:
\begin{quote}
  The differences between this algorithm [DMO by Fourier transform]
  and previously published finite-difference DMO algorithms are
  analogous to the differen\-ces be\-t\-ween
  fre\-qu\-en\-cy-wave\-num\-ber
  \cite[]{GEO43-01-00230048,GEO43-07-13421351} and
  fi\-nite-dif\-fe\-rence \cite[]{Claerbout.fgdp.76} algorithms for
  migration. For example, just as finite-difference migration
  algorithms require approximations that break down at steep dips,
  finite-difference DMO algorithms are inaccurate for large offsets
  and steep dips, even for constant velocity.
\end{quote}
Continuing this analogy, we can observe that both finite-difference
and frequency-domain migration algorithms share a common origin: the
wave equation. The new OC equation, presented in this paper and valid
for all offsets and dips, plays a role analogous to that of the wave
equation for offset continuation and dip moveout algorithms. A
multitude of seismic migration algorithms emerged from the fundamental
wave-propagation theory that is embedded in the wave equation.
Likewise, the fundamentals of DMO algorithms can be traced to the OC
differential equation.

In the first part of the paper, I prove that the revised equation
is, under certain assumptions, kinematically valid. This means that
wavefronts of the offset continuation process correspond to the
reflection wave traveltimes and correctly transform between different
offsets.  Moreover, the wave amplitudes are also propagated correctly
according to the \emph{true-amplitude} criterion \cite[]{GEO58-01-00470066}. 

In the second part of the paper, I relate the offset continuation
equation to different methods of dip moveout. Considering DMO as a
continuation to zero offset, I show that DMO operators can be obtained
by solving a special initial value problem for the OC
equation.  Different known forms of DMO \cite[]{DMObook} appear as special
cases of more general offset continuation operators.

The companion paper \cite[]{GEO68-02-07330744} demonstrates a
practical application of differential offset continuation to seismic
data interpolation.

\section{Introducing the offset continuation equation}
%%%%%%%%%%%%%%%%%%%%%%%%%%%%
 Most of the contents of this paper refer to the following linear
partial differential equation:
\begin{equation}
h \, \left( {\partial^2 P \over \partial y^2} - {\partial^2 P \over \partial
h^2} \right) \, = \, t_n \, {\partial^2 P \over {\partial t_n \,
\partial h}} \,\,\, . 
\label{eqn:OCequation} 
\end{equation}
Equation~(\ref{eqn:OCequation}) describes an {\em artificial}
(non-physical) process of transforming reflection seismic data
$P(y,h,t_n)$ in the offset-midpoint-time domain. In
equation~(\ref{eqn:OCequation}), $h$ stands for the half-offset
($h=(r-s)/2$, where $s$ and $r$ are the source and the receiver
surface coordinates), $y$ is the midpoint ($y=(r+s)/2$), and $t_n$ is the time
coordinate after normal moveout correction is applied:
\begin{equation}
\label{eqn:tnmo}
t_n=\sqrt{t^2-{4 \, h^2 \over v^2}}\;.
\end{equation}
The velocity $v$ is assumed to be known a priori.
Equation~(\ref{eqn:OCequation}) belongs to the class of linear
hyperbolic equations, with the offset $h$ acting as a time-like
variable. It describes a wave-like propagation in the offset
direction.

\subsection{Proof of validity}

A simplified version of the ray method technique \cite[]{cerveny,babich}
can allow us to prove the theoretical validity of
equation~(\ref{eqn:OCequation}) for all offsets and reflector dips by
deriving two equations that describe separately wavefront (traveltime)
and amplitude transformation.  According to the formal ray theory,
the leading term of the high-frequency asymptotics for a reflected
wave recorded on a seismogram takes the form
\begin{equation}
   P\left(y,h,t_n\right) \approx
A_n(y,h)\,R_n\left(t_n-\tau_n(y,h)\right) \;,
\label{eqn:raymethod} 
\end{equation}  
where $A_n$ stands for the amplitude, $R_n$ is the wavelet shape of
the leading high-frequency term, and $\tau_n$ is the traveltime curve
after normal moveout. Inserting~(\ref{eqn:raymethod}) as a trial
solution for~(\ref{eqn:OCequation}), collecting terms that have the
same asymptotic order (correspond to the same-order derivatives of the
wavelet $R_n$), and neglecting low-order terms, we arrive at the set of
two first-order partial differential equations:
\begin{equation}
h \, \left[     {\left( \partial \tau_n \over \partial y \right)}^2 - 
                {\left( \partial \tau_n \over \partial h \right)}^2
     \right] = \, - \, \tau_n \, {\partial \tau_n \over \partial h} \,\,\,,  
\label{eqn:eikonal} 
\end{equation}   
\begin{equation}
\left( \tau_n - 2h \, {\partial \tau_n \over {\partial h}} \right)
\, {\partial A_n \over \partial h} + 2h {\partial \tau_n \over \partial
y}   {\partial A_n \over \partial y} + h A_n \left( {\partial^2 \tau_n
\over {\partial y^2}} - {\partial^2 \tau_n \over {\partial h^2}} 
\right) \, = \, 0 \,\,\,.
\label{eqn:transport} 
\end{equation}

Equation~(\ref{eqn:eikonal}) describes the transformation of
traveltime curve geometry in the OC process analogously to how the
eikonal equation describes the front propagation in the classic wave
theory.  What appear to be wavefronts of the wave motion described by
equation~(\ref{eqn:OCequation}) are traveltime curves of reflected
waves recorded on seismic sections.  The law of amplitude
transformation for high-frequency wave components related to those
wavefronts is given by equation~(\ref{eqn:transport}).  In terms of
the theory of partial differential equations,
equation~(\ref{eqn:eikonal}) is the characteristic equation
for~(\ref{eqn:OCequation}).

\subsection{Proof of kinematic equivalence}
%%%%%%%%%%%%%%%%%%%%%%%%%%%%%%%%%%%%%%%%%%%%%%%%

In order to prove the validity of equation~(\ref{eqn:eikonal}), it is
convenient to transform it to the coordinates of the initial shot
gathers: $s=y-h$, $r=y+h$, and $\tau = \sqrt{\tau_n^2+{{4h^2} \over
    {v^2}}}$. The transformed equation takes the form
\begin{equation}
\left( \tau^2 + {{(r-s)^2} \over {v^2}} \right) \left( {\partial \tau
\over \partial r} -   {\partial \tau \over \partial s} \right) = 2 \, (r-s) \,
\tau \left( {1 \over {v^2}} - {\partial \tau \over \partial r}
{\partial \tau \over \partial s} \right) \,\,\,.
\label{eqn:SCeikonal} 
\end{equation}
Now the goal is to prove that any reflection traveltime function
$\tau(r,s)$ in a constant velocity medium satisfies 
equation~(\ref{eqn:SCeikonal}). 

Let $S$ and $R$ be the source and the receiver locations, and $O$ be a
reflection point for that pair.  Note that the incident ray $SO$ and
the reflected ray $OR$ form a triangle with the basis on the offset
$SR$ ($l=|SR|=|r-s|$).  Let $\alpha_1$ be the angle of $SO$ from the
vertical axis, and $\alpha_2$ be the analogous angle of $RO$ (Figure
\ref{fig:ocoray}). The law of sines gives us the following explicit
relationships between the sides and the angles of the triangle $SOR$:
\begin{eqnarray}
|SO|\,=\,|SR|\, {\cos{\alpha_2} \over
\sin{\left(\alpha_2-\alpha_1\right)}} \,\,\,, 
\label{eqn:triangle1} \\ 
|RO|\,=\,|SR|\, {\cos{\alpha_1} \over
\sin{\left(\alpha_2-\alpha_1\right)}} \,\,\,.
\label{eqn:triangle2} 
\end{eqnarray} 
Hence, the total length of the reflected ray satisfies
\begin{equation}
v \tau = |SO|+|RO|=|SR|\,  {{\cos{\alpha_1}+ \cos{\alpha_2}} \over
\sin{\left(\alpha_2-\alpha_1\right)}} = |r-s|\,{\cos{\alpha} \over
\sin{\gamma}} \,\,\,.
\label{eqn:length} 
\end{equation}
Here $\gamma$ is the reflection angle ($\gamma = (\alpha_2 -
\alpha_1)/2$), and $\alpha$ is the central ray angle ($\alpha =
(\alpha_2 + \alpha_1)/2$), which coincides with the local dip angle of
the reflector at the reflection point.  Recalling the well-known
relationships between the ray angles and the first-order traveltime
derivatives
\begin{eqnarray}
{{\partial \tau} \over {\partial s}} \,=\,{ {\sin{\alpha_1}} \over
{v}} \,\,\,,
\label{eqn:snell1}\\
{{\partial \tau} \over {\partial r}} \,=\, {{\sin{\alpha_2}} \over {v}}  
\label{eqn:snell2}\,\,\,,
\end{eqnarray}  
we can substitute~(\ref{eqn:length}),~(\ref{eqn:snell1}), and~(\ref{eqn:snell2}) into 
(\ref{eqn:SCeikonal}), which  leads to the simple trigonometric equality
\begin{equation}
\cos^2{\left( {\alpha_1 + \alpha_2} \over 2 \right)} +  
\sin^2{\left( {\alpha_1 - \alpha_2} \over 2 \right)}\, = \, 1 -
\sin{\alpha_1} \sin{\alpha_2} \,\,\,. 
\label{eqn:equality} 
\end{equation}
It is now easy to show that equality~(\ref{eqn:equality}) is true for any
$\alpha_1$ and $\alpha_2$, since
\[
\sin^2{a} - \sin^2{b} = \sin{(a+b)}\,\sin{(a-b)}\;.
\]

\inputdir{XFig}

\sideplot{ocoray}{height=2.5in}{Reflection rays in a constant
velocity medium (a scheme).}

Thus we have proved that equation
(\ref{eqn:SCeikonal}), equivalent to~(\ref{eqn:eikonal}), is valid in constant
velocity media independently of the reflector geometry and the offset.
This means that high-frequency asymptotic components of the waves,
described by the OC equation,
are located on the true reflection traveltime curves.  

The theory of characteristics can provide other ways to prove the
kinematic validity of equation~(\ref{eqn:eikonal}), as described by
\cite{me} and \cite{plag}.

%%%%%%%%%%%%%%%%%%%%%%%%%%%%%%%%%%%%%%%%%%%%%%%%%%%
\subsection{Comparison with Bolondi's OC equation}
Equation~(\ref{eqn:OCequation}) and the previously published OC
equation \cite[]{GPR30-06-08130828} differ only with respect to the
single term $\partial^2 P \over {\partial h^2}$. However, this
difference is substantial.  

From the offset continuation characteristic equation
(\ref{eqn:eikonal}), we can conclude that the first-order traveltime
derivative with respect to offset decreases with decreasing
offset. The derivative equals zero at the zero offset, as predicted by the
principle of reciprocity (the reflection traveltime has to be an {\em
  even} function of offset). Neglecting $\left({\partial \tau_n} \over
  {\partial h}\right)^2$ in (\ref{eqn:eikonal}) leads to the
characteristic equation
\begin{equation}
 h \,   {\left( \partial \tau_n \over \partial y \right)}^2  
 = \, - \, \tau_n \, {\partial \tau_n \over \partial h}\;, 
\label{eqn:ITeikonal} 
\end{equation}
which corresponds to the approximate OC equation of
\cite{GPR30-06-08130828}. The approximate equation has the form
\begin{equation}
h \, {\partial^2 P \over \partial y^2} \, = \, t_n \, {\partial^2 P
\over {\partial t_n \, \partial h}}\;.
\label{eqn:bolondi} 
\end{equation}
Comparing equations~(\ref{eqn:ITeikonal}) and (\ref{eqn:eikonal}), we
can note that approximation (\ref{eqn:ITeikonal}) is valid only if
\begin{equation}
{\left( \partial \tau_n \over \partial h \right)}^2 \, \ll\, {\left(
\partial \tau_n \over \partial y \right)}^2 \,\,\,. 
\label{eqn:condition} 
\end{equation}
To find the geometric constraints implied by inequality
(\ref{eqn:condition}), we can express the traveltime derivatives in
geometric terms. As follows from expressions (\ref{eqn:snell1}) and
(\ref{eqn:snell2}),
\begin{eqnarray}
\label{eqn:snells1}
{{\partial \tau} \over {\partial y}} & = & {{\partial \tau} \over
{\partial r}} + {{\partial \tau} \over {\partial s}} \,=\, { {2
\sin{\alpha} \cos{\gamma}} \over {v}}\;, \\
{{\partial \tau} \over
{\partial h}} & = & {{\partial \tau} \over {\partial r}} - {{\partial
\tau} \over {\partial s}} \,=\, { {2 \cos{\alpha} \sin{\gamma}} \over
{v}}\;.
\label{eqn:snells2}
\end{eqnarray}
Expression (\ref{eqn:length}) allows transforming
equations~(\ref{eqn:snells1}) and (\ref{eqn:snells2}) to the form
\begin{eqnarray}
\label{eqn:connection1}
\tau_n \, {{\partial \tau_n} \over {\partial y}} & = & \tau \, {{\partial
\tau} \over {\partial y}} \,=\, 4h\,{{\sin{\alpha} \cos{\alpha}
\cot{\gamma}} \over {v^2}}\;; \\ 
\tau_n \, {{\partial \tau_n} \over {\partial h}} & = &\tau \, {{\partial
\tau} \over {\partial h}} - {{4h} \over {v^2}} \,=\,-\,
4h\,{{\sin^2{\alpha}} \over {v^2}}\;.   
\label{eqn:connection2} 
\end{eqnarray}
Without loss of generality, we can assume $\alpha$ to be positive.
Consider a plane tangent to a true reflector at the reflection
point
(Figure \ref{fig:ocobol}).
The traveltime of a wave, reflected from the plane, has the
known explicit expression
\begin{equation}
\tau\,=\,{2 \over v}\,\sqrt{L^2+h^2\,\cos^2{\alpha}}\,\,\,,   
\label{eqn:CDP} 
\end{equation}
where $L$ is the length of the normal ray from the midpoint. As
follows from combining (\ref{eqn:CDP}) and (\ref{eqn:length}),
\begin{equation}
{\cos{\alpha} \cot{\gamma}} \,=\, {L \over h}   \,\,\,.
\label{eqn:ratio} 
\end{equation}
We can now combine equations~(\ref{eqn:ratio}),
(\ref{eqn:connection1}), and (\ref{eqn:connection2}) to transform
inequality (\ref{eqn:condition}) to the form
\begin{equation}
 h \ll {L \over {\sin{\alpha}}} \,=\, z\, \cot{\alpha}\,\,,   
\label{eqn:hm} 
\end{equation}
where $z$ is the depth of the plane reflector under the midpoint.  For
example, for a dip of 45 degrees, equation~(\ref{eqn:bolondi}) will be
satisfied only for offsets that are much smaller than the depth of the
reflector.

\sideplot{ocobol}{height=2.5in}{Reflection rays and
tangent to the reflector in a constant velocity medium (a scheme).}
  
\subsection{Offset continuation geometry: time rays}
%%%%%%%%%%%%%%%%%%%%%%%%%%%%%%%%%%%%

To study the laws of traveltime curve transformation in the OC
process, it is convenient to apply the method of characteristics
\cite[]{kurant} to the eikonal-type equation~(\ref{eqn:eikonal}). The
characteristics of equation~(\ref{eqn:eikonal}) [{\em
  bi}-characteristics with respect to equation (\ref{eqn:OCequation})]
are the trajectories of the high-frequency energy propagation in the
imaginary OC process. Following the formal analogy with seismic rays,
I call those trajectories {\em time rays}, where the word {\em time}
refers to the fact that the trajectories describe the traveltime
transformation \cite[]{me}.  According to the theory of first-order
partial differential equations, time rays are determined by a set of
ordinary differential equations (characteristic equations) derived
from equation (\ref{eqn:eikonal}) :
\begin{eqnarray}
{{{dy} \over {dt_n}}   =  - {{2 h Y} \over {t_n H}}}\;,\; 
{{{dY} \over {dt_n}}  =  {Y \over t_n}}\;, 
\nonumber \\
{{{dh} \over {dt_n}}  =  {-{1 \over H}+{{2 h} \over t_n}}}\;,\;
{{{dH} \over {dt_n}} = {{Y^2} \over {t_n H}}}\;, 
\label{eqn:rays} 
\end{eqnarray}
where $Y$ corresponds to $\partial \tau_n \over \partial y$ along a
ray and $H$ corresponds to $\partial \tau_n \over \partial h$. In this
notation, equation~(\ref{eqn:eikonal}) takes the form
\begin{equation}
h\, (Y^2-H^2) = -\, t_n H 
\label{eqn:rayeikonal} 
\end{equation}
and serves as an additional constraint for the definition of time
rays.  System~(\ref{eqn:rays}) can be solved by standard mathematical
methods \cite[]{ode}. Its general solution takes the parametric form,
where the time variable $t_n$ is the parameter changing along a time
ray:
\begin{eqnarray}
y(t_n)  =  C_1-C_2\,t_n^2 \; & ; & \;h(t_n)=t_n \sqrt{C_2^2 t_n^2 + C_3}\;;
\\ 
Y(t_n)  =  {{C_2\,t_n}\over C_3}\; & ; & \;H(t_n)={h \over {C_3\,t_n}}
\label{eqn:ray} 
\end{eqnarray}
and $C_1$, $C_2$, and $C_3$ are independent coefficients, constant
along each time ray. To find the values of these coefficients, we can
pose an initial-value problem for the system of differential
equations~(\ref{eqn:rays}).  The traveltime curve $\tau_n(y;h)$ for a
given common offset $h$ and the first partial derivative $\partial
\tau_n \over \partial h$ along the same constant offset section
provide natural initial conditions. A particular case of those
conditions is the zero-offset traveltime curve. If the first partial
derivative of traveltime with respect to offset is continuous, it
vanishes at zero offset according to the reciprocity principle
(traveltime must be an even function of the offset):
\begin{math}
t_0\left(y_0\right)=\tau_n(y;0), 
\left. {\partial \tau_n \over \partial h} \right|_{h=0}=0\,.  
\end{math}
Applying the initial-value conditions to the general
solution~(\ref{eqn:ray}) generates the following expressions for the
ray invariants:
\begin{eqnarray}
C_1 & = & y+h\,{Y \over H}=y_0-{t_0\left(y_0\right) \over
t_0'\left(y_0\right)}\;;\;
C_2={{h\,Y} \over {\tau_n^2\,H}}=
-{1 \over t_0\left(y_0\right)\,t_0'\left(y_0\right)}\;;\;
\nonumber \\
C_3 & = & {h \over {\tau_n\,H}}=
-{1 \over \left(t_0'\left(y_0\right)\right)^2}\;,
\label{eqn:abc} 
\end{eqnarray}
where $t_0'\left(y_0\right)$ denotes the derivative
$\frac{d\,t_0}{d\,y_0}$.  Finally, substituting
equations~(\ref{eqn:abc})
into~(\ref{eqn:ray}), we obtain an explicit parametric form of the ray
trajectories:
\begin{eqnarray}
\label{eqn:yhray1}
y_1\left(t_1\right)  & = & \displaystyle{y+{{h\,Y} \over
{t_n^2\,H}}\,\left(t_n^2-t_1^2\right)= y_0+{{t_1^2-t_0^2\left(y_0\right)} \over
{t_0\left(y_0\right)\,t_0'\left(y_0\right)}}}\;;
\\
h_1^2\left(t_1\right) & = & \displaystyle{{{h\,t_1^2} \over {t_n^3\,H}}\,
\left(t_n^2+t_1^2\,{{h\,Y^2} \over {t_n\,H}}\right)=
t_1^2\,{{t_1^2-t_0^2\left(y_0\right)} \over
{\left(t_0\left(y_0\right)\,t_0'\left(y_0\right)\right)^2}}}\;.
\label{eqn:yhray2}
\end{eqnarray}
Here $y_1$, $h_1$, and $t_1$ are the coordinates of the continued
seismic section. Equations~(\ref{eqn:yhray1}) indicates
that the time ray projections to a common-offset section have a
parabolic form. Time rays do not exist for $t_0'\left(y_0\right)=0$ (a
locally horizontal reflector) because in this case post-NMO offset
continuation transform is not required.

The actual parameter that
determines a particular time ray is the reflection point location.
This important conclusion follows from the known parametric equations
\begin{eqnarray}
\label{eqn:gur1}
t_0(x) & = & \displaystyle{t_v \sec{\alpha}=
t_v(x)\,\sqrt{1+u^2\left(t_v'(x)\right)^2}}\;, 
\\
y_0(x) & = & \displaystyle{x+u t_v\tan{\alpha} =x+u^2\,t_v(x)t_v'(x)}\;,
\label{eqn:gur2}
\end{eqnarray}
where $x$ is the reflection point, $u$ is half of the wave velocity ($u=v/2$), 
$t_v$ is the vertical time (reflector depth divided by $u$), and
$\alpha$ is the 
local reflector dip. Taking into account that the derivative of the zero-offset
traveltime curve is
\begin{equation}
{{dt_0}\over{dy_0}}={{t_0'(x)}\over{y_0'(x)}}={{\sin{\alpha}}\over u}=
{{t_v'(x)} \over \sqrt{1+u^2\left(t_v'(x)\right)^2}}
\label{eqn:gurtx}
\end{equation}
and substituting equations~(\ref{eqn:gur1}) and~(\ref{eqn:gur2})
into~(\ref{eqn:yhray1}) and~(\ref{eqn:yhray2}), we get
\begin{eqnarray}
\label{eqn:rayrp1}
y_1\left(t_1\right) & = &
\displaystyle{x+{{t_1^2-t_v^2\left(x\right)} \over
{t_v\left(x\right)\,t_v'\left(x\right)}}}\;;
\\
u^2 t^2\left(t_1\right) & = & 
\displaystyle{t_1^2\,{{t_1^2-t_v^2\left(x\right)} \over
{\left(t_v\left(x\right)\,t_v'\left(x\right)\right)^2}}}\;,
\label{eqn:rayrp2}
\end{eqnarray}
where $t^2\left(t_1\right)=t_1^2+h_1^2\left(t_1\right)/u^2$.

To visualize the concept of time rays, let us consider some simple
analytic examples of its application to geometric analysis of the
offset-continuation process.

\subsubsection{Example 1: plane reflector}

\inputdir{Math}

The simplest and most important example is the case of a plane dipping
reflector. Putting the origin of the $y$ axis at the intersection of
the reflector plane with the surface, we can express the reflection
traveltime after NMO in the form
\begin{equation}
\tau_n(y,h)=p\,\sqrt{y^2-h^2}\;,
\label{eqn:planett}
\end{equation}
where $p=2\,{ \sin{\alpha} \over v}$, and $\alpha$ is the dip angle. 
The zero-offset traveltime in this case is a straight line:
\begin{equation}
t_0\left(y_0\right)=p\,y_0\;.
\label{eqn:planezott}
\end{equation}
According to equations~(\ref{eqn:yhray1}-\ref{eqn:yhray2}), the time
rays in this case are defined by
\begin{equation}
y_1\left(t_1\right)={t_1^2 \over {p^2\,y_0}}\;;\;
h_1^2\left(t_1\right)=t_1^2\,{{t_1^2-p^2\,y_0^2} \over
{p^4\,y_0^2}}\;;\;
y_0={{y^2-h^2} \over y}\;.
\label{eqn:planerays}
\end{equation}
The geometry of the OC transformation is shown in Figure
\ref{fig:ocopln}.

\plot{ocopln}{width=6in,height=3in}{Transformation of
  the reflection traveltime curves in the OC process: the case of a
  plane dipping reflector. Left: Time coordinate before the NMO
  correction. Right: Time coordinate after NMO. The solid lines
  indicate traveltime curves at different common-offset sections; the
  dashed lines indicate time rays.}

\subsubsection{Example 2: point diffractor}

  The second example is the case of a point diffractor (the left side
  of Figure \ref{fig:ococrv}).  Without loss of generality, the origin
  of the midpoint axis can be put above the diffraction point. In this
  case the zero-offset reflection traveltime curve has the well-known
  hyperbolic form
\begin{equation}
t_0\left(y_0\right)={\sqrt{z^2+y_0^2} \over u}\;,
\label{eqn:pointzott}
\end{equation}
where $z$ is the depth of the diffractor and $u=v/2$ is half of the
wave velocity. Time rays are defined according to
equations~(\ref{eqn:yhray1}-\ref{eqn:yhray2}), as follows:
\begin{equation}
y_1\left(t_1\right)={{u^2\,t_1^2-z^2} \over y_0}\;;\;
u^2\,t^2\left(t_1\right)=u^2\,t_1^2+h_1^2\left(t_1\right)=
u^2\,t_1^2\,{{u^2\,t_1^2-z^2} \over y_0^2}\;.
\label{eqn:pointrays}
\end{equation}


\plot{ococrv}{width=6in,height=3in}{Transformation of
  the reflection traveltime curves in the OC process. Left: the case
  of a diffraction point. Right: the case of an elliptic reflector.
  Solid lines indicate traveltime curves at different common-offset
  sections, dashed lines indicate time rays.}

\subsubsection{Example 3: elliptic reflector}

The third example (the right side of Figure \ref{fig:ococrv}) is the
curious case of a focusing elliptic reflector. Let $y$ be the center
of the ellipse and $h$ be half the distance between the foci of the
ellipse. If both foci are on the surface, the zero-offset
traveltime curve is defined by the so-called ``DMO smile''
\cite[]{GPR29-03-03740406}:
\begin{equation}
t_0\left(y_0\right)={t_n \over h}\,\sqrt{h^2-\left(y-y_0\right)^2}\;,
\label{eqn:smilezott}
\end{equation} 
where $t_n=2\,z/v$, and $z$ is the small semi-axis of the ellipse.
The time-ray equations are
\begin{equation}
y_1\left(t_1\right)=y+{h^2\over {y-y_0}}\,{{t_1^2-t_n^2} \over t_n^2}\;;\;
h_1^2\left(t_1\right)=h^2\,{t_1^2 \over t_n^2}\,
\left(1+
{h^2\over \left(y-y_0\right)^2}\,{{t_1^2-t_n^2} \over t_n^2}
\right)\;.
\label{eqn:smilerays}
\end{equation}
When $y_1$ coincides with $y$, and $h_1$ coincides with $h$, the
source and the receiver are in the foci of the elliptic reflector, and
the traveltime curve degenerates to a point $t_1=t_n$. This remarkable
fact is the actual basis of the geometric theory of dip moveout
\cite[]{GPR29-03-03740406}.

\subsection{Proof of amplitude equivalence} 
%%%%%%%%%%%%%%%%%%%%%%%%%%%%%%%%%%%%%%
Let us now consider the connection between the laws of traveltime
transformation and the laws of the corresponding amplitude
transformation.  The change of the wave amplitudes in the OC process
is described by the first-order partial differential transport
equation~(\ref{eqn:transport}). We can find the general solution of
this equation by applying the method of characteristics. The solution
takes the explicit integral form
\begin{equation}
A_n\left(t_n\right)=A_0\left(t_0\right)\,\exp{\left(\int_{t_o}^{t_n}
\left[h\,\left({\partial^2 \tau_n \over \partial y^2}-
{\partial^2 \tau_n \over \partial h^2}\right)\,
\left(\tau_n\,{\partial \tau_n \over \partial h} \right)^{-1}\right]\,
d\tau_n\right)}\;.
\label{eqn:ampint}
\end{equation}
The integral in equation~(\ref{eqn:ampint}) is defined on a curved
time ray, and $A_n(t_n)$ stands for the amplitude transported along
this ray. In the case of a plane dipping reflector, the ray amplitude
can be immediately evaluated by substituting the explicit traveltime
and time ray equations from the preceding section
into~(\ref{eqn:ampint}). The amplitude expression in this case takes
the simple form
\begin{equation}
A_n\left(t_n\right)=A_0\left(t_0\right)\,\exp{\left(-\int_{t_o}^{t_n}
\frac{d\tau_n}{\tau_n}\right)} = A_0\left(t_0\right)\,{t_0 \over t_n}\;.
\label{eqn:ampplane}
\end{equation}
In order to consider the more general case of a curvilinear reflector,
we need to take into account the connection between the traveltime
derivatives in~(\ref{eqn:ampint}) and the geometry of the reflector.
As follows directly from the trigonometry of the incident and
reflected rays triangle (Figure \ref{fig:ocoray}),
\begin{eqnarray}
h & = & {r-s \over 2}=
D\,{{\cos{\alpha}\,\sin{\gamma}\,\cos{\gamma}} \over
{\cos^2{\alpha}-\sin^2{\gamma}}}\;,
\label{eqn:geoh}\\
y & = & {r+s \over 2}=
x+D\,{{\cos^2{\alpha}\,\sin{\alpha}} \over
{\cos^2{\alpha}-\sin^2{\gamma}}}\;,
\label{eqn:geoy}\\
y_0 & = & x+D\,\sin{\alpha}\;,
\label{eqn:geoy0}
\end{eqnarray}
where $D$ is the length of the normal ray. Let $\tau_0=2\,D/v$ be the
zero-offset reflection traveltime. Combining
equations~(\ref{eqn:geoh}) and~(\ref{eqn:geoy0}) with
(\ref{eqn:length}), we can get the following relationship:
\begin{equation}
a={\tau_n\over\tau_0}={{\cos{\alpha}\,\cos{\gamma}}\over
\left(\cos^2{\alpha}-\sin^2{\gamma}\right)^{1/2}}=
\left(1+{{\sin^2{\alpha}\,\sin^2{\gamma}}\over
{\cos^2{\alpha}-\sin^2{\gamma}}}\right)^{1/2}=
{h\,\over\sqrt{h^2-\left(y-y_0\right)^2}}\;,
\label{eqn:A}
\end{equation}
which describes the ``DMO smile''~(\ref{eqn:smilezott}) found by
\cite{GPR29-03-03740406} in geometric terms.
Equation~(\ref{eqn:A}) allows for a convenient change of variables
in equation~(\ref{eqn:ampint}). Let the reflection angle $\gamma$ be a
parameter monotonically increasing along a time ray. In this case,
each time ray is uniquely determined by the position of the reflection
point, which in turn is defined by the values of $D$ and $\alpha$.
According to this change of variables, we can
differentiate~(\ref{eqn:A}) along a time ray to get
\begin{equation}
{{d\tau_n}\over\tau_n}=-{{\sin^2{\alpha}}\over
{2\,\cos^2{\gamma}\,\left(\cos^2{\gamma}-\sin^2{\alpha}\right)}}\,
d\left(\cos^2{\gamma}\right)\;.
\label{eqn:dt2tg}
\end{equation}
Note also that the quantity $h\,\left(\tau_n\,{\partial \tau_n \over
    \partial h} \right)^{-1}$ in equation~(\ref{eqn:ampint}) coincides
exactly with the time ray invariant $C_3$ found in
equation~(\ref{eqn:abc}). Therefore its value is constant along each
time ray and equals
\begin{equation}
h\,\left(\tau_n\,{\partial \tau_n \over \partial h}\right)^{-1}=
-{v^2 \over 4\, \sin^2{\alpha}}\;.
\label{eqn:c3}
\end{equation}
Finally, as shown in Appendix~\ref{chapter:deriv},
\begin{equation}
\tau_n\,\left({\partial^2 \tau_n \over \partial y^2}-
{\partial^2 \tau_n \over \partial h^2}\right)=4\,
{\cos^2{\gamma}\over v^2}\,\left({\sin^2{\alpha}+DK}\over
{\cos^2{\gamma}+DK}\right)\;,
\label{eqn:curve}
\end{equation}
where $K$ is the reflector curvature at the reflection point.
Substituting (\ref{eqn:dt2tg}),~(\ref{eqn:c3}), and~(\ref{eqn:curve})
into~(\ref{eqn:ampint}) transforms the integral to the form
\begin{eqnarray*}
 \int_{t_o}^{t_n}\left[
h\,\left({\partial^2 \tau_n \over \partial y^2}-
{\partial^2 \tau_n \over \partial h^2}\right)\,
\left(\tau_n\,{\partial \tau_n \over \partial h} \right)^{-1}\right]\,
d\tau_n = 
\end{eqnarray*}
\begin{equation}
 = -{1 \over 2}\,\int_{\cos^2{\gamma_0}}^{\cos^2{\gamma}}
\left({1 \over {\cos^2{\gamma'}-\sin^2{\alpha}}}-
{1 \over
{\cos^2{\gamma'}+DK}}\right)\,d\left(\cos^2{\gamma'}\right) 
\label{eqn:int2g}
\end{equation}
which we can evaluate analytically. The final equation for the
amplitude transformation is
\begin{eqnarray}
A_n & = & A_0\,{\sqrt{\cos^2{\gamma}-\sin^2{\alpha}}\over
\sqrt{\cos^2{\gamma_0}-\sin^2{\alpha}}}\,\left(
{\cos^2{\gamma_0}+DK}\over{\cos^2{\gamma}+DK}\right)^{1/2}=
\nonumber \\
& = & A_0\,{{\tau_0\,\cos{\gamma}}\over{\tau_n\,\cos{\gamma_0}}}\,
\left(
{\cos^2{\gamma_0}+DK}\over{\cos^2{\gamma}+DK}\right)^{1/2}\;. 
\label{eqn:ampcurve}
\end{eqnarray}
In case of a plane reflector, the curvature $K$ is zero, and
equation~(\ref{eqn:ampcurve}) coincides with~(\ref{eqn:ampplane}).
In the general case can be rewritten as
\begin{equation}
A_n={{c\,\cos{\gamma}}\over{\tau_n\,\sqrt{\cos^2{\gamma}+DK}}}\;,
\label{eqn:ampray}
\end{equation}
where $c$ is constant along each time ray (it may vary with the reflection point
location on the reflector but not with the offset). We should compare equation
(\ref{eqn:ampray}) 
with the known expression for the reflection wave amplitude of the leading
ray series term in 2.5-D media \cite[]{cwp}:
\begin{equation}
A={{C_R(\gamma) \Psi}\over G}\;,
\label{eqn:amptrue}
\end{equation}
where $C_R$ stands for the angle-dependent reflection coefficient, $G$ is the
geometric spreading
\begin{equation}
G=v \tau {\sqrt{\cos^2{\gamma}+DK}\over \cos{\gamma}}\;,
\label{eqn:GS}
\end{equation}
and $\Psi$ includes other possible factors (such as the source
directivity) that we can either correct or neglect in the preliminary
processing.  It is evident that the curvature dependence of the
amplitude transformation (\ref{eqn:ampray}) coincides completely with
the true geometric spreading factor (\ref{eqn:GS}) and that the angle
dependence of the reflection coefficient is not accounted for the offset
continuation process. If the wavelet shape of the reflected wave on
seismic sections [$R_n$ in equation~(\ref{eqn:raymethod})] is
described by the delta function, then, as follows from the known
properties of this function,
\begin{equation}
A\,\delta\left(t-\tau(y,h)\right)=\left|{{dt_n} \over {dt}}\right|\,
A\,\delta\left(t_n-\tau_n(y,h)\right) =
{t \over t_n}\,A\,\delta\left(t_n-\tau_n(y,h)\right)\;, 
\label{eqn:deltafun} 
\end{equation}
which leads to the equality
\begin{equation}
A_n=A\,{t \over t_n}\;.
\label{eqn:a2an} 
\end{equation}
Combining equation~(\ref{eqn:a2an}) with equations~(\ref{eqn:amptrue})
and~(\ref{eqn:ampray}) allows us to evaluate the amplitude after
continuation from some initial offset $h_0$ to another offset $h_1$,
as follows:
\begin{equation}
A_1={{C_R(\gamma_0) \Psi_0}\over G_1}\;.
\label{eqn:ampfin}
\end{equation}
According to equation~(\ref{eqn:ampfin}), the OC process described by
equation (\ref{eqn:OCequation}) is amplitude-preserving in the sense
that corresponds to the definition of Born DMO
\cite[]{born,GEO56-02-01820189}. This means that the geometric spreading
factor from the initial amplitudes is transformed to the true
geometric spreading on the continued section, while the reflection
coefficient stays the same. This remarkable dynamic property allows
AVO (amplitude versus offset) analysis to be performed by a dynamic
comparison between true constant-offset sections and the sections
transformed by OC from different offsets.  With a simple trick, the
offset coordinate is transferred to the reflection angles for the AVO
analysis.  As follows from~(\ref{eqn:A}) and~(\ref{eqn:length}),
\begin{equation}
{\tau_n^2 \over \tau\,\tau_0}=\cos{\gamma}\;.
\label{eqn:AVO}
\end{equation}
If we include the ${t_n^2 \over t\,t_0}$ factor in the DMO operator
(continuation to zero offset) and divide the result by the DMO section
obtained without this factor, the resultant amplitude of the reflected
events will be directly proportional to $\cos{\gamma}$, where the
reflection angle $\gamma$ corresponds to the initial offset. Of
course, this conclusion is rigorously valid for constant-velocity
2.5-D media only.

\cite{GEO58-01-00470066} suggest a definition of true-amplitude DMO different
from that of Born DMO. The difference consists of two important
components:
\begin{enumerate}
\item {\em True-amplitude DMO addresses preserving the peak amplitude
    of the image wavelet instead of preserving its spectral density.}
  In the terms of this paper, the peak amplitude corresponds to the
  pre-NMO amplitude $A$ from formula~(\ref{eqn:amptrue}) instead of
  corresponding to the spectral density amplitude $A_n$.  A simple
  correction factor $t \over t_n$ would help us take the difference
  between the two amplitudes into account. Multiplication by $t \over
  t_n$ can be easily done at the NMO stage.
\item {\em Seismic sections are multiplied by time to correct for the
    geometric spreading factor prior to DMO (or, in our case, offset
    continuation) processing.} \end{enumerate} As follows
from~(\ref{eqn:GS}), multiplication by $t$ is a valid geometric
spreading correction for plane reflectors only.  It is the
amplitude-preserving offset continuation based on the OC
equation~(\ref{eqn:OCequation}) that is able to correct for the
curvature-dependent factor in the amplitude. To take into account the
second aspect of Black's definition, we can consider the modified
field $\hat{P}$ such that
\begin{equation}
\hat{P}\left(y,h,t_n\right)=t\,P\left(y,h,t_n\right)\;.
\label{eqn:p2pt}
\end{equation}
Substituting~(\ref{eqn:p2pt}) into the OC equation~(\ref{eqn:OCequation}) transforms the
latter to the form
\begin{equation}
h \, \left( {\partial^2 \hat{P} \over \partial y^2} - {\partial^2 \hat{P} \over
\partial h^2} \right) \, = \, t_n \, {\partial^2 \hat{P} \over {\partial t_n \,
\partial h}}\, -{\partial \hat{P} \over \partial h}\; . 
\label{eqn:BSZequation} 
\end{equation}
Equations~(\ref{eqn:BSZequation}) and~(\ref{eqn:OCequation}) differ
only with respect to the first-order damping term $\partial \hat{P}
\over \partial h$.  This term affects the amplitude behavior but not
the traveltimes, since the eikonal-type equation~(\ref{eqn:eikonal})
depends on the second-order terms only.  Offset continuation operators
based on~(\ref{eqn:BSZequation}) conform to Black's definition of
true-amplitude processing.

\cite{menorm} describe an alternative approach to
confirming the kinematic and amplitude validity of the offset
continuation equation. Applying equation~(\ref{eqn:OCequation})
directly on the Kirchhoff model of prestack seismic data shows that
the equation is satisfied to the same asymptotic order of accuracy
as the Kirchhoff modeling approximation \cite[]{haddon,norm}.

\section{Integral offset continuation operator}
%%%%%%%%%%%%%%%%%%%%%%
Equation~(\ref{eqn:OCequation}) describes a continuous process of
reflected wavefield continuation in the time-offset-midpoint domain.
In order to find an integral-type operator that performs the one-step
offset continuation, I consider the following initial-value
problem for equation~(\ref{eqn:OCequation}):

{\em Given a post-NMO constant-offset section at half-offset $h_1$}
\begin{equation}
\left.P(t_n,h,y)\right|_{h=h_1}=P^{(0)}_1(t_n,y) 
\label{eqn:bound0} 
\end{equation}
{\em and its first-order derivative with respect to offset}
\begin{equation}
\left.\partial P(t_n,h,y)\over \partial h\right|_{h=h_1}=P^{(1)}_1(t_n,y)\;, 
\label{eqn:bound1} 
\end{equation}
{\em find the corresponding section $P^{(0)}(t_n,y)$ at offset $h$.}

Equation~(\ref{eqn:OCequation}) belongs to the hyperbolic type, with
the offset coordinate $h$ being a ``time-like'' variable and the
midpoint coordinate $y$ and the time $t_n$ being ``space-like''
variables.  The last condition~(\ref{eqn:bound1}) is required for the
initial value problem to be well-posed \cite[]{kurant}. From a physical
point of view, its role is to separate the two different wave-like
processes embedded in equation~(\ref{eqn:OCequation}), which are
analogous to inward and outward wave propagation. We will associate
the first process with continuation to a larger offset and the second
one with continuation to a smaller offset.  Though the offset
derivatives of data are not measured in practice, they can be
estimated from the data at neighboring offsets by a finite-difference
approximation. Selecting a propagation branch explicitly, for example
by considering the high-frequency asymptotics of the continuation
operators, can allow us to eliminate the need for
condition~(\ref{eqn:bound1}). In this section, I discuss the exact
integral solution of the OC equation and analyze its asymptotics.

The integral solution of problem~(\ref{eqn:bound0}-\ref{eqn:bound1})
for equation~(\ref{eqn:OCequation}) is obtained in with the help of
the classic methods of mathematical physics
\cite[]{me,Fomel.sepphd.107}. It takes the explicit form
\begin{eqnarray}
P(t_n,h,y) & = &
\int\!\!\int P^{(0)}_1(t_1,y_1)\,G_0(t_1,h_1,y_1;t_n,h,y)\,dt_1\,dy_1
\nonumber \\ 
+ & & 
\int\!\!\int P^{(1)}_1(t_1,y_1)\,G_1(t_1,h_1,y_1;t_n,h,y)\,dt_1\,dy_1\;,
\label{eqn:integral} 
\end{eqnarray}
where the Green's functions $G_0$ and $G_1$ are expressed as
\begin{eqnarray}
G_0(t_1,h_1,y_1;t_n,h,y) & = & \mbox{sign}(h-h_1)\,{H(t_n) \over \pi}\,
{\partial \over \partial t_n}\,\left\{
H(\Theta) \over 
\sqrt{\Theta}\right\}\;,
\label{eqn:green0} \\
G_1(t_1,h_1,y_1;t_n,h,y) & = & \mbox{sign}(h-h_1)\,
{H(t_n) \over \pi}\,h\,
{t_n \over t_1^2}\,\left\{
H(\Theta) \over 
\sqrt{\Theta}\right\}\;,
\label{eqn:green1} 
\end{eqnarray}
and the parameter $\Theta$ is
\begin{equation}
\Theta(t_1,h_1,y_1;t_n,h,y)  = 
\left(h_1^2/t_1^2-h^2/t_n^2\right)\,\left(t_1^2-t_n^2\right)-
\left(y_1-y\right)^2\;.
\label{eqn:gamma}
\end{equation}
$H$ stands for the Heaviside step-function. 
 
From equations~(\ref{eqn:green0}) and~(\ref{eqn:green1}) one can see
that the impulse response of the offset continuation operator is
discontinuous in the time-offset-midpoint space on a surface defined
by the equality
\begin{equation}
\Theta(t_1,h_1,y_1;t_n,h,y)  = 0\;,
\label{eqn:conoid}
\end{equation}
which describes the ``wavefronts'' of the offset continuation process.
In terms of the theory of characteristics \cite[]{kurant}, the surface
$\Theta=0$ corresponds to the characteristic conoid formed by the
bi-characteristics of equation~(\ref{eqn:OCequation}) -- time rays
emerging from the point $\{t_n,h,y\}=\{t_1,h_1,y_1\}$. The
common-offset slices of the characteristic conoid are shown in the
left plot of Figure \ref{fig:cont}.

\plot{cont}{width=6in}{
Constant-offset sections of the characteristic conoid - ``offset
continuation fronts'' (left), and branches of the conoid used in the
integral OC operator (right). The upper part of the plots
(small times) corresponds to continuation to smaller offsets; the 
lower part (large times) corresponds to larger offsets.}

As a second-order differential equation of the hyperbolic type,
equation (\ref{eqn:OCequation}) describes two different processes. The
first process is ``forward'' continuation from smaller to larger
offsets, the second one is ``reverse'' continuation in the opposite
direction.  These two processes are clearly separated in the
high-frequency asymptotics of operator~(\ref{eqn:integral}). To obtain
the asymptotic representation, it is sufficient to note that ${1
  \over \sqrt{\pi}}\, {H(t) \over \sqrt{t}}$ is the impulse response
of the causal half-order integration operator and that $H(t^2-a^2)
\over \sqrt{t^2-a^2}$ is asymptotically equivalent to $H(t-a) \over
{\sqrt{2a}\,\sqrt{t-a}}$ $(t, a >0)$.  Thus, the asymptotical form of
the integral offset-continuation operator becomes
\begin{eqnarray}
P^{(\pm)}(t_n,h,y) & = &
{\bf D}^{1/2}_{\pm\,t_n}\,\int w^{(\pm)}_0(\xi;h_1,h,t_n)\,
P^{(0)}_1(\theta^{(\pm)}(\xi;h_1,h,t_n),y_1-\xi)\,d\xi
\nonumber \\ 
& \pm  & 
{\bf I}^{1/2}_{\pm\,t_n}\,\int w^{(\pm)}_1(\xi;h_1,h,t_n)\,
P^{(1)}_1(\theta^{(\pm)}(\xi;h_1,h,t_n),y_1-\xi)\,d\xi\;.
\label{eqn:asintegral} 
\end{eqnarray}
Here the signs ``$+$'' and ``$-$'' correspond to the type of
continuation (the sign of ${h-h_1}$), ${\bf D}^{1/2}_{\pm\,t_n}$ and
${\bf I}^{1/2}_{\pm\,t_n}$ stand for the operators of causal and
anticausal half-order differentiation and integration applied with
respect to the time variable $t_n$, the summation paths
$\theta^{(\pm)}(\xi;h_1,h,t_n)$ correspond to the two non-negative
sections of the characteristic conoid~(\ref{eqn:conoid}) (Figure
\ref{fig:cont}):
\begin{equation}
t_1=\theta^{(\pm)}(\xi;h_1,h,t_n)=
{t_n \over h}\,\sqrt{{U \pm V} \over 2 }\;,
\label{eqn:summation}
\end{equation}
where $U=h^2+h_1^2-\xi^2$, and $V=\sqrt{U^2-4\,h^2\,h_1^2}$; $\xi$ is
the midpoint separation (the integration parameter), and $w^{(\pm)}_0$
and $w^{(\pm)}_1$ are the following weighting functions:
\begin{eqnarray}
w^{(\pm)}_0 & = & {1 \over \sqrt{2\,\pi}}\,
{\theta^{(\pm)}(\xi;h_1,h,t_n) \over \sqrt{t_n\,V}}\;,
\label{eqn:w0} \\
w^{(\pm)}_1 & = & {1 \over \sqrt{2\,\pi}}\,
{{\sqrt{t_n}\, h_1} \over {\sqrt{V}\,\theta^{(\pm)}(\xi;h_1,h,t_n)}}\;. 
\label{eqn:w1} 
\end{eqnarray}  
Expression~(\ref{eqn:summation}) for the summation path of the OC
operator was obtained previously by \cite{stovas} and
\cite{SEG-1994-1541}. A somewhat different form of it is proposed by
\cite{GEO61-06-18461858}. I describe the kinematic interpretation
of formula~(\ref{eqn:summation}) in Appendix~B.

In the high-frequency asymptotics, it is possible to replace the two
terms in equation~(\ref{eqn:asintegral}) with a single term
\cite[]{GEO68-03-10321042}. The single-term expression is
\begin{equation}
P^{(\pm)}(t_n,h,y)  = 
{\bf D}^{1/2}_{\pm\,t_n}\,\int w^{(\pm)}(\xi;h_1,h,t_n)\,
P^{(0)}_1(\theta^{(\pm)}(\xi;h_1,h,t_n),y_1-\xi)\,d\xi\;,
\end{equation}
where
\begin{eqnarray}
w^{(+)} & = & \sqrt{\theta^{(+)}(\xi;h_1,h,t_n) \over {2\,\pi}}\;
{{h^2-h_1^2-\xi^2} \over {V^{3/2}}}\;, 
\label{eqn:wOC12} \\
w^{(-)} & = & {{\theta^{(-)}(\xi;h_1,h,t_n)} \over \sqrt{2\,\pi t_n}}\;
{{h_1^2-h^2 +\xi^2} \over {V^{3/2}}}\;. 
\label{eqn:wOC21} 
\end{eqnarray}
A more general approach to true-amplitude asymptotic offset
continuation is developed by \cite{tygel}.

The limit of expression~(\ref{eqn:summation}) for the output offset $h$
approaching zero can be evaluated by L'Hospitale's rule. As one would
expect, it coincides with the well-known expression for the summation
path of the integral DMO operator
\cite[]{GPR29-03-03740406}
\begin{equation}
t_1=\theta^{(-)}(\xi;h_1,0,t_n)=
\lim_{h \rightarrow 0} {{t_n \over h}\,\sqrt{{U - V} \over 2 }}=
{{t_n\,h_1} \over \sqrt{h_1^2-\xi^2}}\;.
\label{eqn:DMOsummation}
\end{equation}
I discuss the connection between offset continuation and DMO in the
next section.

\section{Offset continuation and DMO}
%%%%%%%%%%%%%%%%%%%%%%%%%%%%%%%%%%%

Dip moveout represents a particular case of offset continuation for
the output offset equal to zero. In this section, I consider the DMO
case separately in order to compare the solutions of equation
(\ref{eqn:OCequation}) with the Fourier-domain DMO operators, which
have been the standard for DMO processing since Hale's outstanding
work \cite[]{Hale.sepphd.36,GEO49-06-07410757}.

Equation~(\ref{eqn:integral}) transforms to the time-wavenumber domain
with the help of integral tables:
\begin{equation}
\widetilde{P}(t_n,h,k)=
H(t_n)\,\left(\widetilde{P}_0(t_n,h,k) +
t_n\,\widetilde{P}_1(t_n,h,k)\right)\;,
\label{eqn:pp0p1} 
\end{equation}
where
\begin{eqnarray}
\widetilde{P}_0  & = & 
{\partial \over {\partial t_n}}\,
\int_{\left(h_1/h\right)\,t_n}^{t_n}
\widetilde{P}^{(0)}_1\left(\left|t_1\right|,k\right)\,
J_0\left(k\,\sqrt{\left({h^2 \over t_n^2}-{h_1^2 \over t_1^2}\right)\,
\left(t_n^2-t_1^2\right)}\right)\,dt_1\;,
\label{eqn:p0} \\
\widetilde{P}_1  & = & 
\int_{\left(h_1/h\right)\,t_n}^{t_n}
h_1\,\widetilde{P}^{(1)}_1\left(\left|t_1\right|,k\right)\,
J_0\left(k\,\sqrt{\left({h^2 \over t_n^2}-{h_1^2 \over t_1^2}\right)\,
\left(t_n^2-t_1^2\right)}\right)\,{dt_1 \over t_1^2}\;,
\label{eqn:p1} 
\end{eqnarray}
\begin{eqnarray}
\widetilde{P}^{(j)}_1(t_1,k) & = & 
\int\,P\,^{(j)}_1(t_1,y_1)\exp (-iky_1)\,dy_1\;(j=0,1)\;,
\label{eqn:tildej} \\
\widetilde{P}(t_n,h,k)  & = & 
\int\,P(t_n,h,y)\exp (-iky)\,dy\;(j=0,1)\;.
\label{eqn:tilde}
\end{eqnarray}

Setting the output offset to zero, we obtain the following DMO-like
integral operators in the $t$--$k$ domain:
\begin{equation}
\widetilde{P}(t_0,0,k)=
H(t_0)\,\left(\widetilde{P}_0(t_0,k) +
t_0\,\widetilde{P}_1(t_0,k)\right)\;,
\label{eqn:dp0p1} 
\end{equation}
where
\begin{equation}
\widetilde{P}_0(t_0,k)  = 
- {\partial \over {\partial t_0}}\,
\int_{t_0}^{\infty}
\widetilde{P}^{(0)}_1\left(\left|t_1\right|,k\right)\,
J_0\left({{k\,h_1}\over t_1}\,
\sqrt{t_1^2-t_0^2}\right)\,dt_1\;,
\label{eqn:d0} 
\end{equation}
\begin{equation}
\widetilde{P}_1(t_0,k)  = 
- \int_{t_0}^{\infty}
h_1\,\widetilde{P}^{(1)}_1\left(\left|t_1\right|,k\right)\,
J_0\left({{k\,h_1}\over t_1}\,
\sqrt{t_1^2-t_0^2}\right)\,{dt_1 \over t_1^2}\;,
\label{eqn:d1} 
\end{equation}
the wavenumber $k$ corresponds to the midpoint axis $y$, and $J_0$ is
the zeroth-order Bessel function.  The Fourier transform
of~(\ref{eqn:d0}) and (\ref{eqn:d1}) with respect to the time variable
$t_0$ reduces to known integrals \cite[]{table} and creates explicit
DMO-type operators in the frequency-wavenumber domain, as follows:
\begin{equation}
\widetilde{\widetilde{P}}_0(\omega_0,k)  = 
i\,
\int_{-\infty}^{\infty}
\widetilde{P}^{(0)}_1\left(\left|t_1\right|,k\right)\,
{\sin{\left(\omega_0\,|t_1|\,A\right)} \over A}
\,dt_1\;,
\label{eqn:dw0} 
\end{equation}
\begin{equation}
\widetilde{\widetilde{P}}_1(\omega_0,k)  = 
i\, \int_{-\infty}^{\infty}
h_1\,\widetilde{P}^{(1)}_1\left(\left|t_1\right|,k\right)\,
{\sin{\left(\omega_0\,|t_1|\,A\right)} \over A}
{dt_1 \over t_1^2}\;,
\label{eqn:dw1} 
\end{equation}
where 
\begin{equation}
A=\sqrt{1+{(k\,h_1)^2 \over (\omega_0\,t_1)^2}}\;,
\label{eqn:a} 
\end{equation}
\begin{equation}
\widetilde{\widetilde{P}}_j(\omega_0,k)=
\int\,\widetilde{P}_j(t_0,k)\,\exp (i\omega_0 t_0)\,dt_0\;.
\label{eqn:FFT} 
\end{equation}

It is interesting to note that the first term of the continuation to
zero offset~(\ref{eqn:dw0}) coincides exactly with the imaginary part
of Hale's DMO operator \cite[]{GEO49-06-07410757}. However, unlike Hale's,
operator~(\ref{eqn:dp0p1}) is causal, which means that its impulse
response does not continue to negative times. The non-causality of
Hale's DMO and related issues are discussed in more detail by
\cite{stovas}.
%I include a brief summary of this discussion in Appendix~\ref{chapter:hale}.

Though Hale's DMO is known to provide correct reconstruction of the
geometry of zero-offset reflections, it does not account properly for
the amplitude changes \cite[]{GEO58-01-00470066}. The preceding section of this
paper shows that the additional contribution to the amplitude
is contained in the second term of the OC operator
(\ref{eqn:integral}), which transforms to the second term in the DMO
operator~(\ref{eqn:dp0p1}). Note that this term vanishes at the input offset
equal to zero, which represents the case of the inverse DMO operator.

Considering the inverse DMO operator as the continuation from zero
offset to a non-zero offset, we can obtain its representation in the
$t$-$k$ domain from equations~(\ref{eqn:pp0p1}-\ref{eqn:p1}) as
\begin{equation}
\widetilde{P}(t_n,h,k)  = 
H(t_n) {\partial \over {\partial t_n}}\,
\int_{0}^{t_n}
\widetilde{P}_0\left(\left|t_0\right|,k\right)\,
J_0\left({{k\,h}\over t_n}\,
\sqrt{t_n^2-t_0^2}\right)\,dt_0\;,
\label{eqn:i0} 
\end{equation}
Fourier transforming equation~(\ref{eqn:i0}) with respect to the time
variable $t_0$ according to equation~(\ref{eqn:FFT}), we get the
Fourier-domain version of the ``amplitude-preserving'' inverse DMO:
\begin{equation}
\widetilde{P}(t_n,h,k) = 
{H(t_n) \over {2\,\pi}}\,{\partial \over \partial t_n}\,
\int_{-\infty}^{\infty}
\widetilde{\widetilde{P}}_0(\omega_0,k)\, 
{{\sin{\left(\omega_0\,|t_n|\,A\right)} \over {\omega_0\,A}}}
\,d\omega_0\;,
\label{eqn:IDMO} 
\end{equation}
\begin{equation}
A=\sqrt{1+{(k\,h)^2 \over (\omega_0\,t_n)^2}}\;.
\label{eqn:a1} 
\end{equation}

Comparing operator~(\ref{eqn:IDMO}) with Ronen's version of inverse DMO
\cite[]{GEO52-07-09730984}, one can see that if Hale's DMO is denoted
by ${\bf D}_{t_0}\,{\bf H}$, then Ronen's inverse DMO is ${\bf
H^{T}\,D}_{-t_0}$, while the amplitude-preserving inverse~(\ref{eqn:IDMO})
is ${\bf D}_{t_n}\,{\bf H^T}$. Here ${\bf D}_t$ is the derivative
operator $\left( \partial \over \partial t\right)$, and ${\bf H^T}$
stands for the adjoint operator defined by the dot-product test
\begin{equation}
{\bf (Hm,d)=(m,H^{T}d)},
\label{eqn:dot}
\end{equation}
where the parentheses denote the dot product:
\begin{equation}
{\bf (m_1,m_2)}=\int\!\int\,m_1(t_n,y)\,m_2(t_n,y)\,dt_n\,dy\;.
\nonumber
\end{equation}
  
In high-frequency asymptotics, the difference between the amplitudes
of the two inverses is simply the Jacobian term ${d\,t_0 \over
  d\,t_n}$, asymptotically equal to ${t_0 \over t_n}$. This difference
corresponds exactly to the difference between Black's definition of
amplitude preservation \cite[]{GEO58-01-00470066} and the definition used in Born
DMO \cite[]{born,GEO56-02-01820189}, as discussed above. While operator
(\ref{eqn:IDMO}) preserves amplitudes in the Born DMO sense, Ronen's
inverse satisfies Black's amplitude preservation criteria. This means
Ronen's operator implies that the ``geometric spreading'' correction
(multiplication by time) has been performed on the data prior to DMO.

To construct a one-term DMO operator, thus avoiding the estimation of
the offset derivative in~(\ref{eqn:w1}), let us consider the problem
of inverting the inverse DMO operator~(\ref{eqn:IDMO}). One of the
possible approaches to this problem is the least-squares iterative
inversion, as proposed by \cite{GEO52-07-09730984}. This
requires constructing the adjoint operator, which is Hale's DMO (or
its analog) in the case of Ronen's method. The iterative least-squares
approach can account for irregularities in the data geometry
\cite[]{IZO,SEG-1994-1545} and boundary effects, but it is computationally
expensive because of the multiple application of the operators. An
alternative approach is the asymptotic inversion, which can be viewed
as a special case of preconditioning the adjoint operator
\cite[]{SEG-1988-S17.5,SEG-1996-0032}. The goal of the asymptotic inverse is to
reconstruct the geometry and the amplitudes of the reflection events
in the high-frequency asymptotic limit.

According to Beylkin's theory of asymptotic inversion, also known as
the {\em generalized Radon transform} \cite[]{beylkin}, two operators of the
form
\begin{equation}
D(\omega)=\int X(t,\omega)\,M(t)\,
\exp\left[i\omega \phi (t,\omega)\right]\,dt
\label{eqn:b0}
\end{equation}
and
\begin{equation}
\widetilde{M}(t)=\int Y(t,\omega)\,D(\omega)\,
\exp\left[-i\omega \phi (t,\omega)\right]\,d\omega
\label{eqn:b1}
\end{equation}
constitute a pair of asymptotically inverse operators
($\widetilde{M}(t)$ matching $M(t)$ in the high-frequency asymptotics)
if
\begin{equation}
X(t,\omega)\,Y(t,\omega)={Z(t,\omega) \over {2\,\pi}}\;,
\label{eqn:belka}
\end{equation}
where $Z$ is the ``Beylkin determinant''
\begin{equation}
Z(t,\omega)=\left|\partial \omega \over \partial \hat{\omega}\right|\;
\mbox{for}\;\hat{\omega}=\omega\,{\partial \phi(t,\omega) \over \partial t}\;.
\label{eqn:det}
\end{equation}

With respect to the high-frequency asymptotic representation, we can
recast (\ref{eqn:IDMO}) in the equivalent form by moving the time
derivative under the integral sign:
\begin{equation}
\widetilde{P}(t_n,k) \approx 
{H(t_n) \over {2\,\pi}}\,\mbox{Re}\left[
\int_{-\infty}^{\infty}A^{-2}
\widetilde{\widetilde{P}}_0(\omega_0,k)\, 
\exp\left(-i \omega_0\,|t_n|\,A\right)
\,d\omega_0\right]
\label{eqn:IDMO2} 
\end{equation}
Now the asymptotic inverse of~(\ref{eqn:IDMO2}) is evaluated by
means of Beylkin's method (\ref{eqn:b0})-(\ref{eqn:b1}), which leads
to an amplitude-preserving one-term DMO operator of the form
\begin{equation}
\widetilde{\widetilde{P}}_0(\omega_0,k)  = 
\mbox{Im}\left[
\int_{-\infty}^{\infty} B
\widetilde{P}^{(0)}_1\left(\left|t_1\right|,k\right)\,
\exp\left(i \omega_0\,|t_1|\,A\right)
\,dt_1\right]\;,
\label{eqn:born} 
\end{equation}
where 
\begin{equation}
B = A^2 {\partial \over \partial \omega_0}\left(\omega_0\,
{{\partial (t_n\,A)} \over \partial t_n}\right) = 
A^{-1}\,(2\,A^2 - 1)\;.
\label{eqn:jack} 
\end{equation}
  
The amplitude factor~(\ref{eqn:jack}) corresponds exactly to that of
Born DMO \cite[]{born} in full accordance with the conclusions of the
asymptotic analysis of the offset-continuation amplitudes. An
analogous result can be obtained with the different definition of
amplitude preservation proposed by \cite{GEO58-01-00470066}. In
the time-and-space domain, the operator asymptotically analogous to
(\ref{eqn:born}) is found by applying either the stationary phase
technique \cite[]{GEO55-05-05950607,GEO58-01-00470066} or Goldin's method of
discontinuities \cite[]{goldintomo,Goldin.sep.67.171}, which is the
time-and-space analog of Beylkin's asymptotic inverse theory
\cite[]{stovas}. The time-and-space asymptotic DMO operator takes the
form
\begin{equation}
P_0(t_0,y) = {\bf D}^{1/2}_{-t_0}\,\int w_0(\xi;h_1,t_0)\,
P^{(0)}_1(\theta^{(-)}(\xi;h_1,0,t_0),y_1-\xi)\,d\xi\;,
\label{eqn:TADMO}
\end{equation} 
where the weighting function $w_0$ is defined as 
\begin{equation}
w_0(\xi;h_1,t_0)=\sqrt{t_0 \over {2\,\pi}}\, 
{{h_1\,(h_1^2+\xi^2)} \over (h_1^2-\xi^2)^2}\;.
\label{eqn:TAw}
\end{equation}

\section{Offset continuation in the log-stretch domain}
%%%%%%%%%%%%%%%%%%%%%%%%%%%%%%%%%%%%%%%%%%%%%%%%%%%%%
The log-stretch transform, proposed by \cite{GPR30-06-08130828}
and further developed by many other researchers, is a useful
tool in DMO and OC processing. Applying a log-stretch transform of the
form
\begin{equation}
\sigma = \ln\left|t_n \over t_* \right|\;, 
\label{eqn:log}
\end{equation}
where $t_*$ is an arbitrarily chosen time constant, eliminates the
time dependence of the coefficients in equation~(\ref{eqn:OCequation})
and therefore makes this equation invariant to time shifts. After the
double Fourier transform with respect to the midpoint coordinate $y$
and to the transformed (log-stretched) time coordinate $\sigma$, the
partial differential equation~(\ref{eqn:OCequation}) takes the form of
an ordinary differential equation,
\begin{equation}
h\,\left({{d^2 \widehat{\widehat{P}}} 
\over {dh^2}} + k^2\,\widehat{\widehat{P}}\right) =
i\Omega\,{{d \widehat{\widehat{P}}} \over {dh}}\;,
\label{eqn:LSequation}
\end{equation}
where
\begin{equation}
\widehat{\widehat{P}}(h) = \int\!\int P(t_n=t_*\,\exp(\sigma),h,y)\,
\exp(i\Omega\sigma - iky)\,d\sigma\,dy\;. 
\label{eqn:LSFT}
\end{equation}

Equation~(\ref{eqn:LSequation}) has the known general solution,
expressed in terms of cylinder functions of complex order $\lambda =
{{1+i\Omega} \over 2}$ \cite[]{watson}
\begin{equation}
\widehat{\widehat{P}}(h) = 
C_1(\lambda)\,(kh)^{\lambda}\,J_{-\lambda}(kh)+
C_2(\lambda)\,(kh)^{\lambda}\,J_{\lambda}(kh)\;,
\label{eqn:gensol}
\end{equation}
where $J_{-\lambda}$ and $J_{\lambda}$ are Bessel functions, and $C_1$
and $C_2$ stand for some arbitrary functions of $\lambda$ that do not
depend on $k$ and $h$.

In the general case of offset continuation, $C_1$ and $C_2$ are
constrained by the two initial conditions~(\ref{eqn:bound0}) and
(\ref{eqn:bound1}). In the special case of continuation from zero offset, we
can neglect the second term in~(\ref{eqn:gensol}) as vanishing at the zero
offset. The remaining term defines the following operator of inverse
DMO in the ${\Omega,k}$ domain:
\begin{equation}
\widehat{\widehat{P}}(h) = \widehat{\widehat{P}}(0)\,Z_{\lambda}(kh)\;,
\label{eqn:OKDMO}
\end{equation}
where $Z_{\lambda}$ is the analytic function
\begin{eqnarray}
\nonumber
Z_{\lambda}(x) & = & \Gamma(1-\lambda)\,\left(x \over 2\right)^{\lambda}\,
J_{-\lambda}(x)=
{}_0F_1\left(;1-\lambda;-\frac{x^2}{4}\right) \\
& = &
\sum_{n=0}^{\infty} {(-1)^n \over n!}\,
{\Gamma(1-\lambda) \over \Gamma(n+1-\lambda)}\,
\left(x \over 2\right)^{2n}\;,
\label{eqn:z}
\end{eqnarray}
$\Gamma$ is the gamma function and ${}_0F_1$ is the confluent
hypergeometric limit function \cite[]{ab}.

The DMO operator now can be derived as the inversion of operator
(\ref{eqn:OKDMO}), which is a simple multiplication by
$1/Z_{\lambda}(kh)$. Therefore, offset continuation becomes a
multiplication by $Z_{\lambda}(kh_2)/Z_{\lambda}(kh_1)$ (the cascade
of two operators). This fact demonstrates an important advantage of
moving to the log-stretch domain: both offset continuation and DMO are simple
filter multiplications in the Fourier domain of the log-stretched time
coordinate.

In order to compare operator~(\ref{eqn:OKDMO}) with the known versions
of log-stretch DMO, we need to derive its asymptotic representation
for high frequency $\Omega$. The required asymptotic expression
follows directly from the definition of function $Z_{\lambda}$ in
equation~(\ref{eqn:z}) and the known asymptotic representation for a Bessel
function of high order \cite[]{watson}:
\begin{equation}
J_{\lambda}(\lambda z) \stackrel{\lambda \rightarrow \infty}{\approx} 
{{(\lambda z)^{\lambda}\,\exp\left(\lambda\,\sqrt{1-z^2}\right)} \over
{e^{\lambda}\,\Gamma(\lambda+1)\,(1-z^2)^{1/4}\,
\left\{1+\sqrt{1-z^2}\right\}^{\sqrt{1-z^2}}}}\;.
\label{eqn:carlini}
\end{equation}
Substituting approximation~(\ref{eqn:carlini}) into~(\ref{eqn:z}) and
considering the high-frequency limit of the resultant expression
yields
\begin{equation}
Z_{\lambda}(kh) \approx 
\left\{{1+\sqrt{1-\left(kh \over \lambda\right)^2}} \over
2\right\}^{\lambda}\, 
{{\exp\left(\lambda\,\left[1 - \sqrt{1-\left(kh \over
\lambda\right)^2}\right]\right)} \over 
\left(1-\left(kh \over \lambda\right)^2\right)^{1/4}} \approx
{F(\epsilon)\,e^{i\Omega\,\psi(\epsilon)}}\;,
\label{eqn:AOKDMO}
\end{equation}
where $\epsilon$ denotes the ratio ${2\,k\,h} \over
\Omega$,
\begin{equation}
F(\epsilon)=\sqrt{{1+\sqrt{1+\epsilon^2}} \over
{2\,\sqrt{1+\epsilon^2}}}\,
\exp\left({1-\sqrt{1+\epsilon^2}} \over 2\right)\;,
\label{eqn:F}
\end{equation}
and
\begin{equation}
\psi(\epsilon)={1 \over 2}\,\left(1 - \sqrt{1+\epsilon^2} +
\ln\left({1 + \sqrt{1+\epsilon^2}} \over 2\right)\right)\;.
\label{eqn:psi}
\end{equation}

The asymptotic representation~(\ref{eqn:AOKDMO}) is valid for high
frequency $\Omega$ and $|\epsilon| \leq 1$. The
phase function $\psi$ defined in~(\ref{eqn:psi}) coincides precisely
with the analogous term in Liner's \emph{exact log DMO}
\cite[]{GEO55-05-05950607}, which provides the correct
geometric properties of DMO. Similar expressions for the log-stretch
phase factor $\psi$ were derived in different ways by
\cite{GEO61-03-08150820} and \cite{GEO61-04-11031114}.
However, the amplitude term $F(\epsilon)$ differs from the previously
published ones because of the difference in the amplitude preservation
properties.
  
A number of approximate log DMO operators have been proposed in the
literature. As shown by \cite{GEO55-05-05950607}, all of them but
exact log DMO distort the geometry of reflection effects at large
offsets. The distortion is caused by the implied approximations of the
true phase function $\psi$. Bolondi's OC operator
\cite[]{GPR30-06-08130828} implies $\psi(\epsilon) \approx -{\epsilon^2
  \over 8}$, Notfors' DMO \cite[]{GEO52-12-17181721} implies
$\psi(\epsilon) \approx 1 - \sqrt{1+(\epsilon /2)^2}$, and the ``full
DMO'' \cite[]{SEG-1987-S14.1} has $\psi(\epsilon) \approx {1 \over 2}
\ln\left[1-(\epsilon / 2)^2\right]$. All these approximations are
valid for small $\epsilon$ (small offsets or small reflector dips) and
have errors of the order of $\epsilon^4$ (Figure \ref{fig:pha}).  The
range of validity of Bolondi's operator is defined in
equation~(\ref{eqn:hm}).
  
  \sideplot{pha}{height=1.5in}{ Phase functions of the log
    DMO operators. Solid line: exact log DMO; dashed line: Bolondi's
    OC; dashed-dotted line: Bale's full DMO; dotted line: Notfors'
    DMO.  }
  
  In practice, seismic data are often irregularly sampled in space but
  regularly sampled in time. This makes it attractive to apply offset
  continuation and DMO operators in the $\{\Omega,y\}$ domain, where
  the frequency $\Omega$ corresponds to the log-stretched time and
  $y$ is the midpoint coordinate. Performing the inverse Fourier
  transform on the spatial frequency transforms the inverse DMO
  operator~(\ref{eqn:OKDMO}) to the $\{\Omega,y\}$ domain, where the
  filter multiplication becomes a convolutional operator:
\begin{equation}
\widehat{P}(\Omega,h,y) =
{\widehat{F}(\Omega) \over \sqrt{2\,\pi}}\,\int_{|\xi|<h}
{ h \over {h^2-\xi^2}}\,
\widehat{P_0}(\Omega,y-\xi)\,
\exp\left(-{i\Omega \over 2}\,\ln\left(1-{\xi^2 \over h_1^2}\right)\right) 
\,d\xi\;.
\label{eqn:OXDMO} 
\end{equation}
Here $\widehat{F}(\Omega)$ is a high-pass frequency filter:
\begin{equation}
\widehat{F}(\Omega)={{\Gamma(1/2-i\Omega/2)}
\over {\sqrt{1/2}\,\Gamma(-i\Omega/ 2)}}\;.
\label{eqn:hat} 
\end{equation}
At high frequencies $\widehat{F}(\Omega)$ is approximately equal to
$(- i \Omega)^{1/2}$, which corresponds to the half-derivative
operator $\left(\partial \over \partial \sigma \right)^{1/2}$, which,
in turn, is equal to the $\left(t_n {\partial \over \partial t_n}
\right)^{1/2}$ term of the asymptotic OC
operator~(\ref{eqn:asintegral}). The difference between the exact
filter $\widehat{F}$ and its approximation by the half-order
derivative operator is shown in Figure \ref{fig:flt}. This difference
is a measure of the validity of asymptotic OC operators.

\plot{flt}{width=6in}{ Amplitude (left) and phase (right) of
  the time filter in the log-stretch domain. The solid line is for the
  exact filter; the dashed line for its approximation by the
  half-order derivative filter. The horizontal axis corresponds to the
  dimensionless log-stretch frequency $\Omega$.}

Inverting operator~(\ref{eqn:OXDMO}), we can obtain the DMO operator in the
$\{\Omega,y\}$ domain.

\section{Discussion}

The differential model for offset continuation is based on several
assumptions. It is important to fully realize them in order to
understand the practical limitations of this model.
\begin{itemize}
\item The \emph{constant velocity} assumption is essential for
  theoretical derivations. In practice, this limitation is not too
  critical, because the operators act locally. DMO and
  offset continuation algorithms based on the constant-velocity
  assumptions are widely used in practice \cite[]{DMO00-00-04960496}.
\item The \emph{single-mode} assumption does not include multiple
  reflections in the model. If multiple events (with different
  apparent velocities) are present in the data, they might require
  extending the model. Convolving two (or more) differential offset
  continuation operators, corresponding to different velocities, we
  can obtain a higher-order differential operator for predicting
  multiple events.
\item The \emph{continuous AVO} assumption implies that the
  reflectivity variation with offset is continuous and can be
  neglected in a local neighborhood of a particular offset. While the
  offset continuation model correctly predicts the geometric spreading
  effects in the reflected wave amplitudes, it does not account for
  the variation of the reflection coefficient with offset.
\item The \emph{2.5-D} assumption was implicit in the derivation of
  the offset continuation equation. According to this assumption, the
  reflector does not change in the cross-line direction, and we can
  always consider the reflection plane in two dimensions. 
\end{itemize}

\section{Conclusions}
%%%%
I have introduced a partial differential
equation~(\ref{eqn:OCequation}) and proved that the process described
by it provides for a kinematically and dynamically equivalent offset
continuation transform. Kinematic equivalence means that in constant
velocity media the reflection traveltimes are transformed to their
true locations on different offsets. Dynamic equivalence means that,
in the OC process, the geometric spreading term in the amplitudes of
reflected waves transforms in accordance with the laws of geometric
seismics, while the angle-dependent reflection coefficient stays the
same.
  
The offset continuation equation can be applied directly to design OC
operators of the finite-difference type. To construct integral OC
operators, an initial value problem is solved for the
offset continuation equation~(\ref{eqn:OCequation}). For the special
cases of continuation to zero offset (DMO) and continuation from zero
offset (inverse DMO), the OC operators are related to the known forms
of DMO operators: Hale's Fourier DMO, Born DMO, and Liner's ``exact
log DMO.'' The discovery of these relations sheds additional light on
the problem of amplitude preservation in DMO.

\section{Acknowledgments}
Sergey Goldin drew my attention to the role of curvature dependence in
reflected wave amplitudes \cite[]{joint}.  I also acknowledge the
support of Alexey Stovas, who introduced me to the world of DMO and
collaborated with me on developing the asymptotic theory of offset
continuation \cite[]{stovas}. Norman Bleistein collaborated with me on 
confirming the amplitude validity of offset continuation
\cite[]{SEG-1996-1731,menorm}. Shuki Ronen and Richard Bale
pointed out the importance of the log-stretch domain in DMO and
related problems.  I appreciate the encouragement of Jon Claerbout and
Fabio Rocca regarding the analysis of the offset-continuation partial
differential equation.

The financial support for this work was provided by the sponsors of
the Stanford Exploration Project.

%\newpage

\bibliographystyle{seg}
\bibliography{ofcon,SEG,SEP2}

%\newpage

\append{Second-order reflection traveltime derivatives}
%%%%%%%%%%%%%%%%%%%%%%%%%%%%%%%%%%%%%%%%%%%%%%%%%%%%%%%
\label{chapter:deriv}
This appendix contains a derivation of equations connecting
second-order partial derivatives of the reflection traveltime with the
geometric properties of the reflector in a constant velocity medium.
These equations are used in the main text of this paper to describe
the amplitude behavior of offset continuation.  Let $\tau(s,r)$ be the
reflection traveltime from the source $s$ to the receiver $r$.
Consider a formal equality
\begin{equation}
\tau(s,r)=\tau_1\left(s,x(s,r)\right)+\tau_2\left(x(s,r),r\right)\;,   
\label{eqn:t1pt2} 
\end{equation}
where $x$ is the reflection point parameter, $\tau_1$ corresponds to the
incident ray, and $\tau_2$ corresponds to the reflected ray.
Differentiating (\ref{eqn:t1pt2}) with respect to $s$ and $r$ yields
\begin{eqnarray}
{\partial \tau \over \partial s}  & = & 
{\partial \tau_1 \over \partial s} + {\partial \tau \over \partial x}\,
{\partial x \over \partial s}\;,
\label{eqn:dt1ds} \\
{\partial \tau \over \partial r}  & = & 
{\partial \tau_2 \over \partial r} + {\partial \tau \over \partial x}\,
{\partial x \over \partial r}\;.
\label{eqn:dt2dr} 
\end{eqnarray}
According to Fermat's principle, the two-point reflection ray path must
correspond to  the traveltime stationary point. Therefore 
\begin{equation}
{\partial \tau \over \partial x} \equiv 0  
\label{eqn:fermat} 
\end{equation}
for any $s$ and $r$. Taking into account (\ref{eqn:fermat}) while
differentiating (\ref{eqn:dt1ds}) and (\ref{eqn:dt2dr}), we get
\begin{eqnarray}
{\partial^2 \tau \over \partial s^2}  & = & 
{\partial^2 \tau_1 \over \partial s^2} + 
B_1\,
{\partial x \over \partial s}\;,
\label{eqn:d2tds2} \\
{\partial^2 \tau \over \partial r^2}  & = & 
{\partial^2 \tau_2 \over \partial r^2} + 
B_2\,
{\partial x \over \partial r}\;,
\label{eqn:d2tdr2} \\
{\partial^2 \tau \over \partial s \partial r}  & = &  
B_1\,
{\partial x \over \partial r}\;=
B_2\,
{\partial x \over \partial s}\;,
\label{eqn:d2tdsdr}
\end{eqnarray}
where
\[
B_1={\partial^2 \tau_1 \over \partial s \partial x}\;;\;
B_2={\partial^2 \tau_2 \over \partial r \partial x}\;.
\]
Differentiating equation~(\ref{eqn:fermat}) gives us the additional
pair of equations
\begin{eqnarray}
C\,{\partial x \over \partial s}+B_1  & = & 0\;,
\label{eqn:b12c} \\
C\,{\partial x \over \partial r}+B_2  & = & 0\;,
\label{eqn:b22c}
\end{eqnarray}
where
\[
C={\partial^2 \tau \over \partial x^2}=
{\partial^2 \tau_1 \over \partial x^2}+
{\partial^2 \tau_2 \over \partial x^2}\;.
\]
Solving the system (\ref{eqn:b12c}) - (\ref{eqn:b22c}) for $\partial x
\over \partial s$ and $\partial x \over \partial r$ and substituting
the result into (\ref{eqn:d2tds2}) - (\ref{eqn:d2tdsdr}) produces the
following set of expressions:
\begin{eqnarray}
{\partial^2 \tau \over \partial s^2}  & = & 
{\partial^2 \tau_1 \over \partial s^2} -
C^{-1}\,B_1^2\;;
\label{eqn:cs2} \\
{\partial^2 \tau \over \partial r^2}  & = & 
{\partial^2 \tau_2 \over \partial r^2} -
C^{-1}\,B_2^2\;;
\label{eqn:cr2} \\
{\partial^2 \tau \over \partial s \partial r}  & = &  
- C^{-1}\,B_1\,B_2\;.
\label{eqn:csr}
\end{eqnarray}
In the case of a constant velocity medium, expressions (\ref{eqn:cs2}) to
(\ref{eqn:csr}) can be applied directly to the explicit
equation for the two-point eikonal 
\begin{equation}
\tau_1(y,x)=\tau_2(x,y)={\sqrt{(x-y)^2+z^2(x)}\over v}\;.
\label{eqn:twopoint}
\end{equation}
Differentiating (\ref{eqn:twopoint}) and taking into account the trigonometric
relationships for the incident and reflected rays (Figure
\ref{fig:ocoray}), one can 
evaluate all the quantities in (\ref{eqn:cs2}) to (\ref{eqn:csr}) explicitly.
After some heavy algebra, the resultant expressions for the traveltime 
derivatives take the form 
\begin{eqnarray}
{\partial \tau \over \partial s}  = 
{\partial \tau_1 \over \partial s} =
{\sin{\alpha_1}\over v}& \;;\; &
{\partial \tau \over \partial r} =
{\partial \tau_2 \over \partial r} =
{\sin{\alpha_2}\over v}\;;
\label{eqn:tstr} \\
{\partial \tau_1 \over \partial x}  = 
{\sin{\gamma}\over v \cos{\alpha}} & \;;\; &
{\partial \tau_2 \over \partial x} =
- {\sin{\gamma}\over v \cos{\alpha}}\;;
\label{eqn:txx}
\end{eqnarray}
\begin{eqnarray}
B_1  & = & 
{\partial^2 \tau_1 \over \partial s\,\partial x} =
{\cos{\alpha_1}\over{v\,D\,\cos{\alpha}}}\,
\left(-1-{\sin{\gamma}\over\cos{\alpha}}\,\sin{\alpha_1}\right)\;;
\label{eqn:B1} \\ 
B_2  & = &
{\partial^2 \tau_2 \over \partial r\,\partial x} =
{\cos{\alpha_2}\over{v\,D\,\cos{\alpha}}}\,
\left(-1+{\sin{\gamma}\over\cos{\alpha}}\,\sin{\alpha_2}\right)\;;
\label{eqn:B2} 
\end{eqnarray}
\begin{equation}
B_1\,B_2  =  {\cos^6{\gamma}\over v^2\,D^2\,a^4}\;;\;
B_1+B_2 = -2\,{\cos^3{\gamma}\over v\,D\,a^2}\,\left(2\,a^2-1\right)\;;
\label{eqn:B1B2}
\end{equation}
\begin{equation}
{\partial^2 \tau_1 \over \partial x^2} =
{{\cos^2{\gamma}+D\,K}\over{v\,D\,\cos^3{\alpha}}}\,\cos{\alpha_1}\;;\;
{\partial^2 \tau_2 \over \partial x^2} =
{{\cos^2{\gamma}+D\,K}\over{v\,D\,\cos^3{\alpha}}}\,\cos{\alpha_2}\;;
\label{eqn:C1C2} 
\end{equation}
\begin{equation}
C={\partial^2 \tau_1 \over \partial x^2}+{\partial^2 \tau_2 \over \partial x^2}=
2\,\cos{\gamma}\,{{\cos^2{\gamma}+D\,K}\over{v\,D\,\cos^3{\alpha}}}\;.
\label{eqn:C}
\end{equation}
Here $D$ is the length of the normal (central) ray, $\alpha$ is its dip angle
($\alpha={{\alpha_1+\alpha_2}\over 2}$, $\tan{\alpha}=z'(x)$),
$\gamma$ is the reflection angle 
$\left(\gamma={{\alpha_2-\alpha_1}\over 2}\right)$, $K$ is the reflector 
curvature at the reflection point $\left(K=z''(x)\,\cos^3{\alpha}\right)$, and 
$a$ is the dimensionless function of $\alpha$ and $\gamma$ defined in (\ref{eqn:A}).

The equations derived in this appendix were used to obtain the equation
\begin{equation}
\tau_n\,\left({\partial^2 \tau_n \over \partial y^2}-
{\partial^2 \tau_n \over \partial h^2}\right)=
4\,\left(\tau\,{\partial^2 \tau \over \partial s\,\partial r}+
{\cos^2{\gamma}\over v^2}\right)=
4\,
{\cos^2{\gamma}\over v^2}\,\left({\sin^2{\alpha}+DK}\over
{\cos^2{\gamma}+DK}\right)\;,
\label{eqn:curved}
\end{equation}
which coincides with (\ref{eqn:curve}) in the main text.

\append{The kinematics of offset continuation}
%%%%%%%%%%%%%%%%%%%%%%%%%%%%%%%%%%%%%%%%%
\label{chapter:kinem}
This Appendix presents an alternative method to derive equation
(\ref{eqn:summation}), which describes the summation path of the
integral offset continuation operator. The method is based on the
following considerations.

The summation path of an integral (stacking) operator coincides with
the phase function of the impulse response of the inverse operator.
Impulse response is by definition the operator reaction to an impulse
in the input data. For the case of offset continuation, the input is a
reflection common-offset gather. From the physical point of view, an
impulse in this type of data corresponds to the special focusing
reflector (elliptical isochrone) at the depth. Therefore, reflection
from this reflector at a different constant offset corresponds to the
impulse response of the OC operator.  In other words, we can view
offset continuation as the result of cascading prestack common-offset
migration, which produces the elliptic surface, and common-offset
modeling (inverse migration) for different offsets.  This approach
resemble that of \cite{GPR29-03-03740406}.  It was also applied to a
more general case of azimuth moveout (AMO) by
\cite{SEG-1995-1449} and fully generalized by 
\cite{GPR48-01-01350162}.  The geometric approach implies that
in order to find the summation pass of the OC operator, one should
solve the kinematic problem of reflection from an elliptic reflector
whose focuses are in the shot and receiver locations of the output
seismic gather.

In order to solve this problem , let us consider an elliptic surface of
the general form
\begin{equation}
h(x)=\sqrt{d^2-\beta\,(x-x')^2}\;,
\label{eqn:ellips}
\end{equation}
where $0 < \beta < 1$. In a constant velocity medium, the reflection
ray path for a given source-receiver pair on the surface is controlled
by the position of the reflection point $x$.  Fermat's principle
provides a required constraint for finding this position. According to
Fermat's principle, the reflection ray path corresponds to a
stationary value of the travel-time. Therefore, in the neighborhood of
this path,
\begin{equation}
{\partial \tau(s,r,x) \over \partial x} = 0\;,
\label{eqn:fermat1}
\end{equation}
where $s$ and $r$ stand for the source and receiver locations on the
surface, and $\tau$ is the reflection traveltime
\begin{equation}
\tau(s,r,x) =   { \sqrt{h^2(x)+(s-x)^2} \over v} + 
                { \sqrt{h^2(x)+(r-x)^2} \over v}\;.
\label{eqn:length1}
\end{equation}

Substituting equations~(\ref{eqn:length1}) and (\ref{eqn:ellips}) into
(\ref{eqn:fermat1}) leads to a quadratic algebraic equation on the
reflection point parameter $x$.  This equation has the explicit
solution
\begin{equation}
x(s,r)= x' + {{\xi^2+H^2-h^2+\mbox{sign}(h^2-H^2)\,
\sqrt{\left(\xi^2-H^2-h^2\right)^2-4\,H^2\,h^2}\over
{2\,\xi\,(1-\beta)}}}\;,
\label{eqn:reflection}
\end{equation}
where $h=(r-s)/2$, $\xi = y-x'$, $y=(s+r)/2$, and $H^2=d^2\,\left({1
    \over \beta} - 1\right)$. Replacing $x$ in equation
(\ref{eqn:length1}) with its expression (\ref{eqn:reflection}) solves
the kinematic part of the problem, producing the explicit traveltime
expression
\begin{equation}
\tau(s,r)=\left\{
        \begin{array}{lcr}\displaystyle{
{1 \over v} \sqrt{{4\,h^2-\beta\,(f+g)^2} \over {1-\beta}}}
& \mbox{for} & h^2 > H^2 \\ & & \\ \displaystyle{
{1 \over v} \sqrt{{4\,h^2+\beta\,(F+G)^2} \over {1-\beta}}}
& \mbox{for} & h^2 < H^2
        \end{array}
        \right.\;, 
\label{eqn:tau}
\end{equation} 
where 
\begin{eqnarray}
f=\sqrt{(r-x')^2-H^2}\; & , & \;g=\sqrt{(s-x')^2-H^2}\;,
\nonumber \\
F=\sqrt{H^2-(r-x')^2}\; & , & \;G=\sqrt{H^2-(s-x')^2}\;.
\nonumber
\end{eqnarray}

The two branches of equation~(\ref{eqn:tau}) correspond to the
difference in the geometry of the reflected rays in two different
situations. When a source-and-receiver pair is inside the focuses of
the elliptic reflector, the midpoint $y$ and the reflection point $x$
are on the same side of the ellipse with respect to its small
semi-axis. They are on different sides in the opposite case (Figure
\ref{fig:ell}).

\inputdir{Math}

\sideplot{ell}{height=1.in}{.}{
Reflections from an ellipse. The three pairs of reflected rays
correspond to a common midpoint (at 0.1) and different offsets. The
focuses of the ellipse are at 1 and -1.
} 

If we apply the NMO correction, equation (\ref{eqn:tau}) is transformed to
\begin{equation}
\tau_n(s,r)=\left\{
        \begin{array}{lcr}\displaystyle{
{1 \over v} \sqrt{\beta \over {1-\beta}}\,
\sqrt{4\,h^2-(f+g)^2}}
& \mbox{for} & h^2 > H^2 \\ & & \\ \displaystyle{
{1 \over v} \sqrt{\beta \over {1-\beta}}\,
\sqrt{4\,h^2+(F+G)^2}}
& \mbox{for} & h^2 < H^2
        \end{array}
        \right.\;. 
\label{eqn:taun}
\end{equation}
Then, recalling the relationships between the parameters of the
focusing ellipse $r$, $x'$ and $\beta$ and the parameters of the
output seismic gather \cite[]{GPR29-03-03740406}

\begin{equation}
r={ {v\,t_n} \over 2}\;,\;x'=y\;,\;
\beta={t_n^2 \over {t_n^2+{{4\,h^2} \over v^2}}}\;,\;
H=h\;,
\label{eqn:ell2}
\end{equation}
and substituting expressions (\ref{eqn:ell2}) into equation (\ref{eqn:taun}) yields the
expression
\begin{equation}
t_1(s_1,r_1;s,r,t_n)=\left\{
        \begin{array}{lcr}\displaystyle{
{t_n \over {2\,h}}\,
\sqrt{4\,h_1^2-(f+g)^2}}
& \mbox{for} & h_1^2 > h^2 \\ & & \\ \displaystyle{
{t_2 \over {2\,h}}\,
\sqrt{4\,h_1^2+(F+G)^2}}
& \mbox{for} & h_1^2 < h^2
        \end{array}
        \right.\;, 
\label{eqn:final}
\end{equation}
where 
\begin{eqnarray}
f=\sqrt{(r_1-r)\,(r_1-s)}\;,\;g=\sqrt{(s_1-r)\,(s_1-s)}\;,
\nonumber \\
F=\sqrt{(r-r_1)\,(r_1-s)}\;,\;G=\sqrt{(s_1-r)\,(s-s_1)}\;.
\nonumber
\end{eqnarray}

It is easy to verify algebraically the mathematical equivalence of
equation (\ref{eqn:final}) and equation (\ref{eqn:summation}) in the
main text. The kinematic approach described in this appendix applies
equally well to different acquisition configurations of the input and
output data. The source-receiver parameterization used in
(\ref{eqn:final}) is the actual definition for the summation path of
the integral shot continuation operator
\cite[]{SEG-1993-0673,GEO61-06-18461858}. A family of these summation
curves is shown in Figure \ref{fig:shc}.

\plot{shc}{width=6in,height=3in}{.}{
Summation paths of the integral shot continuation. The output source
is at -0.5 km. The output receiver is at 0.5 km. The indexes of the
curves correspond to the input source location.}





%%% Local Variables: 
%%% mode: latex
%%% TeX-master: t
%%% TeX-master: t
%%% TeX-master: t
%%% End: 
