\title{Seismic reflection data interpolation with differential offset
and shot continuation}

\author{Sergey Fomel}

\maketitle

\begin{abstract}
I propose a finite-difference offset continuation filter for
  interpolating seismic reflection data. The filter is constructed
  from the offset continuation differential equation and is applied on
  frequency slices in the log-stretch frequency domain.  Synthetic and
  real data tests demonstrate that the proposed method succeeds in
  structurally complex situations where more simplistic approaches
  fail.
\end{abstract}

\section{Introduction}

Data interpolation is one of the most important problems of seismic
data processing. In 2-D exploration, the interpolation problem arises
because of missing near and far offsets, spatial aliasing and
occasional bad traces. In 3-D exploration, the importance of this
problem increases dramatically because 3-D acquisition almost never
provides a complete regular coverage in both midpoint and offset
coordinates \cite{Biondi.3dsi.99}. Data regularization in 3-D can
solve the problem of Kirchoff migration artifacts \cite{gardnerSEG},
prepare the data for wave-equation common-azimuth imaging
\cite{comaz}, or provide the spatial coverage required for 3-D
multiple elimination \cite{delft}.
\par
\longcite{Claerbout.blackwell.92,gee} formulates the following general
principle of missing data interpolation:
\begin{quote}
  A method for restoring missing data is to ensure that the restored
  data, after specified filtering, has minimum energy.
\end{quote}
How can one specify an appropriate filtering for a given interpolation
problem? Smooth surfaces are conveniently interpolated with Laplacian
filters \cite{GEO39.01.00390048}. Steering filters help us
interpolate data with predefined dip fields
\cite{clappSEG}.
Prediction-error filters in time-space or frequency-space domain
successfully interpolate data composed of distinctive plane waves
\cite{GEO56.06.07850794,gee}. Local plane waves are handled with plane-wave destruction filters \cite{pwd}.
Because prestack seismic data is not
stationary in the offset direction, non-stationary prediction-error
filters need to be estimated, which leads to an accurate but
relatively expensive method with many adjustable parameters
\cite{crawleySEG}.
\par
A simple model for reflection seismic data is a set of hyperbolic
events on a common midpoint gather. The simplest filter for this model
is the first derivative in the offset direction applied after the
normal moveout correction. Going one step beyond
this simple approximation requires taking the dip moveout (DMO) effect
into account \cite{FBR04.07.00070024}. The DMO effect is fully
incorporated in the offset continuation differential equation
\cite{me,ofcon1}.
\par
Offset continuation is a process of seismic data transformation
between different offsets
\cite{GPR29.03.03740406,GPR30.06.08130828,GPR30.06.08290849}.
Different types of DMO operators \cite{dmo} can be regarded as
continuation to zero offset and derived as solutions of an
initial-value problem with the revised offset continuation equation
\cite{ofcon1}. Within a constant-velocity assumption, this equation
not only provides correct traveltimes on the continued sections, but
also correctly transforms the corresponding wave amplitudes
\cite{SEG-1996-1731}. Integral offset continuation operators have been
derived independently by \longcite{Chemingui.sep.82.117},
\longcite{GEO61-06-18461858}, and \longcite{stovas}.  The 3-D analog
is known as azimuth moveout (AMO) \cite{GEO63-02-05740588}. In the
shot-record domain, integral offset continuation transforms to shot
continuation \cite{Schwab.sep.77.117,SEG-1993-0673,SEG-1996-0439}.
Integral continuation operators can be applied directly for missing
data interpolation and regularization
\cite{SEG-1994-1549,SEG-1999-19951998}. However, they don't behave
well for continuation at small distances in the offset space because
of limited integration apertures and, therefore, are not well suited
for interpolating neighboring records. Additionally, as all integral
(Kirchoff-type) operators they suffer from irregularities in the input
geometry. The latter problem is addressed by accurate but expensive
inversion to common offset \cite{Chemingui.sepphd.101}.
\par
In this paper, I propose an application of offset continuation in the
form of a finite-difference filter for Claerbout's method of missing
data interpolation.  The filter is designed in the log-stretch
frequency domain, where each frequency slice can be interpolated
independently.  Small filter size and easy parallelization among
different frequencies assure a high efficiency of the proposed
approach. Although the offset continuation filter lacks the predictive
power of non-stationary prediction-error filters, it is much simpler
to handle and serves as a good \emph{a priori} guess of an
interpolative filter for seismic reflection data. I first test the
proposed method by interpolating randomly missing traces in a
constant-velocity synthetic dataset. Next, I apply offset continuation
and related shot continuation field to a real data example from the
North Sea. Using a pair of offset continuation filters, operating in
two orthogonal directions, I successfully regularize a 3-D marine
dataset. These tests demonstrate that the offset continuation can
perform well in complex structural situations where more simplistic
approaches fail.

\section{Offset continuation}

A particularly efficient implementation of offset continuation results
from a log-stretch transform of the time coordinate
\cite{GPR30.06.08130828}, followed by a Fourier transform of the
stretched time axis. After these transforms, the offset continuation
equation from \cite{ofcon1} takes the form
\begin{equation}
  h \, \left( {\partial^2 \tilde{P} \over \partial y^2} - 
    {\partial^2 \tilde{P} \over \partial h^2} \right) - 
  i\,\Omega \, {\partial \tilde{P} \over   {\partial h}} = 0 \;,
  \label{eqn:OC} 
\end{equation}
where $\Omega$ is the corresponding frequency, $h$ is the half-offset,
$y$ is the midpoint, and $\tilde{P} (y,h,\Omega)$ is the transformed
data. As in other $F$-$X$ methods, equation~(\ref{eqn:OC}) can be
applied independently and in parallel on different frequency slices.
\par
We can construct an effective offset-continuation finite-difference
filter by studying first the problem of wave extrapolation between
neighboring offsets. In the frequency-wavenumber domain, the
extrapolation operator is defined by solving the initial-value problem
on equation~(\ref{eqn:OC}). The solution takes the following form
\cite{ofcon1}:
\begin{equation}
\widehat{\widehat{P}}(h_2) = \widehat{\widehat{P}}(h_1)\,
Z_{\lambda}(kh_2)/Z_{\lambda}(kh_1)\;,
\label{eqn:OKOC}
\end{equation}
where $\lambda = (1 + i \Omega)/2$, and $Z_\lambda$ is the special
function defined as
\begin{eqnarray}
\nonumber
Z_{\lambda}(x) & = & \Gamma(1-\lambda)\,\left(x \over 2\right)^{\lambda}\,
J_{-\lambda}(x)=
{}_0F_1\left(;1-\lambda;-\frac{x^2}{4}\right) \\
& = &
\sum_{n=0}^{\infty} {(-1)^n \over n!}\,
{\Gamma(1-\lambda) \over \Gamma(n+1-\lambda)}\,
\left(x \over 2\right)^{2n}\;,
\label{eqn:z}
\end{eqnarray}
where $\Gamma$ is the gamma function, $J_{-\lambda}$ is the Bessel
function, and ${}_0F_1$ is the confluent hypergeometric limit function
\cite{ab}. The wavenumber $k$ in equation~(\ref{eqn:OKOC}) corresponds
to the midpoint $y$ in the original data domain.  In the
high-frequency asymptotics, operator~(\ref{eqn:OKOC}) takes the form
\begin{equation}
\widehat{\widehat{P}}(h_2) = \widehat{\widehat{P}}(h_1)\,
F(2 k h_2/\Omega)/F(2 k h_1/\Omega)\,
\exp{\left[i\Omega\,\psi\left(2 k h_2/\Omega - 2 k h_1/\Omega\right)\right]}\;,
\label{eqn:AOKOC}
\end{equation}
where
\begin{equation}
F(\epsilon)=\sqrt{{1+\sqrt{1+\epsilon^2}} \over
{2\,\sqrt{1+\epsilon^2}}}\,
\exp\left({1-\sqrt{1+\epsilon^2}} \over 2\right)\;,
\label{eqn:F}
\end{equation}
and
\begin{equation}
\psi(\epsilon)={1 \over 2}\,\left(1 - \sqrt{1+\epsilon^2} +
\ln\left({1 + \sqrt{1+\epsilon^2}} \over 2\right)\right)\;.
\label{eqn:psi}
\end{equation}

Returning to the original domain, we can approximate the continuation
operator with a finite-difference filter with the $Z$-transform
\begin{equation}
\label{eqn:OCpass}
\hat{P}_{h+1}(Z_y) = \hat{P}_{h} (Z_y) \frac{G_1(Z_y)}{G_2(Z_y)}\;.
\end{equation}
The coefficients of the filters $G_1(Z_y)$ and $G_2(Z_y)$ are found by
fitting the Taylor series coefficients of the filter response around
the zero wavenumber.  In the simplest case of 3-point
filters\footnote{An analogous technique applied to the case of
  wavefield depth extrapolation with the wave equation would lead to
  the famous 45-degree implicit finite-difference operator
  \cite{Claerbout.blackwell.85}.}, this procedure uses four Taylor
series coefficients and leads to the following expressions:
\begin{eqnarray}
  \label{eqn:OCnum}
  G_1(Z_y) & = & 1 - \frac{1 - c_1(\Omega) h_2^2 + c_2(\Omega) h_1^2}{6} +
  \frac{1 - c_1(\Omega) h_2^2 + c_2(\Omega) h_1^2}{12}\,
  \left(Z_y + Z_y^{-1}\right)\;, \\
  \label{eqn:OCden}
  G_2(Z_y) & = & 1 - \frac{1 - c_1(\Omega) h_1^2 + c_2(\Omega) h_2^2}{6} +
  \frac{1 - c_1(\Omega) h_1^2 + c_2(\Omega) h_2^2}{12}\,
  \left(Z_y + Z_y^{-1}\right)\;,
\end{eqnarray}
where 
\[
c_1(\Omega) = \frac{3\,(\Omega^2 + 9 - 4
  i\,\Omega)}{\Omega^2\,(3+i\,\Omega)}
\]
and 
\[
c_2(\Omega) =
\frac{3\,(\Omega^2 - 27 - 8 i\,\Omega)}{\Omega^2\,(3+i\,\Omega)}\;.
\]
Figure~\ref{fig:arg} compares the phase characteristic of the
finite-difference extrapolators~(\ref{eqn:OCpass}) with the phase
characteristics of the exact operator~(\ref{eqn:OKOC}) and the
asymptotic operator~(\ref{eqn:AOKOC}). The match between different
phases is poor for very low frequencies (left plot in
Figure~\ref{fig:arg}) but sufficiently accurate for frequencies in the
typical bandwidth of seismic data (right plot in
Figure~\ref{fig:arg}).

Figure~\ref{fig:off-imp} compares impulse responses of the inverse DMO
operator constructed by the asymptotic $\Omega-k$ operator with those
constructed by finite-difference offset continuation. Neglecting
subtle phase inaccuracies at large dips, the two images look similar,
which provides an experimental evidence of the accuracy of the
proposed finite-difference scheme.

When applied on the offset-midpoint plane of an individual frequency
slice, the one-dimensional implicit filter~(\ref{eqn:OCpass})
transforms to a two-dimensional explicit filter with the
2-D $Z$-transform 
\begin{equation}
\label{eqn:gfilt}
G(Z_y,Z_h) = G_1(Z_y) - Z_h G_2(Z_y)\;.
\end{equation}
Convolution with filter~(\ref{eqn:gfilt}) is the regularization
operator that I propose to use for interpolating prestack seismic data.

%I propose to adopt a finite-difference form of the differential
%operator~(\ref{eqn:OC}) for the regularization operator $\bold{D}$. A
%simple analysis of equation~(\ref{eqn:OC}) shows that at small
%frequencies, the operator is dominated by the first term. The form
%${\partial^2 P \over \partial y^2} - {\partial^2 P \over \partial
%  h^2}$ is equivalent to the second mixed derivative in the source and
%receiver coordinates. Therefore, at low frequencies, the offset waves
%propagate in the source and receiver directions. At high frequencies,
%the second term in~(\ref{eqn:OC}) becomes dominating, and the entire
%method becomes equivalent to the trivial linear interpolation in
%offset. The interpolation pattern is more complicated at intermediate
%frequencies.
  
\inputdir{Math}

\plot{arg}{width=6in,bb=36 298 828 566}{Phase of the implicit
  offset-continuation operators in comparison with the exact solution.
  The offset increment is assumed to be equal to the midpoint spacing.
  The left plot corresponds to $\Omega=1$, the right plot to
  $\Omega=10$.}

\inputdir{ocimp}

\plot{off-imp}{width=6in,height=3.5in}{Inverse DMO impulse responses
  computed by the Fourier method (left) and by finite-difference
  offset continuation (right). The offset is 1~km.}

\section{Application}
I start numerical testing of the proposed regularization first on a
constant-velocity synthetic, where all the assumptions behind the
offset continuation equation are valid.  
%
%Encouraged by the results, I
%proceed to tests on the Marmousi synthetic dataset, which is
%associated with a highly inhomogeneous velocity model.

\subsection{Constant-velocity synthetic}

\inputdir{cup}

\sideplot{cup}{width=3in}{Reflector model for the constant-velocity test}

A sinusoidal reflector shown in Figure~\ref{fig:cup} creates
complicated reflection data, shown in Figures~\ref{fig:data}
and~\ref{fig:tslice}. To set up a test for regularization by offset
continuation, I removed 90\% of randomly selected shot gathers from
the input data.  The syncline parts of the reflector lead to
traveltime triplications at large offsets. A mixture of different dips
from the triplications would make it extremely difficult to
interpolate the data in individual common-offset gathers, such as
those shown in Figure~\ref{fig:data}.  The plots of time slices
after NMO (Figure~\ref{fig:tslice}) clearly show that the data are
also non-stationary in the offset direction.  Therefore, a simple
offset interpolation scheme is also doomed.

\plot{data}{width=5.5in,height=7.33in}{Prestack common-offset gathers for
  the constant-velocity test. Left: ideal data (after NMO). Right:
  input data (90\% of shot gathers removed).  Top, center, and
  bottom plots correspond to different offsets.}

\plot{tslice}{width=5.5in,height=7.33in}{Time slices of the prestack
  data for the constant-velocity test. Left: ideal data (after NMO).
  Right: input data (90\% of random gathers removed).  Top, center,
  and bottom plots correspond to time slices at 0.3, 0.4, and 0.5 s.}

\par
Figure~\ref{fig:fslice} shows the reconstruction process on individual
frequency slices. Despite the complex and non-stationary character of
the reflection events in the frequency domain, the offset continuation
equation is able to accurately reconstruct them from the decimated
data.

\plot{fslice}{width=5.5in,height=7.33in}{Interpolation in frequency
  slices.  Left: input data (90\% of the shot gathers removed). Right:
  interpolation output. Top, bottom, and middle plots correspond to
  different frequencies. The real parts of the complex-valued data are
  shown.}

\par
Figure~\ref{fig:all} shows the result of interpolation after the data
are transformed back to the time domain. The offset continuation
result (right plots in Figure~\ref{fig:all}) reconstructs the ideal
data (left plots in Figure~\ref{fig:data}) very accurately even in
the complex triplication zones, while the result of simple offset
interpolation (left plots in Figure~\ref{fig:all}) fails as expected.
The simple interpolation scheme applied the offset derivative 
$\frac{\partial}{\partial h}$
in place of the offset continuation equation and thus did not take
into account the movement of the events across different midpoints.

\plot{all}{width=5.5in,height=7.33in}{Interpolation in common-offset
  gathers.  Left: output of simple offset interpolation.  Right:
  output of offset continuation interpolation. Compare with
  Figure~\ref{fig:data}. Top, center, and bottom plots correspond
  to different common-offset gathers.}

\par
The constant-velocity test results allow us to conclude that, when all
the assumptions of the offset continuation theory are met, it provides
a powerful method of data regularization. 

Being encouraged by the synthetic results, I proceed to a
three-dimensional real data test.

\begin{comment}

\subsection{3-D data regularization with the offset continuation equation}

3-D differential offset continuation amounts to applying two
differential filters, operating on the in-line and cross-line
projections of the offset and midpoint coordinates. The corresponding
system of differential equations has the form
\begin{equation} 
\displaystyle
  \left\{\begin{array}{rcl} 
      \displaystyle
      h_1 \, \left( {\partial^2 \tilde{P} \over \partial y_1^2} - 
        {\partial^2 \tilde{P} \over \partial h_1^2} \right) - 
      i\,\Omega \, {\partial \tilde{P} \over   {\partial h_1}} & = & 0\;; \\
      \displaystyle
      h_2 \, \left( {\partial^2 \tilde{P} \over \partial y_2^2} - 
        {\partial^2 \tilde{P} \over \partial h_2^2} \right) - 
      i\,\Omega \, {\partial \tilde{P} \over   {\partial h_2}} & = & 0\;,
    \end{array}\right.
  \label{eqn:OC-3D} 
\end{equation}
where $y_1$ and $y_2$ correspond to the in-line and cross-line
midpoint coordinates, and $h_1$ and $h_2$ correspond to the in-line
and cross-line offsets.  The projection approach is justified in the
theory of azimuth moveout \cite{amo,GEO63-02-05740588}.
 
The result of a 3-D data regularization test is shown in
Figure~\ref{fig:off4}. The input data is a subset of a 3-D marine
dataset from the North Sea, complicated by salt dome reflections and
diffractions. The same dataset was used previously for testing azimuth
moveout \cite{GEO63-02-05740588}. I used neighboring offsets in the
in-line and cross-line directions and the differential 3-D offset
continuation to reconstruct the empty traces in a selected midpoint
cube. Although the reconstruction is not entirely accurate, it
successfully fulfills the following goals:
\begin{itemize}
\item The input traces are well hidden in the interpolation result. It
  is impossible to distinguish between input and interpolated traces.
\item The main structural features are restored without using any
  assumptions about structural continuity in the midpoint domain. Only
  the physical offset continuity is used.
\end{itemize}

%\plot{off4}{width=5.5in,height=7.33in}{3-D data regularization test.
%  Top: input data, the result of binning in a 50 by 50 meters offset
%  window. Bottom: regularization output. Data from neighboring offset
%  bins in the in-line and cross-line directions were used to
%  reconstruct missing traces.}

\end{comment}

%In the next section, I
%deal with the much more complicated case of Marmousi.

%%\subsection{Marmousi synthetic}

%%The famous Marmousi synthetic was modeled over a very complicated
%%velocity and reflector structure \cite{TLE13-09-09270936}. The dataset
%%has been used in numerous studies of various seismic processing and
%%imaging techniques. Figure~\ref{fig:marm} shows the near and far
%%common-offset gathers from the Marmousi dataset. The structure of the
%%reflection events is extremely complex and contains multiple
%%triplications and diffractions.

%%\plot{marm}{width=6in}{Common-offset gathers of the Marmousi dataset.
%%  Left: near offset. Right: far offset.}

%%\par
%%To test the proposed interpolation method, I set the goal of
%%interpolating the missing near offsets in the Marmousi dataset.
%%Additionally, I attempted to interpolate intermediate shot gathers so
%%that all common-midpoint gathers receive the same offset fold. In the
%%original dataset, both receiver and shot spacing are equal to 25
%%meters, which creates a checkerboard pattern in the offset-midpoint
%%plane. This acquisition pattern is typical for 2-D seismic surveys.
%%\par
%%Interpolation of near offsets can reduce imaging artifacts in
%%different migration methods. \longcite{Ji.sepphd.90} used near-offset
%%interpolation for accurate wavefront-synthesis migration of the
%%Marmousi dataset. He developed an interpolation technique based on
%%the hyperbolic Radon transform inversion. Ji's method produces fairly
%%good results, but is significantly more expensive that the offset
%%continuation approach explored in this paper.
%%\par
%%Figure~\ref{fig:mslice} shows the input and interpolated Marmousi data
%%in the log-stretch frequency domain. We can see that the data in the
%%frequency slices also have a very complicated structure. Nevertheless,
%%the offset continuation method is able to reconstruct the missing
%%portions of the data in a visually pleasing way. The data are not
%%extrapolated off the sides of the common-offset gathers. This behavior
%%is physically reasonable, because such an extrapolation would involve
%%assumptions about unilluminated portions of the subsurface.

%%%\plot{mslice}{width=6in,height=8in}{Interpolation of the Marmousi
%%%  dataset in frequency slices.  Left: input data. Right: interpolation
%%%  output.  Top, center, and bottom plots correspond to different
%%%  frequencies.  Real parts of the complex-valued data are shown.}

%%\par
%%Figure~\ref{fig:mshot} shows one of the shot gathers obtained after
%%transforming the data back into time domain and resorting them into
%%shot gathers. The positive offset part of the shot gather was
%%reconstructed from a common receiver gather by using reciprocity.
%%Comparing the top and bottom plots, we can see that many different
%%events in the original shot gather are nicely extended into near
%%offsets by the interpolation procedure.

%%%\plot{mshot}{width=6in,height=8in}{Interpolation of near offsets in a
%%%  Marmousi shot gather. The shot position is 4500 m.  Top: input data.
%%%  Bottom: interpolation output.}

%%\par
%%In addition to interpolating near offsets, I have reconstructed the
%%intermediate shot gathers in order to equalize the CMP fold.
%%Figure~\ref{fig:mishot} shows an example of an artificial shot gather
%%created by such a reconstruction. An sample CMP gather before and
%%after interpolation is shown in Figure~\ref{fig:mcmp}. Examining the
%%bottom part of the section, we can see that that the interpolation
%%process tends to put more continuity in the near offsets than could be
%%expected from the data. In other places, the interpolation succeeds in
%%producing a visually pleasant result.

%%%\plot{mishot}{width=6in}{This shot gather at 4525 m is a result of
%%%  data interpolation.}

%%%\plot{mcmp}{width=6in,height=8in}{Interpolation of near and
%%%  intermediate offsets in a Marmousi CMP gather. The midpoint position
%%%  is 4500 m. Top: input data.  Bottom: interpolation output.}

%\section{Discussion}

% \par
%In the range of possible interpolation methods \cite{EAE-1998-2051},
%the offset continuation approach clearly stands on the more efficient
%side. The efficiency is achieved both by the small size of the
%finite-difference filter and by the method's ability to decompose and
%parallelize the method across different frequencies. Part of the
%efficiency gain could probably be sacrificed for achieving more
%accurate results.  Here are some interesting ideas one could try:
%\begin{itemize}
%\item Instead of fixing the offset continuation filter in a
%  data-independent way, one could estimate some of its coefficients
%  from the data. In particular, the second term in
%  equation~(\ref{eqn:OC}) can be varied to better account for the
%  effects of variable velocity and amplitude variation with offset.
%  Theoretical extensions of offset continuation to the variable
%  velocity case were studied by \longcite{SEG-1997-1901} and
%  \longcite{SEG-1999-19331936}.

%\item Formulating the problem in the pre-NMO domain would allow us to
%  consider several velocities by convolving several continuation
%  filters. This could be an appropriate approach for interpolating
%  both primary and multiple reflections.
  
%\item Missing data interpolation problems can be greatly accelerated
%  by preconditioning \cite{Fomel.sep.94.sergey1,Fomel.sep.95.sergey1}.
%  Finding an appropriate preconditioning for the offset continuation
%  method is an open problem.  The non-stationary nature of the
%  continuation filter make this problem particularly challenging.

%\end{itemize}

Analogously to integral azimuth moveout operator
\cite{GEO63-02-05740588}, differential offset continuation can be
applied in 3-D for regularizing seismic data prior to prestack imaging.

  
  In the next section, I return to the 2-D case to consider an
  important problem of shot gather interpolation.

  \section{Shot continuation}
  
  Missing or under-sampled shot records are a common example of data
  irregularity \cite{Crawley.sepphd.104}. The offset continuation
  approach can be easily modified to work in the shot record domain.
  With the change of variables $s = y - h$, where $s$ is the shot
  location, the frequency-domain equation~(\ref{eqn:OC}) transforms to
  the equation
  \begin{equation}
    h \, \left( 2\,{\partial^2 \tilde{P} \over \partial s \partial h} - 
      {\partial^2 \tilde{P} \over \partial h^2} \right) - 
    i\,\Omega \, \left({\partial \tilde{P} \over   {\partial h}} -
        {\partial \tilde{P} \over   {\partial s}}\right) = 0 \;.
    \label{eqn:SC} 
  \end{equation}
  Unlike equation~(\ref{eqn:OC}), which is second-order in the
  propagation variable $h$, equation~(\ref{eqn:SC}) contains only
  first-order derivatives in $s$. We can formally write its solution
  for the initial conditions at $s=s_1$ in the form of a phase-shift
  operator:
    \begin{equation}     
      \widehat{\widehat{P}}(s_2) = \widehat{\widehat{P}}(s_1)\,
      \exp{\left[i\,k_h\,\left(s_2-s_1\right)\,
          \frac{k_h\,h-\Omega}{2\,k_h\,h-\Omega}\right]}\;,
    \label{eqn:SCshift} 
  \end{equation}
  where the wavenumber~$k_h$ corresponds to the half-offset~$h$.
  Equation~(\ref{eqn:SCshift}) is in the mixed offset-wavenumber
  domain and, therefore, not directly applicable in practice. However,
  we can use it as an intermediate step in designing a
  finite-difference shot continuation filter. Analogously to the cases
  of plane-wave destruction and offset continuation, shot continuation
  leads us to the rational filter
  \begin{equation}
    \label{eqn:SCpass}
    \hat{P}_{s+1}(Z_h) = 
    \hat{P}_{s} (Z_h) \frac{S(Z_h)}{\bar{S}(1/Z_h)}\;,
  \end{equation}
  The filter is non-stationary, because the coefficients of $S(Z_h)$
  depend on the half-offset $h$. We can find them by the Taylor
  expansion of the phase-shift equation~(\ref{eqn:SCshift}) around
  zero wavenumber $k_h$. For the case of the half-offset sampling
  equal to the shot sampling, the simplest three-point filter is
  constructed with three terms of the Taylor expansion. It takes the
  form
  \begin{equation}
    \label{eqn:SCfilt}
    S(Z_h) = - \left(\frac{1}{12} + i\,\frac{h}{2\,\Omega}\right)\,Z_h^{-1} + 
    \left(\frac{2}{3} - i\,\frac{\Omega^2 + 12\,h^2}{12\,\Omega\,h}\right) +
    \left(\frac{5}{12} + i\,\frac{\Omega^2 + 18\,h^2}{12\,\Omega\,h}\right)\,
    Z_h\;.
  \end{equation}
  
  Let us consider the problem of doubling the shot density.  If we use
  two neighboring shot records to find the missing record between
  them, the problem reduces to the least-squares system
  \begin{equation}
    \label{eqn:SCls}
    \left[\begin{array}{c}
        \bold{S} \\
        \bold{\bar{S}}
      \end{array}\right]\,\bold{p}_s \approx
    \left[\begin{array}{c}
        \bold{\bar{S}}\,\bold{p}_{s-1} \\
        \bold{S}\,\bold{p}_{s+1}
      \end{array}\right]\;,
  \end{equation}
  where $\bold{S}$ denotes convolution with the numerator of
  equation~(\ref{eqn:SCpass}), $\bold{\bar{S}}$ denotes convolution
  with the corresponding denominator, $\bold{p}_{s-1}$ and
  $\bold{p}_{s+1}$ represent the known shot gathers, and $\bold{p}_s$
  represents the gather that we want to estimate. The least-squares
  solution of system~(\ref{eqn:SCls}) takes the form
  \begin{equation}
    \label{eqn:SCsol}
    \bold{p}_s = \left(
      \bold{S}^T\,\bold{S} +
      \bold{\bar{S}}^T\,\bold{\bar{S}}
    \right)^{-1}\,
    \left(\bold{S}^T\,\bold{\bar{S}}\,\bold{p}_{s-1} +
      \bold{\bar{S}}^T\,\bold{S}\,\bold{p}_{s+1}\right)\;.
  \end{equation}
  If we choose the three-point filter~(\ref{eqn:SCfilt}) to construct
  the operators~$\bold{S}$ and $\bold{\bar{S}}$, then the inverted
  matrix in equation~(\ref{eqn:SCsol}) will have five non-zero
  diagonals. It can be efficiently inverted with a direct banded
  matrix solver using the $LDL^T$ decomposition \cite{golub}. Since
  the matrix does not depend on the shot location, we can perform the
  decomposition once for every frequency so that only a triangular
  matrix inversion will be needed for interpolating each new shot.
  This leads to an extremely efficient algorithm for interpolating
  intermediate shot records.

  Sometimes, two neighboring shot gathers do not fully constrain the
  intermediate shot. In order to add an additional constraint, I
  include a regularization term in equation~(\ref{eqn:SCsol}), as
  follows:
  \begin{equation}
    \label{eqn:SCreg}
    \bold{p}_s = \left(
      \bold{S}^T\,\bold{S} +
      \bold{\bar{S}}^T\,\bold{\bar{S}} + 
        \epsilon^2\,\bold{A}^T\bold{A}
    \right)^{-1}\,
    \left(\bold{S}^T\,\bold{\bar{S}}\,\bold{p}_{s-1} +
      \bold{\bar{S}}^T\,\bold{S}\,\bold{p}_{s+1}\right)\;,
  \end{equation}
  where $\bold{A}$ represents convolution with a three-point
  prediction-error filter (PEF), and $\epsilon$ is a scaling
  coefficient. The appropriate PEF can be estimated from
  $\bold{p}_{s-1}$ and $\bold{p}_{s+1}$ using Burg's algorithm
  \cite{GEO37.02.03750376,Burg.sepphd.6,Claerbout.fgdp.76}. A
  three-point filter does not break the five-diagonal structure
  of the inverted matrix.  The PEF regularization attempts to preserve
  offset dip spectrum in the under-constrained parts of the estimated
  shot gather.
  
  Figure~\ref{fig:shotin} shows the result of a shot interpolation
  experiment using the constant-velocity synthetic from
  Figure~\ref{fig:data}. In this experiment, I removed one of the
  shot gathers from the original NMO-corrected data and interpolated
  it back using equation~(\ref{eqn:SCreg}). Subtracting the true shot
  gather from the reconstructed one shows a very insignificant error,
  which is further reduced by using the PEF regularization (right
  plots in Figure~\ref{fig:shotin}).  The two neighboring shot gathers
  used in this experiment are shown in the top plots of
  Figure~\ref{fig:shot3}.  For comparison, the bottom plots in
  Figure~\ref{fig:shot3} show the simple average of the two shot
  gathers and its corresponding prediction error. As expected, the
  error is significantly larger than the error of shot
  continuation. An interpolation scheme based on local dips in the
  shot direction would probably achieve a better result, but it is
  significantly more expensive than the shot continuation scheme
  introduced above.
  
  \plot{shot3}{width=6in,height=7in}{Top: Two synthetic shot gathers
    used for the shot interpolation experiment. An NMO correction has
    been applied. Bottom: simple average of the two shot gathers
    (left) and its prediction error (right).}
  \plot{shotin}{width=6in,height=7in}{Synthetic shot interpolation
    results. Left: interpolated shot gathers. Right: prediction errors
    (the differences between interpolated and true shot gathers),
    plotted on the same scale. Top: without regularization. Bottom:
    with PEF regularization.}
 
 
  A similar experiment with real data from a North Sea marine dataset
  is reported in Figure~\ref{fig:elfshotin}. I removed and
  reconstructed a shot gather from the two neighboring gathers shown
  in Figure~\ref{fig:elfshot3}. The lower parts of the gathers are
  complicated by salt dome reflections and diffractions with
  conflicting dips. The simple average of the two input shot gathers
  (bottom plots in Figure~\ref{fig:elfshotin}) works reasonably well
  for nearly flat reflection events but fails to predict the position
  of the back-scattered diffractions events. The shot continuation
  method works well for both types of events (top plots in
  Figure~\ref{fig:elfshotin}). There is some small and random residual
  error, possibly caused by local amplitude variations.
   
  \plot{elfshot3}{width=6in,height=3.5in}{Two real marine shot gathers
    used for the shot interpolation experiment. An NMO correction has
    been applied.}  
  
  \plot{elfshotin}{width=6in,height=7in}{Real-data shot interpolation
    results.  Top: interpolated shot gather (left) and its prediction
    error (right). Bottom: simple average of the two input shot
    gathers (left) and its prediction error (right).}
  
  Analogously to the case of offset continuation, it is possible to
  extend the shot continuation method to three dimensions. A simple
  modification of the proposed technique would also allow us to use
  more than two shot gathers in the input or to extrapolate missing
  shot gathers at the end of survey lines.

%  \section{Stack and partial stack as a data regularization problem}

\section{Conclusions}

Differential offset continuation provides a valuable tool for
interpolation and regularization of seismic data. Starting from
analytical frequency-domain solutions of the offset continuation
differential equation, I have designed accurate finite-difference
filters for implementing offset continuation as a local convolutional
operator. A similar technique works for shot continuation across
different shot gathers. Missing data are efficiently interpolated by
an iterative least-squares optimization. The differential filters have
an optimally small size, which assures high efficiency.

Differential offset continuation serves as a bridge between integral
and convolutional approaches to data interpolation. It shares the
theoretical grounds with the integral approach but is applied in a
manner similar to that of prediction-error filters in the
convolutional approach.

Tests with synthetic and real data demonstrate that the proposed
interpolation method can succeed in complex structural situations
where more simplistic methods fail.

\section{Acknowledgments}

The financial support for this work was provided by the sponsors of
the Stanford Exploration Project (SEP).

%The 3-D North Sea dataset was released to SEP by Conoco and its
%partners, BP and Mobil. 

For the shot continuation test, I used a
North Sea dataset, released to SEP by Elf Aquitaine.

I thank Jon Claerbout and Biondo Biondi for helpful discussions about
the practical application of differential offset continuation.

\newpage

\bibliographystyle{sep}
\bibliography{MISC,SEP,EAEG,SEGCON,GEOTLE,SEGBKS,GEOPHYSICS,sergey}

%%% Local Variables: 
%%% mode: latex
%%% TeX-master: t
%%% End: 
