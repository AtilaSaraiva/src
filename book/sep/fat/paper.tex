\lefthead{Bevc}
\righthead{Imaging complex structures}
\footer{SEP--84}

\published{SEP Report, 84, 1-10 (1995)}
\title{Imaging complex structures with first-arrival traveltimes}
\email{dimitri@sep.stanford.edu}
\author{Dimitri Bevc}
%\chapter{Imaging complex structures with layer stripping depth migration}
%\CHAPLABEL{Layer}
\def\figdir{./Fig}

\begin{abstract}
I present a layer-stripping Kirchhoff migration algorithm which is
capable of obtaining accurate
images of complex structures by downward continuing
the data and imaging from a lower datum.
%I obtain accurate images of complex structures by
%downward continuing the data and imaging from a lower datum.
I use eikonal traveltimes in a Kirchhoff datuming algorithm for the
downward continuation. After downward continuation, I perform
Kirchhoff migration. 
The method alternates steps of datuming and imaging. Because traveltimes
are computed for each step, the adverse effects of caustics, headwaves, and 
multiple arrivals do not develop.
In principal, this method only requires the same number of
traveltime calculations as a standard migration.
Tests on the Marmousi data set produce excellent results.
\end{abstract}

\section{Introduction}
Kirchhoff migration is generally accepted to be the most efficient
method of imaging 2-D and 3-D prestack seismic data.  The Marmousi
synthetic data set \cite[]{TLE13-09-09270936} has been a popular
testbed for migration algorithms and many researchers have discovered
that Kirchhoff algorithms using first-arrival traveltimes do a poor
job of imaging the target zone
\cite[]{Audebert.sep.80.47,GEO59-05-08100817,GEO58-04-05640575}.  Even
methods which calculate most energetic arrivals and estimate amplitude
and phase do not always result in images which compare favorably with
finite-difference shot-profile migration.

In their 1993 Geophysics article, Geoltrain and Brac ask the
question {\em ``Can we image complex structures with first-arrival
traveltime?''} They conclude that they cannot, and that they should 
either ray trace to find the most energetic arrivals, or calculate
multiple-arrival Green's functions. \cite{Nichols.sepphd.81} calculates 
band-limited Green's functions to estimate the most energetic
arrivals. He estimates not only traveltime, but also amplitude and
phase. My approach is simpler; by breaking up the complex
velocity structure, I am able 
to calculate traveltimes in velocity models where
finite-differencing the eikonal equation is valid.
This results in images comparable to those obtained by 
Nichols' method and by shot-profile migration at a reduced
computational cost.

Like most of the other researchers in the field, I test my method on the
ever-popular Marmousi synthetic.

\section{TRAVELTIME CALCULATION}
\inputdir{green}

Green's functions based on 
first-arrival traveltime calculation methods result in poor images
in structurally complex areas. Several reasons have been given for this
failure:
\begin{itemize}
	\item The high-frequency approximation of ray and eikonal methods breaks
	down in complex velocity models. In rapidly varying velocity models,
	different frequency components of the
	wavefield propagate at different velocities, so summation trajectories
	based on only the high-frequency components may not capture
	the desired events.
	\item When high velocity zones are present, the first-arrivals
	may be non-energetic headwaves.
	\item As energy propagates in complex models, raypaths tend to
	eventually cross. This causes phase shifts and triplications.
	First-arrival traveltimes follow the fastest branch of the
	triplication bow-tie, which is also the low energy branch.
\end{itemize}

Complexity of velocity models and validity of high-frequency approximations
can be defined in various ways. The larger the velocity model, the 
more variation there will tend to be and the more opportunity there will be
for things to go wrong: There will be more opportunity for frequency
components to separate, for headwaves to develop, and for triplications
to occur. 

\plot{velfig}{width=\textwidth}{The Marmousi velocity model.}

The Marmousi velocity model (Figure~\ref{fig:velfig}) results in
complex propagation paths where late energetic arrivals are not fit well 
by first-arrival finite-difference traveltimes.
In Figure~\ref{fig:snap0},
an acoustic modeling program was used to generate snapshots of 
the wavefield from two surface source locations in the Marmousi model.
The corresponding
contours of finite-difference traveltime at 1.05 s have been overlain.
Clearly, the finite-difference traveltime contours do not always correspond
to energetic portions of the wavefield. If these traveltimes were
used for migration, the resulting image would suffer because 
parts of the summation trajectories would not correspond to 
energetic arrivals. By contrast, snapshots for the same source locations
at an earlier time of 0.6 s (Figures~\ref{fig:snapdatum}a and \ref{fig:snapdatum}b)
show that the finite-difference traveltime curves overlay the
high energy portions of the wavefields nicely. This is because there 
has not been enough time for adverse propagation effects to fully develop.
Figures~\ref{fig:snapdatum}c and \ref{fig:snapdatum}d are generated by
starting the acoustic modeling and the finite-difference traveltime
calculation from a depth of 1500 m. The 0.3 s contours correspond nicely
to the high energy portions of the wavefields. There is some deviation
in the shallow part of Figure~\ref{fig:snapdatum}c, but for the most part
the finite-difference traveltime contour fits the bulk of the acoustic 
energy pretty well. Overall, the contours in Figure~\ref{fig:snapdatum}
have not pulled away from the wavefield as they have in Figure~\ref{fig:snap0}.

\plot{snap0}{width=\textwidth}{The result of acoustic wavefield modeling for the Marmousi model overlain
by contours of finite-difference traveltime. Snapshots (a) and (b) are
taken at a time of 1.05 s for two  different source locations.}

\plot{snapdatum}{width=\textwidth}{The result of acoustic wavefield modeling for the Marmousi model overlain
by contours of finite-difference traveltime. Snapshots (a) and (b) are
taken at a time of 0.6 s for the same two source locations depicted in
Figure \protect \ref{fig:snap0}. Snapshots (c) and (d) are taken at a
time of 0.3 s with sources at a depth of 1500 m and the same lateral
positions as in (a) and (b).}

These properties of first-arrival traveltimes, and the observation that
Kirchhoff migration using first-arrival traveltimes usually produces an
acceptable image of the upper 1500 m of the Marmousi synthetic, 
motivated the development of the layer-striping method described here.

Throughout this study, I use a finite difference solution to the
eikonal equation to generate traveltimes \cite[]{GEO56-06-08120821}.
I use my Kirchhoff datuming algorithm \cite[]{Bevc.sep.77.131}, and an
industry standard Kirchhoff migration code to generate the results.

\section{LAYER-STRIPPING KIRCHHOFF MIGRATION}
\inputdir{.}

The layer-stripping migration method can be thought of as a hybrid 
algorithm that incorporates some of the advantages of shot-profile
migration with the efficiency of Kirchhoff migration.
This is illustrated by Figures~\ref{fig:s-g} through~\ref{fig:datum}.

In shot-profile migration, shots and geophones are
alternately downward continued through each depth level.
Figure~\ref{fig:s-g} illustrates this for one line of shots or geophones.  
It is evident that there are many propagation paths from the surface
to the image point, therefore multiple arrivals are handled.
The computation is performed for all frequencies.

By contrast, first-arrival Kirchhoff migration is performed 
by summing data over trajectories defined by the propagation paths
illustrated in Figure~\ref{fig:kirch}. There is only one path linking
each surface position to the image point, therefore multiple arrivals are
not handled. Although crossing paths are not illustrated here, they can occur.

The layer-stripping method is illustrated in Figure~\ref{fig:datum}.
It is a combination of Kirchhoff wave-equation datuming \cite[]{GEO44-08-13291344,GEO49-11-20642067} 
and Kirchhoff migration. 
The computation proceeds as follows:
\begin{enumerate}
	\item Migrate from surface to some depth level $z_1$.
	\item Downward continue to depth level $z_1$.
	\item Migrate from depth level $z_1$.
	\item Downward continue to depth level $z_2$.
	\item Migrate from depth level $z_2$.
	\item etc...
\end{enumerate}
The depth step of downward continuation is much larger than the
$\triangle z$ used in shot-profile migration. Since
there are multiple paths from the surface to the image point,
multiple arrivals are handled. Because the paths are shorter than
in Figure~\ref{fig:kirch}, first-arrival traveltimes are more likely to 
be valid.

\sideplot{s-g}{width=3.0in}{Propagation paths from the surface to an image point for shot 
profile migration.}

\sideplot{kirch}{width=3.0in}{Propagation paths from the surface to an image point for Kirchhoff downward
continuation.}

\sideplot{datum}{width=3.0in}{Propagation paths from the surface to a depth point for layer-stripping
Kirchhoff downward continuation.}

\section{MARMOUSI EXAMPLE}
\inputdir{migration}

Figure~\ref{fig:mig0} is the result of
an industry standard Kirchhoff migration
of the Marmousi synthetic using eikonal traveltimes. The upper
portion is well imaged; however, the anticlinal structure below 2200 m and
the target zone at a lateral position of about 6500 m 
and depth of 2500 m is not imaged.

Figure~\ref{fig:mig1} is generated by downward continuing the data
to a depth of 1500 m in one datuming step. The downward
continued data are then migrated and combined with the previous image of
the upper 2000 m. The anticlinal structure and the target are now clearly
imaged.

Continuing the data to 1500 m in three steps of 500 m each, results
in an even crisper
image of the anticline and the target (Figure~\ref{fig:mig2}).
In both of these images,
the events which unconformably define the top of the anticline,
the anticline events themselves, and the target events, are clearly
imaged.

\plot{mig0}{height=3.0in,width=6.0in}
{Standard Kirchhoff migration using eikonal traveltimes.}

\plot{mig1}{height=3.0in,width=6.0in}
{Migrated image using traveltimes calculated from the surface, and
traveltimes calculated from a depth of 1500 m.
The lower part of the image was obtained by migrating data which
was redatumed to a depth of 1500 m in one step
of downward continuation.}

\plot{mig2}{height=3.0in,width=6.0in}
{Migrated image using traveltimes calculated from the surface, and 
traveltimes calculated from a depth of 1500 m.
The lower part of the image was obtained by migrating data which
was redatumed to a depth of 1500 m in three steps of 500 m each.}

In Figure~\ref{fig:targref}, I compare the images in the vicinity
of the target zone to the velocity model and a filtered reflectivity
model which represents the desired image.
Both images compare favorably to the desired reflectivity.
The image obtained by downward continuing the data in three steps
of 500 m is superior since the events display 
better lateral continuity and the image is clearer.
This is because the traveltimes calculated for each of the 500 m steps
are simpler and better behaved than the traveltimes calculated
for one step of 1500 m.

%\TActiveplot{target}{height=6.0in,width=6.0in}{.}
%{Comparison of images in the target zone for standard migration.}
%{Comparison of images in the target zone for standard migration,
%downward continuation to 1000 m followed by migration, and 
%downward continuation to 1500 m followed by migration.}

\plot{targref}{height=6.0in,width=6.0in}
{Comparison of the velocity, the reflectivity, and the images in the
target zone.}

\section{CONCLUSION}
I obtain images comparable to shot-profile migration results by 
combining wave-equation datuming and Kirchhoff migration into
a layer-stripping migration method. In this case, eikonal traveltimes produce
satisfactory images because the velocity model is subdivided 
and traveltimes are calculated 
under conditions where finite-differencing the eikonal equation is valid.
By dividing the imaging problem in this
way, the traveltimes are better behaved and some multiple arrivals
are accounted for.

\bibliographystyle{seg}
\bibliography{SEP2,SEG}

