\shortnote
\lefthead{Brown, et al.}
\righthead{PEF from sparse data}
\footer{SEP--105}

\title{Test case for PEF estimation with sparse data II}
\email{morgan@sep.stanford.edu, claerbout@stanford.edu, sergey@sep.stanford.edu}
\author{Morgan Brown, Jon Claerbout, and Sergey Fomel}

%\ABS{ 
%	When data have an extremely sparse distribution, the two-stage missing data interpolation
%	of \longcite{gee} is inapplicable.  
%	We present a simple 2-D test case by deconvolving a field of random numbers by
%	a simple dip (steering) filter, then subsample the model to simulate sparse data.  
%	Given the correct filter, we obtain a pleasing interpolation, 
%	and also find that the quality of the result depended strongly on the accuracy
%	of the filter.
%}

\section{ Introduction}

The two-stage missing data interpolation approach of \longcite{gee} (henceforth, the {\em GEE 
approach}) has been applied
with great success \cite{Fomel.sep.95.sergey1,Clapp.sep.97.bob1,Crawley.sep.104} in the past.  
The main strength of the approach lies in the ability of the prediction error filter (PEF) to 
find multiple, hidden correlation in the known data, and then, via regularization, to impose
the same correlation (covariance) onto the unknown model.
Unfortunately, the GEE approach may break down
in the face of very sparsely-distributed data, as the number of valid regression equations
in the PEF estimation step may drop to zero.  
In this case, the most common approach is to simply retreat to regularizing with an isotropic 
differential filter (e.g., Laplacian), which leads to a 
minimum-energy solution and implicitly assumes an isotropic model covariance.
\par
A pressing goal of many SEP researchers is to find a way of estimating a PEF from sparse
data.  Although new ideas are certainly required to solve this 
interesting problem, \longcite{Claerbout.sep.103.jon5} proposes that a standard, simple test 
case first be constructed, and suggests using a known model with vanishing
Gaussian curvature.  In this paper, we present the following, simpler test case, which we
feel makes for a better first step.
\begin{itemize}
	\item {\bf Model:} Deconvolve a 2-D field of random numbers with a simple dip filter, 
	                   leading to a ``plane-wave'' model.
	\item {\bf Filter:} The ideal interpolation filter is simply the dip filter used to 
	                    create the model.
	\item {\bf Data:} Subsample the known model randomly and so sparsely as to make 
	                  conventional PEF estimation impossible.
\end{itemize}
\par
We use the aforementioned dip filter to regularize a least squares estimation of the
missing model points and show that this filter is ideal, in the sense that the
model residual is relatively small.  Interestingly, we found that the characteristics
of the true model and interpolation result depended strongly on the accuracy (dip
spectrum localization) of the dip filter.  We chose the 8-point truncated sinc filter
presented by \longcite{Fomel.sep.105.sergey2}.  We discuss briefly the motivation for
this choice.

\section{ Methodology}

\longcite{gee} presents a two-stage methodology for missing data interpolation.
In the first stage of the so-called {\em GEE approach}, a prediction error filter (PEF) 
is estimated from known data.
In the second stage, the PEF is used in a least squares interpolation scheme to 
regularize the undetermined (missing) model points.  \longcite{Crawley.sep.104}
extends the two-stage procedure to use nonstationary PEF's.
\par
The first stage (PEF estimation) of the two-stage procedure consists of convolving
the unknown filter coefficients with the known data, and adjusting the coefficients
such that the residual is minimized.  Conceptually, in the process of convolution,
a filter template is slid past every point in the data domain.  The GEE approach adheres to the
following convention: unless every point in the filter template overlies known data, the 
regression equation for that output point is ignored, and will not contribute to 
the PEF estimation.
\par
Unfortunately, when the known data is very sparsely distributed, all the regression
equations may depend on missing data, making PEF estimation impossible.  
The motivation of this paper is not to present a new methodology for estimating a PEF 
from sparse data, but instead to create a very simple test case which fulfills the
following criteria:
\begin{enumerate}
	\item  The known data is distributed so sparsely as to render the traditional 
	       GEE two-stage approach ineffective.
	\item  The underlying model is conceptually simple and stationary.
	\item  The ideal PEF for the underlying model is obtainable by common sense.
\end{enumerate}

\subsection{The Test Case}
\inputdir{test}

\longcite{Claerbout.sep.103.jon5} proposes a test case for which the Gaussian curvature
of the model vanishes.  In this paper, we present an even simpler test case.  Given a 
2-D random field, we deconvolve with a known dip (or steering) \cite{Clapp.sep.95.bob1} 
filter to obtain a ``plane wave'' model, as shown in Figure \ref{fig:model}.
To simulate collected ``data'', we sampled the model of Figure \ref{fig:model} at
about 60 points randomly, and about two-thirds of the way down one trace in the center.  
The results are shown in Figure \ref{fig:data}.

\sideplot{model}{width=2.5in,height=2.5in}{ True model - plane waves dipping at $22.5^{\circ}$. }

\plot{data}{width=5.5in,height=2.5in}{ Left: Collected data - one known trace, about 60 
	randomly-selected known data points.  Right: Known data selector. }

\subsection{Digression: Accuracy of Dip Filters}
Given a pure plane wave section, i.e., a wavefield where all events have linear 
moveout, designing a discrete multichannel filter to annihilate events with a given
dip seems a trivial task.  In fact, it is quite a tricky task; an exercise in
interpolation.  For many applications, accuracy considerations make the choice of 
interpolation algorithm critical.  {\em Accuracy} here means localization of the
filter's dip spectrum --- ideally the filter should annihilate only the desired
dip, or a narrow range of dips.  

\sideplot{steering}{width=2.5in,height=2.5in}{ Steering filter schematic. Given a plane wave
	with dip $p$, choose the $a_i$ to best annihilate the plane wave. }

The problem is illustrated in Figure \ref{fig:steering}.  Given a plane wave with
dip $p$, we must set the filter coefficients $a_i$ to best annihilate the plane wave.
Achieving good dip spectrum localization implies a filter with many coefficients, by the 
uncertainly principle \cite{bracewell}.  If computational cost was not an issue, the best
choice would be a sinc function with as many coefficients as time samples.
Realistically, however, a compromise must be found between pure sinc and 
simple linear interpolation.  The reader is referred to Fomel's \shortcite{Fomel.sep.105.sergey2} 
paper, which discusses these issues much more
thoroughly.  The model of Figure \ref{fig:model} was computed using an 8-point
tapered sinc function.  Figure \ref{fig:interp-comp} compares
the result of using, for the same task,  dip filters computed via four different
interpolation schemes: 8-point tapered sinc, 6-point local Lagrange, 4-point cubic convolution,
and simple 2-point linear interpolation.  As expected, we see that the more 
accurate interpolation schemes lead to increased spatial coherency in the model panel.
\longcite{Clapp.sep.103.bob3} has been successful in using as few as 3 coefficients in steering
filters for regularizating tomography problems, so we see that the needed amount 
of steering filter accuracy is a problem-dependent parameter.

\plot{interp-comp}{width=5.0in,height=5.0in}{ Interpolation schemes compared. 
	Deconvolution of random image with labeled steering filters.}

\section{ Interpolation results}

	The plane wave model of Figure \ref{fig:model} dips at $22.5^{\circ}$, so we can easily design a 
	filter to annihilate it.  Using the GEE approach for interpolating missing data \cite{gee}, we 
	interpolate the data of Figure \ref{fig:data}, using the 8-point tapered sinc steering filter
	discussed above.  The results are shown in Figure \ref{fig:correctFill}.  We see that the 
	interpolation is quite good in the center region, where the filter can ``see'' one or more known 
	data points, as evidenced by a nearly uncorrelated model residual.  In the corners, the result 
	is imperfect 
	in regions in which no known data points exist along diagonal tracks.  In order to suppress 
	helix wraparound and other edge effects, we apply zero-padding around the edges of the study 
	region. 

\plot{correctFill}{width=5.5in,height=5.5in}{ Clockwise from upper left: Known data; Interpolation
	regularized with 8-point tapered sinc steering filter; Difference between known model
	and interpolated result;  known model.  }

\section{ Conclusions }

	We presented a 2-D test case for sparse data interpolation and give a good PEF with which to do it.
	The test case renders the traditional GEE two-stage interpolation scheme inapplicable.
	\longcite{Claerbout.sep.103.jon5} suggests a nonlinear iteration, where filter and model 
	are taken as unknown, but the best solution is still a subject of discussion among many SEP
	researchers.  Regardless of the chosen interpolation strategy, the ``correct'' PEF 
	and model are both known in this test case, so it should prove a useful starting point.

\bibliographystyle{seg} 
\bibliography{SEP2,paper} 
