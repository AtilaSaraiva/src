\section{Introduction}
%%%%
The hyperbolic approximation of $P$-wave reflection traveltimes in
common-midpoint gathers plays an important role in conventional
seismic data processing and interpretation. It is well known that
hyperbolic moveout gives exact traveltimes for homogeneous isotropic
or elliptically anisotropic media overlaying a plane dipping
reflector. Deviations from this simple model generally cause departure
from hyperbolic moveout.
%deviations of the true reflection
%moveout from the hyperbolic approximation. 
If the nonhyperbolicity is measurable, we can 
%want to 
take it into account to correct
errors in conventional processing or to obtain additional information
about the medium. To achieve this, however, it is important to know
what causes the $P$-wave moveouts to be nonhyperbolic. Although
seismic anisotropy is one possible reason, it is not always the 
dominant one; others include
the vertical or lateral heterogeneity and reflector curvature. 
%Even if
%nonhyperbolic moveout is not caused by anisotropy, it may be useful to
%consider its presence as an evidence of some ``effective'' anisotropy.
%However, in order to
%provide a correct interpretation, it is important to distinguish among
%the different kinds of effects. 
In this paper, we give a theoretical description of $P$-wave
reflection traveltimes in different models and compare the behavior and degree of
nonhyperbolic moveout caused by various reasons.
%situations when the effect of anisotropy couples with one of the other
%three effects. We provide a theoretical description of these effects
%and compare their influence on $P$-wave reflection moveouts.
\par A transversely isotropic model with a vertical symmetry axis (VTI
medium) is the most commonly used anisotropic model for sedimentary
basins, where the deviation from isotropy is usually attributed to
some combination of fine layering and inherent anisotropy of
shales. One of the first nonhyperbolic approximations for the $P$-wave
reflection traveltimes in VTI media was proposed by
\cite{Muir.sep.44.55} and further developed by
\cite{Dellinger.jse.92.23}.  \cite{GEO51-10-19541966} introduced a
convenient parameterization of VTI media that was used by
\cite{tsvantom} to describe nonhyperbolic reflection moveouts.  \par
We begin with an overview of the weak-anisotropy approximation for
$P$-wave velocities in VTI media and use it for analytic derivations
throughout the paper. First, we consider a vertically heterogeneous
anisotropic layer. For this model, we compare the three-parameter
approximation for the $P$-wave traveltimes suggested by
\cite{tsvantom} with the shifted hyperbola \cite[]{malov,
%Sword.sep.51.313,
castle,nmo}. Next, we examine $P$-wave moveout in VTI media above
%The second case is a
%homogeneous anisotropic medium with 
a curved reflector. 
%In this case,
We analyze the cumulative action of anisotropy, reflector dip, and
reflector curvature, and develop an appropriate three-parameter representation
for the reflection moveout. Finally, we consider 
% the case of 
models characterized by weak lateral heterogeneity and show that 
%with an appropriate choice of the lateral velocity variation, 
it can mimic the influence of transverse isotropy on nonhyperbolic moveout. 
%In conclusion, we discuss possible applications of the theory for moveout 
%modeling and inversion.

