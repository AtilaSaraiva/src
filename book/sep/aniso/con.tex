\section{Conclusions}
Nonhyperbolic reflection moveout of $P$-waves is sometimes considered as
an important indicator of anisotropy. Its correct interpretation,
however, is impossible without taking other factors into account. In
this paper, we have considered three other important factors: vertical
heterogeneity, curvature of the reflector, and lateral heterogeneity.
Each of them can have an influence on nonhyperbolic
behavior of the reflection moveout comparable to that of anisotropy.
In particular, vertical heterogeneity produces a depth-variant
anisotropic pattern that differs from that in VTI media. For
isotropic media, this pattern is reasonably well approximated by the
shifted hyperbola. In a vertically heterogeneous VTI medium, the
parameters of anisotropy should be replaced with their effective
values. For a curved reflector in a homogeneous VTI
medium, we have developed an approximation based on the Taylor series
expansion of the traveltime with both the reflector curvature and the
anisotropic parameters entering the nonhyperbolic term. Lateral
heterogeneity can effectively mimic the influence of virtually any
anisotropy.
\par
The theoretical results of this paper are directly applicable to  
modeling of the nonhyperbolic moveout. The general formulas connecting 
the derivatives of reflection traveltime with those of direct waves are
particularly attractive in this context. For smooth velocity models, 
these formulas reduce the problem of tracing a family of reflected 
rays to tracing only one zero-offset ray. Practical estimation and inversion 
of nonhyperbolic moveout is a different and more difficult problem than is
the forward one. Given that a variety of reasons might cause similar 
nonhyperbolic moveout of $P$-waves, its inversion will be nonunique.  
Nevertheless, the theoretical guidelines provided by the analytical theory 
are helpful for the correct formulation of the inverse problems. They 
explicitly show us which medium parameters we may hope to extract from
the kinematics of long-spread $P$-wave reflection data.

