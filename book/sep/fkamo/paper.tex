\lefthead{Vlad and Biondi}
\righthead{Log-stretch AMO}
\footer{SEP--110}

\title{Effective AMO implementation in the log-stretch,
frequency-wavenumber domain}

\email{nick@sep.stanford.edu, biondo@sep.stanford.edu}
%\keywords{AMO, DMO, azimuth moveout, log-stretch}
\author{Ioan Vlad and Biondo Biondi}
%\ABS{Short notes do not have abstracts}

\shortpaper
\maketitle
\section{Introduction}
Azimuth moveout (AMO), introduced by \cite{GEO63-02-05740588}, is used
as part of the styling goal (in conjunction with a derivative as a
roughener) in \cite{Biondi.sep.110.biondo1}. This paper describes the
implementation of AMO for the above-stated purpose, with a historical
background, proof, and discussion of pitfalls and practical steps.

\section{The Azimuth Moveout}
AMO is conceived as a
cascade of forward and reverse dip moveout (DMO) operators. Thus, the
accuracy and speed of the DMO operator used is highly
important. Computing the DMO in the frequency domain is accurate and
simple, but computationally expensive because the DMO operator is temporally
nonstationary. The technique of logarithmic time-stretching,
introduced by \cite{GPR30-06-08130828} increases the computational
efficiency because the DMO operator is stationary in the log-stretch
domain, and Fast Fourier Transforms can be used instead of slow
Discrete Fourier Transforms. \cite{gardner}, \cite{GEO58-01-00470066} and
\cite{GEO61-03-08150820} derived equivalent and accurate log-stretch,
frequency-wavenumber DMO operators. The implementation of the AMO presented in
this paper is based on the derivation and algorithm in
\cite{GEO61-03-08150820}. 

\section{The log-stretch, frequency-wavenumber AMO in 3D}
Starting from the parametric DMO relations of \cite{GEO58-01-00470066},
\cite{GEO61-03-08150820} derives an expression for a DMO applicable on
2D NMO-ed data. In order to extend the expression to 3D, we only have
to replace the product $kh$ between the wavenumber and half offset
with the dot product of the same quantities, which are vectors in the
case of 3D data. In order to perform AMO from the offset $\vec h_1$ to
the offset $\vec h_2$, we need to cascade one forward DMO from offset
$\vec h_1$ to zero offset with a reverse DMO from zero offset to
offset $\vec h_2$. Thus, applying log-stretch, frequency-wavenumber
AMO on a 3D cube of data $\left. {P\left( {t,m_x ,m_y } \right)}
\right|_{\vec h_1 }$ in order to obtain $\left. {P\left( {t,m_x ,m_y }
    \right)} \right|_{\vec h_2 }$ involves the following sequence of
operations:
\begin{enumerate}
\item Apply log-stretch along the time axis on the $\left. {P\left( {t,m_x ,m_y } \right)} \right|_{\vec h_1 }$ cube, with the formula:
\begin{equation}
\label{eqn:logstretch}
\tau  = \ln \left( {\frac{t}{{t_c }}} \right),
\end{equation}
where $t_c$ is the minimum cutoff time introduced to avoid taking the logarithm of zero. All samples from times smaller than $t_c$ are simply left untouched, the rest of the procedure will be applied to the cube $\left. {P\left( {t > t_c,m_x ,m_y } \right)} \right|_{\vec h_1 }$.
\item 3D forward FFT of the $\left. {P\left( {\tau,m_x ,m_y } \right)} \right|_{\vec h_1 }$ cube. The 3D forward Fourier Transform is defined as follows:
\begin{equation}
\label{eqn:ft}
P\left( {\Omega ,k_x ,k_y } \right) = \int {\int {\int {P\left( {\tau ,m_x ,m_y } \right)} } } \;e^{i\left( {\Omega \tau  - k_x m_x  - k_y m_y } \right)} d\tau \,dm_x {\kern 1pt} dm_y 
\end{equation}
It can be seen that the sign of the transform along the $\tau$-axis is opposite to that over the midpoint axes.
\item For each element of the cube, perform the AMO shift:
\begin{equation}
\label{eqn:amo}
 \left. {P\left( {\Omega ,k_x ,k_y } \right)} \right|_{\vec h_2 }  =
 e^{i\left( {\Phi _1  - \Phi _2 } \right)} \left. {P\left( {\Omega
 ,k_x ,k_y } \right)} \right|_{\vec h_1 } , where
\end{equation}

\begin{equation}
\label{eqn:amophi}
 \Phi _j  = \left\{ {\begin{array}{*{20}c}
   {0,\quad for\;\vec k \cdot \vec h = 0}  \\
   {\vec k \cdot \vec h,\quad for\;\Omega  = 0}  \\
   {\frac{\Omega }{2}\left\{ {\sqrt {1 + \left( {\frac{{2\,\vec k \cdot \vec h}}{\Omega }} \right)^2 }  - 1 - \ln \left[ {\frac{{\sqrt {1 + \left( {\frac{{2\,\vec k \cdot \vec h}}{\Omega }} \right)^2 }  + 1}}{2}} \right]} \right\}\quad otherwise}  \\
\end{array}\;} \right.,\quad
\end{equation}

\begin{equation}
\label{eqn:amophi}
where \; \vec k \cdot \vec h = k_x h_x  + k_y h_y \quad
\end{equation}
 and j can take the values 1 or 2.
The frequency domain variables must have incorporated in their value a $2\pi$ constant (they are defined according to equation (\ref{eqn:ft}))
\item Do reverse 3D FFT in order to obtain the $\left. {P\left( {\tau,m_x ,m_y } \right)} \right|_{\vec h_2 }$ cube.
\item Do reverse log stretch along the time axis and affix to the top of the cube the slices from times smaller than $t_c$. The final result is a $\left. {P\left( {t,m_x ,m_y } \right)} \right|_{\vec h_2 }$ cube.
\end{enumerate}
\inputdir{impresp} 
Figure \ref{fig:impresp1} shows the impulse response of the above
described AMO.
\plot{impresp1}{width=6in,height=4in}{AMO impulse response}

\section{Stretching and aliasing}
\inputdir{Matlab}

For the purpose of this discussion we define stretching of a
single-dimension space as any transformation from one space to
another that has the following property: at least an arbitrarily chosen sequence
of two consecutive, equal in length, intervals in the input space is
transformed into a sequence of two consecutive, $not$ equal in length,
intervals in the output space. Stretching an x-space to a y-space will be denoted as 
\begin{equation}
\label{eqn:str}
y = f(x)
\end{equation}
Two obvious examples of stretching are
\[
\begin{array}{l}
 NMO:\;y = \sqrt {x^2  + \alpha } ,\;and \\ 
 log-stretch:\;y = \log \left( {\frac{x}{\alpha }} \right), \\ 
 \end{array}
\]
where $\alpha$ is a positive real number whose value does not matter
for the purpose of this discussion. As it can be seen in
Fig. \ref{fig:strali}, if we keep the same sampling rate ($\Delta y =
\Delta x$), aliasing can occur when doing the reverse transformation,
from x to y. In order to avoid aliasing, we need to compute $\Delta
y_{\max }$, the largest accceptable sampling rate in the y
domain. This can sometimes lead to a larger number of samples in the
$y$ domain, and thus to larger computational expense. This can be
limited to some extent if the signal in the $x$-space has been
bandpassed, as is often the case with seismic data, with the largest
frequency present in the data ($f_{\max}$) smaller than the Nyquist
frequency given by the sampling rate ($f_{Ny}$). Thus, we can replace in our calculations $\Delta x$ with
\[
\Delta x_{\max }  = \frac{1}{{2f_{\max } }},
\]
which will result in a $\Delta y_{\max }$ larger than that computed using $\Delta x$, the sampling rate in the $x$ space.
\plot{strali}{width=6.5in}{Illustration of how aliasing can occur
while stretching: if the same sampling rate is used for the $y$-space
(lower plot) as for the $x$-space (upper plot), serious aliasing will
occur when transforming back to $x$-space. This will not happen if the
sampling rate in the $y$-space is smaller than or equal to $\Delta y_{\max }$}
In order to compute $\Delta y_{\max }$, we will consider two points in the $x$ space, as seen in Fig. \ref{fig:strali}, such as
\begin{equation}
\label{eqn:xa}
x_b  = x_a  + \Delta x_{\max } 
\end{equation}
and $y_a$ and $y_b$, the images of  $x_a$ and $x_b$ in the $y$ space. Thus,
\[
\Delta y = y_b  - y_a  = f_{(x_a  + \Delta x_{\max } )}  - f_{(x_a )} 
\]
The largest sampling rate in the $y$-space that will not result in aliasing is $\Delta y_{\max }$, the minimum possible value of $\Delta y$. Suppose there is a value $x_m$ that minimizes $\Delta y$. Then,
\[
\Delta y_{\max }  = \left. {\left[ {f_{(x + \Delta x_{\max } )}  - f_{(x)} } \right]} \right|_{x_m } 
\]
In particular, in the case of log-stretch, given by equation (\ref{eqn:logstretch}), if $t_m$ plays the role of $x_m$ from the equation above, then
\begin{equation}
\label{eqn:tempojunk}
\Delta \tau _{\max }  = \left. {\left[ {\log \left( {\frac{{t + \Delta t_{\max } }}{{t_c }}} \right) - \log \left( {\frac{t}{{t_c }}} \right)} \right]} \right|_{t_m }  = \log \left( {1 + \frac{{\Delta t_{\max } }}{{t_m }}} \right)
\end{equation}
$\tau_{\max }$ will be minimum when $t_m$ is as large as possible,
thus minimizing the expression under the logarithm. How large can $t_m$ get? Since the length of the seismic trace is limited to a value $t_{\max }$, 
\[
t_{\mathop{\rm m}\nolimits}   = t_{\max }  - \Delta t_{\max } 
\]
because $t_m$ is the equivalent of $x_a$ from eq. (\ref{eqn:xa}) and Fig. \ref{fig:strali}. Thus, we get
\begin{equation}
\label{eqn:deltataumax}
\Delta \tau _{{\rm max}}  = \log \left( {\frac{{t_{\max } }}{{t_{\max }  - \Delta t_{\max } }}} \right)
\end{equation}

\section{F-K filtering}
\inputdir{impresp}
As it can be seen in Fig. \ref{fig:impresp2}, the impulse response of
the AMO computed in the log-stretch, frequency-wavenumber domain has
some artifacts: high amplitude, large saddle corners. Low temporal
frequencies and high spatial slopes are also present. These artifacts can be eliminated easily
using a f-k filter, which is described below.
\plot{impresp2}{width=6in,height=4in}{AMO impulse response artifacts}

Suppose we want to attenuate all spatial frequencies $k$ that are larger than a certain threshold $ k_{\max }$, where
\begin{equation}
\label{eqn:kdef}
\begin{array}{l}
 k = \sqrt {k_x^2  + k_y^2 } \quad and \\ 
 k_{\max }  = \frac{{2\left| \omega  \right|}}{{v_{\min } }}, \\ 
 \end{array}
\end{equation}
with $\omega$, $k_x$ and $k_y$ being the coordinates in the frequency-wavenumber domain (without logstretch), and $v$ being the minimum apparent velocity of the events that we want the filtered data cube to contain. Thus, the data cube will become:
\begin{equation}
\label{eqn:fkfilt}
P_{filtered} \left( {\omega ,k_x ,k_y } \right) = \left\{ {\begin{array}{*{20}c}
   {P\left( {\omega ,k_x ,k_y } \right)\quad if\quad k \le k_{\max } }  \\
   {e^{ - \varepsilon \left( {k - k_{\max } } \right)^2 } P\left( {\omega ,k_x ,k_y } \right)\quad if\quad k > k_{\max } }  \\
\end{array}} \right.
\end{equation}
Too small an $\varepsilon$ will result in an abrupt transition in the f-k domain, and thus ringing artifacts in the t-x domain. An $\varepsilon$ which is too big will result in no visible filtering of the targeted artifacts. Moreover, $\varepsilon$ depends on the choice of units and the number of samples for the $m_x$ and $m_y$ axes: since the exponential needs to be dimensionless, we have
\[
\varepsilon  = \frac{{\varepsilon _0 }}{{dk_x dk_y }}
\]
where
\[
dk_x  = \frac{1}{{n_x d_x }}\;and\,dk_y  = \frac{1}{{n_y d_y }}.
\]
Thus, the final expression of $\varepsilon$ is 
\begin{equation}
\label{eqn:epsi}
\varepsilon  = \varepsilon _0 n_x d_x n_y d_y ,
\end{equation}
where $\varepsilon _0$ is a value that is hand-picked only once, and embedded in the code. This way, we will not have to change anything at all in the code or in the parameters in order to set $\varepsilon _0$, no matter what the units of the data cube may be. 

The result of the filtering can be seen in Fig. \ref{fig:fkfilter}: 
the slices through the cube are taken at exactly the same locations as
those in Fig. \ref{fig:impresp2}, but now the artefacts are gone. 
\plot{fkfilter}{width=6in,height=4in}{AMO impulse response after f-k filtering}
\section{Cost-cutting avenues}
The largest computational savings come from the use of FFTs for AMO,
instead of slow Fourier integration necessary in the absence of
log-stretch. Standard means of minimizing the CPU time and the amount of memory used
to compute the AMO have also been employed. They include computing the AMO
shift for only half of the elements of the cube in the complex domain,
since the Fourier transform $F$ of a real function is Hermitian:
\begin{equation}
\label{eqn:fourierproperty}
F(s)=F^{*}(-s)
\end{equation}
(where $s$ denotes the frequency domain variable and the star symbol
denotes the complex conjugate). Another way of reducing computational
expenses was through the use of RFFTW and FFTW type Fourier
Transforms \cite[]{fftw}, adaptive to hardware architecture, and
taking advantage of 
the property stated in (\ref{eqn:fourierproperty}). Also, the code was
divided into subroutines in such a way that some quantities
were not computed unnecessarily several times when AMO was applied to
more than one cube of data. Finally, shared memory parallelization
with the OpenMP standard was applied to all the computationally
intensive do loops in the code.

\section{Conclusions}
Azimuth moveout can be successfully implemented in the log-stretch,
frequency-wavenumber domain. It is accurate,
fast, and furthermore it does not have any characteristics that can
result in coding difficulties. 

\bibliographystyle{seg}
\bibliography{AMO,SEP2,SEG}
