\published{SEP report, 92, 83-90 (1996)}

\lefthead{Rickett \& Claerbout}
\righthead{Passive seismic imaging}
\footer{SEP--92}

\title{Passive seismic imaging applied to synthetic data}
%\keywords{noise, modeling }

\email{james@sep.stanford.edu, jon@sep.stanford.edu}

\author{James Rickett and Jon Claerbout}
\maketitle

\begin{abstract}
It can be shown that for a 1-D Earth model illuminated by random plane waves
from below, the cross-correlation of noise traces recorded at two points on the
surface is the same as what would be recorded if one location contained a
shot and the other a receiver. If this is true for real data, it could
provide a way of building `pseudo-reflection seismograms' from background
noise, which could then be processed and used for imaging.
This conjecture is tested on synthetic data from simple 
1-D and point diffractor models,
and in all cases, the kinematics of observed events 
appear to be correct. 
The signal to noise ratio was found to increase as $\sqrt{n}$, where $n$ is
the length of the time series. The number of incident 
plane waves does not directly affect the signal to noise ratio; however,
each plane wave contributes only its own slowness to the common 
shot domain, so that if complete hyperbolas are to be imaged 
then upcoming waves must be incident from all angles. 
\end{abstract}

\section{Introduction}
Conventional seismic reflection methodology relies on having a source
of seismic waves at the surface and studying the reflections from  
impedance contrasts in the earth. 
Ambient noise present in the subsurface is also reflected by impedance 
contrasts in the same way. Therefore, do we need the source or can we 
determine earth structure simply by listening to the 
background noise for a long enough time~?  

With 4-D seismic surveys aimed at monitoring fluid movements becoming more and 
more popular, a number of oil fields are having geophone arrays laid out 
permanently to reduce acquisition costs.  These geophones will 
typically only record while a survey is being shot, and for most of their 
life they will be turned off.  However, if the geophones are left recording 
while surveys are not being shot, we hypothesize the information 
contained in the background seismic energy could be used for imaging the 
subsurface between main surveys.

For this to be realized a technique has to be developed that: firstly, is 
able to extract the useful information from the background noise; and 
secondly, is able to do this quickly enough, ideally in real-time, so 
the huge amounts of raw data that would be recorded would not have to be 
stored.

With this in mind, a technique for creating `pseudo-cmp gathers' 
out of background noise traces using cross-correlation is explored 
and tested on a variety of synthetic models.

\section{CONJECTURE}

\begin{quotation}
By cross-correlating noise traces noise traces recorded at two locations 
on the surface, we can construct the wavefield that would be recorded at 
one of the locations if there was a source at the other.
In this way we can create `pseudo-reflection seismograms' that include 
such effects as NMO and DMO.
\end{quotation}

\subsection{PROOF FOR A 1-D EARTH}
\inputdir{.}
  
The proof of this conjecture for a one dimensional earth is given as a 
problem set in \cite{fgdp}. 
The outline of the derivation that follows uses 
the $Z$ transform approach developed there, where $Z$ is the unit 
delay operator $e^{-i\omega \Delta t}$.

Consider a plane-layered Earth model with the reflection seismology 
geometry shown in Figure~\ref{fig:8-7}. If the system is lossless then the energy
flux through the top layer has to be equal to the 
flux through the half-space below.  Therefore,
\begin{equation}
Y_1 \left\{ R\left(\frac{1}{Z}\right) R(Z) - 
\left[1+R\left(\frac{1}{Z}\right) \right]
[1+R(Z)] \right\} = - Y_k E\left(\frac{1}{Z}\right) E(Z)
\end{equation}
where $Y_1$ is the impedance of the top layer and $Y_k$ is the impedance 
of the half-space, or
\begin{equation}
1 + R\left(\frac{1}{Z}\right) + R(Z) = \frac{Y_k}{Y_1} 
E\left(\frac{1}{Z}\right) E(Z)
\end{equation}

\sideplot{8-7}{width=3in,height=2.5in}{Reflection seismology geometry
- Claerbout~(1979).}

Comparing the reflection seismology geometry with the 
earthquake seismology geometry shown in Figure~\ref{fig:8-8} gives 
$E(Z) = X(Z)$ by reciprocity.  Therefore
\begin{equation}
1 + R\left(\frac{1}{Z}\right) + R(Z) = 
(constant) \; X\left(\frac{1}{Z}\right) X(Z)
\end{equation}
Since $R(Z)=0$ for times less than zero, this means the positive time part 
of the `earthquake' seismogram's auto-correlation function 
equals the reflection seismogram. 

This theorem can be extended to a two dimensional plane-layered Earth 
by considering slant stacks \cite[]{iei}.

\sideplot{8-8}{width=3in,height=2.5in}{Earthquake seismology geometry 
- from Claerbout~(1979). All the waves (1, $X$ and $P$) could be multiplied
 by the same 
random signal, and this would not effect their spectra or auto-correlation
functions.}

\section{SYNTHETIC DATA}
\inputdir{synth}

\cite{scthesis} tested this conjecture on both synthetic 
and real data.  However his field data was very noisy, and he did not draw any 
solid conclusions.

With this as a starting point, however, I continued modeling a single 
reflection, using a program based on the flow \\

\noindent {\tt
loop over each plane wave \{ \\
\hspace*{1cm}  calculate a random slowness, $p$ \\
\hspace*{1cm}  calculate the time delay due to a reflection, $\Delta t$\\
\hspace*{1cm}  loop over each frequency, $\omega$ \{ \\
\hspace*{2cm}    calculate a random amplitude \\
\hspace*{2cm}    loop over each spatial location, x \{  \\
\hspace*{3cm}      multiply each frequency by a factor $(1 + 
r e^{i \omega \Delta t}) \; e^{i \omega p x}$ \\
\hspace*{2cm}   \}  \\
\hspace*{1cm} \}  \\
\}  \\
}

Having produced synthetics, it was then possible to go ahead and 
cross-correlate traces to try and create pseudo-reflection seismograms. 
Figure~\ref{fig:first} is a pseudo-shot gather generated 
by cross-correlating one trace with every other.  The center panel shows how 
the clarity of the signal 
was improved by applying a $\sqrt{-i\omega}$ filter. 
The black line which has been overlain in the left panel 
corresponds to the expected hyperbola which would be observed in a real
shot gather, offset by 0.05~s so it does not obscure the data.
Therefore the kinematics in this case appear to be consistent 
with the conjecture.

\plot{first}{width=6in,height=3in}{
Pseudo-shot gather over model with single horizontal layer and 200 
incoming plane waves. The left panel is raw cross-correlations, the center 
panel has a half differentiation filter applied and the right panel is 
labeled with the correct kinematics shifted by 0.05~s.
}

\subsection{Limited angular bandwidth}

Figure~\ref{fig:build} shows how the hyperbolas build up 
from the individual plane waves. In the left panel there is only
one plane wave present, instead of 200 as in Figure~\ref{fig:first}.  
The trace at zero spatial offset is the autocorrelation 
function, and it has two spike as expected: 
one at zero lag and one corresponding to the reflected event. 
Other traces, corresponding to cross-correlations, have the two same 
spikes, but these are offset in time due to the different 
receiver positions and the apparent velocity of the incident wave.
In the right panel of Figure~\ref{fig:build} there are 5 plane waves incident, 
and a hyperbola is beginning to form.
Only one part of each plane wave adds coherently with the others, the rest 
of the energy is smeared out. 

The corollary of this is that, in order to see a slope in the pseudo-reflection
seismograms, we need incident plane waves with the relevant slowness.
This will become an issue when looking at real data where the direction of
incoming waves will not be spatially white.

\sideplot{build}{width=4in,height=3in}{
Pseudo-shot gather over model with single horizontal layer. Left panel 
has one incoming plane wave and right panel has five.
}

\subsection{More complex Earth models - point diffractors and multiple layers}

The horizontal bed code shown earlier was easy to adapt to test 
slightly more complicated models.
The left panel of Figure~\ref{fig:point} shows a pseudo-shot gather 
with a point diffractor in the subsurface. Again the kinematics are correct, 
as they are in the zero-offset section of the same model shown in the 
right panel of Figure~\ref{fig:point}.  Notice the offset center of the 
hyperbola in the pseudo-shot gather.

\sideplot{point}{width=4in,height=3in}{
Model with single point diffractor. Left panel shows common-shot gather,
and right panel shows zero-offset section.
}

With a model of two horizontal layers, the results (Figure~\ref{fig:two}) 
seem to contain a 
spurious event at the time of the inter-bed multiple. This is not the multiple 
itself but comes from the correlation between the first layer and the
second layer.  If the inter-bed multiple was included in the modeling process, 
it would arrive at the same time but with opposite amplitude and so 
cancel this event out. Therefore this event is an artifact of the modeling 
technique and does not contradict the conjecture.

\sideplot{two}{width=2in,height=3in}{
Common shot gather for a model with two horizontal layers.
}

\subsection{Noise suppression}

So far the reflection coefficients have been very high (0.5) and still the 
images have been relatively noisy. If this 
technique is to be used for looking at real targets, reflection coefficient 
of an order of magnitude smaller will have to be imaged.

Tests showed that the signal to noise ratio decreased as the $\sqrt{n}$, 
where $n$ is the number of time samples, and by increasing the length 
of the time series used I was able to clearly image reflection coefficients
of 0.05, as shown in Figure~\ref{fig:many}.

\plot{many}{width=6in,height=3in}{
Common shot gathers for a model with a single horizontal layer with 
reflection coefficient 0.05. Left panel used 65,000 points in the time series,
the center panel about 130,000 and the right panel about 260,000.
}

A similar study, but this time comparing signal to noise ratio with the 
number of incident plane waves, was also conducted.  Interestingly, 
it showed that more plane waves did not increase the amplitude of the 
observed signal to noise; however, more plane waves do 
improve the shape and definition of the hyperbola.

\section{CONCLUSIONS AND FURTHER WORK}

So far this method has worked well on a variety of simple Earth models 
that have been illuminated by plane waves coming from all angles in 
the subsurface.

Before the method is tested on real data, the following points need to be 
addressed:
\begin{enumerate}
\item
A method of spatially pre-whitening upcoming waves has to be developed, so 
that all dips can be imaged even if the angular distribution of 
upcoming waves is uneven.
\item
The conjecture should be tested with full wave-form modeling 
on more complicated Earth models (multiple dipping beds and 
velocities which vary both laterally and vertically, for example).
\item
In order to avoid storing huge amounts of data, the cross-correlations should
be done in the field in real time.  Implementation of this will be difficult.
\end{enumerate}

\bibliographystyle{seg}
\bibliography{SEP2,passive}





