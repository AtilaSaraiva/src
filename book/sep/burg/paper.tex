% Started 11/29/00

%\shortnote

\lefthead{}
\righthead{}

\renewcommand{\thefootnote}{\fnsymbol{footnote}} 

\title{The Wilson-Burg method of spectral factorization \\
  with application to helical filtering}

\author{Sergey Fomel\/\footnotemark[1],
Paul Sava\/\footnotemark[2],
James Rickett\/\footnotemark[3],
and Jon F. Claerbout\/\footnotemark[2]} 

\footnotetext[1]{\emph{Bureau of Economic Geology, 
Jackson School of Geosciences,
The University of Texas at Austin,
University Station, Box X,
Austin, Texas 78713-8972, USA}}
\footnotetext[2]{\emph{Stanford Exploration Project, 
        Department of Geophysics, 
        Stanford University, Stanford, California 94305, USA}}
\footnotetext[3]{\emph{ChevronTexaco Exploration and Production Technology
    Company, 6001 Bollinger Canyon Road, San Ramon, California 94583-2324}}

\maketitle

\begin{abstract}    
  Spectral factorization is a computational procedure for constructing
      minimum-phase (stable inverse) filters required for recursive
      inverse filtering. We present a novel method of spectral
      factorization. The method iteratively constructs an
      approximation of the minimum-phase filter with the given
      autocorrelation by repeated forward and inverse filtering and
      rearranging the terms. This procedure is especially efficient in
      the multidimensional case, where the inverse recursive filtering
      is enabled by the helix transform.
      
      To exemplify a practical application of the proposed method, we
      consider the problem of smooth two-dimensional data
      regularization. Splines in tension are smooth interpolation
      surfaces whose behavior in unconstrained regions is controlled
      by the tension parameter. We show that such surfaces can be
      efficiently constructed with recursive filter preconditioning
      and introduce a family of corresponding two-dimensional
      minimum-phase filters. The filters are created by spectral
      factorization on a helix.
\end{abstract}

\section{Introduction}

Spectral factorization is the task of estimating a minimum-phase signal from a
given power spectrum. The advent of the helical coordinate system
\cite[]{helix0,GEO63-05-15321541} has led to renewed interest in spectral factorization
algorithms, since they now apply to multi-dimensional problems. Specifically,
spectral factorization algorithms provide the key to rapid multi-dimensional
recursive filtering with arbitrary functions, which in turn has geophysical
applications in preconditioning inverse problems \cite[]{SEG-1998-1851,GEO68-02-05770588},
wavefield extrapolation
\cite[]{SEG-1998-1124,EAE-2000-P0145,SEG-2000-08620865,SEG-2001-10571060}, and
3-D noise attenuation \cite[]{SEG-1999-12311234,EAE-1999-6039,EAE-2001-P167}.

The Kolmogoroff (cepstral or Hilbert transform) method of spectral
factorization \cite[]{kolmog,Claerbout.fgdp.76,oppenheim} is often used by the
geophysical community because of its computational efficiency. However, as a
frequency-domain method, it has certain limitations. For example, the
assumption of periodic boundary conditions often requires extreme amounts of
zero-padding for a stable factorization. This is one of the limitations which
make this method inconvenient for multi-dimensional applications.

The Wilson-Burg method, introduced in this paper, is an iterative algorithm
for spectral factorization based on Newton's iterations. The algorithm
exhibits quadratic convergence. It provides a time-domain approach that is
potentially more efficient than the Kolmogoroff method. We include a detailed
comparison of the two methods.

Recent surveys \cite[]{goodman,kailath} discuss some other methods for spectral
factorization, such as the Schur method \cite[]{schur}, the Bauer method
\cite[]{bauer} and Wilson's original method \cite[]{mywilson}. The latter is noted
for its superb numerical properties. We introduce Burg's modification to this
algorithm, which puts the computational attractiveness of this method to a new
level. The Wilson-Burg method avoids the need for matrix inversion, essential
for the original Wilson's algorithm, reduces the computational effort from
$O(N^3)$ operations to $O(N^2)$ operations per iteration. A different way to
accelerate Wilson's iteration was suggested by \cite{laurie}. We have
found the Wilson-Burg algorithm to be especially suitable for applications of
multidimensional helical filtering, where the number of filter coefficients
can be small, and the cost effectively reduces to $O(N)$ operations.

The second part of the paper contains a practical example of the
introduced spectral factorization method. The method is applied to the
problem of two-dimensional smooth data regularization. This problem
often occurs in mapping potential fields data and in other geophysical
problems. Applying the Wilson-Burg spectral factorization method, we
construct a family of two-dimensional recursive filters, which
correspond to different values of tension in the tension-spline
approach to data regularization \cite[]{GEO55-03-02930305}. We then use
the constructed filters for an efficient preconditioning of the data
regularization problem. The combination of an efficient spectral factorization
and an efficient preconditioning technique provides an attractive practical
method for multidimensional data interpolation. The technique is illustrated
with bathymetry data from the Sea of Galilee (Lake Kinneret) in Israel.

\section{Method description}

Spectral factorization constructs a minimum-phase signal from its
spectrum.  The algorithm, suggested by \cite{mywilson}, approaches
this problem directly with Newton's iterative method. In a
$Z$-transform notation, Wilson's method implies solving the equation
\begin{equation}
S(Z) = A(Z)\bar{A}(1/Z)
\label{eqn:specfac}
\end{equation}
for a given spectrum $S(Z)$ and unknown minimum-phase signal $A(Z)$
with an iterative linearization
\begin{eqnarray}
\nonumber
S(Z) & = & A_t(Z)\bar A_t(1/Z)+
     A_t(  Z)[\bar A_{t+1}(1/Z)-\bar A_t(1/Z)]+
\bar A_t(1/Z)[     A_{t+1}(  Z)-     A_t(  Z)] \\
& = & A_t(  Z) \bar A_{t+1}(1/Z) + \bar A_t(1/Z) A_{t+1}(Z)
- A_t(Z)\bar A_t(1/Z)
\;,
\label{eqn:wilson}
\end{eqnarray}
where $A_t(Z)$ denotes the signal estimate at iteration $t$. Starting from
some initial estimate $A_0(Z)$, such as $A_0(Z)=1$, one iteratively solves the
linear system~(\ref{eqn:wilson}) for the updated signal $A_{t+1}(Z)$. Wilson
\shortcite{mywilson} presents a rigorous proof that
iteration~(\ref{eqn:wilson}) operates with minimum-phase signals provided that
the initial estimate $A_0(Z)$ is minimum-phase. The original algorithm
estimates the new approximation $A_{t+1}(Z)$ by matrix inversion implied in
the solution of the system.

Burg (1998, personal communication) recognized that dividing both
sides of equation (\ref{eqn:wilson}) by $\bar A_t(1/Z) A_t(Z)$ leads
to a particularly convenient form, where the terms on the left are
symmetric, and the two terms on the right are correspondingly strictly
causal and anticausal:
\begin{equation} \label{eqn:burg}
1 \ +\ {S(Z) \over \bar A_t(1/Z)\  A_t(Z)} =
{A_{t+1}(Z) \over A_t(Z)}
\ +\
{\bar A_{t+1}(1/Z) \over \bar A_t(1/Z)}
\end{equation}
\par
Equation~(\ref{eqn:burg}) leads to the Wilson-Burg algorithm, which
accomplishes spectral factorization by a recursive application of
convolution (polynomial multiplication) and deconvolution (polynomial
division). The algorithm proceeds as follows:
\begin{enumerate}
\item Compute the left side of equation~(\ref{eqn:burg}) using forward
  and adjoint polynomial division.
\item Abandon negative lags, to keep only the causal part of the
  signal, and also keep half of the zero lag. This gives us
  $A_{t+1}(Z)/A_t(Z)$.
\item Multiply out (convolve) the denominator $A_t(Z)$. Now we have
the desired result $A_{t+1}(Z)$.
\item Iterate until convergence.  
\end{enumerate}

\tabl{tablerate}{Example convergence of the Wilson-Burg iteration}{
\begin{center}
\begin{tabular}{|r||r|r|r|r|}               \hline
 iter & $a_0$ &  $a_1$ & $a_2$ & $a_3$ \\ \hline \hline
 0   & 1.000000    &  0.000000   &  0.000000   &  0.000000 \\
 1   & 36.523964   & 23.737839   &  6.625787   &  0.657103 \\
 2   & 26.243151   & 25.726116   &  8.471050   &  0.914951 \\
 3   & 24.162354   & 25.991493   &  8.962727   &  0.990802 \\
 4   & 24.001223   & 25.999662   &  9.000164   &  0.999200 \\
 5   & 24.000015   & 25.999977   &  9.000029   &  0.999944 \\
 6   & 23.999998   & 26.000002   &  9.000003   &  0.999996 \\
 7   & 23.999998   & 26.000004   &  9.000001   &  1.000000 \\
 8   & 23.999998   & 25.999998   &  9.000000   &  1.000000 \\
 9   & 24.000000   & 26.000000   &  9.000000   &  1.000000 \\ \hline
\end{tabular}
\end{center}
}

An example of the Wilson-Burg convergence is shown in
Table~\ref{tbl:tablerate} on a simple 1-D signal. The autocorrelation
$S(Z)$ in this case is $1334 + 867 \left(Z + 1/Z\right) + 242
\left(Z^2 + 1/Z^2\right) + 24 \left(Z^3 + 1/Z^3\right)$, and the
corresponding minimum-phase signal is $A(Z) = (2+Z)(3+Z)(4+Z) = 24 +
26 Z + 9 Z^2 + Z^3$. A quadratic rate of convergence is visible from
the table. The convergence slows down for signals whose polynomial
roots are close to the unit circle \cite[]{mywilson}.

The clear advantage of the Wilson-Burg algorithm in comparison with the
original Wilson algorithm is in the elimination of the expensive matrix
inversion step. Only convolution and deconvolution operations are used at each
iteration step.

\subsection{Comparison of Wilson-Burg and Kolmogoroff methods}
The Kolmogoroff (cepstral or Hilbert transform) spectral factorization
algorithm \cite[]{kolmog,Claerbout.fgdp.76,oppenheim} is widely used because of
its computationally efficiency. While this method is easily extended to the
multi-dimensional case with the help of helical transform
\cite[]{SEG-1999-16751678}, there are several circumstances that make the
Wilson-Burg method more attractive in multi-dimensional filtering
applications.

\begin{itemize}
\item The Kolmogoroff method takes $O(N \log N)$ operations, where $N$ is the
  length of the auto-correlation function.  The cost of the Wilson-Burg method
  is proportional to the [number of iterations] $\times$ [filter length]
  $\times N$.  If we keep the filter small and limit the number of iterations,
  the Wilson-Burg method can be cheaper (linear in $N$). In comparison, the
  cost of the original Wilson's method is the [number of iterations] $\times$
  $O(N^3)$.

\item The Kolmogoroff method works in the frequency domain and assumes
  periodic boundary conditions.  Auto-correlation functions,
  therefore, need to be padded with zeros before they are Fourier
  transformed.  For functions with zeros near the unit circle, the
  padding may need to be many orders of magnitude greater than the
  original filter length, $N$. The Wilson-Burg method is implemented
  in the time-domain, where no extra padding is required.
  
\item Newton's method (the basis of the Wilson-Burg algorithm)
  converges quickly when the initial guess is close to the solution.
  If we take advantage of this property, the method may converge in
  one or two iterations, reducing the cost even further.  It is
  impossible to make use of an initial guess with the Kolmogoroff
  method.
  
\item The Kolmogoroff method, when applied to helix filtering,
  involves the dangerous step of truncating the filter coefficients to
  reduce the size of the filter. If the auto-correlation function has
  roots close to the unit circle, truncating filter coefficients may
  easily lead to non-minimum-phase filters. The Wilson-Burg allows us to
  fix the shape of the filter from the very beginning. This does not
  guarantee that we will find the exact solution, but at least we can
  obtain a reasonable minimum-phase approximation to the desired
  filter. The safest practical strategy in the case of an unknown
  initial estimate is to start with finding the longest possible
  filter, remove those of its coefficients that are smaller than a certain
  threshold, and repeat the factoring process again with the shorter
  filter.
\end{itemize}

\subsection{Factorization examples}

\inputdir{helix}

The first simple example of helical spectral factorization is shown in
Figure~\ref{fig:autowaves}. A minimum-phase factor is found by
spectral factorization of its autocorrelation. The result is
additionally confirmed by applying inverse recursive filtering, which
turns the filter into a spike (the rightmost plot in
Figure~\ref{fig:autowaves}.)

\plot{autowaves}{width=4.5in, height=4.5in}{Example of 2-D Wilson-Burg
  factorization. Top left: the input filter. Top right: its
  auto-correlation. Bottom left: the factor obtained by the Wilson-Burg
  method. Bottom right: the result of deconvolution.}

A practical example is depicted in Figure~\ref{fig:laplac}.  The
symmetric Laplacian operator is often used in practice for
regularizing smooth data. In order to construct a corresponding
recursive preconditioner, we factor the Laplacian autocorrelation
(the biharmonic operator) using the Wilson-Burg algorithm.
Figure~\ref{fig:laplac} shows the resultant filter. The minimum-phase
Laplacian filter has several times more coefficients than the original
Laplacian. Therefore, its application would be more expensive in a
convolution application. The real advantage follows from the
applicability of the minimum-phase filter for inverse filtering
(deconvolution). The gain in convergence from recursive filter
preconditioning outweighs the loss of efficiency from the longer
filter.  Figure~\ref{fig:thin42} shows a construction of the smooth
inverse impulse response by application of the $\mathbf{C} = \mathbf{P
  P}^T$ operator, where $\mathbf{P}$ is deconvolution with the
minimum-phase Laplacian. The application of $\mathbf{C}$ is equivalent
to a numerical solution of the biharmonic equation, discussed in the
next section.

\inputdir{laplac}

\plot{laplac}{width=4.5in, height=4.5in}{Creating a minimum-phase
  Laplacian filter.  Top left: Laplacian filter. Top right:
  its auto-correlation (bi-harmonic filter). Bottom left: factor obtained by the Wilson-Burg method
  (minimum-phase Laplacian). Bottom right: the result of deconvolution.}

\plot{thin42}{width=6in, height=2in}{2-D deconvolution with the
  minimum-phase Laplacian. Left: input. Center: output of
  deconvolution.  Right: output of deconvolution and adjoint
  deconvolution (equivalent to solving the biharmonic differential
  equation).}

\section{Application of spectral factorization: \newline
  Regularizing smooth data with splines in tension}

\hyphenation{Woi-now-sky}

The method of minimum curvature is an old and ever-popular approach
for constructing smooth surfaces from irregularly spaced data
\cite[]{GEO39-01-00390048}. The surface of minimum curvature corresponds
to the minimum of the Laplacian power or, in an alternative
formulation, satisfies the biharmonic differential equation.
Physically, it models the behavior of an elastic plate. In the
one-dimensional case, the minimum curvature method leads to the
natural cubic spline interpolation \cite[]{deBoor}. In the
two-dimensional case, a surface can be interpolated with biharmonic
splines \cite[]{sandwell} or gridded with an iterative finite-difference
scheme \cite[]{swain}.  We approach the gridding (data regularization)
problem with an iterative least-squares optimization scheme.
\par
In most of the practical cases, the minimum-curvature method produces
a visually pleasing smooth surface. However, in cases of large changes
in the surface gradient, the method can create strong artificial
oscillations in the unconstrained regions. Switching to lower-order
methods, such as minimizing the power of the gradient, solves the
problem of extraneous inflections, but also removes the smoothness
constraint and leads to gradient discontinuities
\cite[]{galilee}. A remedy, suggested by \cite{schweikert},
is known as \emph{splines in tension}. Splines in tension are
constructed by minimizing a modified quadratic form that includes a
tension term. Physically, the additional term corresponds to tension
in elastic plates \cite[]{timoshenko}. \cite{GEO55-03-02930305}
developed a practical algorithm of 2-D gridding with splines in
tension and implemented it in the popular GMT software
package.
\par
In this section, we develop an application of helical preconditioning
to gridding with splines in tension. We accelerate an iterative data
regularization algorithm by recursive preconditioning with
multidimensional filters defined on a helix \cite[]{GEO68-02-05770588}. The
efficient Wilson-Burg spectral factorization constructs a
minimum-phase filter suitable for recursive filtering.

We introduce a family of 2-D
minimum-phase filters for different degrees of tension.  The filters
are constructed by spectral factorization of the corresponding
finite-difference forms. In the case of zero tension (the original
minimum-curvature formulation), we obtain a minimum-phase version of
the Laplacian filter. The case of infinite tension leads to spectral
factorization of the Laplacian and produces the \emph{helical
  derivative} filter \cite[]{iee}.
\par
The tension filters can be applied not only for data regularization
but also for preconditioning in any estimation problems with smooth
models. Tomographic velocity estimation is an obvious example of such
an application \cite[]{SEG-1998-1218}.

\subsection{Mathematical theory of splines in tension}

The traditional minimum-curvature criterion implies seeking a
two-dimensional surface $f(x,y)$ in region $D$, which corresponds to
the minimum of the Laplacian power:
\begin{equation}
  \label{eqn:l2}
  \iint\limits_{D} \left|\nabla^2 f(x,y)\right|^2\,dx\,dy\;,
\end{equation}
where $\nabla^2$ denotes the Laplacian operator: $ \nabla^2 =
\frac{\partial^2}{\partial x^2} + \frac{\partial^2}{\partial y^2}$.
\par
Alternatively, we can seek $f(x,y)$ as the solution of the biharmonic
differential equation
\begin{equation}
  \label{eqn:bi}
  (\nabla^2)^2 f(x,y) = 0\;.
\end{equation}
\cite{fung} and \cite{GEO39-01-00390048} derive
equation~(\ref{eqn:bi}) directly from~(\ref{eqn:l2}) with the help of
the variational calculus and Gauss's theorem.
\par
Formula~(\ref{eqn:l2}) approximates the strain energy of a thin
elastic plate \cite[]{timoshenko}. Taking tension into account modifies
both the energy formula~(\ref{eqn:l2}) and the corresponding
equation~(\ref{eqn:bi}). \cite{GEO55-03-02930305} suggest the
following form of the modified equation:
\begin{equation}
  \label{eqn:bit}
  \left[(1-\lambda) (\nabla^2)^2 - \lambda (\nabla^2)\right] f(x,y) = 0\;,
\end{equation}
where the tension parameter $\lambda$ ranges from 0 to 1. The
corresponding energy functional is
\begin{equation}
  \label{eqn:l2lam}
  \iint\limits_{D} \left[(1-\lambda)\,\left|\nabla^2 f(x,y)\right|^2
\;+\;
\lambda\,\left|\nabla f(x,y)\right|^2\right]\,dx\,dy\;.
\end{equation}
Zero tension leads to the biharmonic equation~(\ref{eqn:bi}) and
corresponds to the minimum curvature construction. The case of
$\lambda=1$ corresponds to infinite tension. Although infinite tension
is physically impossible, the resulting Laplace equation does have the
physical interpretation of a steady-state temperature distribution. An
important property of harmonic functions (solutions of the Laplace
equation) is that they cannot have local minima and maxima in the free
regions. With respect to interpolation, this means that, in the case
of $\lambda=1$, the interpolation surface will be constrained to have
its local extrema only at the input data locations.

Norman Sleep (2000, personal communication) points out that if the
tension term $\lambda \nabla^2$ is written in the form $\nabla \cdot
(\lambda \nabla)$, we can follow an analogy with heat flow and
electrostatics and generalize the tension parameter~$\lambda$ to a
local function depending on $x$ and $y$. In a more general form,
$\lambda$ could be a tensor allowing for an anisotropic smoothing in
some predefined directions similarly to the steering-filter method
\cite[]{SEG-1998-1851}.

To interpolate an irregular set of data values, $f_k$ at points
$(x_k,y_k)$, we need to solve equation~(\ref{eqn:bit}) under the
constraint
\begin{equation}
  \label{fk}
  f(x_k,y_k) = f_k\;.
\end{equation}
We can accelerate the solution by recursive filter preconditioning. If
$\mathbf{A}$ is the discrete filter representation of the differential
operator in equation~(\ref{eqn:bit}) and we can find a minimum-phase
filter $\mathbf{D}$ whose autocorrelation is equal to $\mathbf{A}$, then
an appropriate preconditioning operator is a recursive inverse
filtering with the filter $\mathbf{D}$. The preconditioned formulation
of the interpolation problem takes the form of the least-squares system 
\cite[]{iee}
\begin{equation}
\mathbf{K}\, \mathbf{D}^{-1} \mathbf{p} \approx  \mathbf{f}_k\;,
\label{eqn:prec2}
\end{equation}
where $\mathbf{f}_k$ represents the vector of known data, $\mathbf{K}$ is
the operator of selecting the known data locations, and $\mathbf{p}$ is
the preconditioned variable: $\mathbf{p} = \mathbf{D\, f}$. After
obtaining an iterative solution of system~(\ref{eqn:prec2}), we
reconstruct the model $\mathbf{f}$ by inverse recursive filtering:
$\mathbf{f} = \mathbf{D}^{-1}\,\mathbf{p}$. Formulating the problem in
helical coordinates \cite[]{helix0,GEO63-05-15321541} enables both the spectral
factorization of $\mathbf{A}$ and the inverse filtering with $\mathbf{D}$.

\subsection{Finite differences and spectral factorization}

\inputdir{tension}

In the one-dimensional case, one finite-difference representation of
the squared Laplacian is as a centered 5-point filter with
coefficients $(1,-4,6,-4,1)$. On the same grid, the Laplacian operator
can be approximated to the same order of accuracy with the filter
$(1/12,-4/3,5/2,-4/3,1/12)$.  Combining the two filters in accordance
with equation~(\ref{eqn:bit}) and performing the spectral
factorization, we can obtain a 3-point minimum-phase filter suitable
for inverse filtering.  Figure~\ref{fig:otens} shows a family of
one-dimensional minimum-phase filters for different values of the
parameter $\lambda$.  Figure~\ref{fig:int} demonstrates the
interpolation results obtained with these filters on a simple
one-dimensional synthetic. As expected, a small tension value
($\lambda=0.01$) produces a smooth interpolation, but creates
artificial oscillations in the unconstrained regions around sharp
changes in the gradient. The value of $\lambda=1$ leads to linear
interpolation with no extraneous inflections but with discontinuous
derivatives. Intermediate values of $\lambda$ allow us to achieve a
compromise: a smooth surface with constrained oscillations.

\sideplot{otens}{width=3in,height=2.5in}{One-dimensional minimum-phase
  filters for different values of the tension parameter $\lambda$. The
  filters range from the second derivative for $\lambda=0$ to the first
  derivative for $\lambda=1$.}

\plot{int}{width=5.5in,height=7.33in}{Interpolating a simple
  one-dimensional synthetic with recursive filter preconditioning for
  different values of the tension parameter $\lambda$. The input data are
  shown on the top. The interpolation results range from a natural
  cubic spline interpolation for $\lambda=0$ to linear interpolation for
  $\lambda=1$.}

\inputdir{Math}

To design the corresponding filters in two dimensions, we define the
finite-difference representation of operator~(\ref{eqn:bit}) on a
5-by-5 stencil. The filter coefficients are chosen with the help of
the Taylor expansion to match the desired spectrum of the operator
around the zero spatial frequency.  The matching conditions lead to
the following set of coefficients for the squared Laplacian:
\begin{center}
\begin{tabular}{|c|c|c|c|c|}
\hline 
-1/60 & 2/5 & 7/30 & 2/5 & -1/60 \\
\hline
2/5 & -14/15 & -44/15 & -14/15 & 2/5 \\
\hline
7/30 & -44/15 & 57/5 & -44/15 & 7/30 \\
\hline
2/5 & -14/15 & -44/15 & -14/15 & 2/5 \\
\hline
-1/60 & 2/5 & 7/30 & 2/5 & -1/60 \\
\hline
\end{tabular}
= 1/60
\begin{tabular}{|c|c|c|c|c|}
\hline 
-1 & 24 & 14 & 24 & -1 \\
\hline
24 & -56 & -176 & -56 & 24 \\
\hline
14 & -176 & 684 & -176 & 14 \\
\hline
24 & -56 & -176 & -56 & 24 \\
\hline
-1 & 24 & 14 & 24 & -1 \\
\hline
\end{tabular}
\end{center}
The Laplacian representation with the same order of accuracy has the
coefficients
\begin{center}
\begin{tabular}{|c|c|c|c|c|}
\hline 
-1/360 & 2/45 & 0 & 2/45 & -1/360 \\
\hline
2/45 & -14/45 & -4/5 & -14/45 & 2/45 \\
\hline
0 & -4/5 & 41/10 & -4/5 & 0 \\
\hline
2/45 & -14/45 & -4/5 & -14/45 & 2/45 \\
\hline
-1/360 & 2/45 & 0 & 2/45 & -1/360 \\
\hline
\end{tabular} =
1/360 
\begin{tabular}{|c|c|c|c|c|}
\hline 
-1 & 16 & 0 & 16 & -1 \\
\hline
16 & -112 & -288 & -112 & 16 \\
\hline
0 & -288 & 1476 & -288 & 0 \\
\hline
16 & -112 & -288 & -112 & 16 \\
\hline
-1 & 16 & 0 & 16 & -1 \\
\hline
\end{tabular}
\end{center}
For the sake of simplicity, we assumed equal spacing in the $x$ and $y$
direction. The coefficients can be easily adjusted for anisotropic
spacing. Figures~\ref{fig:specc} and~\ref{fig:specp} show the spectra
of the finite-difference representations of operator~(\ref{eqn:bit})
for different values of the tension parameter. The
finite-difference spectra appear to be fairly isotropic (independent of
angle in polar coordinates).  They match the exact expressions at small
frequencies.

\plot{specc}{width=4.5in}{Spectra of the finite-difference
  splines-in-tension schemes for different values of the tension
  parameter (contour plots).}

\plot{specp}{width=4.5in}{Spectra of the
  finite-difference splines-in-tension schemes for different values of
  the tension parameter (cross-section plots). The dashed lines show
  the exact spectra for continuous operators.}
\par
Regarding the finite-difference operators as two-dimensional
auto-correlations and applying the Wilson-Burg method of spectral
factorization, we obtain two-dimensional minimum-phase filters
suitable for inverse filtering.  The exact filters contain many
coefficients, which rapidly decrease in magnitude at a distance from
the first coefficient. For reasons of efficiency, it is advisable to
restrict the shape of the filter so that it contains only the
significant coefficients. Keeping all the coefficients that are $1000$
times smaller in magnitude than the leading coefficient creates a
53-point filter for $\lambda=0$ and a 35-point filter for $\lambda=1$,
with intermediate filter lengths for intermediate values of $\lambda$.
Keeping only the coefficients that are $200$ times smaller that the
leading coefficient, we obtain 25- and 16-point filters for
respectively $\lambda=0$ and $\lambda=1$.  The restricted filters do
not factor the autocorrelation exactly but provide an effective
approximation of the exact factors. As outputs of the Wilson-Burg
spectral factorization process, they obey the minimum-phase condition.

\inputdir{gtens}

\plot{splin}{width=6in,height=6in}{Inverse filtering with the tension
  filters. The left plots show the inputs composed of filters and
  spikes. Inverse filtering turns filters into impulses and turns
  spikes into inverse filter responses (middle plots). Adjoint
  filtering creates smooth isotropic shapes (right plots). The tension
  parameter takes on the values 0.3, 0.7, and 1 (from top to bottom).
  The case of zero tension corresponds to Figure~\ref{fig:thin42}.}
\par
Figure~\ref{fig:splin} shows the two-dimensional filters for different
values of $\lambda$ and illustrates inverse recursive filtering, which
is the essence of the helix method \cite[]{GEO63-05-15321541}. The case of
$\lambda=1$ leads to the filter known as \emph{helix derivative}
\cite[]{iee}.  The filter values are spread mostly in two columns. The
other boundary case ($\lambda=0$) leads to a three-column filter,
which serves as the minimum-phase version of the Laplacian. This
filter is similar to the one shown in Figure~\ref{fig:thin42}.  As
expected from the theory, the inverse impulse response of this filter
is noticeably smoother and wider than the inverse response of the
helix derivative.  Filters corresponding to intermediate values of
$\lambda$ exhibit intermediate properties.  Theoretically, the inverse
impulse response of the filter corresponds to the Green's function of
equation~(\ref{eqn:bit}). The theoretical Green's function for the
case of $\lambda=1$ is
\begin{equation}
  \label{eqn:g1}
  G = \frac{1}{2\pi}\ln{r}\;,
\end{equation}
where $r$ is the distance from the impulse:
$r=\sqrt{\left(x-x_k\right)^2 + \left(y-y_k\right)}$. In the case of
$\lambda=0$, the Green's function is smoother at the origin:
\begin{equation}
  \label{eqn:g2}
  G = \frac{1}{8\pi}r^2\ln{r}\;.
\end{equation}
The theoretical Green's function expression for an arbitrary value of
$\lambda$ is unknown\footnote{\cite{mitas} derive an analytical Green's
  function for a different model of tension splines using special functions.}, 
but we can assume that its smoothness lies
between the two boundary conditions.
\par
In the next subsection, we illustrate an application of helical inverse
filtering to a two-dimensional interpolation problem.

\subsection{Regularization example}

We chose an environmental dataset \cite[]{iee} for a simple illustration
of smooth data regularization. The data were collected on a bottom
sounding survey of the Sea of Galilee in Israel \cite[]{zvi}.  The data
contain a number of noisy, erroneous and inconsistent measurements,
which present a challenge for the traditional estimation methods
\cite[]{galilee}.
%Addressing this challenge completely goes beyond the scope of this
%paper.
\par
Figure~\ref{fig:mesh} shows the data after a nearest-neighbor binning
to a regular grid. The data were then passed to an interpolation
program to fill the empty bins. The results (for different values of
$\lambda$) are shown in Figures~\ref{fig:gal} and~\ref{fig:cross}.
Interpolation with the minimum-phase Laplacian ($\lambda=0$) creates a
relatively smooth interpolation surface but plants artificial
``hills'' around the edge of the sea. This effect is caused by large
gradient changes and is similar to the sidelobe effect in the
one-dimensional example (Figure~\ref{fig:int}).  It is clearly seen in
the cross-section plots in Figure~\ref{fig:cross}.  The abrupt
gradient change is a typical case of a shelf break. It is caused by a
combination of sedimentation and active rifting. Interpolation with
the helix derivative ($\lambda=1$) is free from the sidelobe
artifacts, but it also produces an undesirable non-smooth behavior in
the middle part of the image. As in the one-dimensional example,
intermediate tension allows us to achieve a compromise: smooth
interpolation in the middle and constrained behavior at the sides of
the sea bottom.

\sideplot{mesh}{width=3in,height=4in}{The Sea of Galilee dataset after
  a nearest-neighbor binning. The binned data is used as an input for
  the missing data interpolation program.}

\plot{gal}{width=5.5in,height=7.33in}{The Sea of Galilee dataset after
  missing data interpolation with helical preconditioning. Different
  plots correspond to different values of the tension parameter. An
  east-west derivative filter was applied to illuminate the surface.}

\plot{cross}{width=6in,height=6in}{ Cross-sections of the Sea of
  Galilee dataset after missing-data interpolation with helical
  preconditioning.  Different plots correspond to different values of
  the tension parameter.}


\section{Conclusions}

\begin{comment}
We observe a significant (order-of-magnitude) speed-up in the
optimization convergence when preconditioning interpolation problems
with inverse recursive filtering. Since inverse filtering takes almost
the same time as forward convolution, this speed-up translates
straightforwardly into computational time savings.

The savings are hardly noticeable for simple test problems, but they
can have a direct impact on the mere feasibility of iterative
least-squares inversion for large-scale (seismic-exploration-size)
problems.
\end{comment}

The Wilson-Burg spectral factorization method, presented in this paper, allows
one to construct stable recursive filters. The method appears to have
attractive computational properties and can be significantly more efficient
than alternative spectral factorization algorithms. It is particularly
suitable for the multidimensional case, where recursive filtering is enabled
by the helix transform.

We have illustrated an application of the Wilson-Burg method for efficient
smooth data regularization. A constrained approach to smooth data
regularization leads to splines in tension. The constraint is embedded in a
user-specified tension parameter. The two boundary values of tension
correspond to cubic and linear interpolation.  By applying the method of
spectral factorization on a helix, we have been able to define a family of
two-dimensional minimum-phase filters, which correspond to the spline
interpolation problem with different values of tension.  We have used these
filters for accelerating data-regularization problems with smooth surfaces by
recursive preconditioning. In general, they are applicable for preconditioning
acceleration in various estimation problems with smooth models.

\section{Acknowledgments}
This paper owes a great deal to John Burg. We would like to thank him
and Francis Muir for many useful and stimulating discussions. The
first author also thanks Jim Berryman for explaining the variational
derivation of the biharmonic and tension-spline equations. Ralf Ferber and Ali
\"{O}zbek provided helpful reviews.

The financial support for this work was provided by the sponsors of
the Stanford Exploration Project.

\bibliographystyle{seg}
\bibliography{SEG,burg}

%\plot{name}{width=6in,height=}{caption}
%\sideplot{name}{height=1.5in,width=}{caption}
%
%\begin{equation}
%\label{eqn:}
%\end{equation}
%
%\ref{fig:}
%(\ref{eqn:})

%%% Local Variables: 
%%% mode: latex
%%% TeX-master: t
%%% TeX-master: t
%%% End: 
