\lefthead{Chemingui}
\righthead{Fractal media}
\footer{SEP--80}
%
\title{Modeling 3-D anisotropic fractal media}
%
\author{Nizar Chemingui}
%
\begin{abstract}
 This paper presents stochastic descriptions of anisotropic fractal media. 
Second order statistics are used to represent the continuous random field as a 
stationary zero-mean process completely specified by its two-point covariance
function. 
In analogy to the two-dimensional Goff and 
Jordan model for seafloor morphology, I present the von Karman functions as
a generalization to media with exponential correlation functions. 
I also compute a two-state
model by mapping the random field from continuous realizations to a binary 
field. The method can find application in modeling impedances from fractal
media and in fluid flow problems. 
\end{abstract}

\section{Introduction}

Our understanding of the physical phenomena occurring in the earth always 
involves the study of the medium itself. Unfortunately, the earth offers
an unusually complicated medium in which heterogeneities are observed at every
scale. Sometimes the problem is too difficult to deal with deterministically
but it turns out to be quite simply treated by statistical methods. 
Solutions to the one-dimensional problem have targeted the study of reflectivity
series obtained from well logs. 
Seismic impedance can be modeled as a special type of Markov
chain, one which is constrained to have a purely exponential correlation
function \cite{Godfrey}. 
The two-dimensional problem has gained a lot of attention in the 
recent years from studies of seismic scattering in heterogeneous media, e.g.,  
\cite{Wu,Frankel,Goff,Holliger1,Holliger2}.  
Three dimensional simulations can 
be used in fluid flow experiments \cite{Popovici.sep.60.295}.\\ 
\par
This paper presents
a method for simulating three-dimensional anisotropic random fields 
using second order-statistics. The method was introduced by Goff and Jordan
(1988) to model a two-dimensional seafloor morphology. 
I have considered the cases of random media 
characterized by Gaussian, exponential and von Karman correlation functions.
I use the von Karman functions as a generalizations to the exponential 
correlation functions in modeling random sequences. 
This type of correlation function was first introduced by von Karman (1948)
for characterizing the random velocity field of a turbulent medium. It has 
since been frequently used in the statistical literature, studies
of turbulence problems, e.g.\cite{Tatarski},
and studies of random media such as wave scattering, e.g.\cite{Chernov}.
The von Karman functions were identified specifically as belonging to the
class of continuous correlation functions \cite{Matern}.
Holliger et al. (1993) used von Karman covariance functions to model binary
fields and defined ``binarization'' as a mapping of all values in
a continuous field to just two values of the new field.
I have employed their technique to model
two state models (i.e, rock/pore or sandstone/shale) from 
continuous realizations and test the
increase in medium roughness through the ``binarization'' process.
  
\section{RANDOM FIELDS}

  A stochastic model is constructed for the properties of the random medium.
We first construct a distribution function P(x)
for the properties of the medium h(x). From such a probability function, we
can recover the statistical properties of the distribution (i.e., mean, variance
, etc.) through its N-point statistical moments \cite{Goff}.
\begin{eqnarray}
 C_N(x_1,x_2,...,x_N)& =& <h(x_1)...h(x_N)> \nonumber \\
                     & =& \int_{-\infty}^{+\infty}...\int_{-\infty}^{+\infty}h_1...h_NP(
h_1, ...,h_N)dh_1...dh_N \label {eq1}
\end{eqnarray}
where $h_N=h(x_N)$. The key assumption of spatial homogeneity (stationarity) means that the
N-point moments are taken to depend only on the vector joining these points
and not on their absolute positions. 
These moments describe the magnitude and smoothness of the fluctuations
of h(x).

\subsection{Second-order Statistics}

\par

I restricted this research to the study of second-order statistics of random
fields. This means the study of random media characterized by Gaussian 
distributions, where a Gaussian process is completely specified by its first-
and second-order moments. Furthermore if I define the field $h(x)$ to be 
a zero mean process:
\begin{equation}
<h(x)>=\int_{-\infty}^{+\infty}h(x)P(h(x))dh=0 \
\label {eq3}
\end{equation}
then $h(x)$ is fully described by its two-point moment that is its autocovariance 
function which we write as a function of the correlation function:
\begin{equation}
C_{hh}(r)=E[h(x)h(x+r)]=H^2\rho_{hh}(r) \
\label {eq4}
\end{equation}
where $P(h(x))$ is the probability density function of $h(x)$,
$r$ is the lag vector, $E$ is the expected value, $H^2$ is the variance
(i.e. $C_{hh}(0)$) and $\rho_{hh}$ is the three-dimensional correlation function.
Equation~(\ref {eq4}) shows that the random medium can 
be adequately specified by its 
autocorrelation function. More generally, an anisotropic random field can be
described by a monotonically decaying autocorrelation function whose rate of
decay depends on direction. The roughness of the medium is function of the 
decay rate of the correlation. The Fourier transform of the autocorrelation is 
the power spectrum of the field \cite{Bracewell}. Three types of correlation
functions were commonly used in the field of seismic modeling: the Gaussian,
exponential and von Karman functions. These special functions are described by
analytic expressions of their autocorrelations and Fourier transforms. 
\par
Two-dimensional cases have been studied for some
time\cite{Wu,Frankel,Goff,Holliger1,Holliger2}.
Within the last several years, computer capacity and speed have grown rapidly.
It is now feasible to extend our models and simulations to the 
three-dimensional case.

\subsection { von Karman correlation functions }
\inputdir{XFig}

The three-dimensional anisotropic von Karman function is given by \cite{Goff}:
\begin{equation}
C(r)=\frac{4\pi{\nu}H^2r^{\nu}K_{\nu}(r)}{K_{\nu}(0)} \
\label {eq7}
\end{equation}
and its three-dimensional Fourier transform is:
\begin{equation}
P(k)=\frac{4\pi{\nu}H^2}{K_{\nu}(0)}\frac{a_x^2+a_y^2+a_z^2}{{(1+k^2)}^{\nu+\frac {3} {2}}} \
\label {eq8}
\end{equation}
where $r=\sqrt{\frac{x^2}{a_x^2}+\frac{y^2}{a_y^2}+\frac{z^2}{a_z^2}}$,
 $k=\sqrt{k_x^2a_x^2+k_y^2a_y^2+k_z^2a_z^2}$;   
$a_x$, $a_y$ and $a_z$ are the characteristic scales of the medium along the 
3-dimensions and $k_x$, $k_y$ and $k_z$ are the wavenumber components. 
$K_\nu$ is the modified Bessel function of order $\nu$, 
where $0.0<\nu<1.0$ is the 
Hurst number \cite{Mandelbrot1,Mandelbrot2}. The fractal dimension of a stochastic
field characterized by a von Karman autocorrelation is given by:
\begin{equation}
D=E+1-\nu \
\label {eq0}
\end{equation}
where $E$ is the Euclidean dimension i.e., $E=3$ for the three-dimensional problem.
The special case of $\nu=0.5$ yields to the exponential covariance 
function that corresponds to a Markov process \cite{Feller}.
\begin{equation}
C(r)=H^2e^{-r} \
\label {eq9}
\end{equation}
whose three-dimensional Fourier transform is given by:
\begin{equation}
P(k)=H^2\frac{a_x^2+a_y^2+a_z^2}{{(1+k^2)}^{2}} \
\label {eq10}
\end{equation}
%
\plot{karman}{height=3.5in,width=5.in}{Comparison of 1-dimensional isotropic von Karman autocorrelation functions for varying hurst number, $\nu$.}
%
Decreasing the Hurst number, $\nu$, increases the roughness of the medium.
The limiting cases of unity and zero correspond to a smooth Euclidean random 
field and a space-filling field respectively.\\
%
\par
Figure \ref{fig:karman} shows the one-dimensional isotropic von Karman correlation 
function plotted
for different values of $\nu$. The functions have exponential behavior 
but different decay rates.
The higher the slope, the rougher the medium (i.e., the lower is $\nu$).
The exponential behavior is explained by the modified Bessel functions 
$K_\nu(x)$ which in the region $x \gg \nu$ behave as
\begin{equation}
K_\nu(x) \approx \frac {1} {\sqrt{2\pi x}} \exp{(x)} \
\label {eq11}
\end{equation}
%
For comparison of the results, I also include the anisotropic Gaussian 
autocovariance function, which in 3-D has the familiar form:
\begin{equation}
C(r)=H^2e^{-r^2} \
\label {eq5}
\end{equation}
and its 3-dimensional Fourier transform is given by:
\begin{equation}
P(k)=\frac{a_xa_ya_z}{2}H^2e^{\frac{k^2}{4}} \
\label {eq6}
\end{equation}

\section{ FORWARD MODELING }

Continuous random
fields have frequently been used for statistical analyses, modeling perturbed
media, scattering and diffraction studies, fluid flow simulations and,
other related problems. Numerical realizations may describe the 
statistical character of 
random models at all scales. 

\subsection{ Numerical Implementation }
\inputdir{random}

The generation of synthetic random media is done in the wave number domain. 
First, we compute the power spectrum of the field, i.e, the Fourier spectrum of
the autocorrelation function. Then we compute the Fourier spectrum by
multiplying the square root of the power spectrum by a random phase factor 
$e^{2\pi\eta}$ where $\eta$ is a uniform deviate that lies 
in the interval [0,1).
In a final step we apply an inverse fast Fourier transform to 
obtain the spatial domain representation of the random medium.
The numerical implementation of the method is very straightforward, although
special care is required to handle D.C. and Nyquist wavenumbers.\\
Algorithms are similar for the one-, two- and three-dimensional problem
although if we do not care about computer expenses, 1- and 2-D random
sequences can be simply extracted as arrays or sections from 3-D
simulations. 

\plot{gauss}{height=3.0in}{Synthetic random field with anisotropic
Gaussian autocorrelation function; $a_x=15$, $a_y=25$, $a_z=35$.}
\plot{expo}{height=3.0in}{Synthetic random field with anisotropic
exponential autocorrelation function; $a_x=15$, $a_y=25$, $a_z=35$.}
\plot{fractal}{height=3.0in}{Synthetic random field with anisotropic
von Karman autocorrelation function; $a_x=15$, $a_y=25$, $a_z=35$.}

\subsection{ Modeling 3-D random media}

I show three different realizations of an anisotropic model with different
aspect ratios along the three coordinate axes. The model is a 64 points cube 
with characteristic scales $a_x=15$, $a_y=25$, and $a_z=35$. The media are characterized 
by Gaussian, exponential, and van karman autocorrelation functions, respectively.
We notice the increase in model roughness as we move from the Gaussian medium
to the exponential field (i.e, $\nu=.5$) and then to the von Karman field with
$\nu=.2$.\
\par
In the physical world, these fields may represent media at different scales
varying from the microscopic to the megascopic. 

\subsection{ Modeling seismic impedances }
\inputdir{XFig}

Seismic impedances have frequently been modeled as a Markov process. Godfrey
et. al. (1980) modeled impedance as a special type of Markov chain, one
that is constrained to have a purely exponential correlation function.
They tested their method on three actual logs and compared the autocorrelation
function to a best fit exponential curve. Apart from a small geologic noise
component at the origin, their results showed excellent agreement between
the theoretical exponential and the actual autocorrelation on two of the well
logs they considered. For large lags, the actual correlation function
had exponential behavior similar to that of the theoretical curve, but all data
points fell below the synthetic curve showing a faster decay rate. The behavior
of the autocorrelation could very well be interpreted as related to a rougher
distribution than that predicted by the exponential correlation.
A von Karman correlation function with a Hurst number smaller than 0.5
would have given a better fit to the autocorrelation of the impedance series.
The autocorrelation of the impedance function provides information on the
depositional pattern in the sedimentary column i.e, cyclic or transitional 
\cite {ODA}.\
\par
Figure \ref{fig:imped} shows a comparison of one-dimensional random sequences that simulate
synthetic impedances with von Karman correlation functions of varying
fractal dimensions (i.e., Hurst number $\nu$). Again the smaller
the value of $\nu$, the rougher the sequence. The impedance with
exponential correlation seems smooth compared to the ones
generated from autocorrelation functions with values of $\nu$ lower than 0.5.

\plot{imped}{height=3.5in,width=6.in}{Synthetic random sequences
simulating acoustic impedances with von Karman autocorrelation for varying 
Hurst number $\nu$.}

%\section{From Continuous to Binary Fields}
\section{GENERATING TWO-STATE MODELS}
\inputdir{imped}

 In the geophysical world we often deal with heterogeneous media 
whose inhomogeneities are caused by the presence of two different 
types of material
with different mechanical properties. A typical example is the case of a
stratified formation of shale embedded in sandstone. In fluid flow and 
reservoir engineering problems, the rock samples are generally composed of 
a matrix and pore space. Continuously random fields are therefore
inadequate to describe randomness in similar settings. I seek to describe 
a random field in which the medium can be represented as a two-state model. 
This new field is called a binary field and the 
process of deriving the binary field from the continuous field is called 
``binarization'' \cite{Holliger2}. The problem is to relate the statistics
of the binary field to those of the continuous field. 
Holliger et al. (1993) gave a brief description of their mapped two-dimensional binary field which I 
apply in a straightforward generalization to the three-dimensional problem.\
\par
To illustrate the effects of ``binarizing'' a continuous field, let's consider two
examples of random fields with Gaussian and exponential autocorrelation 
functions, respectively. In the first example I simulate a randomly-stratified medium. The
second example is a realization of a random medium with statistically 
isotropic homogeneous inclusions. I like to analyze the change in the medium
properties by comparing the autocorrelation function of the distribution
before and after ``binarization''. For better observation, I limit the analysis 
to the study of the correlation function along one axis, 
i.e, in the x-direction.
\par
Figure \ref{fig:auto-expo} shows the averaged 1-D correlation 
function along the x-axis 
for the randomly layered medium. The solid curve displays the autocorrelation
of the continuous field; the dashed one represents the autocorrelation
of the ``binary'' field. The two functions are noticeably different from one 
another; the slope near the origin is greatly increased after ``binarization''
indicating
a rougher distribution compared to the continuous case. Figure \ref{fig:auto-gauss} shows
the same observations for the isotropic random field with Gaussian 
autocorrelation; again the roughness
of the field has increased as indicated by the steepening in the slope of the
autocorrelation.
\activeplot{layered}{height=3.0in}{}{Synthetic continuous random field with
apparent layering and Gaussian autocorrelation; $a_x=5$, $a_y=80$, $a_z=80$.}
\activeplot{lay-bin}{height=3.0in}{}{Synthetic binary field derived from the
continuous realization of a layered random field with Gaussian autocorrelation.}
\activeplot{isotropic}{height=3.0in}{}{Synthetic continuous random field with
isotropic Gaussian autocorrelation function; $a_x=15$, $a_y=15$, $a_z=15$. }
\activeplot{iso-bin}{height=3.0in}{}{Synthetic binary field derived from the
continuous realization of a random field with Gaussian autocorrelation.}
\activeplot{auto-expo}{height=2.75in}{}{Autocorrelation functions of the
continuous (solid lines) and binary (dashed lines) fields for the layered
random medium with exponential correlation.} 
\activeplot{auto-gauss}{height=2.75in}{}{Autocorrelation functions of the
continuous (solid lines) and binary (dashed lines) fields for the isotropic
random medium with Gaussian correlation} 
%

\section{CONCLUSIONS}

  In this initial study I have tackled the forward problem for modeling anisotropic
fractal media using second-order statistics. The method has close analogy with
the two-dimensional Goff and Jordan model for seafloor morphology. The 
generation of synthetic models is done in the Fourier domain and the
algorithms are similar for the one- two- and three-dimensional problems. 
The von Karman
functions are presented as a generalization of the exponential correlation function 
associated with the Markov process in modeling seismic impedances. The
von Karman functions can be used for better description of statistic lithology
of stratigraphic columns and understanding their depositional pattern. 
I have also computed a two-state model (i.e., rock/pore or sandstone/shale)
by mapping the random field from continuous realizations into a binary field.
Comparisons of the autocorrelation functions of the continuous and binary
fields show that the fractal dimension (i.e, the roughness of the medium) 
increases through the ``binarization'' process.

\section{ FUTURE WORK}

Future goals of this effort will be to formulate the inverse problem
for estimating the characteristic parameters of the anisotropic fractal medium,
i.e, aspect ratios of anisotropy, and Hausdorff (fractal) dimension.
The technique of deriving the binary field from the continuous random field
should be extended to simulate M-state models, where M is the number of
states or rocks composing an impedance well-log.\
\par
I also need to test the method on actual well-log data and demonstrate a better
fit with von Karman correlation functions compared to the exponential fit.
This would would be the first application of the inverse problem. Two and three dimensional problems can find
application in the field of wave scattering and diffraction and in fluid
flow problems.

\append{von Karman covariance function }

Equation (4) in the text represents the autocovariance of a random medium
of fractal nature. The power spectrum of the field corresponds to 
the Fourier transform of its covariance function:
\begin{equation}
P(k)=\int_{-\infty}^{+\infty}\int_{-\infty}^{+\infty}\int_{-\infty}^{+\infty} C(r)e^{-ik\cdot r} d^3r \
\label {eqA1}
\end{equation}
Using the N-dimensional Hankel transform \cite{lord}, the covariance function
and its Fourier transform can be related as follows:
\begin{equation}
p(k)=(2\pi)^{N/2}k^{-N/2+1}\int_{0}^{\infty} r^{N/2}J_{N/2-1}(rk)C(r)dr \
\label {eqA2}
\end{equation}
\begin{equation}
C(r)=(2\pi)^{-N/2}r^{-N/2+1}\int_{0}^{\infty} k^{N/2}J_{N/2-1}(rk)P(k)dk \
\label {eqA3}
\end{equation}
where $J_{N/2-1}$ is the Bessel function of order $N/2-1$.\\
The covariance $C(r)$ in (3) is specified in terms of the function:
\begin{equation}
G_\nu (r)=r^\nu K_\nu (r) \hspace{.3in} 0\leq r < \infty \hspace{.3in} \nu \in [0,1]\ 
\label {eqA4}
\end{equation} 
whose Hankel transform pair has been derived by Lord (1954):
\begin{equation} 
P(k)=\frac {\Gamma{(\nu +N/2)}} {2^{1-N-\nu}\pi{^N/2}} {(1+k^2)}^{-\nu-N/2} \ 
\label {eqA5} 
\end{equation}
where $\Gamma$ is the gamma function defined as:
\begin{equation}
\Gamma (z)= \int_{0}^{\infty} t^{z-1}e^{-t} dt \
\label {eqA6}
\end{equation}
Finally if we normalize $G_\nu (r)$ by $G_\nu (0)$ as in Goff and Jordan (1988), we 
obtain, for the three-dimensional case, the power spectrum of the field 
whose covariance is defined by (4):
\begin{equation}
P(k)=4\pi\nu H^2 {(1+k^2)}^{(-\nu-3/2)} \
\label {eqA7}
\end{equation}

\putbib[SEP,../nizar2/random]
%\putbib[SEP,random]

