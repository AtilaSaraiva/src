\published{SEP report, 105, 67-78, (2000)}

% Started 06/18/2000

%\shortnote
\lefthead{Clapp}
\righthead{Multiple realizations}
\footer{SEP--105}
%\def\figdir{./../bob2/Figs}
\def\beq{\begin {equation}}
\def\eeq{\end {equation}}

\def\bec{\begin {center}}
\def\eec{\end {center}}

\def\D{\displaystyle}
\def\R{\rule [-3mm]{0mm}{8mm}}

\email{bob@sep.stanford.edu}
\title{Multiple realizations using standard inversion techniques}


%\chapter{2-D field tests}
%\label{chap:field2}
\author{Robert G. Clapp}
\maketitle

\begin{abstract}

Stacking operators are widely used in seismic imaging and seismic data
processing. Examples include Kirchhoff datuming, migration, offset
continuation, DMO, and velocity transform. Two primary approaches
exist for inverting such operators. The first approach is iterative
least-squares optimization, which involves the construction of the adjoint
operator. The second approach is asymptotic inversion, where an approximate
inverse operator is constructed in the high-frequency asymptotics.  Adjoint
and asymptotic inverse operators share the same kinematic properties, but
their amplitudes (weighting functions) are defined differently. This paper
describes a theory for reconciling the two approaches. I introduce a pair of
the {\em asymptotic pseudo-unitary} operators, which possess both the property
of being adjoint and the property of being asymptotically inverse. The
weighting function of the asymptotic pseudo-unitary stacking operators is
shown to be completely defined by the derivatives of the operator
kinematics. I exemplify the general theory by considering several particular
examples of stacking operators. Simple numerical experiments demonstrate a
noticeable gain in efficiency when the asymptotic pseudo-unitary operators are
applied for preconditioning iterative least-squares optimization.

\end{abstract}


\section{Introduction}
%In Chapter~\ref{chap:reg} and Appendix~\ref{chap:nonstat} I introduced
%several different methods to characterize the covariance. 
%In each case I presented  a single answer to the interpolation or
%tomography problem.  
\par
When solving a missing data problem, geophysicists and geostatisticians have
very similar strategies.
Each use the known data to characterize the model's covariance.  At SEP we
often characterize the covariance through Prediction Error Filters (PEFs)
\cite[]{gee}.  Geostatisticians build variograms  from the known data
to represent the model's covariance \cite[]{geostat}.  
Once each has some measure of the model covariance they attempt to fill
in the missing data. Here their goals slightly diverge.
The geophysicist solves a global estimation problem and
attempts to create a model whose 
covariance is equivalent to the covariance of the known data.
The geostatistician performs kriging, solving a series of
local estimation problem. Each model estimate
is the linear combination of nearby data points
that best fits their predetermined covariance estimate.
Both of these approaches are in some ways exactly what we want:
given a problem give me  `the answer'.  
\par
The single solution approach
however has a couple significant drawbacks.  First, the solution tends to 
have low spatial frequency. Second, it does not provide information
on model variability or provide error bars on our model estimate.
Geostatisticians  have these
abilities  in their repertoire through
what they refer to as `multiple realizations' or `stochastic simulations'.
They introduce a random component, based on
properties (such as variance) of the data,   to their estimation procedure.  
Each realization's frequency content is more accurate and by
comparing and contrasting  the equiprobable realizations,
model variability can be assessed.
\par
In this paper I present a method to achieve the same goal
using a formulation that better fits into geophysical techniques.  
I modify the model styling goal,
replacing the zero vector with a random vector.  I show  how
the resulting models have a more pleasing texture and can
provide information on variability.


\section{Motivation}
Regularized linear least squares estimation
problems can be written as minimizing the quadratic
function
\beq
Q(m) = \|\data- \bf L \bf m\|^2 + \epsilon^2 \|\bf A \bf m\|^2
\eeq
where $\data$ is our data, $\bf L$ is our modeling operator,
$\reg$ is our regularization operator and we are inverting
for  a model $\bf m$.
Alternately, we can write  them
in terms of  fitting goals,
\beqa
\data &\approx& \bf L \bf m 
\label{eq:geophysics} \\
\nonumber
\zero &\approx& \epsilon  \reg \bf m 
,
\eeqa
For the purpose of this paper I will refer to the first
goal as the {\it data fitting goal} and the second as
the {\it model styling goal}.
Normally we think of {\it data fitting goal } as describing
the physics of the problem.  The {\it model styling goal} is suppose
to provide information about the model character.  Ideally 
$\reg$ should be the inverse model covariance.   In practice
we don't have the model covariance so we attempt  to approximate it through
another operator.   At SEP the regularization operator is  typically one
of the following:
\begin{description}
\item [Laplacian or gradient]  a simple operator that assumes nothing about the model
\item [Prediction Error Filter (PEF)]  a stationary operator  estimated from known portions of the model or  some field with the same properties as the model \cite{gee}
\item [steering filter] a non-stationary operator built from minimal information about the model \cite{Clapp.sep.95.bob1}
\item [non-stationary PEF] a non-stationary operator built from a field with the same properties as the model \cite{Crawley.sep.104}.
\end{description}
A problem with the first three operators is
that while they approximate the model covariance, they
have little concept of model variance.  As a result our model estimates  tend
to have the wrong statistical properties.



\section{Missing Data}
\inputdir{miss}

The missing data problem is probably the simplest to understand and
interpret results.  
We  begin by binning our data onto a regular mesh.
For $\mathbf{L}$ in fitting goals (\ref{eq:geophysics}) we will use a selector 
matrix $\mathbf{J}$,
which is `1' at locations where we have data and `0' at unknown locations.
As an  example, let's  try to interpolate 
a day's worth of  data
collected by SeaBeam (Figure~\ref{fig:sea.init}), which measures
water depth under and to the side of a  ship \cite{gee}. 

\sideplot{init}{width=3.0in,height=3.0in}{Depth of the ocean under ship tracks.}

Figure~\ref{fig:sea.pef} shows the result of estimating a PEF from the
known data locations and then using it to interpolate the entire mesh.
Note how the solution has a lower spatial frequency as we move away from
the recorded data. In addition, the original tracks of the ship are still
clearly visible.  

\sideplot{pef}{width=3.0in,height=3.0in}{Result of using a PEF to
interpolate Figure~\ref{fig:sea.init}, taken from GEE \cite{gee}.}

\par
If we look at a histograms of the known data and our estimated data we
can see the effect of the PEF.  The histogram of the known data has a
nice Gaussian shape.  The predicted
data is much less Gaussian with a much lower variance.  We want estimated 
data to have the same statistical properties as the known data (for
a Gaussian distribution this means matching the mean and variance).
\sideplot{histo}{width=3.0in,height=3.0in}{Histogram for the known data 
(solid lines) and the estimated data (`*').  Note the dissimilar
shapes.}
\par
Geostatisticians are confronted with the same problem. They can produce
smooth, low frequency models through kriging, but must add a little
twist to get model with the statistical properties as the data.
To understand how, a brief review of kriging is necessary.
Kriging estimates each model point by a linear combination of nearby data 
points. For simplicity lets assume that the data has a standard
normal distribution.
The geostatistician find all of the points $m_1 .... m_n$ around the point they
are trying to estimate $m_0$. The vector distance between all data points
$\mathbf{d}_{ij}$
and each data point and the estimation point $\mathbf{d}_{i0}$ are then computed.
Using the predefined covariance function estimate $C$, a covariance
value is then extracted
between all known point pairs $C_{ij}$ and
between known points and
estimation point $C_{i0}$ at the given distances $\mathbf{d}_{ij}$  and
$\mathbf{d}_{i0}$   (Figure~\ref{fig:covar-def}).
They compute the weights   ($w_1 ... w_n$) by solving the set
of equations implied by
\begin{equation}
\left[
\begin{array}{cccc}
C_{11}  &...& C_{1n} & 1 \\
.  &...& . & . \\
.  &...& . & . \\
.  &...& . & . \\
C_{n1}  &...& C_{nn} &  1 \\
1 &...& 1 & 0
\end{array}
\right]
\left[
\begin{array}{c}
w_1 \\
. \\
. \\
. \\
w_n \\
\mu
\end{array}
\right]
=
\left[
\begin{array}{c}
C_{10} \\
. \\
. \\
. \\
C_{n0} \\
1
\end{array} \label{eq:krig}
\right] .
\end{equation}
Estimating $m_0$ is then simply,
\beq
m_0= \sum_{i=1}^{n} w_i m_i.
\eeq
To guarantee that the matrix in equation (\ref{eq:krig})
is invertible geostatisticians approximate
the covariance function
through a linear combination of a limited set of
functions that guarantee that the matrix in equation (\ref{eq:krig}) is
positive-definite and therefore invertible.
% minimizes the variance of 
%Kirgging  models tend to have the same
%low spatial frequency problems we see in the geophysic's 
%approach (Figure~\ref{fig:sea.pef}).

\inputdir{XFig}
\plot{covar-def}{width=6.0in,height=3.0in}{Definition of the terms
in equation (\ref{eq:krig}). A vector is drawn between two points.  The covariance
at the angle and distance describing the vector is then selected.}
\inputdir{miss}

The smooth models provided by kriging  often prove
to be poor representations of earth properties.
A classic example is fluid flow where kriged models  tend to give inaccurate
predictions. The geostatistical solution
is to perform Gaussian stochastic simulation, rather than kriging, to
estimate the field \cite{geostat2}.
There are two major  differences between kriging and simulation. 
The primary difference
is that a random component is introduced into the estimation process.   
Stochastic simulation, or  sequential Gaussian simulation, begins
with a random point being selected in the model space.
They then perform kriging, obtaining
a kriged value $m_0$ and  a kriging variance $\sigma_k$.
Instead of using $m_0$ for the model value we
select a  random number $\beta$
from a  normal distribution.
We use as our model point estimate $m_i$,
\beq
m_i = m_0 + \sigma_k \beta.
\eeq
We then select a new point in the model space and repeat the procedure.
To preserve spatial variability,  a second change is made: 
all the previously estimated points are treated as `data' when estimating
new points guaranteeing that the model matches the covariance estimate. 
By selecting different random numbers (and/or visiting model points
in a different order)  we will get a different, equiprobable model
estimate.
The advantage of the models estimated through simulation is that they
have not only the covariance of they data, but also the variance.
As a result the models estimated by simulation
give more realistic fluid flow measurements compared to a kriged model.
In addition, by trying different realizations fluid flow variability
can be assessed.
\par
%
%In geophysics we can get a measure of something akin to error
%variance by looking at the residual at known data locations
%
%
% Much more difficult with more complex operators (we do not 
% have the luxury of having known locations)
%
%
%
%
%
%
The difference between kriging and simulation has a corollary in our
least squares estimation problem. To see how let's write
our fitting goals in a slightly different format,
\beqa
\bf r_d \pox \mathbf{d}  - \mathbf{J}  \mathbf{m}  \nonumber \\ 
\bf r_m \pox \epsilon \reg \mathbf{m}\label{eq:rbar},
\eeqa
where $\mathbf{r}_d$ is our data residual and $\mathbf{r}_m$ is our model
residual.  The model residual is the result of applying our
covariance estimate $\reg$ to our model estimate.  The larger
the value of a given $\mathbf{r}_m$, the less that model point makes
sense with its surrounding points, given our idea of covariance.
This is similar to kriging variance.
It follows that we might be able to obtain something similar
to the geostatistician's simulations by rewriting our fitting
goals as 
\beqa
\mathbf{d} \pox  \mathbf{J}  \mathbf{m}  \nonumber \\ 
\sigma \mathbf{v} \pox \epsilon \reg \mathbf{m}\label{eq:rand},
\eeqa
where $\mathbf{v}$ is a vector of random normal numbers and $\sigma$ is 
a measure of our estimation uncertainty\footnote{For the missing
data problem $\epsilon$ could be used exclusively.
As  our data fitting goal becomes more complex,
having a separate  $\sigma$ and $\epsilon$ becomes useful.}.

By adjusting $\sigma$
we can change the distribution of
$\mathbf{m}$. For example, let's return to the SeaBeam example.
Figure~\ref{fig:distrib} shows four different model estimations
using a normal distribution and various values for the variance.
Note how the texture of the model changes significantly. If we look
at a histogram of the various realizations (Figure~\ref{fig:distir}),
we see that the correct
distribution is somewhere between our second and third realization.
\par 
We can get an estimate of $\sigma$, or in the case of the missing
data problem $\frac{\sigma}{\epsilon}$, by applying fitting goals 
(\ref{eq:rbar}). If we look at the variance of the model residual $\sigma(\mathbf{m}_r)$  and 
$\sigma(\mathbf{d})$ we can get a good estimate of $\sigma$,
\beq
\sigma = \frac{ \sigma(\mathbf{r}_m) }{ \sigma(\mathbf{d})} \label{eq:sigma.calc} .
\eeq

%At this stage my method for choosing
%$\sigma$ is more guess work than science. The general range of
%$\sigma$, for the missing data problem, can be found by looking at
%the model residuals using $\sigma=0$. An acceptable $\sigma$ is usually
%within a factor of two of the median absolute value of the residuals
%at `known' model locations.  
%
%Choosing an acceptable value of $\sigma$ is sh
%Using the same normal score transform common in geostatistics we
%could emulate any distribution.

\plot{distrib}{width=6in,height=4.5in}{Four different realizations
with increasing $\sigma$  in fitting goals (\ref{eq:rand}).}

\sideplot{distir}{width=3.0in,height=3.0in}{Histogram of
the known data (solid line) and the four different realizations of
Figure~\ref{fig:distrib}.}
\par 
Figure \ref{fig:movie} shows eight different realizations with a
random noise level calculated through equation (\ref{eq:sigma.calc}).
Note how we have done a good job emulating the
distribution of the known data. Each image shows some similar
features but also significant differences (especially note
within the `V' portion of the known data).
\par


\plot{movie}{width=6.0in,height=8.0in}{
Eight different realizations of the SeaBeam interpolation problem and
their histograms. 
Note how the realizations  vary away from the known data points.  }

A potentially attractive feature of setting up the problem in this
manner is that it easy to have both  a space-varying covariance function
(a steering filter or non-stationary PEF) along with a non-stationary
variance. Figure~\ref{fig:non-stat} shows the SeaBeam example
again with the variance increasing from left to right.

\sideplot{non-stat}{width=2.5in,height=2.5in}{Realization where
the variance added to the image increases from left to right.}


\section{Super Dix}
\inputdir{dix}

In general the operator $\mathbf{L}$ in fitting goals (\ref{eq:rand}) is much more
complex than the simple masking operator used in the missing
data problem.  One of the most attractive potential uses for a 
range of equiprobable models is in velocity estimation.  
As a result I decided to next test the methodology on one of the simplest
velocity estimation operators, the Dix equation \cite[]{GEO20-01-00680086}.
\par 
Following the methodology of \cite{Clapp.sep.97.bob1},  I
start from a CMP gather $q(t,i)$ moveout corrected with velocity $v$.
A good starting guess for our RMS velocity function
is the maximum ``instantaneous stack energy'',
\begin{equation}
{\rm stack}(t,v) = \sum_{i=0}^n {\rm NMO_v}(q(t,i))  .
\end{equation}

Not all times have reflections so we don't  weight each $v_{\rm rms}(t)$ 
equivalently. 
Instead we introduce 
a diagonal weighting matrix, $\mathbf{W}$,
found from stack energy at each selected $v_{\rm rms}(t)$.

Our data fitting goal becomes
\begin{equation}
\mathbf{0}
\quad\approx\quad
\mathbf{W}
\left[
\mathbf{C u
-
d}
\right] .
\end{equation}
We are multiplying our RMS function by our time $\tau$ so
must make a slight change in our weighting function.
To give early
times approximately the same priority as later times,
we need to multiply our weighting function by the inverse,
\beq
{\mathbf{W}'} =\frac{{\mathbf{W}}}{\tau} .
\eeq
Next we need to add in regularization. I define
a steering filter operator  ${\bf A}$ that influences
the model to introduce velocity changes that follow structural
dip.
I replace
the zero vector with a random vector and  precondition the problem 
\cite[]{Fomel.sep.95.sergey1}  to get
\beqa
\zero &\approx& {\mathbf{W}'} ( \mathbf{C} \prec \pvar - \data )  \nonumber \\
\sigma \bf v &\approx& \epsilon \pvar \label{eq:mydix} .
\eeqa
\par
To test the methodology I took a 2-D line from a 3-D
North Sea dataset provided by Unocal. 	
Figure~\ref{fig:scale} shows four different realizations
with varying levels of $\sigma$.
\plot{scale}{width=6.0in,height=5.5in}{Four different realization
of fitting goals
(\ref{eq:mydix}) with increasing levels of Gaussian noise in $\bf v$.}

I then chose what I considered a reasonable variability level,
and constructed ten equiprobable models (Figure~\ref{fig:dix-real}).  
Note that the general
shapes of the models are very similar.  What we  see are smaller structural
changes.  For example, look at the range between $.7$s and $1.1$s.
Generally each realization tries to put a high velocity layer in
this region, but
thickness and magnitude varies in the different realizations. 
%Figure~\ref{fig:compare-small} shows the two extremes of the realizations
%for this window of the data. The  average velocity in the two panels varies by
%nearly $100m/s$.
%
\plot{dix-real}{width=6.0in,height=5.5in}{Four of the ten
different realization
of fitting goals
(\ref{eq:mydix}) with constant Gaussian noise in $\bf v$.}
%\plot{compare-small}{width=6.0in,height=3.5in}{The left and center panel
%%represent to extremes of ten realization applying fitting goals (\ref{eq:mydix}). The right panel is the difference between these two realizations.}


%\section{Tomography}
%\input{tomo-rand}
\section{Future Work}
In this paper I glossed over several problems.
First, $\sigma$ should be a space-varying function rather than
the constant I proposed. A bootstrap approach (using
the model residual at one non-linear iteration as our guess
at a space-varying $\sigma$) might prove effective but hasn't been
tested.  How to calculate $\sigma$ for the non-missing data problem
is an open question. In the generic geophysical operator $\bf L$, we often
don't know
which model components are estimated through the data fitting
goal and which are estimated by the model styling goal.
Finally, I made the  assumption that I was dealing with
models with  a normal distribution. 
Whether replacing $\bf v$ with another distribution
or using something similar to the geostatistician's {\it normal-score transform}
would be effective in correctly modeling these distributions is unknown.


\section{Conclusions}
I have  demonstrated a new method for creating equiprobable realizations
using standard geophysical inversion techniques. The character
of  resulting models is
much more consistent than models derived by standard techniques.

\bibliographystyle{seg}
\bibliography{SEG,SEP2,bob}
