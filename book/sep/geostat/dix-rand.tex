\inputdir{dix}

In general the operator $\mathbf{L}$ in fitting goals (\ref{eq:rand}) is much more
complex than the simple masking operator used in the missing
data problem.  One of the most attractive potential uses for a 
range of equiprobable models is in velocity estimation.  
As a result I decided to next test the methodology on one of the simplest
velocity estimation operators, the Dix equation \cite[]{GEO20-01-00680086}.
\par 
Following the methodology of \cite{Clapp.sep.97.bob1},  I
start from a CMP gather $q(t,i)$ moveout corrected with velocity $v$.
A good starting guess for our RMS velocity function
is the maximum ``instantaneous stack energy'',
\begin{equation}
{\rm stack}(t,v) = \sum_{i=0}^n {\rm NMO_v}(q(t,i))  .
\end{equation}

Not all times have reflections so we don't  weight each $v_{\rm rms}(t)$ 
equivalently. 
Instead we introduce 
a diagonal weighting matrix, $\mathbf{W}$,
found from stack energy at each selected $v_{\rm rms}(t)$.

Our data fitting goal becomes
\begin{equation}
\mathbf{0}
\quad\approx\quad
\mathbf{W}
\left[
\mathbf{C u
-
d}
\right] .
\end{equation}
We are multiplying our RMS function by our time $\tau$ so
must make a slight change in our weighting function.
To give early
times approximately the same priority as later times,
we need to multiply our weighting function by the inverse,
\beq
{\mathbf{W}'} =\frac{{\mathbf{W}}}{\tau} .
\eeq
Next we need to add in regularization. I define
a steering filter operator  ${\bf A}$ that influences
the model to introduce velocity changes that follow structural
dip.
I replace
the zero vector with a random vector and  precondition the problem 
\cite[]{Fomel.sep.95.sergey1}  to get
\beqa
\zero &\approx& {\mathbf{W}'} ( \mathbf{C} \prec \pvar - \data )  \nonumber \\
\sigma \bf v &\approx& \epsilon \pvar \label{eq:mydix} .
\eeqa
\par
To test the methodology I took a 2-D line from a 3-D
North Sea dataset provided by Unocal. 	
Figure~\ref{fig:scale} shows four different realizations
with varying levels of $\sigma$.
\plot{scale}{width=6.0in,height=5.5in}{Four different realization
of fitting goals
(\ref{eq:mydix}) with increasing levels of Gaussian noise in $\bf v$.}

I then chose what I considered a reasonable variability level,
and constructed ten equiprobable models (Figure~\ref{fig:dix-real}).  
Note that the general
shapes of the models are very similar.  What we  see are smaller structural
changes.  For example, look at the range between $.7$s and $1.1$s.
Generally each realization tries to put a high velocity layer in
this region, but
thickness and magnitude varies in the different realizations. 
%Figure~\ref{fig:compare-small} shows the two extremes of the realizations
%for this window of the data. The  average velocity in the two panels varies by
%nearly $100m/s$.
%
\plot{dix-real}{width=6.0in,height=5.5in}{Four of the ten
different realization
of fitting goals
(\ref{eq:mydix}) with constant Gaussian noise in $\bf v$.}
%\plot{compare-small}{width=6.0in,height=3.5in}{The left and center panel
%%represent to extremes of ten realization applying fitting goals (\ref{eq:mydix}). The right panel is the difference between these two realizations.}
