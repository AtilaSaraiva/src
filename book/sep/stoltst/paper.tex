
% Started 04/19/00

%\shortnote

\lefthead{}
\righthead{}
\published{Journal of Seismic Exploration, v. 9, 319-335, (2001)}
\title{Evaluating the Stolt-stretch parameter}

\email{sergey@sep.stanford.edu}

\author{Sergey Fomel\/\footnotemark[1] and Louis Vaillant\/\footnotemark[2]}

\footnotetext[1]{\emph{Lawrence Berkeley National Laboratory, 1
    Cyclotron Road, Mail Stop 50A-1148, Berkeley, California 94720,
    USA}, formerly \emph{Stanford Exploration Project, Department of
    Geophysics, Stanford University, Stanford, California 94305, USA}}

\footnotetext[2]{\emph{UNILOG - DO SCL, 37 Rue du Rocher, Paris 75008,
    France}, formerly \emph{Stanford Exploration Project, Department
    of Geophysics, Stanford University, Stanford, California 94305,
    USA}}

\maketitle

\begin{abstract}
  The Stolt migration extension to a variable velocity case describes
  the velocity heterogeneity with a constant parameter, which is
  related to the stretch transformation of the time axis.  We exploit
  a connection between modified dispersion relations and nonhyperbolic
  traveltime approximations to derive an explicit expression for the
  stretch parameter. This analytical expression allows one two achieve
  the highest possible accuracy within the Stolt stretch
  approximation. Using a real data example, we demonstrate an
  application of the explicit Stolt stretch formula for an optimal
  partitioning of the migration velocity in the method of cascaded
  migrations.
\end{abstract}

\section{Introduction}

Although Stolt migration is regarded as the fastest of all the known
seismic migration algorithms, it has a limited applicability because
of the intrinsic constant velocity assumption. The time-stretching
trick proposed in Stolt's classic paper \cite[]{GEO43-01-00230048}
provides an approximate extension of the method to a variable velocity
case.  Implicitly, Stolt stretch transforms reflection traveltime
curves to fit an approximate constant velocity pattern
\cite[]{Levin.sep.35.195,Levin.sep.42.373,Claerbout.blackwell.85}. In other
words, the wave equation with variable velocity is transformed by a
particular stretch of the time axis to an approximate differential
equation with constant coefficients. The two constant coefficients are
an arbitrarily chosen frame velocity and a special non-dimensional
parameter ($W$ in Stolt's original notation). In the constant velocity
case $W$ is equal to 1, and the transformed equation coincides with
the exact constant velocity wave equation. In variable velocity media,
$W$ is generally assumed to lie between $0$ and $1$. As shown by
\cite{GEO52-05-06180643}, the cascaded $f$-$k$ migration
approach can move the value of $W$ for each migration in a cascade
closer to 1, thus increasing the accuracy of the Stolt stretch
approximation.

The $W$ factor is defined by \cite{GEO43-01-00230048} as an
approximate average of a complicated function, which depends on both
time and space coordinates and cannot be computed directly.
Therefore, in practice, the estimation of $W$ is always replaced by a
heuristic guess.  That is why \cite{Levin.sep.35.195} jokingly called the
$W$ parameter ``infamous'', and \cite{GEO52-05-06180643} called it
``esoteric.''

In this paper, we use an analytic technique to evaluate the Stolt
stretch parameter explicitly. The main idea is to constrain this
parameter by fitting the Stolt-stretch traveltime function to the
exact one.  It turns out that in the isotropic case, the $W$ parameter
is connected to the ``parameter of heterogeneity''
\cite[]{ussr,Sword.sep.51.313,castle,nmo}. The definition of
heterogeneity is modified for the case of an anisotropic
(transversally isotropic) media.

We demonstrate an application of the Stolt stretch analytical
expression on a real data example from the North Sea. The velocity
profile is optimally partitioned for the method of cascaded migration,
which allows us to image steeply dipping reflectors at the accuracy
comparable to that of the phase-shift method but at a much smaller
cost.

Although Stolt migration is not currently at the forefront of
geophysical research, it is still widely used in practice
\cite[]{yilmaz,yilmazLE} and keeps recurring in different contexts.
\cite{muir} propose a new interpolation scheme for improving the
practical accuracy of the method. \cite{sava} uses a variation of
Stolt migration - Stolt residual migration \cite[]{GEO61-02-06050607} -
in the context of wave-equation migration velocity analysis.

The growth in computer speed does not automatically make fast
algorithms obsolete, because the amount of processed data tends to
grow at the same rate or even faster. The researchers working in the
field of seismic imaging are often interested in the following
questions: What is the fastest possible migration algorithm?  How
accurate can it get? Stolt migration answers the first question.  The
answer to the second question is developed in this paper.

%--------------------------------------------------------
\section{Stolt Stretch Theory Review}

In order to simplify the references, we start with definitions of the
Stolt migration method. The reader familiar with the Stolt stretch
theory can skip this section and go on to new theoretical results in
the next section.

The basic migration theory reduces post-stack migration to a two-stage
process.  The first stage is a downward continuation of the wavefield
in depth $z$ based on the wave equation
\begin{equation}
{\partial^2 P \over \partial x^2} +
{\partial^2 P \over \partial z^2}  \, = \,{1 \over {v^2(x,z)}}\,
{\partial^2 P \over \partial t^2}\;. 
\label{eqn:wave} 
\end{equation}
The second stage is the imaging condition $t=0$ (here the velocity $v$
is twice as small as the actual wave velocity).  Stolt time migration
performs both stages in one step, applying the frequency-domain
operator
\begin{equation}
\tilde{P_0}\left(k_x, \omega_0\right)=
\tilde{P_v}\left(k_x, \omega_v \left(k,\omega_0 \right)\right) 
\,\left| {{d\omega_v \left(k,\omega_0\right)}\over{d\omega_0}}\right|\;,
\label{eqn:stolt} 
\end{equation}
where
\begin{eqnarray*}
\tilde{P_v}\left(k_x, \omega_v\right) & = & \iint P_v\left(x, t_v\right)
\exp{\left(i \omega_v t_v - i k_x x \right)} \,dt_v\, dx \;\;,  \\
\tilde{P_0}\left(k_x, \omega_0\right) & = & \iint P_0\left(x, t_0\right)
\exp{\left(i \omega_0 t_0 - i k_x x \right)} \,dt_v\, dx \;\;, 
\end{eqnarray*}
$P_0\left(x, t_0\right)$ stands for the initial zero-offset (stacked)
seismic 
section defined on the surface $z=0$,  
$P_v\left(x, t_v\right)$ is the time-migrated section, and $t_v$ is
the vertical traveltime
\begin{equation}
t_v=\int_{0}^{z}{dz' \over {v(x,z')}}\;\;.
\label{eqn:tv} 
\end{equation}
The function $\omega_v \left(k,\omega_0 \right)$ in (\ref{eqn:stolt})
corresponds to the dispersion relation of the wave equation
(\ref{eqn:wave}) and in the constant velocity case has the explicit
expression
\begin{equation}
\omega_v \left(k,\omega_0 \right)=\mbox{sign}\left(\omega_0\right)
\sqrt{\omega_0^2 - v^2 k^2}\;\;.
\label{eqn:dispersion} 
\end{equation}
The choice of the sign in equation (\ref{eqn:dispersion}) is essential
for distinguishing between upgoing and downgoing waves. The upgoing
part of the wavefield is the one used in migration.


The case of a varying velocity complicates the frequency-domain
algorithm and therefore requires special consideration.
\cite{GEO43-01-00230048} suggested the following change of the
time variable (referred to in the literature as {\em Stolt stretch}):
\begin{equation}
s(t)={\left({{2 \over v_0^2}\,\int_0^t\eta d \tau}\right)}^{1/2}\;,
\label{eqn:ss} 
\end{equation}
where $v_0$ is an arbitrarily chosen constant velocity, and $\eta$ is a
function defined by the parametric expressions
\begin{equation}
\eta(\zeta)=\int_0^{\zeta} v(x,z) \,dz \;,\;
\tau(\zeta)=\int_0^{\zeta} { {dz} \over {v(x,z)}}\;\;.
\end{equation} 
Applying equation (\ref{eqn:ss}),we can connect seismic time migration
to the transformed wave equation
\begin{equation}
{\partial^2 P \over \partial x^2} +
W\,{\partial^2 P \over \partial \hat{z}^2} + 2\, {{(1-W)} \over v_0}\,
{{\partial^2 P} \over {\partial \hat{z} \partial \hat{t}}}  \, = 
\,{(2-W) \over v_0^2 }\,
{\partial^2 P \over \partial \hat{t}^2}\;. 
\label{eqn:swave} 
\end{equation}
The variables $\hat{z}$ and $\hat{t}$ correspond to the transformed
depth and time coordinates, which possess the following property: if
$\hat{z}=0$, $\hat{t}=s\left(t_0\right)$, and if $\hat{t}=0$,
$\hat{z}=v_0 s\left(t_v\right)$. $W$ is a varying coefficient defined
as
\begin{equation}
W=a^2+2b\,(1-a^2)\;,
\label{eqn:wstolt} 
\end{equation}
where 
\begin{eqnarray*}
b={{\eta(z)}\over{\eta(\zeta)}}\,,\;
a={{s(\tau)\,v_0\,v(x,z)}\over{\eta(\zeta)}}\,,\;
\tau=\int_0^\zeta{{dz}\over{v(x,z)}}=t+ \int_0^z{{dz'}\over{v(x,z')}}\;.
\end{eqnarray*}

Since the $W$ parameter varies slowly with $x$ and $\hat{z}$, Stolt
suggested to replace it with its average value.  Thus equation
(\ref{eqn:swave}) is then approximated by an equation with constant
coefficients, which has the dispersion relation
\begin{equation}
\widehat{\omega}_v\left(k,\widehat{\omega}_0 \right)=
\left(1-{1\over W}\right) \widehat{\omega}_0+
{{\mbox{sign}\left(\widehat{\omega}_0\right)}\over W}\,
\sqrt{\widehat{\omega}_0^2 - W v_0^2 k^2}\;.
\label{eqn:sdispersion} 
\end{equation}


As outlined above, Stolt's approximate method for migration in
heterogeneous media consists of the following steps:
\begin{enumerate}
\item stretching the time variable according to equation
  (\ref{eqn:ss}),
\item interpolating the stretched time to a regular grid,
\item double Fourier transform,
\item {\em f--k} time migration by the operator (\ref{eqn:stolt}) with
  the dispersion relation (\ref{eqn:sdispersion}),
\item inverse Fourier transform,
\item inverse stretching (that is, shrinking) of the vertical time
  variable on the migrated section.
\end{enumerate}
The value of $W$ must be chosen prior to migration. According to
Stolt's original definition (\ref{eqn:wstolt}), 
the depth variable $z$ gradually changes in the migration process from
zero to $\zeta$, causing the coefficient
$b$ in (\ref{eqn:wstolt}) to change monotonically from 0 to 1. If the velocity $v$ 
monotonically increases with depth, then 
$\eta''(z)={\partial v \over \partial z}\geq 0$, and the average value of $b$ is
\begin{equation}
\bar{b}={1 \over {\zeta \eta(\zeta)}}\, {\int_0^\zeta \eta(z) dz}\leq
{1 \over {\zeta \eta(\zeta)}}\, {\int_0^\zeta {\eta(\zeta) {z \over \zeta} }dz}=
{1 \over 2}\;.
\label{eqn:bishalf} 
\end{equation} 
As follows from equations (\ref{eqn:wstolt}) and (\ref{eqn:bishalf}),
in the case of monotonically increasing velocity, the average value of
$W$ has to be less than 1 ($W$ equals 1 in a constant-velocity case).
Analogously, in the case of a monotonically decreasing velocity, $W$
is always greater than 1.  In practice, $W$ is included in migration
routines as a user-defined parameter, and its value is usually chosen
to be somewhere in the range of 1/2 to 1. The next section describes a
straightforward way to determine the most appropriate value of $W$ for
a given velocity distribution.

\inputdir{Sage}

A useful tool for that purpose is Levin's equation for the traveltime
curve.  \cite{Levin.sep.42.373} applied the stationary
phase technique to the dispersion relation (\ref{eqn:sdispersion}) to
obtain an explicit equation for the summation curve of the integral
migration operator analogous to the Stolt stretch migration. The
equation evaluates the summation path in the stretched coordinate
system, as follows:
\begin{equation}
s\left(t_0\right)=
\left(1-{1\over W}\right) s\left(t_v\right)+
{1\over W}\,
\sqrt{s^2\left(t_v\right) + {W\,
 {{\left(x-x_0\right)^2} \over v_0^2}}}\;,
\label{eqn:levin} 
\end{equation}
where $x_0$ is the midpoint location on a zero-offset seismic section,
and $x$ is the space coordinate on the migrated section.  Equation
(\ref{eqn:levin}) shows that, with the stretch of the time coordinate,
the summation curve has the shape of a hyperbola with the apex at
$\left\{x,s\left(t_v\right)\right\}$ and the center (the intersection
of the asymptotes) at $\left\{x,{\left(1-{1\over
        W}\right)}\,s\left(t_v\right)\right\}$. In the case of
homogeneous media, $W=1$, $s(t)\equiv t$, and
equation~(\ref{eqn:levin}) reduces to the known expression for a
hyperbolic diffraction traveltime curve. It is interesting to note
that inverting equation (\ref{eqn:levin}) for $s\left(t_v\right)$
determines the impulse response of the migration operator:
\begin{equation}
\hat{z}-\hat{z_0}=
\left({1\over Q}-1\right) R \pm
{1\over Q}\,
\sqrt{R^2 - 
{Q\,{\left(x-x_0\right)^2}}}\;,
\label{eqn:sfront} 
\end{equation}
where $R=v_0 \hat{t}$, and $Q=2-W$. Equation (\ref{eqn:sfront}) can be
interpreted as the wavefront from a point source in the
$\{x,\hat{z},\hat{t}\}$ domain of equation (\ref{eqn:swave}).
Wavefronts from a point source in the stretched coordinates for $W<2$
have an elliptic shape, with the center of the ellipse at
$\{x,\hat{z_0}+\left( {1 \over Q}-1\right)\, R \}$ and the semi-axes
$a_x={R \over \sqrt{Q}}$ and $a_z={r \over Q}$. The ellipses stretch
differently for $W<1$ and $W>1$, as shown in Figure \ref{fig:stofro}.
In the upper part that corresponds to the upgoing waves, the ellipses
look nearly spherical, since the radius of the front curvature at the
top apex equals the distance from the source.

\plot{stofro}{width=5.in,height=2.5in}{Wavefronts from a point source
  in the stretched coordinate system. Left: velocity decreases with
  depth (W=1.5). Right: velocity increases with depth (W=0.5).}

%\pagebreak  
\section{EVALUATING THE $W$ PARAMETER}
%%%%%%%%%%%%%%%%%%%%%%%%%%%%%%%%%%%%%

A remarkable connection between the Stolt stretch equation and
different three-parameter traveltime approximations leads to a
constructive estimate of the $W$ parameter.  The first useful
observation is a formal similarity between equation (\ref{eqn:levin})
and Malovichko's approximation for the reflection traveltime curve in
vertically inhomogeneous media \cite[]{ussr,Sword.sep.51.313,castle,nmo}
defined by
\begin{equation}
t_0=
\left(1-{1\over {S\left(t_v\right)}}\right) \,t_v+
{1\over {S\left(t_v\right)}}\,
\sqrt{t_v^2 + {{S\left(t_v\right)}\,
 {{\left(x-x_0\right)^2} \over {v_{rms}^2\left(t_v\right)}}}}\;.
\label{eqn:ussr} 
\end{equation}
In equation (\ref{eqn:ussr}), $v_{rms}$ is the effective (root
mean square) velocity along the vertical ray
\begin{equation}
v_{rms}^2\left(t_v\right)={\eta(z)\over t_v}=
{1 \over t_v}\,\int_{0}^{t_v} v^2(t) \,dt\;,
\label{eqn:vrms} 
\end{equation}
and $S$ is f the {\em parameter of heterogeneity}, defined by the equation:
\begin{equation}
S\left(t_v\right)={1 \over{v_{rms}^4 t_v}}\,\int_{0}^{t_v} v^4(t) \,dt\;.
\label{eqn:heter} 
\end{equation}
In terms of the $S$ parameter, the variance of the squared velocity
distribution along the vertical ray is
\begin{equation}
\sigma^2={1 \over t_v}\,\int_{0}^{t_v} v^4(t) \,dt - v_{rms}^4=v_{rms}^4 (S-1)\;.
\label{eqn:sigma} 
\end{equation}
As follows from equality (\ref{eqn:sigma}), $S\geq 1$ for any type of
velocity distribution ($S$ equals 1 in a constant velocity case). For
most of the distributions occurring in practice, $S$ ranges between 1
and 2.

Since reflection from a horizontal reflector in
vertically-heterogeneous media is kinematically equivalent to
diffraction from a point, we can regard equation (\ref{eqn:ussr}),
which is known as the most accurate three-parameter approximation of
the NMO curve, as an approximation of the summation path for the
post-stack Kirchhoff migration operator. In this case, it has the same
meaning as equation (\ref{eqn:levin}). An important difference between
the two equations is the fact that equation (\ref{eqn:ussr}) is
written in the initial coordinate system and includes coefficients
varying with depth, while equation (\ref{eqn:levin}) applies the
transformed coordinate system and constant coefficients. Using this
fact, we compare the accuracy of the approximations and derive the
following explicit expression, which relates Stolt's $W$ factor to
Malovichko's parameter of heterogeneity:
\begin{equation}
W=1-{{v_0^2\,s^2\left(t_v\right)} \over{v_{rms}^2\left(t_v\right) t_v^2}}\,
\left({{v^2\left(t_v\right)} \over {v_{rms}^2\left(t_v\right)}}
-S\left(t_v\right)
\right)\;\;.
\label{eqn:main} 
\end{equation}

The details of the derivation are given in the appendix. Expression
(\ref{eqn:main}) is derived so as to provide the best possible value
of $W$ for a given depth (or vertical time $t_v$). To get a constant
value for a range of depths, one should take an average of the
right-hand side of (\ref{eqn:main}) in that range.  The error
associated with Stolt stretch can be approximately estimated from
(\ref{eqn:taylor}) as the difference between the fourth-order terms:
\begin{equation}
\delta={{l^4 \over 8}\,{{W\left(t_v\right)-W} \over 
{t_v s^2\left(t_v\right) v_{rms}^2\left(t_v\right) v_0^2}}}\;,
\label{eqn:maxerror} 
\end{equation}
where $W\left(t_v\right)$ is the right-hand side of (\ref{eqn:main}),
and $W$ is the constant value of $W$ chosen for Stolt migration.

\subsection{Analytic Example}
%%%%%%%%%%%%%%%%
A simple analytic example is the case of a constant velocity gradient.
In this case the velocity distribution can be described by the linear
function $v\left(z\right)=v\left(0\right)(1+\alpha z)$. The Stolt
stretch transform for this case can be derived directly from equation
(\ref{eqn:ss}) and takes the form
\begin{equation}
s(t)=\left({e^{2 \alpha v\left(0\right)\,t} -1 -
2 \alpha v\left(0\right)\,t} \over 
{2 \alpha^2 v_0^2}\right)^{1/2}\;.
\label{eqn:ssvz} 
\end{equation}
Let $\kappa$ be the logarithm of the velocity change $v(z)/v(0)$. Then
an explicit expression for $W$ factor is found according
(\ref{eqn:main}) as
\begin{equation}
W={{2\,\kappa}\over{e^{2\,\kappa}-1}}={v^2\left(0\right) \over
v_{rms}^2(z)}\;.
\label{eqn:wvz} 
\end{equation}
In the case of a small $\kappa$', which corresponds to a small depth
or a small velocity gradient, $W \approx 1-\kappa$. In the case of a
large $\kappa$, $W$ monotonically approaches zero. Equation
(\ref{eqn:wvz}) can be a useful rule of thumb for a rough estimation
of $W$.

\subsection{Stolt stretch for anisotropic media}

As follows from the analysis of the reflection moveout in a vertically
heterogeneous transversely isotropic medium \cite[]{Fomel.sep.92.135},
expression~(\ref{eqn:main}) for the Stolt stretch parameter will
remain valid in this case if the values of $v_{rms}$ and $S$ are
computed according to equations
\begin{eqnarray}
v_{rms}^2\left(t_v\right) & = &
{1 \over t_v}\,\int_{0}^{t_v} v^2(t) \left(1 + 2\,\delta(t)\right)\,dt\;,
\label{eqn:TIvrms} \\
S\left(t_v\right) & = & {1 \over{v_{rms}^4 t_v}}\,\int_{0}^{t_v} v^4(t) \,
\left(1 + 2\,\delta(t)\right)^4\,\left(1 + 8\,\eta(t)\right) \,dt\;,
\label{eqn:TIsk}
\end{eqnarray}
where $\delta$ and $\eta$ are the conventional anisotropic parameters
\cite[]{GEO51-10-19541966,GEO60-05-15501566}, which may vary with depth.

As we demonstrate in the next section, the method of cascaded
migrations \cite[]{GEO52-05-06180643} can improve the performance of
Stolt migration in the case of variable velocity
\cite[]{GEO53-07-08810893}.  However, this method affects only the
isotropic part of the model and cannot change the contribution of the
anisotropic parameters.  Therefore, in the anisotropic case, it is
important to incorporate anisotropic parameters into the Stolt stretch
correction.

%--------------------------------------------------------
\section{Application}
\inputdir{elfst}

Following the study by \cite{GEO54-06-07010717}, we selected a
dataset that includes steep dips in order to test the accuracy of our
algorithms. The dataset is courtesy of Elf Aquitaine.  It was recorded
in the North Sea over a salt-dome structure.
Figure~\ref{fig:data-stolt-ststr} shows the data after NMO-stack and
after post-stack Stolt migration, using a constant velocity of 2000
m/s. The Stolt method creates visible undermigrated events on both
sides of the salt body. Using a higher velocity to focus them better
would have created overmigration artifacts at shallow reflectors.
Stolt-stretch migrated section using $W=0.5$ is shown in in
Figure~\ref{fig:data-stolt-ststr}c. It should be compared with an
improved result shown in Figure~\ref{fig:data-ststr-pshift-casc}a.
\par
Using the Stolt-stretch method with the optimal choice for $W$
estimated from equation (\ref{eqn:main}) yields a better focusing of
events at all depths (Figure~\ref{fig:data-ststr-pshift-casc}a),
compared to other values of $W$ (Figures~\ref{fig:data-stolt-ststr}b
and \ref{fig:data-stolt-ststr}c, respectively for $W$ equals 1.0 and
0.5).  The $v(z)$ model used for migration is shown in
Figure~\ref{fig:velocities}a and was obtained by averaging laterally
the reference velocity model.


\plot{data-stolt-ststr}{width=6in,totalheight=9in}{(a) Section of the
  North Sea data, after NMO-stack. (b) Section migrated using Stolt's
  method with $v_0$=2000 m/s. (c) Section migrated using Stolt-stretch
  with an arbitrary value $W=0.5$ for the parameter of heterogeneity.}

\plot{data-ststr-pshift-casc}{width=6in,totalheight=9in}{(a) Section
  migrated with the Stolt-stretch method using the optimal value
  ($\approx 0.67$) for the parameter $W$. (b) Section migrated with
  the phase-shift method. (c) Section migrated using the cascaded
  Stolt-stretch approach (5 velocities).}

\par
The reference method of migration for our study is the phase-shift
method \cite[]{GEO43-07-13421351}. It is known to be perfectly accurate
for all dips up to $90^\circ$ in a $v(z)$ velocity field. A comparison
between the phase-shift migration result
(Figure~\ref{fig:data-ststr-pshift-casc}b) and the section migrated
with the Stolt-stretch approach shows almost no difference for flat
events. However, a more detailed analysis reveals significant errors
for steep events inside and around the salt body. The approximation
made by stretching the time axis breaks for recovering steep events.
\par
A way to overcome the difficulties encountered by Stolt's migration is
to divide the whole process into a cascade, as suggested by
\cite{GEO53-07-08810893}. The theory of cascaded migration proves
that {\it f-k} migration algorithms with a $v(t)$ velocity model like
Stolt-stretch can be performed sequentially as a cascade of $n$
migrations with smaller interval velocities $v_i(t) \; , \;
i=1,\ldots,n$, such then
\begin{equation}
\label{eqn:larner}
v^2(t) = \sum_{i=1}^{n} v_i^2(t)\;.
\end{equation}
At a given vertical traveltime $t$, all the successive velocity models
have to be constant, except the last one \cite[]{GEO52-05-06180643}.
Typically, the first stage is done with a constant velocity model and
can be computed using Stolt's method, which is then accurate for all
dips. Figure~\ref{fig:velocities} illustrates such a cascade of
velocity models in our particular case, with 3 and 5 stages.

\plot{velocities}{width=6in,height=3.5in}{(a) Interval velocity
model $v(t)$ estimated from the 2-D reference model. (b)
Decomposition in a cascade of 3 models, such as $v^2 = v_1^2
+ v_2^2 + v_3^2$. (c) Decomposition in a cascade of 5 models,
such as $v^2 = v_1^2 + v_2^2 + v_3^2 + v_4^2 + v_5^2$}
\par
As a consequence of this decomposition, each intermediate velocity
model shows not only a smaller velocity but also less vertical
heterogeneity. In other words, the Stolt-stretch parameter $W$
estimated for each stage tends to be closer to 1.0, thus reducing the
migration errors due to the
approximation. Figure~\ref{fig:data-ststr-pshift-casc}c shows the
migration result using a 5-stage cascaded scheme. All the successive
values of $W$ were greater than 0.8. There are almost no
differences with the phase-shift result
(Figure~\ref{fig:data-ststr-pshift-casc}b).
\par
\inputdir{imps}
An accuracy of the cascaded stolt-stretch migration is additionally
verified by comparing its impulse response with that of the
phase-shift migration (Figure~\ref{fig:imp-mig3}). The impulses are
generated using the same velocity model as shown in
Figure~\ref{fig:velocities}a.  Figure~\ref{fig:imp-mig} provides a
more detailed comparison. We can see a kinematic difference in the
impulse response of Stolt-stretch compared to phase-shift.  While
Gazdag's phase-shift honor ray bending in any $v(z)$ model,
Stolt-stretch is only designed to make the fitting curve look like an
hyperbola close to the apex \cite[]{Levin.sep.35.195}, and therefore
induces residual migration errors.  As seen in
Figure~\ref{fig:data-ststr-pshift-casc}a, Stolt-stretch result
displays residual hyperbolic migration artifacts that are due to this
fundamental kinematic difference. Cascading Stolt-stretch makes the
impulse response of the migration converge towards the one of
phase-shift.

Figure~\ref{fig:dip-zoom} shows a close-up of the salt body region for
all migration algorithms. The methods have a different accuracy with
respect to steep dips. We notice a gradual improvement of the result
from Stolt-stretch to phase-shift as we increase the number of
velocities in the cascaded Stolt-stretch scheme. In theory, the
migration errors in the cascaded approach can be made as small as
desired by increasing the number of stages. At the limit, it
corresponds to the velocity continuation concept
\cite[]{me,SEG-1997-1762}.
\par
In our case, six stages were enough to obtain a result comparable to
phase-shift. In their comparative study on time migration algorithms,
\cite{GEO54-06-07010717} have shown that four-stage cascaded {\it
  f-k} migration is accurate for dips up to $85^\circ$, which is
almost comparable to phase-shift, accurate for all dips.  It is worth
noting the computational cost difference between the two: on our
example, phase-shift migration is about 80 times more expensive than
Stolt-stretch.

\plot{imp-mig3}{width=6in}{3-D impulses responses of the cascaded
Stolt-stretch (a) and phase-shift (b) operators.}

\plot{imp-mig}{width=6in}{Impulses responses of the different
operators. (a) Stolt-stretch. (b) Phase-shift. (c) and (d) Cascaded
Stolt-stretch, with 3 and 5 velocities, respectively.}

\inputdir{elfst}

\plot{dip-zoom}{width=6in}{Zoom in the salt body area where steep dips
  are located. (a) Migration with the Stolt-stretch method using optimal $W$. 
  (b)
  Migration with the phase-shift method. (c) 
  Migration with the Stolt-stretch method using $W=0.5$. 
  (d) Migration with the
  cascaded Stolt-stretch approach, using 5 velocities.}

%--------------------------------------------------------
\section{Conclusions}

An explicit expression for the Stolt-stretch parameter, derived in this
paper, allows us to achieve optimal accuracy when applying Stolt
migration in vertically heterogeneous media.

Combining an optimal analytical choice for the Stolt-stretch parameter
with the cascaded {\it f-k} migration approach, we manage to obtain
time migration results comparable to Gazdag's phase-shift migration.
The Stolt method is considerably more computer-efficient and remains
accurate for steeply dipping events.

%--------------------------------------------------------
\section{Acknowledgments}

The authors would like to thank the sponsors of the Stanford
Exploration Project for financial support and Elf Aquitaine for
providing the data used in this work.

\bibliographystyle{seg}
\bibliography{stolt,SEG,SEP2}

\append{\ }

In this Appendix, we derive an explicit expression for the
Stolt-stretch parameter $W$ by comparing the accuracy of equations
(\ref{eqn:levin}) and (\ref{eqn:ussr}), which approximate the
traveltime curve in the neighborhood of the vertical ray. It is
appropriate to consider a series expansion of the diffraction
traveltime in the vicinity of the vertical ray:
%\footnote{Though a the
%  power series (\ref{eqn:taylor}) is not the best possible
%  representation of the traveltime curve, it is quite suitable for
%  comparing different approximations in the vicinity of the vertical
%  ray. In the post-stack migration problem, those approximations imply
%  that the reflector dips have zero mean value. If we assumed that the
%  mean dip value on a particular seismic section were different from
%  zero, we could apply expansions different from expansion
%  (\ref{eqn:taylor}).  That curious option is beyond the scope of this
%  paper.}
\begin{equation}
t_0(l)={\left.t_0\right|_{l=0}}+
{1 \over 2}\,{\left.{d^2t_0}\over {dl^2}\right|_{l=0}}l^2+
{1 \over {4!}}\,{\left.{d^4t_0}\over {dl^4}\right|_{l=0}}l^4+\cdots\;\;,
\label{eqn:taylor} 
\end{equation}
where $l=x-x_0$. 
Expansion (\ref{eqn:taylor}) contains only even powers of $l$ because of 
the obvious symmetry of $t_0$ as a function of $l$. 

Matching the series expansions term by term is a constructive method
for relating different equations to each other. The special choice of
parameters $t_v$, $v_{rms}$, and $S$ allows Malovichko's equation
(\ref{eqn:ussr}) to provide correct values for the first three terms
of expansion (\ref{eqn:taylor}):
\begin{eqnarray}
\left.t_0\right|_{l=0} & = & t_v\;;
\label{eqn:mal0}\\
\left.{d^2t_0}\over {dl^2}\right|_{l=0} & = &
{1 \over {t_v v_{rms}^2\left(t_v\right)}}\;;
\label{eqn:mal2}\\
\left.{d^4t_0}\over {dl^4}\right|_{l=0} & = &
-{{3\,S\left(t_v\right)} \over {t_v^3 v_{rms}^4\left(t_v\right)}}\;\;.
\label{eqn:mal4}
\end{eqnarray}
Considering Levin's equation (\ref{eqn:levin}) as an implicit
definition of the function $t_0\left(t_v\right)$, we can iteratively
differentiate it, following the rules of calculus:
\begin{eqnarray*}
\left.{ds}\over {dl}\right|_{l=0} = 
\left. s'\left(t_0\right)\,{{dt_0}\over {dl}}\right|_{l=0}  =  0\;;
\end{eqnarray*}
\begin{equation}
\left.{d^2s}\over {dl^2}\right|_{l=0}  = 
\left.\left(s'\left(t_0\right)\,{{d^2t_0}\over {dl^2}}+
s''\left(t_0\right)\,\left({dt_0}\over {dl}\right)^2\right)\right|_{l=0} = 
\left. s'\left(t_v\right)\,{{d^2t_0}\over {dl^2}}\right|_{l=0}  = 
{1\over {v_0^2 \,s\left(t_v\right)}}\;;
\label{eqn:lev2}
\end{equation}
\begin{eqnarray*}
\left.{d^3s}\over {dl^3}\right|_{l=0}  = 
\left.\left(3\,s''\left(t_0\right)\,{{dt_0}\over {dl}}\,{{d^2t_0}\over {dl^2}}+
s'\left(t_0\right)\,{{d^3t_0}\over {dl^3}}+
s'''\left(t_0\right)\,\left({dt_0}\over {dl}\right)^3\right)\right|_{l=0}  =  0
\end{eqnarray*}
\begin{eqnarray}
\left.{d^4s}\over {dl^4}\right|_{l=0} & = &
\left(6\,s'''\left(t_0\right) \left({dt_0}\over {dl}\right)^2 
{{d^2t_0}\over {dl^2}}+
3\,s''\left(t_0\right) \left({d^2t_0}\over {dl^2}\right)^2 
+ 4s''\left(t_0\right)\,{{dt_0}\over {dl}}\,{{d^3t_0}\over {dl^3}}+ \right.
\nonumber \\  & &
\left. \left. + s'\left(t_0\right)\,{{d^4t_0}\over {dl^4}}+
s^{IV}\left(t_0\right) \left({dt_0}\over {dl}\right)^4\right)\right|_{l=0}  =
\nonumber \\ 
& = & \left.\left(s''\left(t_v\right)\,\left({d^2t_0}\over {dl^2}\right)^2+
s'\left(t_v\right)\,{{d^4t_0}\over {dl^4}}\right)\right|_{l=0}  = 
-{{3\,W} \over {v_0^4 \,s^3\left(t_0\right)}}\;\;.
\label{eqn:lev4}
\end{eqnarray}
Substituting the definition of Stolt stretch transform (\ref{eqn:ss})
into (\ref{eqn:lev2}) produces an equality similar to
(\ref{eqn:mal2}), which means that approximation (\ref{eqn:levin}) is
theoretically accurate in depth-varying velocity media up to the
second term in (\ref{eqn:taylor}). It is this remarkable property that
proves the validity of the Stolt stretch method
\cite[]{Levin.sep.35.195,Claerbout.blackwell.85}. Moreover, equation
(\ref{eqn:levin}) is accurate up to the third term if the value of the
fourth-order traveltime derivative in (\ref{eqn:lev4}) coincides with
(\ref{eqn:mal4}). Substituting equation (\ref{eqn:mal4}) into
(\ref{eqn:lev4}) results in the expression
\begin{equation}
{{1-W}\over {v_0^2\,s^2\left(t_v\right)}}=
{{v^2\left(t_v\right)-S\left(t_v\right)\,v_{rms}^2\left(t_v\right)} \over
{v_{rms}^4\left(t_v\right)\,t_v^2}}\;.
\label{eqn:malvsst} 
\end{equation} 
It is now easy to derive from equation (\ref{eqn:malvsst}) the desired
explicit expression for the Stolt stretch parameter $W$:
equation~(\ref{eqn:main}) in the main text.

