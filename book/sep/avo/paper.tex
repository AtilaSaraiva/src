\lefthead{Ecker \& Lumley}
\righthead{AVO analysis}
\footer{SEP--80}

\def\deg{^{^{\circ}}}

\title{Seismic AVO analysis of methane hydrate structures}
\author{Christine Ecker and David E. Lumley}
\maketitle

\begin{abstract}
Marine seismic data from the Blake Outer Ridge offshore Florida show strong
``bottom simulating reflections'' (BSR) associated with methane hydrate
occurence in deep marine sediments. We use a detailed amplitude versus 
offset (AVO) analysis of these data to explore the validity of models which
might explain the origin of the bottom simulating reflector. After careful
preprocessing steps, we determine a BSR model which can successfully reproduce
the observed AVO responses. The P- and S-velocity behavior predicted by the 
forward modeling is further investigated by estimating the P- and S-impedance
contrasts at all subsurface positions. Our results indicate that the Blake Outer
Ridge BSR is compatible with a model of methane hydrate in sediment, overlaying 
a layer of free methane gas-saturated sediment. The hydrate-bearing
sediments seem to be characterized by a high P-wave velocity of
approximately 2.5 km/s, an anomalously low S-wave velocity of
approximately 0.5 km/s, and a thickness of around 190 meters.
The underlaying gas-saturated sediments have a P-wave velocity of 1.6 km/s, an
S-wave velocity of 1.1 km/s, and a thickness of approximately 250 meters.
\end{abstract}


\section{Introduction}

Bottom simulating reflectors, so called BSRs, parallel the seafloor at subbottom depths
of several hundred meters. Seismic investigations \cite[]{Shipley.aapetgeo.63,Miller.aapetgeo.75,Hyndman2.jgr.97}
indicate that they are 
characterized by large negative reflection coefficients and increasing subbottom
depth with increasing water depth. The base of the stability field for 
methane hydrates appears to be associated with these bottom simulating 
reflectors. Due to the enormous 
amount of methane that is apparently contained within hydrate structures, they
are likely to have a significant ``greenhouse'' effect on future global climate,
and might also represent an important future energy resource 
\cite[]{Kvenvolden.revgeo.31}. Therefore, a good understanding of the origin and
characteristics of the hydrate zones and BSRs is desirable.
Only limited information is available from deep-sea drilling, as the 
risk of heating and destabilizing the initial hydrate conditions during the 
process of drilling is considerably high. Thus, the core samples and well-logs
do not necessarily reflect the correct in situ 
hydrate characteristics and properties.
Consequently, most information is inferred remotely from seismic reflection
data \cite[]{Shipley.aapetgeo.63,Miller.aapetgeo.75,Hyndman.jgr.97,Hyndman2.jgr.97,Singh.science.260}. Most of these investigations, which were 
based mainly 
on AVO responses and synthetic modeling, used primarily P-velocity information,
accessible directly from the seismic data, and neglected possible important 
S-velocity effects entirely.
The exact formation of the hydrate and its formation are still 
controversial, and different models have been proposed to explain the origin
of the BSRs \cite[]{Kvenvolden.aapetgeo.34,Hyndman.jgr.97}.
\par
In this study, we use both P- and S-wave information inferred from synthetic
modeling, velocity and AVO analysis of marine data from 
the Blake Outer Ridge to 
explain the bottom simulating reflector. The validity of the different 
BSR models is explored and the effects of two proposed models can be clearly
discriminated. The reflection amplitude variation with offset can 
be an important indicator of 
free gas at an interface \cite[]{Shuey.geophys.50} and, together with the
estimation of material properties at the interface, considerably 
limits the possible 
explanations of the physical origin of the BSR.
\par
This paper discusses our work with preprocessing, modeling, inversion, and 
interpretation of the methane hydrate seismic data from the Blake Outer
Ridge. Preliminary results of this study were presented by \cite{Ecker.sep.79.161,Ecker.agu.93.370}. 
After careful preprocessing, including
source wavelet deconvolution, trace interpolation, and amplitude
and moveout traveltime calibration, a detailed velocity analysis is performed
on the resulting CMP gathers. The estimated interval velocities are used to
constrain a BSR model which can successfully reproduce the observed AVO
amplitude responses. The impedance structure predicted by the modeled data is
further reinforced by estimating the P- and S-impedance contrasts at all 
subsurface positions. Combining the results of the synthetic modeling and
the impedance inversion, we give an integrated geophysical interpretation of 
the data.

\section{METHANE HYDRATES AND BSR MODELS}

A methane hydrate is an ice-like, crystalline lattice of water molecules in
which gas molecules are trapped physically without the aid of direct
chemical bonds. They are stable under certain pressure and temperature 
conditions (Figure \ref{fig:phase}). Thus, the occurrence of bottom simulating 
reflectors is restricted to two distinct regions: deep oceanic and polar.
In deep oceanic regions, BSRs are found in 
outer continental margins of slopes and rises where cold bottom water is
present. In polar regions, the BSRs are normally associated with permafrost,
both onshore in continental sediments and offshore in sediments of the
continental shelves.

\inputdir{XFig}
\plot{phase}{width=3.5in,height=3.5in}{Phase diagram showing the 
boundary between methane gas and methane hydrate. Redrawn after Kvenvolden
(1993).}

\par
Two models have been proposed to account for gas hydrate formation and thus
the development of BSRs. The first one assumes the local generation of methane
from organic material at the depth of the hydrate. Gradually thickening and 
thus deepening of the methane hydrate zone causes it eventually to subside
into a temperature region where hydrate is unstable. Consequently, free gas
can be present in this region \cite[]{Kvenvolden.aapetgeo.34}. The BSR is caused
by the impedance contrast at the base of the hydrate layer and the top of the 
gas layer. A second model, on the other hand, supports the formation of methane
hydrates through the removal of methane from rising pore fluids being expelled
upwards from deeper in the sediment section \cite[]{Hyndman.jgr.97}. Most of
the methane is generated microbially at depths below the hydrate stability
zone but not at depths sufficient for the formation of thermogenic methane.
Thus free gas does not have to be present below the BSR. In this case, the
BSR can be the consequence of the impedance contrast between overlaying 
sediments containing substantial amounts of high-velocity hydrate and 
underlaying normal velocity brine sediments. 

 
\section{PREPROCESSING STEPS}
\inputdir{bsr}

In the first preprocessing step, the data were corrected for time-varying 
spherical divergence. Next, we performed a single-trace source wavelet 
deconvolution in order to regularize the source wavelet with offset. The
deconvolved data were then bandpass filtered to the original data bandwidth
to remove spurious deconvolutional high-frequency noise. Using an initial
semblance estimate of the best stacking velocity function, a normal moveout
correction of the data was carried out.
\par
Two main assumptions were made to perform the amplitude calibration. First, it
was assumed that an offset-dependent rather than an angle-dependent amplitude
correction was sufficient, since the difference between the maximal angle
of incidence at the seafloor (30$\deg$) and the BSR reflection (33$\deg$) is
negligibly small. Second, we assumed a functional form for the AVO response
of the seafloor reflection, based on the fluid-solid interface Zoeppritz
PP reflection 
coefficient. Based on these assumptions, the amplitude calibration was
performed by scaling each trace to the seabottom amplitude to match the 
predicted seafloor AVO as a function of offset.
\par
Due to the use of a nonlinear streamer to record the data, a trace 
interpolation of the near offset data was necessary to regularize the receiver
cable group spacings. After applying an inverse NMO correction to the
interpolated data, a high resolution NMO stacking velocity analysis was
performed. Having derived a good stacking model for the data, they were
reprocessed in a second iteration using the new velocities. Since it is 
essential for the subsequent impedance contrast estimation that the reflector
moveout is very flat after NMO correction or migration, an additional 
static shift was applied preceding the amplitude scaling to correct for some
small non-hyperbolic, offset-dependent residual moveout in the CMP gathers.
\par
Figure \ref{fig:bsr-ann} shows the final data after preprocessing. 
The gather contains a BSR AVO effect that is representative of the average 
trend along the entire line. Picking the peak amplitudes along the BSR yielded
the AVO curve shown in Figure \ref{fig:bsr-picks}. Starting with a negative 
zero-offset reflection coefficient that was obtained by assuming a seafloor
reflection of approximately 0.2, the amplitudes become increasingly negative
with increasing offset.

\plot{bsr-ann}{width=4.5in}{Final CMP gather after preprocessing
containing an average AVO effect observed along the line.}

\plot{bsr-picks}{width=4.0in}{AVO curve of the amplitude picks along
the BSR. The near offset reflection amplitude was determined by assuming a
seafloor reflection of approximately 0.2.}


\section{MODELING APPROACH}
\inputdir{XFig}

Using the estimated interval velocities, the effects of different impedance 
structures on the BSR AVO response were explored in an attempt to reproduce the
seismic data. Several models were constructed which were constrained to preserve
the average interval velocity of each macro layer. To avoid possible tuning
effects in this first, basic modeling approach, 
all layers were assumed to be thicker than a quarter of a wavelength.
Synthetic AVO amplitude responses were then estimated for the individual models
using Zoeppritz equations and compared with the amplitude responses observed
in the seismic data. 
\par
Figure \ref{fig:orig} shows the initial P- and S-wave interval 
velocities. The P-wave 
velocity was inferred directly from the seismic data, while the S-wave velocity
was determined by assuming a Poisson's ratio of 0.4 which is consistent with
brine saturated sediments.

\plot{orig}{width=2.75in,height=3.25in}{Initial velocity model of the data.}

A considerably high P-wave velocity of approximately 2.5 km/s
was obtained above the BSR which appears
to be underlain by a lower velocity of around 1.6 km/s.
The P-velocity trend for normal
brine-saturated sediments is indicated by the dotted line. Being 
higher than this trend above the BSR and lower below it, the measured P-wave
velocities might be compatible with a model of hydrate-bearing sediment 
overlaying gas-saturated sediments.
\par
In a first attempt to model the observed AVO amplitudes, a thin layer of 
high-velocity, hydrate-bearing sediments was assumed to overlay brine-saturated
sediments. As the measured P-wave interval velocity of 2.5 km/s above the BSR 
has to be preserved, the hydrate layer can not be smaller than a certain
thickness in
order to obtain realistic velocity values for this model. The modeled P- and
S-wave velocities above and below the BSR are shown in Figure 
\ref{fig:thin-hydrate}. The initial model is given by the dotted line whenever the
modeled velocities differ from the initial ones.
The S-wave velocity was obtained by assuming that the Poisson's ratio in the
hydrate-bearing sediments is comparable with that of brine sediments.

\plot{thin-hydrate}{width=2.75in,height=3.25in}{Interval velocities above and 
below the BSR for a thin-hydrate layer overlaying brine sediment. The dotted
lines represent the velocities of the initial model, while the solid lines
give the velocities of this model. The arrows indicate the direction of the
velocity change.}

Using Zoeppritz equations, the AVO trend corresponding to the modeled 
velocities at the transition from hydrate to brine-saturated sediments is 
determined and compared with the one observed in the data 
(Figure \ref{fig:thin-hyd}).

\activeplot{thin-hyd}{width=4.0in}{.}{AVO Curve obtained from the thin-hydrate
model (solid line) compared with the one observed in the data (crosses).}

A comparison of both curves yields that the thin-hydrate model not only failed
in reproducing the near offset reflection coefficients, but also the general 
AVO trend, having slightly increasing amplitudes with increasing offset.
Assuming negligibly small density contributions, the near offset amplitudes
are mainly dependent on the P-wave velocity contrast at the reflector, 
while the AVO trend is 
characterized primarily by the S-wave velocity contrast.
Thus, the AVO response resulting from the thin-hydrate model implies the use
of both incorrect P- and S-wave velocities at the bottom simulating reflector.
Further thinning of the hydrate layer would increase the P-wave
velocity in this zone due to the required preservation of the measured 
interval velocity, and thus the P-wave velocity contrast at the BSR.
This would result in an even more pronounced difference between the observed 
and modeled zero offset reflection amplitudes. Consequently, a thin-hydrate
layer overlaying brine-saturated sediments is not sufficient to explain
the seismic data.
\par
Based on this result, the subsequent modeling attempted to decrease the 
P-wave velocity contrast at the BSR in order to recreate the observed zero 
offset reflection amplitudes. The required decrease was performed by thickening
the hydrate layer, thus yielding a thick-hydrate over brine sediment model. 
An evaluation of the effects of several different velocity combinations on
the reflection amplitudes resulted in the model shown in Figure \ref{fig:brine-vel}.
The estimated P-wave velocity in the hydrate 
corresponds to the measured interval velocity,
yielding a considerable thickness of the hydrate zone. The S-wave velocity was
again obtained using a Poisson's ratio of 0.4 and is thus the same as in the 
initial model.
The AVO curve based on the transition from the hydrate to the brine-saturated
sediments was determined by Zoeppritz modeling and is shown in
Figure \ref{fig:brine}.
\par
The comparison of the modeled AVO responses with those observed 
indicates that this
model could successfully reproduce the zero offset data. This suggests that the
modeled P-wave velocities of 2.5 km/s in the hydrate and 1.6 km/s in
the underlaying sediments might resemble the actual conditions at the BSR.
However, the obtained AVO trend is still contrary to the observed one,
displaying increasingly positive amplitudes with increasing offset. Hence,
a change in Poisson's ratio seems to be required at the transition from the
hydrate-bearing sediments above the BSR to the sediments below the BSR.
\par
Continuously changing the possible velocities in the hydrate zone and the 
characteristics of the underlaying sediments resulted finally in a hydrate layer
characterized by a P-wave velocity of approximately 2.5 km/s and an anomalously
low S-wave velocity of around 0.5 km/s yielding a Poisson's ratio of 0.47.
The hydrates appear to be underlain by sediments having a 
P-wave velocity of 1.6 km/s and an S-wave velocity of 1.1 km/s, yielding a
Poisson's ratio of 0.1 which is consistent with free gas. The final model can 
be seen in Figure \ref{fig:gas-vel}. The initial model is given by the dotted line
whenever the modeled velocities differ from the initial ones.
Based on the determined interval velocities,
the thickness of the hydrate layer was calculated to be approximately 190
meters and the one of the gas layer to be around 250 meters. Neglecting possible
tuning effects, a thin gas layer was not a good model representation,
as it required a decrease in P-wave velocity with respect to the hydrate
layer to preserve the measured interval velocity below the BSR. Thus, it
resulted in a significant 
deviation of the zero offset reflection amplitudes of the model
and the true seismic data.
\par
A comparison of the synthetic AVO curve obtained for the model shown in
Figure \ref{fig:gas-vel} with the amplitude picks obtained
from a representative CMP gather is shown in Figure \ref{fig:gas}. Both the
synthetic and the real data AVO amplitude responses are in good agreement
for near and far offsets. Thus, a significant increase in S-wave velocity and
a simultaneous decrease in P-wave velocity at the transition from 
hydrate-bearing sediments to sediments containing free gas is required to
explain the observed seismic data. 

\notinteractive
\activeplot{brine-vel}{width=2.75in,height=3.25in}{.}{Interval velocities above and
below the BSR for a thick-hydrate over brine sediment model. The modeled 
velocities are correspond to the initial interval velocities.}


\activeplot{brine}{width=4.0in}{.}{AVO curve obtained from the thick-hydrate
model (solid line) compared with the one observed in the data (crosses).}

\notinteractive
\activeplot{gas-vel}{width=2.75in,height=3.25in}{.}{Interval velocities for
hydrate-bearing sediments overlaying gas-saturated sediments. The dotted
lines represent the initial velocities. The arrows indicate where the modeled 
velocities
had to be increased or decreased to match the seismic data.}


\activeplot{gas}{width=4.0in}{.}{Synthetic AVO curve of hydrates overlaying 
sediments saturated with free gas (solid line) compared with an observed
one (crosses).}

The final velocity model for the entire section is shown in Figure
\ref{fig:final}. The deviation from the initial interval velocities is indicated
by the dotted line. While the initial P-wave interval velocities corresponded
to the modeled velocities in the hydrate and gas sediments,
the S-wave velocities
had to be modified with regard to apparently different shear properties in the
hydrates and the gas compared to the brine sediments.
\par
Based on the modeled increase of the S-wave velocity at the bottom of the
of the hydrate stability zone, a large positive S-impedance contrast can be
predicted for the seismic data. On the other hand, a negative P-impedance
contrast can be expected at the BSR due to the decrease in P-wave velocity
at the transition from hydrate to gas. In order to determine the actual effect,
we performed a prestack migration impedance inversion of the seismic data.

\notinteractive
\activeplot{final}{width=2.75in,height=3.25in}{.}{Final modeled interval velocity
model. The dotted line indicates where the initial model differs from
the final model. The arrows describe the direction the velocity had to be 
changed in order to fit the seismic data.}



\section{IMPEDANCE ESTIMATION}

The P- and S-impedance contrasts at each subsurface position were estimated 
by applying a least-squares elastic parameter inversion method 
\cite[]{lumley.sep.70.165,lumley.sep.77.211} to the CMP gathers. This technique
fits the prestack migrated AVO gathers at each pseudo-depth and surface 
position to the theoretical P- and S-impedance curves which are based on
linearized Zoeppritz equations. The least-squares impedance inversion results
of the preprocessed CMP gathers are shown in Figures \ref{fig:pimp-ann} and 
\ref{fig:simp-ann}.


\activeplot{pimp-ann}{width=4.5in}{.}{P-impedance contrast section.}

\activeplot{simp-ann}{width=4.5in}{.}{S-impedance contrast section.}

The P-impedance contrast section shows clearly that the seafloor reflection 
and the BSR have P-impedance contrasts of opposite polarity and approximately
the same magnitude. In a small section above the BSR there is a ``quiet'' 
zone where no diffractions or reflections are visible, which can possibly 
be explained by the presence of disseminated methane hydrate. The 
S-impedance section is dominated by a very strong impedance contrast at the
BSR. Although the seafloor has a much weaker impedance contrast, it is evident 
that both have the same polarity. Below the BSR, a horizontal reflector gives
a strong P-impedance contrast of the same polarity as the seafloor, indicating
an increase in P-wave velocity at the reflector, but weak S-impedance. This may 
be indicative of the base of the gas zone.
\par
Assuming a seafloor reflection of 0.2 and assuming the P- and S-impedance 
contrasts at the seafloor, it is possible to estimate the relative impedance
contrasts of the BSR by determining the average amplitude changes between 
seafloor and BSR. This results in a P-impedance contrast of $-$0.4 at the
BSR which has the same magnitude but opposite polarity to the seafloor.
The S-impedance contrast is very strong and amounts to approximately 0.8
to 1.2, which is two to three times as much as the seafloor impedance 
contrast of the same polarity. This anomalous S-impedance behavior is even
more pronounced by making a simple P$*$S anomaly map shown in Figure 
\ref{fig:PSmap-ann}. Normal impedance structure is plotted dark grey, while
anomalous impedance structure is indicated by white. The high magnitude
contrast has the effect of visually diminishing the S-impedance contrast of
the seafloor (Figure \ref{fig:simp-ann})  compared with the P-impedance contrast
(figure \ref{fig:pimp-ann}) at the seafloor, which are actually the same magnitude.

\activeplot{PSmap-ann}{width=4.5in}{.}{P$*$S impedance contrast.}

The negative P-impedance contrast and the large positive S-impedance contrast
at the BSR are in good agreement with the prediction based on the Zoeppritz 
modeling. The strong positive S-impedance contrast clearly supports the modeled
S-wave velocity behavior of anomalously low S-velocity in the hydrates and 
considerably increased S-velocity in the underlain gas sediments. Based on 
the synthetic modeling and the impedance inversion results, the geology was
interpreted as seen in Figure \ref{fig:interp}. In this interpretation, the
hydrate-bearing sediments are assumed to overlay sediments in which free gas
is trapped. The flat reflector below the BSR might correspond to a gas-water
contact at the base of the free gas zone based on the impedance contrasts
obtained for this reflector.

\notinteractive
\activeplot{interp}{width=4.5in}{.}{Interpretation of the methane hydrate 
seismic data from the Blake Outer Ridge.}
 

\section{CONCLUSIONS}

A detailed AVO analysis was performed on data from the Blake Outer Ridge to
evaluate the origin of the bottom simulating reflector. Reflectivity clearly
discriminates the effects of different models and shows that the observed 
BSR best fits a model of sediments containing substantial amounts of hydrate 
overlaying sediments containing free gas. This modeling result was supported 
by a prestack impedance inversion of the seismic data. A transition from 
hydrate to brine sediments is not consistent with the AVO amplitude responses
and the impedance contrasts. 
\par
Based on the synthetic modeling, the thickness of the hydrate layer was 
determined to be approximately 190 meters. It is characterized by a P-wave
velocity of around 2.5 km/s and an anomalously low S-wave velocity of 0.5 km/s.
The thickness of the gas layer was calculated to be approximately 250 meters.
It has a P-wave velocity of 1.6 km/s and an S-wave velocity of 1.1 km/s, 
yielding a Poisson's ratio of 0.1 which is reasonable for gas. The considerable
thickness of the gas layer might suggest the possibility of it being a source 
rock for the overlaying methane hydrate. 
\par
It has to be considered, however, that 
the synthetic modeling excluded possible tuning effects and thus might 
represent a simplification of the actual conditions. Nevertheless, the 
velocity behavior predicted by the model was reinforced by the prestack
impedance inversion, thus
indicating that a transition from high P-wave velocity and
anomalously low S-wave velocity in the hydrate to low P-wave velocity and high
S-wave velocity in the gas sediments is required. A detailed investigation of this 
unusual behavior is performed by \cite{Ecker.sep.80.christine2}.



\section{Acknowledgments}

We thank the sponsors of the Stanford Exploration Project and Professor Jon
Claerbout for supporting this research. We also thank Keith Kvenvolden, Bill
Dillon and Myung Lee of the USGS for providing us with a copy of the Blake
Outer Ridge seismic data used in this study.


\bibliographystyle{seg}
\bibliography{SEP2}





%\newpage
