\appendix
\section{Wavefield composed of N plane waves}
Here, we extend prediction filter theory to N plane waves. Assume that the 
data have N plane waves with different dipping angles, the dataset can then be 
expressed by

\begin{equation}
	w(t,x) = \sum_{\rm i=1}^N a_{\rm i} \delta(t-t_{\rm i}-p_{\rm i}x)
\end{equation}
Fourier transform along the time axis will give us
\begin{equation}
	W(f,x) = \sum_{\rm i=1}^N a_{\rm i} e^{i 2\pi f (t_{\rm i}+p_{\rm i} x)} = \sum_{\rm i=1}^N E_{\rm i}
\label{eqn:N1}
\end{equation}
where $E_{\rm i}=a_{\rm i} e^{i 2\pi f(t_{\rm i}+p_{\rm i}x)}$.
\par
Trace $(x-j\Delta x)$ is represented by
\begin{equation}
	W(f, x-j\Delta x)=\sum_{\rm i=1}^N P_{\rm i}^{\rm j}E_{\rm i}
\end{equation}
where propagator $P_{\rm i}=e^{i\frac{\pi}{2}p_{\rm i}v}$.
Assuming trace $W(f, x-j\Delta x), j=1,2,...,N$ is known, trace $W(f,x)$ can 
be predicted by a N points prediction filter
\begin{equation}
	W(f,x) = \sum_{\rm j=1}^N C_j W(f,x-j\Delta x)
\label{eqn:N2}
\end{equation}
Inserting equation (\ref{eqn:N1}) into equation (\ref{eqn:N2})
\begin{equation}
	\sum_{\rm i=1}^N E_{\rm i} = \sum_{\rm j=1}^N C_j \sum_{\rm i=1}^N P^{-j}_i E_i = \sum_{\rm i=1}^N E_i \sum_{\rm j=1}^N P^{-j}_i C_j
\end{equation}
For each $E_{\rm i}, i = 1,2,...,N$
\begin{equation}
	1 = \sum_{\rm j=1}^N P^{-j}_i C_j
\label{eqn:N3}
\end{equation}
Equation (\ref{eqn:N3}) can be expressed in matrix form
\begin{equation}
\left[\begin{array}{c}
 1 \\
 1 \\
 . \\
 1 
\end{array}  \right]
=
\left[\begin{array}{cccccc}
 P_1^{-1} & P_1^{-2} & . & . & . & P_1^{\rm -N} \\
 P_2^{-1} & p_2^{-2} & . & . & . & P_2^{\rm -N} \\
    .     &    .     & . & . & . &    .    \\
 P_{\rm N}^{-1} & P_{\rm N}^{-2} & . & . &. & P_{\rm N}^{\rm -N} 
\end{array}  \right]
\left[\begin{array}{c}
 C_1 \\
 C_2 \\
 . \\
 C_{\rm N} 
\end{array}  \right]
\label{eqn:N4}
\end{equation}
Equation (\ref{eqn:N4}) is a Van der Monde system. This system guarantees that 
there is one solution. So every $C_{\rm i}$ is a function of $P_{\rm j}$. 
This means that the prediction filter relies on frequency in the case of 
frequency-independent grid; whereas in the case of frequency-dependent 
grids, we can still get a prediction filter which is independent from 
frequency.
\subsection{Estimating the prediction filter coefficients}
\par
For sake of simplicity, we denote $W_{\rm j}=W(f,j\Delta x)$. $W_{\rm j}$ is 
predictable from $W_{\rm j-1}, W_{\rm j-2},..., W_{\rm j-N}$. 
\begin{equation}
	W_{\rm j}=\sum_{\rm z=1}^{\rm N}C_{\rm z}W_{\rm j-z}
\label{eqn:N5}
\end{equation}
Actually, equation (\ref{eqn:N5}) is a convolution operator. For each frequency
{\tt f}, we have an equation like (\ref{eqn:N5}). 
All frequencies have the same prediction filter $C_1, C_2,..., C_{\rm N}$. 
Usually this is an over-determined system. We can use some optimization 
method to solve it, e.g., conjugate gradient, LSQR, Lanzcos, etc. 

