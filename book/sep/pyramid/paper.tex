\def\figdir{./Fig}
\email{yalei@sep.stanford.edu, shuki@gatwick.Geco-Prakla.slb.com}
\righthead{Frequency dependent grids}
\lefthead{Sun \& Ronen}
\footer{SEP--93}
\title{The pyramid transform and its application to signal/noise separation}
%\keywords{deconvolution, noise }
\author{Yalei Sun and Shuki Ronen}

\begin{abstract}
The ``pyramid transform'' is spatial-resampling of data in the
frequency-space domain with frequency-dependent grids ({\tt FDG}), in which
the spatial-sampling interval is inversely proportional to the frequency.
A cube in the F-X-Y domain is thus transformed to a pyramid.
Wavefields do not contain high wavenumbers in low frequencies.
Therefore, the pyramid transform is invertible when applied to wavefields.
After pyramid transformation, the size of data has shunk dramatically.
This feature can save many resources. However, the main benefit is that the low
frequencies are not over-parameterized which makes frequency-dependent
grids more suitable for inversion and interpolation.
For example, spatial prediction filters become independent of the temporal
frequency in the pyramid domain.  This feature has a great
potential in signal/noise separation and trace interpolation.
In this paper, we investigate the possibility of applying the pyramid 
transform to noise suppression. Our results
show the prediction filter estimated in the pyramid domain can remove the
low temporal frequency random noise, which can not be handled by the
prediction filter estimated from the common frequency-independent grids 
({\tt FIG}). 
\end{abstract}

  
\section{INTRODUCTION}
\par
Seismic data is recorded with constant temporal and spatial 
intervals (e.g., 4ms and 25m). Since seismic velocity does not change with time,
many operators become time invariant and can be applied more economically in 
the temporal frequency domain (sometimes with the aid of the log-stretch 
transform). 
Spatial prediction for signal/noise separation \cite{SEG.1984.S10.1} and
spatial prediction for interpolation \cite{GEO56.06.07850794} are two 
examples in the frequency-space (F-X-Y) domain. In the above two applications, 
the operators are conducted respectively from one frequency to another.
\par
The subset of the data which contain all X-Y samples for one frequency is
called a {\tt frequency slice}.  
All the frequency slices from low to high frequency have the same size in 
the frequency-space domain.  
According to Shannon's sampling theorem, we can use longer spatial interval 
to resample the low frequency data, which also means that the low frequency 
components are usually oversampled.
\par
We present an new way to represent a seismic dataset, which is 
called {\tt pyramid transform}. We take the conventional dataset in the F-X-Y 
domain as input and resample it in accordance with sampling theorem. The 
output is still in the F-X-Y domain. But it is no longer in the form of a cube.
The shape of the output is a pyramid. More specifically, the lower 
the frequency, the smaller the number of data samples. But the pyramid is  
equivalent to the cube in terms of information amount. In other words, the 
{\tt pyramid transform} is invertible. If we conduct the inverse transform, 
we can get back to the input dataset.
\par 
The dataset in the {\tt pyramid domain} has some interesting features. 
For example, the spatial prediction filter is independent of 
temporal frequency. This means that we only need to estimate {\bf one} 
prediction filter from many different frequencies. This feature will make 
those operators more efficient.
\par
In this paper, we first discuss the {\tt pyramid transform} and show how to 
make it invertible. Then we prove the advantages of the pyramid domain 
theoretically. Finally we apply this new scheme to signal/noise separation. 
Our result shows that the spatial prediction filter estimated in the pyramid 
domain can not only remove high temporal frequency noise but also eliminate 
low frequency noise, whereas the prediction filter obtained in the conventional
 way cannot handle low frequency noise. 

\section{PYRAMID TRANSFORM}
\par
Assume v is medium velocity, $\delta x$ is trace interval and $\theta$ is 
dip angle, the maximum threshold frequency that is not aliased is as follows
\cite{SDP00.00.00090522}:
\begin{equation}
        f_{\rm max} = \frac{v}{4\delta x \sin\theta}
\end{equation}
We can rewrite the above formula in another form:
\begin{equation}
        \sin\theta = \frac{v}{4f \delta x} 
\label{eqn:eqn1}
\end{equation}
Equation (\ref{eqn:eqn1}) is the relation between the dip angle and  
the threshold frequency. If $f$ decreases, $\theta$ will increase. 
Since $0^\circ \le \theta \le 90^\circ$, there is a critical frequency 
$\bar{f}$ at which
\begin{equation}  
        \sin\theta = \frac{v}{4\bar{f}\delta x} = 1
\label{eqn:eqn2}
\end{equation}
So we conclude that 
\begin{itemize}
	\item For $f<\bar{f}$, the frequency slice contains information up to 
$90^\circ$. In other words, the data is oversampled. 
	\item For $f>\bar{f}$, the frequency slice only contains information 
$\theta \le \theta_{\rm max} (<90^\circ)$. Dip over $\theta_{\rm max}$ will be  
aliased. And $\theta_{\rm max}$ is defined by
$$
	\theta_{\rm max} = sin^{-1}(\frac{v}{4\bar{f}\delta x})
$$
\end{itemize}
The existence of $\bar{f}$ provides the feasibility of resampling the data in 
the {\tt f-x} domain without losing any information. Resampling means that 
$\delta x$ is no longer constant but a function of frequency, i.e., 
$\Delta x_{\rm f}$. When $f<\bar{f}$ and $\theta=90^\circ$, equation 
(\ref{eqn:eqn1}) can be transferred into 
\begin{equation}
	\Delta x_{\rm f} = \frac{v}{4f}
\label{eqn:eqn3}
\end{equation}
Equation (\ref{eqn:eqn3}) shows that we can subsample the frequency slices using 
larger trace interval. According to sampling theorem, the resampled data 
does not lose any information theoretically. 
We call this resampling scheme ``pyramid transform'' and the output a special 
kind of F-X domain, i.e., {\tt pyramid domain}. 
\par
Figure \ref{fig:sin-value} shows the RHS of equation (\ref{eqn:eqn2}) versus frequency. 
The solid curve represents the frequency-independent grids. 
The dash line represents the frequency-dependent grids. 
From this figure, we can see that all the frequency lower than $\bar{f}=94Hz$ 
have oversampled the data. Therefore, the next question is what is the 
disadvantage of oversampling the data?

\inputdir{model}
\sideplot{sin-value}{width=3.0in,height=2.5in}{$\frac{v}{4f\delta x}$ versus {\tt f}. The solid curve represents the frequency-independent grids. The dash line represents the frequency-dependent grids.}

Unless carefully constrained, an over-parameterized model may
contain illegal components that will fit noisy data better
than a better model which does not include high wavenumbers
for low frequencies. 
\par
Spatial prediction filters become the same for all frequencies
\cite {Nichols.1996}.
Since the prediction filter is independent of frequency, 
it will become more stable and more 
insensitive to the noise. Also, we do not need to estimate independent
prediction filters for each frequency slice separately.
\par
In any case, oversampling will not bring any profit to noise suppression. So 
we will use $\delta x$, not $\Delta x_f$, when $f>\bar{f}$. 
Figure \ref{fig:pyramid} shows that when $f>\bar{f}$, $\delta x$ is actually the 
base of another pyramid. So we can regard the input data $(f>\bar{f})$ as 
frequency- dependent grids directly.
\par
In other words, we can get a series of pyramids corresponding to the 
frequency which is higher than $\bar{f}$. From each pyramid, we can estimate 
a prediction filter. The difference between the first prediction filter (
$f \le \bar{f}$) and the other ($f>\bar{f}$) , 
is that we can apply the first prediction filter to all the frequencies which 
are lower than $\bar{f}$. 
Each of the remaining filters can only be applied to one frequency slice, 
which is also the base of that pyramid. 

\inputdir{XFig}
\plot{pyramid}{width=4.0in,height=2.5in}{The size of the pyramid(in 2-D, it is actually a triangle) is changing when $(f>\bar{f})$.} 
\par
The basic and most important requirement for pyramid 
transform is that this transform is invertible. In other words, ``forward'' 
transform can turn a rectangular {\tt f-x} domain to a pyramid domain and 
``inverse'' transform can turn a pyramid domain back to a rectangular {\tt f-x}
 domain, as shown in Figure \ref{fig:transform}.

\plot{transform}{width=4.0in,height=2.5in}{Schematic demonstration of forward/inverse pyramid transform.}

\inputdir{model}

How do we guarantee this transform invertible? The key is to find a good 
complex-valued interpolation scheme. Here, we use Dave Hale's 8-point sinc 
interpolation program in SU. Some more accurate and complicated interpolation 
scheme can be found in Wade's paper \shortcite{SEG.1988.SP1.6}. 
Figure \ref{fig:forward-inverse-transform} is one example using 8-point sinc 
interpolation. 
The result shows that our scheme is very accurate. There is little difference 
between input and output.

\plot{forward-inverse-transform}{width=4.0in,height=2.5in}{The left is the input dataset, the middle is the result after forward/inverse transform, and the right one is the difference.(Displayed at the same scale)}
 
\section{ESTIMATION OF SPATIAL PREDICTION FILTERS}

\par
In this section, we discuss the theory of estimating the prediction filter for 
plane wave. We use one and two plane waves as examples. First, we prove that 
plane wave is predictable. Then we show how to estimate the spatial prediction 
filter for both frequency-independent grids ({\bf FIG}) and frequency-dependent
 grids ({\bf FDG}). 
The difference is also the advantage of {\bf FDG} over {\bf FIG}. In the 
appendix, we extend our theory to N plane waves. 
\subsection{Wavefield composed of one plane wave}

\par
As shown in Figure \ref{fig:one-plane-wave}, this wavefield in the {\tt t-x} domain
 can be expressed by
\begin{equation}
	w(t,x) = \delta (t-t_1-p_1x)
\end{equation}

\sideplot{one-plane-wave}{width=3.0in,height=2.5in}{A synthetic wavefield composed of a single plane wave.}

After Fourier transform along the time axis, we get the expression in the 
{\tt f-x} domain: 
\begin{equation}
	W(f,x) = e^{i2\pi f (t_1+p_1x)}
\end{equation}
It is easy to show that one trace can be predicted from an adjacent trace:
\begin{equation}
	W(f,x-\Delta x)=e^{i2\pi f (t_1+p_1x-p_1\Delta x)}=W(f,x)e^{-ip_12\pi f \Delta x}
\end{equation}
Or in another form,
\begin{equation}
	W(f,x)=e^{ip_12\pi f \Delta x}W(f,x-\Delta x)
\end{equation}
This means that, each trace can be predicted with the propagator 
$P_1=e^{ip_12\pi f \Delta x}$. Up to now, we have not mentioned {\bf FIG} 
and {\bf FDG}. Therefore, all above conclusions are effective for both cases. 
But when we introduce the details of the two schemes, we will reach different 
conclusions:
\begin{itemize}

	\item Frequency-independent grids \\ \\
	$\Delta x = \delta x$ is constant. The propagator $P_1$ is the function
	of frequency $f$. It will change from one frequency slice to 
	another.

	\item Frequency-dependent grids \\ \\
	From (\ref{eqn:eqn2}), we can choose
\begin{equation}
	\Delta x = \frac{v}{4 f}
\end{equation}
Therefore, the propagator $P_1$ becomes
\begin{equation}
	P_1 = e^{i\frac{\pi}{2}p_1v}
\end{equation}
So the propagator $P_1$ is frequency-invariant. 
\end{itemize}

\subsection{Wavefield composed of two plane waves}

If we have two plane waves, the wavefield (Figure \ref{fig:two-plane-wave}) can be 
expressed by
\begin{equation}
	w(t,x) = a_1\delta(t-t_1-p_1x)+a_2\delta(t-t_2-p_2x)
\end{equation}

\sideplot{two-plane-wave}{width=3.0in,height=2.5in}{A synthetic wavefield composed of two plane waves.}

In the {\rm f-x} domain, it has another expression
\begin{equation}
	W(f,x) = a_1e^{i2\pi f(t_1+p_1x)}+a_2e^{i2\pi f(t_2+p_2x)}
\end{equation}
Or
\begin{equation}
	W(f,x) = E_1+E_2
\label{eqn:P0}
\end{equation}
where $E_1 = a_1e^{i2\pi f(t_1+p_1x)}$ and $E_2=a_2e^{i2\pi f(t_2+p_2x)}$.
The adjacent trace can be expressed by
\begin{equation}
	W(f,x-\Delta x)=P^{-1}_1E_1+P^{-1}_2E_2
\label{eqn:P1}
\end{equation}
Now we cannot predict one trace from another trace, since there are two 
unknown propagators. Consequently, we need the information from another trace.
\begin{equation}
	W(f,x-2\Delta x)=P^{-2}_1E_1+P^{-2}_2E_2 
\label{eqn:P2}
\end{equation}
Equation (\ref{eqn:P1}) and (\ref{eqn:P2}) can be solved for the propagators. But in 
reality, we do not know $p_1$ and $p_2$. So we cannot get the explicit 
form of the propagators. We must use another form of parameter to link the 
the different traces together, which is called a spatial prediction filter.
\par
If we have trace $x-\Delta x$ and trace $x-2\Delta x$, our aim is 
to predict trace $x$. Assume there is a relation between the three traces:
\begin{equation}
	W(f,x) = C_1W(f,x-\Delta x)+C_2W(f,x-2\Delta x)
\end{equation}
Substituting $W$s with equations (\ref{eqn:P0}), (\ref{eqn:P1}) and (\ref{eqn:P2}), we get
\begin{equation}
	E_1+E_2 = C_1(P^{-1}_1E_1+P^{-1}_2E_2)+C_2(P^{-2}_1E_1+P^{-2}_2E_2)
\label{eqn:P3}
\end{equation}
$E_{\rm i}$ can be any value, as long as it is a plane wave. Therefore, 
equation (\ref{eqn:P3}) holds for all of $E_1$ and $E_2$. In particular, we have
\begin{equation}
1 = C_1P^{-1}_1+C_2P^{-2}_1, \hspace*{0.2in}(E_1 = 1, \hspace*{0.1in}E_2 = 0)
\end{equation}
\begin{equation}
1 = C_1P^{-1}_2+C_2P^{-2}_2, \hspace*{0.2in}(E_1 = 0, \hspace*{0.1in}E_2 = 1)
\end{equation}
Solving this linear equation, we find the relation between $C_{\rm i}$ and 
$P_{\rm i}$ $(i=1,2)$. That is
\begin{equation}
	C_1 = P_1 + P_2, \hspace*{0.3in} C_2 = -P_1P_2
\end{equation}

Like the one plane wave case, all above deductions have nothing to do with the 
gridding scheme. All the results are effective for both cases. The differences 
between the two gridding schemes are
\begin{itemize}

	\item Frequency-independent grids\\ \\
	As proved before, since the propagator $P_{\rm i}$ depends on 
	frequency, the coefficients of the prediction filter $C_{\rm i}$ will 
	also depend on frequency.

	\item Frequency-dependent grids\\ \\
	Since the propagator $P_{\rm i}$ is independent of frequency, the 
	coefficients of the prediction filter $C_{\rm i}$ are also independent 
	of frequency.

\end{itemize} 

We can extend our deduction to N plane waves (see the Appendix) and we can get 
similar results: for {\bf FIG}, coefficient $C_{\rm i}$ is frequency-dependent;
 for {\bf FDG}, $C_{\rm i}$ is frequency-independent. This is the basis of 
estimating the spatial prediction filter.


\subsection{Estimating the spatial prediction filter}
\par
We use subscript $k$ to represent the location of the trace, 
i.e., $W_{\rm k} = W(f,k\Delta x)$. If $W_k$ is predictable from $(W_{\rm k-1}, 
W_{\rm k-2}, ..., W_{\rm k-N})$, we have
\begin{equation}
	W_{\rm k} = \sum_{j=1}^N C_{\rm j} W_{\rm k-j}, \hspace*{0.2in}(k=2,3,...,N_2)
\end{equation}
where $N_2$ is the length of the spatial axis.
\par
Determination of the coefficients $C_{\rm j}$ can be treated as an optimization
 problem. Here we choose conjugate gradient code {\tt cgplus} 
\cite{Claerbout.tdf.82} as the solver. Originally, {\tt cgplus} was used for 
real-valued optimization. We extend it to complex-valued optimization.  
Between the two gridding schemes, the most important difference is that
\begin{itemize}

	\item Frequency-independent grids
	\begin{itemize}
		\item Prediction filter is estimated in the rectangular 
		{\tt f-x} domain.
		\item For each frequency slice, we need to solve an 
		optimization problem separately. And the coefficients change 
		from one slice to another.
	\end{itemize}
	\item Frequency-dependent grids
	\begin{itemize}
		\item Prediction filter is estimated in the pyramid domain.
		\item We incorporate all the equations together and only solve 
		one optimization problem. But the coefficients can be applied 
		to all the frequencies within the pyramid. 
	\end{itemize}
\end{itemize}
Hopefully, this difference will save computing resources in 3-D. Also, we can 
expect that the result from {\bf FDG} will more stable.


\section{APPLICATION AND DISCUSSION}
\par
In this section, we compare our new sampling scheme in the application of 
signal/noise separation with F-X noise suppression. We expect that our scheme 
will produce better results. Figure \ref{fig:flow-chart} shows the relation and 
difference between the two implementations:

\inputdir{XFig}
\plot{flow-chart}{width=4.0in,height=3.5in}{The relation between F-X noise suppression and our scheme. We estimate the spatial prediction filter in the pyramid domain. The dashed-line box includes all the operations in the pyramid domain.}
\par
The input dataset is composed of one horizontal event, four dipping events.
This dataset has no random noise. Therefore, it is fully predictable. 
We use this data to test whether subsampling will influence the predictability 
of each frequency slice. Figure \ref{fig:input1} shows the data
in both {\tt t-x} and {\tt f-x} domain.  
\activeplot{input1}{width=6.0in,height=3.5in}{.}{Input data in the {\tt t-x} domain and {\tt f-k} domain.}
\par
F-X noise suppression algorithm estimates the spatial prediction filter for 
each frequency slice and then applies it to the input data. 
Figure \ref{fig:FIG1} shows that there is little difference 
between input and output. This means that if there is no noise or aliased 
energy in the dataset, the prediction filter will produce a dataset which is 
the same as the input.
\activeplot{FIG1}{width=6.0in,height=3.5in}{.}{Result from F-X noise suppression algorithm in the {\tt t-x} domain. The left is the output and the right is the difference between input and output.} 
\par
Figure \ref{fig:FDG1} is the corresponding result from our scheme. It is clear 
that subsampling does not make the frequency slice more unpredictable. All 
the five events have been retained after convolving the prediction filter with 
the input dataset.
\activeplot{FDG1}{width=6.0in,height=3.5in}{.}{Result from Pyramid Transform scheme in the {\tt t-x} domain. The left is the output and the right is the difference between input and output.}
Figure \ref{fig:error1} is the relative residual error for each frequency. 
It is interesting to notice that subsampled dataset is even more predictable 
than the oversampled dataset. 
\activesideplot{error1}{width=3.0in,height=3.5in}{.}{Squared error for both F-X noise suppression and Pyramid scheme. Notice that the subsampled dataset is even more predictable than the oversampled dataset.} 
\par
The second dataset is the same as the first one, except that we introduce 
random noise into it, as shown in Figure \ref{fig:input2}. 
\activeplot{input2}{width=6.0in,height=3.5in}{.}{Input data in the {\tt t-x} domain and {\tt f-k} domain. There is no random noise in the low frequency, high wavenumber part.}
\par
Unfortunately, we have not got very good results from noisy data. As shown in 
Figure \ref{fig:FIG2}, F-X noise suppression cannot remove low frequency noise. 
Therefore, the difference is very small. Although Pyramid scheme can remove 
some parts of low frequency noise, it can also hurt the signal at the same 
time. We will keep working on it and try to get a better result.
\activeplot{FIG2}{width=6.0in,height=3.5in}{.}{Result from F-X scheme. The left is the output, the middle is the difference between input and output, and the right is the output in {\tt f-k} domain. Only some high frequency noise is eliminated from the input. Therefore the difference is very small.} 
\activeplot{FDG2}{width=6.0in,height=3.5in}{.}{Result from Pyramid scheme. The left is the output, the middle is the difference between input and output, and the right is the difference between input and output, and the right is the output in {\tt f-k} domain. Pyramid scheme can remove both high and low frequency noise. But it also remove some signal components.}
\activesideplot{error2}{width=3.0in,height=3.5in}{.}{Squared error for both schemes.} 

\section{CONCLUSION AND FUTURE WORK}
\par
Pyramid transform is a new sampling scheme. It is based on Shannon's sampling 
theorem. Theoretically pyramid transform will not introduce error to the 
dataset. After the pyramid transform, we get a frequency-dependent grid. This
 new grid has a feature that $f\Delta x = const$. This feature 
makes the spatial prediction filter estimated in the pyramid domain 
frequency-independent. We apply this new sampling scheme to signal/noise 
separation. Our result shows that subsampling will not make the data more 
unpredictable. The subsampled data is even more predictable than the 
oversampled one. Pyramid scheme can eliminate low frequency noise, but it 
also hurts the signal as well. We will continue this project and try to get 
better results.
  
\bibliographystyle{sep}\bibliography{MISC,SEP,GEOTLE,SEGCON,SEGBKS,../yalei1/paper}
\appendix
\section{Appendix A}

This appendix shows the pseudo-code used for angle decomposition
indicating the loop order and the link with the theory discussed
in the body of the paper.

% ------------------------------------------------------------
\graybox{
  \singlespacing
  \hrule
  \EICdecomposition
  \hrule
  \sc{Angle decomposition algorithm for extended CIPs.}
}
% ------------------------------------------------------------

