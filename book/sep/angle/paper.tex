%\shortnote

\lefthead{Fomel \& Prucha}
\righthead{Angle-gather time migration}
\footer{SEP--100}

\published{SEP report, 100, 141-150 (1999)}
\title{Angle-gather time migration}

\email{sergey@sep.stanford.edu, marie@sep.stanford.edu}
%\keywords{time migration, AVO}

\author{Sergey Fomel and Marie Prucha}
\maketitle

\begin{abstract} 
Angle-gather migration creates seismic images for different
  reflection angles at the reflector. We formulate an angle-gather
  time migration algorithm and study its properties. The algorithm
  serves as an educational introduction to the angle gather concept.
  It also looks attractive as a practical alternative to conventional
  common-offset time migration both for velocity analysis and for
  AVO/AVA analysis.
\end{abstract}

\section{Introduction}

Angle-gather migration creates seismic images collected by the
reflection angle at the point of reflection. Major advantages of this
approach are apparent in the case of prestack depth migration. As
shown by \cite{Prucha.sep.100.marie1}, the ray 
pattern of angle-gather migration is significantly different from that
of the conventional common-offset migration. The difference can be
exploited for overcoming illumination difficulties of the conventional
depth migration in complex geological areas.
\par
In this paper, we explore the angle-gather concept in the case of
prestack time migration. The first goal of this study is educational.
Since we can develop the complete mathematical theory of angle-gather
time migration analytically, it is much easier to understand the most
basic properties of the method in the time migration domain. The
second goal is practical. Angle gathers present an attractive tool for
post-migration AVO/AVA studies and velocity analysis, and even the
most basic time migration approach can find a valuable place in the
complete toolbox of seismic imaging.
\par
We start with analyzing the traveltime relations for the basic
Kirchhoff implementation of angle-gather time migration. The analysis
follows Fowler's general approach to prestack time migration methods
\cite[]{SEG-1997-1571}. Next, we derive formulas for the amplitude weighting
and discuss some frequency-domain approaches to angle gathers.
Finally, we present simple synthetic tests of the method and discuss
further research directions.

\section{Traveltime considerations}
\inputdir{XFig}

\sideplot{rays}{width=3in}{Reflection rays in a constant-velocity
  medium: a scheme.}

Let us consider a simple reflection experiment in an effectively
constant-velocity medium, as depicted in Figure~\ref{fig:rays}. The
pair of incident and reflected rays and the line between the source
$s$ and the receiver $r$ form a triangle in space. From the
trigonometry of that triangle we can derive simple relationships among
all the variables of the experiment
\cite[]{Fomel.sep.84.179,Fomel.sep.92.159,SEG-1997-1762}.
\par
Introducing the dip angle $\alpha$ and the reflection angle $\gamma$,
the total reflection traveltime $t$ can be expressed from the law of
sines as
\begin{equation}
  t = {\frac{2 h}{v}}\,
  {\frac{\cos(\alpha+\gamma) + \cos(\alpha-\gamma)}{\sin{2\,\gamma}}} =
  {\frac{2 h}{v}\,\frac{\cos{\alpha}}{\sin{\gamma}}}\;,
\label{eqn:h2t}
\end{equation}
where $v$ is the medium velocity, and $h$ is the half-offset between
the source and the receiver. 
\par
Additionally, by following simple trigonometry, we can connect the
half-offset $h$ with the depth of the reflection point $z$, as
follows:
\begin{equation}
  h = {\frac{z}{2}}\,
  {\frac{\sin{2\,\gamma}}{2\,\cos(\alpha+\gamma)\,\cos(\alpha-\gamma)}} =
  {z\,\frac{\sin{\gamma}\,\cos{\gamma}}{\cos^2{\alpha} - \sin^2{\gamma}}}\;.
\label{eqn:z2h}
\end{equation}
\par
Finally, the horizontal distance between the midpoint $x$ and the
reflection point $\xi$ is
\begin{equation}
  \label{eqn:h2x}
  x - \xi = h\,\frac{\cos(\alpha-\gamma)\,\sin(\alpha+\gamma)\,+\,
    \cos(\alpha+\gamma)\,\sin(\alpha-\gamma)}{\sin{2\,\gamma}} =
  h\,\frac{\sin{\alpha}\,\cos{\alpha}}{\sin{\gamma}\,\cos{\gamma}}
\end{equation}
\par
Equations (\ref{eqn:h2t}--\ref{eqn:h2x}) completely define the
kinematics of angle-gather migration. Regrouping the terms, we can
rewrite the three equations in a more symmetric form:
\begin{eqnarray}
  \label{eqn:t}
  t & = & \frac{2\,z}{v}\,
  \frac{\cos{\alpha}\,\cos{\gamma}}{\cos^2{\alpha} - \sin^2{\gamma}}
  \\ \label{eqn:h}
  h & = & z\,
  \frac{\sin{\gamma}\,\cos{\gamma}}{\cos^2{\alpha} - \sin^2{\gamma}}
  \\ \label{eqn:x}
  x - \xi & = & z\,
  \frac{\sin{\alpha}\,\cos{\alpha}}{\cos^2{\alpha} - \sin^2{\gamma}}
\end{eqnarray}
For completeness, here is the inverse transformation from $t$, $h$,
and $x-\xi$ to $z$, $\gamma$, and $\alpha$:
\begin{eqnarray}
  \label{eqn:z}
  z^2 & = & 
  \frac{
    \left[(v\,t/2)^2 - (x-\xi)^2\right]\,
    \left[(v\,t/2)^2 - h^2\right]
    }{(v\,t/2)^2} 
  \\ \label{eqn:gamma}
  \sin^2{\gamma} & = & 
  \frac{h^2\, \left[(v\,t/2)^2 - (x-\xi)^2\right]}
  {(v\,t/2)^4 - h^2\,(x-\xi)^2}
  \\ \label{eqn:alpha}
  \cos^2{\alpha} & = & 
  \frac{(v\,t/2)^2\, \left[(v\,t/2)^2 - (x-\xi)^2\right]}
  {(v\,t/2)^4 - h^2\,(x-\xi)^2}
\end{eqnarray}
The inverse transformation (\ref{eqn:z}-\ref{eqn:alpha}) can be found
by formally solving system (\ref{eqn:t}-\ref{eqn:x}).
\par
The lines of constant reflection angle $\gamma$ and variable dip angle
$\alpha$ for a given position of a reflection (diffraction) point
$\{z,\xi\}$ have the meaning of summation curves for angle-gather
Kirchhoff migration. The whole range of such curves for all possible
values of $\gamma$ covers the diffraction traveltime surface -
``Cheops' pyramid'' \cite[]{Claerbout.blackwell.85} in the $\{t,x,h\}$
space of seismic reflection data.  As pointed out by
\cite{SEG-1997-1571}, this condition is sufficient for proving the
kinematic validity of the angle-gather approach.  For comparison,
Figure \ref{fig:coffset} shows the diffraction traveltime pyramid from
a diffractor at 0.5 km depth.  The pyramid is composed of
common-offset summation curves of the conventional time migration.
Figure \ref{fig:cangle} shows the same pyramid composed of
constant-$\gamma$ curves of the angle-gather migration.

\inputdir{Math}

\sideplot{coffset}{width=3in}{Traveltime pyramid, composed of
  common-offset summation curves.}

\sideplot{cangle}{width=3in}{Traveltime pyramid, composed of
  common-reflection-angle summation curves.}

\par
The most straightforward Kirchhoff algorithm of angle-gather migration
can be formulated as follows:
\begin{itemize}
\item For each reflection angle $\gamma$ and each dip angle $\alpha$,
  \begin{itemize}
  \item For each output location $\{z,\xi\}$,
    \begin{enumerate}
    \item Find the traveltime $t$, half-offset $h$, and midpoint
      $x$ from formulas (\ref{eqn:t}), (\ref{eqn:h}), and
      (\ref{eqn:x}) respectively.
    \item Stack the input data values into the output.
    \end{enumerate}
  \end{itemize}
\end{itemize}
As follows from equations (\ref{eqn:t}-\ref{eqn:x}), the range of
possible $\alpha$'s should satisfy the condition
\begin{equation}
  \label{eqn:range}
  \cos^2{\alpha} > \sin^2{\gamma}\quad\mbox{or}\quad
  |\alpha| + |\gamma| < \frac{\pi}{2}\;.
\end{equation}
The described algorithm is not the most optimal in terms of the
input/output organization, but it can serve as a basic implementation
of the angle-gather idea. The stacking step requires an appropriate
weighting. We discuss the weighting issues in the next section.

\section{Amplitude considerations}

One simple approach to amplitude weighting for angle-gather migration
is based again on Cheops' pyramid considerations. Stacking along the
pyramid in the data space is a double integration in midpoint and
offset coordinates. Angle-gather migration implies the change of
coordinates from $\{x,h\}$ to $\{\alpha,\gamma\}$. The change of
coordinates leads to weighting the integrand by the following Jacobian
transformation:
\begin{equation}
  \label{eqn:jacob}
  dx\,dh = \left| \det \left(
        \begin{array}{cc}
          \frac{\partial x}{\partial \alpha} & 
          \frac{\partial x}{\partial \gamma} \\
          \frac{\partial h}{\partial \alpha} & 
          \frac{\partial h}{\partial \gamma}
        \end{array}
      \right) \right|\,d\alpha\,d\gamma
\end{equation}
Substituting formulas (\ref{eqn:h}) and (\ref{eqn:x}) into equation
(\ref{eqn:jacob}) gives us the following analytical expression for the
Jacobian weighting:
\begin{equation}
  \label{eqn:ajacob}
  W_{\mbox{J}} = \left| \det \left(
      \begin{array}{cc}
        \frac{\partial x}{\partial \alpha} & 
        \frac{\partial x}{\partial \gamma} \\
        \frac{\partial h}{\partial \alpha} & 
        \frac{\partial h}{\partial \gamma}
      \end{array}
    \right) \right| = 
  \frac{z^2}{\left(\cos{\alpha}^2 - \sin{\gamma}^2\right)^2}
\end{equation}
Weighting (\ref{eqn:ajacob}) should be applied in addition to the
weighting used in common-offset migration. By analyzing formula
(\ref{eqn:ajacob}), we can see that the weight increases with the
reflector depth and peaks where the angles $\alpha$ and $\gamma$
approach condition (\ref{eqn:range}).
\par
The Jacobian weighting approach, however, does not provide physically
meaningful amplitudes, when migrated angle gathers are considered
individually. In order to obtain a physically meaningful amplitude, we
can turn to the asymptotic theory of true-amplitude migration
\cite[]{tam,GEO58-08-11121126,tygel}. The true-amplitude weighting provides an
asymptotic high-frequency amplitude proportional to the reflection
coefficient, with the wave propagation (geometric spreading) effects
removed. The generic true-amplitude weighting formula
\cite[]{Fomel.sep.92.267} transforms in the case of 2-D angle-gather
time migration to the form:
\begin{equation}
  \label{eqn:ta}
  W_{\mbox{TA}} = \frac{1}{\sqrt{2\,\pi}}\,
  \frac{\sqrt{L_s\,L_r}}{v\,\cos{\gamma}}\,
  \left|\frac{\partial^2 L_s}{\partial \xi \partial \gamma} +
    \frac{\partial^2 L_r}{\partial \xi \partial \gamma} 
  \right|\;,  
\end{equation}
where $L_s$ and $L_r$ are the ray lengths from the reflector point to
the source and the receiver respectively. After some heavy algebra,
the true-amplitude expression takes the form
\begin{equation}
  \label{eqn:ta1}
  W_{\mbox{TA}} = \frac{2\,z\,\sin{\alpha}}{\sqrt{2\,\pi} v}\,
  \frac{\cos^2{\alpha} + \sin^2{\gamma}}
  {\left(\cos^2{\alpha} - \sin^2{\gamma}\right)^{5/2}}\;.
\end{equation}
Under the constant-velocity assumption and in high-frequency
asymptotic, this weighting produces an output, proportional to the
reflection coefficient, when applied for creating an angle gather with
the reflection angle $\gamma$. Despite the strong assumptions behind
this approach, it might be useful in practice for post-migration
amplitude-versus-angle studies. Unlike the conventional common-offset
migration, the angle-gather approach produces the output directly in
reflection angle coordinates. One can use the generic true-amplitude
theory \cite[]{Fomel.sep.92.267} for extending formula (\ref{eqn:ta1})
to the 3-D and 2.5-D cases.

\section{Examples}

\inputdir{dome}

We created some simple synthetic models with constant velocity backgrounds
to test our angle-gather migration method.  One model is a simple dome 
(Figure~\ref{fig:data}).  The other has a series of flat reflectors of 
various dips (Figure~\ref{fig:data.lines}).  Both of these figures also
show the corresponding data that will be generated by Kirchhoff methods
for zero and far offsets.

\plot{data}{width=6.in}{Left: Model. Center: Data at zero offset. 
Right: Data at far offset.}

\plot{data.lines}{width=6.in}{Left: Model. Center: Data at zero offset. 
Right: Data at far offset.}

\par
\subsection{Dome model}

This model contains a wide range of geologic dips across the dome as well
as having a flat reflector at the base of the dome.  
Figure~\ref{fig:offset.dome} shows the resulting common offset sections from
traditional Kirchhoff migration.  As is expected for such a simple model,
the near and far offset sections are very similar and the stacked section
is almost perfect.  We are more interested in the result of the angle-gather
migration.

Figure~\ref{fig:angle-ta.dome} shows the zero and large angle sections as well
as the stack for angle-gather Kirchhoff migration.  The zero-angle section
is weak but clearly shows the correct shape and position.  The large-angle
section is actually only for $\gamma=25^{\circ}$. 
The reason for this is clear if you consider Figure~\ref{fig:rays}.  At 
greater depths, the rays associated with large reflection angles ($\gamma$)
will not emerge at the surface within the model space.  Therefore at angles
greater than $25^{\circ}$ (the maximum useful angle), the information at later 
times disappears.  

We expect the stacked sections for the offset method and the angle method 
to be identical.  Although we sum over different paths for the offset-domain 
migration (Figure~\ref{fig:coffset}) and the angle-domain migration 
(Figure~\ref{fig:cangle}), the stack should sum all of the same information
together for both methods.  Fortunately, a comparison of the stacked sections
in Figures~\ref{fig:offset.dome}~and~\ref{fig:angle-ta.dome} show that the
results are identical as expected.
 
\plot{offset.dome}{width=6.in}{Left: Migrated offset section at zero offset.
Center: Migrated offset section at far offset.  Right: Stack.}

\plot{angle-ta.dome}{width=6.in}{Left: Migrated angle section at small angle.
Center: Migrated offset section at large angle.  Right: Stack.}

\par
\subsection{Dipping reflectors model}

This model contains fewer dips than the dome model but it allows us to 
see what is happening at later times.  Figure~\ref{fig:offset.lines} 
shows the common offset sections  and stacked section from offset-domain 
Kirchhoff migration.  Once again, they are practically perfect.  The only 
problem is near the bottom of the section where we lose energy because the
data was truncated.

The zero-angle and large-angle sections from the angle-domain migration are
in Figure~\ref{fig:angle-ta.lines}, along with the stacked section.  Once 
again, the zero angle section is very weak and the large angle section only
contains information down to a time of $\approx .85$ seconds, for the same
reason as explained for the dome model.  

Once again, we expect the stacked sections in 
Figures~\ref{fig:offset.lines}~and~\ref{fig:angle-ta.lines} to be the same. 
Although the angle-domain stack is slightly lower amplitude throughout the
section, it is clear that this is a simple scale factor so our expectations
remain intact. 

\plot{offset.lines}{width=6.in}{Left: Migrated offset section at zero offset.
Center: Migrated offset section at far offset.  Right: Stack.}

\plot{angle-ta.lines}{width=6.in}{Left: Migrated angle section at zero angle.
Center: Migrated angle section at large angle.  Right: Stack.}

\par
\subsection{Reflectivity variation with angle}

Amplitude variation with offset (AVO) would not be expected to be very
interesting for the simple models just shown.  Consider  
Figure~\ref{fig:reflect-ta.dome} which contains an offset gather and a 
reflection angle gather taken from space location zero from the dome model in
Figure~\ref{fig:data}.  The offset gather shows exactly what we expect 
for such a model - no variation.  The angle gather also shows no variation
      for angles less than the maximum useful angle ($25^{\circ}$) as discussed in the
previous two subsections.  However, when the angle exceeds the maximum useful
angle, the event increases in amplitude and width.  This is the phenomenon
seen in de Bruin et al. \shortcite{deBruin}. 

\plot{reflect-ta.dome}{width=6.in}{Gathers taken from space location zero
in the dome model.  Left: Offset domain.  Center: Angle domain less than
$25^{\circ}$.  Right: Angle domain.}

\subsection{Velocity sensitivity}

When dealing with real data we almost never know what the true velocity of 
the subsurface is.  Therefore it is important to understand the effects of
velocity on our angle-gather time migration algorithm.  To do this we
simply created data for the dome model in Figure~\ref{fig:data} at a
fairly high velocity (3 km/s) and migrated it using a low velocity (1.5 km/s).
The results are in Figure~\ref{fig:reflect-ta.fast.dome}.  For angles less
than the maximum useful angle ($\gamma=25^{\circ}$), the angle-domain gather
behaves exactly as the offset-domain gather does.  Beyond the maximum
useful angle, the events become even more curved and the amplitudes begin
to change.

The behavior of the angle-gather migration is very similar to that of 
offset-domain migration as long as the limitation of the maximum useful
angle is recognized.  Therefore, we can probably expect angle-gather
migration to behave like offset-domain migration in $v(z)$ media also.  

\plot{reflect-ta.fast.dome}{width=6.in}{Gathers taken from space location
zero inthe dome model and migrated at too low a velocity.  Left: Offset
domain. Center: Angle domain less than $25^{\circ}$.  Right: Angle domain.}  


\section{Frequency-domain considerations}

As pointed out by \cite{Prucha.sep.100.marie1}, the angle gathers
can be conveniently formed in the frequency domain. This conclusion
follows from the simple formula \cite[]{Fomel.sep.92.159}
\begin{equation}
  \label{eqn:freq}
  \tan{\gamma} = \frac{\partial z}{\partial h}\;,
\end{equation}
where $z$ refers to the depth coordinate of the migrated image. In the
frequency-wavenumber domain, formula (\ref{eqn:freq}) takes the
trivial form
\begin{equation}
  \label{eqn:freq1}
  \tan{\gamma} = \frac{k_h}{k_z}\;.
\end{equation}
It indicates that angle gathers can be conveniently formed with the help
of frequency-domain migration algorithms \cite[]{GEO43-01-00230048}.
This interesting opportunity requires further research.

\section{Conclusions}

We have presented an approach to time migration based on angle
gathers. The output of this procedure are migrated angle gathers -
images for constant reflection angles. When stacked together, angle
gathers can produce the same output as the conventional common-offset
gathers. Looking at angle gathers individually opens new possibilities
for amplitude-versus-angle studies and for velocity analysis. 
\par
Our first synthetic tests produced promising results. In the future,
we plan to study the amplitude behavior of angle-gather migration and the
velocity sensitivity more carefully. We also plan to investigate 
the frequency-domain approaches to this method. Initial results indicate
that angle-gather migration is comparable to offset-domain migration for
angles less than the angle at which rays exit the sides of the model, but
further study will hopefully allow us to extract useful information from 
the larger angles as well. Although the major advantages of angle gathers 
lay in the depth migration domain, it is easier to analyze the time migration 
results because of their theoretical simplicity.

%\section{Acknowledgments}

\bibliographystyle{seg}
\bibliography{SEP2,SEG,angle}

%\APPENDIX{A}

%\plot{name}{width=6in,height=}{caption}
%\sideplot{name}{height=1.5in,width=}{caption}
%
%\begin{equation}
%\label{eqn:}
%\end{equation}
%
%\ref{fig:}
%(\ref{eqn:})
