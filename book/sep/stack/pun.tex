
\section{ASYMPTOTIC PSEUDO-UNITARY OPERATOR PAIR}
%%%%%%%%%%%%%%%%%%%%%%%%%%%%%%%%%%%%%%%%%%%%%%%%%
According to the theory of asymptotic inversion, briefly reviewed in
the first part of this paper, the weighting function of the
asymptotically inverse operator is {\em inversely} proportional to the
weighting of the forward operator. On the other hand, the weighting in
the adjoint is {\em directly} proportional to the forward
weighting. This difference allows us to define a hybrid pair of
operators that possess both the property of being adjoint and the
property of being asymptotic inverse. It is appropriate to call a pair
of operators defined in this way {\em asymptotic pseudo-unitary}. The
definition of asymptotic pseudo-unitary operators follows directly
from the combination of definitions (\ref{eqn:inverse}) and
(\ref{eqn:adjoint}). Splitting the derivative operator $|{\bf D}|$ in
(\ref{eqn:inverse}) into the product of two operators, we can write the
forward operator as 
\begin{equation}
S(t,y)= {\bf A}\left[M(z,x)\right]= 
\int\,w^{(+)}(x;t,y)\,|{\bf D}|^{m/2}\,M(\theta(x;t,y),x)\,dx
\label{eqn:forward}
\end{equation}
and its asymptotic pseudo-unitary adjoint as
\begin{equation}
\widetilde{M}(z,x)={\bf \widetilde{A}}[S(t,y)]=
|{\bf D}|^{m/2}\;\int\,w^{(-)}(y;z,x)\,S(\widehat{\theta}(y;z,x),y)\;dy\;.
\label{eqn:backward}
\end{equation}
According to equation (\ref{eqn:whatw}),
\begin{equation}
w^{(+)}\,w^{(-)}={1\over{\left(2\,\pi\right)^m}} \, 
{\sqrt{\left|F\,\widehat{F}\right|\,
\left|\partial \widehat{\theta} \over \partial z\right|^m}}\;.
\label{eqn:wpwm}
\end{equation}
According to equation (\ref{eqn:tildew}),
\begin{equation}
w^{(-)}= w^{(+)}\, 
\left|\partial \widehat{\theta} \over \partial z\right|\;.
\label{eqn:wp2wm}
\end{equation}
Combining equations (\ref{eqn:wpwm}) and (\ref{eqn:wp2wm}) uniquely determines
both weighting functions, as follows:
\begin{eqnarray}
w^{(+)} & = & {1\over{\left(2\,\pi\right)^{m/2}}} \, 
\left|F\,\widehat{F}\right|^{1/4}\,
\left|\partial \widehat{\theta} \over \partial z\right|^{(m-2)/4}\;,
\label{eqn:wp} \\
w^{(-)} & = & {1\over{\left(2\,\pi\right)^{m/2}}} \, 
\left|F\,\widehat{F}\right|^{1/4}\,
\left|\partial \widehat{\theta} \over \partial z\right|^{(m+2)/4}\;.
\label{eqn:wm}
\end{eqnarray}
Equations (\ref{eqn:wp}) and (\ref{eqn:wm}) complete the definition of
asymptotic pseudo-unitary operator pair.
\par
The notion of pseudo-unitary operators is directly applicable in the
situations where we can arbitrarily construct both forward and inverse
operators. One example of such a situation is the velocity transform
considered in the next section of this paper.  In the more common
case, the forward operator is strictly defined by the physics of a
problem. In this case, we can include asymptotic inversion in the
iterative least-squares inversion by means of {\em preconditioning}
\cite[]{jin,lambare}.  The linear preconditioning operator should
transform the forward stacking-type operator to the form
(\ref{eqn:forward}) with the weighting function (\ref{eqn:wp}).
Theoretically, this form of preconditioning should lead to the fastest
convergence of the iterative least-squares inversion with respect to
the high-frequency parts of the model.

If the forward pseudo-unitary operator $\mathbf{A}_p$ can be related to
the forward modeling operator $\mathbf{A}_m$ as $\mathbf{A}_p =
\mathbf{W}_s\,\mathbf{A}_m\,\mathbf{W}_m$, where $\mathbf{W}_s$ and
$\mathbf{W}_m$ are weighting operators in the data and model domains
correspondingly, then preconditioning simply amounts to replacing the
least-squares equation
\begin{equation}
  \label{eq:lse}
S \approx \mathbf{A}_m [M]
\end{equation}
with the equation
\begin{equation}
  \label{eq:precon}
  \mathbf{W}_s [S] \approx 
  \mathbf{W}_s\,\mathbf{A}_m\,\mathbf{W}_m [P] = \mathbf{A}_p [P]\;,
\end{equation}
where $P$ is the preconditioned model. The advantage of using
equation~(\ref{eq:precon}) is in the the fact that the normal operator
$\mathbf{A}_p^T\,\mathbf{A}_p$ is closer (asymptotically) to identity and
therefore should be easier to invert than the original operator
$\mathbf{A}_m^T\,\mathbf{A}_m$ in the least-squares solution~(\ref{eqn:LS}).



