\section{Introduction}
%%%%

Integral (stacking) operators play an important role in seismic
imaging and seismic data processing. The most common applications are
common midpoint stacking, Kirchhoff migration, and dip moveout.  Other
examples include (listed in random order) Kirchhoff datuming,
back-projection tomography, slant stack, velocity transform, offset
continuation, and azimuth moveout. The use of the integral methods
increases in prestack three-dimensional processing because of their
flexibility with respect to irregularities in the data geometry.
\par
An integral operator often is used to represent the forward modeling
problem, and we invert it to solve for the model. In this paper, I
consider two different approaches to inversion. The first is
least-squares inversion, which requires constructing the adjoint
counterpart of the modeling operator.  The second approach is
asymptotic inversion, which aims at reconstructing the high-frequency
(discontinuous) parts of the model. I compare the two approaches and
introduce the notion of \emph{asymptotic pseudo-unitary} operator pair 
that ties them together.
\par
In practice, least squares inversion is often applied as an iterative
process \cite[]{GEO65-05-13641371}. The advantage of connecting it with
the asymptotic inverse theory is the ability to speed up the
iteration. This approach was used, in the context of seismic
migration, by \cite{jin} and \cite{lambare}.  Asymptotic
pseudo-unitary operators, introduced in this paper, provide a more
universal theoretical tool. One can use them to construct an
appropriate preconditioning operator for accelerating the convergence
of the least-squares methods.
\par
The first part of this paper contains a formal definition of a
stacking operator and reviews the theory of asymptotic inversion,
following the fundamental results of \cite{beylkin} and
\cite{goldin,Goldin.sep.67.171}.  According to this theory, the
high-frequency asymptotic inverse of a stacking operator is also a
stacking operator. To connect this theory with the theory of adjoint
operators, I show that the adjoint of a stacking operator can also be
included in the class of stacking operators.  The adjoint operator has
the same summation path as the asymptotic inverse but a different
weighting function.  These two results combine together to form the
definition of asymptotic pseudo-unitary integral operators. I apply
such operators to define a general preconditioning operator for
least-squares inversion. While one can apply Beylkin's theory directly
for constructing an appropriate asymptotic preconditioner,
pseudo-unitary operators accomplish the job in a more straightforward
and computationally attractive way.
\par
The second part of the paper addresses such examples of commonly used
stacking operators as wave-equation datuming, migration, velocity
transform, and offset continuation. The theory is specified for these
particular applications and accompanied by numerical examples. The
examples demonstrate the practical advantages of asymptotic
pseudo-unitary operators.


%%% Local Variables: 
%%% mode: latex
%%% TeX-master: t
%%% TeX-master: t
%%% End: 
