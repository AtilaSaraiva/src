\section{Conclusions}
%%%%
%The mathematical theory of stacking operators leads to the fundamental
%concept of asymptotic inversion. When the integral continuation
%operators are constructed by the asymptotic Green-function solution of
%the partial differential equation, they often appear to be
%asymptotically inverse to the reverse continuation.
\par
Stacking operators such as Kirchoff migration, datuming, dip moveout,
velocity transform, etc. are widely used in seismic imaging and data 
processing, and the need often arises to invert them. 

This paper fills the gap between the concept of asymptotically inverse
operators and the concept of adjoint operators by introducing the notion of
asymptotic pseudo-unitary stacking operators. A pair of asymptotic
pseudo-unitary operators possesses the property of being both adjoint and
asymptotically inverse to each other. The amplitude (weighting) functions of
these operators are completely defined by the derivatives of their 
kinematics (stacking surfaces).

The practical advantage of this unification is in the ability to
construct asymptotically optimal preconditioning for iterative
least-squares solution of inverse problems. Simple preliminary tests
are encouraging, but further practical experience is needed to confirm
the theoretical expectations.



%%% Local Variables: 
%%% mode: latex
%%% TeX-master: t
%%% TeX-master: t
%%% End: 
