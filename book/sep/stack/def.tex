
\section{THEORETICAL DEFINITION OF A STACKING OPERATOR}
%%%%%%%%%%%%%%%%%%%%%%%%%%%%%%%%%%%%%%%%%%%%%%%%%%%

In practice, integration of discrete data is performed by stacking.
In theory, it is convenient to
represent a stacking operator in the form of a continuous integral:
\begin{equation}
S(t,y)= {\bf A}\left[M(z,x)\right]= 
\int\limits_{\Omega} w(x;t,y)\,M(\theta(x;t,y),x)\,dx\;.
\label{eqn:operator}
\end{equation} 
Function $M(z,x)$ is the input of the operator, $S(t,y)$ is the
output, $\Omega$ is the summation aperture, 
$\theta$ represents the summation path, and $w$ stands for the
weighting function. The range of integration (the
operator aperture) may also depend on $t$ and $y$.  Allowing $x$ to be
a two-dimensional variable, we can use definition (\ref{eqn:operator}) to
represent an operator applied to three-dimensional data. Throughout
this paper, I assume that $t$ and $z$ belong to a one-dimensional
space, and that $x$ and $y$ have the same number of dimensions.
\par
The goal of inversion is to reconstruct some function
$\widehat{M}(z,x)$ for a given $S(t,y)$, so that $\widehat{M}$ is in some
 sense close to $M$ in equation (\ref{eqn:operator}).
