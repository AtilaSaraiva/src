\lefthead{Fomel}
\righthead{3-D plane waves}
\footer{SEP--102}
\published{SEP report, 102, 101-110 (1999)}
\title{Plane wave prediction in 3-D}

\email{sergey@sep.stanford.edu}
%\keywords{helix, filtering, 3-D}

\author{Sergey Fomel}

\shortpaper
\maketitle

\section{Introduction}

The theory of plane-wave prediction in three dimensions is described
by \cite{Claerbout.sep.77.19,gee}. Predicting a local plane wave with
T-X filters amounts to finding a pair of two-dimensional filters
for two orthogonal planes in the 3-D space.  Each of the filters
predicts locally straight lines in the corresponding plane. The system
of two 2-D filters is sufficient for predicting all but purely
vertical plane waves. In the latter case, a third 2-D filter for the
remaining orthogonal plane is needed.  \cite{Schwab.sepphd.99}
discusses this approach in more detail.
\par
Using two prediction filters implies dealing with two filtering output
volumes for each input volume. This situation becomes inconvenient
when one
uses the prediction output as a measure of coherency in the input
volume \cite[]{Claerbout.sep.77.19,Schwab.sep.92.29}. Two outputs are
obviously more difficult to interpret than one, and there is no
natural way of combining them into one image. Another difficulty
arises when plane-wave destructors are used for regularizing linear
inverse problems \cite[]{Clapp.sep.95.bob1}. We cannot
apply an efficient recursive preconditioning
\cite[]{Claerbout.sep.97.jon2} unless the regularization operator is
square, or, in other words, only one plane-wave destructor is involved.
\par
Helical filtering \cite[]{GEO63-05-15321541} brings us new tools for
addressing this problem. In this paper, I show how to combine
orthogonal 2-D plane predictors into a single three-dimensional filter
with similar spectral properties. The 3-D filter can then work for
coherency measurements or for preconditioning 3-D inverse problems.
The construction employs the Wilson-Burg method of spectral
factorization, adapted for multidimensional filtering with the help of
the helix transform \cite[]{Sava.sep.97.paul1}.
\par
I use simple synthetic examples to demonstrate the applicability of 
plane-wave prediction to 3-D problems.

\section{Factorizing plane waves}
\inputdir{eplane}

Let us denote the coordinates of a three-dimensional space by $t$,
$x$, and $y$.  A theoretical plane wave is described by the equation
\begin{equation}
\label{eqn:pln}
U(t,x,y) = f (t - p_x x - p_y y)\;,
\end{equation}
where $f$ is an arbitrary function, and $p_x$ and $p_y$ are the plane
slopes in the corresponding direction. It is easy to verify that a
plane wave of the form (\ref{eqn:pln}) satisfies the following system
of partial differential equations:
\begin{equation}
\label{eqn:pde}
\left\{\begin{array}{rcl}\displaystyle
\left(\frac{\partial}{\partial x} + p_x\,\frac{\partial}{\partial t}\right)\,U
& = & 0 \\ \displaystyle
\left(\frac{\partial}{\partial y} + p_y\,\frac{\partial}{\partial t}\right)\,U
& = & 0
\end{array}\right.
\end{equation}
\par
The first equation in (\ref{eqn:pde}) describes plane waves on the
$\{t,x\}$ slices.  In is discrete form, it is represented as a
convolution with the two-dimensional finite-difference filter
$\mathbf{A}_x$. Similarly, the second equation transforms into a
convolution with filter $\mathbf{A}_y$, which acts on the $\{t,y\}$
slices. The discrete (finite-difference) form of equations
(\ref{eqn:pde}) involves a blocked convolution operator:
\begin{equation}
\label{eqn:cnv}
\left(\begin{array}{c}\displaystyle
\mathbf{A}_x \\
\mathbf{A}_y
\end{array}\right)\,\mathbf{U} = \mathbf{0}\;.
\end{equation}
where $\mathbf{U}$ is the model vector corresponding to $U(t,x,y)$.
\par
In many applications, we are actually interested in the spectrum of
the prediction filter, which approximates the inverse spectrum of the
predicted data. In other words, we deal with the square operator
\begin{equation}
\label{eqn:ata}
\left(\begin{array}{cc}\displaystyle \mathbf{A}_x^T & \mathbf{A}_y^T
      \end{array}\right)\,
\left(\begin{array}{c}\displaystyle
\mathbf{A}_x \\
\mathbf{A}_y
\end{array}\right) = 
\mathbf{A}_x^T \mathbf{A}_x + \mathbf{A}_y^T \mathbf{A}_y\;.
\end{equation}
If we were able to transform this operator to the form
$\mathbf{A}^T \mathbf{A}$, where $\mathbf{A}$ is a three-dimensional
minimum-phase convolution, we could use the three-dimensional filter
$\mathbf{A}$ in place of the inconvenient pair of $\mathbf{A}_x$ and
$\mathbf{A}_y$.
\par
The problem of finding $\mathbf{A}$ from its spectrum is known as
spectral factorization. It is well understood for 1-D signals
\cite[]{Claerbout.fgdp.76}, but until recently it was an open problem
in the multidimensional case. Helix transform maps multidimensional
filters to 1-D by applying special boundary conditions and allows us
to use the full arsenal of 1-D methods, including spectral
factorization, on multidimensional problems
\cite[]{GEO63-05-15321541}. A problem, analogous to (\ref{eqn:ata}),
has already occurred in the factorization of the discrete
two-dimensional Laplacian operator:
\begin{equation}
\label{eqn:lap}
\Delta = \nabla^T \nabla = 
\left(\begin{array}{cc}\displaystyle \mathbf{D}_x^T & \mathbf{D}_y^T
      \end{array}\right)\,
\left(\begin{array}{c}\displaystyle
\mathbf{D}_x \\
\mathbf{D}_y
\end{array}\right) = \mathbf{H}^T \mathbf{H}\;,
\end{equation}
where $\mathbf{D}_x$ and $\mathbf{D}_y$ represent the partial derivative
operators along the $x$ and $y$ dimensions, respectively, and the
two-dimensional filter $\mathbf{H}$ has been named \emph{helix
  derivative} \cite[]{gee,Zhao.sep.100.yi1}.
\par
If we represent the filter $\mathbf{A}_x$ with the help of a simple first-order 
upwind finite-difference scheme 
\begin{equation}
\label{eqn:fin}
\mathbf{U}_{x+1}^t - \mathbf{U}_{x}^t + p_x \left(\mathbf{U}_{x+1}^{t+1} - \mathbf{U}_{x+1}^{t}\right) = 0\;,
\end{equation}
then, after the helical mapping to 1-D, it becomes a one-dimensional
filter with the $Z$-transform
\begin{equation}
\label{eqn:hlx}
\mathbf{A}_x (Z) = 1 - p_x Z^{N_t + 1} + (p_x - 1) Z^{N_t}\;,
\end{equation}
where $N_t$ is the number of samples on the $t$-axis. Similarly, the
filter $\mathbf{A}_y$ takes the form
\begin{equation}
\label{eqn:hly}
\mathbf{A}_y (Z) = 1 - p_y Z^{N_t N_x + 1} + (p_y - 1) Z^{N_t N_x}\;.
\end{equation}
The problem is reduced to a 1-D spectral factorization of
\begin{eqnarray}
\nonumber 
& \mathbf{A}_x (1/Z) \mathbf{A}_x (Z) + \mathbf{A}_y (1/Z) \mathbf{A}_y (Z) = 
- p_y \frac{1}{Z^{N_t N_x + 1}} + (p_y - 1) \frac{1}{Z^{N_t N_x}} -
& \\
\nonumber 
& p_x \frac{1}{Z^{N_t + 1}} + (p_x - 1) \frac{1}{Z^{N_t-1}}  + 
\left[p_x (1 - p_x) + p_y (1 - p_y)\right] \frac{1}{Z} +
& \\
\nonumber
& 2 + 
p_x (p_x - 1) + p_y (p_y - 1) + 
\left[p_x (1 - p_x) + p_y (1 - p_y)\right] Z + & \\
&  (p_x - 1) Z^{N_t-1}
- p_x Z^{N_t + 1} + (p_y - 1) Z^{N_t N_x}
- p_y Z^{N_t N_x + 1}\;. &
\label{eqn:spc}
\end{eqnarray}
After a minimum-phase factor of (\ref{eqn:spc}) has been found, we can
use it for 3-D forward and inverse convolution.
\par
All examples in this paper actually use a slightly more sophisticated
formula for 2-D plane-wave predictors:
\begin{equation}
\label{eqn:lax}
\mathbf{A}_x (Z) = 1  + \frac{p_x}{2} (1 - p_x) Z^{N_t - 1} + (p_x^2 - 1) Z^{N_t}
- \frac{p_x}{2} (1 + p_x) Z^{N_t + 1}\;.
\end{equation}
Formula (\ref{eqn:lax}) corresponds to the Lax-Wendroff finite
difference scheme \cite[]{Clapp.sep.95.bob1}.  It provides a valid
approximation of the plane-wave differential equation for 
$-1 \le p_x \le 1$.
\par
\plot{eplane}{width=6in}{3-D plane wave prediction with a 402-point
  filter. Left: $p_x=0.7$, $p_y=0.5$. Right: $p_x=-0.7$, $p_y=0.5$.}

\inputdir{XFig}	
\sideplot{shape}{height=1.5in}{Schematic filter shape for a 26-point 3-D
  plane prediction filter. The dark block represents the leading
  coefficient. There are 9 blocks in the first row and 17 blocks in
  the second row.}

\inputdir{eplane}
\plot{tplane}{width=6in}{3-D plane wave prediction with a 26-point
  filter. Left: $p_x=0.7$, $p_y=0.5$. Right: $p_x=-0.7$, $p_y=0.5$.}
 
Figure~\ref{fig:eplane} shows examples of plane-wave construction. The
two plots in the figure are outputs of a spike, divided recursively
(on a helix) by $\mathbf{A}^{T} \mathbf{A}$, where $\mathbf{A}$ is a 3-D
minimum-phase filter, obtained by Wilson-Burg factorization. The
factorization was carried out in the assumption of $N_t=20$ and
$N_x=20$; therefore, the filter had $N_t N_x +2 = 402$ coefficients.
Using such a long filter may be too expensive for practical purposes.
Fortunately, the Wilson-Burg method allows us to specify the filter
length and shape beforehand. By experimenting with different filter
shapes, I found that a reasonable accuracy can be achieved with a
26-point filter, depicted in Figure~\ref{fig:shape}. Plane-wave
construction for a shortened filter is shown in
Figure~\ref{fig:tplane}. The predicted plane wave is shorter and looks
more like a slanted disk. It is advantageous to deal with short plane
waves if the filter is applied for local prediction of non-stationary
signals.

\inputdir{bob}

\cite{Clapp.sep.105.bob2} has proposed constructing 3-D plane-wave
destruction (steering) filters by splitting. In Clapp's method, the
two orthogonal 2-D filters $\mathbf{A}_x$ and $\mathbf{A}_y$ are
simply convolved with each other instead of forming the
autocorrelation~(\ref{eqn:ata}). While being a much more efficient
approach, splitting suffers from induced anisotropy in the inverse
impulse response. Figure~\ref{fig:bob} illustrates this effect in the
2-D plane by comparing the inverse impulse responses of plane-wave
filters obtained by spectral factorization and splitting.  The
splitting response is evidently much less isotropic.

\plot{bob}{width=4in,height=2in}{Two-dimensional inverse impulse
  responses for filters constructed with spectral factorization (left)
  and splitting (right). The splitting response is evidently much less
  isotropic.}

\par
In the next sections, I address the problem of estimating plane-wave
slopes and show some examples of applying local plane-wave prediction
in 3-D problems.

\section{Estimating plane waves}

It may seem difficult to estimate the plane slope $p_x$ for a
Lax-Wendroff filter of the form (\ref{eqn:lax}) because $p_x$ appears
non-linearly in the filter coefficients. However, using the analytical
form of the filter, we can easily linearize it with respect to the
plane slope and set up a simple iterative scheme: 
\begin{equation}
  \label{eqn:itr}
  p_x^{(k+1)} = p_x^{(k)} + \Delta  p_x^{(k)}\;,
\end{equation}
where $k$ stands for the iteration count, and  $\Delta  p_x^{(k)}$ is found
from the linearized equation
\begin{equation}
  \label{eqn:lin}
\left(\mathbf{A'}_x \mathbf{U}\right) \Delta  p_x = - 
\mathbf{A}_x \mathbf{U}\;,
\end{equation}
where $\mathbf{A'}_x$ is the derivative of $\mathbf{A}_x$ with respect to
$p_x$. To avoid unstable division by zero when solving equation
(\ref{eqn:lin}) for $\Delta p_x$, Adding a regularization equation
\begin{equation}
  \label{eqn:reg}
\epsilon \nabla \Delta  p_x \approx 0\;,
\end{equation}
where $\epsilon$ is a small scalar regularization parameter, I solve
system (\ref{eqn:lin}-\ref{eqn:reg}) in the least-square sense to
obtain a smooth slope variation $\Delta p_x$ at each iteration.  In
practice, iteration process (\ref{eqn:itr}) quickly converges to a
stable estimate of $p_x$.
\section{Examples}
Two simple examples in this section demonstrate an application of 3-D
local plane-wave prediction to the problems of discontinuity
enhancement and missing data interpolation.

\subsection{3-D discontinuity enhancement}
\inputdir{cube}

\sideplot{cube}{height=2.5in}{A synthetic model, showing a fault
  between two plane waves of different slopes.}
\plot{cslope}{width=6in}{Plane wave slope estimates in the $x$ and $y$
  directions (left and right plots, respectively) from the synthetic
  two-plane model.}
\par
Figure~\ref{fig:cube} shows a simple synthetic model of two plane
waves, separated by a plane fault. The slope estimates for the two
orthogonal directions are shown in Figure~\ref{fig:cslope}. We can see
that the estimation procedure correctly identified the regions of
constant slope. Finally, estimating a local 3-D plane-wave predictor by
spectral factorization and convolving the resultant non-stationary
filter with the input model, we obtain the prediction residual, shown
in Figure~\ref{fig:cmain}. In the residual, both plane waves are
effectively destroyed, and we observe a sharp image of the fault
plane. This result compares favorably with results of alternative
methods, collected by \cite[]{Schwab.sepphd.99}.
\sideplot{cmain}{height=2.5in}{Magnitude of the residual after
  convolving the synthetic two-plane model with a local 3-D plane wave
  filter.}
\subsection{3-D missing data interpolation}
\inputdir{qint}

\sideplot{qdome}{height=2.5in}{Claerbout's ``qdome'' synthetic model.}
\plot{qslope}{width=6in}{Plane wave slope estimates in the $x$ and $y$
  directions (left and right plots, respectively) from the ``qdome''
  model.}  

Figures~\ref{fig:qdome} and~\ref{fig:qslope} show Claerbout's
``qdome'' synthetic model \cite[]{Claerbout.sep.77.19,gee} and its
corresponding slope estimates. In a missing data interpolation
experiment, I remove 75\% of the traces in the original model,
arriving at the missing data model, shown in the left plot of
Figure~\ref{fig:qmiss}. The missing data interpolation result is shown
in the right plot of Figure~\ref{fig:qmiss}. Most of the original
signal, except for some high-curvature areas, has been restored. Local
3-D plane-wave predictors allow us to use the efficient interpolation
technique of \cite{Fomel.sep.95.sergey1}, based on recursive
filter preconditioning.

\plot{qmiss}{width=6in}{Left: ``qdome'' model with 75\% of the
  randomly chosen traces removed. Right: result of missing data
  interpolation with a 3-D local plane-wave prediction filter.}

\section{Conclusions}
I have shown that a 3-D plane-wave prediction filter can be
constructed from a pair of two-dimensional filters by using helix
transform and a one-dimensional spectral factorization algorithm.
\par
In all the examples, I used analytical finite-difference filters
instead of more general prediction-error filters.  A similar
factorization idea could be applied to 3-D prediction-error filters.
However, treating non-stationarity in this case is less
straightforward and requires additional care
\cite[]{Crawley.sep.97.sean2,Clapp.sep.100.bob3}.
\par
3-D plane-wave prediction filters can find many interesting
applications in data processing and inversion.  An especially
promising application is solution steering in tomography-type
problems \cite[]{Clapp.sep.95.bob1,Clapp.sep.97.bob2}.


\section{Acknowledgments}
Jon Claerbout suggested the problem of 3-D plane wave prediction and
the idea of its solution.  Unfortunately, he is currently not able to
share the authorship.

\bibliographystyle{seg}
\bibliography{SEG,SEP2,plane}

