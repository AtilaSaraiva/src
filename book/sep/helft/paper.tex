\def\figdir{./Fig} 
\lefthead{Rickett \& Guitton}
\righthead{Helical Fourier transforms}
\footer{SEP--105}
%\keywords{helix, linear filtering, FFT}
\published{SEP Report, 105, 167-176 (2000)}
\title{Multi-dimensional Fourier transforms in the helical coordinate 
system}  

\email{james@sep.stanford.edu, antoine@sep.stanford.edu}
\author{James Rickett and Antoine Guitton}

\maketitle

\begin{abstract}
For every two-dimensional system with helical boundary
conditions, there is an isomorphic one-dimensional system.
Therefore, the one-dimensional FFT of a 2-D function wrapped on a
helix is equivalent to a 2-D FFT.
We show that the Fourier dual of helical boundary conditions is
helical boundary conditions but with axes transposed, and we
explicitly link the wavenumber vector, ${\bf k}$, in a
multi-dimensional system with the wavenumber of a helical 1-D FFT,
$k_h$.   
We illustrated the concepts with an example of multi-dimensional
multiple prediction.
\end{abstract}  

\section{Introduction}
\inputdir{XFig}

If helical boundary conditions \cite[]{geofizhelix} are imposed on a
multi-dimensional system, an isomorphism exists between that system
and an equivalent one-dimensional system.  
Previous authors, for example \cite{gee}, take advantage of
this isomorphism to perform rapid multi-dimensional inverse filtering
by recursion. 

\par
The Fourier analogue of convolution is multiplication: to convolve a
2-D signal with a 2-D filter, take their 2-D Fourier transforms,
multiply them together and return to the original domain.
The relationship between 1-D and 2-D convolution, FFT's and the helix
is illustrated in Figure~\ref{fig:ill}.
With helical boundary conditions, we can take advantage of the
isomorphism described above, and perform multi-dimensional
convolutions by wrapping multi-dimensional signals and filters onto a
helix, taking their 1-D FFT's, multiplying them together, and then
returning to the original domain.

\par
If we can use 1-D FFT's to do 2-D convolutions, the isomorphism due to
the helical boundary conditions must extend into the Fourier domain.  
In this paper, we explore the relationship between 1-D and
multi-dimensional FFT's in helical coordinate systems.  Specifically
we demonstrate the link between the wavenumber vector, ${\bf k}$, in a 
multi-dimensional system, and the wavenumber of a helical 1-D FFT,
$k_h$. 

\sideplot{ill}{width=3.5in}{Relationship between  1-D and 2-D
convolution, FFT's and the helical boundary conditions.}

\section{Theory}
For simplicity, throughout this section we refer to a two-dimensional 
sampled image, ${\bf b}$; however, the beauty of the helical
coordinate system is that everything can be trivially extended to an
arbitrary number of dimensions.  

\par
We employ two equivalent subscripting schemes for
referring to an element of the two-dimensional image, ${\bf b}$.  
Firstly, with two subscripts, $b_{p_x,p_y}$ refers to the element 
that lies $p_x$ increments along the $x$-axis, and $p_y$
increments along the $y$-axis. 
Ranges of $p_x$and $p_y$ are given by 
$0 \leq p_x < N_x$, and 
$0 \leq p_y < N_y$ respectively. 
Helical coordinates suggest an alternative subscripting scheme:
We can use a single subscript, $p_h=p_x + p_y N_x$, 
such that $b_{p_x,p_y} = b_{p_h}$ and the range of $p_h$ is given by 
$0 \leq p_h < N_x N_y$.
Moreover, if we impose helical boundary conditions, we can
treat ${\bf b}$ as a one-dimensional function of subscript $p_h$.

\subsection{Linking 1-D and 2-D FFT's}
Taking the one-dimensional $Z$ transform of ${\bf b}$ in the helical
coordinate system gives
\begin{equation} \label{eqn:slow1dft}
B(Z_h) = \sum_{p_h=0}^{N_x N_y -1} b_{p_h} Z_h^{p_h}.
\end{equation}
Here, $Z_h$ represents the unit delay operator in the sampled
(helical) coordinate system. 
The summation in equation~(\ref{eqn:slow1dft}) can be split into two
components, 
\begin{eqnarray}
B(Z_h) 
& = & \sum_{p_y=0}^{N_y-1} \; \sum_{p_x=0}^{N_x-1} b_{p_x,p_y} 
Z_h^{p_x+ p_y N_x} \\
& = & \sum_{p_y=0}^{N_y-1} \; \sum_{p_x=0}^{N_x-1} b_{p_x,p_y} 
\; Z_h^{p_x} \; Z_h^{N_x p_y}. \label{eqn:halfway}
\end{eqnarray}

\par
Ignoring boundary effects, a single unit delay in the helical
coordinate system is equivalent to a single unit delay on the
$x$-axis;  
similarly, but irrespective of boundary conditions, $N_x$ unit delays
in the helical coordinate system are equivalent to a single delay on
the $y$-axis.   
This leads to the following definitions of $Z_h$ and
$Z_h^{N_x}$ in terms of delay operators, $Z_x$ and $Z_y$, or
wavenumbers, $k_x$ and $k_y$:
\begin{eqnarray}
Z_h    & \approx &  Z_x \; = \; e^{i k_x \Delta x}, 
\label{eqn:zxdef} \\ 
Z_h^{N_x} & = &  Z_y \; = \; e^{i k_y \Delta y}, \label{eqn:zydef}
\end{eqnarray}
where $\Delta x$ and $\Delta y$ define the grid-spacings along the $x$
and $y$-axis respectively.

\par
Substituting equations~(\ref{eqn:zxdef}) and~(\ref{eqn:zydef}) into
equation~(\ref{eqn:halfway}) leaves 
\begin{eqnarray}
B(k_x,k_y) \; = \; B(Z_h) 
& = & \sum_{p_y=0}^{N_y-1} \; \sum_{p_x=0}^{N_x-1}
b_{p_x,p_y} \; Z_x^{p_x} \; Z_y^{p_y} \\
& = & \sum_{p_y=0}^{N_y-1} \; \sum_{p_x=0}^{N_x-1}
b_{p_x,p_y} \; e^{i k_x \Delta x p_x} \; 
e^{i k_y \Delta y p_y}. \label{eqn:slow2dfft}
\end{eqnarray}
Equation~(\ref{eqn:slow2dfft}) implies that, if we ignore boundary 
effects, the one-dimensional FFT of ${\bf b}(x,y)$ in helical
coordinates is equivalent to its two-dimensional Fourier transform. 

\subsection{Wavenumber in helical coordinates}
With the understanding that the 1-D FFT of a multi-dimensional signal
in helical coordinates is equivalent to the 2-D FFT, a natural
question to ask is: how does the helical wavenumber, $k_h$, relate to
spatial wavenumbers, $k_x$ and $k_y$?

\par
The helical delay operator, $Z_h$, is related to $k_h$ 
through the equation,
\begin{equation}
Z_h = e^{i k_h \Delta x}.
\end{equation}
In the discrete frequency domain this becomes
\begin{equation}
Z_h = e^{i q_h \Delta k_h \Delta x},
\end{equation}
where $q_h$ is the integer frequency index that lies in the range, 
$0 \leq q_h < N_x N_y$. 
The uncertainty relationship, 
$\Delta k_h \Delta x = \frac{2 \pi}{N_x N_y}$, allows this to be
simplified still further, leaving
\begin{equation} \label{eqn:zhdef}
Z_h = e^{2 \pi i \; \frac{q_h}{N_x N_y}}.
\end{equation}
If we find a form of $q_h$ in terms of Fourier indices,
$q_x$ and $q_y$, that can be plugged into equation~(\ref{eqn:zhdef})
in order to satisfy equations~(\ref{eqn:zxdef})
and~(\ref{eqn:zydef}), this will provide the link between $k_h$ and
spatial wavenumbers, $k_x$ and $k_y$.

\par
The idea that $x$-axis wavenumbers will have a higher frequency than
$y$-axis wavenumbers, leads us to try a $q_h$ of the form,
\begin{equation} \label{eqn:qanswer}
q_h = N_y q_x + q_y.
\end{equation}
Substituting this into equation~(\ref{eqn:zhdef}) leads to
\begin{eqnarray}
Z_h & = & e^{2 \pi i \; \frac{(N_y q_x + q_y)}{N_x N_y}} \\
& = & e^{2 \pi i \; \left( \frac{q_x}{N_x} + \frac{q_y}{N_x N_y}
\right)}. 
\end{eqnarray}
Since $q_y$ is bounded by $N_y$, for large $N_x$ the second term in
braces $\frac{q_y}{N_x N_y} \approx 0$, and this 
reduces to 
\begin{equation}
Z_h \approx e^{2 \pi i \frac{q_x}{N_x}} = Z_x,
\end{equation}
which satisfies equation~(\ref{eqn:zxdef}).

\par
Substituting equation~(\ref{eqn:qanswer}) into
equation~(\ref{eqn:zhdef}), and raising it to the power of $N_x$ leads 
to: 
\begin{eqnarray}
Z_h^{N_x} & = & e^{2 \pi i \frac{(N_y q_x + q_y)}{N_y}} \\
& = & e^{2 \pi i \; \left( q_x + \frac{q_y}{N_y} \right)}.
\end{eqnarray}
Since $q_x$ is an integer, $e^{2 \pi i q_x} = 1$, and this reduces to  
\begin{equation}
Z_h^{N_x} = e^{2 \pi i \frac{q_y}{N_y}} = Z_y,
\end{equation}
which satisfies equation~(\ref{eqn:zydef}).

\par
Equation~(\ref{eqn:qanswer}), therefore, provides the link we are
looking for between $q_x$, $q_y$, and $q_h$.  It is interesting to
note that not only is there a one-to-one mapping between 1-D and 2-D 
Fourier components, but equation~(\ref{eqn:qanswer}) describes helical
boundaries in Fourier space: however, rather than wrapping around the
$x$-axis as it does in physical space, the helix wraps around the
$k_y$-axis in Fourier space (Figure~\ref{fig:transp}).  This provides 
the link that is missing in Figure~\ref{fig:ill}, but shown in
Figure~\ref{fig:ill2}. 
\sideplot{transp}{width=3.5in}{Fourier dual of helical boundary
conditions is also helical boundary conditions with axis of helix
transposed.} 
\sideplot{ill2}{width=3.5in}{Relationship between 1-D and 2-D
convolution, FFT's and the helix, illustrating the Fourier dual of
helical boundary conditions.}

\par
As with helical coordinates in physical space, 
equation~(\ref{eqn:qanswer}) can easily be inverted to yield
\begin{eqnarray}
k_x & = & \Delta k_x \, q_x = \frac{2 \pi}{N_x \Delta x} 
\left[ \frac{q_h}{N_y}\right], \hspace{0.25in} {\rm and} \\
k_y & = & \Delta k_y \, q_y = \frac{2 \pi}{N_y \Delta y}
\left(
q_h - N_y \left[ \frac{q_h}{N_y}\right]
\right)
\end{eqnarray}
where $[x]$ denotes the integer part of $x$. 

\subsection{Speed comparison}
For a two-dimensional dataset with dimensions, $N_x \times N_y$, the
cost of a 1-D FFT in helical coordinates is proportional to
\begin{equation}
N_x N_y \log \left( N_x N_y \right).
\end{equation}
For the same dataset, the cost of a 2-D FFT is 
\begin{eqnarray}
N_y \left( N_x \log N_x \right) + N_x \left( N_y \log N_y
\right)
& = & N_x N_y \; \left( \log N_x + \log N_y \right)  \nonumber \\
& = & N_x N_y \; \log \left( N_x N_y \right).
\end{eqnarray}
Therefore, the cost of a 1-D helical FFT of a 2-D dataset 
is exactly the same as the cost of an 2-D FFT of the same dataset.
The link between the two leads to no computational advantages in the
number of operations.

However, other differences may lead to computational savings.  For
example, a 2-D FFT with a power-of-two algorithm requires both $N_x$
and $N_y$ to be powers of two. However, the 1-D helical FFT requires
just $N_x N_y$ to be a power of two, and so less zero-padding may be
required.  

The corollary, that a large 1-D FFT 
can be computed (with small inaccuracies) using a 2-D FFT algorithm, 
also leads to potential computational savings.
Two-dimensional FFT's are easier to code to run both in parallel and
out-of-core than 1-D FFT's, leading to significantly faster code and a
lower memory requirement without the additional complexity of
Singleton's algorithm \cite[]{numrec}.

\section{Examples}
\inputdir{spike}

Figure~\ref{fig:spikespec} compares the real part of the 2-D  Fourier
transform of a single spike with the equivalent real part after a 1-D
FFT in helical boundary conditions.  The Fourier transforms are
centered, so that zero frequency is at the center of the plot.  This
has the effect that the artifacts that would appear at the vertical
boundaries ($k_y=0$) of the image are more visible since they appear
at the center of the plot.

\par
Figure~\ref{fig:schlumspec} compares amplitude spectra for a broader 
band 2-D seismic VSP gather.  Artifacts from the helical boundaries
are very difficult to see on the spectra themselves, and the
difference image is very low amplitude.
\plot{spikespec}{width=\textwidth}{Comparison of real part of 2-D  
spectra: (a) input spike (single frequency), (b) real part of
2-D FFT, (c) real part of 1-D helical FFT, and (d)
difference between (b) and (c) clipped to same level.}
\plot{schlumspec}{width=\textwidth}{Comparison of 2-D amplitude
spectra: (a) input 2-D VSP gather, (b) amplitude spectrum from
2-D FFT, (c) amplitude spectrum from 1-D helical FFT, and (d)
difference between (b) and (c) clipped to same level.}

\subsection{Application to the multiple prediction}
\inputdir{mult}
Multiple prediction is the first step in the class of adaptive
multiple suppression methods \cite[]{GEO57-09-11661177}.
In a laterally homogeneous earth, \cite{kelamis:00} show that 
surface-related multiples can be predicted by taking the
multi-dimensional auto-convolution of a common midpoint (CMP) gather.
This auto-convolution reduces to a multiplication in the {\em f-k}
domain, and so it can be performed rapidly with multi-dimensional
FFT's. 

\par
Since multi-dimensional FFT's can be computed with a one-dimensional
Fourier transform in helical coordinates, we can predict multiples by 
wrapping a CMP gather onto a helix, taking its 1-D FFT, squaring the 
result, and returning to the original domain.

\par
We tested this algorithm on a single CMP from the synthetic BP
multiple dataset \cite[]{BPmultiples}.
Figure~\ref{fig:BP2} displays the multiple prediction result using the
helical coordinate system and only a single one-dimensional FFT. 
Theoretically, only first-order multiples should have correct
relative amplitudes, and the source wavelet appears twice in the
multiple prediction.  However, the kinematics of all 
multiples are almost exact, even for higher-order multiples below 5~s 
two-way traveltime.
\plot{BP2}{width=\textwidth}{The left panel shows the multiple model
obtained with the helix and a 1-D FFT. The right panel shows the input
CMP gather with the offset axis reversed to facilitate the comparison.
Some wrap-around effects appear at the top of the multiple model.} 

\section{Conclusion}
We have explicitly found the relationship between multi-dimensional
FFT's and 1-D FFT's on a helix, linking the wavenumber vector, ${\bf
k}$, in a multi-dimensional system with the wavenumber 
of a helical 1-D FFT, $k_h$.  
Specifically, the Fourier dual of helical boundary conditions is
helical boundary conditions but with axes transposed.
We have illustrated the concepts with an example of multi-dimensional
multiple prediction.

\bibliographystyle{seg}
\bibliography{tvdecon,ref}
