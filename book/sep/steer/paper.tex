% Started 07/26/97

\lefthead{Clapp, et al.}
\righthead{Steering filters}
\footer{SEP--95}
\published{SEP report, 95, 27-42 (1997)}
\title{Solution steering with space-variant filters}

\email{bob@sep.stanford.edu, sergey@sep.stanford.edu, jon@sep.stanford.edu}
%\keywords{ interpolation, preconditioning, regularization, helix, least-squares}

\author{Robert G. Clapp, Sergey Fomel, and Jon Claerbout}

\maketitle

\begin{abstract}
Most geophysical problem require some type of regularization.
Unfortunately most regularization schemes produce  ``smeared'' results 
that are often undesirable when applying other criteria (such as geologic
feasibility).
By forming regularization operators in terms of 
recursive steering filters, built from a priori information sources,
we can efficiently guide the solution towards
a more appealing form. The steering
methodology proves effective in interpolating
low frequency functions, such as velocity, 
but performs poorly when encountering multiple
dips and high frequency data.  Preliminary results using steering filters for
regularization in tomography problems are encouraging. 
\end{abstract}

\section{INTRODUCTION}
%background
When attempting to do inversion we are constantly confronted with the
problem of slow convergence. \cite{Claerbout.sep.82.183}
suggested using a preconditioner
to speed up convergence.  Unfortunately it is often difficult to find an
appropriate preconditioner and/or the preconditioner is so 
computationally expensive that
it negates the savings gained by reducing
the  number of iterations \cite[]{Claerbout.tdf.82}.
\cite{Claerbout.sep.95.jon1} proposed
designing helicon-style operators
to  provide  a method to find stable inverses, 
and potentially, appropriate preconditioners
\cite[]{Fomel.sep.95.sergey1,Fomel.sep.94.sergey1}.
\par
In addition, geophysical problems are often under-determined,
requiring some type of regularization.  Unfortunately the simplest,
and most common, regularization techniques tend to create isotropic
features when we would often prefer solutions that follow trends.
This problem is especially prevalent in velocity estimation.  The
result obtained through many inversion schemes produce a velocity
structure that geologists (whose insights are hard to encode into the
regression equations) find unreasonable \cite[]{Etgen}.  Fortunately,
there are often other sources of information that can be encoded into
the regularization operator that allow the inversion to be guided
towards a more appealing result.  For example, in the case of velocity
estimation, reflector dips might be appropriate.

%claim

We create small, space-variant, steering filters from dip or other a priori 
information.
We use the inverse of these filters to form a preconditioner which acts
as our regularization operator.
We show this methodology applied to three different types of problems. In the 
first set of examples we interpolate 
well-log information using reflector dip as the basis for our steering  filters.
For the second set of examples we do a more traditional 
seismic data interpolation problem.  Starting
from a shot gather with a portion of the data missing.  We use a velocity
function to create hyperbolic paths, which in turn are used to construct 
steering filters.
In the final example we show some preliminary
results of using steering filters in conjunction with 
a tomographic  operator to create velocity models which both satisfy
the data and are geologically reasonable.

\section{THEORY/MOTIVATION}
\subsection{Regularization}
\inputdir{XFig}

In general, geophysical problems are ill-posed.
To obtain  pleasing results we impose some type of regularization
criteria such as
diagonal scaling, limiting solutions to large 
singular values \cite[]{Clapp.sep.84.75},  or
minimizing different solution norms \cite[]{Nichols.sep.82.1}.
The 
typical  SEP approach is to minimize the power out of a regularization
operator ($\mathbf A$) applied to the model ($\mathbf m$),
described by the fitting goal
\begin{equation}
0 \approx \mathbf A  \mathbf m .
\label{eq:first}
\end{equation}
Where $\mathbf A$'s spectrum will be the inverse of $\mathbf m$,
so to produce a smooth $\mathbf m$, we need a rough 
$\mathbf A$\cite[]{Claerbout.tdf.82}).
The regularization operator can take many forms, in order of increasing
complexity:
\begin{description}
        \item [Laplacian operator ($\nabla^2$)] The symmetric nature of the
Laplacian leads to isotropic smoothing of the image.
  \item [Steering filters] Simple plane wave annihilation filters which
   tend to orient the data in 
   some preferential direction,
   chosen a priori. These filters can be simple two point filters,
   Figure \ref{fig:steering}, to larger filters that sacrifice compactness for
   more precise dip annihilation.
\sideplot{steering}{height=1.0in}{An example of steering filter. In
this case  preference is given to slopes at 45 degrees.}

        \item  [Prediction Error Filters (PEF)] Like steering filters apply a 
   preferential smoothing direction, but are not limited to a single dip
    and determine their smoothing directions
         from the known data \cite[]{Schwab.sep.94.matt2}.
\end{description}


\subsection{Preconditioning}
Another important consideration is the speed of convergence of the problem.  
The size of most geophysical
problems make direct matrix inversion methods impractical. 
An appealing alternative for linear problems is the family of
conjugate gradient methods.  Unfortunately, 
the operators used in seismic reflection  problems
are often computationally expensive.
As a result it is important to minimize the number of steps it takes
to get to a reasonable solution.
One way that can reduce the number of iterations is by reformulating the 
problem in terms of some new variable ($\mathbf x$) with a preconditioning
operator ($\mathbf B$).  Changing a tradition inversion problem where the
operator ($\mathbf C$) maps the model ($\mathbf m$) to the data ($\mathbf d$),
\begin{equation}
\mathbf d \approx \mathbf C \mathbf m
\label{eq:precondition1}
\end{equation}
we can rewrite 
\begin{equation}
\mathbf d \approx \mathbf C \mathbf B \mathbf x
\end{equation}
where
\begin{equation}
\mathbf m = \mathbf B  \mathbf x.
\label{eq:precondition}
\end{equation}

\subsection{Helix transform}
The next question is  how to choose  $\mathbf B$?
We have three general requirements:
\begin{itemize}
\item it produces relatively smooth (by some criteria) results;
\item it spreads information quickly;
\item and it is computationally inexpensive.
\end{itemize}
By defining our operators via the helix method 
\cite[]{Claerbout.sep.95.jon1} we can meet all of these
requirements.
The helix concept is to transform {\it N}-Dimensional
operators into  1-D operators  to take
advantage of the well developed  1-D theory.
In this case we utilize our ability to construct 
stable inverses from simple, causal filters.  
 We can set $\mathbf B$,  from equation  (\ref{eq:precondition})  to
\begin{equation}
\mathbf B = \mathbf  A^{-1} ,
\label{eq:inverse}
\end{equation}
where $\mathbf A$ is the roughening operator from fitting
goal (\ref{eq:first}), and $\mathbf B$ is simulated using
polynomial division.
If $\mathbf A$ is a small roughening
operator, $\mathbf B$ is a large smoothing operator without the
heavy costs usually associated with larger operators.

\subsection{Steering Filters}
\inputdir{steering}

At this point a discussion of steering filters is appropriate.
Plane waves with a given slope on a discrete grid can be predicted
(destroyed) with compact filters \cite[]{Schwab.sep.94.matt2}. Inverting
such a filter by the helix method, we can create a signal with a given
arbitrary slope extremely quickly. If this slope is expected in the
model, the described procedure gives us a very efficient method of
preconditioning the model estimation problem, fitting goal (\ref{eq:precondition1}).
\par
How can a plane prediction (steering) filter be created? On the helix surface,
the plane wave $P(t,x) = f (t - p x)$ translates naturally into a
periodic signal with the period of $T = N_t + \sigma$, where $N_t$ is
the number of points on the $t$ trace, and $\sigma = \frac{p \triangle
  x}{\triangle t}$, where $\sigma$ is the plane slope, 
\footnote{In
  computational physics, the dimensionless number $\sigma$ is
  sometimes referred to as the CFL (Courant, Friedrichs, and Lewy)
  number \cite[]{sod}.} 
and $\triangle
x$ and $\triangle t$ correspond to the mesh size.
If we design a filter that is two columns long
(assuming the columns go in the $t$ direction), then the \emph{plane
  prediction} problem is simply connected with the
\emph{interpolation} problem: to destroy a plane wave, shift the
signal by $T$, interpolate it, and subtract the result from the
original signal. Therefore, we can formally write
\begin{equation}
  \label{eqn:pfilter}
  \mathbf{P} = I - \mathbf{S}(\sigma)\;,
\end{equation}
where $\mathbf{P}$ denotes the steering filter, $\mathbf{S}$ is
the shift-and-interpolation operator, and $\mathbf{I}$ is the identity
operator.
\par
Different choices for the operator $\mathbf{S}$ in (\ref{eqn:pfilter})
produce filters with different length and prediction power.
A shifting operation corresponds to the filter with the $Z$-transform
$\Sigma(Z) = Z^T$, while the operator $\mathbf{S}$ corresponds to an
approximation of $\Sigma(Z)$ with integer powers of $Z$. One possible
approach is to expand $\Sigma(Z) Z^{-N_t}$ using the Taylor series around
the zero frequency ($Z=1$). For example, the first-order approximation
is
\begin{equation}
  \label{eqn:plinear}
  S_1(Z) = Z^{N_t} \left(1 + \sigma (Z-1)\right) = (1-\sigma) Z^T +
  \sigma Z^{T+1}\;,
\end{equation}
which corresponds to linear interpolation and leads in the
two-dimensional space to the steering filter $\mathbf{P}$ of the
form
\begin{equation}
  \label{eqn:p1}
  \begin{array}{|c|c|}
    \hline
    1 & \\ \hline
    \sigma-1 & -\sigma \\ \hline
  \end{array}
\end{equation}
Filter (\ref{eqn:p1}) is equivalent to the explicit first-order upwind
finite-difference scheme on the plane wave equation
\begin{equation}
  \label{eqn:wave}
  \frac{\partial P}{\partial x} + p\,\frac{\partial P}{\partial t} = 0\;.
\end{equation}
An important property of filter (\ref{eqn:p1}) is that it produces an
exact answer for $\sigma=0$ and $\sigma=1$. The values of $\sigma > 1$
lead to unstable inversion. For negative $\sigma$, the filter is 
reflected:
\begin{equation}
  \label{eqn:p1p}
  \mathbf{P}_1 =
  \begin{array}{|c|c|}
    \hline
     & 1 \\ \hline
    \sigma & -\sigma - 1 \\ \hline
  \end{array}
\end{equation}
The top panel in Figure \ref{fig:steer-lagrange} shows a plane wave, created
by applying the helix inverse of filter (\ref{eqn:p1}) on a single
spike (unit impulse) for the value of $\sigma=0.7$. We see a
noticeable frequency dispersion, caused by the low order of the
approximation.

\plot{steer-lagrange}{width=6in,height=7.5in}{Steering filters with
  Lagrange interpolation. The left and middle plots show the impulse
  responses of steering filters: the top panel corresponds to
  linear interpolation (two-point Lagrange, upwind finite-difference);
  the second top plot, the three-point Lagrange filter (Lax-Wendroff
  scheme); the two bottom plots, the 8-point and 13-point Lagrange
  filters. The right plots in each panel show the corresponding
  average spectrum. The spectrum flattens and the prediction get more
  accurate with an increase of the filter size.}
\par
The second-order Taylor approximation yields
\begin{eqnarray}
  S_2(Z) & = & Z^{N_t-1} \left(1 + 
    \sigma (Z-1) {\frac{(\sigma -1) \,\sigma\,(Z -1)^2}{2}}\right) = 
\nonumber \\
  & & \frac{\sigma\,(\sigma-1)}{2}\, Z^{T-1} +
  \left(1-\sigma^2\right)\, Z^{T} +
  \frac{\sigma\,(\sigma+1)}{2}\, Z^{T+1}\;,   
  \label{eqn:pthree}
\end{eqnarray}
which corresponds to the 2-D filter
\begin{equation}
  \label{eqn:p2}
  \mathbf{P}_2 =
  \begin{array}{|c|c|c|}
    \hline
    & 1 & \\ \hline
    \frac{\sigma\,(1-\sigma)}{2} &  \left(\sigma^2-1\right) & -\frac{\sigma\,(\sigma+1)}{2} \\ \hline
  \end{array}
\end{equation}
and is equivalent to the Lax-Wendroff finite-difference scheme
of equation (\ref{eqn:wave}). The interpolation, implied by filter
(\ref{eqn:p1p}) is a local three-point polynomial (Lagrange)
interpolation. The correspondence of the Taylor series method,
described above, and the Lagrange interpolation can be proved by
induction. In general, the filter coefficients for the second row of
the $N$-th order Lagrangian filter are given by the explicit formula
\begin{equation}
  \label{eqn:coeff}
  a_{k} = \prod_{i \neq k} \frac{(\sigma-\left[\frac{N}{2}\right]-i)}{(k-i)}\;,
\end{equation}
where the $k$ and $i$ range from $0$ to $N$. Such a filter has a
stable inverse for $-\frac{N}{2} \le \sigma \le \frac{N+1}{2}$ and
additionally produces an exact answer for all integer $\sigma$'s in
that range. We would have arrived at the same conclusion if instead of
expanding the $Z$-transform of the filter $\mathbf{S}$ around $Z=1$,
expanded its Fourier transform around the zero frequency. The latter
case corresponds to the ``self-similar'' construction of
\cite{Karrenbach.sepphd.83}. The impulse responses for the helix
inverses of different-order Lagrangian filters are shown in Figure
\ref{fig:steer-lagrange}.
\par
If instead of Taylor series in $Z$, we use a rational (Pad\`{e})
approximation, the filter will get more than one coefficient in the
first row, which corresponds to an implicit finite-difference scheme.
For example, the $[1/1]$ Pad\`{e} approximation is
\begin{equation}
  \label{eqn:pade}
  S_1^1 (Z) = \frac
  {1 + \frac{1 + \sigma}{2}\,(Z-1)}
  {1 + \frac{1 - \sigma}{2}\,(Z-1)} = \frac
  {1-\sigma + (1+\sigma) Z}
  {1+\sigma + (1-\sigma) Z}\;
\end{equation}
which leads to the filter
\begin{equation}
  \label{eqn:p11}
  \mathbf{P}_1^1 =
  \begin{array}{|c|c|}
    \hline
    1 & \frac{1-\sigma}{1+\sigma} \\ \hline
    \frac{\sigma-1}{1+\sigma} & -1 \\ \hline
  \end{array}
\end{equation}
and corresponds to the Crank-Nicolson implicit scheme. 

%The impulse response for the inverse of filter (\ref{eqn:p11}) is
%shown in the top plot of Figure \ref{fig:steer-other}. It shows some
%mild improvements in comparison with the explicit Lagrangian filter of
%the same order. In our experience, filters with more than one
%additional coefficient in the first column behave unstably when
%inverted.

%\plot{steer-other}{width=6in,height=7.5in}{Steering filters with
%  different types of interpolation. The left and middle plots show the
%  impulse responses of the steering filters: the top panel
%  corresponds to first-order Pad\`{e} interpolation (Crank-Nicolson
%  scheme); the second top plot, the (8/2) Pad\`{e} approximation; the two
%  bottom plots, the 8-point and 12-point Lagrange filters. The right
%  plots in each panel show the corresponding average spectrum.}
%\par
%Other types of interpolations could be used for the steering 
%filters \cite[]{Fomel.sep.94.sergey2} The two bottom panels of Figure
%\ref{fig:steer-other} show the impulse responses for the filters, based on
%the tapered sinc interpolation. The filters suffer from high-frequency
%oscillations, but otherwise also perform well.
\par
It is interesting to note that a space-variant convolution with
inverse plane filters can create signals with different shape, which
remains planar only locally. This situation corresponds to a variable
slowness $p$ in the one-way wave equation (\ref{eqn:wave}). Figure
\ref{fig:steer-hyp7} shows an example: predicting hyperbolas with a 7-point
Lagrangian filter.

\plot{steer-hyp7}{width=4in,height=2in}{Creating hyperbolas with a variant
  plane-wave prediction: the impulse response of the inverse 7-point
  time-and-space-variant Lagrangian filter.}
 

\subsection{Space variable filters}
Steering filters are effective in spreading information
along a given direction, but are limited to a single dip.
If it is inappropriate to apply a single smoothing direction to
the entire
model 
there are two general
courses of action:
\begin{description}
\item [Patching] \cite[]{Claerbout.sep.73.391,Schwab.sep.84.271}
Redefine our problem into a series of problems,
each on a small subset of the data where the stationarity assumption is
valid, then recombine the data.  This approach leads to problems in determining
                 subsets where the stationarity condition is satisfied
    and how to effectively remove patching boundaries from the final output.
\item [Space varying filters] Filters that vary with location
but are spatially smooth. In many ways this is the a more appealing
approach.
In the past, space varying filters have not been used because they
impose significant memory issues (a filter at every location) and must
be spatially smooth.
By choosing steering filters for our regularization operator and using
helix enabled polynomial division, these 
weaknesses are significantly diminished.  
We can construct and store relatively
small filters which are much easier to smooth  (smoothing the preferential
dip direction is sufficient).
In addition the polynomial division produced inverse
filters will have an even higher level of smoothness because 
each filter spreads
information over large, overlapping regions at each iteration.
\end{description}

\section{WELL LOG/DIP INTERPOLATION}
\inputdir{qdome}

To illustrate the effectiveness of this method
imagine a simple interpolation problem.
Following the methodology of \cite[]{Fomel.sep.95.sergey1} we first bin the 
data, producing
a model $\mathbf m$, composed of known data $\mathbf m_k$ and unknown data 
$\mathbf m_u$.  We 
have an operator $\mathbf J$ which is simply a diagonal masking operator with 
zeros at known data locations and ones at unknown locations. We can write
$\mathbf m_k$ and $\mathbf m_u$ in terms of $\mathbf m$ and $\mathbf J$,
\begin{eqnarray}
\mathbf m_k &\approx& (\mathbf I\mathbf - \mathbf J)\mathbf m \\
\mathbf m_u &\approx& \mathbf J \mathbf m
\end{eqnarray}
where $\mathbf I$ is the identity matrix.  
We have the preconditioning operator $\mathbf B$, which applies polynomial
division using the helix methodology.  In this case we have a 
single equation in our estimation problem,
\begin{equation}
\mathbf m_k \approx (\mathbf I-\mathbf J) \mathbf B \mathbf x .
\label{eq:easy}
\end{equation}
So the only question that remains is what to use for $\mathbf B$, or 
more specifically $\mathbf B^{-1}$, $\mathbf A$.
\par
For this experiment  
we create a series of well logs by subsampling a 2-D velocity field.
We use as our a priori information source, reflector dips,
to build  our steering filters, and thus our operator $\mathbf A$. 
For this test we  pick our dips  from our ``goal'',
left portion of Figure \ref{fig:reflectors}.
We define areas in which we believe each of these dips to be
approximately correct, and  smooth the overall 
field (right portion of Figure \ref{fig:qdome-reflectors}).

\plot{reflectors}{width=5in,height=3.6in}{Left, a synthetic seismic section with 
four picked reflectors indicated by '*'; right; the dip field constructed from the picked reflectors.} 

%\activesideplot{qdome-dip-field}{width=3in}{}{Smoothed dip field.  Dips range from -5 
%degrees to 40 degrees.}
\par
For the first test, we simulate nine well logs along the survey
(Figure \ref{fig:qdome-combo1}). We use equation (\ref{eq:easy}) as our
fitting goal and  a conjugate gradient solver to estimate $\mathbf x$.
Within 12 iterations we  have a satisfactory 
solution(Figure \ref{fig:qdome-combo1}).
If you look closely, especially near the bottom of the section you
can still see the well locations, but in general the solution 
converges quickly to something 
fairly close to the correct velocity field (Figure \ref{fig:qdome-reflectors}). 


\plot{combo1}{height=3.4in,width=6.0in}{Left, correct velocity field; middle, well subset selected as input;
right, velocity field resulting from interpolation.}

\par
For a more difficult test, we decreased the number of wells, and give
them varying lengths.
In Figure \ref{fig:qdome-combo4} you see that in a few iterations
we  achieve a result quite similar
to our goal. In addition, in areas far away from known data the method
still followed the general dip direction simply at a lower frequency level.

\plot{combo4}{height=3.4in,width=6.0in}{Left model (our goal), middle well logs, and right estimated model
after 12 iterations.}


\section{SHOT-GATHER BASED INTERPOLATION}
\inputdir{shot}

Another possible application for using recursive
steering filters is to interpolate
seismic data.  As an initial test 
we chose to interpolate 
a shot gather.  We used a $v(z)$ velocity function to construct
hyperbolic trajectories, which in turn were used to
construct our dip
field (similar to the seismic dips used in the previous section).
\par
For a first test
we created a synthetic shot gather using a $v(z)=a+bz$ model as input
to a finite difference code.  We then cut
a hole in this shot gather  and attempted
to recover the removed values.
As Figure \ref{fig:combo} shows we did a good job 
recovering the amplitude within a few iterations.

\plot{combo}{width=6in,height=3.8in}{}{Left, synthetic shot gather; center,
holes cut out of shot gather; right, inversion result after 15 iterations.}
\par
To see how the method reacted when it was given data that did not
fit its model (in this case hyperbolic moveout) we used a dataset with
significant noise problems (ground roll, bad traces, etc.).  Using
the same technique as in Figure \ref{fig:combo} 
we ended up with a result which did a fairly
decent job fitting portions of the data where noise content was low,
but a poor job elsewhere (Figure \ref{fig:wz-combo}).  Even  where
the method did the best job of reconstructing the data, it still left
a visible footprint. A more esthetically pleasing result can be achieved
by using the above method followed a more traditional interpolation problem
using  the operator $\mathbf A$ and the fitting goal
\begin{equation}
\mathbf A \mathbf m \approx 0 ,
\label{eq:old}
\end{equation}
where  $\mathbf m$ is initialized with the result of our previous 
inversion problem and not allowed to change at locations where we have data.
The bottom right panel in Figure \ref{fig:wz-combo}  shows
the result of applying a few iterations of fitting goal (\ref{eq:old}) 
to the bottom left result in Figure \ref{fig:wz-combo}.
By using  both methodologies the interpolated data does a much better job 
blending into its
surroundings but still is a poor interpolation result.

\inputdir{wz25}

\plot{wz-combo}{width=6in,height=7in}{Top left, original shot
gather; top right, gather with holes (input); bottom left,
result applying equation \ref{eq:easy}, bottom right, result after
applying equation (\ref{eq:easy}) followed by (\ref{eq:old}).}

\begin{comment}

\section{REGULARIZATION}
The steering filter methodology has the most potential as a regularization 
operator in large inversion problems. For our final example
we use inverse steering-filters in conjunction with another
operator, in this case a tomography operator, to improve the 
inversion result.
For our tomography operator we chose Toldi's 
interval to stacking velocity operator\cite[]{Toldi.sep.43}. 
Generally, \cite{Toldi.sep.43} related perturbations in 
interval slowness
to perturbations in stacking slowness 
in  simple  slowness models.  
\par
We constructed a synthetic interval slowness perturbation model (Figure 
\ref{fig:toldi-steer},  left panel) where the perturbations from zero follow a 
sinusoidal path, and the anomalies go from positive to negative as you go
from left to right.  
We  used Toldi's forward operator to compute
stacking velocities at various depth levels (Figure \ref{fig:toldi-stack}, 
left panel), in this case we simulating collecting stacking velocity at 10
evenly spaced depths (compared to 160 depth locations in our interval
slowness model), assuming a cable length of $2~km$.

\activeplot{toldi-steer}{width=6in,height=3.8in}{}{Left, slowness  perturbation model;
center, inversion result using Laplacian smoother; right, inversion result
using steering filters.}

\activeplot{toldi-stack}{width=6in,height=3.8in}{}{Left, input stacking slowness;
right, calculated stacking slowness of steering filter inversion model.}
\par
We applied a fairly traditional inversion methodology to estimate 
our interval velocity perturbations:

\begin{eqnarray}
{\mathbf T} {\mathbf m} &\approx& {\mathbf d} \\
{\mathbf \epsilon} {\mathbf A} {\mathbf m} &\approx&  0 .
\end{eqnarray}
Where ${\mathbf T}$, is the Toldi operator; ${\mathbf A}$, is a Laplacian smoother; 
${\mathbf m}$, is our interval slowness perturbation model; and ${\mathbf d}$, is
our data, stacking slowness perturbations.
\par
The center panel of Figure \ref{fig:toldi-steer} shows the inversion result.  
We tried a variety of ${\mathbf \epsilon}$ values, selecting one that 
created a rough
model, but did a fairly good job recovering the correct interval velocity
perturbations.
Next, we attempted to recover the interval slowness perturbations, starting
from the same stacking slowness perturbations, using the
steering filter
methodology.  
We constructed our steering filters to follow the sinusoidal pattern of the 
model and changed our fitting goal to:
\begin{eqnarray}
{\mathbf A^{-1} x} &=& {\mathbf m} \\
{\mathbf T} {\mathbf A}^{-1} {\mathbf x} &\approx& {\mathbf d} 
\end{eqnarray}
where ${\mathbf A}$ is our steering filter matrix.  As the right panel of 
Figure \ref{fig:toldi-steer} shows we did a substantially better job 
following ``geology'', with the added benefit of better vertically
constraining  the interval slowness perturbations. 

\end{comment} 


\section{FUTURE WORK AND CONCLUSIONS}
We show that by using helicon enabled inverse 
operators built from small steering filters we can quickly obtain esthetically
pleasing models.
Tests on smooth models, with a single dip at each location proved 
successful.  The methodology does not adequately handle models with
multiple dips at each location and presupposes some knowledge of the
desired final model. 
A different approach would be to estimate
the steering filters ($\mathbf S$) from the experimental data ($\mathbf m$). 
Generally, this leads to a
system of non-linear equations
\begin{equation}
  \label{eqn:system}
  \mathbf{P}(\sigma) \mathbf{m} =
  (\mathbf{I} - \mathbf{S}(\sigma)) \mathbf{m} = \mathbf{0}\;,
\end{equation}
which need to be solved with respect to $\sigma$. One way of solving
system (\ref{eqn:system}) is to apply the general Newton's method,
which leads to the iteration
\begin{equation}
  \label{eqn:newton}
  \sigma_{k} = \sigma_{k-1} + \frac{\mathbf{P}(\sigma_{k-1})
    \mathbf{m}}{\mathbf{S}'(\sigma_{k-1}) \mathbf{m}}\;,
\end{equation}
where the derivative $\mathbf{S}'(\sigma)$ can be computed analytically.
It is interesting to note that if we start with $\sigma =0$ and apply
the first-order filter (\ref{eqn:p1}), then the first iteration of
scheme (\ref{eqn:newton}) will be exactly equivalent to the
slope-estimation method of \cite{Claerbout.blackwell.92}, used by
\cite{Bednar.sep.95.bednarb} for calculating coherency attributes.
Finally, the steering filter
regularization methodology needs to be tried in
conjunction with a variety of operators and 
applied to  real data problems.


\bibliographystyle{seg}
\bibliography{SEG,SEP2,bob}

%\newpage
%\FloatListing
%\APPENDIX{A}



%We used a Fortran90 inversion toolkit \cite[]{Fomel.sep.93.317} to estimate
%the model.  We first read in and binned the data:
%\begin{verbatim}
%call sreed( 'in', raw_data, 4*3*nd)
%call bin2_init (transpose(raw_data(:2,:)), o, d, n)
%call bin2 (raw_data(3,:), mm)
%call bin2_close ()
%\end{verbatim}
%
%
%then constructed the filters for $\mathbf P$  
%and the mask matrix $\mathbf J$ (equation \ref{eq:easy}).
%\begin{verbatim}
%call sreed('angle',angle_in,n123*4)
%do i1=1,n123
%    call small_filter(filt(i1),angle_in(i1),n(1))
%deallocate(angle_in);
%mask =  (data != 0.)
%\end{verbatim}
%
%Constructed and initialized the chain operator $(\mathbf I-\mathbf J) \mathbf F$:
%
%\begin{verbatim}
%call init_mis(model,data,mask,filt)
%\end{verbatim}      
%
%
%where {\tt init\_mis} is:
%\begin{verbatim}
%subroutine init_mis(model,data,mask,filt){
%real :: model(:),data(:)
%logical,pointer :: mask(:)
%type(filters),pointer :: filt(:)
%
%if(size(model)!=size(data))
%  call erexit("expecting model and data to be same size")
%
%call  init_helicon(filt,model)
%call  init_mask(mask,model)
%
%allocate(temp(size(model)))
%
%}
%\end{verbatim}
%
%and then initialized the conjugate gradient solver and passed in our
%chain operator:
%
%\begin{verbatim}
%call init_mis(model,data,mask,filt)
%model=0.
%call cgstep_init()
%call solver(mask_helix_chain, cgstep,niter=niter, dat=data, mask=mask, x=model)
%call cgstep_close()
%\end{verbatim}
%
%where or chain operator is simply:
%\begin{verbatim}
%integer function mask_helix_chain(adj,add,model,data){
%logical :: adj,add
%real :: model(:),data(:)
%
%call chain (mask_op, helidiv2,  adj, add, model, data, temp)
%}
%\end{verbatim}
%
%Finally, construct our output
%model, $\mathbf m_k + (\mathbf I - \mathbf J) \mathbf F \mathbf x$,
%and write out the result:
%
%\begin{verbatim}
%ierr= helidiv2(F,T,model,real_model)
%
%where(mask)
%  real_model=data
%
%call srite( 'out', real_model, 4*n123)
%\end{verbatim}
%
%
