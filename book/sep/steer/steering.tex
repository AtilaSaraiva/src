\inputdir{steering}

At this point a discussion of steering filters is appropriate.
Plane waves with a given slope on a discrete grid can be predicted
(destroyed) with compact filters \cite[]{Schwab.sep.94.matt2}. Inverting
such a filter by the helix method, we can create a signal with a given
arbitrary slope extremely quickly. If this slope is expected in the
model, the described procedure gives us a very efficient method of
preconditioning the model estimation problem, fitting goal (\ref{eq:precondition1}).
\par
How can a plane prediction (steering) filter be created? On the helix surface,
the plane wave $P(t,x) = f (t - p x)$ translates naturally into a
periodic signal with the period of $T = N_t + \sigma$, where $N_t$ is
the number of points on the $t$ trace, and $\sigma = \frac{p \triangle
  x}{\triangle t}$, where $\sigma$ is the plane slope, 
\footnote{In
  computational physics, the dimensionless number $\sigma$ is
  sometimes referred to as the CFL (Courant, Friedrichs, and Lewy)
  number \cite[]{sod}.} 
and $\triangle
x$ and $\triangle t$ correspond to the mesh size.
If we design a filter that is two columns long
(assuming the columns go in the $t$ direction), then the \emph{plane
  prediction} problem is simply connected with the
\emph{interpolation} problem: to destroy a plane wave, shift the
signal by $T$, interpolate it, and subtract the result from the
original signal. Therefore, we can formally write
\begin{equation}
  \label{eqn:pfilter}
  \mathbf{P} = I - \mathbf{S}(\sigma)\;,
\end{equation}
where $\mathbf{P}$ denotes the steering filter, $\mathbf{S}$ is
the shift-and-interpolation operator, and $\mathbf{I}$ is the identity
operator.
\par
Different choices for the operator $\mathbf{S}$ in (\ref{eqn:pfilter})
produce filters with different length and prediction power.
A shifting operation corresponds to the filter with the $Z$-transform
$\Sigma(Z) = Z^T$, while the operator $\mathbf{S}$ corresponds to an
approximation of $\Sigma(Z)$ with integer powers of $Z$. One possible
approach is to expand $\Sigma(Z) Z^{-N_t}$ using the Taylor series around
the zero frequency ($Z=1$). For example, the first-order approximation
is
\begin{equation}
  \label{eqn:plinear}
  S_1(Z) = Z^{N_t} \left(1 + \sigma (Z-1)\right) = (1-\sigma) Z^T +
  \sigma Z^{T+1}\;,
\end{equation}
which corresponds to linear interpolation and leads in the
two-dimensional space to the steering filter $\mathbf{P}$ of the
form
\begin{equation}
  \label{eqn:p1}
  \begin{array}{|c|c|}
    \hline
    1 & \\ \hline
    \sigma-1 & -\sigma \\ \hline
  \end{array}
\end{equation}
Filter (\ref{eqn:p1}) is equivalent to the explicit first-order upwind
finite-difference scheme on the plane wave equation
\begin{equation}
  {\frac{\partial P}{\partial x}} + p\,{\frac{\partial P}{\partial t}} = 0\;.
  \label{eqn:wave}
\end{equation}
An important property of filter (\ref{eqn:p1}) is that it produces an
exact answer for $\sigma=0$ and $\sigma=1$. The values of $\sigma > 1$
lead to unstable inversion. For negative $\sigma$, the filter is 
reflected:
\begin{equation}
  \label{eqn:p1p}
  \mathbf{P}_1 =
  \begin{array}{|c|c|}
    \hline
     & 1 \\ \hline
    \sigma & -\sigma - 1 \\ \hline
  \end{array}
\end{equation}
The top panel in Figure \ref{fig:steer-lagrange} shows a plane wave, created
by applying the helix inverse of filter (\ref{eqn:p1}) on a single
spike (unit impulse) for the value of $\sigma=0.7$. We see a
noticeable frequency dispersion, caused by the low order of the
approximation.

\plot{steer-lagrange}{width=6in,height=7.5in}{Steering filters with
  Lagrange interpolation. The left and middle plots show the impulse
  responses of steering filters: the top panel corresponds to
  linear interpolation (two-point Lagrange, upwind finite-difference);
  the second top plot, the three-point Lagrange filter (Lax-Wendroff
  scheme); the two bottom plots, the 8-point and 13-point Lagrange
  filters. The right plots in each panel show the corresponding
  average spectrum. The spectrum flattens and the prediction get more
  accurate with an increase of the filter size.}
\par
The second-order Taylor approximation yields
\begin{eqnarray}
  S_2(Z) & = & Z^{N_t-1} \left(1 + 
    \sigma (Z-1) {\frac{(\sigma -1) \,\sigma\,(Z -1)^2}{2}}\right) = 
\nonumber \\
  & & \frac{\sigma\,(\sigma-1)}{2}\, Z^{T-1} +
  \left(1-\sigma^2\right)\, Z^{T} +
  \frac{\sigma\,(\sigma+1)}{2}\, Z^{T+1}\;,   
  \label{eqn:pthree}
\end{eqnarray}
which corresponds to the 2-D filter
\begin{equation}
  \label{eqn:p2}
  \mathbf{P}_2 =
  \begin{array}{|c|c|c|}
    \hline
    & 1 & \\ \hline
    \frac{\sigma\,(1-\sigma)}{2} &  \left(\sigma^2-1\right) & -\frac{\sigma\,(\sigma+1)}{2} \\ \hline
  \end{array}
\end{equation}
and is equivalent to the Lax-Wendroff finite-difference scheme
of equation (\ref{eqn:wave}). The interpolation, implied by filter
(\ref{eqn:p1p}) is a local three-point polynomial (Lagrange)
interpolation. The correspondence of the Taylor series method,
described above, and the Lagrange interpolation can be proved by
induction. In general, the filter coefficients for the second row of
the $N$-th order Lagrangian filter are given by the explicit formula
\begin{equation}
  \label{eqn:coeff}
  a_{k} = \prod_{i \neq k} \frac{(\sigma-\left[\frac{N}{2}\right]-i)}{(k-i)}\;,
\end{equation}
where the $k$ and $i$ range from $0$ to $N$. Such a filter has a
stable inverse for $-\frac{N}{2} \le \sigma \le \frac{N+1}{2}$ and
additionally produces an exact answer for all integer $\sigma$'s in
that range. We would have arrived at the same conclusion if instead of
expanding the $Z$-transform of the filter $\mathbf{S}$ around $Z=1$,
expanded its Fourier transform around the zero frequency. The latter
case corresponds to the ``self-similar'' construction of
\cite{Karrenbach.sepphd.83}. The impulse responses for the helix
inverses of different-order Lagrangian filters are shown in Figure
\ref{fig:steer-lagrange}.
\par
If instead of Taylor series in $Z$, we use a rational (Pad\`{e})
approximation, the filter will get more than one coefficient in the
first row, which corresponds to an implicit finite-difference scheme.
For example, the $[1/1]$ Pad\`{e} approximation is
\begin{equation}
  \label{eqn:pade}
  S_1^1 (Z) = \frac
  {1 + \frac{1 + \sigma}{2}\,(Z-1)}
  {1 + \frac{1 - \sigma}{2}\,(Z-1)} = \frac
  {1-\sigma + (1+\sigma) Z}
  {1+\sigma + (1-\sigma) Z}\;
\end{equation}
which leads to the filter
\begin{equation}
  \label{eqn:p11}
  \mathbf{P}_1^1 =
  \begin{array}{|c|c|}
    \hline
    1 & \frac{1-\sigma}{1+\sigma} \\ \hline
    \frac{\sigma-1}{1+\sigma} & -1 \\ \hline
  \end{array}
\end{equation}
and corresponds to the Crank-Nicolson implicit scheme. 

%The impulse response for the inverse of filter (\ref{eqn:p11}) is
%shown in the top plot of Figure \ref{fig:steer-other}. It shows some
%mild improvements in comparison with the explicit Lagrangian filter of
%the same order. In our experience, filters with more than one
%additional coefficient in the first column behave unstably when
%inverted.

%\plot{steer-other}{width=6in,height=7.5in}{Steering filters with
%  different types of interpolation. The left and middle plots show the
%  impulse responses of the steering filters: the top panel
%  corresponds to first-order Pad\`{e} interpolation (Crank-Nicolson
%  scheme); the second top plot, the (8/2) Pad\`{e} approximation; the two
%  bottom plots, the 8-point and 12-point Lagrange filters. The right
%  plots in each panel show the corresponding average spectrum.}
%\par
%Other types of interpolations could be used for the steering 
%filters \cite[]{Fomel.sep.94.sergey2} The two bottom panels of Figure
%\ref{fig:steer-other} show the impulse responses for the filters, based on
%the tapered sinc interpolation. The filters suffer from high-frequency
%oscillations, but otherwise also perform well.
\par
It is interesting to note that a space-variant convolution with
inverse plane filters can create signals with different shape, which
remains planar only locally. This situation corresponds to a variable
slowness $p$ in the one-way wave equation (\ref{eqn:wave}). Figure
\ref{fig:steer-hyp7} shows an example: predicting hyperbolas with a 7-point
Lagrangian filter.

\plot{steer-hyp7}{width=4in,height=2in}{Creating hyperbolas with a variant
  plane-wave prediction: the impulse response of the inverse 7-point
  time-and-space-variant Lagrangian filter.}
