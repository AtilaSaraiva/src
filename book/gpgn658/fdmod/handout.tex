\author{Isaac Newton}
\title{GPGN658: Seismic migration}{FD modeling}

% ------------------------------------------------------------
In this homework, you will write code for time-domain acoustic
finite-differences modeling. This is one of the most basic programs
employed in seismic imaging. I used in class for this program the name
of ``Hello World!'' of seismic imaging. Everyone involved in seismic
imaging should write this program once in their life. This is what you
will do in this homework.

Your assignment is to modify an acoustic finite-differences modeling
program and compute wavefields and data recorded on the surface. You
will use the constant-density and the variable-density acoustic
wave-equations.

% ------------------------------------------------------------

\section{Prerequisites}

Log into your class account and prepare for the assignment:

\begin{enumerate}

\item Update the Madagascar library and 

\texttt{cd \$RSF/book/gpgn658/fdmod}.

You should have in your directory the
following files:

\texttt{fdmod/SConstruct} \\
\texttt{fdmod/handout.tex} \\
\texttt{fdmod/exercise/SConstruct} \\
\texttt{fdmod/exercise/AFDM.c} \\
\texttt{fdmod/exercise/fdutil.c} \\
\texttt{fdmod/exercise/fdutil.h} \\
\texttt{fdmod/exercise/omputil.c} \\
\texttt{fdmod/exercise/omputil.h}

\item 
Open the file \texttt{handout.tex} in your favorite text editor and
change the author name at the top of the file.

\item 
\texttt{cd exercise} \par
to change directory to the programming exercise.

\item 
Run \texttt{scons view} to build all targets described in the
\texttt{SConstruct} file. Watch for figures popping-up on your screen.
Close each figure by typing \texttt{q} with the cursor inside the
window.

You will notice that before execution, the program \texttt{AFDM.c} is
compiled and run to produce results. This is the program that you will
modify in this exercise. All compilation rules are pre-defined in the
\texttt{SConstruct} file and you do not need to modify those.

\item
Run \texttt{scons lock} to prepare your figures to be included in the
main document.

\item 
\texttt{cd gpgn658/fdmod} and run \texttt{scons read} to build and 
read your answer. The following figures are included in your document:
the source wavelet (Figure~\ref{fig:wava}), the velocity
(Figure~\ref{fig:vp}),the density (Figure~\ref{fig:ro}), the
wavefields at a fixed propagation time (Figure~\ref{fig:wa}), and the
data recorded at the surface (Figure~\ref{fig:da}).

% ------------------------------------------------------------
\inputdir{exercise}
% ------------------------------------------------------------
\sideplot{wava}{width=\textwidth}{Source wavelet.}
\sideplot{vp}{width=\textwidth}{Velocity model.}
\sideplot{ro}{width=\textwidth}{Density model.}
\sideplot{wa}{width=\textwidth}{Wavefield.}
\sideplot{da}{width=\textwidth}{Data.}

\end{enumerate}
% ------------------------------------------------------------
\section{Exercise}

\begin{enumerate}
\item 
\texttt{cd exercise} to begin your exercise.

\item 
The program \texttt{AFDM.c} implements time-domain finite-differences
modeling for the constant-density acoustic wave-equation. Your task is
to add the density term to this program. Refer to the course slides
for details about what needs to be added and where. Add comments in
the code to indicate your modifications.

\item 
Run \texttt{scons view} after your code is modified. All figures are
rebuilt with your new code and displayed on screen.

\item 
Run \texttt{scons lock} once you are satisfied with your results. All
figures are copied to the storage directory.

\item 
Add comments to this document indicating the changes to the simulated
data and wavefields. Are your results expected? Describe the data and
wavefield figures indicating what the various events represent.

\item 
\texttt{cd gpgn658/fdmod}, run \texttt{scons handout.read} to build 
your answer. A PDF file is constructed using your newly created
figures and modifications to the text. The modified code is
automatically added to the document.

\end{enumerate}

% ------------------------------------------------------------
\section{Wrap-up}

After you are satisfied that your document looks ok,
print it from the PDF viewer and bring it to class.

% ------------------------------------------------------------
\newpage
\section{AFDM.c}
\tiny
\lstinputlisting{exercise/AFDM.c}
\normalsize


