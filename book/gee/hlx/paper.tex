%\def\Xactiveplot#1#2#3#4{}
%\def\Xactivesideplot#1#2#3#4{}

\title{The helical coordinate}
\author{Jon Claerbout}
\maketitle
\label{paper:hlx}

For many years it has been true that
our most powerful signal-analysis techniques
are in {\em  one}-dimensional space,
while our most important applications are in {\em  multi}-dimensional space.
The helical coordinate system makes a giant step
towards overcoming this difficulty.

\par
Many geophysical map estimation applications appear to be multidimensional,
but actually they are not.
To see the tip of the iceberg, consider this example:
On a two-dimensional cartesian mesh, the function
$
\begin{array}{|r|r|r|r|}
        \hline
        0 & 0 & 0 &0 \\
        \hline
        0 & 1 & 1 &0 \\
        \hline
        0 & 1 & 1 &0 \\
        \hline
        0 & 0 & 0 &0 \\
        \hline
\end{array}
$
\par\noindent
has the autocorrelation
$
\begin{array}{|r|r|r|} \hline
        1 & 2 & 1\\
        \hline
        2 & 4 & 2\\
        \hline
        1 & 2 & 1
        \\ \hline
\end{array}
$.

\par\noindent
Likewise, on a one-dimensional cartesian mesh,
\par\noindent
the function $
%\mathcal{B} =
\begin{array}{|r|r|r|r|r|r|r|r|r|r|r|r|r|} \hline
 1&1&0&0& \cdots& 0& 1&1
        \\ \hline
\end{array}
$
\par\noindent
has the autocorrelation
$ %\mathcal{R} =
\begin{array}{|r|r|r|r|r|r|r|r|r|r|r|r|r|r|r|r|r|r|r|} \hline
 1&2&1&0&\cdots&0&2&4&2&0&\cdots&1&2&1
        \\ \hline
\end{array}
$.


%\par\noindent
%the function $ \bold b = ( 1,1,0,0, \cdots, 0, 1,1) $
%\par\noindent
%has the autocorrelation
%$ \bold r = ( 1,2,1,0,\cdots,0,2,4,2,0,\cdots,1,2,1)$.

\par\noindent
Observe the numbers in the one-dimensional world are identical
with the numbers in the two-dimensional world.
This correspondence is no accident.


\section{FILTERING ON A HELIX}
\inputdir{helicon}
Figure \ref{fig:diamond} shows some two-dimensional shapes
that are convolved together.
The left panel shows an impulse response function,
the center shows some impulses,
and the right shows the superposition of responses.
\plot{diamond}{width=\textwidth,height=.33\textwidth}{
  Two-dimensional convolution
  as performed in one dimension
  by module
  \texttt{helicon} %\vpageref{lst:helicon}
}

\par
A surprising, indeed amazing, fact is that
Figure \ref{fig:diamond} was not computed with a two-dimensional
convolution program.
It was computed with a one-dimensional computer program.
It could have been done with anybody's one-dimensional convolution program,
either in the time domain or in the fourier domain.
This magical trick is done with the helical coordinate system.

\par
A basic idea of filtering, be it in one dimension, two dimensions, or more,
is that you have some filter coefficients and some sampled data;
you pass the filter over the data; 
at each location you find an output by crossmultiplying
the filter coefficients times the underlying data and summing the terms.

\par
The helical coordinate system is much simpler than you might imagine.
Ordinarily, a plane of data is thought of as a collection of columns,
side by side.
Instead, imagine the columns stored end-to-end,
and then coiled around a cylinder.
This is the helix.
Fortran programmers will recognize that fortran's way
of storing 2-D arrays in one-dimensional memory
is exactly what we need for this helical mapping.
Seismologists sometimes use the word ``supertrace''
to describe a collection of seismograms stored ``end-to-end''.

\inputdir{.}

Figure \ref{fig:sergey-helix} shows a helical mesh for 2-D data on a cylinder.
Darkened squares depict a 2-D filter
shaped like the Laplacian operator $\partial_{xx}+\partial_{yy}$.
The input data, the filter, and the output data are all on helical meshes
all of which could be unrolled into linear strips.
A compact 2-D filter like a Laplacian,
on a helix is a sparse 1-D filter with long empty gaps.

\plot{sergey-helix}{width=\textwidth,bb=210 155 630 390}{
  Filtering on a helix.
  The same filter coefficients overlay the same data values
  if the 2-D coils are unwound into 1-D strips.
  (\emph{Mathematica} drawing by Sergey Fomel)
}

\par
Since the values output from filtering can be computed
in any order, we can slide the filter coil
over the data coil in any direction.
The order that you produce the outputs is irrelevant.
You could compute the results in parallel.
We could, however, slide the filter over the data in the screwing order
that a nut passes over a bolt.
The screw order is the same order that would be used
if we were to unwind the coils into one-dimensional strips
and convolve them across one another.
The same filter coefficients overlay the same data values
if the 2-D coils are unwound into 1-D strips.
The helix idea allows us to obtain the same convolution output
in either of two ways, a one-dimensional way, or a two-dimensional way.
I used the one-dimensional way to compute the obviously two-dimensional
result in Figure \ref{fig:diamond}.



\subsection{Review of 1-D recursive filters}
Convolution is the operation we do on polynomial coefficients
when we multiply polynomials.
Deconvolution is likewise for polynomial division.
Often these ideas are described
as polynomials in the variable $Z$.
Take $X(Z)$ to denote the polynomial
whose coefficients are samples of input data,
and let $A(Z)$ likewise denote the filter.
The convention I adopt here is that the first coefficient
of the filter has the value +1, so the filter's polynomial
is $A(Z) = 1 + a_1Z + a_2Z^2 + \cdots$.
To see how to convolve, we now identify the coefficient
of $Z^k$ in the product $Y(Z)=A(Z)X(Z)$.
The usual case ($k$ larger than the number $N_a$ of filter coefficients) is
\begin{equation}
y_k \quad=\quad x_k + \sum_{i=1}^{N_a} a_i x_{k-i}
\label{eqn:convolution}
\end{equation}
Convolution computes $y_k$ from $x_k$ whereas deconvolution
(also called back substitution) does the reverse.
Rearranging (\ref{eqn:convolution}) we get
\begin{equation}
x_k \quad=\quad y_k - \sum_{i=1}^{N_a} a_i x_{k-i}
\label{eqn:deconvolution}
\end{equation}
where now we are finding the output $x_k$ from
its past outputs $x_{k-i}$ and from the present input $y_k$.
We see that the deconvolution process is essentially
the same as the convolution process,
except that the filter coefficients
are used with opposite polarity;
and they are applied to the past {\em  outputs}
instead of the past {\em  inputs}.
That is why deconvolution must be done sequentially
while convolution can be done in parallel.



\subsection{Multidimensional deconvolution breakthrough}
Deconvolution (polynomial division)
can undo convolution (polynomial multiplication).
A magical property of the helix is that we can consider
1-D convolution to be the same as 2-D convolution.
Hence is a second magical property:
We can use 1-D
 {\em  de}convolution to undo convolution,
whether that convolution was 1-D or 2-D.
Thus, we have discovered how to undo 2-D convolution.
We have discovered that 2-D deconvolution on a helix
is equivalent to        1-D deconvolution.
The helix enables us to do multidimensional deconvolution.
\par
Deconvolution is recursive filtering.
Recursive filter outputs cannot be computed in parallel,
but must be computed sequentially
as in one dimension, namely,
in the order that the nut
screws on the bolt.  
\par
Recursive filtering sometimes solves big problems with astonishing speed.
It can propagate energy rapidly for long distances.
Unfortunately, recursive filtering can also be unstable.
The most interesting case, near resonance, is also near instability.
There is a large literature and extensive technology
about recursive filtering in one dimension.
The helix allows us to apply that technology to two (and more) dimensions.
It is a huge technological breakthrough.
\par
In 3-D we simply append one plane after
another (like a 3-D fortran array).
It is easier to code than to explain or visualize
a spool or torus wrapped with string, etc.



\subsection{Examples of simple 2-D recursive filters}
Let us associate $x$- and $y$-derivatives with
a finite-difference stencil or template.
(For simplicity take $\Delta x=\Delta y=1$.)
\begin{equation}
 \frac{\partial }{ \partial x } \eq
 \begin{array}{|c|c|} \hline
   1    & -1
   \\ \hline
 \end{array}
 \label{eqn:partialx}
\end{equation}
\begin{equation}
 \frac{\partial }{ \partial y } \eq
  \begin{array}{|r|} \hline
    1  \\
    \hline
    -1
    \\ \hline
  \end{array}
  \label{eqn:partialy}
\end{equation}
Convolving a data plane with
the stencil (\ref{eqn:partialx})
forms the $x$-derivative of the plane.
Convolving a data plane with
the stencil (\ref{eqn:partialy})
forms the $y$-derivative of the plane.
On the other hand,
{\it deconvolving}
with (\ref{eqn:partialx}) integrates data along the $x$-axis for each $y$.
Likewise, deconvolving
with (\ref{eqn:partialy}) integrates data along the $y$-axis for each $x$.
Next we look at a fully two-dimensional operator
(like the cross derivative $\partial_{xy}$).

\inputdir{helicon}

\par
A nontrivial two-dimensional convolution stencil is
\par
\begin{equation}
        \begin{array}{|r|r|}   \hline
                0    & -1/4 \\
                \hline
                1    & -1/4 \\
                \hline
                -1/4 & -1/4
                \\ \hline
        \end{array}
\label{eqn:filterfour}
\end{equation}
We will convolve and deconvolve a data plane with this operator.
Although everything is shown on a plane,
the actual computations are done in one dimension
with equations
(\ref{eqn:convolution}) and
(\ref{eqn:deconvolution}).
Let us manufacture the simple data plane
shown on the left in Figure \ref{fig:wrap-four}.
Beginning with a zero-valued plane, we add
in a copy of the filter (\ref{eqn:filterfour})
near the top of the frame.
Nearby add another copy with opposite polarity.
Finally add some impulses near the bottom boundary.
The second frame in Figure \ref{fig:wrap-four} is the result
of deconvolution by the filter (\ref{eqn:filterfour})
using the one-dimensional equation (\ref{eqn:deconvolution}).
Notice that deconvolution
turns the filter itself into an impulse,
while it turns the impulses
into comet-like images.
The use of a helix is evident
by the comet images wrapping around the vertical axis.

\plot{wrap-four}{width=.65\textwidth,height=.33\textwidth} { 
  Illustration of 2-D deconvolution.
  Left is the input.
  Right is after deconvolution with
  the filter (\protect\ref{eqn:filterfour})
  as preformed by
  by module
  \texttt{polydiv} %\vpageref{lst:polydiv}
}

The filtering in Figure \ref{fig:wrap-four}
ran along a helix from left to right.
Figure \ref{fig:back-four}
shows a second filtering running from right to left.
Filtering in the reverse direction is the adjoint.
After deconvolving both ways, we have accomplished a symmetrical smoothing.
The final frame undoes the smoothing to bring us exactly back
to where we started.
The smoothing was done with two passes of {\it deconvolution}
and it is undone by two passes of {\it convolution}.
No errors, no evidence remains of any of the boundaries
where we have wrapped and truncated.

\plot{back-four}{width=.65\textwidth,height=.33\textwidth}{ 
  Recursive filtering backwards (leftward on the space axis)
  is done by the {\em  adjoint} of 2-D deconvolution.
  Here we see that 2-D {\it deconvolution} compounded with its adjoint
  is exactly inverted by 2-D {\it convolution} and its adjoint.
}

%\begin{notforlecture}
\par
Chapter \ref{paper:prc} explains the important practical role
to be played by a multidimensional operator for which
we know the exact inverse.  
Other than multidimensional Fourier transformation,
transforms based on polynomial multiplication and division
on a helix are the only known easily invertible linear operators.

%\par
%I found myself with a powerful convolution/deconvolution
%program, but I did not have a bag full of 2-D filters.
%Further, I knew that a great number of filters are unstable
%for deconvolution.
%I recalled a variant of Rouch$\acute{\rm e}$'s theorem
%that if the sum of the absolute values of the filter coefficients
%after the onset pulse is less than the pulse,
%then the filter is ``positive real,''
%hence ``minimum phase,'' hence stable in polynomial division.
%Ironically, it is the ``almost unstable'' filters
%that hold the greatest interest,
%because they are the ones that move energy long distances.
%To make long impulse-responses of deconvolution,
%I chose all the adjustable filter coefficients to be negative
%and to sum to $-1$,
%as does (\ref{eqn:filterfour}).
%For damping, I took the absolute value of the sum to be
%a little less than $1$.

\par
In seismology we often have occasion to steer summation along beams.
Such an impulse response is shown in Figure \ref{fig:dip}.
\plot{wrap-waves}{width=.8\textwidth,height=.4\textwidth}{ 
  A simple low-order 2-D filter whose inverse
  contains plane waves of two different dips.
  One of them is spatially aliased.
}

Of special interest are filters that destroy plane waves.  The inverse
of such a filter creates plane waves.  Such filters are like wave
equations.  A filter that creates two plane waves is illustrated in
figure \ref{fig:wrap-waves}.

\plot{dip}{width=\textwidth,height=.3\textwidth}{ 
  A simple low-order 2-D filter whose inverse times its inverse adjoint,
  is approximately a dipping seismic arrival.
}

\subsection{Coding multidimensional de/convolution}
Let us unroll the filter helix seen in Figure \ref{fig:sergey-helix}
and see what we have.
Start from the idea that a 2-D filter is generally made
from a cluster of values near one another in two dimensions
similar to the Laplacian operator in the figure.
We see that in the helical approach,
a 2-D filter is a 1-D filter containing some long intervals of zeros.
The intervals are about the length of a 1-D seismogram.

\par
Our program for 2-D convolution with a 1-D convolution program,
could convolve with the somewhat long 1-D strip,
but it is much more cost effective to ignore the many zeros,
which is what we do.
We do not multiply by the backside zeros, nor do we even store them in memory.
Whereas an ordinary convolution program would do time shifting
by a code line like {\tt iy=ix+lag},
Module
\texttt{helicon} %\vpageref{lst:helicon}
ignores the many zero filter values on backside of the tube
by using the code {\tt iy=ix+lag[ia]}
where a counter {\tt ia} ranges over the nonzero filter coefficients.
Before operator {\tt helicon} is invoked,
we need to prepare two lists,
one list containing nonzero filter coefficients {\tt flt[ia]},
and the other list containing the corresponding lags {\tt lag[ia]}
measured to include multiple wraps around the helix.
For example, the 2-D Laplace operator
can be thought of as the 1-D filter
\begin{equation}
\label{eqn:2dlapfil}
\begin{array}{|r|r|r|r|r|r|r|r|r|r|r|r|r|r|r|} \hline
1&0&\cdots&0&1&-4&1&0& \cdots&0&1
\\ \hline
\end{array}
%
\quad
\rightarrow{\rm helical \; boundaries}
\quad
\begin{array}{|r|r|r|}  \hline
& 1 & \\
\hline
1 & -4 & 1\\
\hline
& 1 & \\
\hline
\end{array}
\end{equation}
%
%
%\begin{equation}
%\label{eqn:2dlapfil}
%\left[ \begin{array}{ccc}
%& 1 & \\ 1 & -4 & 1\\ & 1 &
%\end{array} \right]
%\xrightarrow{\rm helical \; boundary \; conditions}
%(1,\,0, \; ... \; 0,\,1,\,-4,\,1,\,0, \; ... \; 0,\,1).
%\end{equation}
%
The first filter coefficient in equation (\ref{eqn:2dlapfil})
is $+1$ as implicit to module {\tt helicon}.
To apply the Laplacian on a $1000\times 1000$ mesh
requires the filter inputs:
\par\noindent
\footnotesize
\begin{verbatim}
                i    lag[i]   flt[i]
               ---   ------   -----
                0      999      1
                1     1000     -4
                2     1001      1
                3     2000      1
\end{verbatim}
\normalsize

\par
Here we choose to use
``declaration of a type'',
a modern computer language feature that is absent from Fortran 77.
Fortran 77 has the built in complex arithmetic type.
In module \texttt{helix}
we define a type \texttt{filter}, actually, a helix filter.
After making this definition, it will be used by many programs.
The helix filter consists of three vectors,
a real valued vector of filter coefficients,
an integer valued vector of filter lags,
and an optional vector
that has logical values ``\texttt{true}''~for
output locations that will not be computed
(either because of boundary conditions or because of missing inputs).
The filter vectors are the size of the nonzero filter coefficients
(excluding the leading 1.) while the logical vector is long
and relates to the data size.
The \texttt{helix} module allocates and frees memory for a helix filter.
%By default, the logical vector is not allocated but
%is set to \texttt{null}
%with the \texttt{nullify} operator and ignored.
\moddex{helix}{definition for helix-type filters}{30}{36}{api/c}
\par
For those of you with no C  experience,
the ``\verb#->#'' appearing in the helix module denotes a pointer.
Fortran 77 has no pointers (or everything is a pointer).
%The C, C++, and Java languages use ``\texttt{.}'' to denote pointers.
%C and C++ also have a second type of pointer denoted by ``\texttt{->}''.
The behavior of pointers is somewhat different in each language.
Never-the-less, the idea is simple.
In module \texttt{helicon} %\vpageref{lst:helicon}
you see the expression
\verb#aa->flt[ia]#.
It refers to the filter named \texttt{aa}.
Any filter defined by the \texttt{helix} module
contains three vectors, one of which is named \texttt{flt}.
The second component of the \texttt{flt} vector
in the \texttt{aa} filter
is referred to as
\verb#aa->flt[1]# which
in the example above refers to the value 4.0
in the center of the laplacian operator.
For data sets like above with 1000 points on the 1-axis,
this value 4.0 occurs after 1000 lags,
thus \verb#aa->lag[1]=1000#.

\par
Our first convolution operator
\texttt{tcai1}
%\vpageref{lst:tcai1}
was limited to one dimension and a particular choice of end conditions.
With the helix and C pointers,
the operator
\texttt{helicon} %\vpageref{lst:helicon}
is a {\it multidimensional} filter
with considerable flexibility (because of the \texttt{mis} vector)
to work around boundaries and missing data.
\opdex{helicon}{helical convolution}{35}{54}{api/c}
The code fragment
\verb#aa->lag[ia]#
corresponds to 
\texttt{b-1}
in \texttt{tcai1}.
% \vpageref{lst:tcai1}.


\par
Operator {\tt helicon} did the convolution job for Figure \ref{fig:diamond}.
As with
\texttt{tcai1} 
%\vpageref{lst:tcai1}
the adjoint of filtering is filtering backwards
which means unscrewing the helix.

\par
The companion to convolution is deconvolution.
The module \texttt{polydiv} 
%\vpageref{lst:polydiv}
is essentially the same as
\texttt{polydiv1} 
%\vpageref{lst:polydiv1},
but here it was coded using
our new \texttt{filter} type in
module \texttt{helix} 
%\vpageref{lst:polydiv1}
which will simplify our many future uses of
convolution and deconvolution.
Although convolution allows us to work around missing input values,
deconvolution does not
(any input affects all subsequent outputs),
so \texttt{polydiv} never references \verb#aa->mis[ia]#.
\opdex{polydiv}{helical deconvolution}{39}{70}{api/c}



\section{KOLMOGOROFF SPECTRAL FACTORIZATION}
\sx{Kolmogoroff}
\sx{spectral factorization ! Kolmogoroff}
Spectral factorization addresses a deep mathematical problem not solved
by mathematicians until 1939.
Given any spectrum $|F(\omega)|$,
find a causal time function $f(t)$ with this spectrum.
A causal time function is one that vanishes at negative time $t<0$.
We will mix spectral factorization with the helix idea to find many applications
in geophysical image estimation.
\par
The most abstract method of spectral factorization is that of the Russian mathematician A.N.Kolmogoroff.
I include it here because it is by far the fastest,
so much so that giant problems become practical
such as the solar physics example coming up.
\par
Given that $C(\omega)$ fourier transforms to a causal function of time,
it is next proven that $e^C$ fourier transforms to a causal function of time.
Its filter inverse is $e^{-C}$.
Grab yourself a cup of coffee
and hide yourself away in a quiet place
while you focus on the proof in the next paragraph.
\par
A causal function $c_\tau$ vanishes at negative $\tau$.
Its $Z$ transform $C(Z) = c_0 + c_1 Z + c_2 Z^2 + c_3 Z^3 +\cdots$,
with $Z=e^{i\omega\Delta t}$ is really a Fourier sum.
Its square $C(Z)^2$
convolves a causal with itself so it is causal.
Each power of $C(Z)$ is causal, hence
$e^C=1+C+C^2/2+\cdots$, a sum of causals, is causal.
The time-domain coefficients for $e^C$ could be computed
putting polynomials into power series or faster by Fourier transforms.
The wavelet $e^C$ has inverse $e^{-C}$ also causal.
A causal with a causal inverse is said to be ``minimum phase''.
The filter $1-Z/2$ with inverse $1+Z/2+Z^2/4+\cdots$ is so.
The delay filter $Z^5$ has the noncausal inverse $Z^{-5}$ is not
(output before input).
\par
The next paragraph defines ``Kolmogoroff spectral factorization''.
This arises in applications where one begins with an energy spectrum $|r|^2$
and factors it into an $r e^{i\phi}$ times its conjugate.
The inverse fourier transform of that  $r e^{i\phi}$ is causal.
\par
Relate amplitude $r=r(\omega)$ and phase $\phi=\phi(\omega)$
to a causal time function $c_\tau$.

\par
%\LARGE
\begin{eqnarray}
|r|e^{i\phi} &=&
e^{\ln|r|}e^{i\phi} \ =\
e^{\ln\,|r| + i\phi} \ =\  
e^{c_0+c_1Z+c_2Z^2+c_3Z^3+\cdots} \ =\ 
e^{\sum_{\tau=0} c_\tau Z^\tau}
%\ =\ \exp \left( \sum_{\tau=0} c_\tau Z^\tau\right)
\end{eqnarray}
\par\noindent
%\normalsize
Given a spectrum $r(\omega)$ we will find a filter with that spectrum.
Since $r(\omega)$ is a real even function of $\omega$, so is its logarithm.
Let the inverse Fourier transform of $\ln |r(\omega)|$ be $u_\tau$,
where $u_\tau$ is a real even function of time.
Imagine a real odd function of time $v_\tau$.
\begin{eqnarray}
|r|e^{i\phi} &=& e^{\ln\,|r| + i\phi} \ =\  e^{\sum_\tau (u_\tau+v_\tau) Z^\tau}
\end{eqnarray}
The phase $\phi(\omega)$ transforms to $v_\tau$.
We can assert causality
by choosing $v_\tau$ so that $u_\tau+v_\tau=0$ $ (=c_\tau)$ for all negative $\tau$.
This defines $v_\tau$ at negative $\tau$.
Since $v_\tau$ is odd, it is known at positive lags too.
More simply,
$v_\tau$ is created when $u_\tau$ is multiplied by a step function of size 2.  
This causal exponent
$(c_0,c_1,\cdots)$
creates a causal filter $|r|e^{i\phi}$
with the specified spectrum $r(\omega)$.

\par
We easily manufacture an inverse filter by changing the polarity of the $c_\tau$.
This filter is also causal by the same reasoning.
Thus these filters are causal with a causal inverse.
Such filters are commonly called ``minimum phase''.

\par
Spectral factorization arises in a variety of contexts. Here's one:
Rain drops showering on a tin roof create for you a signal
whose spectrum you can compute,
but what would be the wavelet of a single drop?
Spectral factorization gives it.
Divide this wavelet out from the data
to get a record of impulses, one for each rain drop (theoretically!).
The boiling surface of the sun is coming soon.

\subsection{Kolmogoroff code}
\sx{Kolmogoroff code}
\footnotesize
\begin{verbatim}
subroutine kolmogoroff( n, cx)  # Spectral factorization.
integer              i, n         # input:  cx = amplitude spectrum
complex              cx(n)        # output: cx = FT of min phase wavelet
do i= 1, n                     
        cx(i) = clog( cx(i) )
call ftu( -1., n, cx)
do i= 2, n/2 {                   # Make it causal changing only the odd part.
        cx(i)     = cx(i) * 2.
        cx(n-i+2) = 0.
        }
call ftu( +1., n, cx)
do i= 1, n
        cx(i) = cexp( cx(i))
return; end
\end{verbatim}
\normalsize
\par\noindent
Everyone has their own favorite fourier transform code, so why am I offering mine?
\sx{fourier transform code}
Because you MUST get the scale factors correct.
Few worries if you accidentally replace $e^C$ by $2e^C$,
because your humble plotting program might do that.
But if you accidentally replace $e^C$ by $e^{2C}$,
you have squared it!
\par
\footnotesize
\begin{verbatim}
subroutine ftu( signi, nx, cx )	          # Fourier transform
#   complex fourier transform with traditional scaling (FGDP)
#
#               1         nx          signi*2*pi*i*(j-1)*(k-1)/nx
#   cx(k)  =  -------- * sum cx(j) * e
#              scale     j=1             for k=1,2,...,nx=2**integer
#
#  scale=1 for forward transform signi=1, otherwise scale=1/nx
integer nx, i, j, k, m, istep
real    signi, arg
complex cx(nx), cmplx, cw, cdel, ct
i=1;  while( i<nx) i=2*i
if( i != nx )    call erexit('ftu: nx not a power of 2')
do i= 1, nx
        if( signi<0.)
                cx(i) = cx(i) / nx
j = 1;  k = 1
do i= 1, nx {
        if (i<=j) { ct = cx(j); cx(j) = cx(i); cx(i) = ct }
        m = nx/2
        while (j>m && m>1) { j = j-m; m = m/2 }         # "&&" means .AND.
        j = j+m
        }
repeat {
        istep = 2*k;   cw = 1.;   arg = signi*3.14159265/k
        cdel = cmplx( cos(arg), sin(arg))
        do m= 1, k {
                do i= m, nx, istep
                        { ct=cw*cx(i+k);  cx(i+k)=cx(i)-ct;  cx(i)=cx(i)+ct }
                cw = cw * cdel
                }
        k = istep
        if(k>=nx) break
        }
return; end
\end{verbatim}
\normalsize
\par\noindent
The \texttt{ftu} fast Fourier transform code above
has a restriction that the data length must be a power of 2.
Zero time and frequency are the first point in the vector,
then positive times, then negative times.

\par
It is a exercise for the student to show that
a complex-valued time function has
a positive spectrum that is non-symmetrical in frequency
but it may be factored with the same code.

\subsection{Constant Q medium}
\sx{Q, constant Q medium}
From the absorption law of a material, spectral factorization yields its impulse response.
The most basic absorption law is the {\em constant Q} model.
According to it, for a downgoing wave
the absorption is proportional to the frequency $\omega$,
proportional to time in the medium $z/v$,
and inversely proportional to the ``quality'' $Q$ of the medium.
Altogether the spectrum of a wave passing through a thickness $z$ will be changed by the factor
$ e^{-|\omega|\tau} = e^{-|\omega|(z/v)/Q} $.
This frequency function is plotted in the top line of Figure \ref{fig:futterman}.
\sx{Futterman}
\plot{futterman}{width=\textwidth,height=.41\textwidth}{
        Autocorrelate the bottom signal to get the middle whose FT is the top.
        Spectral factorization works the other way, from top to bottom.
        }

%\end{notforlecture}
\par

The middle function in Figure \ref{fig:futterman}
is the autocorrelation giving on top
the spectrum
$e^{-|\omega|\tau}$.
The third function is the factorization.
An impulse entering the medium comes out with this shape.
There is no physics in this analysis,
only mathematics which assumes the broadened pulse
is causal with an abrupt arrival.
The short wavelengths are concentrated near the sharp corner
while the long wavelengths are spread throughout.
A physical system could cause the pulse to spread further
(effectively by an additional all-pass filter),
but physics cannot make it more compact.
\par
All distances from the source see the same shape, but stretched
in proportion to distance.
The apparent $Q$ is
the traveltime to the source divided by the width of the pulse.


\subsection{Causality in two dimensions}
\sx{filter ! causal in 2-D}
Our foundations,
the basic convolution-deconvolution pair
(\ref{eqn:convolution}) and
(\ref{eqn:deconvolution})
are applicable only to filters with all coefficients {\it after} zero lag.
Filters of physical interest generally concentrate their coefficients near zero lag.
Requiring causality in 1-D and concentration in 2-D leads to shapes such as these:
\begin{equation}
\label{eqn:2dpef}
\begin{array}{ccccc}
        \begin{array}{ccc}
                h  & c &   0 \\
                p  & d &   0 \\
                q  & e &  \bold 1    \\
                s  & f &   a \\
                u  & g &   b
        \end{array}
        &\quad=\quad&
        \begin{array}{ccc}
                h  & c &  \cdot   \\
                p  & d &  \cdot   \\
                q  & e &  \cdot    \\
                s  & f &   a \\
                u  & g &   b
        \end{array}
         &\quad +\quad&
        \begin{array}{ccc}
                \cdot &\cdot &   0 \\
                \cdot &\cdot &   0 \\
                \cdot &\cdot &  \bold 1    \\
                \cdot &\cdot &  \cdot    \\
                \cdot &\cdot &  \cdot  
        \end{array}
  \\
  \\
        {\rm 2-D \ filter}
           &=&
        {\rm variable}
          &\quad +\quad&
        {\rm constrained}
\end{array}
\end{equation}
where $a,b,c,...,u$ are coefficients we will find by least squares.
\par
The complete story is rich in mathematics and in concepts,
but to sum up, filters fall into two categories according to the
numerical values of their coefficients.
There are filters for which equations
(\ref{eqn:convolution}) and
(\ref{eqn:deconvolution})
work as desired and expected.
These filters are called ``minimum phase''.
There are also filters for which
(\ref{eqn:deconvolution}) is a disaster numerically,
the feedback process diverging to infinity.
\par
Divergent cases correspond to physical processes
that are not simply described by initial conditions
but require also reflective boundary conditions
so information flows backwards, i.e.~anticausally.
Equation (\ref{eqn:deconvolution})
only allows for initial conditions.
\par
I oversimplify by trying to collapse an entire book (FGDP)
into a few sentences by saying here that
for any fixed 1-D spectrum there exist many filters.
Of these, only one has
stable polynomial division.
That filter has its energy compacted
as soon as possible after the ``1.0'' at zero lag.

\par
Which side of the little rectangular patch of coefficients
we choose to place the 1.0 is rather arbitrary.
The important matter is that as a matter of principle,
the 1.0 is expected to lie along one side of the little patch.
We never put the ``1'' at the corner of a patch
because that would be excluding locations near the ``1'' that
could be correlated with it.
It is important that beyond the 1.0 (in whatever direction that may be)
the filter coefficients must be zero because in one dimension,
these coefficients lie before zero lag.


\subsection{Causality in three dimensions}
\sx{causality in 3-D}
\sx{filter ! causal in 3-D}
The top plane in Figure~\ref{fig:3dpef} 
is the 2-D filter seen in equation (\ref{eqn:2dpef}).
Geometrically, the three-dimensional generalization of a helix,
Figure~\ref{fig:3dpef} shows a causal filter in three dimensions.
Think of the little cubes as packed with the string of the causal 1-D function.
Under the ``1'' is packed with string, but none above it.
Behind the ``1'' is packed with string, but none in front of it.
The top plane can be visualized as the area around the end of the 1-D string.
Above the top plane are zero-valued anticausal filter coefficients.
\sideplot{3dpef}{width=.76\textwidth}{
        A 3-D causal filter at the starting end of a 3-D helix.
        \viewit{3dpef}
	}
This 3-D cube is like the usual Fortran packing of a 3-D array
with one confusing difference.   The starting location where the ``1''
is located is not at the Fortran (1,1,1) location.
Details of indexing are essential,
but complicated, and found near the end of this chapter.


\subsection{Blind deconvolution and the solar cube}
\sx{deconvolution ! solar cube}
\sx{solar cube}
An area of applications that leads directly to spectral factorization
is ``blind deconvolution.''
Here we begin with a signal.
We form its spectrum and factor it.
We could simply inspect the filter and interpret it,
or we might deconvolve it out from the original data.
This topic deserves a fuller exposition, say for example
as defined in some of my earlier books.
Here we inspect a novel example that incorporates the helix.
\par
Solar physicists have learned how to measure
the seismic field of the sun surface.  It's chaotic.
If you created an impulsive explosion on the surface of the sun,
what would the response be?
James Rickett and I applied the helix idea along with Kolmogoroff
spectral factorization to find the impulse response of the sun.
Figure \ref{fig:solar} shows a raw data cube and the derived impulse response.
The sun is huge so the distance scale is in megameters (Mm).
The United States is 5 Mm wide.
Vertical motion of the sun is measured with a video-camera like device
that measures vertical motion by an optical doppler shift.
From an acoustic/seismic point of view,
the surface of the sun is a very noisy place.
The figure shows time in kiloseconds (Ks).
We see about 15 cycles in 5 Ks which is 1 cycle in about 333 sec.
Thus the sun seems to oscillate vertically with about a 5 minute period.
The top plane of the raw data
in Figure \ref{fig:solar} (left panel)
happens to have a sun spot in the center.
The data analysis here is not affected by the sun spot so please ignore it.
\plot{solar}{width=\textwidth,height=.65\textwidth}{
	Raw seismic data on the sun (left).
	Impulse response of the sun (right)
	derived by Helix-Kolmogoroff spectral factorization.
	\viewit{solar}
	}
\par
The first step of the data processing is
to transform the raw data to its spectrum.
With the helix assumption, computing the spectrum is
virtually the same thing in 1-D space as in 3-D space.
The resulting spectrum was passed to Kolmogoroff spectral factorization code,
a 1-D code.
The resulting impulse response is on the right side of 
Figure \ref{fig:solar}.
The plane we see on the right top is not lag time $\tau=0$;
it is lag time $\tau=2$ Ks.
It shows circular rings, as ripples on a pond.
Later lag times (not shown) would be the larger circles of expanding waves.
The front and side planes show tent-like shapes.
\par
The slope of the tent gives the (inverse) velocity of the wave
(as seen on the surface of the sun).
The horizontal velocity we see on the sun surface turns out
(by Snell's law)
to be the same as that at the bottom of the ray.
On the front face at early times we see the low velocity (steep) wavefronts
and at later times we see the faster waves.
This is because the later arrivals reach more deeply into the sun.
\par
Look carefully, and you can see two (or even three!) tents inside one another.
These ``inside tents'' are the waves that have bounced once (or more!)
from the surface of the sun.
When a ray goes down and back up to the sun surface,
it reflects and takes off again with the same ray shape.
The result is that a given slope on the traveltime curve
can be found again at twice the distance at twice the time.
Very close to $t=0$ see horizontal waveforms
extending only a short distance from the origin.
These are electromagnetic waves of essentially infinite velocity.




\section{FACTORED LAPLACIAN == HELIX DERIVATIVE}
\sx{Laplacian, factored}
I had learned spectral factorization as a method for single seismograms.
After I learned it,
every time I saw a positive function
I'd wonder if it made sense to factor it.
When total field
magnetometers were invented,
I found it as a way to deduce vertical and horizontal
\bxbx{magnetic}
components.
A few pages back you saw how to use factorization
to deduce the waveform passing through an absorptive medium.
Then we saw how the notion of ``impulse response''
applies not only to signals,
but allows use of random noise on the sun
to deduce the 3-D impulse response there.
But the most useful application of spectral factorization so far
is what comes next, factoring the Laplace operator, $-\nabla^2$.
Its fourier transform $-((ik_x)^2+(ik_y)^2) \ge 0$ is positive, so it is a spectrum.
The useful tool we'll uncover I dub the ``helix derivative''.
\par
The signal
\begin{equation}
\label{eqn:2dlapin1d}
\bold r \eq
-\nabla^2
\eq
\begin{array}{|r|r|r|r|r|r|r|r|r|r|r|r|r|r|r|} \hline
-1&0&\cdots&0&-1&4&-1&0& \cdots&0&-1
\\ \hline
\end{array}
\end{equation}
is an autocorrelation function because
it is symmetrical about the ``4,''
and the Fourier transform of $-\nabla^2$ is $-((ik_x)^2+(ik_y)^2) \ge 0$
which is positive for all frequencies $(k_x,k_y)$.
Kolmogoroff spectral-factorization gives this wavelet $\bold h $:
\begin{equation}
\label{eqn:2dlapin1dhalf}
{\bold h}
\eq
\begin{array}{|r|r|r|r|r|r|r|r|r|r|r|r|r|r|r|}
\hline
1.791&-.651&-.044&-.024&\cdots&\cdots&-.044&-.087&-.200&-.558 \\
\hline
\end{array}
\label{eqn:lapfacrow}
\end{equation}
In other words,
the autocorrelation of
(\ref{eqn:2dlapin1dhalf}) is
(\ref{eqn:2dlapin1d}).
This is not obvious from the numbers themselves
because the computation requires a little work,
but dropping all the smaller numbers
allows you a rough check.
\par
In this book section only I use abnormal notation for bold letters.
Here
$\bold h$, $\bold r$ are signals,
while
$\bold H$ and $\bold R$ are images,
neither being matrices or vectors.
Recall from chapter \ref{paper:ajt} that a filter is a signal
packed into a matrix to make a filter operator.
\par
Let the time reversed version of
$\bold h$
be denoted
$\bold h\T$.
This notation is consistent with an idea from
Chapter \ref{paper:ajt} that the adjoint of a filter matrix
is another filter matrix with a reversed filter.
In engineering it is conventional to use the asterisk symbol
``$\ast$'' to denote convolution.
Thus, the idea that the autocorrelation of a 
signal $\bold h$
is a convolution of the
signal $\bold h$
with its time reverse (adjoint)
can be written as
$ {\bold h}\T \ast {\bold h} = {\bold h} \ast {\bold h}\T = {\bold r}$.
\par
Wind the signal $\bold r$ around a
vertical-axis helix to see its two-dimensional shape $\bold R$:
\begin{equation}
\bold r
\quad
\rightarrow{\rm helical \; boundaries}
\quad
\begin{array}{|r|r|r|}  \hline
& -1 & \\
\hline
-1 & 4 & -1\\
\hline
& -1 & \\
\hline
\end{array}
\eq
\bold R
\label{eqn:lap2d}
\end{equation}
This 2-D image (which can be packed into a filter operator)
is the negative of the finite-difference representation
of the Laplacian operator, generally denoted
$\nabla^2 = \frac{\partial^2}{ \partial x^2} + \frac{\partial^2}{ \partial y^2} $.
Now for the magic:
Wind the signal $\bold h$ around the same helix
to see its two-dimensional shape $\bold H$
\begin{equation}
\label{eqn:lapfac}
 {\bold H}
\eq
    \begin{array} {|r|r|r|r|r|r|r|r|r|} \hline
             &      &       &       & 1.791 &  -.651 & -.044  & -.024& \cdots \\
	     \hline 
      \cdots &-.044 & -.087 & -.200 & -.558 &        &        &      &
     \\ \hline
    \end{array}
\end{equation}
In the representation (\ref{eqn:lapfac}) we see the coefficients diminishing
rapidly away from maximum value 1.791.
My claim is that the two-dimensional autocorrelation of (\ref{eqn:lapfac})
is (\ref{eqn:lap2d}).
You verified this idea earlier when the numbers were all ones.
You can check it again in a few moments
if you drop the small values, say 0.2 and smaller.
\par
Physics on a helix can be viewed through the eyes
of matrices and numerical analysis.
This is not easy because the matrices are so huge.
Discretize the $(x,y)$-plane to an $N\times M$ array
and pack the array into a vector of $N\times M$ components.
Likewise pack the Laplacian operator $\partial_{xx}+\partial_{yy}$
into a matrix.
For a $4\times 3$ plane, that matrix is shown in equation (\ref{eqn:huge}).
\begin{equation}
\label{eqn:huge}
-\ \nabla^2 \eq
\left[
\begin{array}{rrrr|rrrr|rrrr}
  4 & -1 & \cdot & \cdot 
& -1 & \cdot & \cdot & \cdot 
& \cdot & \cdot & \cdot & \cdot  \\
  -1  &   4  &   -1     &  \cdot 
&  \cdot &-1 & \cdot  & \cdot 
& \cdot  &  \cdot  &  \cdot &  \cdot    \\
\cdot & -1  &   4  & -1  
&  \cdot  &  \cdot &-1  &  \cdot 
&  \cdot  &  \cdot  &  \cdot  &  \cdot   \\
\cdot &  \cdot & -1 &   4  
&  h  &  \cdot  &  \cdot &-1 
&  \cdot  &  \cdot  &  \cdot  &  \cdot  \\
\hline
  -1  &  \cdot  &  \cdot  &  h 
&   4   &  -1   &  \cdot  &  \cdot 
& -1  &  \cdot  &  \cdot &  \cdot 
\\
   \cdot &-1 & \cdot  &  \cdot 
& -1  &   4  &   -1     &  \cdot 
&  \cdot &-1 & \cdot  &  \cdot 
\\
  \cdot  &  \cdot &-1  &  \cdot 
&  \cdot &   -1     &   4   & -1  
& \cdot  &  \cdot &-1  &  \cdot 
\\
  \cdot  &  \cdot  &  \cdot &-1 
& \cdot  &  \cdot  & -1  &   4  
& h  &  \cdot  &  \cdot &-1 
\\
\hline
  \cdot  &  \cdot  &  \cdot  &  \cdot 
& -1  &  \cdot  &  \cdot  &  h 
&   4   &  -1   &  \cdot  &  \cdot 
\\
  \cdot  &  \cdot  &  \cdot  &  \cdot 
&  \cdot &-1 & \cdot  &  \cdot 
& -1  &   4  &   -1     &  \cdot 
\\
  \cdot  &  \cdot  &  \cdot  &  \cdot 
& \cdot  &  \cdot &-1  &  \cdot 
&  \cdot &   -1     &   4   & -1  
\\
  \cdot  &  \cdot  &  \cdot  &  \cdot 
& \cdot  &  \cdot  &  \cdot &-1 
& \cdot  &  \cdot  & -1  &   4  
\end{array}
\right]
\end{equation}
\par\noindent
The two-dimensional matrix of coefficients for the Laplacian operator
is shown in (\ref{eqn:huge}),
where, 
on a cartesian space, $h=0$,
and in the helix geometry, $h=-1$.
(A similar partitioned matrix arises from packing
a cylindrical surface into a $4\times3$ array.)
Notice that the partitioning becomes transparent for the helix, $h=-1$.
With the partitioning thus invisible, the matrix
simply represents one-dimensional convolution
and we have an alternative analytical approach,
one-dimensional Fourier Transform.
We often need to solve sets of simultaneous equations
with a matrix similar to (\ref{eqn:huge}).
The method we use is triangular factorization.
\par
Although the autocorrelation $\bold r$ has mostly zero values,
the factored autocorrelation $\bold a$ has a great number of nonzero terms.
Fortunately they seem to be converging rapidly (in the middle)
so truncation (of the middle coefficients) seems reasonable.
I wish I could show you a larger matrix, but all I can do is to pack
the signal $\bold a$ into shifted columns of
a lower triangular matrix $\bold A$ like this:
\begin{equation}
\bold A \ = \ \ 
\left[
\begin{array}{rrrrrrrrrrrr}
   1.8& \cdot& \cdot& \cdot& \cdot& \cdot& \cdot& \cdot& \cdot& \cdot& \cdot& \cdot\\
   -.6&   1.8& \cdot& \cdot& \cdot& \cdot& \cdot& \cdot& \cdot& \cdot& \cdot& \cdot\\
\ddots&   -.6&   1.8& \cdot& \cdot& \cdot& \cdot& \cdot& \cdot& \cdot& \cdot& \cdot\\
   -.2&\ddots&   -.6&   1.8& \cdot& \cdot& \cdot& \cdot& \cdot& \cdot& \cdot& \cdot\\
   -.6&   -.2&\ddots&   -.6&   1.8& \cdot& \cdot& \cdot& \cdot& \cdot& \cdot& \cdot\\
 \cdot&   -.6&   -.2&\ddots&   -.6&   1.8& \cdot& \cdot& \cdot& \cdot& \cdot& \cdot\\
 \cdot& \cdot&   -.6&   -.2&  \ddots& -.6&   1.8& \cdot& \cdot& \cdot& \cdot& \cdot\\
 \cdot& \cdot& \cdot&   -.6&  -.2& \ddots&   -.6&   1.8& \cdot& \cdot& \cdot& \cdot\\
 \cdot& \cdot& \cdot& \cdot&  -.6&    -.2&\ddots&   -.6&   1.8& \cdot& \cdot& \cdot\\
 \cdot& \cdot& \cdot& \cdot& \cdot&   -.6&   -.2&\ddots&   -.6&   1.8& \cdot& \cdot\\
 \cdot& \cdot& \cdot& \cdot& \cdot& \cdot&   -.6&   -.2&\ddots&   -.6&   1.8& \cdot\\
 \cdot& \cdot& \cdot& \cdot& \cdot& \cdot& \cdot&   -.6&   -.2&\ddots&   -.6&  1.8
\end{array}
\right]
\label{eqn:lapfacmat}
\end{equation}
If you will allow me some truncation approximations,
I now claim that the laplacian represented by the
matrix in equation (\ref{eqn:huge})
is factored into two parts
$-\nabla^2 =\bold A\T\bold A$
which are upper and lower triangular matrices
whose product forms the autocorrelation seen in (\ref{eqn:huge}).
Recall that triangular matrices
allow quick solutions of simultaneous equations by backsubstitution.
That is what we do with our
deconvolution program.
\par
Spectral factorization produces not merely a causal wavelet
with the required autocorrelation.
It produces one that is stable in deconvolution.
Using $\bold H$ in one-dimensional polynomial division,
we can solve many formerly difficult problems very rapidly.
Consider the Laplace equation with sources (Poisson's equation).
Polynomial division and its reverse (adjoint) gives us
$\bold p =(\bold q/\bold H)/\bold H\T$
which means that we have solved
$\nabla^2 \bold p = -\bold q$
by using polynomial division on a helix.
Using the seven coefficients shown,
the cost is fourteen multiplications
(because we need to run both ways) per mesh point.
An example is shown in Figure \ref{fig:lapfac}.
\plot{lapfac}{width=\textwidth,height=.33\textwidth}{
        Deconvolution by a filter whose autocorrelation
        is the two-dimensional Laplacian operator.
        Amounts to solving the Poisson equation.
        Left is $\bold q$;
        Middle is $\bold q/\bold H$;
        Right is $(\bold q/\bold H)/\bold H\T$.
	}
\par
Figure \ref{fig:lapfac90} contains both the helix derivative and its inverse.
Contrast them to the $x$- or $y$-derivatives (doublets) and their inverses
(axis-parallel lines in the $(x,y)$-plane).
Simple derivatives are highly directional
whereas the helix derivative is only slightly directional
achieving its meagre directionality entirely from its phase spectrum.
\section{HELIX LOW-CUT FILTER}
\sx{filter ! helix low-cut}
Since the autocorrelation of $\bold H$ is
$\bold H\T \ast \bold H = \bold R =-\nabla^2$
is a second derivative,
the operator $\bold H$ must be something like a first derivative.
As a geophysicist, I found it natural to compare
the operator $\frac{\partial}{\partial y}$
with $\bold H$ by applying them to a local topographic map.
The result shown in
Figure \ref{fig:helocut}
is that $\bold H$ enhances drainage patterns whereas
$\frac{\partial}{\partial y}$ enhances mountain ridges.
\plot{helocut}{width=\textwidth,height=1.4\textwidth}{
        Topography, helical derivative, slope south.
	}
\par
The operator $\bold H$ has
curious similarities and differences
with the familiar gradient and divergence operators.
In two-dimensional physical space,
the gradient maps one field to {\em two} fields
(north slope and east slope).
The factorization of $-\nabla^2$ with the helix
gives us the operator $\bold H$
that maps one field to {\em  one} field.
Being a one-to-one transformation
(unlike gradient and divergence)
the operator $\bold H$ is potentially invertible
by deconvolution (recursive filtering).
\par
I have chosen the name
%\footnote{
%        Any fact this basic should be named in some earlier field
%        of mathematics or theoretical physics.
%        Admittedly, the concept exists on an infinite cartesian plane
%        without a helix, but all my codes in a finite space involve the helix,
%        and the helix concept led me to it.
%        }
``helix derivative''
or ``helical derivative'' for the operator $\bold H$.
A flag pole has a narrow shadow behind it.
The helix integral (middle frame of Figure \ref{fig:lapfac90})
and the helix derivative (left frame)
show shadows with an angular bandwidth approaching $180^\circ$.
\par
Our construction makes $\bold H$ have the energy spectrum $k_x^2+k_y^2$,
so the magnitude of the Fourier transform is $\sqrt{k_x^2+k_y^2}$.
It is a cone
centered and with value zero at the origin.
By contrast, the components of the ordinary gradient
have amplitude responses $|k_x|$ and $|k_y|$
that are lines of zero across the
$(k_x,k_y)$-plane.
\par
The rotationally invariant cone in the Fourier domain
contrasts sharply with the nonrotationally invariant
function shape in $(x,y)$-space.
The difference must arise from the phase spectrum.
The factorization (\ref{eqn:lapfac})
is nonunique in that causality
associated with the helix mapping
can be defined along either $x$- or $y$-axes;
thus the operator 
(\ref{eqn:lapfac})
can be rotated or reflected.
\par
In practice we often require an isotropic filter.
Such a filter is a function of $k_r=\sqrt{k_x^2 + k_y^2}$.
It could be represented as a sum of helix derivatives to integer powers.

\par
If you want to see some tracks on the side of a hill,
you want to subtract the hill and see only the tracks.
Usually, however, you don't have a very good model for the hill.
As an 
expedient you could apply a low-cut filter to remove all
slowly variable functions of altitude.
In chapter \ref{paper:ajt} we found the Sea of Galilee
in Figure \ref{fig:galbin} to be too smooth for viewing pleasure
so we made the roughened versions
in Figure \ref{fig:galocut},
%using a filter based on equation (\ref{eqn:lowcut}),
a one-dimensional filter that we could apply
over the $x$-axis or the $y$-axis.
In Fourier space such a filter has a response function of $k_x$
or a function of $k_y$.
The isotropy of physical space tells us
it would be more logical to design a filter that
is a function of
$k_x^2+k_y^2$.
In Figure \ref{fig:helocut} we saw that the helix derivative
$\bold H$
does a nice job.
The Fourier magnitude of its impulse response is $k_r=\sqrt{k_x^2+k_y^2}$.
There is a little anisotropy connected with phase (which way should
we wind the helix, on $x$ or $y$?) but it is
not nearly so severe as that of either component of the gradient,
the two components having wholly different spectra,
amplitude $|k_x|$ or $|k_y|$.

\subsection{Improving low-frequency behavior}
\par
It is nice having the 2-D helix derivative,
but we can imagine even nicer 2-D low-cut filters.
In one dimension we designed a filter with an adjustable parameter,
a cutoff frequency.
In 1-D we compounded
a first derivative (which destroys low frequencies)
with a leaky integration (which undoes the derivative at all other frequencies).
The analogous filter in 2-D would be
$-\nabla^2 /(-\nabla^2 + k_0^2)$,
which would first be expressed as a finite difference
$ (-Z^{-1} + 2.00 - Z) / (-Z^{-1} + 2.01 - Z)$
and then factored as we did the helix derivative.
\par
We can visualize a plot of the magnitude of the 2-D
Fourier transform of the filter equation (\ref{eqn:lapfac}).
It is a 2-D function of $k_x$ and $k_y$ and it should
resemble $k_r=\sqrt{k_x^2+k_y^2}$.
%It does look like this even when the filter (\ref{eqn:lapfac}) has been truncated.
The point of the cone $k_r=\sqrt{k_x^2+k_y^2}$ becomes
rounded by the filter truncation so
$k_r$ does not reach zero at the origin of the $(k_x,k_y)$-plane.
We can force it to vanish at zero frequency
by subtracting .183 from the lead coefficient 1.791.
I did not do that subtraction in Figure
\ref{fig:helgal}
which explains the whiteness in the middle of the lake.
I gave up on playing with both $k_0$ and filter length,
and now merely play with the sum of the filter coefficients.
\plot{helgal}{width=\textwidth,height=.61\textwidth}{
        Galilee roughened by gradient and
        by helical derivative.
	}
\subsection{Filtering mammograms}
\sx{mammogram filtering}
\sx{filtering mammograms}
\sideplot{mam}{width=.8\textwidth,height=1.0\textwidth}{
	Mammogram (medical X-ray).
	The cancer is the ``spoked wheel.''
	(I apologize for the inability of paper publishing technology
	to exhibit a clear grey image.)
	The tiny white circles are metal foil used for navigation.
	The little halo around a circle exhibits the impulse
	response of the helix derivative.	}

\par
I prepared a half dozen medical X-rays like Figure 
\ref{fig:mam}.
The doctor brought her young son to my office one evening
to evaluate the results.
In a dark room I would show the original X-ray on a big screen
and then suddenly switch to the helix derivative.
Every time I did this, her son would exclaim ``Wow!''
The doctor was not so easily impressed, however.
She was not accustomed to the unfamiliar image.
Fundamentally, the helix derivative applied to her data
does compress the dynamic range making weaker features more readily discernible.
We were sure of this from theory and from
various geophysical examples.
The subjective problem was her unfamiliarity with our display.
I found that I could always spot anomalies more quickly
on the filtered display, but then I would feel more comfortable
when I would discover those same anomalies also present
(though less evident) in the original data.
Thinking this through, I decided the doctor would likely have
been more impressed
had I used a spatial low-cut filter instead of the helix derivative.
That would have left the details of her image
(above the cutoff frequency)
unchanged
altering
only the low frequencies,
thereby allowing me to increase the gain.

%\par
%As we set out, keep in mind that many applications call for
%isotropic filters
%(because physics often tells us there is no prefered orientation).
%This requires us to build upon the operator $-\nabla^2$,
%because all isotropic functions are functions of it.
%It might look anisotropic (because rotating
%$45^\circ$ the finite difference representation of $-\nabla^2$
%seems to change it),
%but in Fourier space
%the well sampled data is at low frequencies,
%and the FT of $-\nabla^2$ is the isotropic expression
%$k_x^2+k_y^2$ which is independent of coordinate rotation.
%
%\par
%Recall the one-dimensional causal lowcut filters (\ref{ajt/eqn:lowcut})
%that we invented earlier.
%Applying one of these to its time reverse gives 
%an expression recognizable as a low-cut filter.
%\begin{equation}
%{1-Z^{-1}\over 1-\rho Z^{-1}}\ \ 
%{1-Z     \over 1-\rho Z     }
%\quad\approx\quad
%{\omega^2 \over \omega^2 + \omega_0^2}
%\label{eqn:lofac}
%\end{equation}
%As you look at this expression,
%think of the numerator and denominator
%as expressions that could be sent to a spectral factorization module.
%For example, in 1-D the numerator
%is the autocorrelation $(-1,2,-1)$.
%The denominator is an autocorrelation like $(-1,2.05,-1)$.
%These two autocorrelations can be factored
%by Kolmogoroff or Burg-Wilson methods.
%Equation
%(\ref{eqn:lofac}) shows them as already factored.
%In two dimensions we can use the same approach.
%The factor of the numerator is the helix derivative
%and the factor of the
%denominator being found likewise.

%\par
%The difficulty with medical x-rays is the large required range of amplitude.
%Part of the image is wholly dark and part wholly light.
%With mammograms, the helix filter ``worked'' because it cut out
%the very low spatial frequencies that amount to a background noise
%on the details.
%A less desireable aspect of the helix derivative is that it alters
%the spectrum at high frequencies making the image less familiar
%to long-time practitioners.
%What we really need here is not the helix derivative but a 2-D low-cut filter.

\par
First I had a problem preparing Figure \ref{fig:mam}.
It shows shows the application of the helix derivative
to a medical X-ray.
The problem was that the original X-ray was all positive
values of brightness so there was a massive amount of
spatial low frequency present.
Obviously an $x$-derivative or a $y$-derivative would
eliminate the low frequency, but the helix derivative did not.
This unpleasant surprise arises
because the filter in equation
(\ref{eqn:lapfac})
was truncated after a finite number of terms.
Adding up the terms actually displayed in equation
(\ref{eqn:lapfac}),
they sum to .183 whereas theoretically the sum of all the terms should be zero.
From the ratio of .183/1.791 we can say that the filter
pushes zero frequency amplitude 90\% of the way to zero value.
When the image contains very much zero frequency amplitude,
this is not good enough.
Better results could be obtained with more coefficients,
and I did use more coefficients,
but simply removing the mean saved me
from needing a costly number of filter coefficients.

\inputdir{mam}

\par
A final word about the doctor.
As she was about to leave my office she suddenly asked
whether I had scratched one of her X-rays.
We were looking at the helix derivative
and it did seem to show a big scratch.
What should have been a line was broken into a string of dots.
I apologized in advance and handed her the original film negatives
which she proceeded to inspect.
``Oh,'' she said,
``Bad news. There are calcification nodules along the ducts.''
So the scratch was not a scratch,
but an important detail that had not been noticed on the original X-ray.
Times have changed since then.
Nowadays mammography has become digital
and appropriate filtering is defaulted into their presentation.
\par
In preparing an illustration for here,
I learned one more lesson.
The scratch was small,
so I enlarged a small portion of the mammogram for display.
The very process of selecting a small portion
followed by scaling the amplitude
between maximum and minimum darkness of printer ink
had the effect enhancing the visibility of the scratch on
the mammogram itself.
Now Figure \ref{fig:scratch} shows it to be
perhaps even clearer than on the helix derivative.
\plot{scratch}{width=\textwidth,height=.5\textwidth}{
        Not a scratch.
        Reducing the $(x,y)$-space range of the illustration
	allowed boosting the gain,
	thus making the non-scratch more prominent.
}






\section{SUBSCRIPTING A MULTIDIMENSIONAL HELIX}
Basic utilities transform back and forth between
multidimensional matrix coordinates and helix coordinates.
The essential module used repeatedly in applications
later in this book is
\texttt{createhelix}.
% \vpageref{lst:createhelix}.
We begin here from its intricate underpinnings.

\par
Fortran77 has a concept of a multidimensional array being equivalent
to a one-dimensional array.
Given that the hypercube specification
\texttt{nd=(n1,n2,n3,...)} defines the storage
\texttt{dimension} of a data array,
we can refer to a data element as either
\texttt{dd(i1,i2,i3,...)} or 
\texttt{dd( i1 +n1*(i2-1) +n1*n2*(i3-1) +...)}.
The helix says to refer to the multidimensional data
by its equivalent one-dimensional index
(sometimes called its vector subscript or linear subscript).

\par
The filter, however, is a much more complicated story than the data:
First, we require all filters to be causal.
In other words, the Laplacian doesn't fit very well,
since it is intrinsically noncausal.
If you really want noncausal filters,
you will need to provide your own time shifts outside the tools supplied here.
Second, a filter is usually a small hypercube, say
\texttt{aa(a1,a2,a3,...)}
and would often be stored as such.
For the helix we must store it in a special one-dimensional form.
Either way, the numbers
\texttt{na= (a1,a2,a3,...)}
specify the dimension of the hypercube.
In cube form, the entire cube could be indexed
multidimensionally as \texttt{aa(i1,i2,...)} or it could be indexed
one-dimensionally as \texttt{aa(ia,1,1,...)} or sometimes
\begin{comment}
\footnote{
        Some programming minutia:
        Fortran77 does not allow you to refer to an array
        by both its cartesian coordinates
        and by its linear subscript in the same subroutine.
        To access it both ways, you need a subroutine call,
        or you dimension it as 
        \texttt{data(n1,n2,...)}
        and then you refer to it as
        \texttt{data(id,1,1,...)}.
        Fortran90 follows the same rule outside modules.
        Where modules use other modules,
        the compiler does not allow you to refer
        to data both ways,
        unless the array is declared as
        \texttt{allocatable}.
        }
\end{comment}
\texttt{aa[ia]} by letting \texttt{ia} cover a large range.
When a filter cube is stored in its normal ``tightly packed'' form
the formula for computing
its one-dimensional index
\texttt{ia} is
\begin{verbatim}
  ia = i1 +a1*i2 +a1*a2*i3 + ...
\end{verbatim}
When the filter cube is stored in an array
with the same dimensions as the data,
\texttt{data[n3][n2][n1]},
the formula for \texttt{ia} is
\begin{verbatim}
  ia = i1 +n1*i2 +n1*n2*i3 + ...
\end{verbatim}

\par
%The fortran compiler knows how to convert from the multidimensional
%cartesian indices to the linear index.
%We will need to do that, as well as the converse.
Module \texttt{decart} below contains
two subroutines that
explicitly provide us the transformations
between the linear index
\texttt{i}
and the multidimensional indices
\texttt{ii= (i1,i2,...)}.
The two subroutines have the logical names
\texttt{cart2line} and
\texttt{line2cart}.
\moddex{decart}{helical-cartesian coordinate conversion}{29}{57}{api/c}

\par
The fortran linear index is closely related to the helix.
There is one major difference, however,
and that is the origin of the coordinates.
To convert from the linear index
to the helix lag coordinate,
we need to subtract the fortran linear index of the ``1.0''
which is usually taken at
\texttt{center= (1+a1/2, 1+a2/2, ..., 1)}.
(On the last dimension, there is no shift because nobody stores the
volume of zero values that would occur before the 1.0.)
The \texttt{decart} module fails for negative subscripts.
Thus we need to be careful to avoid thinking of the filter's 1.0 
(shown in Figure \ref{fig:3dpef})
as the origin of the multidimensional coordinate system
although the 1.0 is the origin in the one-dimensional coordinate system.

\par
Even in one dimension
(see the matrix in equation (\ref{eqn:contran1})),
to define a filter {\it operator} we need to know
not only filter coefficients and a filter length,
but we also need to know the data length.
To define a multidimensional filter using the helix idea,
besides the properties intrinsic to the filter,
we also need to know the circumference of the helix,
i.e., the length on the 1-axis of the data's hypercube
as well as the other dimensions
\texttt{nd=(n1,n2,...)}
of the data's hypecube.

\par
Thinking about convolution on the helix,
it is natural to think about the filter and data being stored
in the same way, that is, by reference to the data size.
This would waste so much space, however,
that our helix filter module
\texttt{helix} 
%\vpageref{lst:helix}
instead stores the filter coefficients in one vector
and their lags in another.
The \texttt{i}-th coefficient value
of the filter goes in  \verb#aa->flt[i]# and
the \texttt{i}-th lag \texttt{ia[i]} goes in \verb#aa->lag[i]#.
The lags are the same as the fortran linear index
except for the overall shift of the 1.0 of a cube of 
data dimension \texttt{nd}.
Our module for convolution on a helix,
\texttt{helicon}. % \vpageref{lst:helicon},
has already an implicit
``1.0'' at the filter's zero lag so we do not store it.
(It is an error to do so.)

%\par
%A helix subscript is similar to a fortran vector subscript.
%To convert, we need to keep track of the data size as mentioned above,
%and we need to shift to lag zero the filter's 1.0
%labeled ``A'' in the example below.
%Logic for identifying the first point A on the starting end of the helix
%and those that lie before the first point (and are hence off the helix) is
%\par\noindent\footnotesize
%\begin{verbatim}
%      . . . . .      . . B . .      C C C C C       D D . . .        0 0 0 . .
%      . . . . .      . . B . .      C C C C C       D D . . .        0 0 0 . .
%      . . A . .      . . B . .      . . . . .       D D . . .   =    0 0 0 . .
%      . . . . .      . . B . .      . . . . .       D D . . .        0 0 . . .
%      . . . . .      . . B . .      . . . . .       D D . . .        0 0 . . .
%      
%             Cube locations for (h <= 0) are A | (B & C) | D.
%\end{verbatim}
%\par\noindent\normalsize
%These are locations for which we will not have adjustable filter values
%because the filter lags are zero or negative.
%Using this logic in multidimensional space,

\par
Module
\texttt{createhelix} 
%\vpageref{lst:createhelix}
allocates memory for a helix filter and builds
filter lags along the helix from the hypercube description.
The hypercube description is not the literal cube
seen in Figure~\ref{fig:3dpef} but
some integers specifying that cube:
the data cube dimensions            \texttt{nd},
likewise the filter cube dimensions \texttt{na},
the parameter \texttt{center} identifying
the location of the filter's ``1.0'',
and a \texttt{gap} parameter used in a later chapter.
To find the lag table,
module \texttt{createhelix} %\vpageref{lst:createhelixmod}
first finds the
fortran linear index of the \texttt{center} point on the filter hypercube.
Everything before that has negative lag on the helix and can be ignored.
(Likewise, in a later chapter we see a \texttt{gap} parameter
that effectively sets even more filter coefficients to zero
so their lags can be ignored too.)
Then it sweeps from the center point over the rest of the filter hypercube
calculating for a data-sized cube \texttt{nd},
the fortran linear index of each filter element.
\moddex{createhelix}{constructing helix filter in N-D}{36}{80}{user/gee}
Near
the end of the code you see the calculation of a parameter \texttt{lag0d}.
This is the count of the number of zeros that
a data-sized fortran array would store
in a filter cube before the filter's 1.0.
We need to subtract this shift
from the filter's fortran linear index to get the lag on the helix.

\par
A filter can be represented literally as a multidimensional cube
like equation (\ref{eqn:2dpef}) shows us in two dimensions
or like Figure~\ref{fig:3dpef} shows us in three dimensions.
Unlike the helical form, in literal cube form,
the zeros preceding the ``1.0'' are explicitly present
so \texttt{lag0} needs to be added back in to get the fortran subscript.
To convert a helix filter \texttt{aa} to fortran's multidimensional hypercube
\texttt{cube(n1,n2,...)} is
module \texttt{boxfilter}: %\vpageref{lst:boxfilter}.
\moddex{boxfilter}{Convert helix filter to (n1,n2,...)}{24}{48}{user/gee}
The \texttt{boxfilter} module is normally used
to display or manipulate a filter
that was estimated in helical form
(usually estimated by the least-squares method).

\begin{comment}
\par
The inverse process to \texttt{boxfilter} is to
convert a fortran hypercube to a helix filter.
For this we have module \texttt{unbox}. %\vpageref{lst:unbox}.
It abandons all zero-valued coefficients
such as those that should be zero before the box's 1.0.
It abandons the ``1.0'' as well, because it is implicitly present
in the helix convolution module \texttt{helicon} \vpageref{lst:helicon}.
\moddex{unbox}{Convert hypercube filter to helix}
An example of using \texttt{unbox} would be copying some numbers
such as the factored laplacian in equation (\ref{eqn:lapfac})
into a cube and then converting it to a helix.
\end{comment}

\par
A reasonable arrangement for a small 3-D filter is 
\texttt{na=\{5,3,2\}}
and
\texttt{center=\{3,2,1\}}.
Using these arguments, I used 
\texttt{createhelix} 
%\vpageref{lst:createhelix} 
to create a filter.
I set all the helix filter coefficients to 2.
Then I used module \texttt{boxfilter} 
%\vpageref{lst:boxfilter}
to put it in a convenient form for display.
%After this conversion, the coefficient \texttt{aa[0][1][2]} is 1, not 2.
Finally, I printed it:
\par\noindent
\footnotesize
\begin{verbatim}
          0.000  0.000  0.000  0.000  0.000
          0.000  0.000  1.000  2.000  2.000
          2.000  2.000  2.000  2.000  2.000
          ---------------------------------
          2.000  2.000  2.000  2.000  2.000
          2.000  2.000  2.000  2.000  2.000
          2.000  2.000  2.000  2.000  2.000
\end{verbatim}
\normalsize
\par\noindent
%using \texttt{printn} \vpageref{lst:print}:
%\moddex{print}{print out helix filter coefficients}

\par
Different data sets have different sizes.
To convert a helix filter from one data size to another,
we could drop the filter into a cube with module \texttt{cube}.
Then we could extract it with module \texttt{unbox}
specifying any data set size we wish.
Instead we use module
\texttt{regrid} %\vpageref{lst:regrid}
prepared by Sergey Fomel
which does the job without reference to an underlying filter cube.
He explains his \texttt{regrid} module thus:
        \begin{quotation}
        Imagine a filter being cut out of a piece of paper and
        glued on another paper, which is then rolled to form a helix.
\par            % change this to a blank line and POOF! paragraph disappears.
        We start by picking a random point (let's call it
        \texttt{rand}) in the
        cartesian grid and placing the filter so that its center
        (the leading 1.0) is on top of that point.
        \texttt{rand} should be larger than (or equal to)
        \texttt{center} and
        smaller than \texttt{min (nold, nnew)},
        otherwise the filter might stick outside the grid
        (our piece of paper.)
        \texttt{rand=nold/2} will do (assuming the filter is small),
        although nothing should change
        if you replace \texttt{nold/2} with a random integer array
        between \texttt{center} and \texttt{nold - na}.
\par
        The linear coordinate of \texttt{rand} is \texttt{h0}
        on the old helix and \texttt{h1} on the new helix.
        Recall that the helix lags \verb#aa->lag#
        are relative to the center.
        Therefore, we need to add \texttt{h0}
        to get the absolute helix coordinate (\texttt{h}).
        Likewise, we need to subtract \texttt{h1}
        to return to a relative coordinate system.
        \end{quotation}

\moddex{regrid}{Convert filter to different data size}{24}{43}{user/gee}


\begin{exer}
\item
Observe the matrix
(\ref{eqn:contran1})
which corresponds to
subroutine
\texttt{tcai1}
What is the matrix corresponding to
\texttt{helicon} ?
\end{exer}

%\end{notforlecture}

\clearpage

