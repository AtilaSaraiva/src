% copyright (c) 1998 Jon Claerbout

\title{Noisy data}
\author{Jon Claerbout}
\maketitle
\label{paper:noiz}


Noise comes in two distinct flavors.
First is erratic bursty noise
which is difficult to fit into a statistical model.
It bursts out of our simple models.
To handle this noise we need ``robust'' estimation procedures
which we consider first.

\par
Next is noise that has a characteristic spectrum,
temporal spectrum, spatial spectrum, or dip spectrum.
Such noise is called ``stationary'' noise.
A special case of stationary noise is
low frequency drift of the mean value of a signal.

\par
In real life, we need to handle both bursty noise
and stationary noise at the same time.
We'll try that now.

\section{MEANS, MEDIANS, PERCENTILES AND MODES}
\bxbx{Mean}{mean}s, \bx{median}s, and \bx{mode}s are different averages.
Given some data values $d_i$ for $i=1,2,...,N$,
the arithmetic mean value $m_2$ is
\begin{equation}
m_2 \eq {1 \over N} \ \sum_{i=1}^N \ d_i
\end{equation}
It is useful to notice that this $m_2$ is the solution
of the simple fitting problem
$ d_i \approx m_2$ or
$\bold d \approx m_2$,
in other words, $\min_{m_2} \sum_i (m_2-d_i)^2$ or
\begin{equation}
0 \eq {d \over dm_2} \ \sum_{i=1}^N \ (m_2-d_i)^2
\end{equation}

\par
The median of the $d_i$ values
is found when the values are sorted from smallest to largest
and then the value in the middle is selected.
The median is delightfully well behaved even
if some of your data values happen to be near infinity.
Analytically,
the median arises from the optimization
\begin{equation}
\min_{m_1}\ \sum_{i=1}^N \ |m_1-d_i|
\end{equation}
To see why, notice that the derivative of the absolute value
function is the signum function,
\begin{equation}
{\rm sgn}(x) \eq \lim_{\epsilon \longrightarrow 0} \ \ 
                { x \over |x| + \epsilon }
\end{equation}
The gradient vanishes at the minimum.
\begin{equation}
0 \eq {d \over dm_1} \ \sum_{i=1}^N \ |m_1-d_i|                 
\end{equation}
The derivative is easy and the result is a sum of sgn() functions,
\begin{equation}
0 \eq                \ \sum_{i=1}^N \ {\rm sgn}(m_1-d_i)
\end{equation}
In other words it is a sum of plus and minus ones.
If the sum is to vanish, the number of plus ones
must equal the number of minus ones.
Thus $m_1$ is greater than half the data values and less than the other half,
which is the definition of a median.
The mean is said to minimize the $\ell^2$ norm of the residual
and the median is said to minimize its $\ell^1$ norm.

\par
Before this chapter,
our model building was all based on the $\ell^2$ norm.
The median is clearly a good idea
for data containing large bursts of noise,
but the median is a single value while geophysical models
are made from many unknown elements.
The $\ell^1$ norm offers us the new opportunity
to build multiparameter models
where the data includes huge bursts of noise.
\sx{L-2 norm}
\sx{L-1 norm}
\sx{L-0 norm}

\par
Yet another average is the ``\bx{mode},''
which is the most commonly occurring value.
For example, in the number sequence $(1,1,2,3,5)$ the mode is 1
because it occurs the most times.
Mathematically, the mode minimizes the zero norm of the residual,
namely $\ell^0=|m_0-d_i|^0$.
To see why, notice that when we raise a residual to the zero power,
the result is 0 if $d_i=m_0$, and it is 1 if $d_i \ne m_0$.
Thus, the $\ell^0$ sum of the residuals
is the total number of residuals less those for which $d_i$ matches $m_0$.
The minimum of $\ell^0(m)$ is the mode $m=m_0$.
The zero power function is nondifferentiable at the place of interest so
we do not look at the gradient.

\inputdir{Math}
\plot{norms}{width=6in,height=3.2in}{
  Mean, median, and mode.
  The coordinate is $m$.
  Top is the $\ell^2$, $\ell^1$,
  and $\ell^{1/10}\approx \ell^0$ measures of $m-1$.
  Bottom is the same measures of the data set $(1,1,2,3,5)$.
  (Made with Mathematica.)
}
\par
$\ell^2(m)$ and
$\ell^1(m)$ are convex functions of $m$ (positive second derivative for all $m$),
and this fact leads to
the triangle inequalities $\ell^p(a)+\ell^p(b) \ge \ell^p(a+b)$
for $p\ge 1$
and assures slopes lead to a unique (if $p>1$) bottom.
Because there is no triangle inequality for $\ell^0$,
it should not be called a ``norm'' but a ``measure.''

\par
Because most values are at the mode,
the mode is where a probability function is maximum.
The mode occurs with the maximum likelihood.
It is awkward to contemplate the mode for floating-point values
where the probability is minuscule (and irrelevant)
that any two values are identical.
A more natural concept is to think of the mode
as the bin containing the most values.

%\par  ALL THIS IS WRONG !
%Although the mode is difficult to deal with theoretically
%and there is little literature about it,
%the $\ell^0$-guided approach is evidently better than
%the $\ell^1$-guided approach
%in the debursting task in Figure \FIG{burst}.
%The mode seems to say to throw away bad values
%while the median says to handle their polarity
%as the polarities of all other values.

\subsection{Percentiles and Hoare's algorithm}
\par
The median is the 50-th \bx{percentile}.
After residuals are ordered from smallest to largest,
the 90-th percentile is the value with 10\% of the values
above and 90\% below.
At SEP the default value for clipping plots of field data
is at the 98th percentile.
In other words, magnitudes above the 98-th percentile
are plotted at the 98-th percentile.
Any percentile is most easily defined if the population
of values $a_i$, for $i=1,2,...,n$
has been sorted into order so that $a_i \le a_{i+1}$ for all $i$.
Then the 90-th percentile is $a_k$ where $k=(90n)/100$.

\par
We can save much work by using \bx{Hoare's algorithm}
which does not fully order the whole list,
only enough of it to find the desired quantile.
Hoare's algorithm is an outstanding example
of the power of a recursive function, a function that calls itself.
The main idea is this:
We start by selecting a random value
taken from the list of numbers.
Then we split the list into two piles,
one pile all values greater than the selected,
the other pile all less.
The quantile is in one of these piles, and by looking
at their sizes, we know which one.
So we repeat the process on that pile
and ignore the other other one.
Eventually the pile size reduces to one, and we have the answer.

\par
In Fortran 77 or C it would be natural to split the list
into two piles as follows:
\begin{quotation}
We divide the list of numbers into two groups,
a group below $a_k$ and another group above $a_k$.
This reordering can be done ``in place.''
Start one pointer at the top of the list and another at the bottom.
Grab an arbitrary value from the list
(such as the current value of $a_k$).
March the two pointers towards each other
until you have an upper value out of order with $a_k$
and a lower value out of order with $a_k$.
Swap the upper and lower value.
Continue until the pointers merge somewhere midlist.
Now you have divided the list into two sublists,
one above your (random) value $a_k$ and the other below.
\end{quotation}
Fortran 90 has some higher level intrinsic vector functions
that simplify matters.
When \texttt{a} is a vector and \texttt{ak}
is a value,
\texttt{a>ak} is a vector of logical values that
are true for each component that is larger than \texttt{ak}.
The integer count of how many components
of \texttt{a} are larger than \texttt{ak}
is given by the Fortran intrinsic function \texttt{count(a>ak)}.
A vector containing only values less than \texttt{ak}
is given by the Fortran intrinsic function \texttt{pack(a,a<ak)}.

\par
Theoretically about $2n$ comparisons
are expected to find the median of a list of $n$ values.
The code below (from Sergey Fomel)
for this task is \texttt{quantile}.
\moddex{quantile}{percentile}{22}{48}{filt/lib}

%\par
%An interesting application of medians is eliminating \bx{noise spikes}
%through the use of a running median.
%A \bx{running median} is a median computed in a moving window.
%Figure \ref{fig:median590} shows depth-sounding data from the Sea of Galilee
%before and after a running median of 5 points was applied.
%The data as I received it is 132044 triples; i.e.,
%$(x_i,y_i,z_i)$ where $i$ is measured along the vessel's track.
%In Figure \ref{fig:median590} the
%depth data $z_i$ appears as one long track
%although the surveying was done in several episodes that
%do not always continue the same track.
%For Figure \ref{fig:median590} I first abandoned the last 2044
%of the 132044 triples and all the $(x_i,y_i)$-pairs.
%Then I broke the remaining long signal into
%the 26 strips you see in the figure.
%Typically the depth is a ``U''-shaped function as
%the vessel crosses the lake.
%You will notice that many spikes are missing on the bottom plot.
%For more about these tracks, see Figure \ref{noiz/fig:seegap}.
%\activeplot{median590}{width=6in,height=8.5in}{ER}{
%        Depth of the Sea of Galilee along the vessel's track.
%        }

\subsection{The weighted mean}
The \bx{weighted mean} $m$ is
\begin{equation}
m \eq
{\sum_{i=1}^N \ w_i^2 d_i  \over
 \sum_{i=1}^N \ w_i^2
 }
\end{equation}
where $w_i^2>0$ is the squared weighting function.
This is the solution to the $\ell^2$ fitting problem
$ 0  \approx w_i (m - d_i)$;
in other words,
\begin{equation}
0 \eq {d \over dm} \ \sum_{i=1}^N \ [w_i(m - d_i)]^2
\end{equation}

%\begin{notforlecture}
\subsection{Weighted L.S. conjugate-direction template}
The pseudocode for minimizing the {\it weighted} residual
$\bold 0\approx \bold r = \bold W (\bold F \bold m - \bold d)$
by conjugate-direction method,
is effectively like that for the unweighted method
except that the initial residual is weighted
and the operator $\bold F$ has the premultiplier $\bold W$.
Naturally, the adjoint operator $\bold F'$
has the postmultiplier $\bold W'$.
In some applications the weighting operator $\bold W$
is simply a weighting function or diagonal matrix
(so then $\bold W = \bold W'$)
and in other applications, the weighting operator $\bold W$
may be an operator,
like the derivative along a data recording trajectory
(so then $\bold W \ne \bold W'$).
%\par\newslide
\def\padarrow{\quad\longleftarrow\quad}
\begin{tabbing}
mmmmmm \= mmmmmm \= mmmmm \kill
\> $\bold r \padarrow \bold W (\bold F \bold m - \bold d)$     \\
\> {\rm iterate \{ }                                            \\
\>      \>  $\Delta\bold m  \padarrow \bold F'\bold W'\         \bold r$ \\
\>      \>  $\Delta\bold r\ \padarrow \bold W \bold F \  \Delta \bold m$ \\
\>      \>  $(\bold m,\bold r) \padarrow {\rm cgstep}
            (\bold m,\bold r, \Delta\bold m,\Delta\bold r )$ \\
\>      \> \}                                           
\end{tabbing}

\subsection{Multivariate $\ell^1$ estimation by iterated reweighting}
\sx{L1 or $\ell^1$}
\sx{L2 or $\ell^2$}
The easiest method of model fitting is linear least squares.
This means minimizing the sums of squares of residuals ($\ell^2$).
On the other hand, we often encounter huge noises
and it is much safer to minimize
the sums of absolute values of residuals
($\ell^1$).
(The problem with $\ell^0$ is that there are multiple minima,
so the gradient is not a sensible way to seek the deepest).

\par
There exist specialized techniques for handling $\ell^1$
multivariate fitting problems.
They should work better than the simple
iterative reweighting outlined here.

%\par
%(An inviting idea is to weight the familiar residual by
%the inverse of the square root of the residual from the previous iteration.
%This is wrong.  The required weight does not have the square root,
%as a more careful derivation shows next.)

\par
A penalty function that ranges from $\ell^2$ to $\ell^1$,
depending on the constant $\bar r$ is
\begin{equation}
E(\bold r) \eq \sum_i \left( \sqrt{1+r_i^2/\bar r^2} - 1 \right)
\label{eqn:L12penalty}
\end{equation}
Where
$r_i/\bar r$
is small, the terms in the sum amount to $r_i^2/2\bar r^2$
and where
$r_i^2/\bar r^2$
is large, the terms in the sum amount to $|r_i/\bar r|$.
We define the residual as
\begin{equation}
r_i \eq \sum_j \ F_{ij}m_j - d_i
\label{eqn:usualres}
\end{equation}
We will need
\begin{equation}
{\partial r_i\over\partial m_k}
\eq \sum_j \ F_{ij} \delta_{jk} \eq F_{ik}
\label{eqn:partialroverm}
\end{equation}
where we briefly used the notation that $\delta_{jk}$ is 1 when
$j=k$ and zero otherwise.
Now,
to let us find the descent direction $\Delta \bold m$,
we will compute
the $k$-th component $g_k$ of the gradient $\bold g$.
We have
\begin{equation}
g_k \eq
{\partial E \over\partial m_k}
\eq \sum_i \ {1\over\sqrt{1+r_i^2/\bar r^2}}\ 
{ r_i \over \bar r^2}
\ 
{\partial r_i\over\partial m_k}
\label{eqn:morel1}
\end{equation}

\begin{equation}
\bold g \eq \Delta\bold m
\eq \bold F' \ {\bf diag} \left(  {1\over\sqrt{1+r_i^2/\bar r^2}}
                  \right) \bold r
\label{eqn:gradwt}
\end{equation}
where we have use the notation ${\bf diag}()$ to designate
a diagonal matrix with its argument distributed along the diagonal.

\par
Continuing, we notice that the new weighting of residuals
has nothing to do with the linear relation between model perturbation
and residual perturbation;
that is,
we retain the familiar relations,
$\bold r = \bold F \bold m -\bold d$ and
$\Delta\bold r = \bold F \Delta\bold m $.

\par
In practice we have the question of how to choose $\bar r$.
I suggest that $\bar r$ be proportional to
${\rm median}(|r_i|)$
or some other percentile.

\subsection{Nonlinear L.S. conjugate-direction template}
\sx{nonlinear optimization}
Nonlinear optimization
arises from two causes:
\begin{enumerate}
\item
Nonlinear physics. The operator depends upon
the solution being attained.
\item
Nonlinear statistics.  We need robust estimators
like the $\ell^1$ norm.
\end{enumerate}
The computing template below is useful in both cases.
It is almost the same as the template for weighted
linear least-squares except that the residual
is recomputed at each iteration.
Starting from the usual weighted least-squares template
we simply move the iteration statement a bit earlier.
\begin{tabbing}
mmmmmm \= mmmmmm \= mmmmm \kill
\> {\rm iterate \{ }                                                    \\
\>      \> $\bold r \padarrow \bold F \bold m - \bold d$                \\
\>      \> $\bold W \padarrow {\bf diag}[w(\bold r)]$                    \\
\>      \> $\bold r \padarrow \bold W \bold r $                          \\
\>      \>  $\Delta\bold m  \padarrow \bold F'\bold W'\         \bold r$ \\
\>      \>  $\Delta\bold r\ \padarrow \bold W \bold F \  \Delta \bold m$ \\
\>      \>  $(\bold m,\bold r) \padarrow {\rm cgstep}
             (\bold m,\bold r, \Delta\bold m,\Delta\bold r )$ \\
\>      \> \}                                           
\end{tabbing}
where ${\bf diag}[w(\bold r)]$ is whatever weighting function we choose
along the diagonal of a diagonal matrix.
\par
Now let us see how the weighting functions relate
to robust estimation:
Notice in the code template that $\bold W$ is applied twice
in the definition of $\bold \Delta\bold m$.
Thus $\bold W$ is the square root of the diagonal operator
in equation (\ref{eqn:gradwt}).
\begin{equation}
\bold W \eq \ {\bf diag} \left(  {1\over\sqrt{\sqrt{1+r_i^2/\bar r^2}}}  \right)
\label{eqn:fourthroot}
\end{equation}
\begin{comment}
\par
% begin Sergey's insertion
%  XXX
%Module \texttt{weight\_solver} \vpageref{/prog:weightsolver}
Module \texttt{solver\_irls} \vpageref{/prog:solver_irls}
%
implements the computational template above. In addition to the usual
set of arguments from the \texttt{solver()} subroutine
\vpageref{/prog:smallsolver}, it accepts a user-defined function (parameter
\texttt{wght}) for computing residual weights. Parameters
\texttt{nmem} and \texttt{nfreq} control the restarting schedule of
the iterative scheme.
\moddex{solver_irls}{iteratively reweighted optimization} 
% end Sergey's insertion
\end{comment}
\par
We can ask whether {\tt cgstep()}, which was not designed
with nonlinear least-squares in mind, is doing the right thing
with the weighting function.
First, we know we are doing weighted linear least-squares correctly.
Then we recall that on the first iteration, the conjugate-directions
technique reduces to steepest descent,
which amounts to a calculation of
the scale factor $\alpha$ with
\begin{equation}
\alpha \eq -\ { 
        \Delta \bold r \cdot \bold r
        \over
        \Delta \bold r \cdot \Delta\bold r
        }
\label{eqn:anotheralpha}
\end{equation}
Of course, {\tt cgstep()} knows nothing about the weighting function,
but notice that the iteration loop above nicely inserts
the weighting function both in $\bold r$ and in $\Delta \bold r$,
as required by (\ref{eqn:anotheralpha}).

\par
Experience shows that difficulties arise
when the weighting function varies rapidly from one iteration to the next.
Naturally, the conjugate-direction method,
which remembers the previous iteration, 
will have an inappropriate memory if 
the weighting function changes too rapidly.
A practical approach is to be sure the changes in the weighting function
are slowly variable.

%I propose gradually increasing $\lambda$ from zero to unity
%as the iteration proceeds.

\subsection{Minimizing the Cauchy function}
\sx{Cauchy function}
A good trick
(I discovered accidentally) is to use the weight
\begin{equation}
\bold W \eq \ {\bf diag} \left(  {1\over{\sqrt{1+r_i^2/\bar r^2}}}  \right)
\label{eqn:cauchyweight}
\end{equation}
Sergey Fomel points out that this weight arises from
minimizing the \bx{Cauchy function}:
\begin{equation}
E(\bold r) \eq \sum_i \ \log (1+r_i^2/\bar r^2)
\label{eqn:cauchypenalty}
\end{equation}
A plot of this function is found in Figure \ref{fig:cauchy}.
\plot{cauchy}{width=6in,height=3.2in}{
  The coordinate is $m$.
  Top is Cauchy measures of $m-1$.
  Bottom is the same measures of the data set $(1,1,2,3,5)$.
  Left, center, and right are for
  \protect$\bar r = (2, 1, .2)$. 
}

\par
Because the second derivative is not positive everywhere,
the Cauchy function introduces the possibility of multiple solutions,
but because of the good results we see in Figure \ref{fig:burst},
you might like to try it anyway.
Perhaps the reason it seems to work so well is that
it uses mostly residuals of ``average size,''
not the big ones or the small ones.
This happens because $\Delta\bold m$ is made from $\bold F'$ and
the components of $\bold W^2\bold r$  which are a function
$r_i/(1+r_i^2/\bar r^2)$
that is maximum for those residuals near
$\bar r$.
\par
% begin Sergey's insertion
Module \texttt{irls} \vpageref{/prog:irls} supplies two useful
weighting functions that can be interchanged as arguments to the
reweighted scheme \vpageref{/prog:solver_irls}.
\moddex{irls}{weighting functions for iterative reweighting}{40}{72}{filt/lib}
% end Sergey's insertion
%\end{notforlecture}



\section{NOISE BURSTS}
\inputdir{burst}
Sometimes noise comes in isolated \bx{spikes}.
Sometimes it comes in \bx{bursts} or bunches (like grapes).
Figure \ref{fig:burst} is a simple one-dimensional example
of a periodic signal plus spikes and bursts.
Three processes are applied to this data,
\bx{despike} and two flavors of \bx{deburst}.
Here we will examine the processes used.
(For even better results, see Figure \ref{fig:pefdeburst}.)
\plot{burst}{width=6.0in,height=2.25in}{
  Top is synthetic data with noise spikes and bursts.
  (Most bursts are a hundred times larger than shown.)
  Next is after running medians.
  Bottom is after the two processes described here.
}

\subsection{De-spiking with median smoothing}

\sx{despike}
\sx{median smoothing}
The easiest method to remove spikes is to pass a moving window
across the data and output the median value in the window.
This method of despiking was done in Figure \ref{fig:burst},
which shows a problem of the method:
The window is not long enough to clean the long bursts,
but it is already so long
that it distorts the signal by flattening its peaks.
The window size really should not be chosen in advance
but should depend upon by what is encountered on the data.
This I have not done
because the long-burst problem is solved
by another method described next.

\subsection{De-bursting}
Most signals are smooth, but running medians assume they have no curvature.
An alternate expression of this assumption is that the signal
has minimal curvature
$ 0 \approx h_{i+1} -2 h_{i} + h_{i-1} $;
in other words,
$ \bold 0 \approx \nabla^2 \bold h$.
Thus we propose to create the cleaned-up data $\bold h$
from the observed data $\bold d$ with the fitting problem
\begin{equation}
  \begin{array}{lll}
        0 &\approx & \bold W  (\bold h - \bold d)                       \\
        0 &\approx & \epsilon\  \nabla^2   \bold h
  \end{array}
\end{equation}
where $\bold W$ is a diagonal matrix with weights sprinkled along the diagonal,
and where $\nabla^2$ is a matrix
with a roughener like $(1,-2,1)$ distributed along the diagonal.
This is shown in Figure \ref{fig:burst} with $\epsilon = 1$.
Experience showed similar performances
for $0 \approx \nabla \bold h$ and $0 \approx \nabla^2 \bold h$.
Better results, however, will be found later in Figure
\ref{fig:pefdeburst},
where the $\nabla^2$ operator is replaced
by an operator designed to predict this very predictable signal.

%  THIS IS HORRIBLE.  VERBOSE CODES ARE INCOMPLETE !    :-((
%\par
%What we need now is another solver module
%with the weighting functionality of
%\texttt{weight\_solver} \vpageref{/prog:weightsolver}
%and with the model damping of
%\texttt{reg\_solver} \vpageref{/prog:regsolver}.
%It is in the library under the name \texttt{solver\_mod}.
%(One goal of this book was to show every code,
%but that becomes irritating when many codes are nearly the same.
%This difficulty partly arises because I like to introduce
%only one concept at a time, and it partly arises (I think)
%because Fortran 90 is not a highly object-oriented language.)
%\moddex{deburst1}{remove noise bursts}
%Additionally, we need the identity operator, here named \texttt{copy}[
%\opdex{copy}{remove noise bursts}

%\listing{../../Lib/deburst1.rt}

%\par
%I invite you to experiment with different noises and
%different debursting methods and tell me what you find.
%Notice that we have two parameters to adjust,
%$\epsilon$ and the number of iterations.
%
%(The simple median approach
%can be generalized to an $\ell^1$-norm multivariate fitting problem
%by the method of linear programming as was done
%in FGDP and in my classic paper with Francis \bx{Muir}.
%The reason I do not continue that approach
%is that so far as I know,
%those $\ell^1$ techniques require storing the operator.
%The trouble is that
%we often have problems
%in which the data and model spaces themselves
%strain the available memory
%while the operator size
%(data size {\it times} model size)
%is unreasonably large.
%As you see, iteratively reweighted least squares
%can amount to something like an $\ell^1$-norm approach anyway.)
%
%\subsection{The view from here}
%Multivariate fitting is significantly more complicated than running medians.
%You have struggled to get this high.
%Let us look at the view in several directions.
%\par
%Having done the simple bursty-noise problem in Figure \ref{fig:burst}
%by a multivariate fitting method,
%we are ready to fit noisy data to general linear models, such as 
%\begin{eqnarray}
%       0 &\approx & \bold W  (\bold F \bold h -\bold d )               \\
%       0 &\approx & \epsilon \nabla^2 \bold h
%\end{eqnarray}
%\par
%Please notice that we have have learned to handle
%noise bursts in the {\it residual},
%which is not the same as noise bursts in the {\it data}.
%We are now ready for cases in which the model changes rapidly
%because of being high frequency or having poles or discontinuities.
%This occurs,
%for example, in the altitude function across a cliff,
%where the fitting problem is something like:
%\begin{eqnarray}
%       0 &\approx & \bold W_d  (\bold F \bold h - \bold d )            \\
%       0 &\approx & \bold W_h  \nabla^2 \bold h
%\end{eqnarray}


\section{MEDIAN BINNING}
\inputdir{rbst}
We usually add data into bins.
When the data has erratic noise,
we might prefer to take the median of the values in each bin.
Subroutine \texttt{medbin()}
(in the library, but not listed here)
performs the chore.
It is a little tricky because we first need to find out
how many data values go into each bin,
then we must allocate that space
and copy each data value from its track location to its
bin location.
Finally we take the median in the bin.
A small annoyance with medians
is that when bins have an even number of points,
like two, there no middle.
To handle this problem,
subroutine \texttt{medbin()}
uses the average of the middle two points.

%\progdex{medianbin2}{median in bin}      %XXX used for one figure.
\par
A useful byproduct of the calculation is the residual:
For each data point its bin median is subtracted.
The residual can be used to remove suspicious points
before any traditional least-squares analysis is made.
An overall strategy could be this:
First a coarse binning with many points per bin,
to identify suspicious data values,
which are set aside.
Then a sophisticated least squares analysis
leading to a high-resolution depth model.
If our search target is small, 
recalculate the residual with the high-resolution model
and reexamine the suspicious data values.

\plot{medbin}{width=6in,height=3.8in}{
  Galilee water depth binned and roughened.
  Left is binning with the mean, right with the median.
}
\par
Figure \ref{fig:medbin} compares the water depth in the Sea of Galilee
with and without median binning.
The difference does not seem great here
but it is more significant than it looks.
Later processing will distinguish between empty bins (containing an exact zero)
and bins with small values in them.
Because of the way the depth sounder works,
it often records an erroneously near-zero depth.
This will make a mess of our later processing
(missing data fill)
unless we cast out those data values.
This was done by median binning in Figure \ref{fig:medbin}
but the change is disguised by the many empty bins.

\par
Median binning is a useful tool,
but where bins are so small that they hold only one or two points,
there the median for the bin is the same as the usual arithmetic average.


\section{ROW NORMALIZED PEF}
\sx{row normalized PEF}
We often run into \bx{bursty noise}.
This can overwhelm the estimate of a prediction-error filter.
To overcome this problem we can use a weighting function.
The weight for each row in fitting matrix
(\ref{eqn:exmiss})
is adjusted so that each row has about the same
contribution as each other row.
A first idea is that the weight for the $n$-th row
would be the inverse of the sum of the absolute values of the row.
This is easy to compute:
First make a vector the size of the PEF $\bold a$ but with each element unity.
Second, take a copy of the signal vector $\bold y$
but with the absolute value of each component.
Third, convolve the two.
% ------ Temporarily commented by Sergey
%This is done in subroutine
%\texttt{rnpef1()}.
%\progdex{rnpef1}{row normalized PEF}
%
\noindent
The convolution of the ones with the absolute values
could be the inverse of the weighting function we seek.
However, any time we are forming an inverse we need to think
about the possibility of dividing by zero, how it could arise,
and how divisions by ``near zero'' could be even worse
(because a poor result is not immediately recognized).
Perhaps we should use something between $\ell^1$ and $\ell^2$ or Cauchy.
In any case, we must choose a scaling parameter
that separates ``average'' rows from unusually large ones.
For this choice in subroutine \texttt{rnpef1()}, I chose the median.


\section{DEBURST}
\inputdir{burst}
\par
We can use the same technique to throw out fitting equations
from defective data that we use for missing data.
Recall the theory and discussion leading up to 
Figure \ref{fig:burst}.
There we identified defective data by its lack
of continuity.  We used the fitting equations
$0\approx w_i (y_{i+1} -2y_i + y_{i-1})$
where the weights $w_i$ were chosen
to be approximately the inverse
to the residual $(y_{i+1} -2y_i + y_{i-1})$ itself.
\par
Here we will first use the second derivative
(Laplacian in 1-D) to throw out bad points,
while we determine the PEF.
Having the PEF, we use it to fill in the missing data.

\moddex{pefest}{estimate PEF in 1-D avoiding bad data}{38}{53}{user/gee}
The result of this ``PEF-deburst'' processing
is shown in Figure \ref{fig:pefdeburst}.
\plot{pefdeburst}{width=6.0in,height=3in}{
  Top is synthetic data with noise spikes and bursts.
  (Some bursts are fifty times larger than shown.)
  Next is after running medians.
  Next is Laplacian filter Cauchy deburst processing.
  Last is PEF-deburst processing.
}
\par
Given the PEF that comes out of \texttt{pefest1()}\footnote{
        If you are losing track of subroutines defined earlier,
        look at the top of the module to see what other modules
        it \texttt{use}s.
        Then look in the index to find page numbers of those modules.
        }, subroutine
\texttt{fixbad1()} below convolves it with the data and looks for
anomalous large outputs.  For each that is found, the input data is
declared defective and set to zero.  Then subroutine \texttt{mis1()}
\vpageref{/prog:mis2} is invoked to replace the zeroed values by
reasonable ones.
\moddex{fixbad}{restore damaged data}{41}{48}{user/gee}



\subsection{Potential seismic applications of two-stage infill}
Two-stage data infill has many applications
that I have hardly begun to investigate.
\par {\bf Shot continuation}
is an obvious task for a data-cube extrapolation program.
There are two applications of shot-continuation.
First is the obvious one of repairing holes in data
in an unobtrusive way.
Second is to cooperate with reflection tomographic studies
such as that proposed by Matthias \bx{Schwab}.
\par {\bf Offset continuation} is a well-developed topic because
of its close link with \bx{dip moveout} (\bx{DMO}).
DMO is heavily used in the industry.
I do not know how the data-cube extrapolation code I
am designing here would fit into DMO and stacking,
but because these are such important processes,
the appearance of a fundamentally new tool like
this should be of interest.
It is curious that the DMO operator is traditionally
derived from theory, and the theory requires the
unknown velocity function of depth, whereas here
I propose estimating the offset continuation operator
directly from the data itself, without the need of a velocity model.
\par
Obviously, one application is to extrapolate off the sides of a
\bx{constant-offset section}.
This would reduce migration semicircles
at the survey's ends.
\par
Another application is to extrapolate off the
\bx{cable ends}
of a common-midpoint gather or
a common shot point gather.
This could enhance
the prediction of
multiple reflections
or reduce artifacts in velocity analysis.
\par
Obviously, the methodology and code in this chapter
is easily extendable to four dimensions (prestack 3-D data).
%The application that drove me to putting the code in its
%present form is extending \bx{Kjartansson}-style \bx{tomography}.
%

%\begin{notforlecture}
\section{TWO 1-D PEFS VERSUS ONE 2-D PEF}
\inputdir{duel}
%       Waveforms are commonly recorded
%       on crossing lines such as shown in Figure \FIG{duelin}.
%       Obviously we can fill in between lines
%       by minimizing the power out of a Laplacian operator
%       but for more predictable images than Galilee
%       it might be better to minimize the output energy of a PEF.
%       In practice the problem is, where do we get the PEF?

Here we look at the difference between using two 1-D PEFs,
and one 2-D PEF.
Figure \ref{fig:duelin} shows an example of sparse tracks;
it is not realistic
in the upper-left corner
(where it will be used for testing),
in a quarter-circular disk where
the data covers the model densely.
Such a dense region is ideal for determining the 2-D PEF.
Indeed, we cannot
determine a 2-D PEF from the sparse data lines,
because at any place you put the filter
(unless there are enough adjacent data lines),
unknown filter coefficients will multiply missing data.
So every fitting goal is nonlinear
and hence abandoned by the algorithm.
\plot{duelin}{width=6in,height=3.0in}{
  Synthetic wavefield (left) and as observed over survey lines (right).
  The wavefield is a superposition of waves from three directions.
}

\par
The set of test data shown in Figure \ref{fig:duelin}
is a superposition of three functions like plane waves.
One plane wave looks like low-frequency horizontal layers.
Notice that the various layers vary in strength with depth.
The second wave is dipping about $30^\circ$ down to the right
and its waveform is perfectly sinusoidal.
The third wave dips down $45^\circ$ to the left
and its waveform is bandpassed random noise like the horizontal beds.
These waves will be handled differently by different processing schemes,
so I hope you can identify all three.
If you have difficulty,
view the figure at a grazing angle from various directions.

\par
Later we will make use of the dense data region,
but first let $\bold U$ be the east-west PE operator
and $\bold V$ be the north-south operator
and let the signal or image be $\bold h = h(x,y)$.
The fitting residuals are
\begin{equation}
        \begin{array}{lll}
        \bold 0 &\approx& (\bold I - \bold J) (\bold h - \bold d) \\
        \bold 0 &\approx&  \bold U \ \bold h  \\
        \bold 0 &\approx&  \bold V \ \bold h
        \end{array}
        \label{eqn:maskregression}
\end{equation}
where $\bold d$ is data (or binned data) and $(\bold I-\bold J)$
masks the map onto the data.

\par
Figure \ref{fig:dueleither} shows
the result of using a single one-dimensional PEF
along either the vertical or the horizontal axis.

% APPLIES TO A DIFFERENT SUBROUTINE
%To make the figure I used subroutine \LPROG{maski2}.
%To prepare its input, I used subroutine \GPROG{pef2} to find the 1-D PEFs.
%Although \GPROG{pef2} is designed to find 2-D PEFs
%it is easily invoked to find a 1-D PEF.
%For one figure panel I set {\tt (a1,a2)} equal to {\tt (7,1)}
%and for the other panel I it set equal to {\tt (1,7)}.
%In this figure
%the second PEF {\tt bb(,)} required by the subroutine is taken zero.

\plot{dueleither}{width=6in,height=3.0in}{
  Interpolation by 1-D PEF along the vertical axis (left)
  and along the horizontal axis (right).
}

\par
%The same solver subroutine {\tt maski2()}
%was used for two simultaneous 1-D PEFs.
%For that it was loaded with {\tt aa(7,1)} and {\tt bb(1,7)}.
%Results for two simultaneous 1-D PEFs are in Figure \FIG{duelversus}.
%Again I used the same solver for a 2-D PEF
%loading with {\tt aa(7,4)} (the filter's ``1'' being at (4,1))
%and with {\tt bb(1,1)} being again a zero filter.
%To get this 2-D PEF, I used \GPROG{pef2} and I was
%dependent on the cheating corner of dense data.

%The purpose for cheating here is to establish motivation
%for the more difficult task of doing the nonlinear estimation
%on data lines where cheating would be impossible.
Figure \ref{fig:duelversus} compares
the use of a pair of 1-D PEFs versus a single 2-D PEF
(which needs the ``cheat'' corner in Figure \ref{fig:duelin}.
\plot{duelversus}{width=6in,height=3.0in}{
  Data infilled by a pair of 1-D PEFs (left)
  and by a single 2-D PEF (right).
}
Studying Figure \ref{fig:duelversus} we conclude
(what theory predicts) that
\begin{itemize}
        \item These waves are predictable with a pair of 1-D filters:
        \begin{itemize}
                \item Horizontal (or vertical) plane-wave with random waveform
                \item Dipping plane-wave with a sinusoidal waveform
        \end{itemize}
        \item These waves are predictable with a single 2-D filter:
        \begin{itemize}
                \item both of the above
                \item Dipping plane-wave with a random waveform
        \end{itemize}
\end{itemize}

%\subsection{Inverse masking versus missing data fitting}
%The idea of inverse masking seems like the idea of missing-data fitting
%but their conceptual bases differ.
%Missing data estimation arises from the single fitting problem
%\begin{eqnarray}
%    \bold 0 &\approx & \bold U          \bold h                             \\
%    \bold 0 &\approx & \bold A [\bold J \bold h+(\bold I-\bold J)\bold h ]  \\
%    \bold 0 &\approx & \bold A ( \bold J \bold h + \bold K \bold d )
%\end{eqnarray}
%where $\bold J$ is the usual ``missing-data operator''
%also called the data's free mask,
%$\bold I-\bold J=\bold K$ is the data-known mask,
%and where solving proceeds by iteration,
%implying powers of $\bold A \bold J$ and its adjoint.
%On the other hand,
%\bx{masking inversion} has two or three fitting equations
%like \EQN{maskregression}, say
%\begin{eqnarray}
%\bold 0 &\approx& \bold K ( \bold h - \bold d) \\
%\bold 0 &\approx& \bold U \ \bold h
%\end{eqnarray}
%Solving by iteration
%implies powers of the column operator
%$ \left[ \begin{array}{c} \bold K \\ \bold U \end{array} \right]$
%and its adjoint.
%\par
%The missing-data formulation assures us that
%the solution matches the known data exactly at all stages during the iteration.
%The inverse-mask formulation allows mismatch.
%Therefore, the mask formulation seems inferior.
%In practice, however,
%the mask formulation has many useful generalizations.
%When field data is stored as tracks,
%these tracks can cross, and when they do,
%the data could be inconsistent with itself.
%This often happens.
%It is not an embarrassment but
%an important problem deserving a reformulation (which we tackle next).
%Another generalization is binning
%where the tracking operator throws many data values into the same bin.
%Where data values in a bin differ,
%the solution (bin value) obviously cannot be consistent with each data value.
%Yet another generalization of the masking operator is where data
%is not on a regular mesh but needs to be interpolated from it.
%\end{notforlecture}

\begin{comment}
\section{SPARSE TRACKS IN SATELLITE ALTIMETRY }
\sx{satellite altimetry}
Earth satellites orbiting over the poles go round us in north-flying tracks
and south-flying tracks.
Their orbits would be north-south lines,
but the earth's rotation causes them to appear to drift westward
(with the sun) so the north-going runs leave northwest-going tracks
and the southward runs leave southwest-going tracks.
Figure~\ref{fig:synsat} shows an interesting topography
striated in two non-orthogonal directions
(much like superposed plane waves {\tt ;-)} and some tracks.
For my convenience I have plotted northwest tracks in a vertical
direction and southwest tracks in a horizontal direction.
In reality the track types are not orthogonal, so my displays will
have a shearing distortion (not affecting the analysis).
The satellites considered here measure \bx{sea surface} altitude.
You might think the sea surface should be at the same altitude everywhere,
but the gravitational attraction of the mountains beneath the sea
causes the altitude of sea level to vary from place to place.
It would be an arduous task to survey all the oceans to find their depth
and map it, or to map the strength of gravity on all the ocean surfaces,
but either map looks much like each other
and both look like the satellite altimeter map,
the differences being mainly in calibration and spatial filtering.

\activeplot{synsat}{width=6in,height=1.8in}{ER}{
        Synthetic topography (left),
        northward tracks (center), and
        southward tracks (right).
        The south pole is the lower left-hand corner
        and thirty degrees south latitude is the diagonal.
        }

\par
There is great track density below $30^\circ$ south latitude,
but it is much sparser above.
Most satellite tracks are missing north of the
$30^\circ$ south latitude because
knowledge of the distribution of gravity is required
to guide a ballistic missile to a precise target,
so they are secret.
The tracks north of $30^\circ$ south latitude come from other satellites.

\par
Notice that some tracks are brighter and some are dimmer.
This mimics a calibration problem
that causes each track to have a different mean, or more generally,
a very low-frequency, along-the-track noise.

\par
Although the geometry of Figure~\ref{fig:synsat} is somewhat unusual,
it is an attractive example for study for several reasons.
First, it exhibits the familiar problem that survey lines
are never as dense as we wish them to be.
Second, it is not unusual for crossing lines to have inconsistency.
Third, having dense coverage near sparse coverage
offers us a good chance to study the transition
from a region of high information density to a region of low density
and ambiguity (spatial aliasing).

%\subsection{Conventional wisdom}
\par
The obvious way to fill in between the tracks is with some
kind of smoothness criterion.
For example, we could have the solution
fit the track data exactly and between the tracks
we could have the solution satisfy Laplace's equation.
For Figure~\ref{fig:seasat},
I chose instead to minimize the {\it energy} in the output
of the \bx{Laplacian operator}
(which amounts to satisfying the {\it squared} Laplace equation).
The problem with either of these approaches
is the implicit presumption of isotropy,
i.e., that the interpolation be independent of orientation.
The best result is obviously the last one in  Figure~\ref{fig:seasat}
and we'll explore that next.

\subsection{A deeper model than anisotropy}
\par
From the data south of $30^\circ$ latitude
in Figure~\ref{fig:synsat},
we see the orientation of the two \bx{striations}
(along with the third striation of along-the-track noise).
But there are two orientations in the topography,
not just one, so the problem is not so simple
that we can merely find a skewing and stretching of coordinates
that would make the image appear isotropic.
The approach I advocate is to seek a partial differential equation
that the image will be a solution of.
In reality, a complicated image will satisfy no simple differential equation,
so instead I seek a filter with minimum power out.
I first find this filter in the region of good spatial coverage.
Then I carry it north to the area of sparse coverage
where I find the solution by the principle of minimum power out.
The result is shown on the right in Figure~\ref{fig:seasat}.
The result is overwhelmingly better than the isotropic approach.
The result is almost perfect but the solutions
might still be a little weaker in the middle of large track gaps
because I limited the number of conjugate-direction iterations to 20.
(The figure computes in a few seconds.)
\activeplot{seasat}{width=6in,height=1.8in}{ER}{
        Adjoint reconstruction (left),
        reconstruction by Laplacian (center), and
        reconstruction by 2-D PEF (right).
        }
\par
You might notice that the two striations in the topography
have different textures,
one being string like and the other being rod-like.
This means that the underlying ``waves'' making up the striations
have a different spectrum.
The method used here works equally well (or badly)
if the two spectra are the same.
The fundamental assumption is one of superposition,
an assumption whose validity will be limited according to
the geological mechanism that built the topography.

\subsection{Distribution of sparsity}
\sx{sparsity}
We rarely have data in which the transition from good coverage to poor
is as abrupt as here, but
we generally have data that terminates
even more abruptly at the edges of a survey.
There is also the possibility of coverage that is uniformly sparse,
so that we would have no good region
in which to learn the data statistics.
In principle we should be simultaneously estimating the solution
and its multidimensional prediction-error filter.
Such simultaneous estimation is nonlinear.
Thus there are well-known dangers,
but the problem itself is not ill-conceived or impossible to approach,
as the one-dimensional example ofFGigure~\ref{fig:exp} shows.
Although real problems are nonlinear, 
it is often realistic to approach them textbook-style,
as a sequence of linearized approximations.
Sometimes ingenious tricks can be brought to bear,
as in Figure \ref{fig:lace3}.

\par
Current practice in the seismic-exploration industry
meets the requirements perfectly of a nonlinear problem
that is near to linear,
because the high costs of occupying many data stations
limit the surveys to avoid aliasing at central frequencies,
while allowing aliasing at the highest frequencies
(which define the resolution).

\subsection{Seasat modeling and solution formulation}
\sx{Seasat}
The problem formulation is,
\begin{eqnarray}
\label{eqn:seasat}
\bold 0 &\approx& \bold A_n( \bold T \bold h - \bold d) \\
\bold 0 &\approx& \bold A_h \ \epsilon\ \bold h
\end{eqnarray}
where $\bold A_n$ is a noise whitener,
$\bold A_h$ is a topography whitener (PEF),
$\bold d$ is data,
$\bold h$ is altitude,
$\bold T$ is the operator that makes data tracks from an altitude map, and
$\epsilon$ is the usual damping parameter.
Figure~\ref{fig:seasat} uses two different values of $\bold A_h$,
one a Laplacian and one a PEF.

\par
Subroutine \texttt{icai2()} is yet another 2-D convolution program,
a straightforward generalization of the 1-D subroutine \texttt{icaf1()} \vpageref{/prog:icaf1},
but with the adjoint being the input instead of the filter.
We need it to apply the spatial PEF to the topography
without overflowing the edges.
\progdex{icai2}{convolution}
Subroutine \texttt{track()} shows how tracks are defined from topography,
and it also does the adjoint reconstruction of topography from tracks.
The array {\tt dknow(,,)} is 1 where data is recorded and 0 where it is not.
The subroutine also carries along the $\epsilon \bold I$ operator
with subroutine \texttt{ident()} \vpageref{/prog:ident}.
\progdex{track}{satellite tracks}
\progdex{ident}{identity operator}

\subsection{Along-track noise and crooked tracks}
\sx{crooked tracks}
Subroutine \texttt{trakwit()}
applies a weighting operator $\bold A_n$ to each track.
This \bx{weighting operator} is a little unusual because it
is not a scaling function but an operator.
Recall the unknown mean values or low-frequency noise in the data tracks.
Because of this noise,
there would be huge residuals if we tried to fit the solution altitude
to the crossing altimeter tracks
where they do not match at the point of crossing.
The proper method is to filter the {\it residual}
to eliminate the low-frequency noise in the {\it data.}
Many people filter {\it observed} data to eliminate noise in it,
but when we come to compare such data to {\it modeled} data
we need to filter it with the same filter.
Filtering the residual cleans both data sets at the same time,
with an assuredly identical filter.
\par
The simplest filter that eliminates zero frequency is the
first-order, finite-difference operator $(1,-1)$.
The next simplest is an operator with a narrower spectral notch
around zero frequency, namely  $(1,-1/3,-1/3,-1/3)$, etc.
I found that both worked about equally well in this case.
These are rough prior guesses of the filters that whiten the {\it data}.
A more general solution is to form the {\it residuals}
and then design a whitener for them.
A still more general solution is to find the residuals and the PEF
at the same time.
%\progdex{trakwit}{weight sat.~tracks}
\progdex{trakwit}{weight sat. tracks}
The idea behind
subroutine {\tt track()} applies also to crooked survey lines.
For use with crooked lines, however,
subroutine {\tt trakwit()} would need modification
so that it could
apply the data noise-whitening filter along the actual track
and not simply along the coordinate axis.

\subsection{Seasat optimization}
\sx{Seasat}
To prepare to minimize the fitting residuals (\ref{eqn:seasat}),
we begin by restating the residual
as the output of the operator
that takes a model perturbation        $\Delta \bold h$
and produces the weighted residual perturbation $\Delta \bold r$.
\begin{equation}
\left[ 
\begin{array}{c}
  \             \\ 
  \Delta \bold r \\ 
  \ 
  \end{array} \right] 
\eq
\left[ 
\begin{array}{ccc}
  \bold A_n    & \cdot   & \cdot   \\
  \cdot    & \bold A_n   & \cdot   \\
  \cdot    & \cdot       &   \bold A_h
  \end{array} \right] 
\; \left[ 
\begin{array}{c}
  \bold N \\ 
  \bold S \\ 
  \epsilon \bold I
  \end{array} \right]
\ \left[ \Delta \bold h \right]
\label{eqn:seastat}
\end{equation}
where we have also split the track operator $\bold T$ into two parts,
$\bold N$ making the northwest tracks and
$\bold S$ making the southwest tracks.
\par
To find the best-fitting topography $\bold h=h(x,y)$,
we prepare the solver subroutine \texttt{seatopo()}.
Begin by loading the residual vector
with the negative of the data $\bold r=-\bold d$
and load the solution vector with zeros $\bold h=\bold 0$.
Then proceed as we did with earlier weighted residuals
like \texttt{pef1()} \vpageref{/prog:pef1} or \texttt{pef2()} \vpageref{/prog:pef2}.
First we revise the data to weighted data
by a single application of subroutine \texttt{trakwit()} \vpageref{/prog:trakwit}.
Then begin the usual conjugate-direction iteration
containing first the adjoint of operator
(\ref{eqn:seastat}) and then the forward operator.
The only thing new here is that the {\it filter weighting} cannot
have the output  and input share the same computer memory,
as is allowed
by the more usual {\it scalar weighting} routines,
\texttt{ident()} \vpageref{/prog:ident} and \texttt{diag()} \vpageref{/prog:diag},
so an extra array {\tt dr2} must be declared.
(In a computer language better than Fortran
there would be no need to list separate programs
for each application of weighted least squares---one routine
should meet all needs.)
\progdex{seatopo}{topog. estimation}

\par
Finally, subroutine \texttt{seastat()} builds a topographic map
from satellite data tracks;
the subroutine uses either of two different statistical assumptions.
The first assumption is that the 2-D prediction-error filter (PEF)
for the topography is Laplace's operator.
The alternate assumption is that a previous map is available
which can be used by \texttt{pef2()} \vpageref{/prog:pef2} to compute the required PEF.
The previous map is presumed to be poor where the tracks are missing.
To put this presumption into practice, we set
the map to zero where there are no data tracks,
so {\tt pef2()} will handle such regions as missing data.
Then we define the template of free parameters with subroutine \texttt{setfree()} \vpageref{/prog:setfree};
and we call subroutine \texttt{pef2()} \vpageref{/prog:pef2} to get the PEF.
With the inputs now all prepared, we invoke subroutine \texttt{seatopo()}.
\progdex{seastat}{topog. statistics}

\end{comment}







\section{ALTITUDE OF SEA SURFACE NEAR MADAGASCAR}


\par
A satellite points a radar at the ground and
receives echos we investigate here.
These echos are recorded only over the ocean.
The echo tells the distance from the orbit to the ocean surface.
After various corrections are made for earth and orbit ellipticities
the residual shows tides, wind stress on the surface,
and surprisingly a signal proportional to the depth of the water.
Gravity of mountains on the water bottom pulls water towards them
raising sea level there.

\par
The raw data investigated here\footnote{
	I wish to thank David T. Sandwell
	http://topex.ucsd.edu/
	for providing me with this subset of satellite altimetry data,
	commonly known as Topex-Posidon data.
	}
had a strong north-south tilt
which I\footnote{
	The calculations here were
	all done for us by Jesse Lomask.
	}
removed at the outset.
Figure~\ref{fig:jesse1} gives our first view of altimetry data
(ocean height) from southeast of the island of
Madagascar.

\activeplot{jesse1}{width=6.0in, height=2.31in}{}{
	Sea height under satellite tracks.
	The island of Madagascar is
	in the empty area at $(46^\circ,-22^\circ)$.
	Left is the adjoint $\bold L'\bold d$.
	Right is the adjoint normalized by the bin count,
	${\bf diag}(\bold L'\bold 1)^{-1} \bold L'\bold d$.
	You might notice a few huge, bad data values.
	Overall, the topographic function is too smooth,
	suggesting we need a roughener.
	}
About all we can see is satellite tracks.
The satellite is in a circular polar orbit.
To us the sun seems to rotate east to west
as does the circular satellite orbit.
Consequently, when the satellite moves northward across the site
we get altitude measurements along a SE-NW line.
When it moves southward we get measurements along a NE-SW line.
This data is from the cold war era.
At that time dense data above the $-30^\circ$ parallel was secret
although sparse data was available.
(The restriction had to do with precision guidance of missiles.
Would the missile hit the silo?
or miss it by enough to save the retaliation missile?)

\par
Here are some definitions:
Let components of $\bold d$ be the data,
altitude measured along a satellite track.
The model space is $\bold h$, altitude in the $(x,y)$-plane.
Let $\bold L$ denote the 2-D linear interpolation operator
from the track to the plane.
Let $\bold H$ be the helix derivative,
a filter with response $\sqrt{k_x^2+k_y^2}$.
Except where otherwise noted,
the roughened image $\bold p$ is the preconditioned variable
$\bold p =\bold H \bold h$.
The derivative along a track in data space is ${d\over dt}$.
A weighting function that vanishes when any filter hits a track end
or a bad data point is $\bold W$.

\activesideplot{jesse5}{height=3.6in}{}{
	All the data $\bold d$.
	}
\par
Figure~\ref{fig:jesse5} shows the entire data space,
over a half million data points (actually 537974).
Altitude is measured along many tracks across the image.
In Figure~\ref{fig:jesse5} the tracks are placed end-to-end,
so it is one long vector (displayed in about 50 signal rows).
A vector of equal length is the missing data marker vector.
This vector is filled with zeros everywhere except where
data is missing or known bad or known to be at the ends of the tracks.
The long tracks are the ones that are sparse in the north.

\activesideplot{jesse2}{width=3.0in, height=2.31in}{}{
	The roughened, normalized adjoint,
	$\bold H \ {\bf diag}(\bold L'\bold 1)^{-1} \bold L'\bold d$.
	Some topography is perceptible
	through a maze of tracks.
	}

\par
Figure~\ref{fig:jesse2} brings this information into model space.
Applying the adjoint of the linear interpolation operator $\bold L'$
to the data $\bold d$ gave our first image $\bold L'\bold d$
in model space in Figure~\ref{fig:jesse1}.
The track noise was so large that roughening it made it worse.
A more inviting image arose when I normalized the image before roughening it.
Put a vector of all ones $\bold 1$ into the
adjoint of the linear interpolation operator $\bold L'$.
What comes out $\bold L'\bold 1$
is roughly the number of data points landing in each pixel in model space.
More precisely, it is the sum of the linear interpolation weights.
This then, if it is not zero, is used as a divisor.
The division accounts for several tracks contributing to one pixel.
In matrix formalism this image is
${\bf diag}(\bold L'\bold 1)^{-1} \bold L'\bold d$.
In Figure~\ref{fig:jesse2} this image is roughened
with the helix derivative $\bold H$.

\activeplot{jesse3}{width=6.0in, height=2.31in}{}{
	With a simple roughening derivative in data space,
	model space shows two nice topographic images.
	Let $\bold n$ denote ascending tracks.
	Let $\bold s$ denote descending tracks.
	Left  is $\bold L' {d\over dt} \bold n$.
	Right is $\bold L' {d\over dt} \bold s$.
	}

\par
There is a simple way here to make a nice image---roughen
along data tracks.
This is done in
Figure~\ref{fig:jesse3}.
The result is two attractive images, one for each track direction.
Unfortunately, there is no simple relationship between the two images.
We cannot simply add them because the shadows go in different directions.
Notice also that each image has noticeable tracks that we would
like to suppress further.
\par
A geological side note:
The strongest line, the line that marches along the image from
southwest to northeast is a sea-floor spreading axis.
Magma emerges along this line
as a source growing plates that are spreading apart.
Here the spreading is in the north-south direction.
The many vertical lines in the image are called ``transform faults''.

\par
Fortunately, we know how to merge the data.
The basic trick is to form the track derivative
not on the data (which would falsify it)
but on the residual which
(in Fourier space) can be understood as
choosing a different weighting function for the statistics.
A track derivative on the residual is actually two track derivatives,
one on the observed data, the other on the modeled data.
Both data sets are changed in the same way.

Figure~\ref{fig:jesse10} shows the result.
\activeplot{jesse10}{width=6.0in, height=2.31in}{}{
	All data merged into a track-free image (hooray!)
	by applying the track derivative,
	not to the data, but to the residual.
	Left is $\bold h$
	estimated by
	$\bold 0\approx \bold W {d\over dt}(\bold L\bold h-\bold d)$.
	Right is the roughened altitude, $\bold p = \bold H \bold h$.
	}
The altitude function remains too smooth for nice viewing
by variable brightness,
but roughening it with $\bold H$ makes an attractive image
showing, in the south, no visible tracks.


%\plot{jesse10.res}{width=6.0in, height=2.31in}{
%	Fifty thousand of a half million (537,974) data points
%	of Figure~\ref{fig:jesse10}.
%	Left is the residual $\bold L\bold h -\bold d$.
%	Right is the residual $\bold W {d\over dt} (\bold L\bold h -\bold d)$.
%	Jesse,
%	One thing that was very interesting for Antoine
%	was to carry the residual back into model space.
%	I suspect ours will be boring.  If you do this try several pclips.
%	I'm trying to think up an explanation for this interesting residual.
%	I'm finding it hard to believe the derivative of
%	those ramps is the white noise.  It should be steps?
%	I hope the N-S tilt was removed once and for all?
%	so that I don't need to mention it here.
%	--jon
%	}
%
%\sideplot{jesse10.res2}{width=3.0in, height=2.31in}{
%	Figure~\ref{fig:jesse10} again but the weighted residual
%	residual $\bold W(\bold L\bold h -\bold d)$.
%	The spikes are the zero-values at the track ends.
%	}
%
%\plot{jesse11}{width=6.0in, height=2.31in}{
%	Holes filled with a model space PEF.
%	Starting from the $\bold h$ in Figure~\ref{fig:jesse10}
%	and filling holes with GEE program
%	{\tt Miss} using a PEF $\bold A$ found by {\tt Pef}
%	we get $\bold h$ on the left.
%	The ridge topography is building in the northern region.
%	Right is the roughened altitude $\bold H \bold h$.
%	The northern ridge cannot stand the roughener
%	and the north again becomes dominated by tracks.
%	}
%
%\sideplot{jesse12.1}{width=3.0in, height=2.31in}{
%	The 2-D prediction error $\bold A \bold h$ of model space.
%	It is white by design.
%	}
%
%\plot{jesse6}{width=6.0in, height=2.31in}{
%	An attempt to in-fill with a gradient (100 iterations)
%	without preconditioning.
%	Left is $\bold h$ where
%	$\bold 0 \approx \bold W {d\over dt} (\bold L\bold h-\bold d)$
%	and
%	$\bold 0 \approx \nabla \bold h$.
%	Right is $\bold p =\bold H\bold h$.
%	The lesson here is that regularization with the gradient
%	doesn't build much topography in the north, even with many iterations.
%	We'll probably omit this and its residuals from book,
%	maybe keep in lab.
%	It is puzzling that the signal is so weak on top.
%	Do we need a different $\epsilon$ on the top and bottom?
%	}
%\plot{jesse6.res}{width=6.0in, height=2.31in}{
%	(OMIT FROM BOOK)
%	Residuals 20,000 to 70,000.
%	Left is the residual $\bold L\bold h -\bold d$.
%	Right is residual $\bold W {d\over dt} (\bold L\bold h -\bold d)$.
%	Regularization with $\bold 0 \approx \nabla \bold h$.
%	}
%\sideplot{jesse6.res2}{width=3.0in, height=2.31in}{
%	(OMIT FROM BOOK)
%	The residual of Figure~\ref{fig:jesse6.res}
%	with track-end weight without track derivative.
%	Spikes are zero values where the residual vanishes at track ends.
%	Same data-space subset.
%	}
%\plot{jesse7}{width=6.0in, height=2.31in}{
%	(OMIT FROM BOOK)
%	Necessity of the track derivative.
%	Precondition with the helix.
%	Left is
%	$\bold h = \bold H^{-1}\bold p$ where
%	$\bold p$ was estimated by
%	$\bold 0\approx  \bold L\bold H^{-1} \bold p-\bold d$.
%	Right is $\bold p$.
%	Theoretically this is like regularizing with $\nabla$.
%	}

\par
The north is another story.
We would like the sparse northern tracks
to contribute to our viewing pleasure.
We would like them to contribute to a northern image of the earth,
not to an image of the data acquisition footprint.
\activeplot{jesse8}{width=6.0in, height=2.31in}{}{
	Using the track derivative in residual space 
	and helix preconditioning in model space
	we start building topography in the north.
	Left is $\bold h=\bold H^{-1}\bold p$ where
	$\bold p$ is estimated by
	$ \bold 0 \approx \bold W {d\over dt} (\bold L\bold H^{-1}\bold p-\bold d)$
	for only 10 iterations.
	Right is $\bold p=\bold H\bold h$.
%	Maybe this is the place to say
%	that using the helix slows down
%	each iteration because it is a lot bigger than $\nabla$.
%	In the lab they might compare this to regularization.
	}
This begins to happen in Figure~\ref{fig:jesse8}.
The process of fitting data by choosing an altitude function $\bold h$
would normally include some regularization (model styling),
such as
$\bold 0\approx \nabla \bold h$.
Instead we adopt the usual trick
of changing to preconditioning variables,
in this case $\bold h = \bold H^{-1}\bold p$.
As we iterate with the variable $\bold p$ we watch the images
of $\bold h$ and $\bold p$ and quit either when we are tired,
or more hopefully, when we are best satisified with the image.
This subjective choice is rather like choosing the $\epsilon$
that is the balance between data fitting goals and model styling goals.
The result
in Figure~\ref{fig:jesse8}
is pleasing.
We have begun building topography in the north that continues
in a consistant way with what is in the south.
Unfortunately, this topography does fade out rather quickly
as we get off the data acquisition tracks.



%\plot{jesse8.strt}{width=6.0in, height=2.31in}{
%	(OMIT FROM BOOK)
%	A good starting solution doesn't help much.
%	This is like Figure~\ref{fig:jesse8}
%	but the calculation began at a good
%	starting solution,
%	the regularized result in Figure~\ref{fig:jesse6}.
%	Will omit from book.
%	}

\par
If we have reason to suspect that the geological style north of
the 30th parallel matches that south of it
(the stationarity assumption) we can compute a PEF on the south side
and use it for interpolation on the north side.
This is done in Figure~\ref{fig:jesse9}.
\activeplot{jesse9}{width=6.0in, height=2.31in}{}{
	Given a PEF $\bold A$ estimated on the densely defined southern part
	of the model,
	$\bold p$ was estimated by
	$\bold 0\approx \bold W {d\over dt}(\bold L\bold A^{-1}\bold p-\bold d)$
	for 50 iterations.
	Left is  $\bold h = \bold A^{-1}\bold p$.
	Right is $\bold p=\bold H\bold h$.
%	Figure~\ref{fig:jesse11} used a PEF
%	but it didn't turn out this well.
%	Why's that?
%	A good question.
	}
This is about as good as we are going to get.
Our fractured ridge continues nicely into the north.
Unfortunately, we have imprinted the fractured ridge
texture all over the northern space,
but that's the price we must pay for relying on the stationarity assumption.

\par
The fitting residuals
are shown in Figure~\ref{fig:jesse9_res}.
\activeplot{jesse9_res}{width=6.0in, height=2.31in}{}{
	The residual at
	fifty thousand of the half million (537,974) data points
	in Figure~\ref{fig:jesse9}.
	Left is physical residual $\bold L\bold A^{-1}\bold p -\bold d$.
	Right is fitting residual
	$\bold W {d\over dt} (\bold L\bold A^{-1}\bold p -\bold d)$.
	}
The physical altitude residuals tend to be rectangles,
each the duration of a track.
While the satellite is overflying other earth locations the ocean surface
is changing its altitude.
The fitting residuals (right side) are very fuzzy.
They appear to be ``white'', though with ten thousand points
crammed onto a line a couple inches long, we cannot be certain.
We could inspect this further.
If the residuals turn out to be significantly non-white,
we might do better to change $d\over dt$ to a PEF along the track.


%\sideplot{jesse9.res2}{width=3.0in, height=2.31in}{
%	(OMIT FROM BOOK)
%	The residual $\bold W(\bold L\bold A^{-1}\bold p -\bold d)$.
%	 found from
%	$\bold 0\approx\bold W {d\over dt}(\bold L\bold A^{-1}\bold p-\bold d)$.
%	Same subset of data space.
%	Observe that these residuals all have the same polarity.
%	Why is that?  A good question.
%	I'm confused.
%	}
















\section{ELIMINATING NOISE AND SHIP TRACKS IN GALILEE}

\par
The Sea of Galilee data set has enchanted my colleagues and me
because the data has comprehensible defects
that have frustrated many of our image estimation designs.
The best results found so far were prepared for us here
by Antoine Guitton based on our 2004 Geophysics paper.

\par

We are given depth-sounding data from the Sea of 
Galilee.  The Sea of Galilee is unique
because it is a fresh-water lake below sea-level.
It seems to be connected to the Great Rift (pull-apart)
valley crossing East Africa. The ultimate goal is to produce a good map of
the depth to bottom, and images useful for identifying archaeological,
geological, and geophysical details of the water bottom. In particular,
we hope to identify some ancient shorelines around the lake and meaningful 
geological features inside the lake. The ancient shorelines might
reveal early settlements of archeological interest or old fishing ports.
The pertinence of this data set to our daily geophysical problems is threefold:
(1) We often need to interpolate irregular data.
(2) The data has noise bursts of various types.
(3) The data has systematic error (drift)
which tends to leave data-acquisition tracks in the resulting image.

\par
The Galilee data set was introduced in chapter \ref{paper:iin}
and recently plotted in Figure~\ref{fig:medbin}.
Actually, that figure is a view of 2-D model space.
One of the first things I learned (the hard way) is the importance
of viewing both the model space and the residuals in data space.
\par
\boxit{
	Be sure to plot both model space and data space.
	You should try to understand the results in both spaces
	and might like to watch movies of each as the iteration progresses.
	}

\par
The raw data (Figure \ref{fig:antoine1}),
is distributed irregularly across the lake surface.
It is 132,044 triples $(x_i,y_i,z_i)$, where $x_i$ ranges over about 
12 km, where $y_i$ ranges over about 20 km,
and $z_i$ is depth in multiples of 10 cm.
(It would have been helpful if a fourth value had been included,
the clock-date time $t_i$, of the measurement.)
The ship surveyed a different amount of distance every day of the survey.
Figure \ref{fig:antoine1} displays the whole survey as one long track.
On one traverse across the lake, the depth record is U shaped.
A few V shaped tracks result from deep-water vessel turn arounds.
All depth values (data points) used for building the final map are shown here.
Each point corresponds to one depth measurement inside the lake.
For display convenience, the long signal is broken
into 23 strips of 5718 depth measurements.
We have no way to know that sometimes the ship stops a little while
with the data recorder running;
sometimes it shuts down overnight or longer;
but mostly it progresses at some unknown convenient speed.
So the horizontal axis in data space is a measurement number
that scales in some undocumented way to distance along the track.

\activeplot{antoine1}{width=6in,height=4in}{} {
  The complete Galilee data space.}


\subsection{Attenuation of noise bursts and glitches}


Let $\bold{h}$ be an abstract vector containing as components
the water depth over a 2-D spatial mesh.
Let $\bold{d}$ be an abstract vector whose successive components
are depths along the vessel tracks.
One way to grid irregular data is to minimize the length 
of the residual vector $\bold r_d(\bold h)$:
\begin{equation}
	\bold 0 \quad\approx\quad \bold r_d \quad=\quad \bold B \bold h \ -\  \bold d    \label{eq0}
\end{equation}
where $\bold B$ is a 2-D linear interpolation (or binning) operator 
and $\bold r_d$ is the data residual.
Where tracks cross or where multiple data values end up in the same bin,
the fitting goal (\ref{eq0}) takes an average.
%The bin size is 60 $\times$ 50 m so that the number of data points per bin
%is roughly constant and the aspect ratio of the lake is roughly preserved
%in the number of samples in the vertical and horizontal directions. 
Figure~\ref{fig:medbin}
is a display of simple binning of the raw data.
(Some data points are outside the lake.
These must represent navigation errors.)


\par

Some model-space bins will be empty.
For them we need an additional ``model styling'' goal,
i.e. regularization.
For simplicity we might minimize the gradient.
\begin{equation}
  \begin{array}{lllll}
    \bold 0 &\approx& \bold r_d &=& \bold B \bold h \ -\  \bold d \\
    \bold 0 &\approx& \bold r_h &=& \epsilon \nabla \bold h 
  \end{array} \label{eq1}
\end{equation}
where $\nabla=\left ( \frac{\partial}{\partial x},
\frac{\partial}{\partial y}\right)$ and $\bold r_h$ is the model space
residual.
Choosing a large scaling factor $\epsilon$ will tend to smooth
our entire image, not just the areas of empty bins.
We would like $\epsilon$ to be any number small enough
that its main effect is to smooth areas of empty bins.
When we get into this further, though, we'll see that
because of noise
some smoothing across the nonempty bins is desireable too.


\subsection{Preconditioning for accelerated convergence}

As usual we
precondition by changing variables so
that the regularization operator becomes an identity matrix.
The gradient $\nabla$ in equation (\ref{eq1}) has no inverse, but its
spectrum $-\nabla'\nabla$,
can be factored ($-\nabla'\nabla={\bf H'H}$) into triangular parts 
${\bf H}$ and ${\bf H'}$ where ${\bf H}$ is the helix derivative.
This ${\bf H}$ is invertible by deconvolution.
The quadratic form
$\bold h'\nabla'\nabla\bold h = \bold h'\bold H'\bold H \bold h$
suggests the new preconditioning variable $\bold p = \bold H\bold h$.
The fitting goals in equation (\ref{eq1}) thus become
\begin{equation}
  \begin{array}{lllll}
    \bold 0 &\approx& \bold r_d &=& \bold B \bold {H^{-1}} \bold p \ -\  \bold d \\
    \bold 0 &\approx& \bold r_p &=& \epsilon \bold p
  \end{array} \label{eq2}
\end{equation}
with $\bold r_p$ the residual for the new variable ${\bf p}$.
Experience shows that an iterative solution for ${\bf p}$ converges much
more rapidly than an iterative solution for ${\bf h}$,
thus showing that ${\bf H}$ is a good choice for preconditioning. 
We could view the estimated final map ${\bf h}={\bf H^{-1}p}$,
however in practice, because the depth function is so smooth,
we usually prefer to view the roughened depth $\bold p$.


\par
There is no simple way of knowing beforehand the best value of $\epsilon$.
Practitioners like to see solutions for various values of $\epsilon$.
Practical exploratory data analysis is pragmatic.
Without a simple, clear theoretical basis, analysts
generally begin from ${\bf p=0}$ and then abandon the fitting goal
$\bold 0 \approx  \bold r_p =\epsilon \bold p$.
Effectively, they take $\epsilon=0$.
Then they examine the solution as a function
of iteration, imagining that the solution at larger iterations
corresponds to smaller $\epsilon$ and that the solution at smaller iterations
corresponds to larger $\epsilon$.
In all our explorations, we follow this approach
and omit the regularization in the estimation of the depth maps.
Having achieved the general results we want,
we should include the parameter $\epsilon$ and adjust it until
we see a pleasing result at an ``infinite'' number of iterations.
We should but usually we do not.


\subsection{${\ell^1}$ norm}


Spikes and erratic noise glitches can be suppressed
with an approximate $\ell^1$ norm.
One main problem with the Galilee data is the presence of outliers
in the middle of the lake and at the track ends.
We could attenuate these spikes by editing or applying running median filters.
However, the former involves human labor
while the latter might compromise small details
by smoothing and flattening the signal.
Here we formulate the estimation
to eliminate the drastic effect of the noise spikes.
We introduce a weighting operator that deemphasizes high residuals as follows:
\begin{equation}
  \begin{array}{lllll}
    \bold 0 &\approx& \bold r_d &=& \bold W ( \bold B \bold {H^{-1}} \bold p - \bold d
    )\\
    \bold 0 &\approx&  \bold r_p &=& \epsilon \bold p
  \end{array} \label{eq3}
\end{equation}
with a diagonal matrix $\bf{W}$:
\begin{equation}
{\bf W} = {\bf diag} \left( \frac{1}{(1+r_i^2/\bar{r}^2)^{1/4}} \right)
\end{equation}
where $r_i$ is the residual for one component of $\bold r_d$
and $\bar{r}$ is a prechosen constant. This weighting operator
ranges from $\ell^2$ to $\ell^1$, depending on the constant $\bar{r}$.
We take $\bar{r}=10$ cm
because the data was given to us as integer multiples of 10 cm.
(A somewhat larger value might be more appropriate).


\activeplot{antoine2}{width=\textwidth}{}{Estimated ${\bf p}$
	in a least-squares sense (left) and in an
	$\ell^1$ sense (right).
	Pleasingly, isolated spikes are attenuated.
	Some interesting features are shown by the arrows:
	AS points to few ancient shores,
	O points to some outliers,
	T points to few tracks,
	and R points to a curious feature.
	} 


\par
Figure \ref{fig:antoine2} displays ${\bf p}$ estimated 
in a least-squares sense on the left and in a $\ell^1$ sense on the right 
(equation (\ref{eq3}) with a small $\bar{r}$).
Most of the glitches are no longer visible.
One obvious glitch remains near $(x,y)=(205,238)$.
Evidently a north-south track has a long sequence of biased measurements
that our $\ell^1$ cannot overcome.
Some ancient shorelines in the western and southern parts of the Sea of
Galilee are now easier to identify (shown as AS).
We also start to see a valley in the middle of the lake (shown as R). 
Data outside the lake (navigation errors) have been mostly removed.
Data acquisition tracks (mostly north-south lines and east-west lines,
one of which is marked with a T)
are even more visible after the suppression of the outliers.


\activeplot{antoine3}{width=5in,height=3in}{}{
	East-west cross sections of the lake bottom
	(${\bf h = {\bf H^{-1}p}}$).
	Top with the $\ell^2$ solution.
	Bottom with the $\ell^1$ approximating procedure.
	}

\par
Figure \ref{fig:antoine3}
shows the bottom of the Sea of Galilee (${\bf h = {\bf H^{-1}p}}$)
with $\ell^2$ (top) fitting and  $\ell^1$ (bottom) fitting. 
Each line represents one east-west transect,
transects at half-kilometer intervals on the north-south axis.
The $\ell^1$ result is a nice improvement over the $\ell^2$ maps.
The glitches inside and outside the lake have mostly disappeared.
Also, the $\ell^1$ norm gives positive depths everywhere. 
Although not visible everywhere in all the figures,
topography is produced outside the lake.
Indeed, the effect of regularization is to produce synthetic topography,
a natural continuation of the lake floor surface.

%\subsection{Attenuation of ship tracks}

\par
We are now halfway to a noise-free image.
Figure \ref{fig:antoine2} shows that
vessel tracks overwhelm possible fine scale details.
Next we investigate a strategy based on the idea that
the inconsistency between tracks comes mainly 
from different human and seasonal conditions during the data acquisition. 
Since we have no records of the weather and the time 
of the year the data were acquired
we presume that the depth differences between different acquisition tracks
must be small and relatively smooth along the super track.


\subsection{Abandoned strategy for attenuating tracks}
An earlier strategy to remove the ship tracks is to filter the
residual as follows:
\begin{equation}
  \begin{array}{lllll}
    \bold 0 &\approx& \bold r_d &=& \bold W \frac{d}{ds}( \bold B \bold {H^{-1}} \bold p - \bold d
    ),\\
    \bold 0 &\approx& \bold r_p &=& \epsilon \bold p,
  \end{array} \label{eq5}
\end{equation}
where $\frac{d}{ds}$ is the derivative along the track. The derivative 
removes the drift from the field data (and the modeled data).
An unfortunate consequence of the track derivative
is that it creates more glitches and spiky noise at the track ends and
at the bad data points.
Several students struggled with this idea without good results.

\par
One explanation (of unknown validity)
given for the poor results is that perhaps
the numerical conditioning
of the algebraic problem is worsened by the operators $\bold W$, 
$\frac{d}{ds}$, $\bold B$, and $\bold {H^{-1}}$,
drastically slowing the convergence.
Another explanation is that the operator $d\over ds$ is too simple.
Perhaps we should have a five or ten point low-cut filter---or maybe a PEF.
A PEF could be estimated from the residual itself.
Unfortunately, such a longer filter would smear the bad effect
of noise glitches onto more residuals,
effectively spoiling more measurements.

\par
We concluded that the data is bad only in a very low frequency sense.
Perhaps the lake is evaporating, or it is raining,
or the load in the boat has been changed or shifted.
It's a fact that any very low-frequency reject filter
is necessarily a long filter,
and that means that it must catch many noise spikes.
Thus we should not attempt to filter out the drift from the residual.
Instead we should model the drift.

\par
\boxit{
	In the presence of both noise bursts and
	noise with a sensible spectrum (systematic noise),
	the systematic noise should be modeled while
	the noise bursts should be handled with $\ell^1$.
	}

\subsection{Modeling data acquisition drift}

To model data drift we imagine a vector $\bold q$ of random numbers
that will be passed thru a low-pass filter (like a leaky integrator) $\bold L$.
The modeled data drift is $\bold L\bold q$.
We will solve for $\bold q$.
A price we pay is an increase of the number of unknowns.
Augmenting earlier fitting goals
(\ref{eq3}) we have:
\begin{equation}
  \begin{array}{lllll}
    \bold 0 &\approx& \bold r_d &=& \bold W ( \bold B \bold {H^{-1}} \bold p + \lambda
    \bold L \bold q - \bold d
    ),\\
    \bold 0 &\approx& \bold r_p &=& \epsilon_1 \bold p ,\\
    \bold 0 &\approx& \bold r_q &=& \epsilon_2 \bold q,
  \end{array} \label{eq4}
\end{equation}
where ${\bf h}={\bf H^{-1}p}$ estimates the interpolated map of the lake, and
where ${\bf L}$ is a drift modeling operator (leaky integration),
${\bf q}$ is an additional variable to be estimated,
and $\lambda$ is a balancing constant to be discussed.
We then minimize the misfit function,
\begin{equation}
  g_2(\bold p,\bold q) = \|\bold r_d\|^2+\epsilon_1^2\|\bold r_p\|^2+\epsilon_2^2\|\bold r_q\|^2,
\end{equation}
Now the data $\bold d$ is being modeled in two ways
by two parts which add,
a geological map part $\bold B \bold {H^{-1}} \bold p$
and a recording system drift part $\lambda\bold L \bold q$.
Clearly, by adjusting the balance of
$\epsilon_1$ to
$\epsilon_2$ we are forcing the data to go one way or the other.
There is nothing in the data itself that says which part of
the theory should claim it.


\activeplot{antoine4}{width=6in,height=8.0in}{}{
	Top left: Estimated ${\bf p}$ without attenuation of the
	tracks, i.e., regression (\ref{eq3}). Top right: Estimated ${\bf p}$ 
	with the derivative along the tracks, i.e., regression (\ref{eq5}). 
	Bottom left:  Estimated ${\bf p}$ without 
	tracks, i.e., regression (\ref{eq4}). 
%	S points to a linear, noisy feature that is not removed by the $\ell^1$ norm or the modeling of the tracks.
	Bottom right:
	recorder drift in model space $\bold B'\bold L\bold q$.
}

\par
It is a customary matter of practice to forget the two $\epsilon$s
and play with the $\lambda$.
If we kept the two $\epsilon$s,
the choice of $\lambda$ would be irrelevant to the final result.
Since we are going to truncate the iteration,
choice of $\lambda$ matters.
It chooses how much data energy goes into the equipment drift function
and how much into topography.
Antoine ended out with with $\lambda=0.08$.

\par
There is another parameter to adjust.
The parameter $\rho$ controlling the decay of the leaky integration. 
Antoine found that value ${\rho=0.99}$ was a suitable compromise.
Taking $\rho$ smaller allows the track drift to vary too rapidly thus
falsifying data in a way that falsifies topography.
Taking $\rho$ closer to unity does not allow adequately rapid variation
of the data acquistion system
thereby pushing acquisition tracks into the topography.

\par
Figure \ref{fig:antoine4}
(bottom-left corner)
shows the estimated roughened image ${\bf p}$
with $\lambda\bold L$ data-drift modeling and
(top-left corner) ${\bf p}$ without it.
Data-drift modeling (bottom-left)
yields an image that is essentially track-free without loss of detail.
Top right shows the poor result of applying 
the derivative $d \over ds$ along the tracks.
Tracks are removed but the topography is unclear.
%We show a linear feature that remains after we remove most of the spikes and the tracks. This event probably corresponds to a streak of bad data points that are incorrectly attenuated by our inversion schemes.
%Finally, looking at the attenuated tracks in Figure
%\ref{fig:antoine4} (bottom-right corner) we see that
%no geological feature as leaked in. 

\par
The bottom-right part of
Figure \ref{fig:antoine4} provides important diagnostic information.
The estimated instrumentation drift $\bold L\bold q$ has been transformed
to model space $\bold B'\bold L\bold q$.
We do not like to see hints of geology in this space but we do.
Adjusting $\lambda$ or $\rho$ we can get rid of the geology here,
but then survey tracks will appear in the lake image.
The issue of decomposing data into signal and noise parts
is dealt with further in chapter \ref{paper:pch}.

\par
Figures \ref{fig:antoine5} and \ref{fig:antoine6} show
selected segments of data space.
Examining here the discrepancy between observed data and modeled data
offers us an opportunity to get new ideas.
The top plot is the input data $\bold{d}$.
Next is the estimated noise-free data ${\bf BH^{-1}p}$.
Then the estimated secular variations $\lambda \bold L \bold q$.
Finally residual
$\bold B \bold {H^{-1}} \bold p + \lambda \bold L \bold q - \bold d$
after a suitable number of iterations.
The modeled data in both Figures \ref{fig:antoine5}b and 
\ref{fig:antoine6}b show no remaining spikes.

\activeplot{antoine5}{width=\textwidth,height=3in}{}{
	(a) Track 17 (input data) in Figure \ref{fig:antoine1}. 
	(b) The estimated noise-free data  ${\bf BH^{-1}p}$.
	(c) Estimated drift $\bold L\bold q$.
	(d) Data residual.
	}
\activeplot{antoine6}{width=\textwidth,height=3in}{}{ 
	(a) Track 14 (input data) in Figure \ref{fig:antoine1}. 
	(b) Modeled data, ${\bf BH^{-1}p}$.
	(c) Estimated drift.
	(d) Data-space residual.
	}

\par
The estimated instrument drift is reasonable, mostly under a meter
for measurments with a nominal precision of 10 cm.
There are some obvious problems though.
It is not a serious problem that the drift signal is always positive.
Applying the track derivative means that zero frequency is in the null space.
An arbitrary constant may be moved
from water depth to track calibration.
More seriously, the track calibration fluctuates
more rapidly than we might imagine.
Worse still, Figure \ref{fig:antoine6}c shows
the instrument drift correlates with water depth(!).
This suggests we should have a slower drift function
(bigger $\rho$ or weaker $\lambda$), but Antoine assures me that
this would push data acquisition tracks into the lake image.
If the data set had included the date-time of each measurement
we would have been better able to model drift.
Instead of allowing a certain small change of drift with each measurement,
we could have allowed a small change
in proportion to the time since the previous measurement.

\par
An interesting feature of the data residual in Figure \ref{fig:antoine6}d is
that it has more variance in deep water than in shallow.
Perhaps the depth sounder has insufficient power for deeper water
or for the softer sediments found in deeper water.
On the other hand, this enhanced deep water variance
is not seen in Figure~\ref{fig:antoine5}d which is puzzling.
Perhaps the sea was rough for one day of recording but not another.


%\par
%WHAT TO KEEP FROM THIS PARAGRAPH?
%Looking closely at the residual (Figure
%\ref{fig:antoine6}d), we notice the drift is large where
%the data are noisy (Figure \ref{fig:antoine6}a).  
%It is possible that the day of acquisition was very windy, common 
%weather condition for the Sea of Galilee.
%Thus, the wind forces the water to pile up on one
%side of the lake, which can explain the lower water level on the other side.
%In addition, the strong wind in the middle of the lake induces noisy 
%measurements because of the waves,
%and of the ship's erratic movement.
%Perhaps the depth sounder is more precise in shallow water than in deep.
%This could explain the shape and amplitude of the estimated drift in Figure
%\ref{fig:antoine6}c.


















%\begin{notforlecture}
\subsection{Regridding}
\sx{regridding}

%Because of the weighting $\bold W$,
%which is a function of the residual itself,
%the fitting problems (\ref{eqn:potato}) and (\ref{eqn:pear}) are nonlinear.
%Thus a nonlinear solver is required.
%Unlike linear solvers,
%nonlinear solvers need a good starting approximation
%so they do not land
%in a false minimum.
%(Linear solvers benefit too by
%converging more rapidly when started from a good approximation.)
%I chose the starting solution $\bold h_0$
%beginning from median binning on a coarse mesh.
%Then I refined the mesh with linear interpolation.

\par
We often have an image $\bold h_0$
on a coarse mesh that we would like to use on a refined mesh.
This regridding chore reoccurs on many occasions
so I present reusable code.
When a continuum is being mapped to a mesh,
it is best to allocate to each mesh point
an equal area on the continuum.
Thus we take an equal interval between each point,
and a half an interval beyond the end points.
Given {\tt n} points,
there are {\tt n-1} intervals between them,
so we have
\par\noindent\begin{verbatim}
          min = o - d/2
          max = o + d/2 + (n-1)*d
\end{verbatim}
\par\noindent
which may be back solved to
\par\noindent\begin{verbatim}
          d = (max-min)/n
          o = (min*(n-.5) + max/2)/n
\end{verbatim}
\par\noindent
which is a memorable result for {\tt d} and a less memorable one for {\tt o}.
With these not-quite-trivial results, we can invoke
the linear interpolation operator \texttt{lint2}.
It is designed for data points at irregular locations,
but we can use it for regular locations too.
Operator \texttt{refine2} defines pseudoirregular coordinates
for the bin centers on the fine mesh
and then invokes \texttt{lint2} to
carry data values from the coarse regular mesh to
the pseudoirregular finer mesh.
Upon exiting from \texttt{refine2},
the data space (normally irregular)
is a model space (always regular) on the finer mesh.
\opdex{refine2}{refine 2-D mesh}
Finally, here is the 2-D linear interpolation operator \texttt{lint2},
which is a trivial extension of the 1-D version \texttt{lint1} \vpageref{/prog:lint1}.
\opdex{lint2}{2-D linear interpolation}

