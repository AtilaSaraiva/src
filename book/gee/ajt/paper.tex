% copyright (c) 2001 Jon Claerbout

\title{Basic operators and adjoints}
\author{Jon Claerbout}
\maketitle

\label{paper:ajt}

A great many of the calculations
we do in science and engineering
are really matrix multiplication in disguise.
The first goal of this chapter is to unmask the disguise
by showing many examples.
Second, we see how the 
\bx{adjoint} operator (matrix transpose)
back projects information from data to the underlying model.

\par
Geophysical modeling calculations
generally use linear operators that predict data from models.
Our usual task is to find the inverse of these calculations;
i.e., to find models (or make images) from the data.
Logically, the adjoint is the first step
and a part of all subsequent steps in this \bx{inversion} process.
Surprisingly, in practice the adjoint sometimes does a better job
than the inverse!
This is because the adjoint operator tolerates imperfections
in the data and does not demand that the data provide full information.

\par
Using the methods of this chapter,
you will find that
once you grasp the relationship between operators in general
and their adjoints,
you can obtain the adjoint just
as soon as you have learned how to code
the modeling operator.

\par
If you will permit me a poet's license with words,
I will offer you the following table
of \bx{operator}s and their \bx{adjoint}s: 

\begin{tabular}{p{2em}lp{1em}l}
&\bx{matrix multiply}          &&conjugate-transpose matrix multiply \\
&convolve                      &&crosscorrelate        \\
&truncate                      &&zero pad      \\
&replicate, scatter, spray     &&sum or stack  \\
&spray into neighborhoods      &&sum within bins   \\
&derivative (slope)            &&negative derivative   \\
&causal integration            &&anticausal integration        \\
&add functions                 &&do integrals  \\
&assignment statements         &&added terms   \\
&plane-wave superposition      &&slant stack / beam form       \\
&superpose curves              &&sum along a curve \\
&stretch                       &&squeeze       \\
&scalar field gradient         &&negative of vector field divergence       \\
&upward continue               &&downward continue \\
&diffraction modeling          &&imaging by migration  \\
&hyperbola modeling            &&CDP stacking  \\
&ray tracing                   &&\bx{tomography}       \\
\end{tabular}

\par
The left column above is often called ``\bx{modeling},''
and the adjoint operators on the right are often
used in ``data \bx{processing}.''
\par
When the adjoint operator is
{\em  not}
an adequate approximation to the inverse,
then you apply the techniques of fitting and optimization
explained in Chapter \ref{paper:lsq}.
These techniques require iterative use of the
modeling operator and its adjoint.

\par
The adjoint operator is sometimes called
the ``\bx{back projection}'' operator
because information propagated in one direction (earth to data) is projected
backward (data to earth model).
Using complex-valued operators,
the transpose and complex conjugate go together;
and in \bx{Fourier analysis}, taking the complex conjugate
of $\exp(i\omega t)$ reverses the sense of time.
With more poetic license, I say that adjoint operators
{\em  undo}
the time and phase shifts of modeling operators.
The inverse operator does this too,
but it also divides out the color.
For example, when linear interpolation is done,
then high frequencies are smoothed out,
so inverse interpolation must restore them.
You can imagine the possibilities for noise amplification.
That is why adjoints are safer than inverses.
But nature determines in each application what is the best operator to use,
and whether to stop after the adjoint,
to go the whole way to the inverse,
or to stop partway.

\par 
The operators and adjoints above transform vectors to other vectors.
They also transform data planes to model planes, volumes, etc.
A mathematical operator transforms an ``abstract vector'' which
might be packed full of volumes of information like television
signals (time series) can pack together a movie, a sequence of frames.
We can always think of the operator as being a matrix
but the matrix can be truly huge (and nearly empty).
When the vectors transformed by the matrices are large like
geophysical data set sizes
then the matrix sizes are ``large squared,''
far too big for computers.
Thus although we can always think of an operator as a matrix,
in practice, we handle an operator differently.
Each practical application requires the practitioner to
prepare two computer programs.
One performs the matrix multiply
$\bold y =\bold B \bold x$
and another multiplys by the transpose
$\tilde{\bold x} =\bold B\T \bold y$
(without ever having the matrix itself in memory).
It is always easy to transpose a matrix.
It is less easy to take a computer program that does
$\bold y =\bold B \bold x$
and convert it to another to do
$\tilde{\bold x} =\bold B\T \bold y$.
In this chapter are many examples of increasing complexity.
At the end of the chapter we will see a test for any program pair
to see whether the operators $\bold B$ and $\bold B\T$ are mutually adjoint
as they should be.
Doing the job correctly (coding adjoints without making approximations)
will reward us later when we tackle model and image estimation applications.

\par
Mathematicians often denote the transpose of a matrix
$\bold B$ by $\bold B^{\rm T}$.
In physics and engineering we often encounter complex numbers.
There the adjoint is the complex-conjugate transposed matrix
denoted $\bold B^\ast$.
%In this book, the adjoint is simply written as $\bold F\T$.
What this book calls the adjoint is more properly called the Hilbert adjoint.

%\par
%We will see that computation of the \bx{adjoint}
%is a straightforward adjunct to the operator computation itself,
%and the computed adjoint should be, and generally can be,
%exact (within machine precision).
%We will see
%if the application's operator is computed in an approximate way,
%that it is natural and best that the adjoint
%be computed with adjoint approximations.
%
%Much later in this chapter is a formal definition of adjoint operator.
%Throughout the chapter we handle an adjoint operator as a matrix transpose,
%but we hardly ever write down any matrices or their transposes.
%Instead, we always prepare two routines,
%one that performs 
%$\bold y =\bold A \bold x$
%and another that performs
%$\tilde{\bold x} =\bold A' \bold y$.
%So we need a test that the two routines
%really embody the essential aspects of matrix transposition.
%Although the test is an elegant and useful test
%and is itself a fundamental definition,
%curiously,
%that definition
%%helps us not one bit in constructing
%does not help construct
%adjoint operators,
%so we postpone a formal definition of adjoint
%until after we have seen many examples.


\subsection{Programming linear operators}
\sx{matrix multiply}
The operation $ y_i = \sum_j b_{ij} x_j$ is the multiplication
of a matrix $\bold B$ by a vector $\bold x$.
The adjoint operation is
$\tilde x_j = \sum_i b_{ij} y_i$.
The operation adjoint to multiplication by a matrix
is multiplication by the transposed matrix
(unless the matrix has complex elements,
in which case we need the complex-conjugated transpose).
The following
\bx{pseudocode}
does matrix multiplication
$\bold y=\bold B\bold x$ and multiplication by the transpose
$\tilde{\bold x} = \bold{B}\T \bold{y}$:
\par\noindent
\vbox{\begin{tabbing}
indent \= atechars \= atechars \=  atechars \= \kill
\>if adjoint    \\
\>      \>then erase x  \\
\>if operator itself    \\
\>      \>then erase y  \\
\>do iy = 1, ny \{      \\
\>do ix = 1, nx \{      \\
\>      \>if adjoint    \\
\>      \>      \>x(ix) = x(ix) + b(iy,ix) $\times$ y(iy)       \\
\>      \>if operator itself    \\
\>      \>      \>y(iy) = y(iy) + b(iy,ix) $\times$ x(ix)       \\
\>      \>\}\}
\end{tabbing} }
\par\noindent
Notice that the ``bottom line'' in the program is that $x$ and $y$
are simply interchanged.
The above example is a prototype of many to follow,
so observe carefully the similarities and differences between
the adjoint and the operator itself.

\par
Next we restate the matrix-multiply pseudo code in real code.
\begin{comment}
in a language called \bx{Loptran}\footnote{
        The programming language, Loptran,
        is based on a dialect of Fortran called Ratfor.
        For more details, see Appendix A.},
a language designed for
exposition and research
in model fitting and optimization in physical sciences.
\end{comment}

The module \texttt{matmult}
for matrix multiply and its adjoint
exhibits the style that we will use repeatedly.
At last count there were 53 such routines
(operator with adjoint)
in this book alone.
\opdex{matmult}{matrix multiply}{33}{45}{user/pwd}        
\begin{comment}
Notice that the module \texttt{matmult}
does not explicitly erase its output before it begins,
as does the pseudo code.
That is because Loptran will always erase for you
the space required for the operator's output.
Loptran also defines a logical variable {\tt adj}
for you to distinguish your computation of the adjoint
$\bold x=\bold x+\bold B\T \bold y$
from the forward operation
$\bold y=\bold y+\bold B\bold x$. 
In computerese, the two lines beginning \#\% are macro expansions
that take compact bits of information which expand
into the verbose boilerplate that Fortran requires.
Loptran is Fortran with these macro expansions.
You can always see how they expand by looking at
\url{http://sep.stanford.edu/sep/prof/}.

\par
What is new in Fortran 90, and will be a big help to us,
is that instead of a subroutine with a single entry.
\end{comment}

We have a module with two entries,
one named {\tt \_init} 
for the physical scientist who defines the physical problem by
defining the matrix, and
another named {\tt \_lop} 
for the least-squares problem solver,
the computer scientist who will not be interested
in how we specify $\bold B$, but who will be iteratively computing
$\bold B\bold x$ and $\bold B\T \bold y$
to optimize the model fitting.
\begin{comment}
The lines beginning with {\tt {\#}{\%}} are expanded by Loptran into
more verbose and distracting Fortran 90 code.
The second line in the module \texttt{matmult},
however,
is pure Fortran syntax saying that
{\tt bb} is a pointer to a real-valued matrix.
\end{comment}
\par
To use \texttt{matmult}, two calls must be made,
the first one
\begin{verbatim}
           matmult_init( bb);
\end{verbatim}
is done by the physical scientist after he or she has prepared the matrix.
Most later calls will be done by numerical analysts
in solving code like in Chapter \ref{paper:lsq}.
These calls look like
\begin{verbatim}
            matmult_lop( adj, add, nx, ny, x, y);
\end{verbatim}
where {\tt adj} is the logical variable saying whether we desire
the adjoint or the operator itself,
and where {\tt add} is a logical variable saying
whether we want to accumulate like
$\bold y \leftarrow \bold y+\bold B\bold x$
or whether we want to erase first and thus do
$\bold y \leftarrow \bold B\bold x$.
\begin{comment}
The return value {\tt stat} is an integer parameter,
mostly useless (unless you want to use it for error codes). 
\par
Operator initialization often allocates memory.
To release this memory, you can {\tt call matmult\_close()}
although in this case nothing really happens.
\end{comment}
\par
We split operators into two independent processes,
the first is used for geophysical set up
while the second is invoked by mathematical library code
(introduced in the next chapter)
to find the model that best fits the data.
%in other words, to find the zero-offset trace which sprays
%out to make the best approximations to all the other traces.
Here is why we do so.
It is important that the math code contain nothing about
the geophysical particulars.  This enables us to use
the same math code on many different geophysical applications.
This concept of ``information hiding'' arrived late in
human understanding of what is desirable in a computer language.
\begin{comment}
This feature alone is valuable enough to warrant
upgrading from Fortran 77 to Fortran 90, and likewise from C to C++.
\end{comment}
Subroutines and functions are the way that new programs use old ones.
Object modules are the way that old programs (math solvers)
are able to use new ones (geophysical operators).


\section{FAMILIAR OPERATORS}
The simplest and most fundamental linear operators
arise when a matrix operator reduces to a simple row or a column.
\par\noindent
A {\bf row} \quad\ \ is a summation operation.
\par\noindent
A {\bf column}       is an impulse response.
\vspace{.2in}
\par\noindent
If the inner loop of a matrix multiply ranges within a
\par\noindent
{\bf row,} \quad\ \ the operator is called {\em  sum} or {\em  pull}.
\par\noindent
{\bf column,}       the operator is called {\em  spray} or {\em  push}.
\par\noindent
Generally, inputs and outputs will be high dimensional such as signals or images.
Push gives ugly outputs.   Some output locations will be empty,
each having an erratic number of contributions to it.
Consequently, most data processing (adjoint) is done by {\em pull}.

\par
A basic aspect of adjointness is that the
adjoint of a row matrix operator is a column matrix operator.
For example,
the row operator $[a,b]$
\begin{equation}
y \eq
\left[ \ a \ b \ \right] 
\left[
\begin{array}{l}
        x_1 \\
        x_2
\end{array}
\right] 
\eq
a x_1 + b x_2
\end{equation}
has an adjoint that is two assignments:
\begin{equation}
        \left[
        \begin{array}{l}
                \hat x_1 \\
                \hat x_2
        \end{array}
        \right]
        \eq
        \left[
        \begin{array}{l}
                a \\
                b
        \end{array}
        \right]
        \ y
\end{equation}
\par
\boxit{
The adjoint of a sum of $N$ terms
is a collection of $N$ assignments.
}

\subsection{Adjoint derivative}
In numerical analysis we represent the derivative of a time function
by a finite difference.
We do this by subtracting each two neighboring time points and then
dividing by the sample interval $\Delta t$.
This amounts to convolution with the filter $(1,-1)/\Delta t$.
Omitting the $\Delta t$ we express this concept as:
\begin{equation}
\left[ \begin{array}{c}
        y_1 \\
        y_2 \\
        y_3 \\
        y_4 \\
        y_5 \\
        y_6
        \end{array} \right]
\eq
\left[ \begin{array}{cccccc}
        -1& 1& .& .& .& . \\
         .&-1& 1& .& .& . \\
         .& .&-1& 1& .& . \\
         .& .& .&-1& 1& . \\
         .& .& .& .&-1& 1 \\
         .& .& .& .& .& 0
        \end{array} \right] \ 
\left[ \begin{array}{c}
        x_1 \\
        x_2 \\
        x_3 \\
        x_4 \\
        x_5 \\
        x_6
        \end{array} \right]
 \label{eqn:ruff1}
\end{equation}
\par
The \bx{filter impulse response} is seen in any column
in the middle of the matrix, namely $(1,-1)$. 
In the transposed matrix,
the filter-impulse response
is time-reversed to $(-1,1)$.
So, mathematically,
we can say that the adjoint of the time derivative operation
is the negative time derivative.
This corresponds also to the fact that
the complex conjugate of $-i\omega$ is $i\omega$.
We can also speak of the adjoint of the boundary conditions:
we might say that the adjoint of ``no boundary condition''
is a ``specified value'' boundary condition.
The last row in equation~(\ref{eqn:ruff1}) is optional.
%and depends not on the code shown, but the code that invokes it.
It may seem unnatural to append a null row, but it can be a small
convenience (when plotting) to have the input and output be the same size.
\par

Equation~(\ref{eqn:ruff1}) is implemented by the code
in module \texttt{igrad1}
which does the operator itself (the forward operator)
and its adjoint.
\opdex{igrad1}{first difference}{25}{40}{api/c}

\par \noindent
The adjoint code may seem strange.
It might seem more natural to code the adjoint
to be the negative of the operator itself
and then make the special adjustments for the boundaries.
The code given, however, is correct and requires no adjustments
at the ends.
To see why, notice for each value of \texttt{i},
the operator itself handles one row of
equation~(\ref{eqn:ruff1})
while for each \texttt{i}
the adjoint handles one column.
That's why coding the adjoint in this way
does not require any special work on the ends.
The present method of coding reminds us that
the adjoint of a sum of $N$ terms is a collection of $N$ assignments.
Think of the meaning of
$y_i=y_i+a_{i,j}x_j$
for any particular $i$ and $j$.
The adjoint simply accumulates that same value of $a_{i,j}$
going the other direction
$x_j=x_j+a_{i,j}y_i$.

%A complicated way to think about the adjoint of equation
%(\ref{eqn:ruff1}) is to note that it is the negative of the derivative
%and that something must be done about the ends.
%A simpler way to think about it
%is to apply the idea that the adjoint of a sum of $N$ terms
%is a collection of $N$ assignments.

\begin{comment}
\par
The Ratfor90 dialect of Fortran
allows us to write the inner code of
the \texttt{igrad1} module more simply and symmetrically
using the syntax of C, C++, and Java
where
expressions like {\tt a=a+b} can be written more tersely as {\tt a+=b}.
With this, the heart of module \texttt{igrad1} becomes
\begin{verbatim}
if( adj) {   xx(i+1) += yy(i)
             xx(i)   -= yy(i)
           }
else {       yy(i)   += xx(i+1)
             yy(i)   -= xx(i)
           }
\end{verbatim}
where we see that each component of the matrix is handled both by the
operator and the adjoint.  Think about the forward operator
``pulling'' a sum into {\tt yy(i)}, and think about the adjoint
operator ``pushing'' or ``spraying'' the impulse {\tt yy(i)} back into
{\tt xx()}.
\end{comment}

%Something odd happens at the ends of the adjoint only if we take the
%perspective that the adjoint should have been computed one component
%at a time instead of all together.
%By not taking that view, we avoid that confusion.

\inputdir{igrad1}

\par
Figure \ref{fig:stangrad} illustrates the use of module
\texttt{igrad1} for each north-south line of a topographic map.  We
observe that the gradient gives an impression of illumination from a
low sun angle.

\plot{stangrad}{width=6in,height=8.4in}{ Topography
  near Stanford (top) southward slope (bottom).  }

To apply \texttt{igrad1} along the 1-axis for each point on the 2-axis
of a two-dimensional map, we use the loop
\begin{verbatim}
for (iy=0; iy < ny; iy++) 
     igrad1_lop(adj, add, nx, nx, map[iy], ruf[iy]);
\end{verbatim}
\begin{comment}
On the other hand, to see the east-west gradient, we use the loop
\begin{verbatim}
do ix=1,nx 
      stat = igrad1_lop( adj, add, map(ix,:), ruf(ix,:))
\end{verbatim}
\end{comment}

%\par
%The C/C++/Java family of computer languages allows
%an abbreviation that is handy for all operator routines;
%it's so handy that Bob Clapp has installed it in our version of Ratfor90.
%The essential feature of operator calculations is summing
%and spraying and both are rendered in Fortran as ``A=A+B''.
%In the C/C++/Java languages A=A+B may be abbreviated A+=B;
%for example,
%{\tt xx(i+1) = xx(i+1) + yy(i)} reduces to {\tt xx(i+1) += yy(i)}.
%The more complicated the subscripting,
%the more desirable the abbreviation.
%Also, the abbreviation makes more transparent the nature
%of operator-adjoint pairs.


\subsection{Transient convolution}
\sx{convolution}
\sx{convolution ! transient}

\par
The next operator we examine is convolution.
It arises in many applications; and it could be derived in many ways.
A basic derivation is from the multiplication of two polynomials, say
$X(Z) = x_1 + x_2 Z + x_3 Z^2 + x_4 Z^3 + x_5 Z^4 + x_6 Z^5$ times
$B(Z) = b_1 + b_2 Z + b_3 Z^2 + b_4 Z^3$.\footnote{
	This book is more involved with matrices than with Fourier analysis.
	If it were more Fourier analysis we would choose notation
	to begin subscripts from zero like this:
	$B(Z) = b_0 + b_1 Z + b_2 Z^2 + b_3 Z^3$.}
Identifying the $k$-th power of $Z$ in the product
$Y(Z)=B(Z)X(Z)$ gives the $k$-th row of the convolution transformation
(\ref{eqn:contran1}).
%Matrix multiplication and transpose multiplication still
%fit easily in the same computational framework
%when the matrix has a special form, such as

\begin{equation}
\bold y \eq
\left[ 
\begin{array}{c}
  y_1 \\ 
  y_2 \\ 
  y_3 \\ 
  y_4 \\ 
  y_5 \\ 
  y_6 \\ 
  y_7 \\ 
  y_8
  \end{array} \right] 
\eq
\left[ 
\begin{array}{cccccc}
  b_1 & 0   & 0    & 0   & 0   & 0   \\
  b_2 & b_1 & 0    & 0   & 0   & 0   \\
  b_3 & b_2 & b_1  & 0   & 0   & 0   \\
  0   & b_3 & b_2  & b_1 & 0   & 0   \\
  0   & 0   & b_3  & b_2 & b_1 & 0   \\
  0   & 0   & 0    & b_3 & b_2 & b_1 \\
  0   & 0   & 0    & 0   & b_3 & b_2 \\
  0   & 0   & 0    & 0   & 0   & b_3 
  \end{array} \right] 
\; \left[ 
\begin{array}{c}
  x_1 \\ 
  x_2 \\ 
  x_3 \\ 
  x_4 \\ 
  x_5 \\ 
  x_6
  \end{array} \right]
\eq \bold B \bold x
\label{eqn:contran1}
\end{equation}
Notice that columns of
equation (\ref{eqn:contran1})
all contain the same signal,
but with different shifts.
This signal is called the filter's impulse response.


\par
Equation~(\ref{eqn:contran1}) could be rewritten as
\begin{equation}
\bold y \eq
\left[ 
\begin{array}{c}
  y_1 \\ 
  y_2 \\ 
  y_3 \\ 
  y_4 \\ 
  y_5 \\ 
  y_6 \\ 
  y_7 \\ 
  y_8
  \end{array} \right] 
\eq
\left[ 
\begin{array}{ccc}
  x_1 & 0   & 0    \\
  x_2 & x_1 & 0    \\
  x_3 & x_2 & x_1  \\
  x_4 & x_3 & x_2  \\
  x_5 & x_4 & x_3  \\
  x_6 & x_5 & x_4  \\
  0   & x_6 & x_5  \\
  0   & 0   & x_6
  \end{array} \right] 
\; \left[ 
\begin{array}{c}
  b_1 \\ 
  b_2 \\ 
  b_3 \end{array} \right]
  \eq \bold X \bold b
\label{eqn:contran2}
\end{equation}
In applications we can choose between
$\bold y = \bold X \bold b$
and
$\bold y = \bold B \bold x$.
In one case
the output $\bold y$
is dual to the filter $\bold b$,
and in the other case
the output $\bold y$
is dual to the input $\bold x$.
Sometimes we must solve
for $\bold b$ and sometimes for $\bold x$;
so sometimes we use equation~(\ref{eqn:contran2}) and 
sometimes (\ref{eqn:contran1}).
Such solutions begin from the adjoints.
The adjoint of (\ref{eqn:contran1}) is
\begin{equation}
\left[ 
\begin{array}{c}
\hat  x_1 \\ 
\hat  x_2 \\ 
\hat  x_3 \\ 
\hat  x_4 \\ 
\hat  x_5 \\ 
\hat  x_6
  \end{array} \right] 
\eq
\left[ 
\begin{array}{cccccccc}
  b_1 & b_2 & b_3 & 0   & 0   & 0   & 0   & 0  \\
  0   & b_1 & b_2 & b_3 & 0   & 0   & 0   & 0  \\
  0   & 0   & b_1 & b_2 & b_3 & 0   & 0   & 0  \\
  0   & 0   & 0   & b_1 & b_2 & b_3 & 0   & 0  \\
  0   & 0   & 0   & 0   & b_1 & b_2 & b_3 & 0  \\
  0   & 0   & 0   & 0   & 0   & b_1 & b_2 & b_3 
  \end{array} \right] 
\; \left[ 
\begin{array}{c}
  y_1 \\ 
  y_2 \\ 
  y_3 \\ 
  y_4 \\ 
  y_5 \\ 
  y_6 \\ 
  y_7 \\ 
  y_8 \end{array} \right]
\label{eqn:ajtcontran1}
\end{equation}
The adjoint {\em  \bx{crosscorrelate}s} with the filter
instead of convolving with it (because the filter is backwards).
Notice that each row in
equation (\ref{eqn:ajtcontran1}) contains all the filter coefficients
and there are no rows where the filter somehow uses zero values
off the ends of the data as we saw earlier.
In some applications it is important not to assume zero values
beyond the interval where inputs are given.
\par
The adjoint of (\ref{eqn:contran2}) crosscorrelates a fixed portion
of filter input across a variable portion of filter output.
\begin{equation}
\left[ 
\begin{array}{c}
\hat  b_1 \\ 
\hat  b_2 \\ 
\hat  b_3
  \end{array} \right] 
\eq
\left[ 
\begin{array}{cccccccc}
  x_1 & x_2 & x_3 & x_4 & x_5 & x_6 & 0   & 0   \\
  0   & x_1 & x_2 & x_3 & x_4 & x_5 & x_6 & 0   \\
  0   & 0   & x_1 & x_2 & x_3 & x_4 & x_5 & x_6   
  \end{array} \right] 
\; \left[ 
\begin{array}{c}
  y_1 \\ 
  y_2 \\ 
  y_3 \\ 
  y_4 \\ 
  y_5 \\ 
  y_6 \\ 
  y_7 \\ 
  y_8
  \end{array} \right]
\label{eqn:ajtcontran2}
\end{equation}
\par
Module \texttt{tcai1}
is used for
$\bold y = \bold B \bold x$
and module~\texttt{tcaf1}
is used for
$\bold y = \bold X \bold b$.
\opdex{tcai1}{transient convolution}{45}{50}{user/gee}
\opdex{tcaf1}{transient convolution}{45}{50}{user/gee}

\par
The polynomials 
$X(Z)$,
$B(Z)$, and
$Y(Z)$
are called $Z$ transforms.
An important fact in real life
(but not important here)
is that the $Z$ transforms are
Fourier transforms in disguise.
Each polynomial is a sum of terms
and the sum
amounts to a Fourier sum when we take $Z=e^{i\omega\Delta t}$.
The very expression
$Y(Z)=B(Z)X(Z)$ says that a product in the frequency domain
($Z$ has a numerical value) is a convolution in the time domain
(that's how we multipy polynomials, convolve their coefficients).

%The length of the output of \bx{transient convolution}
%\sx{convolution ! transient}
%is one less than the length of the filter plus the length of the input.
%Notice, however,
%that in these modules,
%the input length plus the filter length equals the output length,
%because these routines erase the extra memory cell.
%So you must be sure you have allocated that memory cell.
%I erase the extra cell to reduce clutter
%in the many 2-D and 3-D modules
%that build upon transient convolution.
%It is important that the extra memory cell be erased and not simply ignored,
%because otherwise it may contain an impossible value
%that would break many modules such as plot routines and inversion routines.
 
\subsection{Internal convolution}
%\begin{notforlecture}
\sx{convolution ! internal}
Convolution is the computational equivalent of
ordinary linear differential operators (with constant coefficients).
Applications are vast,
and end effects are important.
Another choice of data handling at ends
is that zero data not be assumed
beyond the interval where the data is given.
This is important in data where the crosscorrelation changes with time.
Then it is sometimes handled as constant in short time windows.
Care must be taken that zero signal values not be presumed
off the ends of those short time windows;
otherwise,
the many ends of the many short segments
can overwhelm the results.
\par
In the sets (\ref{eqn:contran1}) and (\ref{eqn:contran2}),
the top two equations explicitly assume that the input data vanishes
before the interval on which it is given, and likewise at the bottom.
Abandoning the top two and bottom two equations in (\ref{eqn:contran2})
we get:
\begin{equation}
\left[ 
\begin{array}{c}
  y_3 \\ 
  y_4 \\ 
  y_5 \\ 
  y_6 
  \end{array} \right] 
\eq
\left[ 
\begin{array}{ccc}
  x_3 & x_2 & x_1  \\
  x_4 & x_3 & x_2  \\
  x_5 & x_4 & x_3  \\
  x_6 & x_5 & x_4 
  \end{array} \right] 
\; \left[ 
\begin{array}{c}
  b_1 \\ 
  b_2 \\ 
  b_3 \end{array} \right]
\label{eqn:trunccontran2}
\end{equation}
The adjoint is
\begin{equation}
\left[ 
\begin{array}{c}
\hat  b_1 \\ 
\hat  b_2 \\ 
\hat  b_3
  \end{array} \right] 
\eq
\left[ 
\begin{array}{ccccc}
  x_3 & x_4 & x_5 & x_6  \\
  x_2 & x_3 & x_4 & x_5  \\
  x_1 & x_2 & x_3 & x_4 
  \end{array} \right] 
\; \left[ 
\begin{array}{c}
  y_3 \\ 
  y_4 \\ 
  y_5 \\ 
  y_6 
  \end{array} \right]
\label{eqn:ajtintcontran2}
\end{equation}
The difference between (\ref{eqn:ajtintcontran2})
and (\ref{eqn:ajtcontran2}) is that here 
the adjoint crosscorrelates a fixed portion of {\em  output}
across a variable portion of {\em  input},
whereas with (\ref{eqn:ajtcontran2})
the adjoint crosscorrelates a fixed portion of {\em  input}
across a variable portion of {\em  output}.
\par
In practice we typically allocate equal space for input and output.
Because the output is shorter than the input,
it could slide around in its allocated space,
so its location is specified by an additional parameter called its {\tt lag}.
%The statement {\tt x=y-b+lag} in module
%\texttt{icaf1}
%says that the
%output time {\tt y} aligns with the
% input time {\tt x} for the filter point {\tt b=lag}.
\opdex{icaf1}{convolve internal}{45}{50}{user/gee}
The value of {\tt lag} always used in this book is {\tt lag=1}.
For {\tt lag=1} the module \texttt{icaf1}
implements not equation (\ref{eqn:trunccontran2})
but (\ref{eqn:padconvin}):
\begin{equation}
\left[ 
\begin{array}{c}
  y_1 \\ 
  y_2 \\ 
  y_3 \\ 
  y_4 \\ 
  y_5 \\ 
  y_6 
  \end{array} \right] 
\eq
\left[ 
\begin{array}{ccc}
  0   & 0   & 0    \\
  0   & 0   & 0    \\
  x_3 & x_2 & x_1  \\
  x_4 & x_3 & x_2  \\
  x_5 & x_4 & x_3  \\
  x_6 & x_5 & x_4 
  \end{array} \right] 
\; \left[ 
\begin{array}{c}
  b_1 \\ 
  b_2 \\ 
  b_3 \end{array} \right]
\label{eqn:padconvin}
\end{equation}
It may seem a little odd to put the required zeros at the beginning
of the output, but filters are generally designed so that their
strongest coefficient is the first, namely {\tt bb(1)} so
the alignment of input and output in equation (\ref{eqn:padconvin})
is the most common one.

\inputdir{conv}
\par
The
\bx{end effect}s
of the convolution modules are summarized
in Figure~\ref{fig:conv}.%
\sideplot{conv}{width=0.7\textwidth}{
        Example of convolution end-effects.
        From top to bottom:
        input;
        filter;
        output of {\tt tcai1()};
        output of {\tt icaf1()} also with ({\tt lag=1}).
        }
%\end{notforlecture}

\subsection{Zero padding is the transpose of truncation}
Surrounding a dataset by zeros
(\bx{zero pad}ding)
is adjoint to throwing away the extended data
(\bx{truncation}).
Let us see why this is so.
Set a signal in a vector $\bold x$, and
then to make a longer vector $\bold y$,
add some zeros at the end of $\bold x$.
This zero padding can be regarded as the matrix multiplication
\begin{equation}
\bold y\eq
 \left[ 
  \begin{array}{c}
   \bold I \\ 
   \bold 0
  \end{array}
 \right] 
 \ 
 \bold x
\end{equation}
The matrix is simply an identity matrix $\bold I$
above a zero matrix $\bold 0$.
To find the transpose to zero-padding, we now transpose the matrix
and do another matrix multiply:
\begin{equation}
\tilde {\bold x} \eq
 \left[ 
  \begin{array}{cc}
   \bold I & \bold 0
  \end{array}
 \right] 
\ 
\bold y
\end{equation}
So the transpose operation to zero padding data
is simply {\em  truncating} the data back to its original length.
Module \texttt{zpad1} below
pads zeros on both ends of its input.
Modules for two- and three-dimensional padding are in the
library named {\tt zpad2()} and {\tt zpad3()}.
\opdex{zpad1}{zero pad 1-D}{27}{31}{user/gee}


\subsection{Adjoints of products are reverse-ordered products of adjoints}
Here we examine an example of the general idea that
adjoints of products are reverse-ordered products of adjoints.
For this example we use the Fourier transformation.
No details of \bx{Fourier transformation} are given here
and we merely use it as an example of a square matrix $\bold F$.
We denote the complex-conjugate transpose (or \bx{adjoint}) matrix
with a prime,
i.e.,~$\bold F\T$.
The adjoint arises naturally whenever we consider energy.
The statement that Fourier transforms conserve energy is 
$\bold y\T\bold y=\bold x\T\bold x$ where $\bold y= \bold F \bold x$.
Substituting gives $\bold F\T\, \bold F = \bold I$, which shows that
the inverse matrix to Fourier transform
happens to be the complex conjugate of the transpose of $\bold F$.
\par
With Fourier transforms,
\bx{zero pad}ding and \bx{truncation} are especially prevalent.
Most modules transform a dataset of length of $2^n$,
whereas dataset lengths are often of length $m \times 100$.
The practical approach is therefore to pad given data with zeros.
Padding followed by Fourier transformation $\bold F$
can be expressed in matrix algebra as
\begin{equation}
{\rm Program} \eq
\bold F \ 
 \left[ 
  \begin{array}{c}
   \bold I \\ 
   \bold 0
  \end{array}
 \right] 
\end{equation}
According to matrix algebra, the transpose of a product,
say $\bold A \bold B = \bold C$,
is the product $\bold C\T = \bold B\T \bold A\T$ in reverse order.
So the adjoint routine is given by
\begin{equation}
{\rm Program}\T \eq
 \left[ 
  \begin{array}{cc}
   \bold I & \bold 0
  \end{array}
 \right] 
\
\bold F\T
\end{equation}
Thus the adjoint routine
{\em  truncates} the data {\em  after} the inverse Fourier transform.
This concrete example illustrates that common sense often represents
the mathematical abstraction
that adjoints of products are reverse-ordered products of adjoints.
It is also nice to see a formal mathematical notation
for a practical necessity.
Making an approximation need not lead to collapse of all precise analysis.


\subsection{Nearest-neighbor coordinates}
\sx{nearest neighbor coordinates}
In describing physical processes,
we often either specify models as values given on a uniform mesh
or we record data on a uniform mesh.
Typically we have
a function $f$ of time $t$ or depth $z$
and we represent it by {\tt f(iz)}
corresponding to $f(z_i)$ for $i=1,2,3,\ldots, n_z$
where $z_i = z_0+ (i-1)\Delta z$.
We sometimes need to handle depth as
an integer counting variable $i$
and we sometimes need to handle it as
a floating-point variable $z$.
Conversion from the counting variable to the floating-point variable
is exact and is often seen in a computer idiom
such as either of \begin{verbatim}
            for (iz=0; iz < nz; nz++) {   z = z0 + iz * dz; 
            for (i3=0, i3 < n3; i3++) {  x3 = o3 + i3 * d3;
\end{verbatim}
%{\tt
%   \begin{tabbing}  indent \= \kill
%       \> do iz= 1, nz \{   z = z0 + (iz-1) * dz  \\
%       \> do i3= 1, n3 \{  x3 = o3 + (i3-1) * d3
%   \end{tabbing}
%}
The reverse conversion from the floating-point variable
to the counting variable is inexact.
The easiest thing is to place it at the nearest neighbor.
This is done by solving for {\tt iz}, then adding one half,
and then rounding down to the nearest integer.
The familiar computer idioms are:\begin{verbatim}
        iz = 0.5 + ( z - z0) / dz;
        i3 = 0.5 + (x3 - o3) / d3;
\end{verbatim}
A small warning is in order:
People generally use positive counting variables.
If you also include negative ones,
then to get the nearest integer,
you should do your rounding with the C function {\tt floor()}.

\subsection{Data-push binning}
\sx{nearest neighbor binning}
\sx{data-push binning}
\inputdir{galilee}
A most basic data modeling operation is to copy a number from an $(x,y)$-location on a map
to a 1-D survey data track $d(s)$, where $s$ is a coordinate running along a survey track.
This copying proceeds for all $s$.
The track could be along a straight, curved, or arbitrary line.
Let the coordinate $s$ take on integral values.
Then along with the elements $d(s)$ are the coordinates $(x(s),y(s))$ where on the map the data value $d(s)$ would be recorded.
%This operator, copying $d$ values out from the $(x,y)$-plane
%has an inverse with a name commonly known in practice, ``binning''
%so we will use that name, even though the forward modelling operator does the opposite,
%grabs $d(s)$ values from the $(x,y)$ plane while the adjoint adds $d(s)$ values into the $(x,y)$ mesh,
%and the inverse operator divides that by the number of data values in each bin.
\par
Code for the operator is shown
in module \texttt{bin2}.
\opdex{bin2}{push data into bin}{46}{55}{user/gee}
To invert this data modeling operation, to go from $d(s)$ to $(x(s),y(s))$ requires more than the adjoint operator.
Since each bin contains a different number of data values.
After the adjoint operation is performed,
the inverse operator needs to divide the bin sum by the number of data values that landed in the bin.
It is this inversion operator that is generally called binning (although we will use that name here for the modeling operator).
To find the number of data points in a bin,
we can simply apply the adjoint of \texttt{bin2} to pseudo data of all ones.
To capture this idea in an equation,
let $\bold B$ denote the linear operator
in which the bin value is sprayed to the data values.
The inverse operation,
in which the data values in the bin are summed
and divided by the number in the bin, is represented by
\begin{equation}
\bold m \eq {\bf diag}( \bold B\T \bold 1)^{-1} \bold B\T \bold d
\label{eqn:dpbin}
\end{equation}
Empty bins, of course, leave us a problem since we dare not divide by the zero sum they contain.
That we'll address in chapter \ref{paper:iin}.
In Figure \ref{fig:galbin}, the empty bins contain zero values.
\plot{galbin}{width=\textwidth,height=.66\textwidth}{
  Binned depths of the Sea of Galilee.
}

\subsection{Linear interpolation}
\par
\inputdir{XFig}
The \bx{linear interpolation}
operator is much like the binning operator but a little fancier.
When we perform the forward operation, we take each data coordinate
and see which two model bin centers bracket it.
Then we pick up the two bracketing model values
and weight each of them
in proportion to their nearness to the data coordinate,
and add them to get the data value (ordinate).
The adjoint operation is adding a data value
back into the model vector;
using the same two weights,
the adjoint distributes the data ordinate value
between the two nearest bins in the model vector.
For example, suppose we have a data point near each end of the model
and a third data point exactly in the middle.
Then for a model space 6 points long,
as shown in Figure \ref{fig:helgerud},
we have the operator in (\ref{eqn:lintseq}).
\sideplot{helgerud}{width=0.8\textwidth}{
  Uniformly sampled model space
  and irregularly sampled data space corresponding
  to \protect(\ref{eqn:lintseq}).
}
\begin{equation}
\left[ 
\begin{array}{c}
  d_0 \\ 
  d_1 \\ 
  d_2 
  \end{array} \right] 
\quad \approx \quad
\left[ 
\begin{array}{rrrrrr}
   .7 & .3 &  .  & .  & .  & .  \\
   .  & .  &  1  & .  & .  & .  \\
   .  & .  &  .  & .  & .5 & .5 
  \end{array} \right] 
\left[ 
        \begin{array}{c}
          m_0 \\ 
          m_1 \\ 
          m_2 \\ 
          m_3 \\ 
          m_4 \\ 
          m_5
        \end{array}
\right] 
\label{eqn:lintseq}
\end{equation}
The two weights in each row sum to unity.
If a binning operator were used for the same data and model,
the binning operator would contain a ``1.'' in each row.
In one dimension (as here),
data coordinates are often sorted into sequence,
so that the matrix is crudely a diagonal matrix like equation (\ref{eqn:lintseq}).
If the data coordinates covered the model space uniformly,
the adjoint would roughly be the inverse.
Otherwise,
when data values pile up in some places and gaps remain elsewhere,
the adjoint would be far from the inverse.
\par
Module \texttt{lint1} does linear interpolation and its adjoint.
In chapters \ref{paper:iin} and \ref{paper:mda}
we build inverse operators.
\opdex{lint1}{linear interp}{45}{59}{user/gee}

%\begin{notforlecture}
\subsection{Spray and sum : scatter and gather}
\sx{spray}
\sx{scatter}
\sx{gather}
Perhaps the most common operation is the summing of many values
to get one value.
Its adjoint operation takes a single input value
and throws it out to a space of many values.
The \bx{summation operator} is a row vector of ones.
Its adjoint is a column vector of ones.
In one dimension
this operator is almost too easy for us to bother showing a routine.
But it is more interesting in three dimensions,
where we could be summing or spraying on any of three subscripts,
or even summing on some and spraying on others.
In module \texttt{spraysum},
both input and output are taken
to be three-dimensional arrays.
Externally, however, either could be a scalar, vector, plane, or cube.
For example,
the internal array  {\tt xx(n1,1,n3)}
could be externally the matrix  {\tt map(n1,n3)}.
When
module \texttt{spraysum} is given
the input dimensions and output dimensions stated below,
the operations stated alongside are implied.

\begin{tabular}{p{1em}lll}
&{\tt (n1,n2,n3)}  &{\tt (1,1,1)}     &Sum a cube into a value.      \\
&{\tt (1,1,1)}     &{\tt (n1,n2,n3)}  &Spray a value into a cube.\\
&{\tt (n1,1,1)}    &{\tt (n1,n2,1)}   &Spray a column into a matrix.\\
&{\tt (1,n2,1)}    &{\tt (n1,n2,1)}   &Spray a row into a matrix.\\
&{\tt (n1,n2,1)}   &{\tt (n1,n2,n3)}  &Spray a plane into a cube.\\
&{\tt (n1,n2,1)}   &{\tt (n1,1,1)}    &Sum rows of a matrix into a column.\\
&{\tt (n1,n2,1)}   &{\tt (1,n2,1)}    &Sum columns of a matrix into a row.\\
&{\tt (n1,n2,n3)}  &{\tt (n1,n2,n3)}  &Copy and add the whole cube.
\end{tabular}

If an axis is not of unit length on either input or output,
then both lengths must be the same; otherwise, there is an error.
Normally, after (possibly) erasing the output,
we simply loop over all points on each axis, adding the input to the output.
This implements either a copy or an add, depending on the {\tt add} parameter.
It is either a spray, a sum, or a copy,
according to the specified axis lengths.
\opdex{spraysum}{sum and spray}{42}{56}{user/gee}

%\end{notforlecture}

\subsection{Causal and leaky integration}
\sx{causal integration}
\sx{leaky integration}
\sx{integration ! leaky}
\sx{integration ! causal}
\inputdir{causint}
Causal integration is defined as
\begin{equation}
y(t) \eq \int_{-\infty}^t \ x(\tau )\ d\tau 
\end{equation}
Leaky integration is defined as
\begin{equation}
y(t) \eq \int_0^\infty \ x(t-\tau)\ e^{-\alpha\tau} \ d\tau
\end{equation}
As $\alpha \rightarrow 0$, leaky integration becomes causal integration.
The word ``leaky'' comes from electrical circuit theory
where the voltage on a capacitor would be the integral of the current
if the capacitor did not leak electrons.

\par
Sampling the time axis gives a matrix equation that
we should call causal summation,
but we often call it causal integration.
Equation (\ref{eqn:leakinteq}) represents causal integration for $\rho=1$
and leaky integration for $0<\rho <1$.

\begin{equation}
\bold y \eq
  \left[
        \begin{array}{c}
                y_0 \\
                y_1 \\
                y_2 \\
                y_3 \\
                y_4 \\
                y_5 \\
                y_6 
        \end{array}
  \right]
 \quad = \quad
  \left[
        \begin{array}{cccccccc}
             1 &     0 &     0 &     0 &     0 &     0 &     0 \\
        \rho   &     1 &     0 &     0 &     0 &     0 &     0 \\
        \rho^2 &\rho   &     1 &     0 &     0 &     0 &     0 \\
        \rho^3 &\rho^2 &\rho   &     1 &     0 &     0 &     0 \\
        \rho^4 &\rho^3 &\rho^2 &\rho   &     1 &     0 &     0 \\
        \rho^5 &\rho^4 &\rho^3 &\rho^2 &\rho   &     1 &     0 \\
        \rho^6 &\rho^5 &\rho^4 &\rho^3 &\rho^2 &\rho   &     1 
        \end{array}
  \right]
  \ \ 
  \left[
        \begin{array}{c}
                x_0 \\
                x_1 \\
                x_2 \\
                x_3 \\
                x_4 \\
                x_5 \\
                x_6 
        \end{array}
  \right]
\eq \bold C \bold x
\label{eqn:leakinteq}
\end{equation}
(The discrete world is related to the continuous by
$ \rho = e^{-\alpha\Delta\tau}$ and in
some applications, the diagonal is 1/2 instead of 1.)
Causal integration is
the simplest prototype of a recursive operator.
\sx{recursion ! integration}
The coding is trickier than
that for the operators we considered earlier.
Notice when you compute $y_5$ that it is the sum of 6 terms,
but that this sum is more quickly computed as $y_5 = \rho y_4 + x_5$.
Thus equation~(\ref{eqn:leakinteq}) is more efficiently thought of as
the recursion
\begin{equation}
y_t \eq \rho\; y_{t-1} + x_t
\quad
\quad
\quad t\ {\rm increasing}
\label{eqn:myrecur}
\end{equation}
(which may also be regarded as a numerical representation
of the \bx{differential equation} $dy/dt+y (1-\rho)/\Delta t=x(t)$.)
\par
When it comes time to think about the adjoint, however,
it is easier to think of equation~(\ref{eqn:leakinteq}) than of~(\ref{eqn:myrecur}).
Let the matrix of equation~(\ref{eqn:leakinteq}) be called $\bold C$.
Transposing to get $\bold{C}\T$ and applying it to $\bold y$
gives us something back in the space of $\bold x$,
namely $\tilde{\bold x} = \bold C\T \bold y$.
From it we see that the adjoint calculation,
if done recursively,
needs to be done backwards, as in
\begin{equation}
\tilde x_{t-1} \eq \rho \tilde x_{t} + y_{t-1}
\quad
\quad
\quad t \ {\rm decreasing}
\label{eqn:backrecur}
\end{equation}
Thus the adjoint of causal integration
is \bx{anticausal integration}.
\par
A module to do these jobs is \texttt{leakint}.
The code for anticausal integration is not obvious
from the code for integration and the adjoint coding tricks we
learned earlier.
To understand the adjoint, you need to inspect
the detailed form of the expression $\tilde{\bold x} = \bold C\T \bold y$
and take care to get the ends correct.
Figure \ref{fig:causint} illustrates the program for $\rho = 1$.
\opdex{causint}{causal integral}{35}{46}{api/c}
\par
\sideplot{causint}{width=.8\textwidth}{
  {\tt in1} is an input pulse.
  {\tt C in1} is its causal integral.
  {\tt C' in1} is the anticausal integral of the pulse.
  {\tt in2} is a separated doublet.
  Its causal integration is a box
  and its anticausal integration is a negative box.
  {\tt CC in2} is the double causal integral of {\tt in2}.
  How can an equilateral triangle be built?
}
\par
Later we will consider equations
to march wavefields up towards the earth surface,
a layer at a time, an operator for each layer.
Then the adjoint will start from the earth surface
and march down, a layer at a time, into the earth.

%\begin{notforlecture}
\begin{exer}

\item
Consider the matrix
\begin{equation}
  \left[
        \begin{array}{cccccccc}
             1 &     0 &     0 &     0 &     0 &     0 &     0 \\
             0 &     1 &     0 &     0 &     0 &     0 &     0 \\
             0 &  \rho &     1 &     0 &     0 &     0 &     0 \\
             0 &     0 &     0 &     1 &     0 &     0 &     0 \\
             0 &     0 &     0 &     0 &     1 &     0 &     0 \\
             0 &     0 &     0 &     0 &     0 &     1 &     0 \\
             0 &     0 &     0 &     0 &     0 &     0 &     1 
        \end{array}
  \right]
\end{equation}
and others like it with $\rho$ in other locations.
Show what combination of these matrices will represent
the leaky integration matrix
in equation (\ref{eqn:leakinteq}).  What is the adjoint?

\item
Modify the calculation in Figure~\ref{fig:causint} so that there is
a triangle waveform on the bottom row.
\item
Notice that the triangle waveform is not time aligned
with the input {\tt in2}.
Force time alignment with the operator 
${\bold C\T \bold C}$ or
${\bold C  \bold C\T}$.
\item
Modify \texttt{causint} \vpageref{lst:causint} by changing the diagonal to contain
1/2 instead of 1.
Notice how time alignment changes in Figure~\ref{fig:causint}.
\end{exer}
%\end{notforlecture}




\subsection{Backsolving, polynomial division and deconvolution}

\par
Ordinary differential equations often lead us to the backsolving operator.
For example, the damped harmonic oscillator leads to
a special case of equation
(\ref{eqn:polydiv})
where $(a_3,a_4,\cdots)=0$.
There is a huge literature on finite-difference solutions
of ordinary differential equations
that lead to equations of this type.
Rather than derive such an equation on the basis
of many possible physical arrangements,
we can begin from the filter transformation in (\ref{eqn:contran1})
but put the top square of the matrix on the other side of the equation
so our transformation can be called one of inversion or backsubstitution.
%Let us also force the matrix $\bold B$ to be a square matrix
%by truncating it with
%$\bold T = \left[ \bold I \quad \bold 0 \right]$, say
%$\bold A =\left[ \bold I \quad \bold 0 \right] \bold B = \bold T \bold B$.
To link up with applications in later chapters,
I specialize to 1's on the main diagonal and insert some bands of zeros.
\begin{equation}
\bold A \bold y
\eq
\left[ 
\begin{array}{ccccccc}
   1  & 0   & 0    & 0   & 0   & 0   & 0  \\
  a_1 &  1  & 0    & 0   & 0   & 0   & 0  \\
  a_2 & a_1 &  1   & 0   & 0   & 0   & 0  \\
  0   & a_2 & a_1  &  1  & 0   & 0   & 0  \\
  0   & 0   & a_2  & a_1 &  1  & 0   & 0  \\
  a_5 & 0   & 0    & a_2 & a_1 &  1  & 0  \\
  0   & a_5 & 0    &   0 & a_2 & a_1 & 1
  \end{array} \right] 
\; \left[ 
\begin{array}{c}
  y_0 \\ 
  y_1 \\ 
  y_2 \\ 
  y_3 \\ 
  y_4 \\ 
  y_5 \\ 
  y_6
  \end{array} \right]
\eq
\left[ 
\begin{array}{c}
  x_0 \\ 
  x_1 \\ 
  x_2 \\ 
  x_3 \\ 
  x_4 \\ 
  x_5 \\ 
  x_6
  \end{array} \right] 
\eq
\bold x
\label{eqn:polydiv}
\end{equation}
Algebraically, this operator goes under the various names,
``backsolving'',
``polynomial division'', and
``deconvolution''.
The leaky integration transformation
(\ref{eqn:leakinteq})
is a simple example of backsolving
when $a_1=-\rho$ and $a_2=a_5=0$.
To confirm this, you need to verify that the matrices in
(\ref{eqn:polydiv})
and
(\ref{eqn:leakinteq})
are mutually inverse.

\par
A typical row in equation (\ref{eqn:polydiv}) says
\begin{equation}
x_t \eq y_t \ +\ \sum_{\tau > 0} \ a_\tau\ y_{t-\tau}
\label{eqn:conrecur}
\end{equation}
Change the signs of all terms in equation
(\ref{eqn:conrecur})
and
move some terms to the opposite side
\begin{equation}
y_t \eq x_t \ -\ \sum_{\tau > 0} \ a_\tau\ y_{t-\tau}
\label{eqn:deconrecur}
\end{equation}
Equation (\ref{eqn:deconrecur}) is a recursion
to find $y_t$ from the values of $y$ at earlier times.

\par
In the same way that
equation (\ref{eqn:contran1})
can be interpreted as $Y(Z)=B(Z)X(Z)$,
equation (\ref{eqn:polydiv})
can be interpreted as $A(Z) Y(Z) = X(Z)$
which amounts to
$Y(Z) = X(Z)/A(Z)$.
Thus, convolution is amounts to polynomial multiplication
while the backsubstitution we are doing here
is called deconvolution, and it amounts to polynomial division.
\par
A causal operator is one that uses its present and past inputs
to make its current output.
Anticausal operators use the future but not the past.
Causal operators are generally associated with lower triangular matrices
and positive powers of $Z$, whereas
anticausal operators are associated with upper triangular matrices
and negative powers of $Z$.
A transformation like equation
(\ref{eqn:polydiv})
but with the transposed matrix would require us
to run the recursive solution the opposite direction in time,
as we did with leaky integration.


\par
A module to backsolve equation (\ref{eqn:polydiv}) is \texttt{recfilt}.
\opdex{recfilt}{deconvolve}{49}{75}{api/c}

\par
We may wonder why the adjoint of
$\bold A \bold y = \bold x$
actually is
$\bold A\T \hat{\bold x}= \bold y$.
With the well known fact that
the inverse of a transpose is the transpose of the inverse
we have
\begin{eqnarray}
\bold y          &=& \bold A^{-1} \bold x
\\
\hat \bold x     &=& (\bold A^{-1})\T \bold y
\\
\hat \bold x     &=& (\bold A\T)^{-1} \bold y 
\\
\bold A\T \hat \bold x &=& \bold y
\end{eqnarray}

\subsection{The basic low-cut filter}
Many geophysical measurements contain
very low-frequency noise called ``drift.''
For example, it might take some months to survey the depth of a lake.
Meanwhile, rainfall or evaporation could change the lake level so that
new survey lines become inconsistent with old ones.
Likewise, gravimeters are sensitive to atmospheric pressure,
which changes with the weather.
A magnetic survey of an archeological site would need to contend
with the fact that the earth's main magnetic field is changing randomly
through time while the survey is being done.
Such noises are sometimes called ``secular noise.''

\par
The simplest way to eliminate low frequency noise is
to take a time derivative.
A disadvantage is that the derivative
changes the waveform
from a pulse to a doublet (finite difference).
Here we examine the most basic low-cut filter.
It preserves the waveform at high frequencies;
it has an adjustable parameter
for choosing the bandwidth of the low cut;
and it is causal (uses the past but not the future).

\par
We make a causal lowcut filter (highpass filter) by
two stages which can be done in either order.
\begin{enumerate}
\item
        Apply a time derivative, actually a finite
        difference, convolving the data with $(1,-1)$.
\item
        Integrate, actually to do a leaky integration,
        to deconvolve with $(1,-\rho)$ where numerically,
        $\rho$ is slightly less than unity.
\end{enumerate}
The convolution with $(1,-1)$  ensures that the zero frequency is removed.
The leaky integration almost undoes the differentiation
with $(1,-1)$ but cannot restore the zero frequency.
Adjusting the numerical value of $\rho$ has interesting effects
in the time domain and in the frequency domain.
Convolving the finite difference $(1,-1)$ with the leaky integration 
$(1, \rho, \rho^2, \rho^3, \rho^4, \cdots)$
gives the result
\begin{eqnarray}
\nonumber
&(   1,& \rho, \rho^2, \rho^3, \rho^4, \cdots)
\\
\nonumber
\ -\ 
&(0,& 1, \rho, \rho^2, \rho^3,         \cdots).
\end{eqnarray}
Rearranging, it becomes
\begin{eqnarray}
\nonumber
&(1,&    0,      0,      0,      0, \cdots) \ +\ 
\\
\nonumber
(\rho-1) &(0, &1, \rho, \rho^2, \rho^3, \cdots).
\end{eqnarray}
Since $\rho$ is a tiny bit less than one,
$(1-\rho)$ is a small number.
Thus our filter is an impulse followed by the negative
of a weak decaying exponential $\rho^t$.
If you prefer a time-symmetric (phaseless) filter,
you could follow this one by its time reverse.
\par
Roughly speaking, the cutoff frequency of the filter corresponds
to matching one wavelength to the exponential decay time.
More formally,
the Fourier domain representation of this filter is
$H(Z) = (1-Z)/(1-\rho Z)$
where $Z$ is the unit-delay operator is $Z=e^{i\omega\Delta t}$
and where $\omega$ is the frequency.
The spectral response of the filter is $|H(\omega)|$.
Were we to plot this function we would see it is nearly 1
everywhere except in a small region near $\omega=0$
where it becomes tiny.
It is called a ``low-cut'' filter.
Figure~\ref{fig:galocut} compares a low-cut filter to a finite
difference.
\inputdir{galilee}
\plot{galocut}{width=\textwidth,height=.66\textwidth}{
        The depth of the Sea of Galilee after roughening.
	On the left, the smoothing is done by low-cut
	filtering on the horizontal axis.
	On the right it is a finite difference.
	We know this because of a few scattered impulses
	(navigation failure) outside the lake.
	Both results solve the problem of Figure~\ref{fig:galbin}
	which is too smooth to see interesting features.
        }


\subsection{Smoothing with box and triangle}
\inputdir{XFig}
%
Simple ``\bx{smoothing}'' is a common application of filtering.  A
smoothing filter is one with all positive coefficients.  On the time
axis, smoothing is often done with a single-pole damped exponential
function.  On space axes, however, people generally prefer a
symmetrical function.  We will begin with rectangle and triangle
functions. When the function width is chosen to be long, then the
computation time can be large, but recursion can shorten it
immensely. 
%
\par
%
The inverse of any polynomial reverberates forever, although it might
drop off fast enough for any practical need.  On the other hand, a
rational filter can suddenly drop to zero and stay there.  Let us look
at a popular rational filter, the rectangle or ``\bx{box car}'': 
\begin{equation}
{ \frac{1- Z^5}{ 1-Z} } \eq 1+Z+Z^2+Z^3+Z^4	\label{eqn:box}
\end{equation}
The filter (\ref{eqn:box}) gives a moving average under a {\em
rectangular} window.  This is a basic smoothing filter.  A clever
way to apply it is to move the rectangle by adding a new value at one
end while dropping an old value from the other end.  This approach is
formalized by the polynomial division algorithm, which can be
simplified because so many coefficients are either one or zero.  To
find the recursion associated with $Y(Z)= X(Z)(1-Z^5)/(1-Z)$, we
identify the coefficient of $Z^t$ in $(1-Z)Y(Z)= X(Z)(1-Z^5)$.  The
result is 
\begin{equation}
y_t \eq  y_{t-1} + x_t - x_{t-5}.
\end{equation}
This approach boils down to the program % \texttt{boxconv()} %\vpageref{/prog:boxconv},
which is so fast it is almost free!
\moddex{triangle}{box like smoothing}{175}{180}{api/c}
Its last line scales the output by dividing by the rectangle length.
With this scaling, the zero-frequency component of the input is
unchanged, while other frequencies are suppressed. 

\par
\bxbx{Triangle smoothing}{triangle smoothing} is rectangle smoothing
done twice.  For a mathematical description of the triangle filter, we
simply square equation (\ref{eqn:box}).  Convolving a rectangle
function with itself many times yields a result that mathematically
tends towards a \bx{Gaussian} function.
Despite the sharp corner on the top of the triangle function,
it has a shape that is remarkably similar to a Gaussian. 
Convolve a triangle with itself and you will see a very nice
approximation to a Gaussian (the central limit theorem).
%
\par
%
With filtering, \bx{end effect}s can be a nuisance.  Filtering
increases the length of the data, but people generally want to keep
input and output the same length (for various practical reasons).
This is particularly true when filtering a space axis.  Suppose the
five-point signal $(1, 1,1,1,1)$ is smoothed using the {\tt boxconv()}
program with the three-point smoothing filter $(1,1,1)/3$.  The output
is $5+3-1$ points long, namely, $(1,2,3,3,3,2,1)/3$.  We could simply
abandon the points off the ends, but I like to \bx{fold} them back in,
getting instead $(1+2,3,3,3,1+2)$.  An advantage of the folding is
that a constant-valued signal is unchanged by the smoothing.  This is
desirable since a smoothing filter is a low-pass filter which
naturally should pass the lowest frequency $\omega=0$ without
distortion.  The result is like a wave reflected by a
\bxbx{zero-slope}{zero slope} end condition.  Impulses are smoothed
into triangles except near the boundaries.  What happens near the
boundaries is shown in Figure~\ref{fig:triend}.
\inputdir{triangle}
\sideplot{triend}{width=.8\textwidth}{
	Edge effects when smoothing an impulse with a triangle function.
	Inputs are spikes at various distances from the edge.}
Note that at the boundary, there is necessarily only half a triangle,
but it is twice as tall.
\par
Why might this be useful?
Consider a survey of water depth in an area of the deep ocean.
All the depths are strongly positive with interesting but small variations on them.
Ordinarily we can enhance high frequency fluctuations by one minus a low pass filter,
say $H=1-L$.  If this is to work, however, it is important
that the $L$ truly cancel the $1$ near zero frequency.
%
\par
%
Figure~\ref{fig:triend} was derived from the routine {\tt triangle()}. %
%\progdex{triangle}{conv. with triangle} %
%%%TEMPORTARILY spiced in
\moddex{triangle}{1D triangle smoothing}{182}{188}{api/c}
%\begin{verbatim}
%  module triangle_smooth {
%    use box_smooth
%    contains
%    # Convolve with triangle
%    #
%    # input:        nb      rectangle width (points) (Triangle base twice as wide.)
%    # input:        xx(nd)                is a vector of data.
%    # output:       yy(nd)                may be on top of uu
%    subroutine triangle( nb, nd, xx, yy) {
%      integer,	      intent(in)       ::nb,nd
%      integer                          ::i,np,nq
%      real, dimension (:), intent (in) ::xx
%      real, dimension (:), intent (out)::yy
%      real, dimension (:), allocatable ::pp,qq
%      allocate(pp(nd+nb-1), qq(nd+nb+nb-2))
%      call boxconv( nb, nd, xx, pp);    np = nb+nd-1
%      call boxconv( nb, np, pp, qq);    nq = nb+np-1
%      do i=1,nd   { yy(i)     =             qq(i+nb-1)     }
%      do i=1,nb-1 { yy(i)     =yy(i)      + qq(nb-i  )     } # fold back near end
%      do i=1,nb-1 { yy(nd-i+1)=yy(nd-i+1) + qq(nd+(nb-1)+i)} # fold back far end
%      deallocate(pp,qq)
%    }
%  }
%\end{verbatim}

%%%TEMPORTARILY spiced in

%I frequently use this program, so it is cluttered with extra features.
%For example, the output can share the same location as the input.
%Further, since it is commonly necessary to smooth along the 2-axis
%of a two-dimensional array, there are some Fortran-style pointer
%manipulations to allow the user to smooth either the 1-axis or the
%2-axis or both.  For those of you unfamiliar with Fortran
%matrix-handling tricks, I include below another routine, {\tt
%triangle2()}, that teaches how a two-dimensional array can be
%smoothed over both its 1-axis and its 2-axis.
%%\progdex{triangle2}{conv. w. tri. in 2D}%
%%%%TEMPORTARILY spiced in
%\moddex{triangle2_smooth}{2D triangle smoothing}
%
%%\begin{verbatim}
%%  module triangle2_smooth {
%%    use triangle_smooth
%%    contains
%%    # smooth by convolving with triangle in two dimensions.
%%    #
%%    subroutine triangle2  ( rect1, rect2, n1, n2, xx, yy) {
%%      integer,intent(in) :: rect1, rect2, n1, n2
%%      integer            :: i1,i2
%%      real, intent (in)  ::   xx(:,:)
%%      real, intent (out) ::   yy(:,:)
%%      real, allocatable  :: temp(:,:)
%%      allocate(temp(n1,n2))
%%      do i2= 1, n2 { call triangle( rect1, n1, xx(1:,i2), temp(1:,i2)) }
%%      do i1= 1, n1 { call triangle( rect2, n2, temp(i1,1:), yy(i1,1:)) }
%%      deallocate(temp)
%%    }
%%  }
%\end{verbatim}
%%%TEMPORTARILY spiced in

%\todo{ \item The program {\tt triangle()} has a bug in the unusual case
%where {\tt nb $\ge$ nx}.
%Identify the bug.
%Recode the program to work for all values of {\tt nb}.
%}


% . . END JCS Include
% 
%%%%%%%%%%%%%%%%%%%%%%%%%%%%%%%




%\begin{notforlecture}
\subsection{Nearest-neighbor normal moveout (NMO)}
\par
\bxbx{Normal-moveout}{normal moveout}
correction (\bx{NMO}) is a geometrical correction
of reflection seismic data
that stretches the time axis so that data recorded
at nonzero separation $x_0$ of shot and receiver,
after stretching, appears to be at $x_0=0$.
NMO correction is roughly like time-to-depth conversion
with the equation $v^2 t^2 = z^2 + x_0^2$.
After the data at $x_0$ is stretched from $t$ to $z$,
it should look like stretched data from any other $x$
(assuming these are plane horizontal reflectors, etc.).
In practice, $z$ is not used; rather,
\bx{traveltime depth} $\tau$ is used,
where $\tau =z/v$;
so $t^2 = \tau^2+x_0^2/v^2$.
(Because of the limited alphabet of programming languages,
I often use the keystroke {\tt z} to denote $\tau$.)

\inputdir{cunha}

\par
Typically, many receivers record each shot.
Each seismogram can be transformed by NMO
and the results all added.
This is called ``\bx{stack}ing'' or ``\bx{NMO stack}ing.''
The adjoint to this operation is to begin from
a model which ideally is the zero-offset trace
and spray this model to all offsets.
From a matrix viewpoint, stacking is like a {\em  row} vector
of normal moveout operators and modeling is like a {\em  column}.
An example is shown in Figure \ref{fig:cunha}.
\sideplot{cunha}{width=.8\textwidth}{
  Hypothetical model,
  synthetic data, and model image.
}

\par
A module that does reverse moveout is \texttt{hypotenusei}.
Given a zero-offset trace, it makes another at non-zero offset.
The adjoint does the usual normal moveout correction.
\opdex{hypotenusei}{inverse moveout}{42}{73}{user/gee}
(My 1992 textbook (PVI) illustrates many additional features of normal moveout.)
A companion routine \texttt{imospray}
loops over offsets and makes a trace for each.
The adjoint of {\tt imospray} is the industrial process of moveout and stack.
\opdex{imospray}{inverse NMO spray}{47}{62}{user/gee}

\par
This is the first time we have built an operator (moveout and stack)
from a collection of other operators (moveout at various offsets)
and there are new possibilities for confusion and program bugs.
As earlier with the matrix multiplication operator,
to use the \texttt{imospray} operator,
there are two steps, one that we use to set things up
\begin{verbatim}
        imospray_init( slow, x0,dx, t0,dt, nt,nx);
\end{verbatim}
and another step that we use a lot in the next chapter
for analysis and data fitting.
\begin{verbatim}
        imospray_lop( adj, add, n1, n2, stack, gather);
\end{verbatim}
Later we'll see programs that are not operators.
\begin{comment}
Every program that is an operator
(contains {\tt {\#}{\%} {\_}init}
      and {\tt {\#}{\%} {\_}lop})
is expanded by Loptran to a longer Fortran code
where the
      {\tt {\_}lop}
function begins by
(potentially) erasing the output (when \texttt{add=.false.}).
This
potential erasing is done in both
\texttt{hypotenusei} and
\texttt{imospray}.
\end{comment}
Consider the the adjoint of spraying which is stacking.
Here the occurance of the
\texttt{add=true} in
\texttt{imospray} assures we do not erase the stack each time
we add in another trace.
\begin{comment}
Because of Loptran we don't explicitly see that
\texttt{imospray\_lop}
has its own potential erase of its output
which we'd turn off if we wanted to add one stack upon another.
\end{comment}

%\subsection{Model-pull binning}
%\sx{model-pull binning}
%\sx{binning ! model-pull}
%An alternative to pushing each data point into the nearest bin,
%is looping over all the bins on the uniform mesh and,
%for each bin, pulling in the nearest data value.
%Unlike pushing each data value exactly once,
%we pull a one data value into each bin
%so we never have empty bins.
%The trouble is that each data point is not used exactly once;
%it may be used many times or never.
%What you really want is no empty bins
%and no ignored data points,
%but you don't get that with either data push or bin pull.
%An example of interpolating some well depth-information
%onto a map is shown in Figure \ref{fig:pushpull90}.
%\activesideplot{pushpull90}{width=3in,height=1.5in}{}{
%        Well depths mapped with data push (left)
%        and bin pull (right).
%        }
%
%\par
%Both data push and bin pull are needed in practice.
%Earlier we saw module \texttt{dpbin2} \vpageref{/prog:dpbin2}
%that pushes in two dimensions.
%Here we see module \texttt{binpull1}
%that pulls in one dimension.
%\opdex{binpull1}{pull data into bin}
%It is an easy exercise to extend \texttt{binpull1}
%to a two-dimensional module {\tt binpull2}
%which is found in the library but not shown here.
%
%\par
%The topic of bin pull has much more
%intellectual content than it might appear.
%For large data spaces and model spaces, module
%\texttt{binpull1} is unnecessarily slow,
%because it is proportional to the product of the sizes of the two spaces,
%while a deeper analysis based on sorting
%(in $k$ dimensions, $k$-D trees) can reduce the cost immensely.
%To reduce the cost of
%using my simple code,
%I have split
%\texttt{binpull1}
%into two parts, an initialization part
%{\tt binpull1i()},
%and the operator part \texttt{binpull1}.
%The slow part is the initialization.
%

%\end{notforlecture}

%XXX
\subsection{Coding chains and arrays}

With a collection of operators,
we can build more elaborate operators.
An amazing thing about matrices is that their elements
may themselves be matrices.
A row is a matrix containing two matrices.
This is done by subroutine \texttt{row0}
also in module \texttt{smallchain3}.
An operator product
$\bold A = \bold B \bold C$
is represented in the subroutine 
\texttt{ chain2( op1, op2, ...)}.
As you read these codes, please remember the
output is the last argument only when the output is $\bold d$.
When the output is $\bold m$, the output is the second from last.




\moddex{chain}{operator chain and array}{26}{46}{api/c}
\moddex{chain}{operator chain and array}{70}{90}{api/c}

%\moddex{solver_tiny}{}

\section{ADJOINT DEFINED: DOT-PRODUCT TEST}

\sx{adjoint ! defined}
Having seen many examples of \bx{space}s, operators, and adjoints,
we should now see more formal definitions
because abstraction
helps us push these concepts to their limits.

\subsection{Definition of a vector space}
An operator transforms a \bx{space} to another space.
Examples of spaces are model space $\bold m$
and data space $\bold d$.
We think of these spaces as vectors
whose components are packed with numbers,
either real or complex numbers.
The important practical concept is that
not only does this packing include one-dimensional spaces like signals,
two-dimensional spaces like images, 3-D movie cubes,
and zero-dimensional spaces like a data mean, etc,
but spaces can be sets of all the above.
One space that is a set of three cubes is
the earth's magnetic field, which has three components;
and each component is a function of a three-dimensional space.
(The 3-D {\em  physical space} we live in is not
the abstract {\em  \bx{vector space}}
of models and data so abundant in this book.
Here the word ``space'' without an adjective means the vector space.)
\par
A more heterogeneous example of a vector space is \bx{data tracks}.
A depth-sounding survey of a lake can make a vector space that is
a collection of tracks,
a vector of vectors
(each vector having a different number of components,
because lakes are not square).
This vector space of depths along tracks in a lake
contains the depth values only.
The $(x,y)$-coordinate information
locating each measured depth value
is (normally) something outside the vector space.
A data space could also be a collection of echo soundings,
waveforms recorded along tracks.

\par
We briefly recall information about vector spaces found in elementary books:
Let $\alpha$ be any scalar.
Then if $\bold d_1$ is a vector and $\bold d_2$ is conformable
with it, then other vectors are
$\alpha \bold d_1$ and $\bold d_1 + \bold d_2$.
The size measure of a vector is a positive value called a norm.
The norm is usually defined to be the \bx{dot product}
(also called the $L_2$ \bx{norm}), say $\bold d \cdot \bold d$.
For complex data it is
$\bar{\bold{d}} \cdot \bold{d}$
where $\bar{\bold{d}}$ is the complex conjugate of $\bold{d}$.
A notation that does transpose and complex conjugate at the same time is $\bold d\T\,\bold d$.
In theoretical work the ``size of a vector'' means the vector's norm.
In computational work the ``size of a vector'' means the
number of components in the vector.

\par
Norms generally include a \bx{weighting function}.
In physics,
the norm generally measures a conserved quantity
like energy or momentum,
so, for example,
a weighting function for magnetic flux is permittivity.
In data analysis,
the proper choice of the weighting function is
a practical statistical issue,
discussed repeatedly throughout this book.
The algebraic view of a weighting function is that
it is a diagonal matrix
with positive values $w(i)\ge 0$ spread along the diagonal,
and it is denoted $\bold W = {\bf diag}[w(i)]$.
With this weighting function
the $L_2$ norm of a data space is denoted
$\bold d\T \bold W \bold d$.
Standard notation for norms uses a double absolute value,
where $||\bold d||=\bold d\T \bold W \bold d$.
A central concept with norms is the triangle inequality,
$ ||\bold d_1   +   \bold d_2|| \le ||\bold d_1|| + ||\bold d_2|| $
whose proof you might recall (or reproduce with the use of dot products).

\subsection{Dot-product test for validity of an adjoint}
\par
There is a huge gap between the
conception of an idea and putting it into practice.
During development, things fail far more often than not.
Often, when something fails,
many tests are needed to track down the cause of failure.
Maybe the cause cannot even be found.
More insidiously, failure may be below the threshold of detection
and poor performance suffered for years.
The \bx{dot-product test}
enables us to ascertain whether
the program for the adjoint of an operator
is precisely consistent with the operator itself.
It can be, and it should be.

Conceptually, the idea of matrix transposition is simply ${a}_{ij}'=a_{ji}$.
In practice, however, we often encounter matrices far too large
to fit in the memory of any computer.
Sometimes it is also not obvious how to formulate the process at hand
as a matrix multiplication.
(Examples are differential equations and fast Fourier transforms.)
What we find in practice is that an operator and its adjoint 
are two routines. 
The first amounts to the matrix multiplication $ \bold F \bold m$.
The adjoint routine computes $\bold F\T \bold d$,
where $\bold F\T$ is the \bx{conjugate-transpose} matrix.
In later chapters we will be solving huge sets of simultaneous equations,
in which both routines are required.
If the pair of routines are inconsistent,
we may be doomed from the start.
The dot-product test is a simple test for verifying that the two 
routines are adjoint to each other.

\par
I'll tell you first what the dot-product test is,
and then explain how it works.
Take a model space vector $\bold m$ filled with random numbers, and likewise
a data space  vector $\bold d$ filled with random numbers.
Use your forward modeling code to compute
\begin{eqnarray}
\bold m & \Leftarrow & {\rm random}
\\
\bold d & \Leftarrow & {\rm random}
\\
\bold {\hat d} &=& \bold F \bold m
\\
\bold {\hat m} &=& \bold F\T \bold d
\end{eqnarray}
You should find these two inner products equal:
\begin{equation}
\bold {\hat m} \cdot \bold m  =
\bold {\hat d} \cdot \bold d
\end{equation}
If they are, it means what you coded for $\bold F\T$ is indeed the adjoint of $\bold F$.
There is a glib way of saying why this must be so:
\begin{eqnarray}
\bold d\T ( \bold F \bold m ) &=& ( \bold d\T \bold F )  \bold m   \\
\bold d\T ( \bold F \bold m ) &=& ( \bold F\T \bold d )\T \bold m 
\label{eqn:bilin2}
\end{eqnarray}
This glib way is easily made concrete with explicit summation.
We may express $\sum_i \sum_j d_i F_{ij} m_j$ in two different ways.
\begin{eqnarray}
\sum_i d_i ( \sum_j F_{ij} m_m) &=& \sum_j (\sum_i d_i F_{ij}     ) m_j
\\
                                &=& \sum_j (\sum_i     F_{ij} d_i ) m_j
\\
 \bold d\T \cdot (\bold F\bold m)     &=& (\bold F\T \bold d) \cdot  \bold m
\\
 \bold d\T \cdot \bold{\hat d}        &=& \bold{\hat m}\cdot  \bold m
\end{eqnarray}
Should $\bold F$ contain complex numbers, the dot-product test is a comparison
for both real parts and for imaginary parts.

\par
The program for applying the dot product test is \texttt{dottest}
\vpageref{lst:dottest}. 
The C way of passing a linear operator
as an argument is to specify the function interface. Fortunately, we
have already defined the interface for a generic linear operator. To
use the \texttt{dottest} program, you need to initialize an operator
with specific arguments (the \texttt{\_init} subroutine) and then pass
the operator itself (the \texttt{\_lop} function) to the test program.
You also need to specify the sizes of the model and data vectors so
that temporary arrays can be constructed. The program runs the dot
product test twice, second time with \texttt{add = true} to test if
the operator can be used properly for accumulating results, for example.
$\bold d \leftarrow \bold d+\bold F\bold m$.

%\begin{notforlecture}
\moddex{dottest}{dot-product test}{52}{64}{api/c}
%\end{notforlecture}

\par
I ran the dot product test on many operators
and was surprised and delighted to find
that for small operators
it is generally satisfied to an accuracy near the computing precision.
For large operators, precision can become and issue.
Every time I encountered a relative discrepancy of $10^{-5}$ or more
on a small operator (small data and model spaces)
I was later able to uncover a conceptual or programming error.
Naturally,
when I run dot-product tests, I scale the implied matrix to a
small size both
to speed things along, and to be sure that
boundaries are not overwhelmed by the much larger interior.

\par
Do not be alarmed if the operator you have defined has \bx{truncation} errors.
Such errors in the definition of the original operator
should be matched by
like errors in the adjoint operator.
\sx{adjoint ! truncation errors}
If your code passes the \bx{dot-product test},
then you really have coded the adjoint operator.
In that case,
to obtain inverse operators,
you can take advantage of the standard methods of mathematics.

\par
We can speak of a \bx{continuous function} $f(t)$
or a \bx{discrete function} $f_t$.
For continuous functions we use integration,
and for discrete ones we use summation.
In formal mathematics, the dot-product test 
{\em  defines} the adjoint operator,
except that the summation in the dot product
may need to be changed to an integral.
The input or the output or both can be given
either on a continuum or in a discrete domain.
So the dot-product test
$\hat {\bold m} \cdot \bold m  =
\hat {\bold d} \cdot \bold d
$
could have an integration on one side of the equal sign
and a summation on the other.
Linear-operator theory is rich with concepts not developed here.


\subsection{Automatic adjoints}
Computers are not only able to perform computations;
they can do mathematics.
Well known software is Mathematica and Maple.
Adjoints can also be done by symbol manipulation.
For example Ralf Giering
offers a program for converting
linear operator programs into their adjoints.
Actually, it does more than that.
He says:\footnote{\url{http://www.autodiff.com/tamc/}}
\begin{quote}
Given a Fortran routine (or collection of routines) for a function,
TAMC produces Fortran routines for the computation
of the derivatives of this function.
The derivatives are computed in the reverse mode (adjoint model)
or in the forward mode (tangent-linear model).
In both modes Jacobian-Matrix products can be computed.
\end{quote}


\subsection{The word ``adjoint''}
In mathematics the word ``\bx{adjoint}'' has two meanings.
One of them, the so-called \bx{Hilbert adjoint},
is the one generally found in physics and engineering
and it is the one used in this book.
\sx{adjoint ! operator}
\sx{operator ! adjoint}
In linear algebra is a different matrix,
called the \bx{adjugate} matrix.
It is a matrix whose elements
are signed cofactors (minor determinants).
For invertible matrices,
this matrix is the \bx{determinant} times the \bx{inverse matrix}.
It can be computed without ever using division,
so potentially the adjugate can be useful in applications
where an inverse matrix does not exist.
Unfortunately, the adjugate matrix is sometimes called the adjoint matrix,
particularly in the older literature.
Because of the confusion of multiple meanings of the word adjoint,
in the first printing of PVI, I avoided the use of the word and
substituted the definition, ``\bx{conjugate transpose}''.
Unfortunately this was often abbreviated to ``conjugate,''
which caused even more confusion.
Thus I decided to use the word adjoint
and have it always mean the Hilbert adjoint
found in physics and engineering.

\begin{comment}
\subsection{Matrix versus operator}
\sx{operator}
Here is a short summary of where we have been and where we are going:
Start from the class of linear operators, add subscripts
and you get matrices.  Examples of operators without subscripts
are routines that solve differential equations and
routines that do fast Fourier transform.
What people call ``sparse matrices'' are often not really matrices
but operators,
because they are not defined by data structures
but by routines that apply them to a vector.
With sparse matrices you easily can do
$\bold A(\bold B(\bold C\bold x))$
but not
$(\bold A\bold B\bold C)\bold x$.
\par
Although a linear operator does not have defined subscripts,
you can determine what would be the operator value at any subscript:
by applying the operator to an impulse function, you would get a matrix column.
The adjoint operator is one from which we can extract the transpose matrix.
For large spaces this extraction is unwieldy,
so to test the validity of adjoints,
we probe them with random vectors,
say $\bold x$ and $\bold y$,
to see whether
$\bold y\T(\bold A\bold x)=(\bold A\T\bold y)\T \bold x$.
Mathematicians define adjoints by this test,
except that instead of using random vectors,
they say ``for all functions,'' which includes the continuum.
\par
This defining test makes adjoints look mysterious.
Careful inspection of operator adjoints,
however, generally reveals that they are built up from simple matrices.
Given adjoints
$\bold A\T$,
$\bold B\T$,
and
$\bold C\T$,
the adjoint of
$\bold A\bold B\bold C$
is
$\bold C\T\bold B\T\bold A\T$.
Fourier transforms and linear-differential-equation solvers
are chains of matrices,
so their adjoints can be assembled
by the application of adjoint components in reverse order.
The other way we often see complicated operators being built from simple ones
is when operators are put into components of matrices,
typically a
$1\times 2$
or
$2\times 1$
matrix containing two operators.
An example of the adjoint of a two-component column operator is
\begin{equation}
 \left[
  \begin{array}{c}
   \bold A \\
   \bold B
  \end{array}
 \right]\T
\eq
 \left[
  \begin{array}{cc}
   \bold A\T & \bold B\T
  \end{array}
 \right]
\end{equation}
\par
Although in practice an operator might be built from matrices,
fundamentally,
a matrix is a data structure whereas
an operator is a procedure.
A matrix is an operator if its subscripts are hidden
but it can be applied to a space, producing another space.
\par
As matrices have inverses, so do linear operators.
You don't need subscripts to find an inverse.
The
conjugate-gradient method and
conjugate-direction method
explained in the next chapter
are attractive methods of finding them.
They merely apply
$\bold A$
and
$\bold A\T$
and use inner products to find coefficients
of a polynomial in
$\bold A\bold A\T$
that represents the inverse operator.
\par
Whenever we encounter a \bx{positive-definite} matrix we should recognize
its likely origin in a nonsymmetric matrix
$\bold F$
times its adjoint.
Those in
natural sciences often work on solving simultaneous equations but fail
to realize that they should return to the origin of the equations
which is often a fitting goal; i.e.,
applying an operator to a model should yield data,
i.e.,
$\bold d  \approx  \bold d_0 + \bold F(\bold m-\bold m_0)$
where the operator
$\bold F$
is a partial derivative matrix
(and there are potential underlying nonlinearities).
This begins another story with new ingredients,
weighting functions and statistics.
\end{comment}

\subsection{Inverse operator}
A common practical task is to fit a vector of observed data
$\bold d_{\rm obs}$
to some modeled data
$\bold d_{\rm model}$
by the adjustment of components in a vector of model parameters $\bold m$.
\begin{equation}
\bold d_{\rm obs}
\quad\approx\quad
\bold d_{\rm model}
\eq
\bold F \bold m
\label{eqn:wefit}
\end{equation}
A huge volume of literature establishes theory for two estimates
of the model,
$\hat {\bold m}_1$ and
$\hat {\bold m}_2$, where
\begin{eqnarray}
\hat {\bold m}_1 &=& (\bold F\T\bold F)^{-1}\bold F\T\bold d
                                                \label{eqn:overdeterm} \\
\hat {\bold m}_2 &=& \bold F\T(\bold F\bold F\T)^{-1}\bold d
                                                \label{eqn:underdeterm}
\end{eqnarray}
Some reasons for the literature being huge are the
many questions
about the existence, quality, and cost
of the inverse operators.
Before summarizing that,
let us quickly see why these two solutions are reasonable.
Inserting equation
(\ref{eqn:wefit})
into equation
(\ref{eqn:overdeterm}),
and inserting equation
(\ref{eqn:underdeterm})
into equation
(\ref{eqn:wefit}),
we get the reasonable statements:
\begin{eqnarray}
\hat {\bold m}_1 &=& (\bold F\T\bold F)^{-1}(\bold F\T\bold F)\bold m \eq\bold m
                                                \label{eqn:overimplication} \\
\hat {\bold d}_{\rm model} &=&
                     (\bold F\bold F\T)(\bold F\bold F\T)^{-1}\bold d \eq\bold d
                                                \label{eqn:underimplication}
\end{eqnarray}
Equation (\ref{eqn:overimplication}) says that the estimate $\hat {\bold m}_1$
gives the correct model $\bold m$
if you start from the modeled data.
Equation (\ref{eqn:underimplication}) says that the model estimate $\hat {\bold m_2}$
gives the modeled data if we derive $\hat{\bold m}_2$
from the modeled data.
Both of these statements are delightful.
Now let us return to the problem of the inverse matrices.
\par
Normally a rectangular matrix does not have an inverse.
Surprising things often happen, but commonly,
when $\bold F$ is a tall matrix 
(more data values than model values)
then the matrix for finding
$\hat{\bold m}_1$
is invertible while that for finding
$\hat{\bold m}_2$
is not,
and when the matrix is wide instead of tall
(the number of data values is less than the number of model values)
it is the other way around.
In many applications neither
$\bold F\T\bold F$ nor
$\bold F\bold F\T$ 
is invertible.  This difficulty is
solved by ``\bx{damping}'' as we will see in later chapters.
If it happens that
$\bold F \bold F\T$
or
$\bold F\T \bold F$
equals $\bold I$ (unitary operator),
then the adjoint operator $\bold F\T$ is the inverse $\bold F^{-1}$
by either equation (\ref{eqn:overdeterm}) or (\ref{eqn:underdeterm}).

\par
Current computational power limits matrix inversion
jobs to about $10^4$ variables.
This book specializes in big problems,
those with more than about $10^4$ variables.
The iterative methods we learn here for giant problems
are also excellent for smaller problems,
so we rarely here speak of inverse matrices
or worry much if neither
$\bold F \bold F\T$
nor
$\bold F\T \bold F$ is an identity.

\begin{exer}
\item
Consider the matrix
\begin{equation}
  \left[
        \begin{array}{cccccccc}
             1 &     0 &     0 &     0 &     0 &     0 &     0 \\
             0 &     1 &     0 &     0 &     0 &     0 &     0 \\
             0 &  \rho &     1 &     0 &     0 &     0 &     0 \\
             0 &     0 &     0 &     1 &     0 &     0 &     0 \\
             0 &     0 &     0 &     0 &     1 &     0 &     0 \\
             0 &     0 &     0 &     0 &     0 &     1 &     0 \\
             0 &     0 &     0 &     0 &     0 &     0 &     1 
        \end{array}
  \right]
\end{equation}
and others like it with $\rho$ in other locations.
Show what combination of these matrices will represent
the leaky integration matrix
in equation (\ref{eqn:leakinteq}).  What is the adjoint?
\item
Modify the calculation in Figure~\ref{fig:causint} so that there is
a triangle waveform on the bottom row.
\item
Notice that the triangle waveform is not time aligned
with the input {\tt in2}.
Force time alignment with the operator 
${\bold C\T \bold C}$ or
${\bold C  \bold C\T}$.
\item
Modify \texttt{leakint}
by changing the diagonal to contain
1/2 instead of 1.
Notice how time alignment changes in Figure~\ref{fig:causint}.
\item
Suppose a linear operator $\bold F$ has
its input in the discrete domain and
its output in the continuum.
How does the operator resemble a matrix?
Describe the operator $\bold F\T$ that has
its input in the discrete domain and
its output in the continuum.
To which do you apply the words
``scales and adds some functions,''
and to which do you apply the words
``does a bunch of integrals''?
What are the integrands?

%\item
%Generally,
%a computer representation of a one-dimensional function is a series of values
%(ordinates at discrete, uniformly spaced coordinates).
%Suppose instead we wish to interlace function values with
%function derivatives,
%say $ y_0,y'_1, y_2,y'_3, y_4,y'_5, \cdots$ where
%prime denotes derivative.
%A simple transformation with this flavor,
%going from values to interlaced values and derivatives, is
%\begin{equation}
%\bold L \quad = \quad
%{1 \over \sqrt{2} }
%\left[ \begin{array}{ccccccccc}
%        1& 1& .& .& .& .&  .&  . \\
%        .& 1&-1& .& .& .&  .&  . \\
%        .& .& 1& 1& .& .&  .&  . \\
%        .& .& .& 1&-1& .&  .&  . \\
%        .& .& .& .& 1& 1&  .&  . \\
%        .& .& .& .& .& 1& -1&  . \\
%        .& .& .& .& .& .&  1&  1 
%       \end{array} \right]
%\end{equation}
%Although I know an exact inverse operator to $\bold L$ (based
%on $\bold L\bold L'$ being tridiagonal),
%I have chosen the scale factor so that the tridiagonal matrix
%$\bold L'\bold L$ is an identity matrix
%approximation for low frequencies.
%This is true because
%$\bold L'\bold L \bold 1 = \bold 1$ (except for end effects),
%where $\bold 1$ is a vector of 1's; i.e.,  it is a low-frequency signal.
%
%       \begin{enumerate}
%
%       \item
%       Write a module for $\bold L$ and $\bold L'$.
%       Do the dot-product test.
%
%       \item
%       Given a signal represented in ``signal-derivative-interlace'' form,
%       write a routine for filtering this signal where the adjoint
%       is the filter, and another where the adjoint is the input.

%       \item
%       Evaluate $\bold L'\bold L$ and verify that
%       $\bold L'\bold L \bold 1 = \bold 1$ (except for end effects).

%       \item
%       If you don't like the inverse approximation notice that we
%       could represent the derivative by the bilinear approximation
%       (see PVI) which has an exact inverse.  Explain.

%       \item
%       Suggest an application where it might be helpful to have
%       interlaced signal and derivative.

%       \item
%       A question arises in the obvious extension to two dimensions,
%       $L_xL_y$ or
%       $L_yL_x$ or
%       $(L_xL_y + L_yL_x)/2$.
%       What belongs in location $q$ in the data space
%       \begin{equation}
%       \left[ \begin{array}{ccc}
%                a& b& a \\
%                b& q& b \\
%                a& b& a \\
%               \end{array} \right]
%       \end{equation}
%       where $a$ denotes function values and $b$ denotes derivatives?
%\end{enumerate}

\end{exer}

\clearpage

 

%\end{notforlecture}

