\title{Seismic wavefield extrapolation by curved-ray phase-shift}
\author{Sergey Fomel}

\maketitle

\begin{abstract}
This is an abstract.
\end{abstract}

\section{Introduction}

Phase-shift method of wavefield continuation is a popular tool for
wave-equation imaging. While the original method of
\cite{GEO43-07-13421351} is applicable only for wave extrapolation through
constant-velocity layers, there are several extensions for approximate wave
extrapolation through variable velocities: PSPI (phase-shift plus
interpolation) \cite[]{GEO49-02-01240131}, extended split-step
\cite[]{GEO55-04-04100421,SEG-1992-0917}, Fourier finite differences
\cite[]{GEO59-12-18821893}, nonstationary phase-shift
\cite[]{GEO64-04-10671078}, etc.

In the PSPI method and its variations, one extrapolates the wavefield several
times through the same layer using a set of constant velocities and then
combines the results according to the local velocity. The accuracy of the
method depends on the number of velocities used. However, even with the
maximum possible number of extrapolation velocities, the method is only
approximate in the variable-velocity case, because it essentially assumes
straight-ray extrapolation. The straight-ray approximation can be sufficiently
accurate in the case of very small steps in depth. Improving this
approximation can allow for larger extrapolation steps and/or smaller number
of extrapolation velocities, which should result in noticeable computational
savings.

In this paper, I derive phase-shift extrapolation operators extending the
straight-ray approximation to locally parabolic rays. This construction leads
to second-order accurate wave-propagation operators. The accuracy improvement
is verified by numerical experiments.

Instead of assuming a constant local velocity, I assume a medium with a
locally constant gradient of slowness squared. Two particular extensions
correspond to the cases of vertical and horizontal gradients.

\section{PSPI and straight rays}

In the limiting case of the maximum number of extrapolation velocities, the
PSPI method \cite[]{GEO49-02-01240131} can be written in the form of a
pseudo-differential operator \cite[]{GEO64-04-10671078}
\begin{equation}
  \label{eq:pspi}
  W(z,x) = \int \widehat{W}(z+\Delta z,k)\,
  e^{i\,\omega\,\Delta z\,\sqrt{s^2(z,x)-(k/\omega)^2}-i\,k\,x}\,d k\;,
\end{equation}
where $W(z,x)$ is the wavefield at depth $z$, $\widehat{W}(z,k)$
is the spatial Fourier transform:
\begin{equation}
  \label{eq:ft}
  \widehat{W}(z,k) = \int W(z,y)\,e^{i\,k\,y}\,d y\;,
\end{equation}
$s(z,x)$ is the local slowness and $\omega$ is temporal frequency.

Substituting equation~(\ref{eq:ft}) into~(\ref{eq:pspi}), we can rewrite the
PSPI operator in the form
\begin{equation}
  \label{eq:pspi2}
  W(z,x) = \int W(z+\Delta z,y)\,\int
  e^{i\,\omega\,\Delta z\,\sqrt{s^2(z,x)-(k/\omega)^2}+i\,k\,(y-x)}\,
  d k\,d y\;.
\end{equation}
Evaluating the inner integral by the stationary-phase method, one can
establish the equivalence, in the high-frequency asymptotics, of the PSPI
extrapolator and the Kirchhoff extrapolation operator
\begin{equation}
  \label{eq:kirch}
  W(z,x) \approx \int 
  W(z+\Delta z,y)\,
  \frac{\Delta z}{D}\,
  \left[\frac{i\,\omega\,s(z,x)}{D}\right]^{m/2}\,
  e^{i\,\omega\,s(z,x)\,D}\,d y\;,
\end{equation}
where $D = \sqrt{(\Delta z)^2+(y-x)^2}$ is the straight-ray
distance between points $\{z,x\}$ and $\{z+\Delta z,y\}$, and $m$ corresponds
to the dimensionality of $x$ and $y$ ($m=1$ in the 2-D case and $m=2$ in the
3-D case).
  
\inputdir{XFig}

\plot{scheme}{width=\textwidth}{Local ray propagation (a scheme). Conventional
  phase-shift extrapolation approximates local ray trajectories inside each
  layer by straight lines (left). Curved-ray phase shift assumes curved
  trajectories to better approximate the true local character of wave
  propagation (right).}

In the general variable-velocity case, the traveltime $T$ between points
$\{z,x\}$ and $\{z+\Delta z,y\}$ follows a curved ray
(Figure~\ref{fig:scheme}). The following steps are needed to find the
appropriate phase-shift term:
\begin{enumerate}
\item Trace a ray from $\{z+\Delta z,y\}$ using the starting ray parameter 
  \begin{equation}
    \label{eq:py}
    p = -\frac{\partial T}{\partial y} = \frac{k}{\omega}\;.
  \end{equation}
\item Stop the ray when it hits the $z$ level. Record the output
  position $x$ and the traveltime  $T$ along the ray.
\item The phase-shift operator is then
  \begin{equation}
    \label{eq:ps}
    e^{i\omega\left[T + p\,(y-x)\right]}\;.
    \end{equation}
    The first term in equation~(\ref{eq:ps}) accounts for the traveltime,
    while the second term accounts for the shift of the lateral position.
  \item In order to use equation~(\ref{eq:ps}) in the wavenumber domain, it is
    necessary to express it in the form independent of $y$ but possibly
    dependent on the slowness $s(z,x)$ and other characteristics at the output
    location $\{z,x\}$. This transformation is possible in some special cases.
    In the case of a locally constant velocity, $T = s(z,x)\,D$. Expressing
    the $(y-x)$ term from equation~(\ref{eq:py}) and substituting it in
    equation~(\ref{eq:ps}), we arrive at the PSPI operator~(\ref{eq:pspi}),
    which has
    \begin{equation}
      \label{eq:simp}
      T+p\,(y-x) = \sqrt{s^2(z,x)-p^2}\;.
    \end{equation}

    Another special case is that of a constant
    gradient of slowness squared. It is considered in the next sections.
\end{enumerate}

\section{Vertical velocity gradient}

If the slowness $s(z,x)$ varies only with depth, the phase-shift term can be
expressed analytically as
\begin{eqnarray}
\nonumber
  T + p\,(y-x) & = &
  \int\limits_{z}^{z+\Delta z} 
  \frac{s^2(\zeta)\,d \zeta}{\sqrt{s^2(\zeta)-p^2}}
  - p \int\limits_{z}^{z+\Delta z} \frac{p\,d \zeta}{\sqrt{s^2(\zeta)-p^2}} \\
  & = & \int\limits_{z}^{z+\Delta z} \sqrt{s^2(\zeta)-p^2}\,d \zeta\;.
  \label{eq:dz}
\end{eqnarray}
Let us assume that the slowness squared $s^2(z)$ has a constant vertical
gradient between $z$ and $z+\Delta z$: $s^2(z)=s_0^2+2\,g\,(z-z_0)$. In this
case, ray trajectories have a parabolic form, and the integral in
equation~(\ref{eq:dz}) can be evaluated analytically to yield
\begin{eqnarray}
\nonumber
  \int\limits_{z}^{z+\Delta z} \sqrt{s_0^2+2\,g\,(\zeta-z_0)-p^2}\,d \zeta 
  & = & \left.
  \frac{2}{3}\,\frac{\left[s_0^2+2\,g\,(\zeta-z_0)-p^2\right]^{3/2}}{2\,g}
  \right|_{z}^{z+\Delta z} \\
  \nonumber 
  & = &
  \frac{2}{3}\,\Delta z\,
  \frac{\psi_1^2+\psi_2^2+\psi_1\,\psi_2}{\psi_1+\psi_2} \\
  & = & 
  \frac{2}{3}\,\Delta z\,
  \left(\psi_1+\psi_2-\frac{1}{1/\psi_1+1/\psi_2}\right)\;,
  \label{eq:zint}
\end{eqnarray}
where $\psi_1 = \sqrt{s^2(z)-p^2}$ and $\psi_2 = \sqrt{s^2(z+\Delta z)-p^2}$.
We can see that, in this case, the exact phase-shift operator is a combination
of shifts with velocities at the top and at the bottom of the layer. This
combined shift corresponds to waves propagating along parabolic rays in the
vertically-variable velocity.

In Appendix~A, I derive equation~(\ref{eq:zint}) in an alternative way by an
analytical solution of the Helmholtz equation. Appendix~B contains a
derivation of the corresponding one-way wave equation.

\subsection{Accuracy comparison}

\inputdir{Math}

How accurate is the phase shift expression~(\ref{eq:zint})? It is instructive
to compare its accuracy with that of the straight-ray approximation for the
case of velocity linearly increasing with depth: $1/s(z)=v_0+a\,(z-z_0)$. In
this case, ray trajectories have a circular form \cite[]{GEO01-01-00090022}, and
the phase integral has an analytical expression
\begin{equation}
\int\limits_{z}^{z+\Delta z} \sqrt{s^2(\zeta)-p^2}\,d \zeta =
\Delta z\,\frac{\psi_2\,s_1 - \psi_1\,s_2+
  s_1\,s_2\,\ln{\left(\frac{\psi_1+s_1}{\psi_2+s2}\right)}}{s_1-s_2}\;,
\label{eq:circl}
\end{equation}
where $s_1 = s(z)$ and $s_2=s(z+\Delta z)$. The relative errors in replacing
the exact expression~(\ref{eq:circl}) with the curved (parabolic) ray
approximation~(\ref{eq:zint}) or with the straight-ray approximation
$\psi_1\,\Delta z$ depend on the velocity contrast at the top and bottom of
the layer $(s2-s1)/s1$ and on the angle of wave propagation $\theta =
\arcsin{(p/s_1)}$. They are plotted in Figure~\ref{fig:errcomp}. We can see
that the error of replacing circular rays with parabolic rays is negligible
even with a high velocity contrast, while the error of the conventional
straight-ray phase shift can be significant, especially at large angles of
wave propagation.

\plot{errcomp}{height=\textheight}{Relative phase shift errors of parabolic
  ray (solid line) and straight ray (dashed line) approximations as functions
  of the propagation angle. A linear velocity function $1/s(z)=v_0+a\,(z-z_0)$
  is assumed. Different plots correspond to different velocity contrasts: a --
  0.1\%, b -- 1\%, c -- 10\%. In all cases, the relative error of the
  straight-ray approximation (conventional phase shift) climbs to 100\% at
  $90^{\circ}$ propagation angle.}

\inputdir{test}

Figure~\ref{fig:err} shows an accuracy comparison by a numerical experiment.
Figure~\ref{fig:err}a shows a migration impulse response computed by a phase
shift with a small depth step $\Delta z = 1\,\mbox{m}$ assuming a linear
velocity distributon with $v_0 = 1\,\mbox{km/s}$ and $a=2\,\mbox{1/s}$. This
impulse response is used as a reference for comparing different
approximations. The error of the straight-ray approximation is negligible when
the step size (layer thickness) is $\Delta z = 1\,\mbox{m}$
(Figure~\ref{fig:err}b) but becomes large when the layer thickness increases
to the more realistic values of 10~m (Figure~\ref{fig:err}d) and 40~m
(Figure~\ref{fig:err}f). The curved-ray (parabolic) approximation is
practically zero at $\Delta z = 10\,\mbox{m}$ (Figure~\ref{fig:err}c) and
negligible at 40~m (Figure~\ref{fig:err}e).

\plot{serr}{width=\textwidth}{Numerical accuracy comparison. a - Migration
  impulse response in a linear velocity distribution $1/s(z) = 1 + 2 z$
  computed with  $\Delta z = 1\,\mbox{m}$ and used for reference. b - Error of
  the straight-ray phase-shift for $\Delta z = 1\,\mbox{m}$. c - Error of
  the curved-ray phase-shift for $\Delta z = 10\,\mbox{m}$. d - Error of
  the straight-ray phase-shift for $\Delta z = 10\,\mbox{m}$. e - Error of
  the curved-ray phase-shift for $\Delta z = 40\,\mbox{m}$. f - Error of
  the straight-ray phase-shift for $\Delta z = 40\,\mbox{m}$.  
  All panels are plotted at the same clip.}

\section{Horizontal velocity gradient}

The case of the locally constant lateral gradient of slowness squared is
somewhat more complicated. Similarly to the previous case, the rays are
parabolic but their curvature is now affected by lateral changes in velocity.

Assuming that the slowness squared can be described locally as the linear 
function $s^2(x) = s_0^2 + 2\,g\,(x-x_0)$, where $g$ is half of the lateral
gradient, we can write the ray tracing system \cite[]{cerveny}
\begin{eqnarray}
  \label{eq:xray}
  \frac{d\,x}{d\,\sigma} & = & p\;, \\
  \label{eq:zray}
  \frac{d\,z}{d\,\sigma} & = & \sqrt{s^2(x)-p^2}\;, \\
  \label{eq:pray}
  \frac{d\,p}{d\,\sigma} & = & g\;, \\
  \label{eq:tray}
  \frac{d\,T}{d\,\sigma} & = & s^2(x) = s_0^2 + 2\,g\,(x-x_0)\;,
\end{eqnarray}
where $\sigma$ is a parameter increasing along a ray. The system has an
analytical solution, which leads to the expressions
\begin{eqnarray}
  \label{eq:ztrace}
  \Delta z & = & \sigma\,\sqrt{s^2(x) - (p - g\,\sigma)^2}\;, \\
  \label{eq:ttrace}
  T + p\,(y-x) & = & \left(s^2(x) - p^2\right)\,\sigma
    + \frac{3}{2}\,g\,p\,\sigma^2 - \frac{2}{3}\,g^2\,\sigma^3\;.
\end{eqnarray}
Evaluating $\sigma$ from equation~(\ref{eq:ztrace}) and substituting it into
equation~(\ref{eq:ttrace}), we should be able to obtain an exact phase-shift
factor to use in a PSPI-like extrapolation method. Even though an exact
analytical solution of the fourth-order equation~(\ref{eq:ztrace}) is
possible, it is inconvenient for practical use. One possibility is to solve
equation~(\ref{eq:ztrace}) numerically and tabulate the solution for every
value of $p/s$ and $g\,\Delta z/s^2$. A practical alternative is to
approximate the exact solution using a perturbation approach. Assuming a small
gradient $g$ or a small depth step $\Delta z$, we can write the Taylor series
expansion around $g=0$ by differentiating the parametric relationship in
equations~(\ref{eq:ztrace}-\ref{eq:ttrace}):
\begin{eqnarray}
  \label{eq:staylor}
  \sigma & = & \frac{\Delta z}{\psi}\,
  \left[1 - \frac{p\,(g\,\Delta z)}{\psi^3} + 
    \frac{\left(\psi^2+5\,p^2\right)\,(g\,\Delta z)^2}
    {2\,\psi^6} + O\left((g\,\Delta z)^3\right)\right]\;, \\
  \label{eq:taylor}
  T + p\,(y-x) & = & \Delta z\,\psi\,\left[1
  + \frac{p\,(g\,\Delta z)}{2\,\psi^3} 
  - \frac{\left(\psi^2+3\,p^2\right)\,(g\,\Delta z)^2}
  {6\,\psi^6} + O\left((g\,\Delta z)^3\right)\right]\;,
\end{eqnarray}
where $\psi = \sqrt{s^2(x)-p^2}$.
To avoid the danger of division by zero in equation~(\ref{eq:taylor}), we can
select a non-linear function, which has the Taylor expansion similar
to~(\ref{eq:taylor}). One such function is
\begin{equation}
  \label{eq:pspix}
  T + p\,(y-x) \approx
  \Delta z\,\sqrt{\frac{1}{3}\,\psi^2+
    \frac{2}{3}\,\sqrt{\psi^4+
      3\,p\,g\,\Delta z\,\psi - \left(g\,\Delta z\right)^2}}\;.
\end{equation}

\subsection{Accuracy comparison}

\inputdir{Math}

The error of approximation~(\ref{eq:pspix}) is plotted in
Figure~\ref{fig:xerr}, where it is compared with the error of the straight-ray
approximation. Approximation~(\ref{eq:pspix}) appears accurate for a wide
range of propagation angles even for high velocity contrasts, where the error
of the conventional straight-ray phase shift (PSPI approximation) is
significant.

\plot{xerr}{height=\textheight}{Relative phase shift errors of
  approximation~(\ref{eq:pspix}) (solid line) and the straight-ray
  approximation (dashed line) as functions of the propagation angle. Different
  plots correspond to different velocity contrasts ($g\,\Delta z/s^2(x)$): a
  -- 0.1\%, b -- 1\%, c -- 10\%. In all cases, the relative error of the
  straight-ray approximation (conventional phase shift) climbs to 100\% at
  $90^{\circ}$ propagation angle.}

\newpage

\bibliographystyle{segnat}
\bibliography{SEG,pspi}

\newpage

\APPENDIX{A}
\section{Curved-ray phase-shift extrapolator and the Helmholtz equation}

In this appendix, I derive the curved-ray (parabolic) phase-shift extrapolator
in a $v(z)$ medium by solving the Helmholtz equation analytically. Assuming
that the wavefield $W(z,x)$ satisfies the Helmholtz equation
\begin{equation}
  \label{eq:helm}
  \frac{\partial^2 W}{\partial z^2} +
  \frac{\partial^2 W}{\partial x^2} +
  \omega^2\,s^2(z,x)\,W = 0
\end{equation}
and that the slowness $s(z,x)$ is a linear function of depth $z$: $s^2(z) =
s_0^2 + 2\,g\,(z-z_0)$, one can transform equation~(\ref{eq:helm}) 
to the wavenumber domain
\begin{equation}
  \label{eq:helm2}
  \frac{d^2 \widehat{W}}{d z^2} + 
  \omega^2\,\left(s_0^2 + 2\,g\,(z-z_0)  - p^2\right)\,
  \widehat{W} = 0
\end{equation}
and solve it analytically with the help of the following transformations:
\begin{enumerate}
\item Change from $z$ to a new variable
  \begin{equation}
    \label{eq:z2}
    \zeta(z) = \frac{2}{3}\,\omega\,
    \frac{
      \left(s_0^2 + 2\,g\,(z-z_0)  - p^2\right)^{3/2}}
    {2\,g}\;.
  \end{equation}
\item Change from $\widehat{W}(z)$ to a new function 
  $V(\zeta)$ such that $\widehat{W}(z) = \zeta^{1/3}\,V(\zeta)$.
\end{enumerate}
With these changes, equation~(\ref{eq:helm2}) transforms to
\begin{equation}
  \label{eq:bessel}
  \zeta^2\,\frac{d^2 V}{d \zeta^2} + 
  \zeta\,\frac{d V}{d \zeta} + 
  \left(\zeta^2-\frac{1}{9}\right)\,V(\zeta) = 0\;,
\end{equation}
which is the Bessel equation of order $1/3$. 
Therefore, $V(\zeta)$ is simply a Bessel function of order $1/3$. 
We can write the analytical solution of equation~(\ref{eq:helm2}) in the form
\begin{equation}
  \label{eq:anal}
  \widehat{W}(z) = \widehat{W}(z+\Delta z)\,\left(\frac{\zeta_1}{\zeta_2}\right)^{1/3}\,
  \frac{H_{1/3}^{(2)}(\zeta_1)}{H_{1/3}^{(2)}(\zeta_2)}\;,
\end{equation}
where $\zeta_1 = \zeta(z)$, $\zeta_2 = \zeta(z+\Delta z)$, and $H_{1/3}^{(2)}$
is the Hankel function (Bessel function of the third kind) of order $1/3$. The
choice of the Hankel function for propagating upward traveling waves is
justified by considering the high-frequency asymptotics of the analytical
solution~(\ref{eq:anal}).

In the high-frequency asymptotic regime $\zeta \rightarrow \infty$, and 
the Hankel function behaves as \cite[]{watson}
\begin{equation}
  \label{eq:asymp}
  H_{1/3}^{(2)}(\zeta) \approx \sqrt{\frac{2}{\pi\,\zeta}}\,
  e^{-i\,(\zeta - 5\,\pi/12)}\;.
\end{equation}
Correspondingly, the analytical solution~(\ref{eq:anal}) transforms
to
\begin{equation}
  \label{eq:wkbj}
  \widehat{W}(z) \approx \widehat{W}(z+\Delta z)\,\left(\frac{\zeta_2}{\zeta_1}\right)^{1/6}\,
  e^{i\,(\zeta_2 - \zeta_1)}
  = \widehat{W}(z+\Delta z)\,\sqrt{\frac{\psi_2}{\psi_1}}\,
  \exp{\left(i\,\omega\,\frac{2}{3}\,\Delta z\,
      \frac{\psi_1^2+\psi_2^2+\psi_1\,\psi_2}{\psi_1+\psi_2}\right)}
  \;,
\end{equation}
where $\psi_1 = \sqrt{s^2(z)-p^2}$ and $\psi_2 = \sqrt{s^2(z+\Delta
  z)-p^2}$. The phase term in equation~(\ref{eq:wkbj}) corresponds
precisely to equation~(\ref{eq:zint}), derived in a different way. The
amplitude term $\sqrt{\psi_2/\psi_1}$ is a common WKBJ amplitude transmission
factor \cite[]{stolt}.

\newpage

\APPENDIX{B}

\section{Curved-ray one-way wave equation}

It is instructive to obtain a one-way equivalent of the two-way Helmholtz
equation in the curved ray approximation. According to
equation~(\ref{eq:wkbj}), the curved-ray solution behaves at high frequencies
as
\begin{equation}
  \label{eq:behave}
  \widehat{W}(z) \approx A\,\zeta^{-1/6}\,e^{-i\,\zeta}\;,
\end{equation}
where $\zeta$ is defined in equation~(\ref{eq:z2}), and $A$ does not depend on
$\zeta$. Differentiating equation~(\ref{eq:behave}), we obtain
\begin{equation}
  \label{eq:ow1}
  \frac{d \widehat{W}}{d \zeta} \approx 
  - \left(i + \frac{1}{6\,\zeta}\right)\,\widehat{W}
\end{equation}
and
\begin{equation}
  \label{eq:ow2}
  \frac{d \widehat{W}}{d z} = 
\frac{d \widehat{W}}{d \zeta}\,\frac{d \zeta}{d z}
\approx 
  - \left(i\,\omega\,\psi + \frac{g}{2\,\psi^2}\right)\,
  \widehat{W}\;,
\end{equation}
where $\psi = \sqrt{s^2(z)-p^2}$. Equation~(\ref{eq:ow2}) is
equivalent to the true-amplitude one-way wave equation of
\cite[]{SEG-2003-09250928}, derived in a different way. Extrapolation
operator~(\ref{eq:wkbj}) provides an analytical solution of
equation~(\ref{eq:ow2}) under the assumption that gradient of the
slowness squared $g=s(z)\,s'(z)$ does not depend on $z$.


%%% Local Variables: 
%%% mode: latex
%%% TeX-master: t
%%% End: 
