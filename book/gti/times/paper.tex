%\documentclass[12pt]{handout}

%\begin{document}
\newcommand{\boxit}[1]{\noindent\fbox{\parbox{0.95\textwidth}{#1}}}

\author{Sergey Fomel}
\title{Geometry of Waves}

\maketitle

In the introductory lecture, we talked about the geometry of seismic
reflectors as seen in seismic images and the geometry of seismic
reflection events as seen in field data. The connection between the
two geometries is established via the geometry of seismic wave
propagation.

\section{What is a wave?}

\inputdir{XFig}

A wave (any wave) can be defined as a disturbance propagating with a
finite speed. If a wave propagation process starts with a source in a
limited area in space and we take a snapshot of the wave after a
finite period of time, we will see two distinct regions: a region that
``does not know yet'' about the existence of the wave and a region
that ``has already seen the wave'' (Figure~\ref{fig:wave}). The
boundary between these two regions is the wavefront.

\sideplot{wave}{width=0.7\textwidth}{A wave is a disturbance
  propagating with a finite speed. At a fixed point in time, the
  boundary between the unperturbed region and the region already
  touched by the wave propagation is the wavefront.}

Of course, everything in the physical world propagates with a finite
speed. However, we can distinguish between two fundamentally different
kinds of propagation processes: wave propagation and diffusion. A
fundamental difference between them is the existence of wavefronts. In
a diffusion process, the transition between the unperturbed and
unperturbed regions is so smooth that is impossible to draw a
wavefront. In a wave propagation process, there is always some
discontinuity on or across the wavefront. The fact that seismic energy
propagates in a wave form allows us to use the geometrical properties
of wavefronts in seismic imaging.

\section{Eikonal}
How can we describe the wavefront geometry mathematically? One
possible approach is to define the function $T(\mathbf{x})$, which
assigns the traveltime from the source to points $\mathbf{x}$ in
space. The traveltime function is sometimes called
\emph{eikonal}\footnote{The term ``eikonal'' is derived from the Greek
word for ``image'' and is a relative of the word ``icon''.}. It is
defined for a fixed wave source.

A wavefront at a particular time $t$ is the collection of all points
$\mathbf{x}$  for which $T(\mathbf{x}) = t$. In the simplest case of a
point source at $\mathbf{x}_0$ in a homogeneous isotropic material,
the traveltime simply equals the distance to the source divided by the
velocity $V$:
\begin{equation}
  \label{eq:const}
  T(\mathbf{x}) = \frac{\left|\mathbf{x}-\mathbf{x}_0\right|}{V}\;.
\end{equation}
We will use the notation $|\mathbf{a}|$ to denote the Euclidian length
of the vector $\mathbf{a}=\{a_1,a_2,a_3\}$ so that
$|\mathbf{a}|=\sqrt{a_1^2+a_2^2+a_3^2}$. In the case of a constant
isotropic velocity, the wavefront at time $t$ is a sphere with the
radius $V\,t$ centered at the source point $\mathbf{x}_0$
(Figure~\ref{fig:cone}).

\sideplot{cone}{width=0.9\textwidth}{In a constant isotropic velocity,
  the traveltime function $T(\mathbf{x})$ is a cone. Its horizontal
  sections are spherical wavefronts (circular in 2-D). Its vertical
  sections are hyperboloids (hyperbolas in 2-D).}

Wavefronts are \emph{level sets} or iso-surfaces of the traveltime
function $T(\mathbf{x})$. In the locations where the traveltime gradient 
\[
\nabla T = \left\{
\frac{\partial T}{\partial x_1},
\frac{\partial T}{\partial x_2},
\frac{\partial T}{\partial x_3}\right\}
\]
is defined, $\nabla T$ is a vector pointing in the direction of the
maximum change in $T$ or, in other words, in the direction normal to
the local wavefront. To describe the front evolution in time, it is
sufficient to prescribe a velocity to the front propagation in the
normal direction. Such a velocity is called \emph{phase velocity}. The
general front propagation equation is
\begin{equation}
  \label{eq:front}
  \left|\nabla T\right| = \frac{1}{V(\ldots)}\;,
\end{equation}
where $V$ is the phase velocity, which can depend on a number of
different characteristics. In the case of isotropic wavefront
propagation, the velocity depends only on the location, and
equation~(\ref{eq:front}) takes the form
\begin{equation}
  \label{eq:eikonal}
  \left|\nabla T\right| = \frac{1}{V(\mathbf{x})}\;.
\end{equation}
Equation~(\ref{eq:eikonal}) is known as the \emph{eikonal equation}.

In the case of \emph{anisotropy}, the phase velocity depends both on
the spacial position and on the direction of the local wavefront
normal. The general anisotropic eikonal equation is
\begin{equation}
  \label{eq:anisoeikonal}
  \left|\nabla T\right| = 
  \frac{1}{V\left(\mathbf{x},\frac{\nabla T}{\left|\nabla T\right|}\right)}\;.
\end{equation}
We will limit the following analysis to the isotropic case and will return
to anisotropy later in the course.

\section{Ray tracing}
Another way to approach the wavefront propagation problem is to track
the movement of individual points on the wavefront. For the purposes
of this analysis, it is convenient to rewrite the eikonal
equation~(\ref{eq:eikonal}) in the form
\begin{equation}
  \label{eq:hamilton}
  H(\mathbf{x},\mathbf{p}) = \mathbf{p} \cdot \mathbf{p} - 
  S^2(\mathbf{x}) = 0\;,
\end{equation}
where the vector $\mathbf{p}$ denotes $\nabla T$ and $S$ is
the slowness: $S(\mathbf{x}) = \frac{1}{V(\mathbf{x})}$ . According to
the eikonal equation, the function $H(\mathbf{x},\mathbf{p})$ is equal
to zero everywhere and does not change along rays (the trajectories of
individual wavefront points). If $\sigma$ denotes a variable moving
along a ray, then the differential $d\,H/d\,\sigma$ should be
zero. Differentiating equation~(\ref{eq:hamilton}), we obtain
\begin{equation}
  \label{eq:dsigma}
  \frac{d\,H}{d\,\sigma} = 
  2\,\mathbf{p} \cdot \frac{d\,\mathbf{p}}{d\,\sigma} -
  2\,S(\mathbf{x})\,\nabla S \cdot \frac{d\,\mathbf{x}}{d\,\sigma} = 0\;.
\end{equation}
Note that we have not specified the precise meaning of $\sigma$. It
is just some parameter that changes monotonically along each ray
(Figure~\ref{fig:ray}). The choice of $\sigma$ becomes constrained if
we split equation~(\ref{eq:dsigma}) into a system of ordinary
differential equations\footnote{This procedure is known in mathematics
as the Hamilton-Jacobi theory.}
\begin{eqnarray}
  \label{eq:xray}
  \frac{d\,\mathbf{x}}{d\,\sigma} & = & \mathbf{p}\;, \\
  \label{eq:pray}
  \frac{d\,\mathbf{p}}{d\,\sigma} & = & S(\mathbf{x})\,\nabla S\;.
\end{eqnarray}
Equations~(\ref{eq:xray}-\ref{eq:pray}) can be combined into one
second-order equation for the ray trajectory $\mathbf{x}(\sigma)$:
\begin{equation}
  \label{eq:ray2}
  \frac{d^2\,\mathbf{x}}{d\,\sigma^2} =  S(\mathbf{x})\,\nabla S
\end{equation}
The $\sigma$ variable in equations~(\ref{eq:xray}-\ref{eq:ray2}) has a defined
meaning and the physical dimensions of distance squared divided by time.
Starting with an initial point $\mathbf{x}(0)=\mathbf{x}_0$ and an initial
direction vector $\mathbf{x}'(0)=\mathbf{p}(0)=\mathbf{p}_0$, one can, in
principle, trace the ray trajectory $\mathbf{x}(\sigma)$. Repeating this
process for every point on the wavefront, we can track the wavefront movement.
The numerical approaches to ray tracing will be discussed in the next lecture.
If the slowness $S(\mathbf{x})$ is constant, $\mathbf{p}$ does not change with
$\sigma$, and all rays become straight lines that can be described with 
equation
\begin{equation}
  \label{eq:straight}
  \mathbf{x}(\sigma) = \mathbf{x}_0 + \mathbf{p}_0\,\sigma\;.
\end{equation}

\sideplot{ray}{width=0.45\textwidth}{Rays are trajectories of
  individual points on the propagating wavefronts. The $\sigma$
  parameter changes monotonically along each individual ray.}

How does traveltime change along a ray? According to the chain rule,
the derivative of any scalar function $f(\mathbf{x})$ along the
$\sigma$ direction satisfies
\[
\frac{d\,f}{d\,\sigma} = \nabla f \cdot
\frac{d\,\mathbf{x}}{d\,\sigma}\;.
\]
Similarly, for the traveltime function $T(\mathbf{x})$,
\begin{equation}
  \label{eq:tray0}
  \frac{d\,T}{d\,\sigma} = \nabla T \cdot
  \frac{d\,\mathbf{x}}{d\,\sigma}
\end{equation}
or, using the notation $\mathbf{p} = \nabla T$ and
equations~(\ref{eq:hamilton}) and~(\ref{eq:xray}),
\begin{equation}
  \label{eq:tray}
  \frac{d\,T}{d\,\sigma}
  = \mathbf{p} \cdot \mathbf{p} =
  S^2(\mathbf{x})\;.
\end{equation}
Solving equation~(\ref{eq:tray}) along with equations~(\ref{eq:xray})
and ~(\ref{eq:pray}), we can track the wavefront changes along
individual rays. In the constant velocity case, the traveltime changes
linearly as
\begin{equation}
  \label{eq:tconst}
  T(\sigma) = S^2\,\sigma\;.
\end{equation}

According to equation~(\ref{eq:tray}), the traveltime derivative
$d\,T/d\,\sigma$ is always positive. This means that the traveltime
increases monotonically along a ray. Therefore, it can be used instead
of $\sigma$ as a ray trajectory variable. Dividing
equations~(\ref{eq:xray}--\ref{eq:pray}) by equation~(\ref{eq:tray}),
we obtain the ray tracing system defined in terms of traveltime, as
follows:
\begin{eqnarray}
  \label{eq:xrayt}
  \frac{d\,\mathbf{x}}{d\,T} & = & 
  \frac{\mathbf{p}}{S^2(\mathbf{x})} = 
  \frac{1}{S(\mathbf{x})}\,\frac{\mathbf{p}}{|\mathbf{p}|} =
  V(\mathbf{x})\,\frac{\mathbf{p}}{|\mathbf{p}|}\;, \\
  \label{eq:prayt}
  \frac{d\,\mathbf{p}}{d\,T} & = & \frac{\nabla S}{
    S(\mathbf{x})} = - \frac{\nabla V}{V(\mathbf{x})}\;.
\end{eqnarray}
The vector $d\,\mathbf{x}/d\,T$ is the \emph{group velocity}. It points
  in the direction of the ray trajectory and describes the velocity of
  an individual point on the wavefront. In the isotropic case, the
  group velocity points in the direction normal to the wavefront and
  coincides in value with the phase velocity.

There are other alternative ways to define the ray tracing system. A
curious property of system~(\ref{eq:xray}--\ref{eq:pray}) is the
mechanical analogy described in the next section.

\section{Mechanical analogy}
As noticed by Sir Isaac Newton, there is a certain analogy between the
trajectory of a ray and the trajectory of a mechanical particle moving
in a potential field. The total energy $E$ of a moving particle is
composed of the kinetic energy and the potential energy:
\begin{equation}
  \label{eq:energy}
  E(\mathbf{x},\mathbf{v}) = m\,\frac{\mathbf{v} \cdot
    \mathbf{v}}{2} - U(\mathbf{x})\;,
\end{equation}
where $m$ is the particle mass, $\mathbf{v}$ is the particle velocity,
and $- U(\mathbf{x})$ is the potential energy. An example of a
mechanical system described by equation~(\ref{eq:energy}) is a planet
rotating around the Sun. In this case, $U$ is the gravitational
potential
\begin{equation}
  \label{eq:sun}
  U(\mathbf{x}) = G\,\frac{m\,M}{|\mathbf{x}-\mathbf{x}_0|}\;,
\end{equation}
where $G$ is the gravitational constant, $M$ is the sun mass, and
$\mathbf{x}_0$ is the sun position.
The total energy is conserved in the propagation process and does not
change with the time $t$. Therefore,
\begin{equation}
  \label{eq:dt}
  \frac{d\,E}{d\,t} = m\,\mathbf{v} \cdot
  \frac{d\,\mathbf{v}}{d\,t} - \nabla U \cdot
  \frac{d\,\mathbf{x}}{d\,t} = 0\;.
\end{equation}
The corresponding ``ray tracing'' system
takes the form
\begin{eqnarray}
  \label{eq:xtraj}
  \frac{d\,\mathbf{x}}{d\,t} & = & \mathbf{v} \\
  \label{eq:vtraj}
  \frac{d\,\mathbf{v}}{d\,t} & = & \frac{\nabla U}{m}\;.
\end{eqnarray}
The second-order equivalent of the first-order
system~(\ref{eq:xtraj}-\ref{eq:vtraj}) is the equation
\begin{equation}
  \label{eq:xtraj2}
  \frac{d^2\,\mathbf{x}}{d\,t^2} = \frac{\nabla U}{m}\;,
\end{equation}
which simply states Newton's second law of mechanics. The $\nabla U$
term is the potential force that causes the particle acceleration. In
the case of the planet motion, the gravitational force is directed
towards the sun. In the absence of the external force, the velocity
$\mathbf{v}$ remains constant (Newton's first law and the principle of
inertia). This is analogous to the ray trajectory remaining a straight
line in the absence of velocity variations.

The analogy between particle trajectories and ray trajectories is
summarized in Table~\ref{tab:anal}. 

\begin{table}
  \centering
  \begin{tabular}{|c||c|} \hline
    Particle & Ray \\ \hline \hline
    Position $\mathbf{x}$ & Position $\mathbf{x}$ \\ \hline
    Time $t$ & Natural ray coordinate $\sigma$ \\ \hline
    Velocity $\mathbf{v}$ & Ray parameter (slowness vector) $\mathbf{p} = \nabla T$  \\ \hline
    Acceleration 
    $(\nabla U)/m$ & $S(\mathbf{x})\,\nabla S$ \\ \hline
  \end{tabular}
\caption{Analogy between physical quantities of a moving mechanical
  particle and those of a ray.}
\label{tab:anal}
\end{table}

\section{Snell's law}
How many independent equations does the ray tracing
system~(\ref{eq:xray}-\ref{eq:pray}) contain? On the first glance,
there are six independent equations: three for the components of
$\mathbf{x}(\sigma)$ and three for the components of
$\mathbf{p}(\sigma)$. However, we can also use the eikonal
equation~(\ref{eq:hamilton}), which serves as an additional constraint
on the parameters. The eikonal equation states that the length of the
vector $\mathbf{p}$ should always equal the local slowness
$S(\mathbf{x})$. It is sufficient to specify $\mathbf{p}$ with only
two parameters (such as the direction angles) and use the local
slowness to find the third parameter (such as the vector length). In
this way, the system will be reduced to five independent equations.

Additional simplifications are possible for particular kinds of the
velocity fields. An especially important case is the velocity that
changes only with depth. Most of the Earth's upper crust is composed
of horizontally-layered sediments, making vertical changes in seismic
velocities the predominant factor in seismic exploration. To handle
this case mathematically, let us partition the spatial location vector
$\mathbf{x}$ into the horizontal and vertical coordinates, as follows:
$\mathbf{x}=\{\mathbf{y},z\}$. Analogously, the $\mathbf{p}$ vector
gets partitioned into horizontal and vertical components: 
$\mathbf{p}=\{\mathbf{p}_y,p_z\}$. If the velocity $V(\mathbf{x})$
depends only on the $z$ coordinate, then, according to
equation~(\ref{eq:pray}), only the $p_z$ component of the ray vector
$\mathbf{p}$ changes along the ray. The conservation of the lateral
component $\mathbf{p}_y$ is the manifestation of Snell's law.

\boxit{In a $V(z)$ medium, the lateral components of the slowness
  vector $\mathbf{p}$ remain constant along each ray trajectory.}
  
If $\alpha$ is the angle of the ray measured from the vertical, then
the vertical component of $\mathbf{p}$ can be expressed as $p_z =
\frac{\cos{\alpha}}{V}$, and the length of the lateral component is
$|\mathbf{p}_y| = \frac{\sin{\alpha}}{V}$. Consider a horizontal
interface between two layers with velocities $V_1$ and $V_2$
(Figure~\ref{fig:snell}). The conservation of $\mathbf{p}_y$ leads to
the equation of refraction, which is the best known form of Snell's
law:
\begin{equation}
  \label{eq:snell}
  \frac{\sin{\alpha_1}}{V_1} =  \frac{\sin{\alpha_2}}{V_2}
\end{equation}

\sideplot{snell}{width=0.7\textwidth}{Snell's law governs the
  refraction of rays across horizontal interfaces.}

In the next lecture, we will discuss how Snell's law and other
fundamental principles of geometrical seismics get applied in practice
for numerical wavefront propagation.

\section{Acknowledgments and references}
An authoritative reference on all aspects of the seismic ray theory is
\cite[]{cerveny}. Useful material can also be found in
\cite[]{STI00-00-03630363}. The level-set approach to eikonal-type
equations is developed by \cite{sethian}. Sergey Goldin and James
Sethian had a major influence on my understanding of geometrical seismics.

\bibliographystyle{seglike}
\bibliography{times,SEG}

%\end{document}

