\author{Pierre Fermat}
%%%%%%%%%%%%%%%%%%%%%%
\title{Homework 3}

\begin{abstract}
This homework has two parts. In the theoretical part, you will derive
analytical reflection traveltimes using Fermat's principle. In the
computational part, you will experiment with numerical traveltime
computation methods and measure their accuracy.
\end{abstract}

\definecolor{frame}{rgb}{0.905,0.905,1.0}
\lstset{language=Python,backgroundcolor=\color{frame},showstringspaces=false}

\section{Introduction}

Start by running
\begin{verbatim}
> cd ~/geo391
> svn update
\end{verbatim}

\section{Theoretical part}

You can either write your answers on paper or edit them in
\verb#~/geo391/hw2/paper.tex#. Please show all the mathematical
derivations that you do.

In the theoretical part of this homework, you will derive an
analytical expression for the reflection traveltime for a special
reflector geometry. 

\begin{enumerate}
\item Consider a reflector (a salt dome) with a hyperbolic shape
\begin{equation}
  \label{eq:dome}
  z(x) = \sqrt{a^2 + b^2\,x^2}
\end{equation}
Place the source and receiver at the surface and assume a constant
isotropic velocity $V$. Express the reflection traveltime as a sum of
the two-point traveltimes from $\{s,0\}$ to $\{x,z(x)\}$ and from
$\{x,z(x)\}$ to $\{r,0\}$.
\begin{equation}
  \label{eq:that}
  \widehat{T}(s,r) = 
\end{equation}

\item What principle can you use to constrain the position of the
  reflection point? Write the corresponding equation.

\item Using the explicit analytical form~(\ref{eq:dome}), solve your
  equation for $x$.

\item Substitute the value of $x$ in equation~(\ref{eq:that}) to find
  an explicit analytical expression for the reflection traveltime as a
  function of $s$, $r$, $a$, and $b$.

\item Check your formula by considering the limit of $a=0$ when the
  hyperbolic reflector turns into a line.
\end{enumerate}

\section{Computational part}

\begin{enumerate}

\item In the first part of the computational assignment, you will
  investigate the numerical accuracy of finite-difference eikonal solvers.

  \inputdir{eikonal}
  
  Figure~\ref{fig:exact} shows wavefronts in a constant velocity
  gradient medium computed with the analytical two-point ray tracing
  formula from Homework 2.

  \plot{exact}{width=\textwidth}{Wavefronts in a constant velocity
    gradient medium computed with the analytical formula.}

  By computing traveltimes numerically at different sampling intervals
  and comparing the numerical result with the analytical one, we can
  get an experimental estimate of the numerical error behavior. The
  error is shown in Figure~\ref{eq:err}. The right plot in
  Figure~\ref{eq:err} displays the error in log-log coordinates. The
  slope of the line in these coordinates shows directly the rate of
  convergence of the numerical method. For example, a first-order
  accurate method should have a slope of one.

  \plot{err}{width=\textwidth}{Left: average error of the
    finite-difference eikonal solver as a function of grid spacing.
    Right: the same on a log-log plot. The slope of the curve on the
    log-log plot indicates the order of the numerical accuracy.}

  We can also measure the computational time of the method in its
  relationship with the number of grid points (Figure~\ref{fig:cpu}).
  If the computational time grows linearly with the grid size, the
  method is said to have an $O(N)$ efficiency.

  \sideplot{cpu}{width=\textwidth}{Computational time of the
    finite-difference eikonal solver as a function of the grid size.}

  \begin{enumerate}
  \item Change directory 
\begin{verbatim}
> cd ~/geo391/hw3/eikonal
\end{verbatim}
  \item Run
\begin{verbatim}
> scons view
\end{verbatim}
    to generate figures and display them on your screen.  
  \item In the \texttt{SConstruct} file, find the parameter that defines the order of accuracy for the eikonal solver. Change the order from $1$ to $2$ and recompute the results. Does the numerical accuracy change? What is the experimental order of accuracy? Does the computational time change? 
  \item Instead of an analytical solution for a constant velocity gradient, let us try an analytical solution for a constant gradient of slowness squared. Uncomment the part of the \texttt{SConstruct} file that defines a velocity model with the constant gradient of slowness squared. Change the analytical solution for the two-point traveltime to an appropriate formula. Recompute the figures and check your results.
    
  \lstinputlisting[frame=single]{eikonal/SConstruct}

  \item After you are done, run
\begin{verbatim}
> scons lock
> scons -c
\end{verbatim} 
  \end{enumerate}
  
\item In the second-part of the computational assignment, you will
  investigate the numerical accuracy of one-point ray tracing.

  \inputdir{raytrace}

  Figure~\ref{eq:ray} shows the same constant velocity gradient model
  and a ray computed by analytical one-point ray tracing using
  formulas from Homework 2. The ray has the shape of a circular arc.

  \plot{ray}{width=\textwidth}{Analytical ray tracing in a constant
    velocity gradient medium. A single ray is displayed.}

  Figure~\ref{eq:rayerr} displays errors of numerical ray tracing for
  different sizes of the time step. 

  \sideplot{rayerr}{width=\textwidth}{Average error of the
    one-point ray tracer as a function of the time step size.}

\begin{enumerate}
  \item Change directory 
\begin{verbatim}
> cd ~/geo391/hw3/raytrace
\end{verbatim}
  \item Run
\begin{verbatim}
> scons view
\end{verbatim}
    to generate figures and display them on your screen.  
  \item Edit the \texttt{SConstruct} file to plot the numerical error
    on a log-log scale similarly to Figure~\ref{fig:err}. What is the
    order of accuracy of this numerical method?

    \lstinputlisting[frame=single]{raytrace/SConstruct}

  \item After you are done, run
\begin{verbatim}
> scons lock
> scons -c
\end{verbatim} 
  \end{enumerate}
   
\item In some cases, no analytical solution exists for measuring the
  accuracy of your numerical scheme. You can still get an estimate of
  the error by computing the solution at different grid sizes and
  comparing them with computations at the finest scale.

  \inputdir{sigsbee}

  Figure~\ref{fig:eiko} shows numerical wavefronts from a
  finite-difference solution of the eikonal equation in the Sigsbee
  velocity model.

  \plot{eiko}{width=\textwidth}{Wavefronts in the Sigsbee velocity
    model computed with a finite-difference eikonal solver.}

  \begin{enumerate}    
  \item Change directory 
\begin{verbatim}
> cd ~/geo391/hw3/sigsbee
\end{verbatim}
  \item Run
\begin{verbatim}
> scons view
\end{verbatim}
    to generate figures and display them on your screen.  
  \item Edit the \texttt{SConstruct} file to include subsampling of
    the velocity model and plotting numerical error versus sampling similar to the previous examples.
    
\lstinputlisting[frame=single]{sigsbee/SConstruct}

\item Include your figure in \verb#~/geo391/hw2/paper.tex# following an analogy with previous examples.
  
\item After you are done, run
\begin{verbatim}
> scons lock
> scons -c
\end{verbatim}

\end{enumerate}

\item Edit the file
\verb#~/geo391/hw2/paper.tex# in your favorite editor and change the
first line to have your name instead of Euler's. Run
\begin{verbatim}
> scons pdf
\end{verbatim}
and submit your result (file \texttt{paper.pdf}) on paper or by
e-mail.

\end{enumerate}
