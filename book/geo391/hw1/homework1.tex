\author{Paul Sava}
\title{Homework 1}

\def\bea{\begin{eqnarray}}
\def\eea{  \end{eqnarray}}

\def\beq{\begin{equation}}
\def\eeq{  \end{equation}}

\def\req#1{(\ref{#1})}

\def\lp{\left (}
\def\rp{\right)}

\def\lb{\left [}
\def\rb{\right]}

\def\pbox#1{ \fbox {$ \displaystyle #1 $}}

% ------------------------------------------------------------
% trace
\def\tr{\texttt{tr}\;}

% divergence
\def\DIV#1{\nabla \cdot #1}

% curl
\def\CURL#1{\nabla \times #1}

% gradient
\def\GRAD#1{\nabla #1}

% Laplacian
\def\LAPL#1{\nabla^2 #1}
% ------------------------------------------------------------

% stress tensor
\def\ssten{{\bf \sigma}}

\def\ssmat{
\lp \matrix {
 \sigma_{11} &  \sigma_{12}   &  \sigma_{13} \cr
 \sigma_{12} &  \sigma_{22}   &  \sigma_{23} \cr
 \sigma_{13} &  \sigma_{23}   &  \sigma_{33} \cr
} \rp
}

% strain tensor
\def\eeten{{\bf \epsilon}}

\def\eemat{
\lp \matrix {
 \epsilon_{11} &  \epsilon_{12}   &  \epsilon_{13} \cr
 \epsilon_{12} &  \epsilon_{22}   &  \epsilon_{23} \cr
 \epsilon_{13} &  \epsilon_{23}   &  \epsilon_{33} \cr
} \rp
}

% elastic tensor
\def\CC{{\bf C}}

% identity tensor
\def\I{\;{\bf I}}

% particle displacement vector
\def\uu{{\bf u}}

% particle velocity vector
\def\vv{{\bf v}}

% force vector
\def\ff{{\bf f}}

% distance vector
\def\xx{{\bf x}}

% wavenumber vector
\def\kk{{\bf k}}

% normal vector
\def\nn{{\bf n}}

% source vector
\def\ss{{\bf s}}

% receiver vector
\def\rr{{\bf r}}

\def\Fop#1{\mathcal{F}     \lb #1 \rb}
\def\Fin#1{\mathcal{F}^{-1}\lb #1 \rb}

\def\dtwo#1#2  {\frac{\partial^2 #1}{\partial #2^2         }   }
\def\done#1#2  {\frac{\partial   #1}{\partial #2           }   }


% ------------------------------------------------------------

\section{Prerequisites}

\begin{enumerate}
\item Log into your class account and prepare for the assignment:

\begin{itemize}
\item \texttt{cd ~/geo391} and run \texttt{scons update} \par
to bring the latest class assignment to your local directory.
\item \texttt{cd ~/geo391/hw1} \par
to change directory to your current homework.
\end{itemize}

\item Open the file \texttt{homework1} in your favorite text editor
and change the name at the top of the file.
Run \texttt{scons homework1.read} to build and read this file.

\item This homework has $2$ parts. 
Part $A$ is mandatory (max $100$ points).
Part $B$ is optional but with extra credit (max $50$ points).

If you answer only part $A$, \texttt{cd hw1a},
run the commands in this directory according to the 
instructions. When you are done, run \texttt{scons lock} and
return to the \texttt{hw1} directory.

If you also answer part $B$, \texttt{cd hw1b},
run the commands in this directory according to the 
instructions. When you are done, run \texttt{scons lock} and
return to the \texttt{hw1} directory.

When you are done, uncomment the ``plot'' lines in this document
to include your figures, add your answers, and run
\texttt{scons homework1.read}. 
When you are happy with your answer, print the document and
bring it to class. Don't forget to update your name on the file.

\end{enumerate}

% ------------------------------------------------------------

\section{Part A: SConstruct assignment}

\begin{enumerate}

\item In this assignment, you will get familiar with 
the computer library called \texttt{RSF} and the reproducible testing
environment using \texttt{scons}.
You will run a number of commands defined in the 
\texttt{SConstruct} file, modify them, 
observe and comment on the results, and finally include comments
in this homework document.
We will use this setup in future labs, so try to get familiar with
it as much as possible to avoid being distracted in future labs.

\item \texttt{cd ~/geo391/hw1/hw1a}

\item open the \texttt{SConstruct} file in your favorite text editor.

\item Observe typical commands in the \texttt{SConstruct} file:
\begin{itemize}
\item \texttt{Flow} describes a flow with which we produce 
one or more outputs using one or more inputs and execution rules and
parameters. The generic format is:
\par
\texttt{ Flow(output,input,command) }
\par
For example
\par
\texttt{ Flow('mm0','vel','window n2=\%(nx)d min2=\%(ox)g' \% par)}
\par
produces output \texttt{mm0.rsf} from input \texttt{vel.rsf} using the rule
\par
\texttt{'window n2=\%(nx)d min2=\%(ox)g'} where parameters
\texttt{nx} and \texttt{ox} are extracted from the dictionary \texttt{par}.

\item \texttt{Plot} produces an output plot file 
in \texttt{vplot} format, that can be displayed using the program 
\texttt{xtpen}.
For example
\par
\texttt{ Plot(mm,igrey('allpos=y bias=4715',par))}
\par
produces output \texttt{mm.vpl} from input 
\texttt{mm.rsf} using the rule defined by the function 
\texttt{igrey} with arguments \texttt{allpos=y bias=4715}.

\item \texttt{Result} is similar with \texttt{Plot} except
that outputs

\end{itemize}

\item In this homework, you will analyze wave propagation modeled
using the acoustic wave-equation. 
You will observe and discuss $3$ pre-defined models and 
$2$ pre-defined acquisition geometries (experiments).
Feel free to add more examples by replicating the rules in the
\texttt{SConstruct} file.

\item 
For each model-experiment pair, you will observe
plots for the model and a wavefield snapshot (acquisition overlain),
and recorded data.

Each triplet model-wavefield-data is annotated with a balloon
label with a letter (e.g. A). 
Modify the \texttt{x0} and \texttt{y0} parameters in the 
\texttt{SConstruct} file to indicate a consistent event across 
all 3 figures for each model. By default, the labels are in
arbitrary positions that do not point to anything in particular.

Use section ``Discussion'' to list all of them in this document 
and describe what you wanted to indicate. For example, you can 
point at a reflector on the model plot,
a wave reflected from this reflector in the wavefield plot and
the same reflected wave in the acquired data.

You can identify direct arrivals, reflected arrivals, 
multiple reflection or anything else that you might find 
interesting.
This exercise is about being creative and seeing things,
as well as getting used to the  \texttt{SConstruct} setup.

\end{enumerate}

% ------------------------------------------------------------

\inputdir{hw1a}

\section{Discussion}

\begin{enumerate}

\item Model 0, experiment 0, label A
%      \plot*{me-0-0}{width=3.0in}{Model 0, experiment 0}
%      \plot*{ww-0-0}{width=3.0in}{Model 0, experiment 0}
%      \plot*{dd-0-0}{width=3.0in}{Model 0, experiment 0}

\item Model 0, experiment 1, label B
%      \plot*{me-0-1}{width=3.0in}{Model 0, experiment 1}
%      \plot*{ww-0-1}{width=3.0in}{Model 0, experiment 1}
%      \plot*{dd-0-1}{width=3.0in}{Model 0, experiment 1}

\item Model 1, experiment 0, label C
%      \plot*{me-0-0}{width=3.0in}{Model 1, experiment 0}
%      \plot*{ww-0-0}{width=3.0in}{Model 1, experiment 0}
%      \plot*{dd-0-0}{width=3.0in}{Model 1, experiment 0}

\item Model 1, experiment 1, label D
%      \plot*{me-0-0}{width=3.0in}{Model 1, experiment 1}
%      \plot*{ww-0-0}{width=3.0in}{Model 1, experiment 1}
%      \plot*{dd-0-0}{width=3.0in}{Model 1, experiment 1}

\item Model 2, experiment 0, label E
%      \plot*{me-0-0}{width=3.0in}{Model 2, experiment 0}
%      \plot*{ww-0-0}{width=3.0in}{Model 2, experiment 0}
%      \plot*{dd-0-0}{width=3.0in}{Model 2, experiment 0}

\item Model 2, experiment 1, label F
%      \plot*{me-0-0}{width=3.0in}{Model 2, experiment 1}
%      \plot*{ww-0-0}{width=3.0in}{Model 2, experiment 1}
%      \plot*{dd-0-0}{width=3.0in}{Model 2, experiment 1}

\end{enumerate}

When you are done, run \texttt{scons lock} to archive your pictures 
so they can be included in your answer.
Don't forget to uncomment the plots in \texttt{homework1.tex}.

% ------------------------------------------------------------

\inputdir{hw1b}

\section{Part B: programming assignment (extra credit)}

\begin{enumerate}

\item \texttt{cd ~/geo391/hw1/hw1b}

\item The file \texttt{AFDM.c} is a \texttt{C} program
for time-domain acoustic finite-difference modeling.
The \texttt{SConstruct} file contains all the rules for compiling and
running tests with this program. Every change in the program will
automatically recompile it and create an updated executable,
so you don't need to worry about doing anything special about this.
Just make your changes and run the tests using \texttt{scons}.

The program is set for models with constant density.
You will modify this program to include the variable-density term.
Consult the class notes for details.

\item
You will use the Python function \texttt{amodel} (already
defined in the SConstruct file) to run tests for both constant-density
and variable-density modeling.

The lines
\par
\texttt{ amodel('dv','wv','wav','vv','rr','sc','rc','free=n dens=n') }
\par
\texttt{ amodel('dr','wr','wav','vv','rr','sc','rc','free=n dens=y') }
\par
generate
\begin{itemize}
\item the data files \texttt{dv} and \texttt{dr} 
corresponding to constant-density
and variable-density, respectively. 
You are basically running the same program, but the flag
\texttt{dens=y} indicates if you use or ignore density.

\item the wavefield files \texttt{dv} and \texttt{dr} 
corresponding to constant-density
and variable-density, respectively. 
\end{itemize}

The source wavelet is in \texttt{wav},
the velocity in \texttt{vv}, and
the density  in \texttt{rr}. 

The   source coordinates are in \texttt{sc} and
the receiver coordinates are in \texttt{rc}.

All input files are predefined for you, so you don't need to change them.
However, feel free to duplicate the rules in the \texttt{SConstruct} 
file and experiment at will.

\item
When you are finished modifying the program, run the commands:

\begin{itemize}

\item \texttt{scons mcomp.view} to view the velocity and density models.

\item \texttt{scons wvmovie.view} to view a movie of the pulse in
\texttt{wav} propagate through the medium with constant density.

\item \texttt{scons wrmovie.view} to view a movie of the pulse in
\texttt{wav} propagate through the medium with variable density.

\item \texttt{scons wcomp} to compare snapshots of the wavefield
for the two cases.

\item \texttt{scons dcomp} to compare the data recorded at the surface
for the two cases.

\end{itemize}

Initially, the left and right panels of \texttt{wcomp} and 
\texttt{dcomp} are identical. After you modify the program, the 
two panels will change. Are you sure that you obtained correct 
results? If yes, explain why?

\end{enumerate}

When you are done, run \texttt{scons lock} to archive your pictures 
so they can be included in your answer.
Don't forget to uncomment the plots in \texttt{homework1.tex}.

%\plot{mcomp}{width=6.0in}{Model}
%\plot{wcomp}{width=6.0in}{Wavefield}
%\plot{dcomp}{width=6.0in}{Data}

% ------------------------------------------------------------

\section{Wrap-up}

When you are done, rebuild this document and 
print your answers. Remember to uncomment the plots
so the figures you created are included in the document.

This homework is due in class $2$ weeks
from the day in which it was handed-out.

% ------------------------------------------------------------

\section{Resources}

\begin{itemize}
\item RSF
\item scons
\item Python
\item C
\end{itemize}
