\author{Thomas Bayes}
%%%%%%%%%%%%%%%%%%%%
\title{Homework 5}

\begin{abstract}
  This homework has two parts. 
  \begin{enumerate}
  \item Theoretical questions related to covariance estimation.
  \item Missing data interpolation for ocean floor topography using 
  covariance estimation. 
  \end{enumerate}
\end{abstract}

\section{Prerequisites}

Completing the computational part of this homework assignment requires
\begin{itemize}
\item \texttt{Madagascar} software environment available from \\
\url{http://www.ahay.org/}
\item \LaTeX\ environment with \texttt{SEG}\TeX\ available from \\ 
\url{http://www.ahay.org/wiki/SEGTeX}
\end{itemize}
To do the assignment on your personal computer, you need to install
the required environments. Please ask for help if you don't know where
to start.

The homework code is available from the \texttt{Madagascar} repository
by running
\begin{verbatim}
svn co http://svn.code.sf.net/p/rsf/code/trunk/book/geo391/hw5
\end{verbatim}

\section{Theory}

\begin{enumerate}
\item The following equality for the posterior model covariance was given in lecture notes without a proof:
\begin{equation}
\label{eq:cpost}
\left(\mathbf{F}^T\,\mathbf{C}_n^{-1}\,\mathbf{F} + \mathbf{C}_{m}^{-1}\right)^{-1}=
\mathbf{C}_m-\mathbf{C}_m\,\left(\mathbf{F}\,\mathbf{C}_{m}\,\mathbf{F}^T + \mathbf{C}_n\right)^{-1}\,\mathbf{C}_m\;.
\end{equation}
Prove it.
\item Suppose that we use the gradient operator for data interpolation:
\begin{equation}
\label{eq:grad}
\min\,\left|\nabla \mathbf{m}\right|^2\;.
\end{equation}  
This approach roughly corresponds to minimizing the surface area and
represents the behavior of a soap film or a thin rubber sheet.

The corresponding inverse model covariance operator is the negative Laplacian
$\mathbf{C}_m^{-1}=\nabla^T\,\nabla=-\nabla^2$. The corresponding
covariance operator corresponds to the Green's function $G(\mathbf{x})$ that solves
\begin{equation}
\label{eq:green}
-\nabla^2 G = \delta(\mathbf{x}-\mathbf{x}_0)\;.
\end{equation}
In 2-D, the Green's function has the form 
\begin{equation}
\label{eq:g2d}
G(\mathbf{x}) = \displaystyle k - \frac{\ln |\mathbf{x}-\mathbf{x}_0|}{2\pi}
\end{equation}
with some constant $k$.

To derive equation~(\ref{eq:g2d}), we can introduce polar coordinates
around $\mathbf{x}_0$ with the radius $r= |\mathbf{x}-\mathbf{x}_0|$
and note that the Laplacian operator for a radially-symmetric function
$\phi(r)$ in polar coordinates takes the form
\begin{equation}
\label{eq:polar}
\nabla^2 \phi = \displaystyle \frac{1}{r}\,\frac{d}{dr}\,\left(r\,\frac{d \phi}{dr}\right)
\end{equation}  
Away from the point $\mathbf{x}_0$, solving
\begin{equation}
\label{eq:away}
\frac{1}{r}\,\frac{d}{dG}\,\left(r\,\frac{d \phi}{dr}\right) = 0
\end{equation}
leads to $G(r) = k_1\,\ln r + k_2$. To find the constant $k_1$, we can
integrate $\nabla^2 G$ over a circle with some small radius $\epsilon$
around the origin and apply the Green's theorem
\begin{equation}
\label{eq:integrate}
-1 = \iint \nabla^2 G dx\,dy = \oint \nabla G \cdot \vec{ds} = 
\int\limits_{0}^{2\pi} \left.\frac{\partial G}{\partial r}\right|_{r=\epsilon}\,\epsilon\,d\theta = 2\pi\,k_1\;.
\end{equation}

Derive the model covariance function $G(\mathbf{x})$ which corresponds
to replacing equation~(\ref{eq:grad}) with equation
\begin{equation}
\label{eq:lap}
\min\,\left|\nabla^2 \mathbf{m}\right|^2
\end{equation} 
and approximates the behavior of a thin elastic plate.

\end{enumerate} 

\section{Missing ocean-floor data interpolation}
\inputdir{seabeam}

SeaBeam is an apparatus for measuring water depth both directly under
a boat and somewhat off to the sides of the boat's track. In this part
of the assignment, we will use a benchmark dataset from \cite{gee}:
SeaBeam data from a single day of acquisition. The original data are
shown in Figure~\ref{fig:data}a.

\plot{data}{width=0.8\textwidth}{(a) Water depth measurements from one day
  of SeaBeam acquisition. (b) Mask for locations of known data.}

Program~\texttt{interpolate.c} implements two alternative methods:
regularized inversion using convolution with a multi-dimensional
filter and preconditioning, which uses the inverse operation
(recursive deconvolution or polynomial division on a helix) for model
reparameterization.

Our first attempt is to use the Laplacian-factor filter from the
previous homework. The results from the two methods are shown in
Figure~\ref{fig:lseabeam}. They are not very successful in hiding
the ``acquisition footprint''.

\plot{lseabeam}{width=0.8\textwidth}{Left: missing data interpolation using regularization by convolution with a Laplacian factor. 
Right: missing data interpolation using model reparameterization by
deconvolution (polynomial division) with a Laplacian factor.}

Next, we will try to estimate the model covariance from the available
data. The covariance can be estimated using two alternative methods:
the fast Fourier transform (FFT) method \cite[]{jim} and the
prediction-error filter (PEF) method \cite[]{morgan}. Random
realizations of model patterns from the two methods and the
corresponding 2-D Fourier spectra are shown in Figures~\ref{fig:rand2}
and~\ref{fig:rand}.

\plot{rand2}{width=0.8\textwidth}{Left: data pattern generated using FFT method for covariance estimation.
Right: its Fourier spectrum.}

\plot{rand}{width=0.8\textwidth}{Left: data pattern generated using PEF method for inverse covariance estimation.
Right: its Fourier spectrum.}


Our new attempt to interpolate missing data using the helical
prediction-error filter is shown in Figure~\ref{fig:seabeam}.

\plot{seabeam}{width=0.8\textwidth}{Left: missing data interpolation using regularization by 
convolution with a prediction-error filter. 
Right: missing data interpolation using model reparameterization by
deconvolution (polynomial division) with a prediction-error filter.}

\lstset{language=c,numbers=left,numberstyle=\tiny,showstringspaces=false}
\lstinputlisting[frame=single,title=seabeam/interpolate.c]{seabeam/interpolate.c}

\newpage

\lstset{language=python,numbers=left,numberstyle=\tiny,showstringspaces=false}
\lstinputlisting[frame=single,title=seabeam/SConstruct]{seabeam/SConstruct}

Your task:
\begin{enumerate}
\item Change directory to \texttt{hw5/seabeam}
\item Run 
\begin{verbatim}
scons view
\end{verbatim}
to reproduce the figures on your screen.
\item Modify the \texttt{SConstruct} file to implement the following tasks:
\begin{enumerate}
\item Find the number of iterations required for both methods shown in Figure~\ref{fig:seabeam} to achieve similar results.
\item To provide a more quantitative comparison, modify the
\texttt{interpolate.c} program to output a measure of convergence
(such as the least-squares misfit) as a function of the number
of iterations\footnote{You can study the interfaces to
the \texttt{sf\_solver} and
\texttt{sf\_solver\_prec} programs. See
\url{https://sourceforge.net/p/rsf/code/HEAD/tree/trunk/api/c/bigsolver.c}}. Generate figures comparing convergence with and without preconditioning.
\end{enumerate}
\item A method for generating multiple realizations of missing data interpolation is:
\begin{enumerate}
\item Start with a random realization $\mathbf{m}_0$ such as the one shown in \ref{fig:rand} or \ref{fig:rand2}.
\item Instead of estimating $\mathbf{m}$ such that $\mathbf{K}\,\mathbf{m} = \mathbf{d}$, 
      estimate $\mathbf{x}$ such that 
\[
\mathbf{K}\,\mathbf{x} = \mathbf{d}- \mathbf{K}\,\mathbf{m_0}\;.
\]
\item The estimate for $\mathbf{m}$ is then  $\widehat{\mathbf{m}} = \widehat{\mathbf{x}}+\mathbf{m_0}$.
\end{enumerate}
Implement several realizations of missing data interpolation using
several realizations of $\mathbf{m}_0$. You can do it by modifying
either \texttt{SConstruct} or \texttt{interpolate.c}.
\item Include your results in the paper.
\item \textbf{EXTRA CREDIT} for implementing missing data interpolation using the FFT method for model covariance.
\end{enumerate}

\section{Completing the assignment}

\begin{enumerate}
\item Change directory to \texttt{hw5}.
\item Edit the file \texttt{paper.tex} in your favorite editor and change the
  first line to have your name instead of Bayes's.
\item Run
\begin{verbatim}
sftour scons lock
\end{verbatim}
to update all figures.
\item Run
\begin{verbatim}
sftour scons -c
\end{verbatim}
to remove intermediate files.
\item Run
\begin{verbatim}
scons pdf
\end{verbatim}
to create the final document.
\item Submit your result (file \texttt{paper.pdf}) on paper or by
e-mail.
\end{enumerate}

\bibliographystyle{seg}
\bibliography{bayes}
