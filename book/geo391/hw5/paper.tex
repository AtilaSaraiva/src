\author{Eliakim Hastings Moore}
%%%%%%%%%%%%%%%%%%%%%%%
\title{Homework 5}

\begin{abstract}
  This homework has the following parts:
  \begin{enumerate}
  \item A theoretical question related to covariance estimation.
  \item Missing data interpolation using compressive properties of the 2-D Fourier transform.
  \item Compression of sand dune images.
  \end{enumerate}
\end{abstract}

\section{Prerequisites}

Completing the computational part of this homework assignment requires
\begin{itemize}
\item \texttt{Madagascar} software environment available from \\
\url{http://www.ahay.org/}
\item \LaTeX\ environment with \texttt{SEG}\TeX\ available from \\ 
\url{http://www.ahay.org/wiki/SEGTeX}
\end{itemize}
To do the assignment on your personal computer, you need to install
the required environments. Please ask for help if you don't know where
to start.

The homework code is available from the \texttt{Madagascar} repository
by running
\begin{verbatim}
svn co http://svn.code.sf.net/p/rsf/code/trunk/book/geo391/hw5
\end{verbatim}

\section{Theory}

Suppose that we use the gradient operator for data interpolation:
\begin{equation}
\label{eq:grad}
\min\,\left|\nabla \mathbf{m}\right|^2\;.
\end{equation}  
This approach roughly corresponds to minimizing the surface area and
represents the behavior of a soap film or a thin rubber sheet.

The corresponding inverse model covariance operator is the negative Laplacian
$\mathbf{C}_m^{-1}=\nabla^T\,\nabla=-\nabla^2$. The corresponding
covariance operator corresponds to the Green's function $G(\mathbf{x})$ that solves
\begin{equation}
\label{eq:green}
-\nabla^2 G = \delta(\mathbf{x}-\mathbf{x}_0)\;.
\end{equation}
In 2-D, the Green's function has the form 
\begin{equation}
\label{eq:g2d}
G(\mathbf{x}) = \displaystyle A - \frac{\ln |\mathbf{x}-\mathbf{x}_0|}{2\pi}
\end{equation}
with some constant $A$.

To derive equation~(\ref{eq:g2d}), we can introduce polar coordinates
around $\mathbf{x}_0$ with the radius $r= |\mathbf{x}-\mathbf{x}_0|$
and note that the Laplacian operator for a radially-symmetric function
$\phi(r)$ in polar coordinates takes the form
\begin{equation}
\label{eq:polar}
\nabla^2 \phi = \displaystyle \frac{1}{r}\,\frac{d}{dr}\,\left(r\,\frac{d \phi}{dr}\right)
\end{equation}  
Away from the point $\mathbf{x}_0$, solving
\begin{equation}
\label{eq:away}
\frac{1}{r}\,\frac{d}{dr}\,\left(r\,\frac{d G}{dr}\right) = 0
\end{equation}
leads to $G(r) = A + B\,\ln r$. To find the constant $B$, we can
integrate $\nabla^2 G$ over a circle with some small radius $\epsilon$
around the origin and apply the Green's theorem
\begin{equation}
\label{eq:integrate}
-1 = \iint \nabla^2 G dx\,dy = \oint \nabla G \cdot \vec{ds} = 
\int\limits_{0}^{2\pi} \left.\frac{\partial G}{\partial r}\right|_{r=\epsilon}\,\epsilon\,d\theta = 2\pi\,B\;.
\end{equation}

Derive the model covariance function $G(\mathbf{x})$ which corresponds
to replacing equation~(\ref{eq:grad}) with equation
\begin{equation}
\label{eq:lap}
\min\,\left|\nabla^2 \mathbf{m}\right|^2
\end{equation} 
and approximates the behavior of a thin elastic plate.

\section{Projection onto convex sets}
\inputdir{pocs}

\sideplot{hole}{width=\textwidth}{Seismic depth slice
  after removing selected parts of the data.}

The goal of the next exercise is to figure out if one can use compactness
of the Fourier transform to reconstruct missing data. The missing
parts are created artificially by cutting holes in the original data
(Figure~\ref{fig:hole}).

\multiplot{2}{fft-data,fft-hole}{width=0.4\textwidth}{Fourier transform of the original data (a) and data with holes
  with holes (b). The absolute value is displayed}

\sideplot{fft-mask}{width=0.8\textwidth}{Fourier-domain mask for selecting a convex set.}

Figures~\ref{fig:fft-data} and~\ref{fig:fft-hole} show the digital
Fourier transform of the original data and the data with holes. We
observe again that the support of the data in the Fourier domain is
compact thanks to the data smoothness. Cutting holes in the physical
domain creates discontinuities that make the Fourier response spread
beyond the original support. Figure~\ref{fig:fft-mask} shows a
Fourier-domain mask designed to contain the support of the original
data.

To accomplish the task of missing data interpolation, we will use an
iterative method known as POCS (\emph{projection onto convex
sets}). By definition, a convex set $\mathcal{C}$ is a set of
functions such that, for any $f_1(\mathbf{x})$ and $f_2(\mathbf{x})$
from the set, $g(\mathbf{x}) = \lambda\,f_1(\mathbf{x}) +
(1-\lambda)\,f_2(\mathbf{x})$ (for $0 \le \lambda \le 1$) also belongs
to the set. A projection onto a convex set means finding a function in
the set that is of the shortest distance to the given function. The
POCS theorem states that if one wants to find a function that belongs
to the intersection of two convex sets $C_1$ and $C_2$, the task can
be accomplished iteratively by alternating projections onto the two
sets.

In our example, $C_1$ is the set of all functions that are equal to
the known data outside of the holes. $C_2$ is the set of all functions
that have a predifined compact support in the Fourier domain (and
therefore are smooth in the physical domain). The algorithm consists
of the following steps:
\begin{enumerate}
\item Apply 2-D Fourier transform. 
\item Multiply by a Fourier-transform mask to enforce compact support.
\item Apply inverse 2-D Fourier transform.
\item Replace data outside of the holes with known data.
\item Repeat.
\end{enumerate}
The output after 5 iterations is shown in Figure~\ref{fig:pocs}.

\sideplot{pocs}{width=\textwidth}{Missing data interpolated by
  iterative projection onto convex sets.}

\lstinputlisting[frame=single]{pocs/SConstruct}

Your task:
\begin{enumerate}
\item Change directory to \texttt{hw5/pocs}
\item Run 
\begin{verbatim}
scons view
\end{verbatim}
to reproduce the figures on your screen.
\item Additionally, you can run
\begin{verbatim}
scons pocs.vpl
\end{verbatim}
to see a movie of different iterations.
\item By modifying appropriate parameters in the \texttt{SConstruct} file and repeating computations,
find out
\begin{enumerate}
\item How many iterations are required for convergence?
\item How large can you make the holes and still be able to achieve a reasonably good reconstruction?
\end{enumerate}
\item \textbf{EXTRA CREDIT} for finding a different convex set or a different iteration strategy for either faster or more accurate missing data reconstruction.
\end{enumerate}

\section{Compression of Sand Dune Images}
\inputdir{dunes}

\sideplot{dune}{width=\textwidth}{Image of sand dunes in a river.}

Figure~\ref{fig:dune} shows an image of sand dunes at the bottom of a
river\footnote{courtesy of Ryan Ewing}. In this part of the assignment, you
will try to compress the image by applying different transforms.

\multiplot{2}{cos,sort}{width=0.45\textwidth}{(a) Sand dunes image in the cosine Fourier domain. (b) Fourier coefficients sorted by absolute value and displayed on the logarithmic (decibel) scale.}

Figure~\ref{fig:cos} shows the image after applying the \emph{cosine
transform} (a version of the Fourier transform that keeps coefficients
real). Notice both compactness and sparsity in the Fourier domain. To
analyze the sparsity pattern, Figure~\ref{fig:sort} shows Fourier
coefficients after sorting them by absolute value. The rate 
of coefficient decay is a measure of sparsity.

\plot{inv}{width=\textwidth}{(a) Sand dunes image reconstructed after thresholding. (b) Compression noise.}

Figure~\ref{fig:inv} shows the result of shrinkage (soft thresholding)
of Fourier coefficients using 1\% threshold and the difference between
the reconstructed image and the true image.

Your task:
\begin{enumerate}
\item Change directory to \texttt{hw5/dunes}
\item Run 
\begin{verbatim}
scons view
\end{verbatim}
to reproduce the figures on your screen.
\item Modify the \texttt{SConstruct} file to adjust the threshold percentage to the level that makes noise negligible.
\item  Modify the \texttt{SConstruct} file to use DWT (the \emph{digital wavelet transform}) instead of the cosine transform. 
Compare the results. Which transform has more sparsity and provides better compression? 
\item \textbf{EXTRA CREDIT} for finding and implementing a transform with an even better compression.
\end{enumerate}

\newpage

\lstset{language=python,numbers=left,numberstyle=\tiny,showstringspaces=false}
\lstinputlisting[frame=single,title=dunes/SConstruct]{dunes/SConstruct}

\newpage

\section{Completing the assignment}

\begin{enumerate}
\item Change directory to \texttt{hw5}.
\item Edit the file \texttt{paper.tex} in your favorite editor and change the
  first line to have your name instead of Moore's.
\item Run
\begin{verbatim}
sftour scons lock
\end{verbatim}
to update all figures.
\item Run
\begin{verbatim}
sftour scons -c
\end{verbatim}
to remove intermediate files.
\item Run
\begin{verbatim}
scons pdf
\end{verbatim}
to create the final document.
\item Submit your result (file \texttt{paper.pdf}) on paper or by
e-mail.
\end{enumerate}

