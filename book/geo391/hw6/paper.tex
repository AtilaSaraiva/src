\author{Eliakim Hastings Moore}
%%%%%%%%%%%%%%%%%%%%
\title{Homework 6}

\begin{abstract}
  This homework has three computational parts. 
  \begin{enumerate}
  \item Compression of sand dune images.
  \item Attenuation of surface-wave noise in 
        seismic data collected over sand dunes using match filtering.
  \item Predicting data patterns using match filtering.
  \end{enumerate}
\end{abstract}

\section{Prerequisites}

Completing the computational part of this homework assignment requires
\begin{itemize}
\item \texttt{Madagascar} software environment available from \\
\url{http://www.ahay.org/}
\item \LaTeX\ environment with \texttt{SEG}\TeX\ available from \\ 
\url{http://www.ahay.org/wiki/SEGTeX}
\end{itemize}
To do the assignment on your personal computer, you need to install
the required environments. Please ask for help if you don't know where
to start.

The homework code is available from the \texttt{Madagascar} repository
by running
\begin{verbatim}
svn co http://svn.code.sf.net/p/rsf/code/trunk/book/geo391/hw6
\end{verbatim}

\section{Compression of Sand Dune Images}
\inputdir{dunes}

\sideplot{dune}{width=\textwidth}{Image of sand dunes in a river.}

Figure~\ref{fig:dune} shows an image of sand dunes at the bottom of a
river\footnote{courtesy of Ryan Ewing}. In this part of the assignment, you
will try to compress the image by applying different transforms.

\multiplot{2}{cos,sort}{width=0.45\textwidth}{(a) Sand dunes image in the cosine Fourier domain. (b) Fourier coefficients sorted by absolute value and displayed on the logarithmic (decibel) scale.}

Figure~\ref{fig:cos} shows the image after applying the \emph{cosine
transform} (a version of the Fourier transform that keeps coefficients
real). Notice both compactness and sparsity in the Fourier domain. To
analyze the sparsity pattern, Figure~\ref{fig:sort} shows Fourier
coefficients after sorting them by absolute value. The rate 
of coefficient decay is a measure of sparsity.

\plot{inv}{width=\textwidth}{(a) Sand dunes image reconstructed after thresholding. (b) Compression noise.}

Figure~\ref{fig:inv} shows the result of shrinkage (soft thresholding)
of Fourier coefficients using 1\% threshold and the difference between
the reconstructed image and the true image.

Your task:
\begin{enumerate}
\item Change directory to \texttt{hw6/dunes}
\item Run 
\begin{verbatim}
scons view
\end{verbatim}
to reproduce the figures on your screen.
\item Modify the \texttt{SConstruct} file to adjust the threshold percentage to the level that makes noise negligible.
\item  Modify the \texttt{SConstruct} file to use DWT (the \emph{digital wavelet transform}) instead of the cosine transform. 
Compare the results. Which transform has more sparsity and provides better compression? 
\item \textbf{EXTRA CREDIT} for finding and implementing a transform with an even better compression.
\end{enumerate}

\newpage

\lstset{language=python,numbers=left,numberstyle=\tiny,showstringspaces=false}
\lstinputlisting[frame=single,title=dunes/SConstruct]{dunes/SConstruct}

\section{Match Filtering for Attenuation of Surface Seismic Waves}
\inputdir{match}

\sideplot{data}{width=\textwidth}{Seismic shot record from sand dunes in the Middle East. The data are contaminated by ground roll propagating in the sand.}

Figure~\ref{fig:data} shows a section out of a seismic shot record
collected over sand dunes in the Middle East. The data are
contaminated by ground roll propagating in the sand. A major data
analysis task is to separate the signal (reflection waves) from the
noise (surface waves).

\plot{spec0}{width=0.8\textwidth}{Data spectrum. Solid line -- original data. Dashed line -- initial noise model and signal model.}

A quick look at the data spectrum (Figure~\ref{fig:spec0}) shows that the
noise is mostly concentrated at low frequencies. We can use this fact
to create a noise model by low-pass filtering.

\plot{noise0}{width=\textwidth}{(a) Noise model created by low-pass filtering of the original data. (b) Result of subtraction of the noise model from the data.}

Figure~\ref{fig:noise0} shows the noise model from low-pass filtering
and inner muting and the result of subtracting this model from the
data. Our next task is to match the model to the true noise by solving
the least-squares optimization problem
\begin{equation}
\label{eq:ls}
\min \|\mathbf{N}\,\mathbf{f} - \mathbf{d}\|^2\;,
\end{equation}
where $\mathbf{d}$ is the data, $\mathbf{f}$ is a \emph{matching filter}, and
$\mathbf{N}$ represents convolution of the noise model
$\mathbf{n}_0$ with the filter. After minimization, $\mathbf{n} =
\mathbf{N}\,\mathbf{f}$ becomes the new noise model, and
$\mathbf{d}-\mathbf{n}$ becomes the estimated signal. Match filtering
is implemented in program \texttt{match.c}. Some parts of this program
are left out for you to fill.

\lstset{language=c,numbers=left,numberstyle=\tiny,showstringspaces=false}
\lstinputlisting[frame=single,title=match/match.c]{match/match.c}

Your task:
\begin{enumerate}
\item Change directory to \texttt{hw6/match}
\item Run 
\begin{verbatim}
scons view
\end{verbatim}
to reproduce the figures on your screen.
\item Modify the \texttt{match.c} file to fill in missing parts.
\item Test your modifications by running the dot product test.
\begin{verbatim}
scons dot.test
\end{verbatim}
Repeating this several times, make sure that the numbers in the test match.
\item  Modify the \texttt{SConstruct} file to display the results of match filtering
and include them in your assignment.
\item \textbf{EXTRA CREDIT} for improving the results by finding either better parameters or a better algorithm. 
\end{enumerate}

\newpage

\lstset{language=python,numbers=left,numberstyle=\tiny,showstringspaces=false}
\lstinputlisting[frame=single,title=match/SConstruct]{match/SConstruct}

\newpage

\section{Match Filtering Continued}

Both the sand dune image and the seismic record are sections from 3-D
data (time-lapse measurements in one case and 3-D seismic acquisition
in the other).
\begin{enumerate}
\item In either of the examples, try using match filtering to
predict one 3-D frame from another. You may need to modify the program
to make the filter two-dimensional. 
\item Include your results in the paper.
\end{enumerate}

%\newpage

\section{Completing the assignment}

\begin{enumerate}
\item Change directory to \texttt{hw6}.
\item Edit the file \texttt{paper.tex} in your favorite editor and change the
  first line to have your name instead of Moore's.
\item Run
\begin{verbatim}
sftour scons lock
\end{verbatim}
to update all figures.
\item Run
\begin{verbatim}
sftour scons -c
\end{verbatim}
to remove intermediate files.
\item Run
\begin{verbatim}
scons pdf
\end{verbatim}
to create the final document.
\item Submit your result (file \texttt{paper.pdf}) on paper or by
e-mail.
\end{enumerate}


