\author{Thomas Bayes}
%%%%%%%%%%%%%%%%%%%%
\title{Homework 6}

\begin{abstract}
  This homework has two computational parts:
  \begin{enumerate}
  \item Stochastic simulation of natural patterns.
  \item Missing data interpolation for ocean floor topography.
  \end{enumerate}
\end{abstract}

\section{Prerequisites}

Completing the computational part of this homework assignment requires
\begin{itemize}
\item \texttt{Madagascar} software environment available from \\
\url{http://www.ahay.org/}
\item \LaTeX\ environment with \texttt{SEG}\TeX\ available from \\ 
\url{http://www.ahay.org/wiki/SEGTeX}
\end{itemize}
To do the assignment on your personal computer, you need to install
the required environments. Please ask for help if you don't know where
to start.

The homework code is available from the \texttt{Madagascar} repository
by running
\begin{verbatim}
svn co http://svn.code.sf.net/p/rsf/code/trunk/book/geo391/hw6
\end{verbatim}

\section{Natural patterns}
\inputdir{pattern}

In this section we will extract multidimensional spatial patterns from
natural images using the method of \cite{textures}. Four examples,
shown in Figures~\ref{fig:horizon}-\ref{fig:your}, contain:
\begin{enumerate}
\item Seismic horizon slice.
\item A slice from a CT-scan of a rock sample.
\item A remote-sensing satellite image.
\item Your own data (to be replaced by you).
\end{enumerate}
In each of the cases, we follow the same workflow:
\begin{enumerate}
\item Remove a linear trend from the data.
\item Estimate a multi-dimensional prediction-error filter (PEF) on a helix.
\item Apply the inverse of the estimated PEF to random normally-distributed numbers
  to create a random spatial texture, which shares the covariance with the input. 
\end{enumerate}

\plot{horizon}{width=0.8\textwidth}{Pattern extraction from a seismic time horizon.}
\plot{square}{width=0.8\textwidth}{Pattern extraction from a CT-scan of a rock sample.}
\plot{sat}{width=0.8\textwidth}{Pattern extraction from a satellite image.}
\plot{your}{width=0.8\textwidth}{Pattern extraction from your own data.}

\lstset{language=python,numbers=left,numberstyle=\tiny,showstringspaces=false}
\lstinputlisting[frame=single]{pattern/SConstruct}

Your task:
\begin{enumerate}
\item Change directory to \verb#geo391/hw6/pattern#
\item Run 
\begin{verbatim}
scons view
\end{verbatim}
to reproduce the figures on your screen.
\item Modify the \texttt{SConstruct} file to replace Figure~\ref{fig:your} with the figure containing your own data.
\item Why does the method fail in extracting some of the patterns? Did it succeed in extracting patterns from your data?
\end{enumerate}

\section{Missing ocean-floor data interpolation}
\inputdir{seabeam}

SeaBeam is an apparatus for measuring water depth both directly under
a boat and somewhat off to the sides of the boat's track. In this part
of the assignment, we will use a benchmark dataset from \cite{gee}:
SeaBeam data from a single day of acquisition. The original data are
shown in Figure~\ref{fig:data}a.

\plot{data}{width=0.8\textwidth}{(a) Water depth measurements from one day
  of SeaBeam acquisition. (b) Mask for locations of known data.}

Program~\texttt{interpolate.c} implements two alternative methods:
regularized inversion using convolution with a multi-dimensional
filter and preconditioning, which uses the inverse operation
(recursive deconvolution or polynomial division on a helix) for model
reparameterization.

Our first attempt is to use the Laplacian-factor filter from the
previous homework. The results from the two methods are shown in
Figure~\ref{fig:lseabeam}. They are not very successful in hiding
the ``acquisition footprint''.

\plot{lseabeam}{width=0.8\textwidth}{Left: missing data interpolation using regularization by convolution with a Laplacian factor. 
Right: missing data interpolation using model reparameterization by
deconvolution (polynomial division) with a Laplacian factor.}

Next, we will try to estimate the model covariance from the available
data. The covariance can be estimated using two alternative methods:
the fast Fourier transform (FFT) method \cite[]{jim} and the
prediction-error filter (PEF) method \cite[]{textures}. Random
realizations of model patterns from the two methods and the
corresponding 2-D Fourier spectra are shown in Figures~\ref{fig:rand2}
and~\ref{fig:rand}.

\plot{rand2}{width=0.8\textwidth}{Left: data pattern generated using FFT method for covariance estimation.
Right: its Fourier spectrum.}

\plot{rand}{width=0.8\textwidth}{Left: data pattern generated using PEF method for inverse covariance estimation.
Right: its Fourier spectrum.}


Our new attempt to interpolate missing data using the helical
prediction-error filter is shown in Figure~\ref{fig:seabeam}.

\plot{seabeam}{width=0.8\textwidth}{Left: missing data interpolation using regularization by 
convolution with a prediction-error filter. 
Right: missing data interpolation using model reparameterization by
deconvolution (polynomial division) with a prediction-error filter.}

\lstset{language=c,numbers=left,numberstyle=\tiny,showstringspaces=false}
\lstinputlisting[frame=single,title=seabeam/interpolate.c]{seabeam/interpolate.c}

%\newpage

\lstset{language=python,numbers=left,numberstyle=\tiny,showstringspaces=false}
\lstinputlisting[frame=single,title=seabeam/SConstruct]{seabeam/SConstruct}

Your task:
\begin{enumerate}
\item Change directory to \texttt{hw5/seabeam}
\item Run 
\begin{verbatim}
scons view
\end{verbatim}
to reproduce the figures on your screen.
\item Modify the \texttt{SConstruct} file to implement the following tasks:
\begin{enumerate}
\item Find the number of iterations required for both methods shown in Figure~\ref{fig:seabeam} to achieve similar results.
\item To provide a more quantitative comparison, modify the
\texttt{interpolate.c} program to output a measure of convergence
(such as the least-squares misfit) as a function of the number
of iterations\footnote{You can study the interfaces to
the \texttt{sf\_solver} and
\texttt{sf\_solver\_prec} programs. See
\url{https://sourceforge.net/p/rsf/code/HEAD/tree/trunk/api/c/bigsolver.c}}. Generate figures comparing convergence with and without preconditioning.
\end{enumerate}
\item A method for generating multiple realizations of missing data interpolation is:
\begin{enumerate}
\item Start with a random realization $\mathbf{m}_0$ such as the one shown in Figures~\ref{fig:rand2} or \ref{fig:rand}.
\item Instead of estimating $\mathbf{m}$ such that $\mathbf{K}\,\mathbf{m} = \mathbf{d}$, 
      estimate $\mathbf{x}$ such that 
\[
\mathbf{K}\,\mathbf{x} = \mathbf{d}- \mathbf{K}\,\mathbf{m_0}\;.
\]
\item The estimate for $\mathbf{m}$ is then  $\widehat{\mathbf{m}} = \widehat{\mathbf{x}}+\mathbf{m_0}$.
\end{enumerate}
Implement several realizations of missing data interpolation using
several realizations of $\mathbf{m}_0$. You can do it by modifying
either \texttt{SConstruct} or \texttt{interpolate.c}.
\item Include your results in the paper.
\item \textbf{EXTRA CREDIT} for implementing missing data interpolation using the FFT method for model covariance.
\end{enumerate}

\section{Completing the assignment}

\begin{enumerate}
\item Change directory to \texttt{hw6}.
\item Edit the file \texttt{paper.tex} in your favorite editor and change the
  first line to have your name instead of Bayes's.
\item Run
\begin{verbatim}
sftour scons lock
\end{verbatim}
to update all figures.
\item Run
\begin{verbatim}
sftour scons -c
\end{verbatim}
to remove intermediate files.
\item Run
\begin{verbatim}
scons pdf
\end{verbatim}
to create the final document.
\item Submit your result (file \texttt{paper.pdf}) on paper or by
e-mail.
\end{enumerate}

\bibliographystyle{seg}
\bibliography{hw6}
