\author{Isaac Newton}
%%%%%%%%%%%%%%%%%%%%%%
\title{Homework 4}

\begin{abstract}
In this lab, we will use shot-record migration to create 
images in complex velocity media.
\end{abstract}

\definecolor{frame}{rgb}{0.905,0.905,1.0}
\lstset{language=Python,backgroundcolor=\color{frame},showstringspaces=false}

\section{Introduction}

Start by running
\begin{verbatim}
> cd ~/geo391
> svn update
\end{verbatim}

% ------------------------------------------------------------
\section{Sigsbee 2A migration}
% ------------------------------------------------------------

\begin{enumerate}
%%
\item
In this exercise, you will modify parameters controlling 
shot-record wavefield extrapolation migration.
Everything is set-up for migration. Your task is to
identify relevant parameters, change them to generate 
new figures, include them in this document and discuss
the importance/effect of your various changes.

\inputdir{sigsbee}
\plot{slo}{width=6.0in}{}


%% 
\item Change directory 

\begin{verbatim}
> cd ~/geo391/hw3/sigsbee
\end{verbatim}
  \item Run
\begin{verbatim}
> scons view
\end{verbatim}
to generate figures and display them on your screen.  

%%
\item
Open the \texttt{SConstruct} file and 
find the dictionary named \texttt{par}.
Parameters $ns$, $js$, $fs$ control sampling on the shot axis,
and parameters $nw$, $jw$, $fw$ control sampling on the 
frequency axis (``n''=number, ``j''=jump, ``f''=first).

%%
\item
Locate the loop over migration configuration.
A local dictionary  named \texttt{loc} allows you to 
make local changes to some parameters without affecting
the others.
Two examples are included in this document
(Figures~\ref{fig:img0} and \ref{fig:img1}).
You will include and discuss more figures.

\plot{img0}{width=6.0in}
{Monochromatic migration for one shot 
($ns=1$, $nw=1$, $ow=2$).}

\plot{img1}{width=6.0in}{ 
Monochromatic migration for three shots 
($ns=3$, $fs=50$, $js=150$, $nw=1$).}

%%
\item
Keep the number of frequencies fixed 
(for example $nw=1$, $ow=2$),
and modify the number of shots used in migration
(change $ns$, $js$, $os$).
How many shots do you need in order to see the main 
features of the structure?
Include a new figure and discuss your observations.

%% 
 % \plot ...
 %%


%%
\item
Keep the number of shots fixed ($ns=1$) 
and change the migration frequency (change $ow$).
The migrated images will change accordingly and
the subsurface illumination will change.
Modify the location of the shot (change $fs$)
and observe illumination patterns, especially around
the salt body.
How does illumination change with frequency?
Include figures for images at different frequencies
and discuss your observations.

%% 
 % \plot ...
 %%

%%
\item
Keep the number of shots fixed ($ns=1$) 
and change the number of frequencies 
(change $nw$, $jw$, $ow$).
How does the image change with increased frequency band?
Include figures for shots at different locations in the 
image and discuss your observations.

%% 
 % \plot ...
 %%

%%
\item
Increase the number of shots 
(change $ns$, $js$, $os$)
and frequencies 
(change $nw$, $jw$, $ow$).
until you are satisfied with the quality of the migrated image.
How did you decide when to stop?
What parameters did you use?
Include one or more new figures and discuss your 
observations and choice of parameters.

%% 
 % \plot ...
 %%

%%
\item
This is a free-form questions.
Experiment with migration parameters on your own.
Include and discuss another migration configuration
of your choice. Discuss your figure and observations.

%% 
 % \plot ...
 %%

%%    
\item After you are done, run

\begin{verbatim}
> scons lock
> scons -c
\end{verbatim} 
  
\end{enumerate}


\section{}


\section{Wrap-up}

\begin{enumerate}  
\item 
Edit the file
\verb#~/geo391/hw4/paper.tex# in your favorite editor and change the
first line to have your name. Run
\begin{verbatim}
> scons pdf
\end{verbatim}
and submit your result (file \texttt{paper.pdf}) on paper or by
e-mail.

\end{enumerate}
