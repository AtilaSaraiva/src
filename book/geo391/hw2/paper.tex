\author{Leonard Euler} 
%%%%%%%%%%%%%%%%%%%%%%
\title{Homework 2}

\begin{abstract}
This homework has two parts. In the theoretical part, you will derive
analatycal solutions for the one-point and two-point ray tracing
problems in a medium with a constant gradient of velocity. In the
computational part, you will experiment with a real dataset from...
\end{abstract}

\section{Introduction}

Start by running
\begin{verbatim}
> cd ~/geo391
> svn update
> cd hw2
> scons read
\end{verbatim}
to display the homework on your screen. Edit the file
\texttt{~/geo391/hw2/paper.tex} in your favorite editor and change the
first line to have your name instead of Euler's. 

At any time, you can
produce a PDF file by running
\begin{verbatim}
> scons pdf
\end{verbatim}
or display it on your screen by running
\begin{verbatim}
> scons read
\end{verbatim}

\section{Theoretical part}

You can either write your answers on paper or edit them in
\texttt{~/geo391/hw2/paper.tex}. Please show all the derivations that
you do.

\begin{enumerate}
\item In class, we used a mysterious parameter $\sigma$ to represent a
  variable continuously increasing along a ray. There are other
  variables that can play a similar role.
\begin{enumerate}
\item Use the chain rule $\frac{d}{d \sigma} = \frac{d}{d T}\,\frac{d
T}{d \sigma}$ to transform the isotropic ray tracing system
\begin{eqnarray}
\label{eq:xsigma}
\frac{d \mathbf{x}}{d \sigma} & = & \mathbf{p} \\
\label{eq:psigma}
\frac{d \mathbf{p}}{d \sigma} & = & S(\mathbf{x})\,\nabla S \\
\label{eq:tsigma}
\frac{d T}{d \sigma} & = & S^2(\mathbf{x})
\end{eqnarray}
into the equivalent system
\begin{eqnarray}
\label{eq:xt}
\frac{d \mathbf{x}}{d T} = & & \hfill \\
\label{eq:pt}
\frac{d \mathbf{p}}{d T} = & & \hfill
\end{eqnarray}
that uses time $T$ as the ray parameter. Check physical dimensions.
\item Use the chain rule to find a parameter $\xi$ such that
\begin{equation}
\label{eq:pxi}
\frac{d \mathbf{p}}{d \xi} = - \nabla V = \frac{\nabla S}{S^2(\mathbf{x})}
\end{equation}
What are the physical dimensions of $\xi$?
\item Show that
\begin{equation}
  \label{eq:txi}
  \frac{d T}{d \xi} = V(\mathbf{x})\;,
\end{equation}
where $V=1/S$ is the velocity.
\item Find
\begin{eqnarray}
  \label{eq:xxi}
  \frac{d \mathbf{x}}{d \xi} = & & \hfill 
\end{eqnarray}
\end{enumerate}
\item We will use a new $\xi$ parameterization to solve the ray
tracing problem analytically for the special case when the velocity
distribution has a constant gradient
\begin{equation}
  \label{eq:v}
  V(\mathbf{x}) = V(\mathbf{x}_0) + \mathbf{g}_0 \cdot (\mathbf{x} - \mathbf{x}_0)\;.
\end{equation}
\begin{enumerate}
\item Solving the one-point ray tracing problem for $\mathbf{x}(\xi)$
and $\mathbf{p}(\xi)$ with the initial conditions  $\mathbf{x}(0)=\mathbf{x}_0$ and
 $\mathbf{p}(0)=\mathbf{p}_0$, show that
\begin{equation}
\label{eq:psol}
\mathbf{p}(\xi) = \mathbf{p}_0 - \mathbf{g}_0\,\xi\;.
\end{equation}
\item Express
\begin{eqnarray}
\label{eq:vsol}
V(\mathbf{x}) = \frac{1}{S(\mathbf{x})} = \frac{1}{\sqrt{\mathbf{p} \cdot \mathbf{p}}} = & &
\end{eqnarray}
in terms of $\mathbf{g}_0$, $\mathbf{p}_0$ and $\xi$.
\item Define $a = \mathbf{p} \cdot (\mathbf{x} - \mathbf{x}_0)$. Find
\begin{eqnarray}
\label{eq:axi}
\frac{d a}{d \xi} = & & 
\end{eqnarray}
and show that 
\begin{equation}
\label{eq:asol}
a(\xi) = V(\mathbf{x}_0)\,\xi
\end{equation}
and
\begin{equation}
\label{eq:a0sol}
\mathbf{p}_0 \cdot (\mathbf{x} - \mathbf{x}_0) = V(\mathbf{x})\,\xi
\end{equation}
\item One way to seek the solution for the one-point ray tracing
  problem is to look for scalars $\alpha$ and $\beta$ in the representation
\begin{equation}
  \label{eq:xsol}
  \mathbf{x}(\xi) = \mathbf{x}_0 + \alpha(\xi)\,\mathbf{p}_0 + \beta(\xi)\,\mathbf{g}_0\;.
\end{equation}
Under what condition does the linear system of equations
\begin{eqnarray}
  \label{eq:linsys1}
  V(\mathbf{x})\,\xi = \mathbf{p}_0 \cdot (\mathbf{x} - \mathbf{x}_0) & = & 
  \alpha\,\mathbf{p}_0 \cdot \mathbf{p}_0 + \beta\,\mathbf{p}_0 \cdot \mathbf{g}_0 \\
  \label{eq:linsys2}
  V(\mathbf{x}) - V(\mathbf{x}_0) = \mathbf{g}_0 \cdot (\mathbf{x} - \mathbf{x}_0) & = &
  \alpha\,\mathbf{p}_0 \cdot \mathbf{g}_0 + \beta\,\mathbf{g}_0 \cdot \mathbf{g}_0
\end{eqnarray}
have a unique solution for $\alpha$ and $\beta$?
\end{enumerate}
\item (EXTRA CREDIT) For an extra credit, switch to solving the
  two-point ray tracing problem. 
\begin{enumerate}
\item Find $\alpha$ and $\beta$ from
  equations~(\ref{eq:linsys1}-\ref{eq:linsys2}).
\item Express the squared distance between the ray end points
  \begin{equation}
    \label{eq:d2} 
    |\mathbf{x} - \mathbf{x}_0|^2 = (\mathbf{x} - \mathbf{x}_0) \cdot (\mathbf{x} - \mathbf{x}_0) =
    \alpha^2\,\mathbf{p}_0 \cdot \mathbf{p}_0 + 2 \alpha\,\beta\,\mathbf{p}_0 \cdot \mathbf{g}_0 +
    \beta^2\,\mathbf{g}_0 \cdot \mathbf{g}_0 =
  \end{equation}
  in terms of $\mathbf{g}_0$,$\mathbf{p}_0$, and $\xi$. 
\item In the two-point problem, $|\mathbf{x} - \mathbf{x}_0|$ is known, and the two unknown
  parameters are $\xi$ and $(\mathbf{p}_0 \cdot \mathbf{g}_0)$.
  Express $(\mathbf{p}_0 \cdot \mathbf{g}_0)$ from your
  equation~(\ref{eq:vsol}) and substitute it into your
  equation~(\ref{eq:d2}). Solve for $\xi$. How many physically
  meaningful solutions can you obtain?
\item Finally, use $(\mathbf{p}_0 \cdot \mathbf{g}_0)$ and $\xi$
  expressed in terms of $|\mathbf{x} - \mathbf{x}_0|$, $|\mathbf{g}_0|$, $V(\mathbf{x}_0)$, and
  $V(\mathbf{x})$ and substitute them into the one-point traveltime
  solution obtained by integrating
  equation~(\ref{eq:txi})\footnote{$\mbox{arccosh}(x)$ is the inverse
    hyperbolic cosine function defined as $\mbox{arccosh}(x) = \ln\left(x + \sqrt{x^2-1}\right)$.}
  \begin{equation}
    \label{eq:t1}
    T(\xi) = \frac{1}{|\mathbf{g}_0|}\,\mbox{arccosh}\left(1 + \frac{|\mathbf{g}_0|^2\,V(\mathbf{x})\,V^2(\mathbf{x}_0)\,\xi^2}
      {V(\mathbf{x})+V(\mathbf{x}_0) - \mathbf{p}_0 \cdot \mathbf{g}_0\,V(\mathbf{x})\,V^2(\mathbf{x}_0)\,\xi}\right)\;.
  \end{equation}
  Your result will be the analytical two-point traveltime
  \begin{eqnarray}
    \label{eq:t2}
    \widehat{T}(\mathbf{x}_0,\mathbf{x}) = & & \hfill \ 
  \end{eqnarray}
\end{enumerate}
\end{enumerate}

\newpage

\section{Computational part}



