\author{Joseph Fourier}
%%%%%%%%%%%%%%%%%%%%%
\title{Homework 2}

\begin{abstract}
  This homework has four parts. 
  \begin{enumerate}
  \item Theoretical questions related to digital filtering.
  \item Analyzing data by applying a running median filter.
  \item Analyzing data by applying Fourier compression.
  \item Applying one of these algorithms to your own data.
  \end{enumerate}
\end{abstract}

\section{Digital filtering}

You can either write your answers on paper or edit them in the file
\verb#hw2/paper.tex#. Please show all the mathematical
derivations that you perform.

\begin{enumerate}

\item You are given a table of numbers $x_1, x_2, \ldots, x_N$ and $y_1, y_2, \ldots, y_N$ and 
want to fit a parabolic model $y(x) = a + b\,x + c\,x^2$ to them.

\begin{enumerate}
\item Using the method of least squares, find a set of liner equations for coefficients $a$, $b$, and $c$.
\item Using the method of least squares, find $a$, $b$, and $c$ for the case of $N=2$.
\end{enumerate}

\item The matrix in equation~(\ref{eq:conv}) represents a convolution operator with zero boundary conditions.
\begin{equation}
\label{eq:conv}
\mathbf{F} = \left[\begin{array}{llllll}
f_1 & f_0 & 0   & 0   & 0   & 0   \\
f_2 & f_1 & f_0 & 0   & 0   & 0   \\
f_3 & f_2 & f_1 & f_0 & 0   & 0   \\
0   & f_3 & f_2 & f_1 & f_0 & 0   \\
0   & 0   & f_3 & f_2 & f_1 & f_0 \\
0   & 0   & 0   & f_3 & f_2 & f_1 \\
\end{array}\right]\;.
\end{equation}

The operator is implemented in the C code below.

\lstset{language=c,numbers=left,numberstyle=\tiny,showstringspaces=false}
\lstinputlisting[frame=single,firstline=15]{conv.c}

\begin{enumerate}
\item Modify the matrix and the program to implement periodic boundary conditions.
\item Add the code for the adjoint (matrix transpose) operator.
\end{enumerate}

\item Consider the parabolic filter $F(Z)$ defines as
\begin{equation}
\label{eq:parab}
F(Z) = 1 + 4 Z + 9 Z^2 + \ldots + N^2 Z^{N-1}\;.
\end{equation} 

\begin{enumerate}
\item Show that this filter can be implemented with recursive filtering (polynomial division).
\item What is the advantage of recursive filtering? Does it depend on $N$?
\end{enumerate}

\item Using helical boundary conditions, a two-dimensional filter $F_2(Z_1,Z_2)$ can be mapped 
to a one-dimensional filter $H(Z)$ according to equation
\begin{equation}
\label{eq:helix}
H(Z) = F_2\left(Z,Z^{N_1}\right)\;,
\end{equation} 
where $N_1$ is the length of the first (fast) axis. How would you map
a three-dimensional filter $F_3(Z_1,Z_2,Z_3)$?

\end{enumerate}

\newpage

\section{Running median and running mean filters}
\inputdir{running}

\plot{bay}{width=\textwidth}{Digital elevation map of the San Francisco
  Bay Area.}

We return the digital elevation map of the San Francisco Bay Area, shown in Figure~\ref{fig:bay}.

In this exercise, we will separate the data into ``signal'' and
``noise'' by applying running mean and median filters.  The result of
applying a running median filter is shown in
figure~\ref{fig:med,res}. Running median effectively smooths the data
by removing local outliers.

\multiplot{2}{med,res}{width=0.45\textwidth}{Data separated into signal (a) and noise (b) by applying a running median filter.}

\plot{time}{width=\textwidth}{CPU time for computing the running median with different algorithms as a function of the window size. 
The dashed line corresponds to slow sorting in program \texttt{running.c}, the solid line corresponds to the fast \emph{quantile} algorithm.}

The algorithm is implemented in program \texttt{running.c}.

\lstinputlisting[frame=single]{running/running.c}

\begin{enumerate}
\item Change directory to \verb#geo391/hw2/running#
\item Run 
\begin{verbatim}
scons view
\end{verbatim}
to reproduce the figures on your screen.
\item Modify the \texttt{running.c} program and the \texttt{SConstruct} file to compute running mean instead of running median. Compare the results.
\item \textbf{EXTRA CREDIT} for improving the efficiency of the running median algorithm. Figure~\ref{fig:time} compares the efficiency of running median computations using the slow sorting from function \texttt{median} in program \texttt{running.c} and the fast quantile algorithm (library function \verb#sf_quantile# ). Your goal is to make the algorithm even faster. Consider parallelization, reusing previous window results, other fast sorting strategies, etc.
\end{enumerate}

\lstset{language=python,numbers=left,numberstyle=\tiny,showstringspaces=false}
\lstinputlisting[frame=single]{running/SConstruct}

\section{Fourier compression}
\inputdir{fourier}

\plot{fft}{width=\textwidth}{Absolute value of the Fourier transform of the digital elevation data. The frame inside shows a window selected for compression.}

The goal of your next assignment is to find a compressed
representation of the data in the Fourier transform
domain. Figure~\ref{fig:fft} shows the Fourier transform of the
digital elevation data from Figure~\ref{fig:bay}. We can see that most
of the energy gets concentrated near the center (zero frequency). 

There are two alternative ways to compress data in the Fourier domain:
\begin{itemize}
\item One approach is to
select a range of frequencies that contain the most important
information. An advantage of this approach is the ability to subsample
the original data by transforming back from a windowed range of frequencies.
The results from this method are shown in Figure~\ref{fig:sig,cut}.
\item Another approach is to zero all Fourier coefficients below a certain threshold value, regardless of which frequencies they represent.  
 The results from this method are shown in Figure~\ref{fig:thr,noi}. Figure~\ref{fig:hist} shows a selected threshold plotted against the histogram of Fourier coefficients.
\end{itemize}

\multiplot{2}{sig,cut}{width=0.45\textwidth}{Data separated into signal (a) and noise (b) by applying Fourier compression with windowing.}
\multiplot{2}{thr,noi}{width=0.45\textwidth}{Data separated into signal (a) and noise (b) by applying Fourier compression with thresholding.}

\sideplot{hist}{width=0.8\textwidth}{Normalized histogram of Fourier coefficients (by absolute value). The vertical line shows a selected threshold.}

\begin{enumerate}
\item Change directory to \verb#geo391/hw2/fourier#
\item Run 
\begin{verbatim}
scons view
\end{verbatim}
to reproduce the figures on your screen.
\item Modify the \texttt{SConstruct} file to decrease the size of the window so that the noise level increases in Figure~\ref{fig:cut}. How do you measure the noise level? Find a level that you find negligibly small.
\item Modify the \texttt{SConstruct} file to increase the threshold value so that the compression achieves the same quality as in the previous case. The noise level in Figure~\ref{fig:noi} should match that in Figure~\ref{fig:cut}.
\item Compare the number of nonzero Fourier coefficients in both cases. Which method achieves a better compression?
\item \textbf{EXTRA CREDIT} for finding a way for a better compression of the data in the Fourier domain. Your data reconstruction should have 
the same noise level, yet the number of non-zero coefficients in the Fourier domain should be smaller.
\end{enumerate}

\lstinputlisting[frame=single]{fourier/SConstruct}

\newpage

\section{Your own data}

Your final task is to apply one of the data analysis techniques of the
previous sections to your own data:
\begin{enumerate}
\item Select a dataset suitable for
running mean/median filters or for Fourier compression.
\item Apply one the algorithm of the previous sections and choose
  appropriate parameters.
\item Include the results in your homework.
\end{enumerate}

\section{Completing the assignment}

\begin{enumerate}
\item Change directory to \verb#geo391/hw2#.
\item Edit the file \texttt{paper.tex} in your favorite editor and change the
first line to have your name instead of Fourier's.
\item Run
\begin{verbatim}
sftour scons lock
\end{verbatim}
and
\begin{verbatim}
scons pdf
\end{verbatim}
\item Submit your result (file \texttt{paper.pdf}) by printing it out
  or by e-mail.
\end{enumerate}
