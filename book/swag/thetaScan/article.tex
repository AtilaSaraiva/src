../macro.tex

\published{Geophysics, 76, no. 3, WA31-WA42, (2011)}

\lefthead{Alkhalifah}
\righthead{TI traveltimes in complex media}

\def\beq{\begin{equation}}
\def\eeq{\end{equation}}
\def\beqa{\begin{eqnarray}}
\def\eeqa{\end{eqnarray}}

\title{Traveltime approximations for transversely isotropic media with an inhomogeneous background}

\email{tariq.alkhalifah@kaust.edu.sa}

\author{Tariq Alkhalifah}

\maketitle

\begin{abstract}

A transversely isotropic model with a \geosout{tilt in the axis of} \new{tilted} symmetry \geouline{axis}
(TI) is regarded as one of the most effective 
approximations to the Earth subsurface, especially for imaging
purposes. However, we commonly utilize
 this model  by setting the axis of symmetry normal to \geosout{the direction
 of} the reflector\geosout{ dip}. 
This assumption may be accurate in many places,  but deviations \geouline{from this assumption}
will cause errors in the wavefield description. Using perturbation theory and Taylor's series, 
I expand the solutions of the eikonal equation for \geosout{the} 2D transversely isotropic
\geosout{case} \geouline{media} with respect to the independent parameter $\theta$, the angle the
\geosout{TI} tilt \geouline{of the axis of symmetry}
makes with the vertical, in a generally {\emph{inhomogeneous}} TI \new{background} with
a vertical axis of symmetry (VTI)\old{background}. I do an additional expansion 
 in terms of the independent \geouline{(anellipticity)}
 parameter $\eta$ in a generally inhomogeneous elliptically
anisotropic background medium. These new TI traveltime solutions are
given by \geosout{polynomials as a function of powers of} \geouline{expansions in} $\eta$ and $\theta$ 
with coefficients  extracted from solving linear first-order partial differential 
equations.  Pade approximations are used to enhance the accuracy of
the representation by \geosout{transient behavior} \geouline{predicting
the behavior of the higher-order terms} of the expansion.
A \geosout{homogeneous medium} simplification of the expansion \geouline{for homogenous media} provides  
nonhyperbolic moveout descriptions of the traveltime  \geosout{as a function of the tilt in the symmetry axis} \geouline{for TI models}
that 
are more accurate than other recently derived approximations.
 In addition, for 3D media, I develop traveltime approximations \geosout{with
 respect to} \geouline{using Taylor's series type of} expansions in the azimuth of the axis of symmetry. The
 coefficients of all
 these expansions can also \geosout{serve to} provide us with the medium sensitivity
 gradients (Jacobian) for nonlinear tomographic-based inversion for \geosout{such parameters} the tilt in the symmetry axis\old{ direction}.

%However, using simple approximations we can relate the measured vertical and NMO velocity to those valid for the symmetry direction.
\end{abstract}

\newpage

\section{Introduction}

The nature of sedimentation and thin layering in the Earth subsurface induces wave propagation
characteristics that can be better described by considering the medium to be anisotropic. Specifically, since the layering has a general preferred direction, 
\geosout{where features within the layering plane are isotropic,} we find that the transversely isotropic (TI) assumption to be the most practical \geouline{type of}
anisotropy to represent big parts of the subsurface. The tilt in this case is naturally set in the direction normal to the layering
\cite[]{SEG-2000-09650968,audebert:P185,ttt,alkhalifah:A19}.
Thus, \geouline{this type of model} \geosout{a general TI medium type of a model} approximates 
a big portion of the anisotropy resulting from the thin layering. Developing simple traveltime formulations
for such a model helps in many applications, including traveltime tomography and integral-based \geouline{Kirchhoff} imaging. 
The vertical symmetry axis (VTI) \new{medium} is a special case of
TI in which the \geosout{tilt} \geouline{symmetry axis} is normal to the typically horizontal acquisition surface, and thus results in simpler formulations.

Traveltimes are conventionally evaluated by solving a nonlinear partial differential equation (PDE), better known as
the eikonal equation. Among the most known methods for solving this equation are
 ray tracing and the finite-difference \old{approximation} \new{approximations}. Finite-difference solutions 
of the eikonal equation have been
recognized as one of the most efficient means of traveltime
calculations
\cite[]{GEO55-05-05210526,GEO56-06-08120821,Popovici.sep.70.245,GPR49-02-01650178}.  Some of
main advantages of this method in comparison to ray tracing
 include the ability to directly provide solutions on regular grids, a
complete coverage of the solution space, and a high numerical
robustness. \geouline{On the down side, finite-difference based
  solutions typically include only the first arrivals, which might not 
be even the most energetic ones} \cite[]{bookC}.
In anisotropic media, traveltime computation is dependent on more than one \geosout{velocity} \geouline{parameter} field. 
However, through careful parametrization of the
TI medium,
$P$-wave traveltimes in 3D\new{, under the acoustic assumption,} become dependent on only three \geosout{velocity} \geouline{parameter} fields and two angles.
\geouline{These parameters include the tilt-direction velocity, $v_t$, the normal-moveout \old{like} \new{equivalent} velocity, $v=v_{t} \sqrt{1+2 \delta}$}
 [where $\delta$ corresponds to the symmetry direction \cite[]{tsvankin:479}], and the anellipticity parameter 
 $\eta=\frac{\epsilon-\delta}{1+2 \delta}$ (\new{with $\epsilon$
 also defined with respect to} \old{again in} the symmetry direction).
 \geosout{These parameters include the normal-moveout velocity, $v$ (in the
 tilt direction), an anisotropy 
parameters referred to as $\eta$ (or the horizontal velocity, $v_h$), and the tilt-direction velocity $v_t$. }
This is evident in the eikonal equation
for TI media
developed by \cite{GEO63-02-06230631,GEO65-04-12391250}. If the \geosout{TI} \geouline{symmetry axis}  is
not vertical, two additional parameters are needed to describe the tilt in
3D, the angle $\theta$ that the \geosout{tilt} \new{symmetry} \geouline{axis} makes
with the vertical \geosout{axis along a vertical plane} and
\geouline{the azimuth}
$\phi$\geosout{, the azimuth} \old{that} \new{of} the vertical \geouline{symmetry-axis} plane
 \new{with respect} the $x$-axis [\cite{tsvankin:479}].

The process of finding a stable solution for
the TI (or even the VTI) eikonal equation using finite-difference schemes is
generally hard, especially since such a process requires finding the root of a quartic
equation at each computational step \cite[]{wang:T129}. However, traveltime computation for a 
slightly more simplified, but not practical, \geouline{elliptically} anisotropic model\geosout{, referred
to as elliptical anisotropy,} is far more efficient. The reason for the high efficiency
is that elliptical anisotropy has the
same order \geouline{of} complexity (nonlinearity) in the eikonal equation as does the isotropic equation. Thus, though elliptical
anisotropy represents an uncommon model in practice, 
it provides some flexibility in treating the difference between
vertical and horizontal velocities, or in other words, the flexibility of stretching the depth axis
to obtain accurate reflection depths in imaging
\cite[]{TLE20-05-05240527,GEO60-05-14951513,ohlsen:1600}. 
However, elliptical anisotropy does not provide accurate focusing
for media of \geouline{typical non-elliptical} TI anisotropy \cite[]{GEO59-09-14051418}.
 It will be used here however as \geosout{a launching pad (a}
 \geouline{the} background medium \geosout{)} for the perturbation
 expansions.

The forward problem, whether it is traveltime calculation or wavefield modeling,  is a major component of the inversion process. If we model wavefields,
we can use those wavefields to generate synthetic data and compare them with measured data as part of what we refer to as wavefield inversion.
Likewise, forward traveltime calculation is used to measure the traveltime misfit with those extracted from the data in what is referred to as
traveltime tomography. If we assume that one or two of the parameters are constant, the gradient of the objective function with respect to these parameters
can be calculated analytically and that usually
helps the inversion process. In this paper, \new{I} \old{we} develop simplified formulations for traveltime calculation that can help in resolving
\old{anisotropic} \new{anisotropy} parameters, specifically the tilt angle.

 \cite{etascan} developed an eikonal-based scanning scheme to search
 for the anisotropy parameter $\eta$ that can provide the best
 traveltime fit \geouline{to the data} in a general inhomogeneous background medium. In this earlier paper, I derived first-order
 linear \geouline{partial differential equations (PDEs)} governing the coefficients of expanding the traveltime
 solution for VTI media in terms of the independent parameter $\eta$
 from a background elliptical anisotropic model. I use \geouline{the} Shanks transform [\cite{Bender}] to enhance the accuracy of the expansion to a point in
 which the homogeneous-medium versions of it provided exceptional accuracy in describing the traveltime compared to other well-known published
 moveout equations \cite[]{GEO65-04-13161325}. 
I also suggested a simple angle transformation to make the method work for a tilted symmetry axis with a known tilt
 direction (i.e. in the direction normal to the \geouline{layering}\geosout{dip}). However, if the tilt \geouline{direction} is unknown, the $\eta$ estimation will certainly suffer from this limitation.

In this paper, I derive multi-parameter expansions of \geosout{the traveltimes} \geouline{traveltime}
as a function of the symmetry-axis \geosout{direction} angles and $\eta$ with coefficients estimated using
linearized forms of the eikonal equation. The accuracy of such an expansion is again further enhanced using Shanks transform and Pade
approximations to obtain \geosout{better} higher-order representation. I use a homogeneous-background medium version of the approximation to test its
accuracy and then \geosout{look at} \geouline{examine} the 3D case where the \geosout{tilt} \geouline{axis direction} is described by
azimuth as well.

\section{The TI eikonal and expansion in $\theta$}

In VTI media, the eikonal equation \cite[]{GEO63-02-06230631}
 \geosout{for an} \geouline{in the} acoustic \geosout{medium} approximation has the form:
\beqa
{v^2} (1+2 \eta) \,{\left(\left(\frac{\partial \tau}{\partial x}\right)^2+\left(
\frac{\partial \tau}{\partial y}\right)^2 \right)} + 
    {{{v_t}}^2}\,{\left(\frac{\partial \tau}{\partial z}\right)^2}\,
     \left( 1 - 2 \eta {v^2} \,{\left( \left(\frac{\partial \tau}{\partial x}\right)^2 +
\left(\frac{\partial \tau}{\partial y}\right)^2 \right)} \right)=1,
\label{eq:eikonal}
\eeqa
where $\tau (x,y,z)$ is the traveltime (eikonal) measured from the source to
a point with the coordinates $(x, y, z)$, and $v_{t}$ and $v$  are the
velocity and NMO\old{-like} \new{velocity} (=$v_{t} \sqrt{1+2 \delta}$), respectively, \old{in the symmetry direction} 
described with respect to the symmetry direction at
that point.  To formulate
a well-posed initial-value problem \geosout{on} \geouline{for} equation~\ref{eq:eikonal}, it is
sufficient to specify $\tau$ at some closed surface and to choose one
of the two solutions: the wave going from or \geosout{to} \geouline{toward} the
source.
The level of nonlinearity in this quartic  (in terms of $\tau$) equation is higher than that for the
isotropic or elliptically anisotropic eikonal equations. This results in much more complicated 
finite-difference \geosout{solution} approximations of the VTI eikonal equation.

 For a tilted \geosout{in the} TI \geouline{medium}, the traveltime derivatives in equation~\ref{eq:eikonal} are taken with respect to the tilt direction, and
 thus, \geosout{the 
 equation on a regular Cartesian grid (tilt independent)} we have to
 rotate 
the derivatives in equation~\ref{eq:eikonal} using the following Jacobian in 3D:
\beq
\left(
\begin{array}{ccc}
 \cos\phi  \cos\theta &  \sin\phi \cos\theta & \sin\theta \\
 - \sin\phi  &  \cos\phi & 0  \\
-\cos\phi \sin\theta &  -\sin\phi \sin\theta & \cos\theta
\end{array}
\right),
\label{eq:J}
\eeq
\geouline{to obtain an eikonal equation corresponding to the conventional computational coordinates governed by the acquisition surface.} 
In equation~\ref{eq:J}, $\theta$ is the angle of the symmetry axis measured from the vertical and $\phi$ corresponds to the azimuth of the vertical plane
that contains the symmetry axis measured from the $x$-axis (the axis of \geouline{the} source-receiver \geouline{direction}).  Setting \geouline{initially}
$\phi$=0, for \geosout{initial} simplicity, allows us to obtain the eikonal \geouline{equation} for 2D TI media given by:
 \beqa
& & {v^2} (1+2 \eta) \,{\left(\cos\theta \frac{\partial \tau}{\partial x} + \sin\theta \frac{\partial \tau}{\partial z}\right)^2 } + \nonumber \\
& &{{{v_t}}^2}\,{\left( \cos\theta \frac{\partial \tau}{\partial z}-\sin\theta \frac{\partial \tau}{\partial x} \right)^2}\,
     \left( 1 - 2 \eta {v^2} \,{ \left( \cos\theta \frac{\partial \tau}{\partial x} +\sin\theta \frac{\partial \tau}{\partial z} \right)^2} \right)=1.
\label{eq:eikonalti}
\eeqa 
The full 3D version of this
equation is \new{stated} in Appendix D.

\inputdir{XFig}

\sideplot{scanTheta}{width=4in}{A schematic plot showing the relation between a background traveltime field for $\theta$=0 
and that when $\theta$ is larger than zero. The round dot at the top of the $\theta=0$ plane represents a source.}

\geouline{Solving} equation~\ref{eq:eikonalti} \geosout{can be solved} numerically requires
 solving a quartic equation (instead of the quadratic in the isotropic and elliptical
anisotropic case) at each computational step. Alternatively, it  can be solved using perturbation theory \cite[]{Bender}
by approximating equation~\ref{eq:eikonalti} with a series of simpler linear equations. Considering $\theta$
constant and small, we can represent the traveltime solution as a series expansion in $\theta$. This will result in a solution that is globally representative in the
space domain and, despite the approximation of small $\theta$, 
the accuracy for even large $\theta$, as we will see later, is high. The constant-$\theta$ assumption
assumes a factorized medium \cite[]{alkhalifah:1139}
 in $\theta$ (useful for smooth $\theta$ \geouline{estimation applications}\geosout{model development application}). However,
all other velocities and parameters including $v_t$, $v$ (or $\delta$) and $\eta$ are allowed to vary\geosout{ freely}.
Figure~\ref{fig:scanTheta} illustrates
the concept of the global expansion as we predict the traveltime for
any $\theta$ from its behavior at $\theta=0$ for the full traveltime
field
using, in this case, a quadratic approximation. Specifically, we
 substitute the following trial solution,
\begin{equation}
 \tau(x,z) \approx \tau_0(x,z) +\tau_1(x,z) \sin\theta+ \tau_2(x,z)  \sin^{2}\theta,
\label{eqn:n0e}
\end{equation}
where $\tau_0$,  $\tau_1$, and $\tau_2$ are coefficients of the
expansion with \geosout{units} \geouline{dimensions} of traveltime, into the eikonal equation~\ref{eq:eikonalti}.
For practical purposes, I consider here only three terms of the expansion. 
As a result \geosout{and} (as shown in Appendix A), $\tau_0$ satisfies the eikonal equation for VTI anisotropy, while $\tau_1$ and $\tau_2$
 satisfy linear first-order partial differential equations having \geosout{, as shown in Appendix A,} the following general form \geouline{(see Appendix A)}:
\begin{equation}
{v^2} (1+2 \eta) \,{\frac{\partial \tau_{0}}{\partial x}  \frac{\partial \tau_{i}}{\partial x}  } + 
    {{{v_t}}^2}\,{\frac{\partial \tau_{0}}{\partial z}  \frac{\partial \tau_{i}}{\partial z}}\,
     - 2 \eta {v^2} {v_{t}^2} \frac{\partial \tau_{0}}{\partial z} \frac{\partial \tau_{0}}{\partial x}  
     \left( \,{ \frac{\partial \tau_{0}}{\partial z}  \frac{\partial \tau_{i}}{\partial x}} +
    { \frac{\partial \tau_{0}}{\partial x}  \frac{\partial \tau_{i}}{\partial z}} \right) =
f_i(x,z),
\label{eqn:allorder}
\end{equation}
with $i=1,2$. The functions $f_i(x,z)$ get more complicated \geouline{for larger $i$} \geosout{as $i$ is larger} and depend on terms that can be evaluated 
only sequentially. Therefore, these linear partial differential equations must be solved \geouline{in the order of increasing $i$} \geosout{in succession} starting with $i=1$. 

\section{Expansion in terms of $\theta$ and $\eta$}

Though the expansion in terms of $\theta$ in the previous section allowed us to estimate traveltimes for \geouline{a}  tilted symmetry axis, it also required 
that we solve the eikonal \geouline{equation} for a VTI medium, which
is relatively challenging. \geouline{For inversion purposes,} it also required knowledge of $\eta$, which might not be possible
in TI media using initially a VTI approximation, especially if the tilt is large.
 However, an expansion in $\eta$, in addition to $\theta$ \geouline{(from their zero values)}, 
 will result in an elliptically anisotropic background medium \geosout{, which is easily 
 solvable} and it will allow us to search for both $\eta$ and $\theta$, simultaneously, \geouline{considering that the elliptical anisotropy model is known}.
 
 The two-parameter expansion can be \geouline{obtained} \geosout{represented} by substituting the following trial solution:
 \begin{equation}
 \tau(x,z) \approx \tau_{0}(x,z) +\tau_{\eta}(x,z) \eta+\tau_{\theta}(x,z) \sin\theta+ \tau_{\eta_2}(x,z)  \eta^{2}+ \tau_{\eta \theta}(x,z)  \eta \sin\theta+ \tau_{\theta_2}(x,z)  \sin^{2}\theta
\label{eqn:n0ee}
\end{equation}
into equation~\ref{eq:eikonalti} resulting in linear first-order partial differential equations having the following general form:
\begin{equation}
v_t^2 \frac{\partial \tau _{0}}{\partial z} \frac{\partial \tau _i}{\partial z}+
v^2 \frac{\partial \tau _{0}}{\partial x} \frac{\partial \tau _i}{\partial x} =
f_i(x,z),
\label{eqn:allorder}
\end{equation}
with\old{, now,} $i=\eta,\theta,\eta_2,\eta \theta,\theta_2$, and $\tau_{0}$ satisfies the eikonal equation for an
 elliptical anisotropic background model. Again, the function
 $f_i(x,z)$ gets more complicated for $i$ corresponding to the
 second-order term and it depends on terms for the first order and
 background medium \geouline{solutions}. 
Therefore, these linear partial differential equations also must be solved in succession 
starting with $i=\eta$ and $i=\theta$. As soon as
the $\tau_{\eta}$,  and $\tau_{\eta_2}$ \geosout{functions} \geouline{coefficients} are evaluated, they can be used, as  \cite{etascan} showed,
 to estimate the traveltime using the first-sequence \geouline{of} Shanks transform \cite[]{Bender}\geosout{ in $\eta$},
and as shown in Appendix B, has the form:
\beq
\tau(x,z) \approx \tau_{0}(x,z)+ \tau_{\theta}(x,z) \sin\theta+ \tau_{\theta_2}(x,z)  \sin^{2}\theta
+\frac{\eta  \left(\tau_{\eta}(x,z)+ \tau_{\eta \theta}(x,z) \sin\theta  \right)^2}{\tau_{\eta}(x,z)+ \tau_{\eta \theta}(x,z) \sin\theta -\eta  \tau _{\eta_2}(x,z)}.
\label{eqn:shanks1m1}
\eeq
The $\theta$ expansion does not adapt well to the Shanks transform
requirements for \geosout{improvements} \geouline{predicting the
  behavior of the higher-order terms in $\theta$}.
In this case, the second-order approximation in the $\theta$ expansion
is sufficient.

 For $\eta$ and $\theta$ scan
applications, the coefficients ($\tau_{0}$, $\tau_{\eta}$, $\tau_{\theta}$, $\tau_{\eta_2}$, $\tau_{\eta \theta}$, and $\tau_{\theta_2}$) 
need to be evaluated only once and can be used with 
equation~\ref{eqn:shanks1m1} to search for the best traveltime fit
\geouline{to those traveltimes extracted from the data}.


\section{A homogeneous model test}

Though the equations above are developed for a general {\emph{inhomogeneous}} background medium, I examine their accuracy in representing
TI traveltime and traveltime moveout in the homogeneous case. This is convenient since most parameter scan-type applications \new{(i.e. semblance
velocity analysis)}
are \geouline{performed considering} \geosout{applied for} an effective homogeneous medium.

As shown in Appendix C, \old{we} \new{I} use the simple traveltime relation for an elliptically anisotropic homogeneous background 
\geosout{assumption} to recursively solve for the 
coefficients of the traveltime expansion in $\theta$ and  $\eta$, and thus, obtain
analytical representations for \geosout{traveltimes} \geouline{coefficients} $\tau_{0}$, $\tau_{\theta}$,
$\tau_{\eta}$, $\tau_{\theta_2}$, $\tau_{\eta_2}$, and $\tau_{\eta \theta}$. Setting
$\eta=0$,  to allow for a simplified presentation,
 \old{we} \new{I} obtain an analytical representation \geouline{of traveltime} for tilted elliptical anisotropy given by
\beqa
\tau(x,z)= \sqrt{\frac{x^2}{v^2}+\frac{z^2}{v_t^2}}
   \left(1+\frac{\left(v_t^2-v^2\right) x z \sin\theta }{v^2 z^2+v_t^2
   x^2} + \right. \nonumber \\ \left. \frac{\sin^2\theta \left(-v^4 z^4+v^2 v_t^2
   \left(x^4+z^4\right)-v_t^4 x^4\right)}{2 \left(v^2 z^2+v_t^2
   x^2\right)^2}\right),
\label{eqn:eikeqn2}
\eeqa
where the source is located at $x=0$ and $z=0$.

This \old{type} formula \new{type} (equation~\ref{eqn:eikeqn2}) basically represents a moveout equation for traveltime in TI media as a function of offset (or $x$)
and can be compared with \geouline{equations developed explicitly to represent the moveout in TI media}\geosout{other equations
out there}. \cite{GEO68-05-16001610} derived the exact quartic moveout coefficient (i.e., the fourth-order term of the Taylor series expansion for squared traveltime) for pure (nonconverted) reflections in arbitrarily anisotropic, heterogeneous media. They also linearized the P-wave quartic coefficient in the anisotropy parameters for homogeneous tilted TI media above a horizontal and dipping reflector. For a horizontal TI layer, the P-wave fourth-order Taylor series with linearized expressions for both the NMO velocity \cite[]{GEO65-01-02320246} and the quartic coefficient \cite[]{GEO68-05-16001610} is given by:
\beqa
t^2(X)= t_0^2+\frac{\left(1-2 \delta
   +2 \epsilon  \sin^2\theta-14 (\epsilon-\delta) \sin^2\theta \cos^2\theta\right)}{v_t^2} X^2 +A_{4} X^4,
\label{eqn:eikeqnT}
\eeqa
where
\beqa
A_{4} =- \frac{2 \eta \cos^4\theta}{v_t^4 t_0^2},
\label{eqn:eikeqnT2}
\eeqa
$t_0$ is the two-way zero-offset time and $X$ is the offset ($=2x$).
For zero tilt, it reduces to
\beqa
t^{2}(X,\theta=0)=t_0^2+\frac{\left(1-2 \delta \right)}{v_t^2} X^2 -\frac{2 \eta}{v_t^4 t_0^2 } X^4,
\label{eqn:eikeqnnT}
\eeqa
which \new{reveals} \old{demonstrates} the \new{additional} \old{level of} approximation
 involved in this equation as even the second-order term is \old{a} linearized \new{with respect to} \old{approximation
in} $\delta$. Meanwhile, setting $\theta=0$ in equation~\ref{eqn:eikeqn2} yields
an accurate description of the second-order term of the moveout with the NMO velocity in the denominator instead.

\cite{grechka:D1} suggested rewriting equation~\ref{eqn:eikeqnT} in the following form:
\beqa
t^2(X)= t_0^2+\frac{\left(1-2 \delta
   +2 \epsilon  \sin^2\theta-14 (\epsilon-\delta) \sin^2\theta \cos^2\theta\right)}{v_t^2} X^2 +\frac{2 \eta A_{4}}{2 \eta- v^2
 (1+2 \eta) A_{4} X^{2}} X^4,
\label{eqn:eikeqnTT}
\eeqa
\geouline{which typically provides higher-order accuracy as the additional offset component
in the denominator of the fourth-order term tends to predict the
behavior at very large offsets.}
   
\cite{GEO56-12-20902101} presented an anisotropic approximation
 of the group velocity as a function of symmetry angle. These velocities can serve to
obtain moveout equations as well, and they are given by
\beqa
t(X)=\frac{\sqrt{X^2+4 z^2} \sqrt{2 (\delta -\epsilon ) \sin^4(\theta -\psi
   )-2 \delta  \sin ^2(\theta -\psi )+1}}{v_t},
   \label{eqn:eiksena}
\eeqa
where \new{the ray angle,} $\psi$, \old{the ray angle,} satisfies
\beq
\psi= \tan^{-1}\left(\frac{X}{2z}\right).
   \label{eqn:eiksena2}
\eeq

Using an elliptical anisotropic background model with axis-direction
velocity equal to 2 km/s velocity, $\delta=0.2$, and tilt angle $\theta$=$20^{o}$, 
I compare \geouline{the traveltime errors of} the moveout equations extracted from our eikonal based formulations with
those that are used for pure moveout approximations, equations~\ref{eqn:eikeqnTT} and~\ref{eqn:eiksena}. 
For a reflector at depth $z$=2 km, Figure~\ref{fig:errTEll3f} shows the percentage traveltime errors as a function of
offset for the equations given above. Clearly, equations~\ref{eqn:eikeqnTT} and~\ref{eqn:eiksena}, given by the solid grey and
dashed curves, respectively,
 are less accurate in describing the traveltime behavior overall than
the new formula (solid black curve),
equation~\ref{eqn:eikeqn2}. The moveout equations have faired well
for the vertical direction but performed poorly for larger offset.
\inputdir{Math}
\plot{errTEll3f}{width=6in}{The relative traveltime error as a function of offset for an elliptically anisotropic
 model with $v$=2 km/s, $\delta=0.2$, $\theta=20^{0}$, and a reflector depth $z$=2 km for
the new expansion in $\theta$ (solid black curve), the nonhyperbolic moveout equation~\protect\ref{eqn:eikeqnT}  (solid gray curve), 
and for the Sena approximation
(dashed black curve).}

Moreover, for a more practical case of TI, in which $\eta$=0.2, the errors, as shown in Figure~\ref{fig:errTTI3fNew}, 
are \geosout{less} \geouline{smaller} overall for the new equation (solid black curve) than the other
approximations. Despite the larger error of our approximation near vertical, the departure
of the other approximations from the accurate value at large offset
reflect their near-offset based expansion. In the new equation, the
expansion is with respect to small tilt, and thus, has \geouline{an unbiased} representation of the traveltime with respect to
offset. As a result, the accuracy of the new equation is higher over all.
\plot{errTTI3fNew}{width=6in}{The relative 
traveltime error as a function of offset for a model with $v$=2 km/s, $\eta$=0.2, $\theta=20^{0}$, and a reflector depth $z$=2 km for
the new expansion in $\theta$ (solid black curve), the nonhyperbolic moveout equation~\protect\ref{eqn:eikeqnT}  (dashed gray curve), 
and~\protect\ref{eqn:eikeqnTT} (solid gray curve) and for Sena's approximation
(dashed black curve).}

The analytical equations developed in this section \old{was} \new{were}
 meant to test the accuracy of the perturbation theory approximations applied to the eikonal
equation. Though they show 
high accuracy in representing the moveout, the \new{general perturbation formulations} are not meant to be
only used as  an alternative to other available equations derived for
the homogeneous case. \new{Their perturbation from a background inhomogeneous model allows us to predict traveltime in more
complex media.}

\section{The vertical direction}

The vertical direction in the conventional seismic experiment
 is critical as the depth mistie (or vertical velocity) and the NMO velocity are typically measured \geosout{in the} \geouline{with respect to the} vertical direction
regardless of the tilt in the symmetry angle. Specifically, the vertical velocity is extracted from the well check shots (typically vertical) and the moveout
\geouline{velocity given} by the second derivative of traveltime with respect to \geouline{phase} angle, is
measured in the vertical direction ($x=0$). 
Setting $x=0$ in equation~\ref{eqn:eikeqn2}, and considering
\geouline{the} two-way traveltime, $t=2\tau$,
 yields the following relation for the traveltime in the vertical direction:
\beqa
t(x=0,z)=2 \frac{z}{v_{t}} \left(1-\frac{1}{2} \sin^{2}\theta \left(1-\frac{v_{t}^{2}}{v^{2}} \right) \right),
\label{eqn:eikeqn3}
\eeqa
which \geosout{clearly} includes terms related to the symmetry-axis \geouline{direction}. From Figure~\ref{fig:errTEll3f} we can see how well this
approximation predicted the vertical traveltime. This equation is convenient to use along with
well information to predict the velocities along the symmetry
axis. Actually, for $\eta \ne 0$ \geosout{not equal to zero} (\geouline{anelliptic} TI media), 
applying Pade approximations \cite[]{Bender} to the $\eta$ and $\sin\theta$ expansions, yields the following relation for traveltime in the vertical direction:
\beqa
t(x=0,z)=2 \frac{z}{v_{t}} \frac{v^{2}}{v^{2} +\frac{1}{2} \sin^{2}\theta \left(v^{2}-v_{t}^{2} \right)}.
\label{eqn:eikeqn4}
\eeqa
Note that this equation is independent \geouline{of $\eta$}, which implies that for small tilt angles, $\eta$ has practically no influence on the 
vertical traveltime.
This fact helps us \geosout{to} better construct the background
elliptical anisotropy, as $\eta$ has little influence on this process
for small tilt angles. Combine the above equations with those related to the 
NMO (stacking) velocity in the vertical direction:
\beqa
\frac{1}{t_0 v_{nmo}^{2}} = \frac{1}{2} \frac{\partial^{2} t}{\partial x^{2}}(x=0,z)= \frac{v_{t}}{z}
	\frac{v^{2} +\frac{3}{2} \sin^{2}\theta \left(v^{2}-v_{t}^{2} \right)}{v^{4}},
\label{eqn:eikeqn5}
\eeqa
where $t_0$ is given by equation~\ref{eqn:eikeqn3}.
\geosout{In the case} \geouline{If} we use Shanks transform again for higher-order accuracy, such an NMO velocity equation has the following form:
\beqa
\frac{1}{t_0 v_{nmo}^{2}} = \frac{z}{v_{t}} \frac{8 v_t^2 v^4-2 \left(4 v^2-9 v_t^2\right) \left(v^2-v_t^2\right)^2 \sin ^4\theta+24 v_t^2 \left(v^2-v_t^2\right) v^2 \sin ^2\theta}{z^2 \left(v^2 \left(\sin ^2\theta+2\right)-v_t^2 \sin ^2\theta\right)^3},
\label{eqn:eikeqn6}
\eeqa
where $t_0$ is given by equation~\ref{eqn:eikeqn5}.
Again, these equations are independent of $\eta$. The combination of equations~\ref{eqn:eikeqn3} and~\ref{eqn:eikeqn5}
or equations~\ref{eqn:eikeqn4} and~\ref{eqn:eikeqn6} for a known tilt direction can
be used to estimate the velocity along the \old{tilt} \new{symmetry axis}
 as well as $\delta$ ($=\frac{v^{2}-v_{t}^{2}}{2 v_{t}^{2}}$). Specifically, the above formulations provide a
mechanism to evaluate the symmetry-direction \new{(anisotropy)} parameters from \new{measurements obtained using
 the conventional seismic experiment, vertical and stacking velocities.}

\begin{comment}
In fact, we had to evaluate the fourth-order traveltime derivative,
\beqa
 \frac{\partial^{4} t}{\partial x^{4}}(x=0,z) &=& \frac{z}{v_{t}} 
 \frac{12}{v^2 z^4 \left(v^2 \left(-\sin ^2\theta-2\right)+v_t^2 \sin ^2\theta\right)^5}	\nonumber \\
& & \left(16 v_t^4 v^8 (8 \eta +1)-32 \left(v^{2}-v_t^{2}\right) v_t^2 v^6 \left(2 v^2-5 v_t^2 (2 \eta +1)\right) \sin ^2(\theta
   ) \right. \nonumber \\
& &   \left. -\left(v^2-v_t^2\right)^4 \left(32 v^4-96 v_t^2 v^2+3 v_t^4 (29-48 \eta )\right) \sin ^8\theta \right. \nonumber \\
& &   \left. +8 \left(v^2-v_t^2\right)^3 v^2
   \left(8 v^4-26 v_t^2 v^2+3 v_t^4 (26 \eta +3)\right) \sin ^6\theta \right. \nonumber \\
& &   \left. +8 \left(v^2-v_t^2\right)^2 v^4 \left(16 v^4-72 v_t^2 v^2+5
   v_t^4 (20 \eta +11)\right) \sin ^4\theta \right),
\label{eqn:eikeqn5}
\eeqa
to start seeing dependency on $\eta$. Note that we are using the traveltime
expansions in terms of medium parameters ($\eta$ and $\theta$) to estimate the expansion
with respect to offset, $x$. 
\end{comment}

\geouline{Focusing on the performance of approximation~\ref{eqn:shanks1m1} for small offsets
allows us to predict the accuracy  of 
equations~\ref{eqn:eikeqn3} and~\ref{eqn:eikeqn5} as they are derived
from equation~\ref{eqn:shanks1m1}.}
Figure~\ref{fig:verticaleta0} repeats the example of
Figure~\ref{fig:errTEll3f} with a tilt of $5^{o}$ (a), $10^{o}$
(b), and $20^{o}$ (c), and a focus on small offsets (near vertical). Of
course, errors increase with the increase in the tilt angle as all
approximations are \geosout{with respect to} \geouline{for} small tilt angles from vertical. Though the vertical \geouline{direction} error in the case of the
new equations is higher, the slope of the error is almost zero indicating
that the new equation should provide a good estimation of the NMO velocity (extracted from the second derivative of traveltime with respect to offset). This feature
is critical since the errors associated with the other approximations
(i.e. \cite{GEO56-12-20902101}) for the
NMO representation \geouline{are} large. This also explains the higher accuracy of
the new equations at
higher offsets as the error gradient is small. \geouline{It is also clear from
Figure~\ref{fig:verticaleta0} that using Pade
approximations to predict the higher-order terms of the expansion in
$\theta$ did not increase the accuracy much (the difference between
solid and dashed black curves). This is also \old{observe} \new{observed} for
anelliptic TI as we will see next.}
\plot{verticaleta0}{width=\textwidth}{The relative traveltime error as a function of
  offset (near zero offset) for an elliptical anisotropic
 model with $v$=2 km/s, $\delta=\epsilon=0.2$, $\theta=20^{0}$, and a reflector depth $z$=2 km for
the new expansion (equation~\protect\ref{eqn:shanks1m1}) in $\theta$
(dashed black curve), the new expansion with Pade approximation in $\theta$ (solid black curve), 
and for the Sena approximation
(solid gray curve). The plots correspond to (a) a 5 degree tilt,
(b) a 10 degree tilt, and (c) a 20 degree tilt.}

For anelliptic TI media with $\delta=0.1$ and $\epsilon=0.2$, \old{we} \new{I} obtain similar results. Figure~\ref{fig:vertical}
shows the traveltime error over a limited offset for three symmetry
tilt angles: (a) $5^{o}$, (b) $10^{o}$, and (c) $20^{o}$. Again,
the accuracy of the new equations is apparent in the slope of the
error \geouline{near zero offset}, implying that the NMO velocity representation is highly
accurate. The errors in the vertical velocity, though, \geosout{is} \geouline{are} the largest
for the new equations. However, the vertical velocity genrally has less influence than the NMO velocity on time processing objectives.
\plot{vertical}{width=\textwidth}{The relative traveltime error as a function of
  offset (near zero offset) for a TI
 model with $v$=2 km/s, $\delta=0.1$, $\epsilon=0.2$, $\theta=20^{0}$, and a reflector depth $z$=2 km for
the new expansion (equation~\protect\ref{eqn:shanks1m1}) in $\theta$
(dashed black curve), the new expansion with Pade approximation in $\theta$ (solid black curve), 
and for the Sena approximation
(solid gray curve). The plots correspond to (a) a 5 degree tilt,
(b) a 10 degree tilt, and (c) a 20 degree tilt.}

Finally, for a TI medium with $\delta=0.0$ and $\epsilon=0.2$
(Figure~\ref{fig:verticaleta2}) we observe
similar behavior with \geosout{less} \geouline{smaller} overall \geouline{relative} errors compared to
Figures~\ref{fig:verticaleta0} and ~\ref{fig:vertical}. As $\delta$
decreases, \geouline{the anisotropy influence} near the vertical direction decreases, and the effect of the
tilt is less pronounced. However, \geosout{in the larger tilt case}\geouline{when the tilt is larger},
Figure~\ref{fig:verticaleta2}c, the errors are large and comparable
to those in Figures~\ref{fig:verticaleta0}c and ~\ref{fig:vertical}c as the influence
of $\eta$ starts to appear.
\plot{verticaleta2}{width=\textwidth}{The traveltime error as a function of
  offset (near zero offset) for a TI anisotropic
 model with $v$=2 km/s, $\delta=0.0$, $\epsilon=0.2$, $\theta=20^{0}$, and a reflector depth $z$=2 km for
the new expansion (equation~\protect\ref{eqn:shanks1m1}) in $\theta$
(dashed black curve), the new expansion with Pade approximation in $\theta$ (solid black curve), 
and for the Sina approximation
(solid gray curve). The plots correspond to (a) a 5 degree tilt,
(b) a 10 degree tilt, and (c) a 20 degree tilt.}

In the above examples we note that despite the inferior accuracy of
equations~\ref{eqn:eikeqn3} and~\ref{eqn:eikeqn5} in representing
the vertical traveltime, with \geouline{relative} errors that could reach 0.3
percent for the 20 degree tilt case, it has superior qualities in
predicting the NMO velocity. This phenomenon is
explained by the fact that the new equations are expansions with
respect to tilt angle, while the other equations are expansions with respect
to offset (or anisotropy parameters), thus they provide better accuracy near zero offset. In contrast,
the new equations tend to be more offset independent and better
represent the moveout over larger offsets.


\section{The symmetry-axis azimuth and the 3-D case}

In 3D, the tilt of the symmetry axis is defined by an angle, $\theta$, measured from vertical, and the azimuth, $\phi$, of the vertical plane that 
contains the \geosout{tilt} \geouline{symmetry axis}. Thus, $\phi$ is an angle measured \geosout{along} \geouline{in} the horizontal plane from a given axis 
within that plane. To implement
an expansion with respect to $\phi$, we must consider $\phi$ to be generally small. Since \geosout{most} seismic acquisition is \geouline{often} performed
in the dip direction of the structure, and we anticipate that the tilt is influenced by the  presumed subsurface structure (folding), it would be
reasonable to measure $\phi$ from the acquisition \geosout{line} direction. In this case, \old{we} \new{I} can consider $\phi$ to be small, and thus, 
approximate the traveltime
solution \geouline{of the eikonal equation} with the following expansion:
\beqa
 \tau(x,y,z) \approx \tau_0(x,y,z) +\tau_1(x,y,z) \sin\phi+  \tau_2(x,y,z) \sin^2\phi,
\label{eqn:eikeqn2A}
\eeqa
to be inserted in the acoustic eikonal equation for TI in 3D given by
\beqa
a_4 v_t^4 \left(\frac{\partial \tau }{\partial z}\right)^4+a_3 v_t^3
   \left(\frac{\partial \tau }{\partial z}\right)^3+a_2 v_t^2
   \left(\frac{\partial \tau }{\partial z}\right)^2+a_1 v_t
   \frac{\partial \tau }{\partial z}+a_0=0,
   \label{eqn:eikeqn3A}
\eeqa
where $a_{0}$, $a_{1}$, $a_{2}$, $a_{3}$, and $a_{4}$ are stated in Appendix D.
This is a complicated eikonal \geouline{equation} that is highly nonlinear, and thus,
justifies our effort to simplify it through perturbation theory.

Substituting the trial solution, equation~\ref{eqn:eikeqn2A},  into the eikonal equation for 3D 
TI media, equation ~\ref{eqn:eikeqn3A}, yields a polynomial expansion in the powers of $\sin\phi$. The zero-order term of this polynomial
represents the eikonal equation for TI media for the zero-azimuth case (the 2-D result). The coefficient of the $\sin\phi$ term yields
a first-order PDE for $\tau_{1}$. \geosout{however} For simplicity, \geouline{it is} shown here for the case of elliptical anisotropy \geouline{($\eta=0$)} as
   \beqa
& & \left(\left(v^2 \cos ^2\theta+v_t^2 \sin ^2\theta\right) \frac{\partial \tau_0}{\partial x}+\left(v^2-v_t^2\right) \sin\theta \cos\theta \frac{\partial \tau_0}{\partial z} \right) \frac{\partial \tau_1}{\partial x} \nonumber \\ &+& v^2 \frac{\partial \tau_0}{\partial y} \frac{\partial \tau_1}{\partial y} \nonumber \\ &+&
\left(\left(v^2 \sin ^2\theta+v_t^2 \cos ^2\theta\right) \frac{\partial \tau_0}{\partial z} +\left(v^2-v_t^2\right) \sin\theta \cos
  \theta \frac{\partial \tau_0}{\partial x} \right)  \frac{\partial \tau_1}{\partial z}  \nonumber \\ &=& \left(v^2-v_t^2\right) \sin\theta  \frac{\partial \tau_0}{\partial y}
   \left(\cos\theta \frac{\partial \tau_0}{\partial z}-\sin\theta
    \frac{\partial \tau_0}{\partial x} \right).
\label{eqn:eikeqnA20}
\eeqa
We can obtain similar PDEs for the other coefficients of the
expansion, but obtaining the 2D background TI model will be a
challenging task. We can use the 2D version of the equation, developed earlier, to
manage this task; however, the influence of ignoring the azimuth on that
process could be critical.

A more realistic implementation is achieved by expanding from an elliptically anisotropic \geouline{background} with a vertical symmetry axis background
in terms of $\eta$, $\theta$, and $\phi$, simultaneously. However, since a vertical symmetry axis has no particular
azimuth (a singularity), I replace the tilt and azimuth by $n_{x}$ and $n_{y}$ ($n_{x} = \sin\theta \cos\phi$ and $n_{y}= \sin\theta \sin\phi$). 
For simplicity and to be able to include the resulting equations in this paper, I only consider first-order terms of the Taylor's series expansion,
and thus, consider the following trial solution:
 \beqa
 \tau(x,y,z) \approx \tau_{0}(x,y,z) +\tau_{\eta}(x,y,z) \eta+\tau_{n_{x}}(x,y,z) n_{x}+ \tau_{n_{y}}(x,y,z)  n_{y}.
\label{eqn:sol}
\eeqa
In Appendix D, we look at higher-order expansions in $\eta$ to
\geouline{take advantage} \geosout{make use}
of the Shanks transform \geouline{properties}\geosout{representation}, which considerably helps in the $\eta$ case. 
Inserting equation ~\ref{eqn:sol} into equation~\ref{eqn:eikeqn3A}
and equating terms of similar powers of the independent parameters
($\eta$, $n_{x}$, and $n_{y}$) yields an elliptically
anisotropic eikonal \geouline{equation with a} \geosout{for a} vertical symmetry axis from the zeroth-order term and first order PDEs from the 
other terms that have the same form as equation~\ref{eqn:allorder}. The source functions are shown in Appendix D.

\geouline{Assuming a} \geosout{Using a background} homogeneous-medium \geouline{background} \geosout{approximation} yields an
analytic relation, as shown in Appendix D [equation~\ref{eqn:homo14d}]: \geosout{given by}
\beqa
 \tau(x,y,z) & \approx & \sqrt{\frac{x^2}{v^2} +\frac{y^2}{v^2}+\frac{z^2}{v_t^2}} -\frac{v_t^4 \eta  \left(x^2+y^2\right)^2
   \sqrt{\frac{x^2+y^2}{v^2}+\frac{z^2}{v_t^2}}}{\left(v^2 z^2+v_t^2
   \left(x^2+y^2\right)\right)^2} \nonumber \\ &+& \frac{\left(v_t^2-v^2\right) z
   \sin\theta}{v^2 v_t^2
   \sqrt{\frac{x^2+y^2}{v^2}+\frac{z^2}{v_t^2}}} \left( x \cos\phi +y \sin\phi \right) .
\label{eqn:solhomo}
\eeqa
From this first-order approximation in all three parameters ($\eta$, $n_{x}$, and $n_{y}$), we can
see that by setting $v_t=v$ ($\delta=0$), the last term in the
equation, \geouline{that contains the} \geosout{related to} tilt \geouline{component,}
equals zero. This happens also if $\theta=0$ (axis is vertical), and
the traveltime is dependent only on $\eta$. The
$\eta$-term (second on the right hand side) in this first-order approximation has no dependence on
\geosout{tilt} \geouline{symmetry angle}. Also, note that this equation has a complex \geouline{variation with} \geosout{representation
in} offset (i.e. nonhyperbolic), however, reducing to a hyperbolic equation if $\eta$ and the
symmetry angle are equal to zero.


\section{Discussion}

The main objective of the \geouline{newly developed expressions} \geosout{expansions} is parameter estimation in complex media.
Specifically, the perturbation PDEs developed here are with respect to a background generally inhomogeneous, and
possibly anisotropic, medium. If a generally inhomogeneous \geouline{isotropic} velocity
field is available (for example from conventional 
migration velocity analysis), in addition to a map of the well-to-seismic misties, which
can be used to develop a vertical velocity field, then an elliptical
anisotropic model with a vertical symmetry \geouline{axis} can be constructed.
We can use this model to solve for traveltimes in elliptically
anisotropic media as a background model, as well as to solve for the expansion coefficients using equations~\ref{eqn:allorder}. 
These coefficients can be used with, for example, equation ~\ref{eqn:sol} to search 
explicitly for the $\eta$, and tilt angles $\theta$ and $\phi$ in 3D that provides the best traveltime fit to the data. 
This process can be implemented in a semblance-type search or incorporated as part of a
tomographic inversion. Though the scans are based on an underline factorized  assumption in the perturbation parameters,
$\eta$ and the tilt angles, we can allow them to vary 
 smoothly with location, and thus, produce effective values. The conversion of these
effective values to interval ones in generally inhomogeneous media is not trivial and might require a tomographic treatment
of its own.

\plot{etaTheta}{width=\textwidth}{The traveltime difference between the TTI model computed using equation~\protect\ref{eqn:eikeqn2} and the elliptically
anisotropic with a vertical symmetry axis background model for (a) an offset of 1 km, (b) an offset of 2 km, and (c) an offset of 4 km.  
The medium has $v_{t}$=2 km/s, $v$=2 km/s ($\delta=0$), and a reflector depth, $z$=2 km.}
In 3D, the search for $\eta$ and the symmetry direction angles can be applied \geouline{either} 
sequentially or \geosout{can be applied} to all the parameters at once. A sequential search, though
faster and easier, may propagate some of the errors of an initial
(wrong) tilt \geouline{into} the estimation of the parameter $\eta$ \cite[]{ttt}. The
search for all three parameters simultaneously would reduce such errors, but it will suffer from a null space based on the tradeoff between
$\eta$ and the tilt angles. Conventionally, the information for $\eta$ \geosout{is} \geouline{could be} extracted, especially for small tilt angles from vertical,
which is assumed here, from long offsets and dipping reflectors. The tilt information resolution in 3D requires a 3D coverage
(i.e. wide azimuth or even narrow azimuth for small tilt azimuth). There is also a general tradeoff, even in 2D media, between $\eta$ and the
tilt angle, which may require some a priori information for $\eta$ or constraining the tilt angle to be normal to the reflector dip \cite[]{SEG-2000-09650968}.
Figure~\ref{fig:etaTheta} shows the dependence of traveltime in the 2D case, based on equation~\ref{eqn:eikeqn2}, on the 
parameters $\eta$ and $\theta$. For a single offset, clearly there are combinations of $\eta$ and $\theta$ (given by the contour lines) that provide
equal traveltimes. Nevertheless these curves clearly vary from one offset to another. Specifically, near offsets (Figure~\ref{fig:etaTheta}a,
where $x/z=0.5$), for mostly small tilt angles, show little
dependence of traveltime on $\eta$, and more dependence on $\theta$. 
On the other hand, as the tilt angle increases the resolution of $\eta$ increases as
its influence starts to affect even the shorter offsets. The dependence of the traveltime
on $\eta$ increases for $x/z=1$ (Figure~\ref{fig:etaTheta}b) and $x/z=2$ 
(Figure~\ref{fig:etaTheta}c), which implies that $\eta$ is better resolved at large offsets for small tilt angles. However, it is resolved even better for
large tilt angles in all cases. Meanwhile, larger offsets with reasonable $\eta$ values result in less dependence of \geouline{the} traveltime on tilt angle.

The availability of multi-offset data will increase our chances in resolving both $\eta$ and the tilt angle in 2D. The addition
of multi azimuth should help resolve the tilt in 3D.
Of course, the accuracy of resolving these parameters will depend 
mainly on how well we \geosout{pin down} \geouline{estimate} the original elliptically anisotropic background medium. However, we can always go back
and improve on our velocity picks once an approximate effective $\eta$ and tilt-angle fields are estimated. There are probably
many \geouline{other more sophisticated} \geosout{smarter} ways to explore this parameter matrix, however, the equations introduced here provides the basis for doing
so.

\section{Conclusions}

Expanding the traveltime solutions of the TI eikonal equation in a power series in terms of independent
 parameters, like the tilt angle $\theta$ in 2D, provides an efficient tool to estimate $\theta$ in a generally {\emph{inhomogeneous}} background 
medium. Additional expansions from a background elliptical anisotropic medium with a vertical symmetry axis allows us
to search for the anisotropy parameter $\eta$ and the tilt angles simultaneously, even in 3D media. 
For a homogeneous background, \old{we} \new{I} obtain analytic nonhyperbolic moveout equations
for anisotropic media that are \geosout{exceptional in its simplicity and accuracy} \geouline{generally simple, and yet accurate}. Nevertheless, the formulations
provide $\eta$ and tilt-angles estimation capabilities \old{valid} for a general inhomogeneous background medium. 

\section{Acknowledgments}

I am grateful to KAUST for its financial support. I also thank Andrej Bona, Ilya Tsvankin, Ivan
Psencik, and Andres Pech for their critical and helpful reviews of the paper.

\bibliographystyle{seg}
%\bibliography{paper,SEG,TARIQ}
\bibliography{SEG,TARIQ,paper,SEP2}

\appendix
\section{Appendix A: Expansion in $\theta$}

To derive a traveltime equation in terms of perturbations in $\theta$, we first establish the form for the 
governing equation for TI media given by the eikonal representation.
The eikonal equation for \geouline{$p$-waves in} TI media in 2D (for simplicity) is 
given by
 \beqa
& &{v^2} (1+2 \eta) \,{\left(\cos\theta \frac{\partial \tau}{\partial x} + \sin\theta \frac{\partial \tau}{\partial z}\right)^2 } + \nonumber \\
   & & {{{v_t}}^2}\,{\left( \cos\theta \frac{\partial \tau}{\partial z}-\sin\theta \frac{\partial \tau}{\partial x} \right)^2}\,
     \left( 1 - 2 \eta {v^2} \,{ \left( \cos\theta \frac{\partial \tau}{\partial x} +\sin\theta \frac{\partial \tau}{\partial z} \right)^2} \right)=1.
\label{eq:eikonalti2}
\eeqa 
To
solve equation~\ref{eq:eikonalti2} through perturbation theory, we  assume that $\theta$ is small, and thus, a trial solution can be expressed 
as a series expansion in $\sin\theta$ given by
\begin{equation}
 \tau(x,z) \approx \tau_0(x,z) +\tau_1(x,z) \sin\theta+ \tau_2(x,z)  \sin^{2}\theta,
\label{eqn:n0e2}
\end{equation}
where $\tau_0$,  $\tau_1$ and  $\tau_2$ are coefficients of the expansion given in units of traveltime, and, for practicality,
terminated at the
second power of $\sin\theta$.
Inserting the trial solution, equation~\ref{eqn:n0e2}, into equation~\ref{eq:eikonalti2} yields a long formula, but by setting $\sin\theta=0$,
\old{we} \new{I} obtain the zeroth-order term given by
\beqa
{v^2} (1+2 \eta) \,{\left(\frac{\partial \tau_{0}}{\partial x}\right)^2} + 
    {{{v_t}}^2}\,{\left(\frac{\partial \tau_{0}}{\partial z}\right)^2}\,
     \left( 1 - 2 \eta {v^2} \,{ \left(\frac{\partial \tau_{0}}{\partial x}\right)^2} \right)=1,
\label{eq:eikonal2}
\eeqa
which is the eikonal formula for VTI anisotropy. By equating the coefficients of the powers of the
independent parameter $\sin\theta$, in succession,
we end up first with the coefficients of first-power in
$\sin\theta$, simplified by using equation~\ref{eq:eikonal2}, and given by
\beqa
 v^2 \frac{\partial \tau _1}{\partial x}
   \left( (2 \eta +1) \frac{\partial \tau _0}{\partial x}-2
   v_t^2 \eta  \frac{\partial \tau _0}{\partial x} \left(\frac{\partial
   \tau _0}{\partial z}\right)^2\right)+  v_t^2 \frac{\partial \tau _1}{\partial
   z} \left(\frac{\partial \tau _0}{\partial z}-2 v^2  \eta
    \left(\frac{\partial \tau _0}{\partial x}\right)^2 \frac{\partial
   \tau _0}{\partial z}\right)   = \nonumber \\
    2 v^2 v_t^2 \eta \left(  \frac{\partial
   \tau _0}{\partial x} \left(\frac{\partial \tau _0}{\partial
   z}\right)^3 -   \left(\frac{\partial \tau _0}{\partial x}\right)^3
   \frac{\partial \tau _0}{\partial z} \right) - v^2 (2 \eta +1) \frac{\partial \tau _0}{\partial x}
   \frac{\partial \tau _0}{\partial z}+ v_t^2 \frac{\partial \tau
   _0}{\partial x} \frac{\partial \tau _0}{\partial z},
\label{eqn:forder}
\eeqa
which is a first-order linear partial differential equation in $\tau_1$.
The coefficient of $\sin\theta^{2}$, with some manipulation, has the
following form
\beqa
2 v^2 \frac{\partial \tau_2}{\partial x} \left( (2 \eta +1)
   \frac{\partial \tau _0}{\partial x}-2 v_t^2 \eta  \frac{\partial
   \tau _0}{\partial x} \left(\frac{\partial \tau _0}{\partial
   z}\right)^2\right) +2 v_t^2   \frac{\partial \tau_2}{\partial z} \left( \frac{\partial \tau _0}{\partial
   z}-2 v^2  \eta  \left(\frac{\partial \tau _0}{\partial
   x}\right)^2 \frac{\partial \tau _0}{\partial
   z}\right) = \nonumber \\
       v^2 (2 \eta +1) \left(\frac{\partial \tau
   _0}{\partial x}\right)^2+2 v^2 v_t^2 \eta  \left(\frac{\partial
   \tau _0}{\partial z}\right)^4+2 v_t^2 \frac{\partial \tau
   _1}{\partial x} \frac{\partial \tau _0}{\partial z}-v_t^2
   \left(\frac{\partial \tau _1}{\partial z}-\frac{\partial \tau
   _0}{\partial x}\right)^2+v_t^2 \left(\frac{\partial \tau
   _0}{\partial z}\right)^2    \nonumber \\
+4 v^2 v_t^2 \eta  \frac{\partial \tau _1}{\partial x}
   \left(\frac{\partial \tau _0}{\partial z}\right)^3-12 v^2 v_t^2
   \eta  \left(\frac{\partial \tau _0}{\partial x}\right)^2
   \left(\frac{\partial \tau _0}{\partial z}\right)^2+2 v^2 v_t^2 \eta
    \left(\frac{\partial \tau _1}{\partial x}\right)^2
   \left(\frac{\partial \tau _0}{\partial z}\right)^2  \nonumber \\
   +12 v^2 v_t^2
   \eta  \frac{\partial \tau _0}{\partial x} \frac{\partial \tau
   _1}{\partial z} \left(\frac{\partial \tau _0}{\partial
   z}\right)^2-12 v^2 v_t^2 \eta  \left(\frac{\partial \tau
   _0}{\partial x}\right)^2 \frac{\partial \tau _1}{\partial x}
   \frac{\partial \tau _0}{\partial z}+8 v^2 v_t^2 \eta  \frac{\partial
   \tau _0}{\partial x} \frac{\partial \tau _1}{\partial x}
   \frac{\partial \tau _1}{\partial z} \frac{\partial \tau _0}{\partial
   z}   \nonumber \\
   -v^2 (2 \eta +1) \left(\frac{\partial \tau _1}{\partial
   x}+\frac{\partial \tau _0}{\partial z}\right)^2+2 v^2 v_t^2 \eta 
   \left(\frac{\partial \tau _0}{\partial x}\right)^2
   \left(\frac{\partial \tau _1}{\partial z}-\frac{\partial \tau
   _0}{\partial x}\right)^2-2 v^2 (2 \eta +1) \frac{\partial \tau
   _0}{\partial x} \frac{\partial \tau _1}{\partial z},
\label{eqn:sorder}
\eeqa
which is again a first-order linear partial differential equation in
$\tau_2$ with an obviously 
more complicated source function given by the right-hand side. 
Though the equation seems complicated, many of the variables of the source function (right-hand side) can be evaluated during the evaluation of 
equations~\ref{eq:eikonal2} and~\ref{eqn:forder} in a fashion that will not add much to the cost.

\appendix
\section{Appendix B: Expansion in $\theta$ and $\eta$}

For an expansion in $\theta$ and $\eta$, simultaneously, \old{we} \new{I} use the following trial solution:
 \begin{equation}
 \tau(x,z) \approx \tau_{0}(x,z) +\tau_{\eta}(x,z) \eta+\tau_{\theta}(x,z) \sin\theta+ \tau_{\eta_2}(x,z)  \eta^{2}+ \tau_{\eta \theta}(x,z)  \eta \sin\theta+ \tau_{\theta_2}(x,z)  \sin^{2}\theta,
\label{eqn:n0ee2}
\end{equation}
in terms of the coefficients $\tau_{i}$, where the $i$ corresponds to $\eta,\theta,\eta_2,\eta \theta$, and $\theta_2$. 
Inserting the trial solution, equation~\ref{eqn:n0ee2}, into equation~\ref{eq:eikonalti2} yields again a long formula, 
but by setting both $\sin\theta=0$ and $\eta=0$,
\old{we} \new{I} obtain the zeroth-order term given by
\begin{equation}
v^2(x,y,z) \left(\frac{\partial \tau_{0}}{\partial x}\right)^2  +
v_t^2(x,y,z) \left(\frac{\partial \tau_{0}}{\partial z}\right)^2 = 1\;,
\label{eqn:ellip}
\end{equation}
which is simply the eikonal formula for elliptical anisotropy. By equating the coefficients of the powers 
of the independent parameter $\sin\theta$ and $\eta$, in succession starting with first powers of the two parameters,
we end up first with the coefficients of first-power in $\sin\theta$ and zeroth power in $\eta$, 
simplified by using equation~\ref{eqn:ellip}, and  given by
\begin{equation}
 v^2 \frac{\partial \tau_{0}}{\partial x} \frac{\partial \tau_{\theta}}{\partial x}+ v_t^2 \frac{\partial \tau_{0}}{\partial z} \frac{\partial \tau
   _{\theta}}{\partial z} = - \left(v^2-v_t^{2}\right)
   \frac{\partial \tau_{0}}{\partial x} \frac{\partial \tau_{0}}{\partial
   z},
\label{eqn:fordert}
\end{equation}
which is a first-order linear partial differential equation in $\tau_{\theta}$. The 
coefficients of zero-power in $\sin\theta$ and the first-power in $\eta$ is given by
\begin{equation}
v^2
   \frac{\partial \tau_{0}}{\partial x} \frac{\partial \tau_{\eta}}{\partial
   x}+v_t^2 \frac{\partial \tau _{0}}{\partial z} \frac{\partial \tau_{\eta}}{\partial z}  = -  \left(v^2 \left(\frac{\partial \tau_{0}}{\partial x}\right)^2-v^2
   v_t^2 \left(\frac{\partial \tau_{0}}{\partial x}\right)^2
   \left(\frac{\partial \tau_{0}}{\partial z}\right)^2\right),
\label{eqn:fordere}
\end{equation}
The coefficients of the square terms in $\sin\theta$, with some manipulation, results in the following relation
\beqa
2 v^2 \frac{\partial \tau _{0}}{\partial x}
   \frac{\partial \tau _{\theta_2}}{\partial x}+2 v_t^2 \frac{\partial \tau _{0}}{\partial z}
   \frac{\partial \tau _{\theta_2}}{\partial z} = 
   v^2 \left(\frac{\partial \tau _{0}}{\partial
   x}\right)^2-v^2 \left(\frac{\partial \tau _{\theta}}{\partial
   x}\right)^2-   \nonumber \\   2 \left(v^{2}-v_t^{2}\right)
   \frac{\partial \tau _{\theta}}{\partial x}
   \frac{\partial \tau _{0}}{\partial z}-2 \left(v^{2}-v_t^{2}\right)
   \frac{\partial \tau _{0}}{\partial x} \frac{\partial
   \tau _{\theta}}{\partial z}- \nonumber \\ v_t^2 \left(\frac{\partial \tau _{0}}{\partial
   x}\right)^2-v_t^2 \left(\frac{\partial \tau _{\theta}}{\partial
   z}\right)^2-\left(v^{2}-v_t^{2}\right)
  \left(\frac{\partial \tau _{0}}{\partial
   z}\right)^2,
\label{eqn:sordert}
\eeqa
which is again a first-order linear partial differential equation in $\tau_{\theta_2}$ with an obviously more complicated source function given by the right hand side.
The coefficients of the square terms in $\eta$, with also some manipulation, results in the following relation
\beqa
2 v^2 \frac{\partial \tau _{0}}{\partial x} \frac{\partial \tau
   _{\eta_2}}{\partial x}+2 v_t^2 \frac{\partial \tau _{0}}{\partial z}
   \frac{\partial \tau _{\eta_2}}{\partial z} = 
   4 v_t^2 v^2 \frac{\partial \tau _{0}}{\partial x} \frac{\partial \tau
   _{0}}{\partial z} \left(\frac{\partial \tau _{\eta}}{\partial x}
   \frac{\partial \tau _{0}}{\partial z}+   \frac{\partial \tau _{0}}{\partial
   x} \frac{\partial \tau _{\eta}}{\partial z}\right)-   \nonumber \\ v^2
   \left(\frac{\partial \tau _{\eta}}{\partial x}\right)^2-   4 v^2
   \frac{\partial \tau _{0}}{\partial x} \frac{\partial \tau _{\eta}}{\partial
   x}-v_t^2 \left(\frac{\partial \tau _{\eta}}{\partial
   z}\right)^2,
\label{eqn:sordere}
\eeqa
which is again a first-order linear partial differential equation in $\tau_{\eta_2}$ with a again complicated source function.

Finally, the coefficients of the first-power terms in both $\sin\theta$ and $\eta$ 
results also in a first-order linear partial differential  equation in $\tau_{\eta \theta}$ given by
\beqa
& & 2 v^2 \frac{\partial
   \tau _{0}}{\partial x} \frac{\partial \tau _{\eta \theta}}{\partial x}+2
   v_t^2 \frac{\partial \tau _{0}}{\partial z} \frac{\partial \tau
   _{\eta \theta}}{\partial z}  = \nonumber \\
& &  4 v_t^2 v^2 \frac{\partial \tau _{0}}{\partial x} \frac{\partial \tau
   _{0}}{\partial z} \left(\frac{\partial \tau _{0}}{\partial z}
   \left(\frac{\partial \tau _{\theta}}{\partial x}+\frac{\partial \tau
   _{0}}{\partial z}\right)+\frac{\partial \tau _{0}}{\partial x}
   \left(\frac{\partial \tau _{\theta}}{\partial z}-\frac{\partial \tau
   _{0}}{\partial x}\right)\right)  \nonumber \\ & & -2 v^2 \frac{\partial \tau _{0}}{\partial
   x} \frac{\partial \tau _{\eta}}{\partial z}-4 v^2 \frac{\partial \tau
   _{0}}{\partial x} \left(\frac{\partial \tau _{\theta}}{\partial
   x}+\frac{\partial \tau _{0}}{\partial z}\right)-2 v^2 \frac{\partial
   \tau _{\eta}}{\partial x} \left(\frac{\partial \tau _{\theta}}{\partial
   x}+\frac{\partial \tau _{0}}{\partial z}\right)  \nonumber \\  & &+2 v_t^2
   \frac{\partial \tau _{\eta}}{\partial x} \frac{\partial \tau _{0}}{\partial
   z}-2 v_t^2 \frac{\partial \tau _{\eta}}{\partial z} \left(\frac{\partial
   \tau _{\theta}}{\partial z}-\frac{\partial \tau _{0}}{\partial x}\right).
\label{eqn:torder}
\eeqa
Though the equation seems complicated, many of the variables of the source function (right hand side) can be evaluated during the evaluation of 
equations~\ref{eqn:fordert} and~\ref{eqn:fordere} in a fashion that will not add much to the cost.

Using Shanks transforms \cite[]{Bender} we can isolate and remove the most transient behavior of the expansion~\ref{eqn:n0ee2} 
in $\eta$ (the $\theta$ expansion did not improve with such a treatment) by first defining the 
following parameters:
\beqa
A_0 &=& \tau_{0}+ \tau_{\theta} \sin\theta+ \tau_{\theta_2}  \sin^{2}\theta  \nonumber \\
A_1 &= &A_{0} + \left(\tau_{\eta}+ \tau_{\eta \theta} \sin\theta  \right) \eta    \nonumber \\
A_2 &=& A_{1}  + \tau_{\eta_2} \eta^2 
\label{eqn:shanks}
\eeqa
The first sequence of Shanks transforms uses  $A_0$, $A_1$, and $A_2$, and thus, is given by
\beqa
\tau(x,z) \approx \frac{A_0 A_2-A_1^2}{A_0-2 A_1+A_2} = \tau_{0}(x,z)+ \tau_{\theta}(x,z) \sin\theta+ \tau_{\theta_2}(x,z)  \sin^{2}\theta \nonumber \\
+\frac{\eta  \left(\tau_{\eta}(x,z)+ \tau_{\eta \theta}(x,z) \sin\theta  \right)^2}{\tau_{\eta}(x,z)+ \tau_{\eta \theta}(x,z) \sin\theta -\eta  \tau _{\eta_2}(x,z)}.
\label{eqn:shanks1}
\eeqa


\appendix
\section{Appendix C: The homogeneous medium case}

To develop analytical traveltime representation for TI media, \old{we} \new{I} start with a background velocity model that is homogeneous. 
The expansion here will be with respect to $\eta$ and $\theta$ from a background elliptical anisotropic model.
In this case, the traveltime \geouline{from a point source at $x=0$ and $z=0$} is given by the following simple relation in 2-D:
\beq
\tau_{0}(x,z) = \sqrt{\frac{x^2}{v^{2}}+\frac{z^2}{v^{2}_{t}}},
\label{eqn:homo1}
\eeq
which satisfies the eikonal equation~\ref{eqn:ellip}. Using equation~\ref{eqn:homo1}, \old{we} \new{I} evaluate $\frac{\partial \tau _{0}}{\partial x}$ and
$\frac{\partial \tau _{0}}{\partial z}$ and insert them into equation~\ref{eqn:fordert} to solve the first-order linear equation to obtain 
\beq
\tau_{\theta}(x,z) = \frac{\left(v_t^2-v^2\right) x z
   \sqrt{\frac{x^2}{v^2}+\frac{z^2}{v_t^2}}}{v^2 z^2+v_t^2 x^2},
\label{eqn:homo2}
\eeq
as well as insert them into equation~\ref{eqn:fordere} and solve the equation to obtain
\beq
\tau_{\eta}(x,z) = -\frac{v_t^4 x^4 \sqrt{\frac{x^2}{v^2}+\frac{z^2}{v_t^2}}}{\left(v^2
   z^2+v_t^2 x^2\right)^2},
\label{eqn:homo22}
\eeq

\old{We} \new{I} now evaluate $\frac{\partial \tau _{\theta}}{\partial x}$ and $\frac{\partial \tau _{\theta}}{\partial z}$ and use them to solve equation~\ref{eqn:sordert}.
After some tedious algebra, I obtain
\beq
\tau_{\theta_2}(x,z) = \frac{\sqrt{\frac{x^2}{v^2}+\frac{z^2}{v_t^2}} \left(-v^4 z^4+v^2 v_t^2
   \left(x^4+z^4\right)-v_t^4 x^4\right)}{2 \left(v^2 z^2+v_t^2
   x^2\right)^2}.
\label{eqn:homo3}
\eeq
I also evaluate $\frac{\partial \tau _{\eta}}{\partial x}$ and $\frac{\partial \tau _{\eta}}{\partial z}$ 
and use them to solve equation~\ref{eqn:sordere} to obtain
\beq
\tau_{\eta_2}(x,z) = \frac{3 v_t^6 x^6 \sqrt{\frac{x^2}{v^2}+\frac{z^2}{v_t^2}} \left(4 v^2
   z^2+v_t^2 x^2\right)}{2 \left(v^2 z^2+v_t^2 x^2\right)^4}.
\label{eqn:homo32}
\eeq

Finally, I solve equation~\ref{eqn:torder}.
After some tedious algebra once again, I obtain
\beq
\tau_{\eta \theta}(x,z) = -\frac{v_t^4 x^3 z \sqrt{\frac{x^2}{v^2}+\frac{z^2}{v_t^2}}
   \left(\left(3 v^2+v_t^2\right) x^2+4 v^2 z^2\right)}{\left(v^2
   z^2+v_t^2 x^2\right)^3}.
\label{eqn:homo4}
\eeq

Using the first sequence of Shanks transform, equation~\ref{eqn:shanks1}, applied to the Taylor's series expansion, we obtain an
analytical equation that describes traveltime as a function of $\eta$ and $\theta$. 

\begin{comment}
the following:
\beqa
\tau(x,z)=\sqrt{\frac{x^2}{v^2}+\frac{z^2}{v_t^2}} \left( \frac{ 2
   \left(v^{2}-v_t^{2}\right) z \sin ^3\theta \left(v_t^{2} x^4-v^{2} z^4\right) \left(v^2 z^2+v_t^2
   x^2\right) \left(\left(3 v^2+v_t^2\right) x^2+4 v^2 z^2\right)}{2 \left(v^2 z^2+v_t^2 x^2\right){}^2
   \left(2 v^4 z^3 \left(3 x^2 \sin\theta+x z+4 z^2 \sin (\theta
   )\right)+2 v_t^2 v^2 x^2 z \left(3 x^2 \sin\theta+6 x z \eta +2 x
   z+5 z^2 \sin\theta\right)+v_t^4 x^4 (x (3 \eta +2)+2 z \sin
  \theta)\right)} + \right. \nonumber \\ \frac{2 x
   \left(v^2 z^2+v_t^2 x^2\right){}^2 \left(2 v^4 z^4+4 v^2 v_t^2 x^2
   z^2 (3 \eta +1)+v_t^4 x^4 (\eta +2)\right)}{2 \left(v^2 z^2+v_t^2 x^2\right){}^2
   \left(2 v^4 z^3 \left(3 x^2 \sin\theta+x z+4 z^2 \sin (\theta
   )\right)+2 v_t^2 v^2 x^2 z \left(3 x^2 \sin\theta+6 x z \eta +2 x
   z+5 z^2 \sin\theta\right)+v_t^4 x^4 (x (3 \eta +2)+2 z \sin
  \theta)\right)} + \nonumber \\  \left.
      \frac{x \sin ^2\theta
   \left(18 v^6 \left(v_t^2-v^2\right) z^8-4 v_t^4 x^6 z^2 \left(v^4 (6
   \eta +2)+v_t^2 v^2 (9 \eta -1)+v_t^4 (\eta -1)\right)-v^2 v_t^2 x^4
   z^4 \left(22 v^4+v_t^2 v^2 (99 \eta +4)+v_t^4 (29 \eta -26)\right)-4
   v^4 x^2 z^6 \left(3 v^4+v_t^2 v^2 (3 \eta +7)+v_t^4 (13 \eta
   -10)\right)+\left(v-v_t\right) v_t^6 \left(v+v_t\right) x^8 (3 \eta
   +2)\right)+2 z \sin\theta \left(8 v^8 z^8+v_t^6 x^8 \left(v^2
   (4-15 \eta )-v_t^2 (\eta -4)\right)+v^2 v_t^4 x^6 z^2 \left(3 v^2
   (4-9 \eta )-5 v_t^2 (\eta -4)\right)+4 v^6 \left(v^2+7 v_t^2\right)
   x^2 z^6-4 v^4 v_t^2 x^4 z^4 \left(3 v^2 (\eta -1)+v_t^2 (\eta
   -9)\right)\right)}{2 \left(v^2 z^2+v_t^2 x^2\right){}^2
   \left(2 v^4 z^3 \left(3 x^2 \sin\theta+x z+4 z^2 \sin (\theta
   )\right)+2 v_t^2 v^2 x^2 z \left(3 x^2 \sin\theta+6 x z \eta +2 x
   z+5 z^2 \sin\theta\right)+v_t^4 x^4 (x (3 \eta +2)+2 z \sin
  \theta)\right)}  \right).
\label{eqn:homo5}
\eeqa
\end{comment}

For 3-D media, \old{we} \new{I} include the azimuth angle as we will see next.


\appendix
\section{Appendix D: Expansion in 3D}

\geouline{The eikonal equation for $P$-waves in a TI medium with a tilt in the symmetry axis satisfies the following relation,}
\beqa
a_4 v_t^4 \left(\frac{\partial \tau }{\partial z}\right)^4+a_3 v_t^3
   \left(\frac{\partial \tau }{\partial z}\right)^3+a_2 v_t^2
   \left(\frac{\partial \tau }{\partial z}\right)^2+a_1 v_t
   \frac{\partial \tau }{\partial z}+a_0=0,
   \label{eqn:eikeqn3D}
\eeqa
where
\beqa
a_0 &=& 2 v^2 v_t^2 \eta  \sin ^2\theta \cos ^2\theta \left(\cos\phi \frac{\partial \tau }{\partial x}-\sin\phi \frac{\partial
   \tau }{\partial y}\right)^4-v^2 (2 \eta +1) \cos ^2\theta
   \left(\cos\phi \frac{\partial \tau }{\partial x}-\sin\phi
   \frac{\partial \tau }{\partial y}\right)^2 \nonumber \\  &+& 2 v^2 v_t^2 \eta  \sin
   ^2\theta \left(\sin\phi \frac{\partial \tau }{\partial x}+\cos\phi \frac{\partial \tau }{\partial y}\right)^2 \left(\cos\phi
   \frac{\partial \tau }{\partial x}-\sin\phi \frac{\partial \tau
   }{\partial y}\right)^2   \nonumber \\ &-& v^2 (2 \eta +1) \left(\sin\phi
   \frac{\partial \tau }{\partial x}+\cos\phi \frac{\partial \tau
   }{\partial y}\right)^2  - v_t^2 \sin ^2\theta \left(\cos\phi
   \frac{\partial \tau }{\partial x}-\sin\phi \frac{\partial \tau
   }{\partial y}\right)^2+1,  \\ 
   a_1 &=& -\frac{2}{v_t} \sin\theta \cos\theta
   \left(\cos\phi \frac{\partial \tau }{\partial x}-\sin\phi
   \frac{\partial \tau }{\partial y}\right) \left(v_t \left(2 v^2 \eta 
   \left(\sin ^2\theta \cos\phi \frac{\partial \tau }{\partial x}
   \left(2 v_t \sin\phi \frac{\partial \tau }{\partial
   y}-1\right)  \right. \right. \right. \nonumber  \\  &+& \left. \left. \left.  v_t \left(\frac{\partial \tau }{\partial x}\right)^2
   \left(\cos ^2\theta \cos ^2\phi+\sin ^2(\phi
   )\right)+\frac{\partial \tau }{\partial y} \left(v_t \frac{\partial
   \tau }{\partial y} \left(\cos ^2\theta \sin ^2\phi+\cos ^2(\phi
   )\right) \right. \right. \right. \right. \nonumber  \\  &+& \left. \left. \left. \left. \sin ^2\theta \sin\phi\right)\right)-v_t\right)+v^2
   (2 \eta +1)\right), \\  a_{2} &=& \frac{1}{4}  \left(v^2 \eta 
   \left(-4 \sin ^2\theta (3 \cos (2\theta )+2) \cos (2\phi )
   \left(\left(\frac{\partial \tau }{\partial
   x}\right)^2-\left(\frac{\partial \tau }{\partial y}\right)^2\right)+8
   \sin ^2\theta (3 \cos (2\theta )+2) \right. \right. \nonumber \\ & & \left. \left.  \sin (2\phi ) \frac{\partial
   \tau }{\partial x} \frac{\partial \tau }{\partial y}+(2 \cos (2
  \theta )+3 \cos (4\theta )+3) \left(\left(\frac{\partial \tau
   }{\partial x}\right)^2+\left(\frac{\partial \tau }{\partial
   y}\right)^2\right)\right)-4 \cos ^2\theta\right) \nonumber \\  &-&  \frac{v^2 (2 \eta +1)
   \sin ^2\theta}{v_t^2},  \\  a_3 &=& \frac{v^2 \eta  \sin (4\theta ) \left(\cos
  \phi \frac{\partial \tau }{\partial x}-\sin\phi \frac{\partial
   \tau }{\partial y}\right)}{v_t}, \\  a_{4} &=& \frac{v^2 \eta  \sin ^2\theta \cos
   ^2\theta}{v_t^2}.
\label{eqn:eiktti}
\eeqa

To develop equations for the coefficients of a traveltime expansion in 3D from a background elliptical anisotropy with
a vertical symmetry axis \old{we} \new{I} use vector notations ($n_{x}$ and $n_{y}$) to describe the tilt angles, where the components of this 2D
vector describe the projection of the symmetry axis  on each of the $x-z$ and $y-z$ planes, respectively. As a result,
\beq
n_{x} = \sin\theta \cos\phi,
\label{eqn:nx}
\eeq
and
\beq
n_{y} = \sin\theta \sin\phi.
\label{eqn:ny}
\eeq
Using these two equations to solve for $\sin\theta$ and $\sin\phi$ and plugging them into equation~\ref{eqn:eikeqn3D} yields
an eikonal for TTI media in terms of $n_x$ and $n_{y}$. Thus, inserting the following trial solution
\beqa
 \tau(x,y,z) \approx \tau_{0}(x,y,z) +\tau_{\eta}(x,y,z) \eta+\tau_{\eta_{2}}(x,y,z) \eta^{2}+\tau_{n_{x}}(x,y,z) n_{x}+ \tau_{n_{y}}(x,y,z)  n_{y},
\label{eqn:sol2}
\eeqa
where $\eta$, $n_{x}$, and $n_{y}$ are independent
parameters and small, into the eikonal equation yields an extremely long equation. 
Again, setting the coefficients of the independent parameters ($\eta$, $n_{x}$, and $n_{y}$) to zero in the equation gives
the eikonal equation for elliptical anisotropy with vertical symmetry axis. On the other hand, the coefficients of the first power
of the independent parameters yield:
\beqa
 v^2 \frac{\partial \tau _0}{\partial x}
   \frac{\partial \tau _{\eta }}{\partial x}+v^2 \frac{\partial \tau
   _0}{\partial y} \frac{\partial \tau _{\eta }}{\partial y}+v_t^2
   \frac{\partial \tau _0}{\partial z} \frac{\partial \tau _{\eta
   }}{\partial z} &=& v^2 \left( \left( v_t^2 \left(\frac{\partial \tau _0}{\partial z}\right)^2 -1 \right)
   \left(\left(\frac{\partial \tau _0}{\partial
   x}\right)^2+\left(\frac{\partial \tau _0}{\partial
   y}\right)^2\right)\right),  \nonumber \\
  v^2 \frac{\partial \tau _0}{\partial y} \frac{\partial \tau
   _{n_x}}{\partial y}+v^2 \frac{\partial \tau _0}{\partial x}
   \frac{\partial \tau _{n_x}}{\partial x}+v_t^2 \frac{\partial \tau
   _0}{\partial z} \frac{\partial \tau _{n_x}}{\partial
   z}  &=& -\left(v^{2}-v_t^{2}\right)  \frac{\partial \tau
   _0}{\partial x} \frac{\partial \tau _0}{\partial z},  \nonumber \\
   v^2 \frac{\partial \tau _0}{\partial y} \frac{\partial \tau
   _{n_y}}{\partial y}+v^2 \frac{\partial \tau _0}{\partial x}
   \frac{\partial \tau _{n_y}}{\partial x}+v_t^2 \frac{\partial \tau
   _0}{\partial z} \frac{\partial \tau _{n_y}}{\partial
   z}  &=& -\left(v^{2}-v_t^{2}\right)  \frac{\partial \tau
   _0}{\partial y} \frac{\partial \tau _0}{\partial z},
   \label{eqn:pde3d}
   \eeqa
corresponding to $\eta$, $n_{x}$, and $n_{y}$, respectively. 

%Since,
%the background medium is elliptical anisotropic with a vertical
%symmetry axis $v_t$ and $v$ in equations~\ref{eqn:pde3d} correspond
%  to the background vertical direction.

The coefficient of the $\eta^{2}$ term, for higher accuracy in $\eta$, is given by
\beqa
& & 2 v^2 \frac{\partial \tau
   _0}{\partial x} \frac{\partial \tau _{\eta _2}}{\partial x}+2 v^2 \frac{\partial \tau _0}{\partial y}
   \frac{\partial \tau _{\eta _2}}{\partial y} +2 v_t^2
   \frac{\partial \tau _0}{\partial z} \frac{\partial \tau _{\eta
   _2}}{\partial z}= 4 v_t^2 v^2 \left(\frac{\partial \tau _0}{\partial z}\right)^2
   \left(\frac{\partial \tau _0}{\partial x} \frac{\partial \tau _{\eta
   }}{\partial x}+\frac{\partial \tau _0}{\partial y} \frac{\partial
   \tau _{\eta }}{\partial y}\right)   \nonumber \\ &+& 4 v_t^2 v^2 \frac{\partial \tau
   _0}{\partial z} \left(\left(\frac{\partial \tau _0}{\partial
   x}\right)^2+\left(\frac{\partial \tau _0}{\partial
   y}\right)^2\right) \frac{\partial \tau _{\eta }}{\partial z}-v^2
   \left(\left(\frac{\partial \tau _{\eta }}{\partial
   x}\right)^2 + \left(\frac{\partial \tau _{\eta }}{\partial y}\right)^2 \right)-4 v^2 \left( \frac{\partial \tau _0}{\partial x}
   \frac{\partial \tau _{\eta }}{\partial x}+
   \frac{\partial \tau _0}{\partial y} \frac{\partial \tau _{\eta
   }}{\partial y} \right) \nonumber \\ &-& v_t^2
   \left(\frac{\partial \tau _{\eta }}{\partial z}\right)^2.
   \label{eqn:or2}
\eeqa
These first-order PDEs, when solved, provide traveltime approximations using equation~\ref{eqn:sol2} for 3D TI media in a generally inhomogeneous
elliptical anisotropic background.

For a homogeneous medium simplification, the traveltime is given by the following analytical relation in 3-D
elliptical anisotropic media:
\beq
\tau_{0}(x,y,z) = \sqrt{\frac{x^2+y^{2}}{v^{2}}+\frac{z^2}{v^{2}_{t}}},
\label{eqn:homo13d}
\eeq
which satisfies the eikonal equation~\ref{eqn:ellip} in 
3D. Using equation~\ref{eqn:homo13d}, \old{we} \new{I} evaluate $\frac{\partial \tau _{0}}{\partial x}$, $\frac{\partial \tau _{0}}{\partial y}$ and
$\frac{\partial \tau _{0}}{\partial z}$ and insert them into equations~\ref{eqn:pde3d} to solve these first-order linear equations to obtain 
\beqa
\tau_{\eta}(x,y,z) &=&-\frac{v_t^4 \left(x^2+y^2\right)^2
   \sqrt{\frac{x^2+y^2}{v^2}+\frac{z^2}{v_t^2}}}{\left(v^2 z^2+v_t^2
   \left(x^2+y^2\right)\right)^2}, \nonumber \\
\tau_{n_{x}}(x,y,z)&=& \frac{\left(v_t^2-v^2\right) x z}{v^2 v_t^2
   \sqrt{\frac{x^2+y^2}{v^2}+\frac{z^2}{v_t^2}}}, \nonumber \\
\tau_{n_{y}}(x,y,z) &=& \frac{\left(v_t^2-v^2\right) y z}{v^2 v_t^2
   \sqrt{\frac{x^2+y^2}{v^2}+\frac{z^2}{v_t^2}}},
\label{eqn:homo14d}
\eeqa
respectively.

\old{We} \new{I} now evaluate $\frac{\partial \tau _{\eta}}{\partial x}$, $\frac{\partial \tau _{\eta}}{\partial y}$, and $\frac{\partial \tau _{\eta}}{\partial z}$ 
and use them to solve equation~\ref{eqn:or2}.
After some tedious algebra, I obtain
\beq
\tau_{\eta_{2}}(x,y,z) = \frac{3 v_t^6 \left(x^2+y^2\right)^3
   \sqrt{\frac{x^2+y^2}{v^2}+\frac{z^2}{v_t^2}} \left(4 v^2 z^2+v_t^2
   \left(x^2+y^2\right)\right)}{2 \left(v^2 z^2+v_t^2
   \left(x^2+y^2\right)\right)^4}.
\label{eqn:homo3d}
\eeq

The application of Pade approximation on the expansion in $\eta$, by finding a first order polynomial representation
 in the denominator, 
yields a TI equation that is accurate for large $\eta$ \cite[]{etascan},
as well as small tilt, given by
\beqa
\tau(x,y,z) &\approx& \frac{1}{v^2 v_t^2
   \sqrt{\frac{x^2+y^2}{v^2}+\frac{z^2}{v_t^2}} \left(2 v^4 z^4+4 v^2
   v_t^2 z^2 (3 \eta +1) \left(x^2+y^2\right)+v_t^4 (3 \eta +2)
   \left(x^2+y^2\right)^2\right)} \nonumber \\ & & \left(2 v^6 z^5 (z- \sin\theta \left( x \cos\phi+y \sin
  \phi \right))-2 v_t^2 v^4 z^3 \left(\left((6 \eta +2)
   \left(x^2+y^2\right)-z^2\right)  \right. \right. \nonumber \\ & & \left. \left. (\sin\theta \left( x \cos\phi+y \sin
  \phi \right)) -3 z (2 \eta +1)
   \left(x^2+y^2\right)\right)-v_t^4 v^2 z \left(x^2+y^2\right) \right. \nonumber \\ & & \left.
      \left(\left((3 \eta +2) \left(x^2+y^2\right)-4 z^2 (3 \eta +1) \right)
    \sin\theta \left( x \cos\phi+y \sin
  \phi \right)-z (13
   \eta +6) \left(x^2+y^2\right)\right)  \right. \nonumber \\ & & \left.  +v_t^6 \left(x^2+y^2\right)^2
   \left(x^2 (\eta +2)+z (3 \eta +2) \sin\theta \left( x \cos\phi+y \sin
  \phi \right)+y^2 (\eta +2)\right)\right).
  \label{eqn:homo3df}
  \eeqa


