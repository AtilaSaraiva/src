\section{Angle decomposition}

In downward continuation methods, theoretical analysis of angle
gathers can be reduced to analyzing the geometry of reflection in the
simple case of a dipping reflector in a locally homogeneous medium
\cite[]{SavaFomel.segab2.2005}. The behavior of plane waves in the
vicinity of the reflection point is sufficient for deriving
relationships for local reflection traveltime derivatives
\cite[]{STI00-00-03630363}.  The geometry of the reflection ray paths
is depicted in \rfg{Reflection}.

\plot{rayparameters}{width=0.3\textwidth}{A schematic plot
  depicting the relation between source and receiver ray-parameter
  vectors ($\pps$ and $\ppr$) and that of the space-lag and position
  ($\ppl$ and $\ppx$). The angle $\theta$ 
  corresponds to the phase angle direction of the plane wave.}

Using the standard notations for the source and receiver coordinates:
$\ss=\xx+\hh$ and $\rr=\xx-\hh$, the traveltime from a source to a
receiver is a function of all spatial coordinates of the seismic
experiment $t=t(\xx,\hh)$.  Differentiating $t$ with respect to all
components of the vectors $\xx$ and $\hh$, and using the standard
notations \geouline{to represent slownesses} ${\pp_{\alpha}}=\nabla_{\alpha} t$, where $\alpha=(\xx,\hh,\ss,\rr)$, we can write:
%
\bea 
\ppx &=& \ppr + \pps \label{eqn:quadd}  \;, \\
\ppl &=& \ppr - \pps \label{eqn:quaddd} \;.
\eea
%
By analyzing the geometric relations of various vectors at an image
point (\rfgs{rayparameters}), we can write the following trigonometric
expressions:
%
\bea 
|\ppl|^2 &=& |\pps|^2 +|\ppr|^2 - 2 |\pps| |\ppr| \cos(2 \theta) \label{eqn:quaddd1} \;, \\
|\ppx|^2 &=& |\pps|^2 +|\ppr|^2 + 2 |\pps| |\ppr| \cos(2 \theta) \label{eqn:quaddd2} \;.
\eea

Defining $\kkx$ and $\kkl$ as the position and lag (or offset)
wavenumber vectors, we can replace $\pp=\kk/\ww$. Using the
trigonometric \geouline{identities} \geosout{identity}
%
\bea
1-\cos(2\theta) = 2\sin^2(\theta) \label{eqn:quaddd3} \;, \\
1+\cos(2\theta) = 2\cos^2(\theta) \label{eqn:quaddd4} \;,
\eea
%
and assuming $|\pps|=|\ppr|=s(\theta)$, where $s(\theta)=1/v_p(\theta)$
is the phase slowness as a function of phase angle at an image
location, we obtain the following relations:
%
\bea
|\kkl|^2 &=& (2 \ww s(\theta) \sin(\theta))^2 \label{eqn:quaddd34} \;, \\
|\kkx|^2 &=& (2 \ww s(\theta) \cos(\theta))^2 \label{eqn:quaddd35} \;, \\
\,\,\,\,\, \; \;  \;\kkl . \kkx = 0 \label{eqn:quaddd355} \;.
\eea

We can eliminate from \reqs{quaddd34}-~\ren{quaddd35} the
dependence on \geouline{the} depth \geouline{axis} and obtain an angle decomposition formulation
prior to imaging. \geouline{Thus, if} \geosout{If} we eliminate $\kz$ and $\klz$, \geosout{as well} we obtain
the expression:
%
\bea \label{eqn:quaddd366}
(\kx^2 + \ky^2) (2 \ww s(\theta) \sin{\theta})^2 +
(\klx^2+\kly^2) (2 \ww s(\theta) \cos{\theta})^2 = \nonumber \\
(\kx\kly-\ky\klx)^2 + 
(2 \ww s(\theta) \sin{\theta})^2
(2 \ww s(\theta) \cos{\theta})^2 \;.
\eea
%
The quadratic \req{quaddd366} can be used to map data from space-lag
gathers ($\klx$, $\kly$) into angle coordinates $\theta$, prior to
imaging. For \geosou2{2-D} \geoulin2{2D} data, \req{quaddd366} takes the simpler form
%
\bea \label{eqn:quaddd36}
\kx^2  (2 \ww s(\theta) \sin{\theta})^2 +
\klx^2 (2 \ww s(\theta) \cos{\theta})^2 = \nonumber \\
(2 \ww s(\theta) \sin{\theta})^2 
(2 \ww s(\theta) \cos{\theta})^2,
\eea
%
which can be solved for an explicit mapping of $\klx$ to $\theta$.

Note that the angle decomposition formula \ren{quaddd36} reduces to a
form simpler than that shown by \cite{alkhalifah:2009} for VTI
media. This angle decomposition is particularly useful in imaging via
downward continuation, as discussed next.

