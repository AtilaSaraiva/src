\section{Domain of applicability}

The reflector dip TI tilt \geouline{constraint} \geosout{constrain} introduced here for imaging simplification purposes is not applicable everywhere. In fact, setting such a \geouline{constraint} \geosout{constrain} inherently suggests smooth interfaces as it is impossible to impose such a \geouline{constraint} \geosout{constrain} on a diffractor. The smooth interface is also required by the plane wave assumption used in the angle gather development. Thus, we are suggesting DTI as a model development tool in which this suggested assumption is typically used in areas like the Gulf of Mexico. Thus, the DTI model must be extracted from reflections that adhere to this \geouline{constraint} \geosout{constrain}, which do not include salt flanks. This is convenient in building the model around the salt and even subsalt. While the top-of-salt reflections do not adhere to this \geouline{constraint} \geosout{constrain}, the bottom reflections do as isotropy is a special case of DTI with the anisotropy parameters $\eta=\delta=0$. In addition, subsalt reflections also satisfy this \geouline{constraint} \geosout{constrain} whether such reflections are within isotropic media or an assumed DTI condition. It does not matter that the rays may have traveled through media that is VTI or general TTI, what matters is the behavior at the reflection point for applications like DTI imaging or angle gather analysis.

