\begin{abstract}
  The computational tools for imaging in transversely isotropic media
  with tilted axes of symmetry (TTI) are complex and in most cases do
  not have an explicit \geouline{closed-form} \geosout{closed form}
  representation. \geouline{As discussed in this paper,} developing
  \geosout{these} \geouline{such} tools for a TTI medium with tilt
  constrained to be normal to the reflector dip (DTI) reduces their
  complexity and allows for closed-form representations. \geosout{In
    fact,} \geouline{We show that, for the homogeneous case}
  zero-offset migration in such a medium \geosout{for the homogeneous
    case is represented by} \geouline{can be performed using} an
  isotropic operator scaled by the velocity of the medium in the tilt
  direction. \geouline{We also show that,} for the nonzero-offset
  case, the reflection angle is always equal to the incidence angle,
  and thus, the velocities for the source and receiver waves at the
  reflection point are equal and explicitly dependent on the
  reflection angle. This fact allows us to develop explicit
  representations for angle decomposition as well as moveout formulas
  for analysis of extended images obtained by wave-equation
  migration. Although setting the tilt normal to the reflector dip may
  not be valid everywhere (i.e., salt flanks), it can be used in the
  process of velocity model building where such constrains are useful
  and typically used.
\end{abstract}
