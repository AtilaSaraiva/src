\section{Conclusions}

Constraining the tilt of a transversely isotropic medium normal to the reflector dip (DTI) allows for explicit formulations of plane waves around the scattering point. These formulations form the basis for angle decomposition or the moveout analysis in the extended image condition domain. As a result, DTI is a convenient model for anisotropy parameter estimation in media in which such models are applicable. This model also allows us to use the general TTI assumption in a simplified form that better fits the information embedded in the recorded data.


\section{Acknowledgments}
We are grateful to KAUST and to the sponsors of the Center for Wave Phenomena at Colorado School of Mines for their support. We thank the associate editor and the reviewers for their critical review of the letter.
