\published{Applied Geophysics, 12, 55-63 (March 2015)}

\title{Seismic dip estimation based on the two-dimensional Hilbert transform and its application in random noise attenuation}
%\keywords{Two-dimensional Hilbert transform, random noise attenuation, structure protection, nonstationary polynomial fitting, local seismic dip }


\renewcommand{\thefootnote}{\fnsymbol{footnote}}

\author{Cai Liu\/\footnotemark[1], Changle Chen\/\footnotemark[1], Dian Wang\/\footnotemark[3]\/\footnotemark[1], Yang Liu\/\footnotemark[1], Shiyu Wang\/\footnotemark[1], and Liang Zhang\/\footnotemark[2]}

\footnotetext[1]{\emph{College of Geo-exploration Science and Technology, Jilin University, Changchun 130026, China.}}
\footnotetext[2]{\emph{Qian An Oil Factory, Jilin Oilfield, CNPC, Songyuan 138000, China.}}
\footnotetext[3]{\emph{Corresponding Author: Dian Wang (Email: dianwang@jlu.edu.cn)}}

\lefthead{Liu et al.}
\righthead{Dip estimation using Riesz transform}
\maketitle


\begin{abstract}
In seismic data processing, random noise seriously affects the seismic
data quality and subsequently the interpretation.  This study aims to
increase the signal-to-noise ratio by suppressing random noise and
improve the accuracy of seismic data interpretation without losing
useful information. Hence, we propose a structure-oriented polynomial
fitting filter. At the core of structure-oriented filtering is the
characterization of the structural trend and the realization of
nonstationary filtering. First, we analyze the relation of the
frequency response between two-dimensional (2D) derivatives and the 2D
Hilbert transform (Riesz transform). Then, we derive the noniterative
seismic local dip operator using the 2D Hilbert transform to obtain
the structural trend. Second, we select polynomial fitting as the
nonstationary filtering method and expand the application range of the
nonstationary polynomial fitting. Finally, we apply variableamplitude
polynomial fitting along the direction of the dip to improve the
adaptive structureoriented filtering. Model and field seismic data
show that the proposed method suppresses the seismic noise while
protecting structural information.
\end{abstract}

\section{introduction}

Random noise, which refers to any unwanted features in data, commonly
contaminates seismic data. Random noise sources in seismic exploration
are roughly divided into three categories. First, there are external
disturbances such as wind and human activities. Second, there is
electronic instrument noise. Third, there is the irregular
interference owing to seismic explosions.  Random noise attenuation is
a significant step in seismic data processing. In particular, the
extent of noise suppression in poststack data directly affects the
accuracy of subsequent processing and interpretation.  Presently,
several different random noise attenuation methods are
available. \cite{Liu06} presented a 2D multilevel median filter for
random noise attenuation, whereas \cite{liu091} used a 1D time-varying
window median filter. \cite{Bekara09} used the empirical mode
decomposition (EMD) method and proposed a filtering technique for
random noise attenuation in seismic data. \cite{Liu09} proposed a
high-order seislet transform for random noise attenuation. \cite{Li12}
applied morphological component analysis to suppress random noise and
\cite{Liu12} proposed a novel method of random noise attenuation based
on local frequency-domain singular value decomposition
(SVD). \cite{Maraschini13} assessed the effect of nonlocal means
random noise attenuator on coherency. \cite{Li13} used time-frequency
peak filtering to suppress strong noise in seismic data. \cite{Liu13}
used f-x regularized nonstationary autoregression to suppress random
noise in 3D seismic data. The abovementioned random noise attenuation
methods are limited by their lack of protection of structural
information. For example, improper filtering may blur small faults,
which may also make the displacement of larger fault continuous and
consequently make layers appear continuous instead of
faulted. Obviously, this hinders fault interpretation, and makes
denoising and protecting structural information
important. \cite{Fehmers03} applied structureoriented filtering to
fast structural interpretation. \cite{Hoeber06} applied nonlinear
filters, such as median, trimmed mean, and adaptive Gaussian, over
planar surfaces parallel to the structural dip. \cite{Fomel06}
suggested the method of plane-wave construction by using model
reparameterization. \cite{Liu10} applied nonlinear structure-enhancing
filtering by using plane-wave prediction to preserve structural
information. \cite{Liu11} proposed a poststack random noise
attenuation method by using weighted median filter based on local
correlation and tried to balance the protection of fault information
and noise attenuation.

Structure-oriented filtering includes structure prediction and
filtering. Seismic dip is at the core of structure prediction; for, we
can use seismic dip to determine structural trends and achieve
structure protection. \cite{Ottolini82} used local slant stack to
formulate a local seismic dip estimation method. \cite{Fomel02}
proposed a seismic dip estimation method based on the plane-wave
destruction (PWD) filter. \cite{Schleicher09} compared different
methods of local dip computations. The selection of filtering methods
in structure-oriented filters is critical and polynomial fitting has
been successfully applied to seismic data denoising.
\cite{Lu09} used edge-preserving polynomial fitting
to suppress random seismic noise. This method achieves
better results when the trajectories of seismic events are
linear or the amplitudes along the trajectories are not
constant. \cite{Liu111} proposed a novel seismic noise
attenuation method by using nonstationary polynomial
fitting \cite[]{Fomel09} and shaping regularization \cite[]{Fomel07}
for constraining the smoothness of the polynomial coefficients.

In this paper, we discuss the two-dimensional (2D) Hilbert transform
and use it to derive the formula for the dip in the plane wave,
construct a stable algorithm for estimating the dip, and improve the
computational efficiency of Fomel's method \cite[]{Fomel02} without
minimizing the precision of the dip estimation.  Finally, we use
synthetic model and field seismic data to demonstrate the
applicability of the proposed method.

\section{theory}
The extraction of structural information and the selection of
effective filtering methods are critical to structure-oriented
filters. Because of the time-space relation in seismic data,
structural information must satisfy kinematics and kinetics
equations. The dip of seismic events reveals structural features. This
study is the first to discuss a calculation method for the local
seismic dip.

\subsection{Noniterative local dip calculation}
Following the local plane-wave equation \cite[]{Fomel02}
\begin{equation}
  \label{eq:1}
  \frac{\partial{P(x,t})}{\partial{x}}+{\sigma}(x,t)
       \frac{\partial{P(x,t})}{\partial{t}}=0,
\end{equation}
we define the local dip of seismic data
\begin{equation}
  \label{eq:2}
  {\sigma}(x,t)=-\frac{\partial{P(x,t})}{\partial{x}}{\slash}
    \frac{\partial{P(x,t})}{\partial{t}},
\end{equation}
where $P(x,t)$ is the seismic wave field and ${\sigma}(x, t)$ is the
local seismic dip as a function of time $t$ and distance $x$.
However, in actual computations, because the local dip is used to
determine the direction of a seismic event, we ignore the dimensions
and sampling interval; thus, ${\sigma}$ only depends on the sampling
data and the local dimensionless dip is defined as
\begin{equation}
  \label{eq:3}
  {\sigma}=-(\frac{\partial{P(x,t})}{\partial{x}}{\slash}
    \frac{\partial{P(x,t})}{\partial{t}})\cdot
      \frac{{\Delta}x}{{\Delta}t}=
     -\frac{\partial{P}}{\partial{x}}{\slash}\frac{\partial{P}}{\partial{y}},
\end{equation}
where $\partial{P}{\slash}\partial{x}$ and
$\partial{P}{\slash}\partial{y}$ are the partial derivatives of the
seismic wave field in the $x-$ and $y-$direction, respectively, and
${\Delta}x$ and ${\Delta}t$ are the respective sampling intervals in
the $x-$ and $y-$direction.

Using equation~\ref{eq:3}, we compute the local dip by using the
specific values of the space- and time-directional derivatives. Hence,
we first discuss the derivative operator.

The ideal differentiator frequency response is
\begin{equation}
  \label{eq:4}
  F_{IDD}({\omega})=i{\omega},-{\pi}\leq\omega\leq\pi.
\end{equation}
The ideal differentiator frequency response is multiplied by a
frequency-dependent linear function in the frequency domain. The
direct calculation of the derivative of the signal in the time domain
enhances the high-frequency random noise and reduces the dip
accuracy. Thus, we analyze the frequency response of the derivative
operator and the frequency response of the Hilbert transform. We
derive the Hilbert transform (Appendix A) and the approximate partial
derivative by using the finite impulse response (FIR) filter
\cite[]{Pei01}.  We use a 2D Hilbert transform to approximate the
partial derivatives of the wave field, which reduces the side effect
of strong high-frequency random noise owing to the derivative
algorithm.

The redefined noniterative local dip of the seismic data is
\begin{equation}
  \label{eq:5}
  {\sigma} =-(\frac{\partial{P}}{\partial{x}}{\slash}
       \frac{\partial{P}}{\partial{y}})=
       -\frac{{FFT^{-1}}{[\tilde{P}(x)]}}{{FFT^{-1}}{[\tilde{P}(y)]}}
       =-\frac{FFT^{-1}[\frac{1}{\sqrt{c_{x}}}\tilde{P}(x)]}{FFT^{-1}
         [\frac{1}{\sqrt{c_{y}}}\tilde{P}(y)]}{\approx}-\frac{FFT^{-1}
         [H_{HT}(x)]}{FFT^{-1}[H_{HT}(y)]}{\approx}-\frac{H_{HTx}}{H_{HTy}},
\end{equation}
where $\tilde{P}(x)$ is the frequency response function of the partial
derivative in the $x-$direction and $\tilde{P}(y)$ is the frequency
response function of the partial derivative in the $y-$direction. The
dimensions are ignored in the derivation and $c$ does not depend on
the time and space sampling intervals; thus, we take
$c_x=c_y$. $H_{HT}(x)$ is the frequency response function of the
Hilbert transform in the $x-$direction and $H_{HT}(y)$ is the
frequency response function of the Hilbert transform in the
$y-$direction. $H_{HTx}$ and $H_{HTy}$ are the components of the 2D
Hilbert transform in the $x-$ and $y-$direction, respectively. Using
equation~\ref{eq:5}, we calculate the local seismic dip attribute by
using the 2D Hilbert transform instead of the derivative operation.
Because division is required in equation~\ref{eq:5} and the
denominator might become zero, we add the nonzero constant
$\varepsilon$ in the denominator
\begin{equation}
  \label{eq:6}
  \sigma{\approx}-\frac{H_{HTx}}{H_{HTy}+\varepsilon}.
\end{equation}
\cite{Fomel07} proposed the shaping regularization
for imposing regularization constraints in estimation problems and
defined the local seismic attributes. In this paper, we use the same
method to constrain the division and smooth the local dip by using the
Gaussian smooth operator as the regularization operator.

\inputdir{sigmoid}
 \multiplot{4}{noise,nrt,nrx,rizdip}{width=0.47\columnwidth}{Local
              seismic dip based on the 2D Hilbert transform. Synthetic
              seismic data (a), time component of the 2D Hilbert
              transform (b), space component of the 2D Hilbert
              transform (c), and local seismic dip (d).}

To show the validity of the proposed dip calculation method, we
construct a synthetic seismic model and add white Gaussian random
noise, as shown in Figure~\ref{fig:noise,nrt,nrx,rizdip}a.  The
components of the 2D Hilbert transform in the $x-$ and $y-$direction
are shown in Figures~\ref{fig:noise,nrt,nrx,rizdip}b and
\ref{fig:noise,nrt,nrx,rizdip}c, respectively. We obtain the dip of
the seismic data by using the ratio of the two components and
calculate the smoothing constraints, as shown in
Figure~\ref{fig:noise,nrt,nrx,rizdip}d. We see that the calculation
results can accurately reflect the dip value of the original data at
different locations, such as the tilted layers at the top the
underlying strata with the sinusoidal fluctuations, and the fault
location. Using the 2D Hilbert transform and shaping regularization,
we obtain the smooth local dip attribute.

Another effective calculation method of the time-varying and
space-variant seismic local dip is based on the plan-wave destruction
(PWD) filter proposed by \cite{Fomel02}. The PWD filter realizes the
plane-wave propagation across different traces, while the total energy
of the propagating wave stays invariant, by using an all-pass digital
filter in the time domain and a Taylor expansion of the all-pass
filter frequency. We obtain the relation of the PWD and
space-time-varying local seismic dip by using the Gauss-Newton
algorithm to solve the nonlinear problem of local seismic dip.  This
method can be essentially understood as solving an implicit
finite-difference scheme for the local planewave equation. The
disadvantage of the PWD-based calculation method is its slow
computation speed, which is especially worse at higher order
conditions.  The computational cost of the proposed method is
proportional to $2N_{x}\times N_{t}$, where $N_{x}\times N_{t}$ is the
data size, whereas the computational efficiency of the PWD-based dip
estimation method is proportional to $N_{iter}\times N_{x} \times
N_{t}$, where $N_{iter}$ is the number of iterations. Hence, to
achieve similar accuracy, the dip estimation method based on the 2D
Hilbert transform requires a smaller number of iterations than the
PWD-based method.  

The dip of seismic events controls the trend of the
constructed seismic model; thus, next, we need to apply filtering
along the trend. The selected filtering method must simultaneously
suppress the seismic noise and protect structural information.

\subsection{Nonstationary polynomial fitting}
Traditional stationary regression is used to estimate the coefficients
$a_{i}, i=1,2,\dots,N$ by minimizing the prediction error between a
``master'' signal s($\mathbf{x}$) (where $\mathbf{x}$ represents the
coordinates of a multidimensional space) and a collection of slave
signals $L_{i}(\mathbf{x}), i = 1, 2,\dots ,N$
\cite[]{Fomel09}
\begin{equation}
  \label{eq:7}
  E(\mathbf{x})=s(\mathbf{x})-\sum_{i=1}^{N}a_{i}L_{i}(\mathbf{x}).
\end{equation}
When $\mathbf{x}$ is 1D and $N= 2$, $L_{1}(\mathbf{x})=1$ and
$L_{2}(\mathbf{x})=x$ , the problem of minimizing $E(\mathbf{x})$
amounts to fitting a straight line $a_{1}+a_{1}x$ to the master
signal.  Nonstationary regression is similar to equation~\ref{eq:7}
but allows the coefficients $a_{i}(\mathbf{x})$ to vary with
$\mathbf{x}$, and the error
\cite[]{Fomel09}
\begin{equation}
  \label{eq:8}
  E(\mathbf{x})=s(\mathbf{x})-\sum_{i=1}^{N}a_{i}(\mathbf{x})L_{i}(\mathbf{x})
\end{equation}
is minimized to solve for the multinomial coefficients $a_{i}(\mathbf{x})$. The
minimization becomes an ill-posed problem because $a_{i}(\mathbf{x})$ rely on
the independent variables $\mathbf{x}$. To solve the ill-posed problem, we
constrain the coefficients $a_{i}(\mathbf{x})$. Tikhonov's regularization
\cite[]{Tikhonov63} is a classical regularization method that amounts
to the minimization of the following functional \cite[]{Fomel09}
\begin{equation}
  \label{eq:9}
  F(a)=\|E(\mathbf{x})\|^{2}+
    \varepsilon^{2}\sum_{i=1}^{N}\|\mathbf{D}[a_{i}(\mathbf{x})]\|^2 ,
\end{equation}
where $\mathbf{D}$ is the regularization operator and $\varepsilon$ is
a scalar regularization parameter. When $\mathbf{D}$ is a linear
operator, the least-squares estimation reduces to linear inversion
\cite[]{Fomel09}
\begin{equation}
  \label{eq:10}
  \mathbf{a}=\mathbf{A}^{-1}\mathbf{d} ,
\end{equation}
where
\begin{equation}
\begin{split}
  \nonumber
  \mathbf{a} & =[a_1(x)a_2(x)\cdots a_N(x)]^T ,\\
  \nonumber
  \mathbf{d} & =[L_1(x)s(x)L_2(x)s(x)\cdots L_N(x)s(x)]^T,
\end{split}
\end{equation}
and the elements of matrix $\mathbf{A}$ are
\begin{equation}
  \nonumber 
  A_{ij}({\mathbf{x}})=L_i({\mathbf{x}})L_j({\mathbf{x}})+\varepsilon^2
         \delta_{ij}\mathbf{D}^T\mathbf{D} \;.
\end{equation}

\inputdir{linefit}
\plot{compare}{width=0.9\columnwidth}{Least-squares linear fitting 
              compared with nonstationary polynomial fitting.}

Next, we use a simple signal to simulate the variation of the
amplitude of a nonstationary event with random noise (dashed line in
Figure~\ref{fig:compare}). In Figure~\ref{fig:compare}, the dot dashed
line denotes the results of the least-squares linear fitting and the
solid line denotes the results of the nonstationary polynomial
fitting. We compare the least-squares linear fitting and nonstationary
polynomial fitting results, and we find that the nonstationary
polynomial fitting models the curve variations more accurately for
events with variable amplitude, particularly for $40<x<60$.

\section{Synthetic data tests}

\inputdir{sigmdenoi}
\multiplot{5}{elpf,delpf,sdip,median,dmedian}{width=0.47\columnwidth}{Analysis 
             of results using different structure-oriented
             filtering. Nonstationary polynomial fitting (a),
             difference profile of nonstationary polynomial fitting
             (b), local PWD-based dip (c), median filter (d), and
             difference profile of median filter (e).}

We construct a new structure-oriented filtering method based on the 2D
Hilbert transform with nonstationary polynomial fitting and apply it
to synthetic data (Figure~\ref{fig:noise,nrt,nrx,rizdip}a).  The local
seismic dip (Figure~\ref{fig:noise,nrt,nrx,rizdip}d) controls the
trend of the event, and we apply nonstationary polynomial fitting
along the direction of the dip for fast structural interpretation
using structure-oriented filtering. We achieve continuous model
protection in the direction of dip, and noise attenuation and fault
protection because of the use of nonstationary polynomial
fitting. Nine sampling points are used in the structure-oriented
filtering and five sampling points in the nonstationary polynomial
fitting. The filtering results are shown in
Figure~\ref{fig:elpf,delpf,sdip,median,dmedian}a and the difference
profile is shown in Figure~\ref{fig:elpf,delpf,sdip,median,dmedian}b.
Figure~\ref{fig:elpf,delpf,sdip,median,dmedian}a shows that the upper
tilted layer, the lower sinusoidal layer, and the fault information
are preserved, while the noise is clearly suppressed. Random noise
constitutes most of the difference profile without any tilted layer
and fault information left because the local seismic dip cannot
reflect the trend of the layers and owing to the attenuation of the
limited effective information. To compare the proposed method with the
PWD-based local dip estimation method
(Figure~\ref{fig:elpf,delpf,sdip,median,dmedian}c) with similar dip
accuracy (Figure~\ref{fig:noise,nrt,nrx,rizdip}d), we choose the
median filter and show the results in
Figure~\ref{fig:elpf,delpf,sdip,median,dmedian}d and the difference
profile in Figure~\ref{fig:elpf,delpf,sdip,median,dmedian}e. We
compare the two profiles after the application of the median
filter. We find more useful structural information than the method we
proposed.  That means the method we proposed has better effect.

\section{Field data tests}
\inputdir{fieldata}
\multiplot{4}{slice,dipr,elpfr,delpfr}{width=0.47\columnwidth}{Comparison 
              of processing results. Field data (a), Local dip (b),
              After filtering (c), Difference profile (d).}

For field data processing, we chose the 2D profile of 3D poststack
data \cite[]{Liu13}. The shallow structures are simple planar layers
and the deep structures are complex curved layers. First, we use the
proposed method, which is based on the 2D Hilbert transform, to
compute the corresponding local seismic dip attribute
(Figure~\ref{fig:slice,dipr,elpfr,delpfr}b). From
Figure~\ref{fig:slice,dipr,elpfr,delpfr}b, we see that the dip changes
smoothly and steadily in the midshallow layer corresponding to the
continuous event in the profile, whereas the variation of the dip in
the deep layer is relatively larger, which characterizes the bending
event in the mid-deep layer.  

The trend of the local seismic events
can be determined by using the dip attribute; thus, we select the
filtering window, which is determined by the dip, and use
nonstationary polynomial fitting for filter processing. The window
size of the structure-oriented data consists of 11 sampling points and
the window size of the nonstationary polynomial fitting comprises
seven sampling points. 

Figure~\ref{fig:slice,dipr,elpfr,delpfr}c
shows the denoising results. We see that the random noise in the raw
profile is suppressed, the whole section is clearer, and the
continuity of the plane event (0.1s-0.3s) in the shallow layer and the
curved event (below 0.3s) in the deep layer has improved.  The
difference profile (Figure~\ref{fig:slice,dipr,elpfr,delpfr}d) shows
that the removed noise is mainly irrelevant random noise and the
information is well preserved.

\section{Conclusions}

We propose a seismic dip estimate method based on the 2D Hilbert
transform. We compute the stable dip by using the noniterative
approximation relation within the middle frequency band, and improve
the computational efficiency relative to the iterative dip algorithm
based on the PWD filter. We combine the proposed method with
nonstationary polynomial fitting to suppress the seismic random noise
using the computed local seismic dip. We predict the seismic structure
trend using the structure-oriented window based on the seismic dip,
while balancing the random noise attenuation and signal preservation
via filtering with the nonstationary polynomial fitting. The proposed
method suppresses the seismic noise and strongly depend on the of dip
trend prediction. The accuracy of computed dip is directly affected by
filtering. The method is not applicable at strong noise conditions. We
use synthetic model and field data processing, to demonstrate the
applicability of the proposed method.

\appendix
\section{Appendix A: Hilbert transform derivation for approximating the partial derivative}
\subsection{Derivation of the FIR transfer function for the frequency response of digital differentiators}

First, to characterize the FIR for signal differentiators, we
transform the Leibniz series
$\displaystyle\frac{2{\rm{arcsin}}x}{\sqrt{1-x^2}}$ to power series
\cite[]{Lehmer85}
\begin{equation}
  \label{eq:1a}
  \frac{2{\rm{arcsin}}x}{\sqrt{1-x^2}}=2x
       \left[1+\sum_{m=1}^{\infty}\frac{(2{\rm{m}})!!}
        {(2{\rm{m}}+1)!!}x^{2m}\right].
\end{equation}
We substitute sin$\displaystyle(\frac{\omega}{2})$ for $x$, and after
rearrangement and truncation of the first $M$ terms, we obtain
\begin{equation}
\begin{split}
  \label{eq:2a}
  & \frac{\omega}{\sqrt{1-{\rm{sin}}^{2}\displaystyle\frac{\omega}{2}}}\\
  & =2{\rm{sin}}\frac{\omega}{2}\left[1+\sum_{m=1}^{M}\frac{(2{\rm{m}})!!}
   {(2{\rm{m}}+1)!!}
 \left(\frac{1-{\rm{cos}}\omega}{2}\right)^{m}+
    o((\frac{1-{\rm{cos}}\omega}{2})^{M+1})\right]
\end{split}
\end{equation}
and after manipulation
\begin{equation}
\begin{split}
  \label{eq:3a}
  \omega & =2{\rm{sin}}\frac{\omega}{2}{\rm{cos}}\frac{\omega}
    {2}\left[1+\sum_{m=1}^{M}\frac{(2{\rm{m}})!!}{(2{\rm{m}}+1)!!}
 \left(\frac{1-{\rm{cos}}\omega}{2}\right)^{m}+
    o((\frac{1-{\rm{cos}}\omega}{2})^{M+1})\right]\\
         & ={\rm{sin}}\omega\left[1+\sum_{m=1}^{M}
       \frac{(2{\rm{m}})!!}{(2{\rm{m}}+1)!!}
\left(\frac{1-{\rm{cos}}\omega}{2}\right)^{m}+
       o((\frac{1-{\rm{cos}}\omega}{2})^{M+1})\right].
\end{split}
\end{equation}
We ignore the higher order terms and we obtain the $(2M+2)$th-order
causal transfer function of the derivative operator as
\begin{equation}
  \label{eq:4a}
  \hat{F}_{DD}(z)\approx-\frac{1-z^{-2}}{2}\left\{z^{-M}+
     \sum_{m=1}^{M}\frac{(2{\rm{m}})!!}{(2{\rm{m}}+
       1)!!}{\cdot}z^{-(M-m)}\left[-\frac{(1-z^{-1})^2}{4}\right]^m\right\}.
\end{equation}

\subsection{Derivation of the FIR transfer function for the frequency response of the Hilbert transform}
The ideal frequency response of the Hilbert transform is expressed as
\begin{equation}
  \label{eq:5a}
  H_{IHT}(\omega)=-i\,{\rm{sgn}}\,\omega=
-i\frac{\omega}{\left|\omega\right|}=\left\{ \begin{array}{ll}
i, & \textrm{$-\pi<\omega<0$}\\
-i, & \textrm{$0<\omega<\pi$}
\end{array} \right.
.
\end{equation}
From equations~\ref{eq:4} and \ref{eq:5a}, we obtain the difference as
$–1/\left|\omega\right|$.  For
\begin{equation}
  \label{eq:6a}
  {\rm{sgn}}\,x=\frac{x}{\sqrt{x^2}}=xf(x^2),x\neq0
\end{equation}
and $f(u)=\displaystyle\frac{1}{\sqrt{u}},u>0$, the Taylor series 
of $f(u)$ at center c is expressed
\begin{equation}
  \label{eq:7a}
  f(u)=\frac{1}{\sqrt{c}}\left[1+\sum_{m=1}^{\infty}
  \frac{(2{\rm{m}}-1)!!}{(2{\rm{m}})!!}\left(1-\frac{u}{c}\right)^m\right],
\end{equation}
where $(2{\rm{m}}-1)!!=1 \cdot 3 \cdot 5 \dots (2 {\rm{m}}- 1)$, $(2{\rm{m}})!!=1\cdot 3\cdot 5\dots(2{\rm{m}})$.
Consequently, the signum function sgn$x$ is expressed
\begin{equation}
  \label{eq:8a}
  {\rm{sgn}}\,x=\frac{x}{\sqrt{c}}\left[1+\sum_{m=1}^{\infty}
  \frac{(2{\rm{m}}-1)!!}{(2{\rm{m}})!!}\left(1-\frac{x^2}{c}\right)^m\right].
\end{equation}
We substitute sin$\omega$ for $x$, based on
sgn$\omega$=sgn(sin$\omega$) for $–\pi<\omega<\pi$, truncate the
series at the first M terms, and obtain the sinusoidal power series of
the signum function as
\begin{equation}
\label{eq:9a}
\rm{sgn}\,\omega=\frac{\rm{sin}\omega}{\sqrt{c}} \left[1+
   \sum_{m=1}^{M}\frac{(2m-1)!!}{(2m)!!}\left(1-\frac{\rm{sin}^2
   \omega}{c}\right)^m+\circ((1-\frac{\rm{sin^2\omega}}{c})^{M+1})\right]
\end{equation}

The series in \ref{eq:9a} converges for $\displaystyle
-1<1-\frac{{\rm{sin}}\omega}{c}<1$; that is, $c$ has to be larger than
1/2. On the other hand, the expansion center $c$ in the $x$-domain is
associated to the frequency center in the $\omega$-domain via the
relation $c={\rm{sin}}^2\omega_c$. Therefore, $c={\rm{sin}}^2\omega_c$
must be less than or equal to 1. Accordingly, $c$ is constrained by
$1/2<c\;{\leq}\;1$ and the corresponding $\omega_c$ is within the
range $[\pi/4,\pi/2]$. Clearly, the ideal frequency response is well
approximated within the middle frequency band. Multiplying \ref{eq:9a}
by $- i$ and substituting $\displaystyle\frac{z-z^{-1}}{2i}$ for
sin$\omega$, the transfer function for the zero phase FIR of the
Hilbert transform is expressed as
\begin{equation}
\label{eq:10a}
H_{HT}\rm{(z,c)}\approx-\frac{z-z^{-1}}{2\sqrt{c}}
    \left\{1+\sum_{m=1}^{M}\frac{(2m-1)!!}{(2m)!!}\left[1+\frac{1}{c} 
    \left( \frac{z-z^{-1}}{2} \right)^2\right]^m \right\}
\end{equation}
To obtain the causal transfer function, $H_{HT}(z,c)$ is
multiplied by $z^{-2M-1}$ and the resultant transfer function of
the FIR Hilbert transform of the (2$M$+2)th-order is
\begin{equation}
\label{eq:11a}
\hat{H}_{HT}\rm{(z,c)}\approx-\frac{1-z^{-2}}{2\sqrt{c}}
   \left\{z^{-2M}+\sum_{m=1}^{M}\frac{(2m-1)!!}{(2m)!!}z^{-2(M-m)}
    \left[z^{-2}+\frac{1}{c} \left( \frac{1-z^{-2}}{2} 
     \right)^2\right]^m \right\}
\end{equation}
For $M$=0, the transfer functions of equations~\ref{eq:4a}
and \ref{eq:11a} are approximated as
\begin{equation}
\label{eq:12a}
\hat{H}_{HT}\rm{(z,c)}\approx-\frac{1-z^{-2}}{2\sqrt{c}}
\end{equation}
\begin{equation}
\label{eq:13a}
\hat{F}_{DD}\rm{(z)}\approx-\frac{1-z^{-2}}{2}
\end{equation}
We compare equations~\ref{eq:12a} and \ref{eq:13a}, and we conclude
that these two transfer functions in middle frequency band of the
frequency domain differ by the constant coefficient
$\displaystyle\frac{1}{\sqrt{c}}$.

\bibliographystyle{seg}
\bibliography{dipest}
