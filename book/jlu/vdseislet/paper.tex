\published{Geophysics, 80, no. 6, WD117-WD128, (2015)}
 
\title{Signal and noise separation in prestack seismic data using velocity-dependent seislet transform}

\renewcommand{\thefootnote}{\fnsymbol{footnote}}

\address{
\footnotemark[1] College of Geo-exploration Science and Technology,\\
Jilin University \\
No.938 Xi minzhu street, \\
Changchun, China, 130026 \\
\footnotemark[2] Bureau of Economic Geology,\\
John A. and Katherine G. Jackson School of Geosciences \\
The University of Texas at Austin \\
University Station, Box X \\
Austin, TX, USA, 78713-8924}

\author{Yang Liu\footnotemark[1], Sergey Fomel\footnotemark[2], Cai Liu\footnotemark[1]}

\footer{TCCS-10}
\lefthead{Liu et al.}
\righthead{VD-seislet transform}
\maketitle

\begin{abstract}
The seislet transform is a wavelet-like transform that analyzes
seismic data by following varying slopes of seismic events across
different scales and provides a multiscale orthogonal basis for
seismic data. It generalizes the discrete wavelet transform (DWT) in
the sense that DWT in the lateral direction is simply the seislet
transform with a zero slope. Our earlier work used plane-wave
destruction (PWD) to estimate smoothly varying slopes.  However, PWD
operator can be sensitive to strong noise interference, which makes
the seislet transform based on PWD (PWD-seislet transform)
occasionally fail in providing a sparse multiscale representation for
seismic field data. We adopt a new velocity-dependent (VD) formulation
of the seislet transform, where the normal moveout equation serves as
a bridge between local slope patterns and conventional moveout
parameters in the common-midpoint (CMP) domain. The velocity-dependent
(VD) slope has better resistance to strong random noise, which
indicates the potential of VD seislets for random noise attenuation
under 1D earth assumption. Different slope patterns for primaries and
multiples further enable a VD-seislet frame to separate primaries from
multiples when the velocity models of primaries and multiples are well
disjoint. Results of applying the method to synthetic and field-data
examples demonstrate that the VD-seislet transform can help in
eliminating strong random noise. Synthetic and field-data tests also
show the effectiveness of the VD-seislet frame for separation of
primaries and pegleg multiples of different orders.
\end{abstract}

\section{Introduction}
Signal and noise separation is a persistent problem in seismic
exploration. Sometimes noise is divided into random noise and coherent
noise.  Many authors have developed effective methods of eliminating
random noise. \cite{Ristau01} compared several adaptive filters, which
they applied in an attempt to reduce random noise in geophysical
data. \cite{Karsli06} applied complex-trace analysis to seismic data
for random-noise suppression, recommending it for low-fold seismic
data.  Some transform methods were also used to deal with seismic
random noise, e.g., the discrete cosine transform \cite[]{Lu07}, the
curvelet transform \cite[]{Neelamani08}, and the seislet transform
\cite[]{Fomel10a}. If seismic events are planar (lines in
2D data and planes in 3D data) or locally planar, one can predict
seismic events by using prediction techniques in the $f$-$x$ domain
\cite[]{Canales84,Sacchi01,Liu13} or the $t$-$x$ domain
\cite[]{Claerbout92,Fomel02,Sacchi09,Liu15}.

Multiple reflections are one kind of coherent noise, especially in
marine environments. Wave-equation based algorithms for attenuating
multiples have rapidly developed since 1990s and usually consist of
two steps, namely multiple prediction
\cite[]{Verschuur92,Berkhout97,Weglein97} and 
adaptive subtraction
\cite[]{GEO68-01-03460354,Guitton04,Fomel09a}.
However, these algorithms need to calculate a full wavefield, which is
\old{occasionally} \new{often} a computational bottleneck for their application,
especially in the 3D case. Another popular class of demultiple
techniques is based on variants of the Radon transform
\cite[]{Foster92}. Several revised Radon transforms have been proposed 
for multiple attenuation
\cite[]{Hunt96,Zhou96,Wang03,Hargreaves03}. Radon-transform based 
methods often fail to provide accurate separation because of their
non-sparsity in characterizing seismic data, although they can be
improved by high-resolution methods
\cite[]{Sacchi95,Herrmann00,Trad03}. Despite their usual
classification as noise, multiples can penetrate deeply enough into
the subsurface to illuminate the prospect zone. In this sense,
multiples can also be viewed as a viable signal, rather than noise
\cite[]{Reiter91,Youn01,Berkhout06}. \cite{Brown05} proposed a
least-squares joint imaging of pegleg multiples and primaries and
discussed separation of pegleg multiples and primaries in prestack
data.

In seismic data analysis, it is common to represent signals as sums of
plane waves by using multidimensional Fourier
transforms. The discrete
wavelet transform (DWT) is often preferred to the Fourier transform
for characterizing digital images, because of its ability to localize
events in both time and frequency domains
\cite[]{ripples,mallat}. However, DWT may not be optimal for
describing data that consist of plane waves. Wavelet-like transforms
that explore directional characteristics of images have found
important applications in seismic imaging and data
analysis \cite[]{Chauris08,Herrmann08b}. \cite{Fomel06} investigated
the possibility of designing a wavelet-like transform tailored
specifically to seismic data and introduced it as the seislet
transform. \cite{Fomel10a} further \old{improved} \new{developed} the seislet framework and
proposed additional applications. The original 2D seislet transform
utilizes local data slopes estimated by plane-wave destruction (PWD)
filters \cite[]{Fomel02,Chen13a,Chen13b}. However, a PWD operator can
be sensitive to strong interference, which makes the seislet transform
based on PWD (PWD-seislet transform) occasionally fail in
characterizing noisy signals.

In this paper, we develop a velocity-dependent (VD) concept
\cite[]{Liu13}, where local slopes in prestack data are evaluated from
moveout parameters estimated by conventional velocity-analysis
techniques. We implement a VD-seislet transform and propose its
application for signal and noise separation. We expect the new
VD-seislet transform to provide better compression ability for
reflection events away from interference of strong random noise. We
also provide an application of VD-seislet transform for separating
primaries from pegleg multiples of different orders. We test the
performance of VD-seislet transform using synthetic and field data.

\section{Theory}

\subsection{Review of seislets}
The seislet transform was introduced by \cite{Fomel06} and extended by
\cite{Fomel10a} and \cite{Liu10}. The seislet construction is based on 
the discrete wavelet transform (DWT) combined with seismic data
patterns, such as \new{local} slopes or frequencies. \cite{Fomel02} developed a
local plane-wave destruction (PWD) operation to predict local
plane-wave events, where an all-pass digital filter is used to
approximate the time shift between two neighboring traces. The inverse
operation, plane-wave construction \cite[]{Fomel06b,Fomel10b},
predicts a seismic trace from its neighbors by following locally
varying slopes of seismic events and has been used for designing a
PWD-seislet transform, which is a particular kind of the seislet
transforms \old{with} \new{based on} slope pattern\new{s}. \cite{Liu13} proposed a
velocity-dependent (VD) slope as a pattern in VD-seislet transform,
where the normal moveout (NMO) equation serves as a bridge between
local slopes and scanned NMO velocities.

To define seislet transform, \old{one can} \new{we} follow the general recipe of the
lifting scheme for the discrete wavelet transform, as described by
\cite{athome}. The construction is reviewed in Appendix A. 
Designing pattern-based prediction operator $\mathbf{P}$ and update
operator $\mathbf{U}$ for seismic data is key in the seislet
framework. In the seislet transform, the basic data components can be
different, e.g., traces or common-offset gathers, and the prediction
and update operators shift components according to different patterns.

The prediction and update operators for a simple seislet transform are
defined by modifying the biorthogonal wavelet construction in
equations from Appendix A as follows:
\begin{eqnarray}
  \label{eq:sp}
  \mathbf{P[e]}_k & = & \left(\mathbf{R}_k^{(+)}[\mathbf{e}_{k-1}] + 
  \mathbf{R}_k^{(-)}[\mathbf{e}_{k}]\right)/2 \\
  \label{eq:su}
  \mathbf{U[r]}_k & = & \left(\mathbf{R}_k^{(+)}[\mathbf{r}_{k-1}] + 
    \mathbf{R}_k^{(-)}[\mathbf{r}_{k}]\right)/4\;,
\end{eqnarray}
where $\mathbf{e}_k$ is even components of data at the $k$th transform
scale, $\mathbf{r}_k$ is residual difference between the odd component
of data $\mathbf{o}$ and its prediction from the even component at the
$k$th transform scale, \old{the details are shown in Appendix A,} and
$\mathbf{R}_k^{(+)}$ and $\mathbf{R}_k^{(-)}$ are operators that
predict a component from its left and right neighbors correspondingly
by shifting them according to their patterns. \new{The details are
explained in Appendix A.}

To get the relationship between prediction operator
$\mathbf{R}_k$ and slope pattern $\sigma$, the plane-wave destruction 
operation \cite[]{Fomel02} can be defined in a linear operator notation as
\begin{equation}
  \label{eq:pwd}
  \mathbf{d} = \mathbf{D(\sigma)\,s}\;,
\end{equation}
where seismic section $\mathbf{s} = \left[\mathbf{s}_1 \; 
\mathbf{s}_2 \; \ldots \; \mathbf{s}_N\right]^T$ is a collection of traces, \new{and}
$\mathbf{d}$ is the destruction residual. \old{and the expression of the
nonstationary plane-wave destruction operator $\mathbf{D}$ can be
found in \cite{Fomel02}.} The general structure of $\mathbf{D}$ is
defined as follows \cite[]{Fomel06b,Fomel10b}
\begin{equation}
  \label{eq:d}
  \mathbf{D(\sigma)} =
  \left[\begin{array}{ccccc}
      \mathbf{I} & 0 & 0 & \cdots & 0 \\
      - \mathbf{R}_{1,2}(\sigma_1) & \mathbf{I} & 0 & \cdots & 0 \\
      0 & - \mathbf{R}_{2,3}(\sigma_2) & \mathbf{I} & \cdots & 0 \\
      \cdots & \cdots & \cdots & \cdots & \cdots \\
      0 & 0 & \cdots & - \mathbf{R}_{N-1,N}(\sigma_{N-1}) & \mathbf{I} \\
    \end{array}\right]\;,
\end{equation}
where $\mathbf{I}$ stands for the identity operator, $\sigma_i$ is
local slope pattern, and $\mathbf{R}_{i,j}(\sigma_i)$ is an operator
for prediction of trace $j$ from trace $i$ according to the slope
pattern $\sigma_i$. A trace is predicted by shifting it according to
the local seismic event slopes. Prediction of a trace
from a distant neighbor can be accomplished by simple recursion, i.e.,
predicting trace $k$ from trace $1$ is simply
\begin{equation}
  \label{eq:pred}
  \mathbf{R}_{1,k} = \mathbf{R}_{k-1,k}\,
  \cdots\,\mathbf{R}_{2,3}\,\mathbf{R}_{1,2}\;.
\end{equation}
If $\mathbf{s}_r$ is a reference trace, then the prediction of trace
$\mathbf{s}_k$ is $\mathbf{R}_{r,k} \mathbf{s}_r$.

The predictions need to operate at different scales, which, in this
case, mean different separation distances between the data elements,
e.g., traces in PWD-seislet transform. Equations~\ref{eq:sp} and
\ref{eq:su}, in combination with the forward and inverse lifting
schemes, provide a complete definition of the seislet framework. For
different kinds of slope-based seislets, one needs to define the
corresponding slope pattern $\sigma$.

\subsection{VD-slope pattern for primary reflections}
The kinematic description of a seismic event is an essential step for
several developments in seismic data processing. Local slope is
one important kinematic pattern for seismic data in the time-space
domain. PWD provides a constructive algorithm for estimating local slopes
\cite[]{Claerbout92,Fomel02,Schleicher09,Chen13a,Chen13b} and can 
combine with a seislet framework to implement the PWD-seislet. Local
slant stack \cite[]{Ottolini83} is another standard tool for
calculating slopes.

Under 1D earth assumption, one can consider the classic 
hyperbolic model of primary reflection moveouts at near offsets
\cite[]{Dix55}:
\begin{equation}
  \label{eq:nmo}
  t(x) = \sqrt{t_0^2 + \frac{x^2}{v^2(t_0)}}\;,
\end{equation}
where $t_0$ is the zero-offset traveltime, $t(x)$ is the corresponding
primary traveltime recorded at offset $x$, and $v(t_0)$ is the
stacking, or root mean square (RMS) velocity, which comes from a
standard velocity scan. As follows from equation~\ref{eq:nmo}, the
traveltime slopes $\sigma= dt/dx$ in CMP gathers are given by
\begin{equation}
  {\sigma(t,x)} = {\frac{x}{t(x)\,v^2(t_0,x)}}\;.
  \label{eq:pnmo}
\end{equation}
This calculation is reverse to the one used in \old{velocity estimation} \new{NMO} by
velocity-independent imaging \cite[]{Ottolini83b,Fomel07b}.  To
calculate local slopes of primaries, we need to know $v(t_0, x)$ at
each time-space location ($t_0,x$). This can be accomplished by
simultaneously scanning both $t_0$ and $v(t_0,x)$ according to the
hyperbolic NMO equation at each $x$-coordinate position or by
\emph{time-warping}. In this paper, we use the time-warping \new{algorithm} to calculate
$v(t_0, x)$. \old{the time-warping} \new{Time warping} performs mapping between different
coordinates\old{,}\new{:} if one has sampled functions $f(x)$ and $y(x)$, the
mapping operation finds sampled $f(y)$ \cite[]{Burnett09,Casasanta11}.

After the VD-slope pattern of primaries is calculated, we can design
pattern-based prediction and update operators $\mathbf{R}_k$ by using
plane-wave construction for the VD-seislet transform to represent only
primary reflections. When VD-seislet transform is applied to a CMP
gather, random noise spreads over different scales while the
predictable reflection information gets compressed to large
coefficients at small scales. A simple thresholding \old{method} \new{operation} can easily
remove small coefficients. Finally, applying the inverse
VD-seislet transform reconstructs the signal while attenuating random
noise.

\subsection{VD-slope pattern for pegleg multiples}

In a laterally homogeneous model, the NMO equation~\ref{eq:nmo}
flattens primary events on a CMP gather with offset $x$ and time $t$
to its zero-offset traveltime $t_0$. \cite{Brown05} use an analogous
NMO equation for pegleg multiples under \new{locally} 1D earth assumption. For
example, first-order pegleg can be kinematically approximated by
pseudo-primary with the same offset but with an additional zero-offset
traveltime $\tau$. The NMO equation for an $m$th-order pegleg multiple
is generalized to
\begin{equation}
  \label{eq:mnmo}
  t_m(x) = \sqrt{(t_0+m\tau)^2 + \frac{x^2}{v_m^2(t_0)}}\;,
\end{equation}
where $t_m(x)$ is the corresponding multiple traveltime recorded at
offset $x$ and the effective RMS velocity $v_m$ is
defined according to Dix's equation as:
\begin{equation}
  \label{eq:mveff}
  v_m(t_0) = \sqrt{\frac{t_0v^2(t_0)+m\tau\,v^2(\tau)}{t_0+m\tau}}\;.
\end{equation}
In marine seismic data, $v(\tau)$ is constant water velocity, and
it assumes that we are able to pick zero-offset traveltime
$\tau$ of the water bottom. According to the definition of slopes for
primaries (equation~\ref{eq:pnmo}), slopes for pegleg multiples can be
calculated analogously by:
\begin{equation}
  \sigma_m(t,x) = {\frac{x}{t_m(x)\,v_m^2(t_0,x)}}\;.
  \label{eq:mdip}
\end{equation}
Equation~\ref{eq:mdip} provides the estimation of multiple slopes,
which we use to define VD-seislet frame for representing pegleg
multiples of different orders.

\subsection{Separation of primaries and pegleg multiples using VD-seislet frame}
Once the VD-seislet transform is defined, it can be applied to analyze
signals composed of multiple wavefields, e.g., primaries and multiples
of different orders. If a range of slopes are chosen and a VD-seislet
transform is constructed for each of them, then all the transforms
together will constitute an overcomplete
representation. Mathematically, if $\mathbf{F}_i$ is the VD-seislet
transform for the $i$th slope pattern (corresponding to primaries or
pegleg multiples of different orders), then, for any data vector
$\mathbf{d}$,
\begin{equation}
\label{eq:frame}
\sum\limits_{i=1}^N \|\mathbf{F}_i\,\mathbf{d}\|^2 = 
\sum\limits_{i=1}^N \mathbf{d}^T\,\mathbf{F}_i^T\,\mathbf{F}_i\,\mathbf{d} =
\sum\limits_{i=1}^N \|\mathbf{d}\|^2 = N\,\|\mathbf{d}\|^2\;,
\end{equation}
which means that all transforms taken together constitute a 
\emph{tight frame} with constant $N$ \cite[]{mallat}.

Because of its overcompleteness, a frame representation for a given
signal is not unique. In order to assure that different wavefield
components do not leak into other parts of the frame, it is
advantageous to employ \old{sparseness}\new{sparsity}-promoting inversion
\cite[]{Fomel10a}. We use a nonlinear shaping-regularization scheme
\cite[]{Fomel08} and define sparse decomposition as an iterative
thresholding process \cite[]{Daubechies04}
\begin{eqnarray}
  \label{eq:shape4}
  \hat{\mathbf{f}}_{k+1} & = & \mathbf{S}[\mathbf{F}\,
     \mathbf{d}+(\mathbf{I}-\mathbf{F}\,
   \mathbf{B})\,\hat{\mathbf{f}}_{k}]\;, \\
   \mathbf{f}_{k+1} & = & \mathbf{f}_k + \mathbf{F}\,
   \mathbf{d}- \mathbf{F}\,\mathbf{B} \hat{\mathbf{f}}_{k+1}\;,
   \label{eq:bregman} 
\end{eqnarray}
where $\mathbf{f}_k$ are coefficients of the seislet
frame at $k$th iteration, $\hat{\mathbf{f}}_k$ is an auxiliary
quantity, $\mathbf{S}$ is a soft thresholding operator, $\mathbf{F}$
and $\mathbf{B}$ are frame construction and deconstruction operators
\begin{eqnarray}
  \label{eq:f}
  \mathbf{F} & \equiv & 
\left[\begin{array}{cccc}\mathbf{F}_1 &\mathbf{F}_2 & 
    \cdots &\mathbf{F}_N\end{array}\right]^T\;, \\
\mathbf{B} & \equiv &  
\left[\begin{array}{cccc}\mathbf{F}_1^{-1} &\mathbf{F}_2^{-1} & 
    \cdots &\mathbf{F}_N^{-1}\end{array}\right]\;.
    \label{eq:b}
\end{eqnarray}

\old{Equation~\ref{eq:shape4} converges to the solution of a least-squares
optimization problem regularized by a sparsity constraint
\begin{equation}
\label{eq:lsproblem}
\min_{\mathbf{f}}\|\mathbf{B}\mathbf{f}-\mathbf{d}\|_2^2+\epsilon \|\mathbf{f}\|_1\;,
\end{equation}
where the first term is the $L_2$ norm of the data misfit and the
second term is the $L_1$ norm of the model, which promotes sparsity.

Assuming that equation~\ref{eq:shape4} converges, one can express 
its convergence point as
\begin{equation}
\label{eq:lsolution}
\hat{\mathbf{f}}=[\mathbf{I}+\mathbf{S}(\mathbf{F}\mathbf{B}-
\mathbf{I})]^{-1}\mathbf{S}\mathbf{F}\mathbf{d}\;,
\end{equation}
which is precisely the shaping regularization equation proposed by
\cite{Fomel07a}.} The iteration\old{s} \new{in equations} \ref{eq:shape4} and \ref{eq:bregman}
start\new{s} with $\mathbf{f}_0=\mathbf{0}$ and
$\hat{\mathbf{f}_0}=\mathbf{F}\mathbf{d}$ and \old{are} \new{is} related to the
linearized Bregman iteration \cite[]{Cai09}, which converges to the
solution of the constrained minimization problem:
\begin{equation}
  \label{eq:convex}
  \min_{\mathbf{f}}\|\mathbf{f}\|_1 \; s.t.\; \mathbf{Bf} = \mathbf{d}\;.
\end{equation}

Separated wavefield can be calculated by
$\mathbf{d}_i=\mathbf{B}\mathbf{M}_i\mathbf{f}_\eta$, where $\eta$ is
iteration number, masking operator $\mathbf{M}_i$ is a diagonal matrix
as
\begin{equation}
  \label{eq:M}
  \mathbf{M}_i =
  \left[\begin{array}{ccccccc}
      \mathbf{0} & \cdots & \cdots & \cdots & \cdots & \cdots & \mathbf{0}\\
      \cdots & \cdots & \cdots & \cdots & \cdots & \cdots & \cdots \\
      \mathbf{0} & \cdots & \mathbf{0} & \mathbf{0} & \mathbf{0} & \cdots & \mathbf{0} \\
       \mathbf{0} & \cdots & \mathbf{0} & \mathbf{I}_{i,i} & \mathbf{0} & \cdots & \mathbf{0}\\
      \mathbf{0} & \cdots & \mathbf{0} & \mathbf{0} & \mathbf{0} & \cdots & \mathbf{0} \\
      \cdots & \cdots & \cdots & \cdots & \cdots & \cdots & \cdots \\
      \mathbf{0} & \cdots & \cdots & \cdots & \cdots & \cdots & \mathbf{0}\\
    \end{array}\right]_{N\times N}\;,
\end{equation}
and $\mathbf{d}_i$ corresponds to
the signal of interest (e.g., primaries or multiples of selected
order). \old{One needs to} \new{We} calculate all patterns for primaries and
multiples, and then \new{apply} sparse decomposition (equation\new{s}~\ref{eq:shape4} and
\ref{eq:bregman}) \old{will} \new{to} separate primaries from multiples. In practice,
a small number of iterations is usually sufficient for convergence and
for achieving both model sparseness and data recovery. \old{However,
similar to Radon transform, the proposed method need an assumption
that primaries and multiples correspond to different velocity models.}

 \section{Synthetic Data Examples}

\subsection{Validation of slope estimation and random noise elimination}
\inputdir{nmo}
A simple synthetic example is shown in Figure~\ref{fig:synt,sdip}a.
The synthetic data were generated by applying inverse NMO with
time-varying velocities and represent perfectly
hyperbolic events.  Figure~\ref{fig:synt,sdip}b shows local event
slopes measured from the data using PWD algorithm
\cite[]{Fomel02}. PWD provides an accurate slope field for noise-free
data.  Figure~\ref{fig:noise,pdip}a and
\ref{fig:noise,pdip}b show the data after adding normally distributed
random noise and local slopes from PWD, respectively. Compared with
Figure~\ref{fig:synt,sdip}b, PWD fails in finding exact slope field
because of strong random noise. Next, we calculate slopes using NMO
velocities from velocity scan. Picked NMO velocities
(Figure~\ref{fig:svsc2,vdip}a) are close to the exact velocity because
velocity scan is less sensitive to strong random noise. As a
consequence, VD slopes calculated from equation~\ref{eq:pnmo} provide
a more accurate result (Figure~\ref{fig:svsc2,vdip}b).

  \multiplot{2}{synt,sdip}{width=0.4\columnwidth}{Synthetic
         data (a) and slopes calculated by PWD (b).}

  \multiplot{2}{noise,pdip}{width=0.4\columnwidth}{Synthetic
         noisy data (a) and slopes calculated by PWD (b).}

  \multiplot{2}{svsc2,vdip}{width=0.4\columnwidth}{Velocity
         scanning (dash line: exact velocity, solid line: picked
         velocity) (a) and VD slopes (b).}

A direct application of the seislet transform is denoising. We apply
both PWD-seislet and VD-seislet transforms on the noisy data
(Figure~\ref{fig:noise,pdip}a). Figure~\ref{fig:pseis,vseis,ccomp}a and
\ref{fig:pseis,vseis,ccomp}b show the transform coefficients of PWD-seislet
and VD-seislet, respectively. The hyperbolic events are compressed in
both transform domains. Notice that PWD-seislet coefficients get more
concentrated at small scale than those of VD-seislet because part\new{s} of
the random noise \old{is} \new{are} also compressed along inaccurate PWD
slopes. Meanwhile, random noise gets spread over different scales in
the VD-seislet domain, while the predictable reflection information
gets compressed to large coefficients at small scales, which makes
signal and noise display different amplitude
characteristics. Figure~\ref{fig:pseis,vseis,ccomp}c shows a
comparison between the decay of coefficients sorted from large to
small in the PWD-seislet transform and the VD-seislet
transform. Seislet transform can compress the seismic events with
coincident wavelets, if the slopes of the reflections are correct, the
sparse large coefficients only correspond to the stacked reflection
events. However, when the slopes of the reflections are not accurate,
the stacked amplitude values for inconsistent wavelets will create
more coefficients with smaller values. VD slopes are less affected by
strong random noise than PWD slopes, \old{therefore,
they show} \new{which results in} a faster decay of the VD-seislet
coefficients. A simple thresholding method can easily remove the small
coefficients of random noise. Figure~\ref{fig:pclean,vclean}a and
\ref{fig:pclean,vclean}b display the denoising results by using
PWD-seislet transform and VD-seislet transform, respectively. The
events after PWD-seislet transform denoising show serious distortion
while VD-seislet transform produces a reasonable denoising result. For
numerically comparison, we use the signal-to-noise ratio (SNR) defined
as $SNR=10\log_{10}\frac{||\mathbf{s}||_2^2}
{||\mathbf{s}-\hat{\mathbf{s}}||_2^2}$, where $\mathbf{s}$ is the
noise-free signal and $\hat{\mathbf{s}}$ is the denoised signal. The
original SNR of the noisy data (Figure~\ref{fig:noise,pdip}a) is
-12.53 dB. The SNR of the denoised results using the PWD-seislet
transform (Figure~\ref{fig:pclean,vclean}a) and the VD-seislet
transform (Figure~\ref{fig:pclean,vclean}b) are 0.53 dB and 1.94 dB,
respectively.

   \multiplot{3}{pseis,vseis,ccomp}{width=0.4\columnwidth}{PWD-seislet
         coefficients (a), VD-seislet coefficients (b), and transform
         coefficients sorted from large to small, normalized, and
         plotted on a decibel scale (Solid line - VD-seislet
         transform. Dashed line - PWD-seislet transform) (c).}

   \multiplot{2}{pclean,vclean}{width=0.4\columnwidth}{Denoising
         result using different transforms. PWD-seislet transform (a)
         and VD-seislet transform (b).}

\subsection{Separation of primaries and pegleg multiples}
\inputdir{haskell}
Next, we use a synthetic CMP gather (Figure~\ref{fig:hask,vscan}a) to
test separation of primaries and pegleg multiples by VD-seislet
frame. This gather was generated by \cite{Lumley94} using
Haskell-Thompson elastic modeling and a well log from the Mobil AVO
dataset \cite[]{key}. The gather contains primaries and water-bottom
multiples of different orders.

To separate pegleg multiples from primaries, we transform the
data using VD-seislet frame by involving different VD slope fields
(Figure~\ref{fig:dips,frame}a) according to equations~\ref{eq:pnmo}
and
\ref{eq:mdip}. The primary velocities and calculated velocities of
different-order pegleg multiples are shown in
Figure~\ref{fig:hask,vscan}b. The estimated curves of multiple
velocities indicate accurate trends in velocity spectra. \old{The proposed
method utilizes} \new{We use} a nonlinear shaping-regularization scheme
(equations~\ref{eq:shape4}-\ref{eq:bregman}) to separate different
wavefield components, which are shaped to be sparse in the
corresponding VD-seislet frame domain
(Figure~\ref{fig:dips,frame}b). In this example, the pattern number \new{in equation~\ref{eq:frame}}
$N$ is selected to \new{be} 4\old{ in equation~\ref{eq:frame}}. The
separated wavefields are shown in
Figure~\ref{fig:comp0,comp1,comp2,comp3} and display \old{reasonable} \new{reasonably accurate}
separation results.

 \multiplot{2}{hask,vscan}{width=0.4\textwidth}{Synthetic model (a)
               and velocity trends of primaries and multiples (b).}
 \multiplot{2}{dips,frame}{width=0.4\textwidth}{VD slopes (a) and
                           VD-seislet coefficients (b).}
\multiplot{4}{comp0,comp1,comp2,comp3}{width=0.4\textwidth}{Separated
              wavefields. Primaries (a), first-order multiples (b),
              second-order multiples (c), and third-order multiples
              (d).}

 \section{Field Data Examples}

We test VD-seislet denoising by employing a field land data provided
by Geofizyka Torun Sp. Z.o.o, Poland from FreeUSP
website\footnote{http://www.freeusp.org/RaceCarWebsite/TechTransfer/Tutorials/Processing\_2D}. Figure~\ref{fig:cmps,vdip,vdseis,vddenoise}a
show the CMP gathers after removing most of ground roll. The strong
random noise makes reflection events hardly visible. However, velocity
analysis from equation~\ref{eq:nmo} can still produce a reasonable
velocity field. Equation~\ref{eq:pnmo} converts RMS velocity to
seismic pattern (Figure~\ref{fig:cmps,vdip,vdseis,vddenoise}b), which
displays the hyperbolic slopes from negative to positive in CMP
gathers and varying slopes in common-offset
section. VD-seislet transform utilizes the slopes to compress
reflections along offset
axis. Figure~\ref{fig:cmps,vdip,vdseis,vddenoise}c shows the
VD-seislet coefficients, in which the small dynamic range of seislet
coefficients implies a good compression ratio.  If we choose the
significant coefficients at the coarse scale, e.g., scale $<$ 8, and
zero out difference coefficients at the finer scales, the inverse
transform effectively removes incoherent noise from the gathers
(Figure~\ref{fig:cmps,vdip,vdseis,vddenoise}d).

\inputdir{usp}
   \multiplot{4}{cmps,vdip,vdseis,vddenoise}{width=0.47\columnwidth}{Field CMP
         gathers (a), VD slopes (b), VD-seislet coefficients (c), and
         denoising result using VD-seislet transform (b).}

Next, we test the proposed algorithm to separate multi-wavefields on a
single CMP gather from the Viking Graben (Mobil AVO) dataset
\cite[]{key}. The field data are shown in
Figure~\ref{fig:cmp,rvscan}a. We pick the primary velocities by muting
spectra energy of multiples. Multiple RMS velocities with different
orders (equation~\ref{eq:mveff}) follow the pseudo-primary NMO
equation~\ref{eq:mnmo}. The velocity spectra of primaries and
multiples are shown in
Figure~\ref{fig:cmp,rvscan}b. Equations~\ref{eq:pnmo} and
\ref{eq:mdip} convert velocities to slopes, which help the VD-seislet frame
separate primaries from different-order pegleg multiples (we only
display three orders). Figure~\ref{fig:rcomp0,rcomp1,rcomp2,rcomp3}
displays the separated primaries and different-order multiples. The
corresponding velocity spectra are shown in
Figure~\ref{fig:rvscan0,rvscan1,rvscan2,rvscan3}. After separating
different wavefields, the velocity spectra confirm that the signals
get concentrated around their respective trends.

\inputdir{demultiple}
  \multiplot{2}{cmp,rvscan}{width=0.4\columnwidth}{Field CMP gather
               (a) and velocity trends of primaries and multiples (b).}
  \multiplot{4}{rcomp0,rcomp1,rcomp2,rcomp3}{width=0.4\columnwidth}{Separated
                 wavefields. Primaries (a), first-order multiples (b),
                 second-order multiples (c), and third-order multiples
                 (d).}
\multiplot{4}{rvscan0,rvscan1,rvscan2,rvscan3}{width=0.4\columnwidth}{Velocity
                 spectra of different wavefields. Primaries (a),
                 first-order multiples (b), second-order multiples
                 (c), and third-order multiples (d).}

\section{Discussion}
What are the limitations of the proposed algorithms? First,
VD-seislet transform \old{cannot properly deal} \new{may have difficulties in dealing} with nonhyperbolic \old{travel
times, however,} \new{moveouts. However, it is possible to extend it to} nonhyperbolic moveout equation \cite[]{Casasanta11}, which would provide the
possibility to handle large offsets and anisotropy
\cite[]{Fomel01b}. \old{Next, the velocity scan occasionally arises
additional errors, it is possible to pre-denoise followed by
PWD-seislet transform, but the results depend on careful parameter
selection for preprocessing methods.} Compared with PWD-seislet
transform, VD-seislet transform provides more accurate
representation for seismic events with class II AVO anomalies
\cite[]{Rutherford89} that cause seismic amplitudes to go through a
polarity reversal \old{PWD-seislet transform may fail since PWD slopes
depend on amplitude value of seismic data, however, VD slopes are more
accurate} by using advanced velocity analysis, e.g., AB
semblance \cite[]{Sarkar01,Sarkar02,Fomel09b}. Finally, the proposed
method works \old{for the velocity models} \new{best for the
relatively deep water situations} where primaries and multiples are
well \old{disjoint, for instance, it may fail in the case of shallow
water multiples.} \new{separated}. \old{Therefore, the current
VD-seislet transform provide an alternative tool for analyzing CMP
gathers with strong random noise in structurally simple areas or
disjoint pegleg multiples.}

\section{Conclusions}

We have introduced the VD-seislet transform, a new domain for
analyzing prestack reflection data in CMP domain. The new transform is
able to compress reflection data away from strong random
noise. \old{The} \new{In this formulation, the} normal moveout
equation serves as a bridge between local slopes and scanned
velocities that are not sensitive to strong random noise and
aliasing. We also used the explicit relationship between slopes and
velocities of primary and multiple events to extend VD-seislet
transform to a VD-seislet frame. We have shown example applications of
VD-seislet transform to signal and noise separation. Other traditional
processing tasks such as data interpolation can be
also \old{simply} \new{easily} defined in the VD-seislet domain.

\section{Acknowledgments}
We thank Valentina Socco, Adriana Citlali Ramirez, and three anonymous
reviewers for helpful suggestions, which improved the quality of the
paper. This work is partly supported by National Natural Science
Foundation of China (Grant No. 41430322 and 41274119) and 863
Programme of China (Grant No. 2012AA09A2010). All results are
reproducible in the Madagascar open-source software environment
\cite[]{m8r}.

\append{The lifting scheme for DWT}

The lifting scheme \cite[]{Sweldens95} provides a convenient approach
for defining wavelet transforms by breaking them down into the
following steps:
\begin{enumerate}
\item Divide data into even and odd components, $\mathbf{e}$ and
  $\mathbf{o}$.
\item Find a residual difference, $\mathbf{r}$, between the odd
  component and its prediction from the even component:
  \begin{equation}
    \label{eq:c}
    \mathbf{r}  = \mathbf{o} - \mathbf{P[e]}\;,
  \end{equation}
  where $\mathbf{P}$ is a \emph{prediction} operator.
\item Find a coarse approximation, $\mathbf{c}$, of the data by
  updating the even component:
  \begin{equation}
    \label{eq:r}
    \mathbf{c}  = \mathbf{e} + \mathbf{U[r]}\;,
  \end{equation}
  where $\mathbf{U}$ is an \emph{update} operator.
\item The coarse approximation, $\mathbf{c}$, becomes the new
  data, and the sequence of steps is repeated at the next scale.
\end{enumerate}

The Cohen-Daubechies-Feauveau (CDF) 5/3 biorthogonal wavelets
\cite[]{Cohen92} are constructed by making the prediction operator a
linear interpolation between two neighboring samples,
\begin{equation}
  \label{eq:p}
  \mathbf{P[e]}_k = \left(\mathbf{e}_{k-1} + \mathbf{e}_{k}\right)/2\;,
\end{equation}
and by constructing the update operator to preserve the running
average of the signal \cite[]{athome}, as follows:
\begin{equation}
  \label{eq:u}
  \mathbf{U[r]}_k = \left(\mathbf{r}_{k-1} + \mathbf{r}_{k}\right)/4\;.
\end{equation}

Furthermore, one can create a high-order CDF 9/7 biorthogonal wavelet
transform by using CDF 5/3 biorthogonal wavelets twice with different
lifting operator coefficients \cite[]{Lian01}. The transform is
easily inverted according to reversing the steps above:       
\begin{enumerate}
\item Start with the coarsest scale data representation $\mathbf{c}$
      and the coarsest scale residual $\mathbf{r}$.
\item Reconstruct the even component $\mathbf{e}$ by reversing the 
      operation in equation~\ref{eq:r}, as follows:
\begin{equation}
    \label{eq:e}
    \mathbf{e} = \mathbf{c} - \mathbf{U[r]}\;,
  \end{equation}
\item Reconstruct the odd component 
  $\mathbf{o}$ by reversing the operation in equation~\ref{eq:c}, as follows:
  \begin{equation}
    \label{eq:o}
    \mathbf{o} = \mathbf{r}  + \mathbf{P[e]}\;,
  \end{equation} 
\item Combine the odd and even components to generate the data at 
      the previous scale level and repeat the sequence of steps.
\end{enumerate}



\bibliographystyle{seg}
\bibliography{SEG,SEP2,geo2014}

