\title{Developers}
\author{Jeff Godwin}
\maketitle

The goal of this brief tutorial is to demonstrate the key concepts
behind writing programs for Madagascar.  Since the Madagascar API is
well documented, we focus on a high level view of the development
process and the core design functionality.  For technical details of
how to interface with the Madagascar API for a specific programming
language, we refer the reader to the current online documentation
mentioned at the end of this tutorial.

By the end of this tutorial, you should understand the basic design of
Madagascar programs, and where to find more information about how to
develop programs using your programming language of choice.

\section{Core design}

As previously mentioned, the main way for programs to communicate in
Madagascar is through RSF files.  Thus, Madagascar programs are
expected to read RSF files, process them, and then write RSF files as
outputs that can be then used in other RSF programs.  At the highest
level of abstraction, we can consider RSF programs as black boxes that
simply input and output RSF files and do something to them in between.
From a practical standpoint, your programs will first read RSF files
from disk into memory (or progressively do so) as hypercubes.  Once
the file is read,the program processes the hypercubes using routines
that you design or external libraries.  After processing, you then
write the hypercubes to RSF files on disk.

In pseudocode this process looks something like:
\begin{verbatim}
data = make_array()

input = rsf_input()

rsf_read(input,data) 

. . . process data . . .

output = rsf_output()

rsf_write(output,data)
\end{verbatim}

\section{Core API}

Madagascar provides core APIs for each language to ease the process of
reading/writing files.  Additionally, the core API provides functions
to parse command line variables that can be used to control the
execution of your programs.

In some languages, the API is extended to allow you to access commonly
used functions from the RSF main library.  For example, you can
use the FFT library that is contained in Madagascar.  

\section{Program design philosophy}

While Madagascar does not strictly have a design requirement for
programs to enter the main distribution, there are some general
guidelines to programs that we would like developers to follow.  In
particular, we would like developers to: design programs that have
error handling and parameter checking, that accept command line
arguments to control important parameters in the program, and write
programs that are limited in scope.  For example, a program that is
limited in scope is a program that computes the Fourier transform of a
real-valued signal and outputs a complex-valued RSF file.  Conversely,
a program that overreaches in its scope, would be a program that
conducts a long series of processing completely in another language
(say C or Fortran).  You should avoid designing programs with too much
scope, because you cannot fully leverage the advantages of SCons and
Python, if everything is happening inside a C or Fortran program.
\setlength{\unitlength}{1in}
\begin{figure}
\begin{picture}(5,0.5)(0,0)
    \put(0,0){\framebox(1,0.5){RSF file}}
    \put(1,0.25){\vector(1,0){1}}
    \put(2,0){\framebox(1,0.5){Program}}
    \put(3,0.25){\vector(1,0){1}}
    \put(4,0){\framebox(1,0.5){RSF file}}
\end{picture}
\caption{A Madagascar program reads RSF files, processes them, and then outputs them at the most fundamental level.}
\end{figure}

\section{Final thoughts}

With this background, and some additional information provided below,
you should be able to start writing your own Madagascar programs to
process data and implement algorithms that are not provided with
Madagascar by default.  We welcome you to the developer community, and
would greatly appreciate it if you would consider releasing your
programs to the community as a whole.

If you have further questions, please feel free to ask the RSF mailing lists.

\section{Further information}

For more information about using the API for a particular language,
please see:
\url{http://ahay.org/wiki/Guide_to_madagascar_API}.

For more information about developing Madagascar programs in general
see:
\url{http://ahay.org/wiki/Adding_new_programs_to_Madagascar}.

For more information about contributing your programs see:
\url{http://ahay.org/wiki/Contributing_new_programs_to_Madagascar}.

For a full reference of the C API see:
\url{http://www.ahay.org/RSF/book/rsf/manual/manual_html/}
