\chapter{Vplot graphics}

The final results of a data analysis project are often represented by figures. Madagascar includes a versatile plotting library for generating scientific illustrations called Vplot. Vplot defines a data format for storing figures. Figures stored in that format can be displayed on the screen or transformed into other formats for inclusion in scientific papers, presentation slides, or other documents.

\section{Vplot history}

Vplot was developed in 1980s at the Stanford Exploration Project
(SEP). At that time, there was no universal format for graphics files
and no univeral libraries for representing graphics on hardware
devices, such as monitors or printers. The purpose of the Vplot was to
provide such a format so that the same plots could be easily
transferred between devices
\cite[]{Cole.sep.60.349,Dellinger.sep.61.327}.


The \texttt{man} page for Vplot says
  \begin{quote}
    Vplot originally stood for `vector plot', but this name is now
    insufficient because vplot supports not only vectors, but also
    filled areas (either raster patterns or hatched), text (Hershey
    fonts), and raster images (including grey-scale dithered ones on
    monochrome devices such as laser printers).
    \end{quote}

Joe Dellinger was the main developer of Vplot, other major
contributors at Stanford were Steve Cole and Dave Nichols, with
additional contributions by other SEP students. After Vplot was
adopted by Madagascar, new filter programs (called ``pens'') for
converting Vplot files to other formats were added.

\section{Vplot format}
\inputdir{test}

Vplot breaks all graphics into primitives that are encoded in the file
in a mixed text and binary form. The source of the Vplot library is
contained in the Madagascar source tree under
\texttt{\$RSFSRC/plot/lib} with example main functions in
\texttt{\$RSFSRC/plot/test}.

For example, here is a portion of a function that draws a diamond
shape to reproduce a figure (Figure~\ref{fig:croshyp}) from \cite[]{bei}:

\lstset{language=c,numbers=left,numberstyle=\tiny,showstringspaces=false}
\lstinputlisting[firstline=106,lastline=128,frame=single,title=\$RSFSRC/plot/test/croshyp.c]{\RSF/plot/test/croshyp.c}

The function calls are easy to decipher: \texttt{vp\_umove} sets the
starting point (in predefined ``user'' coordinates),
\texttt{vp\_udraw} draws a line to the next point. Four lines form a
diamond shape.

\sideplot{croshyp}{width=\textwidth}{A figure reproduced from (Claerbout, 2006)
  shows an example of using Vplot library.}

A documentation of the programming interface to different functions in
the library can be obtained by running \texttt{man libvplot}.

\subsection{Python interface}

It is also possible to generate Vplot files from Python using the
Python interface to the library defined in \texttt{vplot} module.

For example, here is an equivalent (slighly more concise) Python code
for generating a diamond shape in Figure ~\ref{fig:croshyp}:

\lstset{language=python,numbers=left,numberstyle=\tiny,showstringspaces=false}
\lstinputlisting[firstline=32,lastline=42,frame=single,title=\$RSFSRC/plot/test/croshyp.py]{\RSF/plot/test/croshyp.py}

\subsection{sfpldb and sfplas}

Vplot files use a mixture of binary and text code, which is not easy
to decipher, Two convenient programs, \texttt{sfpldb} and
\texttt{sfplas}, allow the user to convert a Vplot file to a pure text representation and back to Vplot.

For a quick demonstration, let us create an example Vplot file:
\begin{verbatim}
bash$ sfspike n1=10 k1=5 | sfsmooth rect1=3 | sfgraph title=Example > test.vpl
\end{verbatim}

Suppose that we want to change the axis label from ``Time (s)'' to ``Distance (km)'' without regenerating the file. This can be done by
\begin{enumerate}
\item converting the Vplot file to a text file
\begin{verbatim}
bash$ < test.vpl sfpldb > test.txt
\end{verbatim}
\item editing the text file \texttt{test.txt}
\item converting back to Vplot
\begin{verbatim}
bash$ < test.txt sfplas > test2.vpl
\end{verbatim}
\end{enumerate}

Aternatively, we can do it using a Unix pipe and \texttt{sed} text editor, as follows:
\begin{verbatim}
bash$ < test.vpl sfpldb | sed s/Time\ \(s\)/Distance\ \(km\)/ | sfplas > test2.vpl
\end{verbatim}

\subsection{sfvplotdiff and reproducibility testing}

\section{Graphics programs}

\subsection{sfdots}

\subsection{sfgraph}

\subsection{sfgraph3}

\subsection{sfwiggle}

\subsection{sfbargraph}

\subsection{sfcontour}

\subsection{sfcontour3}

\subsection{sfthplot}

\subsection{sfgrey}

\subsection{sfgrey3}

\subsection{sfgrey4}

\subsection{sfbox}

\section{Pen programs}

\subsection{xtpen}

\subsection{oglpen}

\subsection{pspen}

\section{Vplot and interactivity}

\bibliographystyle{seg}
\bibliography{SEP2,vplot}
