\chapter{Guide to RSF format}
%\email{sergey.fomel@beg.utexas.edu}
%\author{Sergey Fomel}
%\lefthead{Fomel}
%\righthead{RSF format}

%\maketitle

Madagascar programs operate on files using a universal file format
called RSF (from Regularly Sampled Format). This chapter explains both
the design of this format and the rational behind it.

\section{Principles}

The main design principle behind the RSF file format is KISS (``Keep It
Simple, Stupid!''). The RSF format is borrowed from the SEPlib data format
originally designed at the Stanford Exploration Project
\cite[]{Claerbout.sep.70.413}. The format is made as simple as possible for
maximum convenience, transparency and flexibility.

According to the Unix tradition, common file formats should be in a readable
textual form so that they can be easily examined and processed with universal
tools.  \cite{taoup} writes:
\begin{quote}
  To design a perfect anti-Unix, make all file formats binary and opaque, and
  require heavyweight tools to read and edit them.
\end{quote}
\begin{quote}
  If you feel an urge to design a complex binary file format, or a complex
  binary application protocol, it is generally wise to lie down until the
  feeling passes.
\end{quote}

Storing large-scale datasets in a text format may not be economical. RSF
chooses the next best thing: it allows data values to be stored in a binary
format but puts all data attributes in text files that can be read by humans
and processed with universal text-processing utilities.

\subsection{Example}
\inputdir{format}

Let us first create some synthetic RSF data.
\begin{verbatim}
bash$ sfmath n1=1000 output='sin(0.5*x1)' > sin.rsf
\end{verbatim}

Open and read the file \texttt{sin.rsf}.
\begin{verbatim}
bash$ cat sin.rsf
sfmath  rsf/rsf/rsftour:        fomels@egl      Sun Jul 31 07:18:48 2005

        o1=0
        data_format="native_float"
        esize=4
        in="/tmp/sin.rsf@"
        x1=0
        d1=1
        n1=1000
\end{verbatim}
The file contains nine lines with simple readable text. The first line
shows the name of the program, the working directory, the user and
computer that created the file and the time it was created (that
information is recorded for accounting purposes). Other lines contain
parameter-value pairs separated by the ``='' sign. The ``in''
parameter points to the location of the binary data. Before we discuss
the meaning of parameters in more detail, let us plot the data.
\begin{verbatim}
bash$ < sin.rsf  sfwiggle title='One Trace' | sfpen
\end{verbatim}
On your screen, you should see a plot similar to Figure~\ref{fig:sin1}.

\plot{sin1}{width=5in}{An example sinusoid plot.}
 
Suppose you want to reformat the data so that instead of one trace of a
thousand samples, it contains twenty traces with fifty samples each. Try
running
\begin{verbatim}
bash$ < sin.rsf sed 's/n1=1000/n1=100 n2=10/' > sin10.rsf 
bash$ < sin10.rsf sfwiggle title=Traces | sfpen
\end{verbatim}
or (using pipes)
\begin{verbatim}
bash$ < sin.rsf sed 's/n1=1000/n1=50 n2=20/' | sfwiggle title=Traces | sfpen
\end{verbatim}
On your screen, you should see a plot similar to Figure~\ref{fig:sin2}.

\plot{sin2}{width=5in}{An example sinusoid plot, with data reformatted to twenty traces.}

What happened? We used \texttt{sed}, a standard Unix line editing utility to
change the parameters describing the data dimensions. Because of the
simplicity of this operation, there is no need to create specialized data
formatting tools or to make the \texttt{sfwiggle} program accept additional
formatting parameters. Other general-purpose Unix tools that can be applied on
RSF files include \texttt{cat}, \texttt{echo}, \texttt{grep}, etc. 

An alternative way to obtain the previous result is to run
\begin{verbatim}
bash$ ( cat sin.rsf; echo n1=50 n2=20 ) > sin10.rsf 
bash$ < sin10.rsf sfwiggle title=Traces | sfpen
\end{verbatim}
In this case, the \texttt{cat} utility simply copies the contents of the
previous file, and the \texttt{echo} utility appends new line ``\texttt{n1=50
  n2=20}''. A new value of the \texttt{n1} parameter overwrites the old value
of \texttt{n1=1000}, and we achieve the same result as before.

Of course, one could also edit the file by hand with one of the general
purpose text editors. For recording the history of data processing, it is
usually preferable to be able to process files with non-interactive tools.

\section{Header and Data files}

A simple way to check the layout of an RSF file is with the \texttt{sfin}
program.
\begin{verbatim}
bash$ sfin sin10.rsf
sin10.rsf:
    in="/tmp/sin.rsf@"
    esize=4 type=float form=native
    n1=50          d1=1           o1=0
    n2=20          d2=?           o2=?
        1000 elements 4000 bytes
\end{verbatim}
The program reports the following information: the location of the data file
(\texttt{/tmp/sin.rsf\@}), the element size (4 bytes), the element
type (floating point), the element form (native), the hypercube dimensions
($50 \times 20$), axis scaling (1 and unspecified), and axis origin (0 and
unspecified). It also checks the total number of elements and bytes in the
data file.

Let us examine this information in detail. First, we can verify that the data
file exists and contains the specified number of bytes:
\begin{verbatim}
bash$ ls -l /tmp/sin.rsf@
-rw-r--r--  1 sergey users 4000 2004-10-04 00:35 /tmp/sin.rsf@
\end{verbatim}
4000 bytes in this file are required to store $50 \times 20$ floating-point
4-byte numbers in a binary form. Thus, the data file contains nothing but the
raw data in a contiguous binary form.

\subsection{Datapath}

How did the RSF program (\texttt{sfmath}) decide where to put the data file?
In the order of priority, the rules for selecting the data file name and the
data file directory are as follows:
\begin{enumerate}
\item Check \texttt{out=} parameter on the command line. The parameter
  specifies the output data file location explicitly.
\item Specify the path and the file name separately. 
  \begin{itemize}
  \item The rules for the path
  selection are:
  \begin{enumerate}
  \item Check \texttt{datapath=} parameter on the command line. The parameter
    specifies a string to prepend to the file name. The string may contain the
    file directory.
  \item Check \texttt{DATAPATH} environmental variable. It has the same meaning
    as the parameter specified with \texttt{datapath=}.
  \item Check for \texttt{.datapath} file in the current directory. The file may
    contain a line 
\begin{verbatim}
datapath=/path/to_file/
\end{verbatim}
    or
\begin{verbatim}
machine_name datapath=/path/to_file/
\end{verbatim}
    if you indent to use different paths on different platforms.
  \item Check for \texttt{.datapath} file in the user home directory.
  \item Put the data file in the current directory (similar to \texttt{datapath=./}).
  \end{enumerate}
  \item The rules for the filename selection are:
    \begin{enumerate}
    \item If the output RSF file is in the current directory, the name of the
      data file is made by appending \@.
    \item If the output file is not in the current directory or if it is
    created temporarily by a program, the name is made by appending random
    characters to the name of the program and selected to be unique.
  \end{enumerate}
  \end{itemize}
\end{enumerate}

Here are some examples:
\begin{description}
\item[Example 1] \ \\
\begin{verbatim}
bash$ sfspike n1=10 out=test1 > spike.rsf
bash$ grep in spike.rsf
        in="test1"
\end{verbatim}
\item[Example 2] \ \\
\begin{verbatim}
bash$ sfspike n1=10 datapath=/tmp/ > spike.rsf
bash$ grep in spike.rsf
        in="/tmp/spike.rsf@"
\end{verbatim}
\item[Example 3] \ \\
\begin{verbatim}
bash$ DATAPATH=/tmp/ sfspike n1=10 > spike.rsf
bash$ grep in spike.rsf
        in="/tmp/spike.rsf@"
\end{verbatim}
\item[Example 4] \ \\
\begin{verbatim}
bash$ sfspike n1=10 datapath=/tmp/ > /tmp/spike.rsf
bash$ grep in /tmp/spike.rsf
in="/tmp/sfspikejcARVf"
\end{verbatim}
\end{description}

\subsubsection{Packing header and data together}

While the header and data files are separated by default, it is also possible
to pack them together into one file. To do that, specify the program's
``\texttt{out}'' parameter as \texttt{out=stdout}. Example:
\begin{verbatim}
bash$ sfspike n1=10 out=stdout > spike.rsf
bash$ grep in spike.rsf
Binary file spike.rsf matches
bash$ sfin spike.rsf
spike.rsf:
    in="stdin"
    esize=4 type=float form=native
    n1=10          d1=0.004       o1=0          label1="Time" unit1="s"
        10 elements 40 bytes
bash$ ls -l spike.rsf
-rw-r--r--  1 sergey users 196 2004-11-10 21:39 spike.rsf
\end{verbatim}
If you examine the contents of \texttt{spike.rsf}, you will find that it
starts with the text header information, followed by special
symbols, followed by binary data. 

Packing headers and data together may not be a good idea for data processing
but it works well for storing data: it is easier to move the packed file
around than to move two different files (header and binary) together while
remembering to preserve their connection. Packing header and data together is
also the current mechanism used to push RSF files through Unix pipes.

\subsection{Type}

The data stored with RSF can have different types: character, unsigned
character, integer, floating point, complex. By default, single
precision is used for numbers (\texttt{int} and \texttt{float} data
types in the C programming language) but double precision and other
integer types (\texttt{short} and \texttt{long}) are also
supported. The number of bytes required for represent these numbers
may depend on the platform.

\subsection{Form}

The data stored with RSF can also be in a different form: ASCII, native
binary, and XDR binary. Native binary is often used by default. It is the
binary format employed by the machine that is running the application. On
Linux-running PC, the native binary format will typically correspond to the
so-called little-endian byte ordering. On some other platform, it might be
big-endian ordering. XDR is a binary format designed by Sun for exchanging
files over network. It typically corresponds to big-endian byte ordering. It
is more efficient to process RSF files in the native binary format but, if you
intend to access data from different platforms, it might be a good idea to
store the corresponding file in an XDR format. RSF also allows for an ASCII
(plain text) form of data files. 

Conversion between different types and forms is accomplished with
\texttt{sfdd} program. Here are some examples. First, let us create synthetic
data.
\begin{verbatim}
bash$ sfmath n1=10 output='10*sin(0.5*x1)' > sin.rsf
bash$ sfin sin.rsf
sin.rsf:
    in="/tmp/sin.rsf@"
    esize=4 type=float form=native
    n1=10          d1=1           o1=0
        10 elements 40 bytes
bash$ < sin.rsf sfdisfil
   0:             0        4.794        8.415        9.975        9.093
   5:         5.985        1.411       -3.508       -7.568       -9.775
\end{verbatim}
Converting the data to the integer type:
\begin{verbatim}
bash$ < sin.rsf sfdd type=int > isin.rsf
bash$ sfin isin.rsf
isin.rsf:
    in="/tmp/isin.rsf@"
    esize=4 type=int form=native
    n1=10          d1=1           o1=0
        10 elements 40 bytes
bash$ < isin.rsf sfdisfil
   0:    0    4    8    9    9    5    1   -3   -7   -9
\end{verbatim}
Converting the data to the ASCII form:
\begin{verbatim}
bash$ < sin.rsf sfdd form=ascii > asin.rsf
bash$ < asin.rsf sfdisfil
   0:             0        4.794        8.415        9.975        9.093
   5:         5.985        1.411       -3.508       -7.568       -9.775
bash$ sfin asin.rsf
asin.rsf:
    in="/tmp/asin.rsf@"
    esize=0 type=float form=ascii
    n1=10          d1=1           o1=0
        10 elements
bash$ cat /tmp/asin.rsf@
0 4.79426 8.41471 9.97495 9.09297 5.98472 1.4112 -3.50783
-7.56803 -9.7753
\end{verbatim}

\subsection{Hypercube}

While RSF stores binary data in a contiguous 1-D array, the conceptual
data model is a multidimensional hypercube. By convention, the
dimensions of the cube are defined with parameters \texttt{n1},
\texttt{n2}, \texttt{n3}, etc.  The fastest axis is \texttt{n1}.
Additionally, the grid sampling can be given by parameters
\texttt{d1}, \texttt{d2}, \texttt{d3}, etc. The axes origins are given
by parameters \texttt{o1}, \texttt{o2}, \texttt{o3}, etc. Optionally,
you can also supply the axis label strings: \texttt{label1},
\texttt{label2}, \texttt{label3}, etc., and axis units strings:
\texttt{unit1}, \texttt{unit2}, \texttt{unit3}, etc. 

\section{Compatibility with other file formats}

It is possible to exchange RSF-formatted data with several other popular data formats.

\subsection{Compatibility with SEPlib}

RSF is mostly compatible with its predecessor, the SEPlib file format.
However, there are several important differences:
\begin{enumerate}
\item SEPlib program typically use the element size (\texttt{esize=}
parameter) to distinguish between different data types:
\texttt{esize=4} corresponds to floating point data, while
\texttt{esize=8} corresponds to complex data. The typical type
handling mechanism in RSF is different: RSF looks at
\texttt{data\_format=} to determine the data type.
\item The default data form in SEPlib programs is
typically XDR and not native as it is in RSF. 
\item It is possible to pipe the
output of RSF programs to SEPlib:
\begin{verbatim}
bash$ sfspike n1=1 | Attr want=min
minimum value = 1 at 1
\end{verbatim}
However, piping the output of SEPlib programs to RSF (or, for that matter, any
other non-SEPlib programs) will result in an unterminated process. Do not try
\begin{verbatim}
bash$ Spike n1=1 | sfattr want=ming
\end{verbatim}
That happens because SEPlib uses sockets for piping and expects a socket
connection from the receiving program. RSF passes data through regular Unix
pipes.
\item SEP3D is an extension of SEPlib for operating with irregularly sampled
  data \cite[]{Biondi.sep.92.343}. There is no equivalent of it in RSF for
  the reasons explained in the beginning of this chapter. Operations with
  irregular datasets are supported through the use of auxiliary input files
  that represent the geometry information.
\end{enumerate}

\subsection{Reading and writing SEG-Y and SU files}

The SEG-Y format is based on the proposal of \cite{GEO40-02-03440352}.
It was revised in 2002\footnote{See \url{http://seg.org/publications/tech-stand/seg_y_rev1.pdf}.}. The
SU format is a modification of SEG-Y used in Seismic Unix
\cite[]{TLE16-07-10451049}.

To convert files from SEG-Y or SU format to RSF, use the \texttt{sfsegyread}
program. Let us first manufacture an example file using SU utilities
\cite[]{su}:
\begin{verbatim}
bash$ suplane > plane.su
bash$ segyhdrs < plane.su | segywrite tape=plane.segy
\end{verbatim}
To convert it to RSF, use either
\begin{verbatim}
bash$ sfsuread < plane.su tfile=tfile.rsf endian=0 > plane.rsf
\end{verbatim}
or
\begin{verbatim}
bash$ sfsegyread < plane.segy tfile=tfile.rsf \
hfile=file.asc bfile=file.bin > plane.rsf
\end{verbatim}
The endian flag is needed if the SU file originated from a little-endian
machine such as Linux PC.

Several files are generated. The standard output contains an RSF file with the
data (32 traces with 64 samples each):
\begin{verbatim}
bash$ sfin plane.rsf
plane.rsf:
    in="/tmp/plane.rsf@"
    esize=4 type=float form=native
    n1=64          d1=0.004       o1=0
    n2=32          d2=?           o2=?
        2048 elements 8192 bytes
\end{verbatim}
The contents of this file are displayed in Figure~\ref{fig:plane}.
The \texttt{tfile} is an RSF integer-type file with the trace headers (32
headers with 71 traces each):
\begin{verbatim}
bash$ sfin tfile.rsf
tfile.rsf:
    in="/tmp/tfile.rsf@"
    esize=4 type=int form=native
    n1=71          d1=?           o1=?
    n2=32          d2=?           o2=?
        2272 elements 9088 bytes
\end{verbatim}
The contents of trace headers can be quickly examined with the 
\texttt{sfheaderattr} program.
The \texttt{hfile} is the ASCII header file for the whole record.
\begin{verbatim}
bash$ head -c 242 file.asc
C      This tape was made at the
C                                                                              
C      Center for Wave Phenomena                         
\end{verbatim}
The  \texttt{bfile} is the binary header file.

\plot{plane}{width=5in}{The output of \texttt{suplane}, converted to RSF and
  displayed with \texttt{sfwiggle}.}

To convert files back from RSF to SEG-Y or SU, use the \texttt{sfsegywrite}
program and reverse the input and output:
\begin{verbatim}
bash$ sfsuwrite > plane.su tfile=tfile.rsf endian=0 < plane.rsf
\end{verbatim}
or
\begin{verbatim}
bash$ sfsegywrite > plane.segy tfile=tfile.rsf \
hfile=hfile bfile=bfile < plane.rsf
\end{verbatim}

If \texttt{hfile=} and \texttt{bfile=} are not supplied to
\texttt{sfsegywrite}, the corresponding headers will be generated on
the fly. The trace header file can be generated with
\texttt{sfsegyheader}. Here is an example:
\begin{verbatim}
bash$ sfheadermath < plane.rsf output=N+1 | sfdd type=int > tracl.rsf
bash$ sfsegyheader < plane.rsf tracl=tracl.rsf > tfile.rsf
bash$ sfsegywrite  < plane.rsf tfile=tfile.rsf > plane.segy
\end{verbatim}

\subsubsection{Unusual trace header keys}

Sometimes, SEG-Y files deviate from the standard by creating additional
trace header keys. If, for example, you find out that the SEG-Y file
contains an additional trace header key stored in bytes 225-226, you
can either remap one of the standard two-byte keys
\begin{verbatim}
bash$ sfsegyread < file.segy tfile=tfile.rsf gut=224 > file.rsf
\end{verbatim}
or create a new key
\begin{verbatim}
bash$ sfsegyread < file.segy tfile=tfile.rsf \
key1=mykey key1_len=2 mykey=224 > file.rsf
\end{verbatim}
Any number of additional keys can be created this way.

\subsection{Reading and writing ASCII files}

Reading and writing ASCII files can be accomplished with the \texttt{sfdd}
program. For example, let us take an ASCII file with numbers
\begin{verbatim}
bash$ cat file.asc
1.0 1.5 3.0
4.8 9.1 7.3
\end{verbatim}
Converting it to RSF is as simple as
\begin{verbatim}
bash$ echo in=file.asc n1=3 n2=2 data_format=ascii_float > file.rsf
bash$ sfin file.rsf
file.rsf:
    in="file.asc"
    esize=0 type=float form=ascii
    n1=3           d1=?           o1=?
    n2=2           d2=?           o2=?
        6 elements
\end{verbatim}
For more efficient input/output operations, it might be advantageous to
convert the data type to native binary, as follows:
\begin{verbatim}
bash$ echo in=file.asc n1=3 n2=2 data_format=ascii_float | \
sfdd form=native > file.rsf
bash$ sfin file.rsf
file.rsf:
    in="/tmp/file.rsf@"
    esize=4 type=float form=native
    n1=3           d1=?           o1=?
    n2=2           d2=?           o2=?
        6 elements 24 bytes
\end{verbatim}

Converting from RSF to ASCII is equally simple:
\begin{verbatim}
bash$ sfdd form=ascii out=file.asc < file.rsf > /dev/null
bash$ cat file.asc
1 1.5 3 4.8 9.1 7.3
\end{verbatim}
You can use the \texttt{line=} and \texttt{format=} parameters in
\texttt{sfdd} to control the ASCII formatting:
\begin{verbatim}
bash$ sfdd form=ascii out=file.asc \
line=3 format="%3.1f " < file.rsf > /dev/null
bash$ cat file.asc
1.0 1.5 3.0
4.8 9.1 7.3
\end{verbatim}
An alternative is to use \texttt{sfdisfil}.
\begin{verbatim}
bash$ sfdisfil > file.asc col=3 format="%3.1f " number=n < file.rsf
bash$ cat file.asc
1.0 1.5 3.0
4.8 9.1 7.3
\end{verbatim}

\bibliographystyle{seg}
\bibliography{format,SEG,SEP2}
