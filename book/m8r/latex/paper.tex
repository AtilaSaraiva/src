\chapter{Writing reproducible papers, books, and reports}

Chapter~\ref{chapter:scons} describes some of the SCons tools used by
Madagascar users to write papers, books, and reports with reproducible
examples. In this chapter, we will take a closer look at an example paper.

\section{Overview of \LaTeX and SEG\TeX}

\LaTeX is a document markup language developed originally by Leslie
Lamport in early 1980s \cite[]{latex} as an extension of Donuld
Knuth's \TeX system \cite[]{tex}. \LaTeX is widely used in different
fields of science and engineering.

\LaTeX files are text files...

SEG\TeX is a collection of \LaTeX macros developed for geophysical
publications, such as \emph{Geophysics} articles, SEG (Society of
Exploration Geophysicists) Annual Meeting abstracts, etc. SEG\TeX is
an open-source project maintained by a group of volunteers. It
integrates well with Madagascar to support the practice of
\emph{reproducible research}, as formulated by Jon Claerbout: ...

\LaTeX processing scripts included with Madagascar can turn a single
document into multiple outputs: paper in a journal, chapter in a book,
an expanded abstract for a conference, etc. It can also produce
multiple output formats, namely PDF and HTML.

\section{Example paper}

\subsection{Multiple outputs}

\section{Including papers into books and reports}

\section{Acknowledgments}

The \LaTeX-processing scripts in Madagascar follow a system developed
earlier at SEP (Stanford Exploration Project) for implementing Jon
Claerbout's vision of reproducible research \cite[]{sep}.

\bibliographystyle{seg}
\bibliography{latex}
