\chapter{Madagascar community}

As an open-source project, Madagascar not only provides you with
software tools for accomplishing varous tasks, it also connects you
with a worldwide community of users and developers. There are multiple
ways to get engaged with the community to receive help and exchange
ideas. Because Madagascar development is a community effort, this
ability is crucial.

\section{Online tools}

Modern online tools conviently connect users across the globe and
allow them to share ideas, receive help from the community, and
provide help to others. 

\subsection{Wiki pages}

The Madagascar web site\footnote{\url{http://ahay.org/}} is maintaned
using wiki technology and the Mediawiki
platform\footnote{\url{https://www.mediawiki.org/wiki/MediaWiki}},
similar to the one used by Wikipedia. 

Any Madagascar user can open an account on the platform and edit the
content of the web pages. The possibility to add translations into
multiple languages is also available.

\subsection{GitHub}

The development version of the Madagascar source code is openly
available on GitHub\footnote{\url{https://github.com/ahay}}. GitHub is currently
the largest provider of Internet hosting and version control for
open-source sotfware projects.

Version control is important both for keeping track of changes so that
new introduced bugs can be quickly identified and fixed and for
allowing multiple people to collaborate on the same code.

In its history, Madagascar went through different version control
systems: CVS, Subversion, and now Git. The development version was
initially hosted by
SourceForge\footnote{\url{https://sourceforge.net/projects/rsf/}} but
switched to GitHub, once it became a clear winner in
popularity\footnote{\url{http://ahay.org/blog/2015/07/28/hello-github/}}. We
continue to use SourceForge for distributing the stable version of the
package,

In addition to hosting the development version, GitHub provides
convenient tools for Madagascar developers to discuss changes in the
code and for users to engage with the developers by submitting bug
reports and feature requests. 

\subsection{Mailing lists}

Mailing lists are a traditional way to connect the community and allow
users and developers to communicate. Madagascar maintains two mailing
lists: for
developers\footnote{\url{https://lists.sourceforge.net/lists/listinfo/rsf-devel}}
and for
users\footnote{\url{https://lists.sourceforge.net/lists/listinfo/rsf-user}}. New
members of the community can get quick help by asking questions on the
mailing lists.

\subsection{LinkedIn}

A LinkedIn
group\footnote{\url{https://www.linkedin.com/groups/1847746/}},
Madagascar Users, was created in 2009. At the time of this writing, it
connects nearly 700 members.

\subsection{Development blog}

One of the earliest tools in Madagascar development was a blog...

\subsection{Twitter}

\subsection{Slack}

\section{Schools and Working Workshops}

\section{Contributing to Madagascar}

\subsection{GNU GPL license}

\subsection{Continuous integration}
