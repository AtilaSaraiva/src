\chapter{Introduction}

Welcome to the world of Madagascar, the open-source package for
multidimensional data analysis! This book...

\section{What is Madagascar?}

The Madagascar software package is designed for analysis of
large-scale multidimensional data, such as those occurring in
exploration geophysics. Madagascar provides a framework for
reproducible research. By “reproducible research” we
refer to the discipline of attaching software codes and data to
computational results reported in publications. The package contains a
collection of (a) computational modules, (b) data-processing scripts,
and (c) research papers. Madagascar is distributed on SourceForge
under a GPL v2 license https://sourceforge.net/projects/rsf/. By
October 2013, more than 70 people from different organizations around
the world have contributed to the project, with increasing
year-to-year activity. The Madagascar website is http://www.ahay.org/.

Reproducible research, as defined by Jon Claerbout [1], refers to the
discipline of attaching software code and data to scientific
publications, in order to enable independent verification and
replication of computational experiments. The so-called
``Claerbout's principle'' [2,3] states that
``An article about computational science in a scientific
publication is not the scholarship itself, it is merely advertising of
the scholarship. The actual scholarship is the complete software
development environment and the complete set of instructions which
generated the figures.'' The Madagascar software package
implements a computational environment that is designed both for
conducting computational experiments in the area of large-scale
geophysical data analysis and for attaching links to software code and
data in scientific publications in order to enable reproducible
research. As of October 2013, Madagascar includes more than 120
scientific papers and book chapters complete with software codes
necessary for independent verification and replication of
computational results (see
$http://www.ahay.org/wiki/Reproducible_Documents$).  The work on the
Madagascar project started in 2003, and the beta version of the
package was publicly released in June 2006. Since then, many people
have joined the project and contributed to the code. The 1.0 version
was released in 2010 and tested by an open community. The community
stays in touch using mailing lists, social networks, and annual
meetings.  Although the main applications have focused so far on
applied geophysics and exploration seismology in particular, the core
package is suitable for other scientific fields that require
reproducible analysis of large-scale multidimensional data.

\section{Madagascar design principles}

The design of Madagascar follows the Unix principle: ``Write
programs that do one thing and do it well. Write programs to work
together. Write programs to handle text streams, because that is a
universal interface.'' [4] Analysis of complex multidimensional
data, such as those occurring in exploration seismology requires
multiple steps. In addition, the data size can be too large for
storing data objects in memory (a typical modern seismic survey
generates terabytes of data). We break the data analysis chain into
multiple steps by writing short programs that implement individual
steps (``do one thing and do it well'') with control
parameters specified on the Unix command line. The programs act as
filters (``work together'') by taking input from a file
on disk or from a Unix pipe and writing either to disk or to another
pipe. We adopt a universal data format, called RSF (regularly sampled
file). The RSF format is based on a text description (``because
that is a universal interface'') that points to the raw binary
data stored in a separate file. Conceptually, an RSF file represents a
regularly sampled multi-dimensional hypercube, while the corresponding
binary data are stored (or passed through a Unix pipe) in simple
contiguous arrays for optimally efficient input/output operations [5].
To assemble data analysis workflows from individual programs, we have
adopted SCons, a Python-based makelike utility [6]. SCons
configuration files (SConstruct scripts) are written in Python and
specify the database of dependencies between input files, programs,
and target files. SCons supports other useful features, such as
multithreaded execution. In our extension of SCons, we define four
specific commands for establishing data-processing dependencies [7]:

``Fetch'' escribes a rule for downloading data
files from a remote data server or a local data directory. 
 
``Flow'' describes a rule (command or Unix pipeline) for
generating one or more target files from one or more (or none) source
files.
 
``Plot'' is similar to ``Flow'' but the target file is a figure.  

``Result'' is similar to ``Plot'' but the
target file is a final ``result'' figure for inclusion
in a publication.  

One can think of the Madagascar environment as existing on three
different levels that correspond to three different stages of research
activities of a computational scientist:

One can think of the Madagascar environment as existing on three
different levels that correspond to three different stages of research
activities of a computational scientist:

1. Implementing a new computational algorithm for data analysis. This
level involves writing low-level programs (command-line modules).

2. Testing a new algorithm or a new workflow by applying them to
data. This level involves assembling workflows from existing
command-line modules and tuning their parameters through repeated
computational experiments to achieve the desired result.

3. Publishing new results. Results from computational experiments
(figures in our case) get referenced in papers and included in
publications.

We adopt SCons for the third level as well, to simplify creation of
documents that include results from the second level. Customized SCons
commands create documents from LaTeX sources with output either in PDF
or HTML format. The HTML format is produced using LaTeX-2HTML [8]. In
the HTML version, reproducible figures are followed by links to
SConstruct scripts from level 2 and low-level programs from level 1 in
order to let the reader verify the details of the computational
experiment and reproduce it.

\inputdir{figures}
   \plot{architecture}{width=0.9\textwidth}{Madagascar Software Architecture.}

\section{History of the project}

\section{Book conventions}

\subsection{Note on reproducibility}

