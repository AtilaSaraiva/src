\def\pcscwp{
Center for Wave Phenomena \\ 
Colorado School of Mines \\ 
psava@mines.edu
}

\def\pcscover{
\author[]{Paul Sava}
\institute{\pcscwp}
\date{}
\logo{WSI}
\large
}

\def\WSI{\textbf{WSI}~}

% ------------------------------------------------------------
% colors
\def\red#1{\textcolor{red}{#1}}
\def\green#1{\textcolor{green}{#1}}
\def\blue#1{\textcolor{blue}{#1}}
\def\yellow#1{\textcolor{yellow}{#1}}
\def\magenta#1{\textcolor{magenta}{#1}}

\def\black#1{\textcolor{black}{#1}}
\def\white#1{\textcolor{white}{#1}}
\def\gray#1{\textcolor{gray}{#1}}

\definecolor{DarkGreen}{rgb}{0,0.5,0}
\definecolor{DarkBlue}{rgb}{0,0,0.5}
\definecolor{DarkRed}{rgb}{0.5,0,0}
\definecolor{DarkYellow}{rgb}{0.5,0,0}
\definecolor{LightRed}{rgb}{1.000,0.752,0.796}
\definecolor{LightGreen}{rgb}{0.792,1.000,0.439}
\definecolor{LightBlue}{rgb}{0.690,0.886,1.000}
\definecolor{LightYellow}{rgb}{1.000,0.925,0.545}
\definecolor{DarkGray}{rgb}{0.45,0.45,0.45}
\definecolor{LightGray}{rgb}{0.90,0.90,0.90}

\def\darkgreen#1{\textcolor{DarkGreen}{#1}}
\def\darkblue#1 {\textcolor{DarkBlue}{#1}}
\def\darkred#1  {\textcolor{DarkRed}{#1}}
\def\lightred#1 {\textcolor{LightRed}{#1}}

\def\lightgray#1{\textcolor{LightGray}{#1}}
\def\darkgrey#1 {\textcolor{DarkGray}{#1}}

% ------------------------------------------------------------
% madagascar
\def\mg{\darkgreen{\sc madagascar~}}
\def\latex{\darkgreen{\sc \LaTeX}}
\def\mex#1{ \normalsize \red{ \texttt{#1} } \large }
\def\mvbt#1{\small{\blue{\begin{semiverbatim}#1\end{semiverbatim}}}}

% ------------------------------------------------------------
% equations
\def\bea{\begin{eqnarray}}
\def\eea{  \end{eqnarray}}

\def\beq{\begin{equation}}
\def\eeq{  \end{equation}}

%\def\req#1{(\ref{#1})}

\def\lp{\left (}
\def\rp{\right)}

\def\lb{\left [}
\def\rb{\right]}

\def\pbox#1{ \fbox {$ \displaystyle #1 $}}

\def\non{\nonumber \\ \nonumber}

% ------------------------------------------------------------
% REFERENCE (equations and figures)
\def\rEq#1{Equation~\ref{eqn:#1}}
\def\req#1{equation~\ref{eqn:#1}}
\def\rEqs#1{Equations~\ref{eqn:#1}}
\def\reqs#1{equations~\ref{eqn:#1}}
\def\ren#1{\ref{eqn:#1}}

\def\rFg#1{Figure~\ref{fig:#1}}
\def\rfg#1{Figure~\ref{fig:#1}}
\def\rFgs#1{Figures~\ref{fig:#1}}
\def\rfgs#1{Figures~\ref{fig:#1}}
\def\rfn#1{\ref{fig:#1}}

% ------------------------------------------------------------
% field operators

% trace
\def\tr{\texttt{tr}\;}

% divergence
\def\DIV#1{\nabla \cdot {#1}}

% curl
\def\CURL#1{\nabla \times {#1}}

% gradient
\def\GRAD#1{\nabla {#1}}

% Laplacian
\def\LAPL#1{\nabla^2 {#1}}

\def\dellin{
\lb
\begin{matrix}
\done{}{x} \; \done{}{y} \; \done{}{z}
\end{matrix}
\rb
}

\def\delcol{
\lb
\begin{matrix}
\done{}{x} \non
\done{}{y} \non
\done{}{z}
\end{matrix}
\rb
}

\def\aveclin{
\lb
\begin{matrix}
a_x \; a_y \; a_z
\end{matrix}
\rb
}


% ------------------------------------------------------------

% elastic tensor
\def\CC{{\bf C}}

% identity tensor
\def\I{\;{\bf I}}

% particle displacement vector
\def\uu{{\bf u}}

% particle velocity vector
\def\vv{{\bf v}}

% particle acceleration vector
\def\aa{{\bf a}}

% force vector
\def\ff{{\bf f}}

% wavenumber vector
\def\kk{{\bf k}}

% ray parameter vector
\def\pp{{\bf p}}

% distance vector
\def\xx{{\bf x}}
\def\kkx{{\kk_\xx}}
\def\ppx{{\pp_\xx}}

\def\yy{{\bf y}}

% normal vector
\def\nn{{\bf n}}
\def\ns{\nn_s}
\def\nr{\nn_r}

% source vector
\def\ss{{\bf s}}
\def\kks{{\kk_\ss}}
\def\pps{{\pp_\ss}}

% receiver vector
\def\rr{{\bf r}}
\def\kkr{{\kk_\rr}}
\def\ppr{{\pp_\rr}}

% midpoint vector
\def\mm{{\bf m}}
\def\kkm{{\kk_\mm}}
\def\ppm{{\pp_\mm}}

% offset vector
\def\ho{{\bf h}}
\def\kkh{{\kk_\ho}}
\def\pph{{\pp_\ho}}

% space-lag vector
\def\hh{ {\boldsymbol{\lambda}} }
\def\kkl{{\kk_\hh}}
\def\ppl{{\pp_\hh}}

% CIP vector
\def\cc{ {\bf c}}

% time-lag scalar
\def\tt{\tau}
\def\tts{\tt_s}
\def\ttr{\tt_r}

% frequency
\def\ww{\omega}

%
\def\dd{{\bf d}}

\def\bb{{\bf b}}
\def\qq{{\bf q}}

\def\ii{{\bf i}} % unit vector
\def\jj{{\bf j}} % unit vector

\def\lo{{\bf l}}

% ------------------------------------------------------------

\def\Fop#1{\mathcal{F}     \lb #1 \rb}
\def\Fin#1{\mathcal{F}^{-1}\lb #1 \rb}

% ------------------------------------------------------------
% partial derivatives

\def\dtwo#1#2{\frac{\partial^2 #1}{\partial #2^2}}
\def\done#1#2{\frac{\partial   #1}{\partial #2  }}
\def\dthr#1#2{\frac{\partial^3 #1}{\partial #2^3}}
\def\mtwo#1#2#3{ \frac{\partial^2#1}{\partial #2 \partial#3} }

\def\larrow#1{\stackrel{#1}{\longleftarrow}}
\def\rarrow#1{\stackrel{#1}{\longrightarrow}}

% ------------------------------------------------------------
% elasticity 

\def\stress{\underline{\textbf{t}}}
\def\strain{\underline{\textbf{e}}}
\def\stiffness{\underline{\underline{\textbf{c}}}}
\def\compliance{\underline{\underline{\textbf{s}}}}

\def\GEOMlaw{
\strain = \frac{1}{2} 
\lb \GRAD{\uu} + \lp \GRAD{\uu} \rp^T \rb
}

\def\HOOKElaw{
\stress = \lambda \; tr \lp \strain \rp {\bf I} + 2 \mu \strain 
}

\def\CONSTITUTIVElaw{
\stress = \stiffness \;\strain 
}


\def\NEWTONlaw{
\rho \ddot{\uu} = \DIV{\stress}
}

\def\NAVIEReq{
\rho \ddot\uu =
\lp \lambda + 2\mu \rp \GRAD{\lp \DIV{\uu} \rp}
             - \mu     \CURL{   \CURL{\uu}}
}

% ------------------------------------------------------------

% potentials
\def\VP{\boldsymbol{\psi}}
\def\SP{\theta}

% stress tensor
\def\ssten{{\bf \sigma}}

\def\ssmat{
\lp \matrix {
 \sigma_{11} &  \sigma_{12}   &  \sigma_{13} \cr
 \sigma_{12} &  \sigma_{22}   &  \sigma_{23} \cr
 \sigma_{13} &  \sigma_{23}   &  \sigma_{33} \cr
} \rp
}

% strain tensor
\def\eeten{{\bf \epsilon}}

\def\eemat{
\lp \matrix {
 \epsilon_{11} &  \epsilon_{12}   &  \epsilon_{13} \cr
 \epsilon_{12} &  \epsilon_{22}   &  \epsilon_{23} \cr
 \epsilon_{13} &  \epsilon_{23}   &  \epsilon_{33} \cr
} \rp
}


% plane wave kernel
\def\pwker{A e^{i k \lp \nn \cdot \xx - v t \rp}}


% ------------------------------------------------------------
% details for expert audience (math, cartoons)
\def\expert{
\colorbox{red}{\textbf{\LARGE \white{!}}}
}

% ------------------------------------------------------------
% image, data, wavefields

\def\RR{R}

\def\UU{W}
\def\US{{\UU_s}}
\def\UR{{\UU_r}}

\def\DD{D}
\def\DS{{\DD_s}}
\def\DR{{\DD_r}}

\def\UUw{\UU}
\def\USw{{\UU_s}}
\def\URw{{\UU_r}}

\def\DDw{\DD}
\def\DSw{{\DD_s}}
\def\DRw{{\DD_r}}

% perturbations

\def\ds{\Delta s}
\def\dl{\Delta l}
\def\di{\Delta \RR}
\def\du{\Delta \UU}

\def\dRR{\Delta \RR}
\def\dUU{\Delta \UU}
\def\dUS{\Delta \US}
\def\dUR{\Delta \UR}

\def\dtt{\Delta \tt}
\def\dhh{\Delta \hh}

% ------------------------------------------------------------
% Green's functions

\def\GG{G}

\def\GS{{\GG_s}}
\def\GR{{\GG_r}}

% ------------------------------------------------------------
% elastic data, wavefields

\def\eRR{\textbf{\RR}}

\def\eDS{{\textbf{\DD}_s}}
\def\eDR{{\textbf{\DD}_r}}
\def\eDD{{\textbf{\DD}}}

\def\eUS{{\textbf{\UU}_s}}
\def\eUR{{\textbf{\UU}_r}}
\def\eUU{{\textbf{\UU}}}

% ------------------------------------------------------------
% sliding bar
\def\tline#1{
\put(95,-3){\small \blue{time}}
\put(-4,-1){\small \blue{0}}
\thicklines
\put( 0,0){\color{blue} \vector(1,0){100}}
\put(#1,0){\color{red}  \circle*{2}}
}

% ------------------------------------------------------------
% arrow on figure
\def\myarrow#1#2#3{
\thicklines
\put(#1,#2){\color{green} \vector(-1,-1){5}}
\put(#1,#2){\color{green} \textbf{#3}}
}

\def\bkarrow#1#2#3{
\thicklines
\put(#1,#2){\color{black} \vector(-1,-1){5}}
\put(#1,#2){\color{black} \textbf{#3}}
}

\def\anarrow#1#2#3#4{
\thicklines
\put(#1,#2){\color{#4} \vector(-1,-1){5}}
\put(#1,#2){\color{#4} \textbf{#3}}
}

\def\myvec#1#2#3#4{
\thicklines
\put(#1,#2){\color{green} \rotatebox{#4}{\vector(4,0){20}}}
\put(#1,#2){\color{green} \textbf{#3}}
}

% ------------------------------------------------------------
% circle on figure
\def\mycircle#1#2#3{
\thicklines
\put(#1,#2){\color{green} \circle{#3}}
}

% ------------------------------------------------------------
% note on figure
\def\mynote#1#2#3{
\put(#1,#2){\color{green} \textbf{#3}}
}

\def\biglabel#1#2#3{
\put(#1,#2){\Large \textbf{#3}}
}

\def\wlabel#1#2#3{ \white{ \biglabel{#1}{#2}{#3} }}
\def\klabel#1#2#3{ \black{ \biglabel{#1}{#2}{#3} }}
\def\rlabel#1#2#3{ \red{   \biglabel{#1}{#2}{#3} }}
\def\glabel#1#2#3{ \green{ \biglabel{#1}{#2}{#3} }}
\def\blabel#1#2#3{ \blue { \biglabel{#1}{#2}{#3} }}
\def\ylabel#1#2#3{ \yellow{\biglabel{#1}{#2}{#3} }}

% ------------------------------------------------------------
% centering
\def\cen#1{ \begin{center} \textbf{#1} \end{center}}
\def\cit#1{ \begin{center} \textit{#1} \end{center}}
\def\ctt#1{ \begin{center} \texttt{#1} \end{center}}

% emphasis (bold+alert)
\def\bld#1{ \textbf{\alert{#1}}}

% huge fonts
\def\big#1{\begin{center} {\LARGE \textbf{#1}} \end{center}}
\def\hug#1{\begin{center} {\Huge  \textbf{#1}} \end{center}}

% ------------------------------------------------------------
% separator
\def\sep{ \vfill \hrule \vfill}
\def\itab{ \hspace{0.5in}}
\def\nsp{\\ \vspace{0.1in}}

% ------------------------------------------------------------
% integrals

\def\tint#1{\!\!\!\int\!\! #1 dt}
\def\xint#1{\!\!\!\int\!\! #1 d\xx}
\def\wint#1{\!\!\!\int\!\! #1 d\ww}
\def\aint#1{\!\!\!\alert{\int}\!\! #1 d\alert{\xx}}

\def\esum#1{\sum\limits_{#1}}
\def\eint#1{\int\limits_{#1}}

% ------------------------------------------------------------
\def\CONJ#1{\overline{#1}}
\def\MOD#1{\left| {#1} \right|}

% ------------------------------------------------------------
% imaging components

\def\IC{\colorbox{yellow}{\textbf{I.C.}}\;}
\def\WR{\colorbox{yellow}{\textbf{W.R.}}\;}
\def\WE{\colorbox{yellow}{\textbf{W.E.}}\;}
\def\SO{\colorbox{yellow}{\textbf{SOURCE}}\;}
\def\WS{\colorbox{yellow}{\textbf{W.S.}}\;}

% ------------------------------------------------------------
% summary/take home message
\def\thm{take home message}

% ------------------------------------------------------------
\def\dx{\Delta x}
\def\dy{\Delta y}
\def\dz{\Delta z}
\def\dt{\Delta t}

\def\dhx{\Delta h_x}
\def\dhy{\Delta h_y}

\def\kz{{k_z}}
\def\kx{{k_x}}
\def\ky{{k_y}}

\def\kmx{k_{m_x}}
\def\kmy{k_{m_y}}
\def\khx{k_{h_x}}
\def\khy{k_{h_y}}

\def\why{ \alert{\widehat{{\khy}}}}
\def\whx{ \alert{\widehat{{\khx}}}}

\def\lx{{\lambda_x}}
\def\ly{{\lambda_y}}
\def\lz{{\lambda_z}}

\def\klx{k_{\lambda_x}}
\def\kly{k_{\lambda_y}}
\def\klz{k_{\lambda_z}}

\def\mx{{m_x}}
\def\my{{m_y}}
\def\mz{{m_z}}
\def\hx{{h_x}}
\def\hy{{h_y}}
\def\hz{{h_z}}

\def\sx{{s_x}}
\def\sy{{s_y}}
\def\rx{{r_x}}
\def\ry{{r_y}}

% ray parameter (absolute value)
\def\modp#1{\left| \pp_{#1} \right|}

% wavenumber
\def\modk#1{\left| \kk_{#1} \right|}

% ------------------------------------------------------------
\def\kzwk{ {\kz^{\kk}}}
\def\kzwx{ {\kz^{\xx}}}
 
\def\PSk#1{e^{\red{#1 i \kzwk \dz}}}
\def\PSx#1{e^{\red{#1 i \kzwx \dz}}}
\def\PS#1{ e^{\red{#1 i k_z   \dz}}}

\def\TT{t}
\def\TS{t_s}
\def\TR{t_r}

\def\oft{\lp t \rp}
\def\ofw{\lp \ww \rp}

\def\ofx{\lp \xx \rp}
\def\ofh{\lp \hh \rp}
\def\ofk{\lp \kk \rp}
\def\ofs{\lp \ss \rp}
\def\ofr{\lp \rr \rp}
\def\ofz{\lp   z \rp}

\def\ofxt{\lp \xx, t  \rp}
\def\ofst{\lp \ss, t  \rp}
\def\ofrt{\lp \rr, t  \rp}

\def\ofxw{\lp \xx, \ww  \rp}
\def\ofsw{\lp \ss, \ww  \rp}
\def\ofrw{\lp \rr, \ww  \rp}

\def\ofxm{\lp \xx,\hh \rp}

\def\ofxmp{\lp \xx+\hh \rp}
\def\ofxmm{\lp \xx-\hh \rp}

\def\ofmm{\lp \mm      \rp}
\def\ofmz{\lp \mm, z   \rp}
\def\ofmw{\lp \mm, \ww \rp}
\def\ofkm{\lp \kkm     \rp}

% ------------------------------------------------------------
% source/receiver data and wavefields

\def\dst{$\DS\ofst$}
\def\drt{$\DR\ofrt$}
\def\ust{$\US\ofxt$}
\def\urt{$\UR\ofxt$}

\def\dsw{$\DS\ofsw$}
\def\drw{$\DR\ofrw$}
\def\usw{$\US\ofxw$}
\def\urw{$\UR\ofxw$}

% ------------------------------------------------------------
\def\Nx{N_x}
\def\Ny{N_y}
\def\Nz{N_z}
\def\Nt{N_t}
\def\Nw{N_{\ww}}
\def\Nm{N_{\mm}}

\def\Nlx{N_{\lambda_x}}
\def\Nly{N_{\lambda_y}}
\def\Nlz{N_{\lambda_z}}
\def\Nlt{N_{\tau}}

\def\wmin{\ww_{min}}
\def\wmax{\ww_{max}}
\def\zmin{z_{min}}
\def\zmax{z_{max}}
\def\tmin{t_{min}}
\def\tmax{t_{max}}
\def\lmin{\hh_{min}}
\def\lmax{\hh_{max}}
\def\xmin{\xx_{min}}
\def\xmax{\xx_{max}}

% ------------------------------------------------------------
% course qualifiers

\def\fun{\hfill \alert{concepts}}
\def\pra{\hfill \alert{applications}}
\def\fro{\hfill \alert{frontiers}}


% ------------------------------------------------------------
% wavefield extrapolation
\def\ws{ {\ww s} }

\def\kows{\lp \frac{\kx}{\ws} \rp}

\def\kmws{\lp \frac{\modk{\mm}}{\ws} \rp}
\def\kzws{\lp \frac{\kz}       {\ws} \rp}

\def\S{\lb\frac{\modk{\mm}}{\ws  }\rb}
\def\C{\lb\frac{\modk{\mm}}{\ws_0}\rb}
\def\K{\lb\frac{\modk{\mm}}{\ww  }\rb}

\def\Cs{\lb\frac{\modk{\mm}^2}{\lp \ws_0 \rp^2}\rb}

\def\SSR#1{  \sqrt{ \lp \ww {#1} \rp^2 - \modk{\mm}^2} }

\def\SQRsum#1{\sum\limits_{n=1}^{\infty} \lp -1 \rp^n
		\displaystyle{\frac{1}{2} \choose n} #1}

\def\TSE#1#2#3#4{\sum\limits_{#4=#3}^{\infty} \lp -1 \rp^#4
		\displaystyle{#2 \choose #4} {#1}^#4}

\def\onefrac#1#2{\frac{#2^2}{a_#1+b_#1 #2^2}}
\def\SQRfrac#1{
	\sum\limits_{n=1}^{\infty}
	\onefrac{n}{#1} }

\def\dkzds { \left. \frac{d {\kz}}  {d s} \right|_{s_b} }
\def\SSX#1#2{\sqrt{ 1 - \lb \frac{\MOD{#2}}{#1} \rb^2} }
\def\SST#1#2{1 + \sum_{j=1}^N c_j \lb \frac{\MOD{#2}}{#1} \rb^{2j} }

% ------------------------------------------------------------
% acknowledgment
\def\ackcwp{\cen{the sponsors of the\\Center for Wave Phenomena\\at\\Colorado School of Mines}}

% ------------------------------------------------------------
% citation in slides
\def\talkcite#1{{\small \sc #1}}

% ------------------------------------------------------------
\def\ise{GPGN302: Introduction to EM and Seismic Exploration}
\def\inv{GPGN409: Inversion}

% ------------------------------------------------------------
\def\model{m}
\def\data {d}

\def\Lop{ {\mathbf{L}}}
\def\Sop{ {\mathbf{S}}}
\def\Eop{ {\mathbf{E}}}
\def\Iop{ {\mathbf{I}}}
\def\Aop{ {\mathbf{A}}}
\def\Pop{ {\mathbf{P}}}
\def\Fop{ {\mathbf{F}}}
\def\Kop{ {\mathbf{K}}}


% ------------------------------------------------------------
\def\mybox#1{
  \begin{center}
    \fcolorbox{black}{yellow}
    {\begin{minipage}{0.8\columnwidth} {#1} \end{minipage}}
  \end{center}
}

\def\hibox#1{
  \begin{center}
    \fcolorbox{black}{LightGreen}
    {\begin{minipage}{0.8\columnwidth} {#1} \end{minipage}}
  \end{center}
}

% ------------------------------------------------------------
% Nota Bene
\def\nbnote#1{
  \vfill
  \begin{center}
    \colorbox{LightGray}
    {\begin{minipage}{\columnwidth} {\textbf{\black{\large N.B.}} #1} \end{minipage}}
  \end{center}
}

\def\notabene#1{
  \begin{leftbar}
    {\sc Nota Bene:~} #1
  \end{leftbar}
}

\def\sidebar#1{
  \begin{leftbar}
    {#1}
  \end{leftbar}
}


\def\highlight#1{
  \begin{center}
    \colorbox{LightRed}
    {\begin{minipage}{0.95\columnwidth} {#1} \end{minipage}}
  \end{center}
}

% ------------------------------------------------------------
\def\pcsshaded#1{
  \definecolor{shadecolor}{rgb}{0.8,0.8,0.8}
  \begin{shaded} {#1} \end{shaded}
  \definecolor{shadecolor}{rgb}{1.0,1.0,1.0}
}

\def\blueshade#1{
  \definecolor{shadecolor}{rgb}{0.690,0.886,1.000}
    \begin{shaded}
      {#1}
    \end{shaded}
  \definecolor{shadecolor}{rgb}{1.0,1.0,1.0}
}

\def\grayshade#1{
  \definecolor{shadecolor}{rgb}{0.8,0.8,0.8}
  \begin{shaded}
    {#1}
  \end{shaded}
  \definecolor{shadecolor}{rgb}{1.0,1.0,1.0}
}

\def\yellowshade#1{
  \definecolor{shadecolor}{rgb}{1.0,1.0,0.0}
  \begin{shaded}
    {#1}
  \end{shaded}
  \definecolor{shadecolor}{rgb}{1.0,1.0,1.0}
}


% ------------------------------------------------------------
\def\postit#1{
  \begin{center}
    \colorbox{yellow}
    {\begin{minipage}{0.80\columnwidth} {#1} \end{minipage}} 
  \end{center}
}

% ------------------------------------------------------------
\def\graybox#1{
  \begin{center}
    \colorbox{LightGray}
    {\begin{minipage}{1.00\columnwidth} {#1} \end{minipage}}
  \end{center}
}

\def\whitebox#1{
  \begin{center}
    \colorbox{white}
    {\begin{minipage}{1.00\columnwidth} {#1} \end{minipage}}
  \end{center}
}

\def\yellowbox#1{
  \begin{center}
    \colorbox{LightYellow}
    {\begin{minipage}{1.00\columnwidth} {#1} \end{minipage}}
  \end{center}
}

\def\greenbox#1{
  \begin{center}
    \colorbox{LightGreen}
    {\begin{minipage}{1.00\columnwidth} {#1} \end{minipage}}
  \end{center}
}

\def\bluebox#1{
  \begin{center}
    \colorbox{LightBlue}
    {\begin{minipage}{1.00\columnwidth} {#1} \end{minipage}}
  \end{center}
}

\def\redbox#1{
  \begin{center}
    \colorbox{LightRed}
    {\begin{minipage}{1.00\columnwidth} {#1} \end{minipage}}
  \end{center}
}

\def\hyellow#1{ \colorbox{yellow} #1 }
\def\hgreen #1{ \colorbox{green}  #1 }

% ------------------------------------------------------------
% boxes for vectors and matrices

\def\pcsbox#1#2#3#4{
  % #1 = hmax
  % #2 = height
  % #3 = width
  % #4 = text
  \begin{picture}(#3,#1)
    \linethickness{0.5mm}
    % 
    \multiput(0,#1)(#3, 0){2}{\line(0,-1){#2}}
    \multiput(0,#1)(0,-#2){2}{\line(+1,0){#3}}
    % 
    \put(1,-10){#4}
  \end{picture}
}

% annotate block equations
\def\pcssym#1#2{
  \begin{picture}(3,#1)
    \put(1,-10){#2}
  \end{picture}
}

% block equation sign
\def\pcsops#1#2#3#4{
  \begin{picture}(#3,#1)
    \put(0,#2){#4}
  \end{picture}
}

% ------------------------------------------------------------
% two color boxes
\def\sidebyside#1#2{
  \begin{center}
    \colorbox{LightBlue}{
      \begin{minipage}{1.0\columnwidth} {#1} \end{minipage}
    }
    \colorbox{LightYellow}{
      \begin{minipage}{1.0\columnwidth} {#2} \end{minipage}
    }
  \end{center}
}

% upward pointing arrow
\def\uparrow#1#2#3{
  \thicklines
  \put(#1,#2){\color{green} \vector(0,+1){5}}
  \put(#1,#2){\color{green} \textbf{#3}}
}

% acknowledge figure source
\def\ackfig#1#2#3{\blabel{#1}{#2}{\normalsize \sc #3}}

../macro.tex

\def\xh{ {{\x}_h} }
\def\yh{ {{\y}_h} }
\def\th{ {{\t}_h} }

%\def\geosout#1{ \sout{#1} }
\def\geosout#1{}
%\def\geouline#1{ \uline{#1} }
\def\geouline#1{#1}

% ------------------------------------------------------------

\published{Journal of Petroleum Exploration and Production Technology, 1, 43-49, (2011)}
\author{Paul Sava}
\title{Micro-earthquake monitoring with sparsely-sampled data}
%\institute{Center for Wave Phenomena, 
%  Colorado School of Mines, 
%  1500 Illinois Str., Golden, CO 80401, USA}
%\date{Received: / Accepted:}
\maketitle

\begin{abstract}
  The computational tools for imaging in transversely isotropic media
  with tilted axes of symmetry (TTI) are complex and in most cases do
  not have an explicit \geouline{closed-form} \geosout{closed form}
  representation. \geouline{As discussed in this paper,} developing
  \geosout{these} \geouline{such} tools for a TTI medium with tilt
  constrained to be normal to the reflector dip (DTI) reduces their
  complexity and allows for closed-form representations. \geosout{In
    fact,} \geouline{We show that, for the homogeneous case}
  zero-offset migration in such a medium \geosout{for the homogeneous
    case is represented by} \geouline{can be performed using} an
  isotropic operator scaled by the velocity of the medium in the tilt
  direction. \geouline{We also show that,} for the nonzero-offset
  case, the reflection angle is always equal to the incidence angle,
  and thus, the velocities for the source and receiver waves at the
  reflection point are equal and explicitly dependent on the
  reflection angle. This fact allows us to develop explicit
  representations for angle decomposition as well as moveout formulas
  for analysis of extended images obtained by wave-equation
  migration. Although setting the tilt normal to the reflector dip may
  not be valid everywhere (i.e., salt flanks), it can be used in the
  process of velocity model building where such constrains are useful
  and typically used.
\end{abstract}

%\keywords{microearthquakes, imaging, interferometry}

% ------------------------------------------------------------
\section{Introduction}

%% 
 % about wavefield imaging methodology
 %%
Seismic imaging based on the single scattering assumption, also known
as Born approximation, consists of two main steps: \textit{wavefield
  reconstruction} which serves the purpose of propagating recorded
data from the acquisition surface back into the subsurface, followed
by an \textit{imaging condition} which serves the purpose of
highlighting locations where scattering occurs.
%
This framework holds both when the source of seismic waves is located
in the subsurface and the imaging target consists of locating this
source, as well as when the source of seismic waves is located on the
acquisition surface and the imaging target consists of locating the
places in the subsurface where scattering or reflection occurs. In
this paper, I concentrate on the case of imaging seismic sources
located in the subsurface, although the methodology discussed here
applies equally well for the more conventional imaging with artificial
sources.

%% 
 % micro-earthquakes location techniques
 %%
An example of seismic source located in the subsurface is represented
by micro-earthquakes triggered by natural causes or by fluid injection
during reservoir production or fracturing. One application of
micro-earthquake location is monitoring of fluid injection in brittle
reservoirs when micro-earthquake evolution in time correlates with
fluid movement in reservoir formations.  Micro-earthquakes can be
located using several methods including double-difference algorithms
\cite[]{WaldhauserEllsworth:2000}, Gaussian-beam migration
\cite[]{SEG-2004-03540357,rentsch:S33}, diffraction stacking
\cite[]{Gajewski:2007} or \geosout{reverse-time migration}
\geouline{time-reverse imaging} \cite[]{Gajewski:2005,Artman.meq}.

%% 
 % RTM for micro-earthquake location
 %%
Micro-earthquake location using \geosout{reverse-time migration}
\geouline{time-reverse imaging}, which is also the technique advocated in
this paper, follows the same general pattern mentioned in the
preceding paragraph: wavefield-reconstruction backward in time
followed by an imaging condition extracting the image, i.e. the
location of the source. The main difficulty with this procedure is
that the onset of the micro-earthquake is unknown, i.e. time $\tm=0$
is unknown, so the imaging condition cannot be simply applied as it is
usually done in zero-offset migration. Instead, an automatic search
needs to be performed in the back-propagated wavefield to identify the
locations where wavefield energy focuses. This process is difficult
and often ambiguous since false focusing locations might overlap with
locations of wavefield focusing. This is particularly true when
imaging using an approximate model which does not explain all random
fluctuations observed in the recorded data. This problem is further
complicated if the acquisition array is sparse, e.g. when receivers
are located in a borehole. In this case, the sparsity of the array
itself leads to artifacts in the reconstructed wavefield which makes
the automatic picking of focused events even harder.

% ------------------------------------------------------------
\inputdir{XFig}
% ------------------------------------------------------------
\multiplot{4}{delta0,delta1,delta2,delta3}{width=0.22\textwidth}
{Schematic representation of focus constructions using time
  reversal. Each line in the plots represents a wavefront
  reconstructed at the source from a given receiver. The panels
  represent the following cases: (a) dense acquisition, complete
  angular coverage and correct velocity, (b) dense acquisition,
  partial angular coverage and correct velocity, (c) dense
  acquisition, partial angular coverage and incorrect velocity, and
  (d) sparse acquisition, partial angular coverage and incorrect
  velocity. Panel (d) represents the worst case scenario for
  micro-earthquake imaging.}

%% 
 % the problem of sparse data acquisition
 %%
The process by which sampling artifacts are generated is explained in
\rfgs{delta0}-\rfn{delta3}. Each segment in \rfg{delta0} corresponds
to a wavefront reconstructed from a receiver. For dense, uniform
\geouline{and wide-aperture} receiver coverage and for reconstruction
using accurate velocity, the wavefronts overlap at the source
position, \rfg{delta1}. \geouline{This idealized situation resembles the
  coverage typical for medical imaging, although the physical
  processes used are different.} However, if the velocity used for
wavefield reconstruction is inaccurate, then the wavefronts do not all
overlap at the source position, \rfg{delta2}, thus leading to imaging
artifacts. Likewise, if receiver sampling is sparse, reconstruction at
the source position is incomplete, \rfg{delta3}, even if the velocity
used for reconstruction is accurate. The cartoons depicted in
\rfgs{delta0}-\rfn{delta3} represent an ideal situation with receivers
surrounding the seismic source, which is not typical for seismic
experiments. In those cases, source illumination is limited to a range
which correlates with the receiver coordinates.

In general, artifacts caused by unknown velocity fluctuations and
receiver sampling overlap and, although the two phenomena are not
equivalent, their effect on the reconstructed wavefields are
analogous. As illustrated in the following sections, the general
character of those artifacts is that of random wavefield
fluctuations. Ideally, the imaging procedure should attenuate those
random wavefield fluctuations irrespective of their cause in order to
support automatic source identification.

% ------------------------------------------------------------
\section{Conventional imaging condition}

Assuming data $\DD{\xm,\tm}$ acquired at coordinates $\xm$ function of
time $\tm$ (e.g. in a borehole) we can reconstruct the wavefield
$\VV{\xm,\ym,\tm}$ at coordinates $\ym$ in the imaging volume using an
appropriate Green's function $\GG{\xm,\ym,\tm}$ corresponding to the
locations $\xm$ and $\ym$ (\rfg{coord})
%
\beq \label{eqn:green}
\VV{\xm,\ym,\tm} = \DD{\xm,\tm} *_t \GG{\xm,\ym,\tm} \;,
\eeq
%
where the symbol $*_t$ indicates time convolution.  The total
wavefield $\UU{\ym,\tm}$ at coordinates $\ym$ due to data recorded at
all receivers located at coordinates $\xm$ is represented by the
superposition of the reconstructed wavefields $\VV{\xm,\ym,\tm}$:
%
\beq \label{eqn:linear}
\UU{\ym,\tm} = \intxm \VV{\xm,\ym,\tm} \;.
\eeq
%
A conventional imaging condition (CIC) applied to this reconstructed
wavefield extracts the image $\IA{CIC}{\ym}$ as the wavefield at time
$\tm=0$
%
\beq \label{eqn:cic}
\IA{CIC}{\ym} = \UU{\ym,\tm=0} \;.
\eeq
%
This imaging procedure succeeds if several assumptions are fulfilled:
first, the velocity model used for imaging has to be accurate; second,
the numeric solution to the wave-equation used for wavefield
reconstruction has to be accurate; third, the data need to be sampled
densely and uniformly on the acquisition surface. In this paper, I
assume that the first and third assumptions are not fulfilled. In
these cases, the imaging is not accurate because contributions to the
reconstructed wavefield from the receiver coordinates do not interfere
constructively, thus leading to imaging artifacts. As indicated
earlier, this situation is analogous to the case of imaging with an
inaccurate velocity model, e.g. imaging with a smooth velocity of data
corresponding to geology characterized by rapid velocity variations.

%% 
 % talk about interferometry as a fix to this problem
 % cite Sava and Borcea
 %%

Different image processing procedures can be employed to reduce the
random wavefield fluctuations. The procedure advocated in this paper
uses interferometry for noise cancellation. Interferometric procedures
can be formulated in various frameworks, e.g. coherent interferometric
imaging \cite[]{Borcea.GEO.2006} or wave-equation migration with an
interferometric imaging condition \cite[]{SavaPoliannikov.geo.iic}.

\sideplot{coord}{width=\textwidth}{Illustration of the variables $\xm$
  and $\ym$ used for the description of the conventional and
  interferometric imaging procedures.}

% ------------------------------------------------------------
\section{Interferometric imaging condition}

Migration with an interferometric imaging condition (IIC) uses the
same generic framework as the one used for the conventional imaging
condition, i.e. wavefield reconstruction followed by an imaging
condition.  However, the difference is that the imaging condition is
not applied to the reconstructed wavefield directly, but it is applied
to the wavefield which has been transformed using pseudo Wigner
distribution functions (WDF) \cite[]{Wigner.wdf}. By definition, the
zero frequency pseudo WDF of the reconstructed wavefield
$\UU{\ym,\tm}$ is
%
\beq \label{eqn:wdf}
\WA{}{\ym,\tm} = \intth \intyh
\UU{\ym-\frac{\yh}{2},\tm-\frac{\th}{2}}
\UU{\ym+\frac{\yh}{2},\tm+\frac{\th}{2}} \;,
\eeq
%
where $Y$ and $T$ denote averaging windows in space and time,
respectively. In general, $Y$ is three dimensional and $T$ is one
dimensional. Then, the image $\IA{IIC}{\ym}$ is obtained by extracting
the time $\tm=0$ from the pseudo WDF, $\WA{}{\ym,\tm}$, of the
wavefield $\UU{\ym,\tm}$:
%
\beq \label{eqn:iic}
\IA{IIC}{\ym} = \WA{}{\ym,\tm=0} \;.
\eeq
%
The interferometric imaging condition represented by \reqs{wdf} and
\ren{iic} effectively reduces the artifacts caused by the random
fluctuations in the wavefield by filtering out its rapidly varying
components \cite[]{SavaPoliannikov.geo.iic}. In this paper, I use this
imaging condition to attenuate noise caused by sparse data sampling or
noise caused by random velocity variations. As suggested earlier, the
interferometric imaging condition attenuates both types of noise at
once, since it does not explicitly distinguish between the various
causes of random fluctuations.

%% 
 % discussion about window shape/size
 %%

The parameters $Y$ and $T$ defining the local window of the pseudo WDF
are selected according to two criteria \cite[]{Cohen.tfa.1995}. First,
the windows have to be large enough to enclose a representative
portion of the wavefield which captures the random fluctuation of the
wavefield. Second, the window has to be small enough to limit the
possibility of cross-talk between various events present in the
wavefield. Furthermore, cross-talk can be attenuated by selecting
windows with different shapes, for example Gaussian or
exponentially-decaying.  \geouline{Therefore, we could in principle
  define the transformation in} \req{wdf} \geouline{more generally} as
%
\beq \label{eqn:wdfgeneral}
\WA{}{\ym,\tm} = 
\int d\th W_T\lp \tm,\th \rp
\int d\yh W_Y\lp \ym,\yh \rp
\UU{\ym-\frac{\yh}{2},\tm-\frac{\th}{2}}
\UU{\ym+\frac{\yh}{2},\tm+\frac{\th}{2}} \;,
\eeq
%
where $W_T$ and $W_Y$ are weighting functions which could represent
Gaussian, boxcar or any other local functions (Artman 2011, personal
communication).  For simplicity, in all examples presented in this
paper, the space and time windows are rectangular with no tapering and
the size is selected assuming that micro-earthquakes occur
sufficiently sparse, i.e. the various sources are located at least
twice as far in space and time relative to the wavenumber and
frequency of the considered seismic event. Typical window sizes used
here are $11$ grid points in space and $5$ grid points in
time. \geosout{A complete discussion of optimal window selection falls
  outside the scope of this paper.}

% ------------------------------------------------------------
\section{Example}

%% 
 % describe experiment: setup and objectives
 %%

I exemplify the interferometric imaging condition method with a
synthetic example simulating the acquisition geometry of the passive
seismic experiment performed at the San Andreas Fault Observatory at
Depth \cite[]{Chavaria:2003,VasconcelosEtAl.eos.safod}. This numeric
experiment simulates waves propagating from three micro-earthquake
sources located in the fault zone, \rfg{meqs}, which are recorded in a
deviated well located at a distance from the fault. For the imaging
procedure described in this paper, the micro-earthquakes represent the
seismic sources. This experiment uses acoustic waves, corresponding to
the situation in which we use the P-wave mode recorded by the
three-component receivers located in the borehole, \rfgs{saf1-rev3}
and \rfn{saf2-rev4}. The three sources are triggered $40$~ms apart and
the triggering time of the second source is conventionally taken to
represent the origin of the time axis.

The goal of this experiment is to locate the source positions by
focusing data recorded using dense acquisition in media with random
fluctuations or by focusing data recorded using sparse acquisition
arrays in media without random fluctuations. In the first case, the
imaging artifacts are caused by the fact that data are imaged with a
velocity model that does not incorporate all random fluctuations of
the model used for data simulation, while in the second case, the
imaging artifacts are caused by the fact that the data are sampled
sparsely in the borehole array. \geouline{The third case is a combination
of acquisition with two sparse arrays, and imaging with an inaccurate
velocity model.}

% ------------------------------------------------------------
\inputdir{XFig}
% ------------------------------------------------------------
\sideplot{meqs}{width=0.65\textwidth}{Geometry of the sources used in
  the numeric experiment. The horizontal and vertical separation
  between sources is $250$~m. The sources are triggered with $40$~ms
  delays in the order indicated by their numbers. Time $\tm=0$ is
  conventionally set to the the triggering moment of source $2$.}

% ------------------------------------------------------------
\inputdir{saf1}
% ------------------------------------------------------------
\multiplot{2}{wfl3-14,rev3}{width=\textwidth} {(a) Wavefields
  simulated in random media and (b) data acquired with a dense
  receiver array. Overlain on the model and wavefield are the
  positions of the sources and borehole receivers. The \geosout{hashed}
  \geouline{boxed} area corresponds to the images depicted in
  \rfgs{saf1-uuu3}-\rfn{saf1-wig3}.}

% ------------------------------------------------------------
\inputdir{saf2}
% ------------------------------------------------------------
\multiplot{2}{wfl4-14,rev4}{width=\textwidth} {(a) Wavefields
  simulated in smooth media and (b) data acquired with a sparse
  receiver array. Overlain on the model and wavefield are the
  positions of the sources and borehole receivers. The \geosout{hashed}
  \geouline{boxed} area corresponds to the images depicted in
  \rfgs{saf2-uuu4}-\rfn{saf2-wig4}.}

% ------------------------------------------------------------
\inputdir{saf3}
% ------------------------------------------------------------
\multiplot{2}{wfl3-14,rev3}{width=\textwidth} {\geouline{(a) Wavefields
    simulated in random media and (b) data acquired with two sparse
    receiver arrays. Overlain on the model and wavefield are the
    positions of the sources and borehole receivers. The boxed area
    corresponds to the images depicted in}
  \rfgs{saf3-uuu3}-\rfn{saf3-wig3}. \geouline{The top traces in panel (b)
    correspond to the vertical array, and the other traces correspond
    to the sparse deviated array.}}

% ------------------------------------------------------------
\inputdir{saf1}
% ------------------------------------------------------------
\multiplot{2}{uuu3,wig3}{width=\textwidth}{ Images corresponding to
  migration of the \textit{densely-sampled data} (\rfg{saf1-rev3})
  modeled in the \textit{random velocity} by (a) conventional I.C. and
  (b) interferometric I.C. using the background velocity. The
  left-most panel shows focusing at source $1$, the middle panel shows
  focusing at source $2$, and the right-most panel shows focusing at
  source $3$. The overlain dots represent the exact source
  positions. }

% ------------------------------------------------------------
\inputdir{saf2}
% ------------------------------------------------------------
\multiplot{2}{uuu4,wig4}{width=\textwidth}{ Images corresponding to
  migration of the \textit{sparsely-sampled data} (\rfg{saf2-rev4})
  modeled in the \textit{background velocity} by (a) conventional
  I.C. and (b) interferometric I.C. using the background velocity. The
  left-most panel shows focusing at source $1$, the middle panel shows
  focusing at source $2$, and the right-most panel shows focusing at
  source $3$. The overlain dots represent the exact source positions.
}

% ------------------------------------------------------------
\inputdir{saf3}
% ------------------------------------------------------------
\multiplot{2}{uuu3,wig3}{width=\textwidth}{\geouline{Images corresponding
    to migration of the dual \textit{sparsely-sampled data}}
  (\rfg{saf3-rev3}) \geouline{modeled in the \textit{random velocity} by
    (a) conventional I.C. and (b) interferometric I.C. using the
    background velocity. The left-most panel shows focusing at source
    $1$, the middle panel shows focusing at source $2$, and the
    right-most panel shows focusing at source $3$. The overlain dots
    represent the exact source positions.} }

%% 
 % show wavefield snapshots (movie) in the vicinity of t=0
 %%

\rFgs{saf1-uuu3}-\rfn{saf1-wig3}, \rfn{saf2-uuu4}-\rfn{saf2-wig4}
\geouline{and} \rfn{saf3-uuu3}-\rfn{saf3-wig3} show the wavefields
reconstructed in reverse time around the target location. From left to
right, the panels represent the wavefield at different times. As
indicated earlier, the time at which source $2$ focuses is selected as
time $\tm=0$, although this convention is not relevant for the
experiment and any other time could be selected as reference. The
experiment depicted in \rFgs{saf1-uuu3}-\rfn{saf1-wig3} corresponds to
modeling in a model with random fluctuations and migration in a smooth
background model. In this experiment, the data used for imaging are
densely-sampled in the borehole, \geouline{i.e. there are $81$ receivers
  separated by approximately $12$~m}. In contrast, the experiment
depicted in \rfn{saf2-uuu4}-\rfn{saf2-wig4} corresponds to modeling
and migration in the smooth background model. In this experiment, the
data are sparsely-sampled in the borehole, \geouline{i.e. there are only
  $6$ receivers obtained by selecting every $16$th receiver from the
  original set}.  In all cases, panels (a) correspond to imaging with
a conventional imaging condition, i.e. simply select the reconstructed
wavefield at various times, and panels (b) correspond to imaging with
the interferometric imaging condition, i.e. select various times from
the wavefield transformed with a pseudo-WDF of $11$ grid points in
space and $5$ grid points in time. For this example, WDF window
corresponds to $44$~m in space and $2$~ms in time.

\rFg{saf1-uuu3} shows significant random fluctuations caused by
wavefield reconstruction using an inaccurate velocity model. The
fluctuations caused by the random velocity and encoded in the recorded
data are not corrected during wavefield reconstruction and they remain
present in the model. Likewise, \rfg{saf2-uuu4} shows significant
random fluctuations caused by reconstruction using the sparse borehole
data. However, the \geosout{zero-frequency} pseudo WDF applied to the
reconstructed wavefields attenuates the rapid wavefield fluctuations
and leads to sparser, better focused images that are easier to use for
source location. This conclusion applies equally well for the
experiments depicted in \rfgs{saf1-uuu3}-\rfn{saf1-wig3} or
\rfn{saf2-uuu4}-\rfn{saf2-wig4}.

\geouline{The final example corresponds to the case of acquisition with
  two separate sparse arrays, \rfgs{saf3-wfl3-14}-\rfn{saf3-rev3}. As
  expected, the wavefields are far less noisy after the application of
  the WDF, and the focusing is increased due to the larger array
  aperture. This facilitates an automatic procedure for focusing
  identification, since most of the spurious noisy is eliminated from
  the image.  }

%% 
 % discussion about 3D imaging from a 1D acquisition
 %%

Finally, I note that the 2D imaging results from this example show
better focusing than what would be expected in 3D. This is simply
because the 1D acquisition in the borehole cannot constrain the 3D
location of the micro-earthquakes, i.e. the azimuthal resolution is
poor, especially if scatterers are not present in the model used for
imaging. This situation can be improved by using data acquired in
several boreholes or by using additional information extracted from
the wavefields, e.g. polarization of multicomponent data.

% ------------------------------------------------------------
\section{Conclusions}

The interferometric imaging condition used in conjunction with
\geosout{reverse-time migration} \geouline{time-reverse imaging} reduces the
artifacts caused by random velocity fluctuations that are
unaccounted-for in imaging and by the sparse wavefield sampling on the
acquisition array. The images produced by this procedure are crisper
and support automatic picking of micro-earthquake
locations. \geouline{Imaging with sparse arrays allows increased aperture
  for identical acquisition cost with that of a narrower but denser
  array. At the same time, a larger aperture improves focusing of the
  events, thus facilitating automatic event identification.} The
interferometric imaging procedure has a similar structure to
conventional imaging and the moderate cost increase is proportional to
the size of the windows used by the pseudo Wigner distribution
functions. The source positions obtained using this procedure can be
used to monitor fluid injection or for studies of naturally occurring
earthquakes in fault zones.

% ------------------------------------------------------------
\section{Acknowledgment}
This work is supported by the sponsors of the Center for Wave
Phenomena at Colorado School of Mines and by a research grant from
ExxonMobil.
%
The reproducible numeric examples in this paper use the Madagascar
open-source software package freely available from
\url{http://www.reproducibility.org}.



\bibliographystyle{seg}
\bibliography{SEG2005,PCS,MISC,BOOKS}
