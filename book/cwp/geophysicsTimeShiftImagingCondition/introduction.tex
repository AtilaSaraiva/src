\section{Introduction}
A key challenge for imaging in complex areas is accurate
determination of a velocity model in the area under investigation. 
Migration velocity analysis is based on the principle that
image accuracy indicators are optimized when
data are correctly imaged. 
A common procedure for velocity analysis is to examine the
alignment of images created with multi-offset data.
An optimal choice of image analysis can be done in the 
angle domain which is free of some complicated
artifacts present in surface offset gathers in complex areas
\cite[]{GEO69-02-05620575}.
\par
Migration velocity analysis after migration by wavefield
extrapolation requires image decomposition
in scattering angles relative
to reflector normals. Several methods have 
been proposed for such decompositions
\cite[]{GEO55-09-12231234,
SEG-1999-08240827,
SEG-2000-08300833,
GEO67-03-08830889,
XieWu.adcig,
GEO68-03-10651074,
SEG-2003-08890892,
Fomel.seg.3dadcig,
GEO69-05-12831298}.
These procedures require decomposition of 
extrapolated wavefields in variables that are
related to the reflection angle.

A key component of such image decompositions 
is the imaging condition.
A careful implementation of the imaging condition preserves
all information necessary to decompose images in their
angle-dependent components. 
The challenge is efficient and reliable construction of these
angle-dependent images for velocity or amplitude analysis.

In migration with wavefield extrapolation, a prestack imaging condition
based on spatial shifts of the source and receiver wavefields allows
for angle-decomposition
\cite[]{GEO67-03-08830889,SavaFomel.pag}.
Such formed angle-gathers describe reflectivity as a
function of reflection angles and are powerful tools for migration
velocity analysis (MVA) or amplitude versus angle analysis (AVA).
However, due to the large expense of space-time 
cross-correlations, especially in three dimensions, 
this imaging methodology is not used routinely in data processing.

This paper presents a different form of imaging condition.
The key idea of this new method is to use
time-shifts instead of space-shifts 
between wavefields computed from sources and receivers.
Similarly to the space-shift imaging condition, an image is built by
space-time cross-correlations of subsurface wavefields,
and multiple lags of the time cross-correlation are preserved in the image.
Time-shifts have physical meaning that can be related directly
to reflection geometry, similarly to the procedure used for
space-shifts. 
Furthermore, time-shift imaging is cheaper to apply than space-shift 
imaging, and thus it might alleviate some of the difficulties
posed by costly cross-correlations in 3D space-shift imaging condition.

The idea of a time-shift imaging condition is related to the idea of
depth focusing analysis 
\cite[]{SEG-1986-S7.6,
GEO57-12-16081622,GEO58-08-11481156,
SEG-1995-0465,SEG-1996-0463}.
The main novelty of our approach is that we employ time-shifting 
to construct angle-domain gathers for prestack depth imaging.

The time-shift imaging concept is applicable to Kirchhoff 
migration, migration by wavefield extrapolation, or 
reverse-time migration.
We present a theoretical analysis of this new imaging condition,
followed by a physical interpretation leading to angle-decomposition.
Finally, we illustrate the method with images of the complex Sigsbee 2A
dataset \cite[]{SEG-2002-21222125}.

