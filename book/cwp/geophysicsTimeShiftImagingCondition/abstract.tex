\begin{abstract}
Seismic imaging based on single-scattering approximation 
is based on analysis of the match between the 
source and receiver wavefields at every image location.
Wavefields at depth are functions of space and time and are
reconstructed from surface data either by integral 
methods (Kirchhoff migration) or by differential methods
(reverse-time or wavefield extrapolation migration).
Different methods can be used to analyze wavefield matching,
of which cross-correlation is a popular option.
Implementation of 
a simple imaging condition requires time cross-correlation 
of source and receiver wavefields, followed by extraction
of the zero time lag.
A generalized imaging condition operates by cross-correlation
in both space and time, followed by image extraction at
zero time lag. 
Images at different spatial cross-correlation lags are indicators of
imaging accuracy and are also used for image angle-decomposition.

In this paper, we introduce an alternative prestack imaging condition
in which we preserve multiple lags of the time cross-correlation.
Prestack images are described as functions of 
time-shifts as opposed to space-shifts
between source and receiver wavefields.
This imaging condition is applicable to migration by
Kirchhoff, wavefield extrapolation or reverse-time techniques. 
The transformation allows construction of
common-image gathers presented as function of either
time-shift or reflection angle at every location in space.
Inaccurate migration velocity is revealed by angle-domain
common-image gathers with non-flat events.
Computational experiments using a synthetic dataset
from a complex salt model
demonstrate the main features of the method.
\end{abstract}

