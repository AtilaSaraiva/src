\section{VERTICAL HETEROGENEITY}
%%%%%%%%%%%%%%%%%%%%%%%%%%%%%%%
Vertical heterogeneity is another reason for nonhyperbolic
moveout. We start this section by reviewing well-known results for
isotropic media. Although these results can be interpreted
in terms of an effective anisotropy, we show that it has different properties
than those for the VTI model. We then extend the theory to vertically heterogeneous
VTI media and perform a comparative analysis of various three-parameter 
nonhyperbolic approximations.

\inputdir{Sage}

\subsection{Vertically heterogeneous isotropic model} 
%%%%%%%%%%%%%%%%%%%%%%
Nonhyperbolicity of reflection moveout in vertically heterogeneous
isotropic media has been extensively studied using the
Taylor series expansion in the powers of the offset
\cite[]{der,taner,GPR21-04-07830795}. The most important property of
vertically heterogeneous media is that the ray parameter 
\[
p \equiv {{\sin{\psi}(z)} \over {V_z(z)}}
\]
does not change along
any given ray (Snell's law). This fact leads to the explicit parametric
relationships 
\begin{eqnarray}
t(p) & = & 2\,\int_{0}^{z}\,{{dz} \over {V_z(z)\,\cos{\psi(z)}}} =
\int_{0}^{t_z}\,{{dt_z} \over {\sqrt{1 - p^2\,V_z^2(t_z)}}}\;,
\label{eqn:tp} \\ 
l(p) & = & 2\,\int_{0}^{z}\,{{dz}\,\tan{\psi(z)}} =
\int_{0}^{t_z}\,{{p\,V_z^2(t_z)\,dt_z} \over 
{\sqrt{1 - p^2\,V_z^2(t_z)}}}\;,
\label{eqn:hp}
\end{eqnarray}
where
\begin{equation}
t_z = t(0) = 2\,\int_{0}^{z}\,{{dz} \over {V_z(z)}}\;.
\label{eqn:tz}
\end{equation}
Straightforward differentiation of parametric equations (\ref{eqn:tp})
and (\ref{eqn:hp}) yields the first four coefficients of the Taylor
series expansion
\begin{equation}
t^2(l) = a_0 + a_1\,l^2 + a_2\,l^4 + a_3\,l^6 + \ldots
\label{eqn:taylor}
\end{equation}
in the vicinity of the vertical zero-offset ray.  Series
(\ref{eqn:taylor}) contains only even powers of the offset $l$ because
of the reciprocity principle: the pure-mode reflection traveltime is an even
function of the offset.  The Taylor series coefficients for the isotropic case
are defined as follows:
\begin{eqnarray}
a_0 & = & t_z^2\;,
\label{eqn:a0} \\
a_1 & = & {1 \over V_{rms}^2}\;,
\label{eqn:a1} \\
a_2 & = & {{1 - S_2} \over {4\,t_z^2\,V_{rms}^4}}\;,
\label{eqn:a2} \\
a_3 & = & {{2\,S_2^2 -  S_2 - S_3} \over {8\,t_z^4\,V_{rms}^6}}\;,
\label{eqn:a3}
\end{eqnarray}
where 
\begin{eqnarray}
V_{rms}^2 = M_1 \, ,
\label{eqn:a31}
\end{eqnarray}
\begin{eqnarray}
M_k & = & {1 \over t_z}\,\int_{0}^{t_z}\,V_z^{2k}(t)\,dt \qquad
(k = 1, \, 2, \, \ldots)\;,
\label{eqn:mk} \\
S_k & = & {M_k \over {V_{rms}^{2k}}} \qquad (k = 2, \, 3, \, \ldots)\;.
\label{eqn:sk}
\end{eqnarray}
Equation (\ref{eqn:a1}) shows that, at small offsets, the reflection
moveout has a hyperbolic form with the normal-moveout velocity $V_n$
equal to the root-mean-square velocity $V_{rms}$. At large offsets,
however, the hyperbolic approximation is no longer accurate.  Studying
the Taylor series expansion (\ref{eqn:taylor}), \cite{malov}
introduced a three-parameter approximation for the reflection
traveltime in vertically heterogeneous isotropic media.
%\cite{malov,Sword.sep.51.313}. 
His equation has the form of a
shifted hyperbola \cite[]{castle,nmo}:
\begin{equation}
t(l) = \left(1 - {1 \over S}\right)\,t_0 + 
{1 \over S} \sqrt{t_0^2 + S\,{l^2 \over V_n^2}}\;.
\label{eqn:malovichko}
\end{equation}
\par
If we set the zero-offset traveltime $t_0$ equal to the vertical
traveltime $t_z$, the velocity $V_n$ equal to $V_{rms}$, and the {\em
parameter of heterogeneity} $S$ equal to $S_2$, equation
(\ref{eqn:malovichko}) guarantees the correct coefficients $a_0$, $a_1$, and
$a_2$ in the Taylor series (\ref{eqn:taylor}). Note that the parameter $S_2$
is related to the variance $\sigma^2$ of the squared velocity
distribution, as follows:
\begin{equation}
\sigma^2 = M_2 - V_{rms}^4 = V_{rms}^4\,(S_2 -1)\;.
\label{eqn:sigma2}
\end{equation}
According to equation (\ref{eqn:sigma2}), this parameter is always greater
than unity (it equals 1 in homogeneous media). In many 
practical cases, the value of $S_2$ lies between~$1$ and~$2$.  We can
roughly estimate the accuracy of approximation (\ref{eqn:malovichko}) at
large offsets by comparing the fourth term of its Taylor series with
the fourth term of the exact traveltime expansion (\ref{eqn:taylor}).
According to this estimate, the error of Malovichko's approximation is
\begin{equation}
{{\Delta t^2(l)} \over t^2(0)} = {1 \over 8} (S_3-S_2^2)\,
\left({l \over {t_0\,V_n}}\right)^6\;.
\label{eqn:malerror}
\end{equation}
As follows from the definition of the parameters $S_k$ 
[equations~(\ref{eqn:sk})] and the Cauchy-Schwartz inequality, the expression
(\ref{eqn:malerror}) is always nonnegative. 
This means that the shifted-hyperbola approximation
tends to overestimate traveltimes at large offsets. As the offset
approaches infinity, the limit of this approximation is 
\begin{equation}
\lim_{l \rightarrow \infty} t^2(l) = {1 \over S}\,{l^2 \over V_n^2}\;.
\label{eqn:Mhlimit}
\end{equation}
\par
Equation (\ref{eqn:Mhlimit}) indicates that the effective horizontal
velocity for Malovichko's approximation (the slope of the shifted
hyperbola asymptote) differs from the normal-moveout velocity. One 
can interpret this difference as evidence of some \emph{effective}
depth-variant anisotropy. However, the anisotropy implied in
equation (\ref{eqn:malovichko}) differs from the true anisotropy in a 
homogeneous transversely isotropic medium [see equation (\ref{eqn:vg})]. 
To reveal this difference, let us substitute
the effective values $t(l) = {\sqrt{4\,z^2 + l^2} / {V_g(\psi)}}$,
$~t_0 = {2\,z / V_z}$, $~l = 2\,z\,\tan{\psi}$, and $S = {V_x^2 / V_n^2}$ 
into equation (\ref{eqn:malovichko}). After eliminating the
variables $z$ and $l$, the result takes the form
\begin{equation}
{1 \over {V_g(\psi)}} = {1 \over V_z}\,\left\{
\cos{\psi}\left(1 - {V_n^2 \over V_x^2}\right) +
\sqrt{{V_z^2 \over V_x^2}\,\sin^2{\psi} +
{V_n^4 \over V_x^4}\,\cos^2{\psi}}\right\}.
\label{eqn:mal2vg}
\end{equation}
If the anisotropy is induced by vertical heterogeneity, $V_x \ge V_n \ge V_z$.  
Those inequalities follow from the definitions of $V_{rms}$, $t_z$,
$S_2$, and the Cauchy-Schwartz inequality. They reduce to equalities only
when velocity is constant.  Linearizing expression
(\ref{eqn:mal2vg}) with respect to Thomsen's anisotropic parameters $\delta$
and $\epsilon$, we can transform it to the form analogous to
that of equation (\ref{eqn:vgeta}):
\begin{equation}
V_g^2(\psi) = V_z^2\,\left[\,1 + 2\,\delta\,\sin^2{\psi} + 
2\,\eta\,(1 - \cos{\psi})^2\right]\;.
\label{eqn:vgvz}
\end{equation}
Figure \ref{fig:nmofrz1,nmofrz2} illustrates the difference between the VTI
model and the effective anisotropy implied by the Malovichko approximation. 
The differences are noticeable in both the shapes of the effective wavefronts
(Figure \ref{fig:nmofrz1}) and the moveouts (Figure \ref{fig:nmofrz2}).

\multiplot{2}{nmofrz1,nmofrz2}{width=2.8in,height=1.4in}{Comparison of the
wavefronts (a) and moveouts (b) in the VTI (solid) and vertically
inhomogeneous isotropic media (dashed). The values of the effective 
vertical, horizontal, and NMO velocities are the same
in both media and correspond to Thomsen's parameters $\epsilon = 0.2$ and
$\delta = 0.1$.}
%??? -- Label the plots with a and b. 
%??? -- Remove the numbers from the b-plot.
%??? -- Replace the dash-dotted line with dashed.}   

\par
In deriving equation (\ref{eqn:vgvz}), we have assumed the correspondence
\begin{equation}
S =  {V_x^2 \over V_n^2} = 
{{1 + 2\,\epsilon} \over {1 + 2\,\delta}} \approx 1 + 2\,\eta\;.
\label{eqn:sone}
\end{equation}
We could also have chosen the value of the parameter of heterogeneity $S$ that
matches the coefficient $a_2$ given by equation (\ref{eqn:a2}) with
the corresponding term in the Taylor series (\ref{eqn:TItaylor}).Then, the value of $S$ is \cite[]{GEO62-06-18391854}
\begin{equation}
S = 1 + 8\,\eta\;.
\label{eqn:stwo}
\end{equation}
The difference between equations (\ref{eqn:sone}) and (\ref{eqn:stwo}) is an
additional indicator of the fundamental difference between homogeneous 
VTI and vertically heterogeneous isotropic media. The three-parameter 
anisotropic approximation (\ref{eqn:TIapprox}) can match the
reflection moveout in the isotropic model up to 
the fourth-order term in the Taylor series expansion if the value of
$\eta$ is chosen in accordance with equation (\ref{eqn:stwo}). We can
estimate the error of such an approximation with an equation analogous
to (\ref{eqn:malerror}): 
\begin{equation}
{{\Delta t^2(l)} \over t^2(0)} = {1 \over 8} (S_3-2 + 3\,S_2-2\,S_2^2)\,
\left({l \over {t_0\,V_n}}\right)^6\;.
\label{eqn:tsverror}
\end{equation}
The difference between the error estimates (\ref{eqn:malerror}) and
(\ref{eqn:tsverror}) is
\begin{equation}
{{\Delta t^2(l)} \over t^2(0)} = {1 \over 8} (2 - S_2)\,(S_2-1)\,
\left({l \over {t_0\,V_n}}\right)^6\;.
\label{eqn:tsvmal}
\end{equation}
For usual values of 
%the parameter of heterogeneity 
$S_2$, which range from $1$ to $2$, the expression (\ref{eqn:tsvmal}) is 
positive. This means that the anisotropic approximation
(\ref{eqn:TIapprox}) overestimates traveltimes in the isotropic
heterogeneous model even more than does the shifted hyperbola 
(\ref{eqn:malovichko}) shown in Figure~\ref{fig:nmofrz2}. 
Below, we examine which of the two approximations is more suitable when
the model includes both vertical heterogeneity and anisotropy.

\subsection{Vertically heterogeneous VTI model} 
%%%%%%%%%%%%%%%%%%%%%%%%%%%%%%%%%%%%%%%%%%%%%%%
In a model that includes vertical heterogeneity and anisotropy, both
factors influence bending of the rays. The weak anisotropy
approximation, however, allows us to neglect the effect of anisotropy on ray
trajectories and consider its influence on traveltimes only. This
assumption is analogous to the linearization, conventionally done for
tomographic inversion. Its application to weak anisotropy has been
discussed by \cite{GEO61-06-18831894}. According to the
linearization assumption, we can retain isotropic equation
(\ref{eqn:hp}) describing the ray trajectories and rewrite equation
(\ref{eqn:tp}) in the form
\begin{equation}
t(p) = 2\,\int_{0}^{z}\,{{dz} \over {V_g(z,\psi(z))\,\cos{\psi(z)}}}\;,
\label{eqn:TItp} 
\end{equation}
where $V_g$ is the anisotropic group velocity, which varies both with the
depth $z$ and with the ray angle $\psi$ and has the expression
(\ref{eqn:vg}). Differentiation of the parametric traveltime equations
(\ref{eqn:TItp}) and (\ref{eqn:hp}) and linearization with respect to Thomsen's
anisotropic parameters shows that the general form of equations
(\ref{eqn:a0})--(\ref{eqn:a3}) remains valid if we replace the definitions 
of the root-mean-square velocity $V_{rms}$ and the parameters $M_k$ 
by 
\begin{eqnarray}
V_{rms}^2 & = & {1 \over t_z}\,\int_{0}^{t_z}\,V_z^{2}(t)\,
\left[1 + 2\,\delta(t)\right]\,dt\;,
\label{eqn:TIvrms} \\
M_k & = & {1 \over t_z}\,\int_{0}^{t_z}\,V_z^{2k}(t)\,
\left[1 + 2\,\delta(t)\right]^{2k}\,\left[1 + 8\,\eta(t)\right]\,dt \qquad 
(k = 2,3, \ldots)\;.
\label{eqn:TImk} 
%\\
%S_k & = & {M_k \over {V_{rms}^{2k}}}\qquad (k = 2, 3, \ldots)\;.
%\label{eqn:TIsk}
\end{eqnarray}
%??? Am I right removing this eq? It is the same as eq.(29). --
In homogeneous media, expressions (\ref{eqn:TIvrms}) 
and (\ref{eqn:TImk}) transform series (\ref{eqn:taylor}) with
coefficients (\ref{eqn:a0})--(\ref{eqn:a3}) into the form equivalent to
series (\ref{eqn:TItaylor}). Two important conclusions follow from 
%the mathematical form of 
equations (\ref{eqn:TIvrms}) and (\ref{eqn:TImk}). 
First, if the mean value of the anisotropic coefficient $\delta$ 
is less than zero, the presence of anisotropy can reduce the difference
between the effective root-mean-square velocity and the effective vertical 
velocity $\widehat{V}_z=z/t_z$. In this case, the influence of anisotropy
and heterogeneity partially cancel each other, and the moveout curve
may behave at small offsets as if the medium were homogeneous and
isotropic. This behavior has been noticed by
\cite{GEO58-10-14541467}. On the other hand, if the anellipticity
coefficient $\eta$ is positive and different from zero, it can
significantly increase the values of the heterogeneity parameters
$S_k$ defined by equations~(\ref{eqn:sk}). Then, the nonhyperbolicity 
of reflection moveouts at
large offsets is stronger than that in isotropic media.
\par
To exemplify the general theory, let us consider a simple analytic
model with constant anisotropic parameters and the vertical velocity
linearly increasing with depth according to the equation
\begin{equation}
V_z(z) = V_z(0)\,(1 + \beta \,z) = V_z(0)\,e^{\kappa(z)}\;,
\label{eqn:vzlin} 
\end{equation}
where $\kappa$ is the logarithm of the velocity change. In this case,
the analytic expression for the RMS velocity $V_{rms}$ is found
from equation (\ref{eqn:TIvrms}) to be
\begin{equation}
V_{rms}^2 = V_z^2(0)\,(1 + 2\,\delta)\,{{e^{2\kappa}-1}\over {2\,\kappa}}\;,
\label{eqn:vrmslin} 
\end{equation}
while the mean vertical velocity is
\begin{equation}
\widehat{V}_z = {z \over t_z} = 
V_z(0)\,{{e^{\kappa}-1}\over {\kappa}}\;,
\label{eqn:hatvzlin} 
\end{equation}
where $\kappa=\kappa(z)$ is evaluated at the reflector depth.
Comparing equations (\ref{eqn:vrmslin}) and (\ref{eqn:hatvzlin}), we can see 
that the squared RMS velocity $V_{rms}^2$ equals to the squared mean velocity
$\widehat{V}_z^2$ if
\begin{equation}
1 + 2\,\delta = {{2\,\left(e^\kappa - 1\right)} \over
{\kappa\,\left(e^\kappa + 1\right)}}\;.
\label{eqn:deltalin} 
\end{equation}
For small $\kappa$, the estimate of $\delta$ from equation
(\ref{eqn:deltalin}) is
\begin{equation}
\delta \approx - {\kappa^2 \over 24}\;.
\label{eqn:appdeltalin} 
\end{equation}
For example, if the vertical velocity near the reflector is twice
that at the surface (i.e., $\kappa = \ln 2 \approx 0.69$), 
having the anisotropic
parameter $\delta$ as small as~$-0.02$ is sufficient to cancel out the
influence of heterogeneity on the normal-moveout velocity.  The values of
parameters $S_2$ and $S_3$, found from equations~(\ref{eqn:sk}), (\ref{eqn:TIvrms}) and~(\ref{eqn:TImk}), are
\begin{eqnarray}
S_2 & = & (1 + 8\,\eta)\,\kappa\,{{e^{2\kappa}+1} \over {e^{2\kappa} - 1}}\;,
\label{eqn:lins2} \\
S_3 & = & {4 \over 3}\,
(1 + 8\,\eta)\,\kappa^2\,{{e^{4\kappa} + e^{2\kappa}+1} \over 
{\left(e^{2\kappa} - 1\right)^2}}\;.
\label{eqn:lins3}
\end{eqnarray} 
Substituting equations (\ref{eqn:lins2}) and (\ref{eqn:lins3}) into 
the estimates (\ref{eqn:malerror}) and (\ref{eqn:tsverror}) and linearizing 
them both in $\eta$ and in $\kappa$, we
find that the error of anisotropic traveltime approximation
(\ref{eqn:TIapprox}) in the linear velocity model is 
\begin{equation}
{{\Delta t^2(l)} \over t^2(0)} = 
-{{\kappa^2\,(1 - 8\,\eta)} \over 12}\,
\left({l \over {t_0\,V_n}}\right)^6\;,
\label{eqn:lintsverr}
\end{equation}
while the error of the shifted-hyperbola approximation
(\ref{eqn:malovichko}) is
\begin{equation}
{{\Delta t^2(l)} \over t^2(0)} = 
\left({{\kappa^2\,(1 - 8\,\eta)} \over 24} - \eta\right)\,
\left({l \over {t_0\,V_n}}\right)^6\;.
\label{eqn:linmalerr}
\end{equation}
%??? Please put the correct sign into one of the previous two eqs. Otherwise,
%eq.(53) does not seem correct -- that is why I thought it is incorrect. --
Comparing 
equations (\ref{eqn:lintsverr}) and (\ref{eqn:linmalerr}), we 
conclude that if the medium is elliptically anisotropic $(\eta=0)$, 
the shifted hyperbola can be twice as accurate as the anisotropic equation 
(assuming the optimal choice of parameters). The accuracy of the latter,
however, increases when the anellipticity coefficient $\eta$ grows and
becomes higher than that of the shifted hyperbola if $\eta$ satisfies
the approximate inequality 
\begin{equation}
\eta \geq {\kappa^2 \over {8\,(1 + \kappa^2)}}\;.
\label{eqn:lineta}
\end{equation}
%??? -- I got from equations (\ref{eqn:lintsverr}) and (\ref{eqn:linmalerr})
%\[
%  \eta \geq {\kappa^2 \over {4\,(3 + 2\,\kappa^2)}}\;.
%\]
%Who is right?
For instance, if $\kappa = \ln 2$, inequality~(\ref{eqn:lineta}) yields
$\eta \geq 0.03$, a quite small value.

%\begin{comment}
%\subsection{Stolt Stretch}
%%%%%%%%%%%%%%%%%%%%%
%Stolt stretch
%\cite{GEO43.01.00230048,Levin.sep.42.373,Claerbout.blackwell.85} is
%a method of extending constant-velocity frequency-domain migration to
%the case of a vertically variable velocity. The method consists of
%stretching the time axis according to the equation
%\begin{equation}
%\tau(t_z)=
%\left({{2 \over V_0^2}\,\int_0^{t_z}\,t\,V_{rms}^2(t)\,dt}\right)^{1/2}\;,
%\label{eqn:ss} 
%\end{equation}
%double Fourier transform, and migration according to the dispersion
%relation
%\begin{equation}
%\omega_m(k,\omega_0)=
%\left(1-{1\over W}\right)\,\omega_0+
%{{\mbox{sign}(\omega_0)}\over W}\,
%\sqrt{\omega_0^2 - W\,V_0^2\,k_x^2}\;,
%\label{eqn:sdispersion} 
%\end{equation}
%where $V_0$ is a constant frame velocity, $\omega_0$ and $\omega_m$
%are the frequencies before and after the migration, corresponding to
%the stretched time coordinate, $k_x$ is the wavenumber, and $W$ is a
%constant parameter ($W=1$ in the constant velocity case). Fomel
%\shortcite{Fomel.sep.84.61} has shown that the optimal choice of the
%Stolt stretch parameter $W$ for a particular traveltime depth $t_z$ is
%given by the expression
%\begin{equation}
%W=1-{{V_0^2\,\tau^2\left(t_z\right)} \over{V_{rms}^2\left(t_z\right) t_z^2}}\,
%\left({{V^2\left(t_z\right)} \over {V_{rms}^2\left(t_z\right)}}
%-S_2\left(t_z\right)
%\right)\;\;.
%\label{eqn:main} 
%\end{equation}
%This expression remains valid in the case of a vertically
%heterogeneous VTI medium if the values of $V_{rms}$ and $S_2$ are
%computed according to equations (\ref{eqn:TIvrms}) and
%(\ref{eqn:TIsk}). The method of cascaded migrations
%\cite{GEO52.05.06180643} can improve the performance of Stolt
%migration in the case of variable velocity \cite{GEO53.07.08810893}.
%However, this method affects only the isotropic part of the model and
%cannot change the contribution of the anisotropic parameters.
%Therefore, in the anisotropic case, it is important to incorporate
%anisotropic parameters into the Stolt stretch correction.
%\end{comment}


