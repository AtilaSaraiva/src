\append{NORMAL MOVEOUT BEYOND THE NIP THEOREM}
%%%%%%%%%%%%%%%%%%%%%%%%%%%%%%%%%%%%%%%%%%%%%
In this Appendix, we derive equations that relate traveltime
derivatives of the reflected wave, evaluated at the zero offset point,
and traveltime derivatives of the direct wave, evaluated in the
vicinity of the zero-offset ray. Such a relationship for second-order
derivatives is known as the NIP (normal incidence point) theorem
\cite[]{chergr,hubkrey,GEO48-08-10511062}. Its extension to high-order
derivatives is described by \cite{fomel}.

Reflection traveltime in any type of model can be considered as a
function of the source and receiver locations $s$ and $r$ and the
location of the reflection point $x$, as follows:
\begin{equation}
t(y,h) = F\left(y,h,x(y,h)\right)\;,
\label{eqn:ttsum}
\end{equation}
where $y$ is the midpoint $\left(y = {{s + r} \over 2}\right)$, $h$ is
the half-offset $\left(h = {{r - s} \over 2}\right)$, and the
function $F$ has a natural decomposition into two parts corresponding
to the incident and reflected rays:
\begin{equation}
F(y,h,x) = T(y-h,x) + T(y+h,x)\;,
\label{eqn:f2t}
\end{equation}
where $T$ is the traveltime of the direct wave. Clearly, at the
zero-offset point,
\begin{equation}
t(y,0) = 2 T(y,x)\;,
\label{eqn:2t2t0}
\end{equation}
where $x=x(y,0)$ corresponds to the reflection point of the
zero-offset ray.

Differentiating equation (\ref{eqn:ttsum}) with respect to the half-offset
$h$ and applying the chain rule, we obtain
\begin{equation}
{{\partial t} \over {\partial h}} = {{\partial F} \over {\partial h}}
+ {{\partial F} \over {\partial x}}\,{{\partial x} \over {\partial h}}\;.
\label{eqn:ttp}
\end{equation}
According to Fermat's principle, one of the
fundamental principles of ray theory, the ray
trajectory of the reflected wave corresponds to an
extremum value of the traveltime. Parameterizing
the trajectory in terms of the reflection point
location $x$ and assuming that $F$ is a smooth
function of $x$, we can write Fermat's principle
in the form 
\begin{equation}
{{\partial F} \over {\partial x}} = 0\;.
\label{eqn:ferma}
\end{equation}
Equation (\ref{eqn:ferma}) must be satisfied for any values of $x$ and
$h$. Substituting this equation into equation (\ref{eqn:ttp}) leads to the
equation
\begin{equation}
{{\partial t} \over {\partial h}} = {{\partial F} \over {\partial h}}\;.
\label{eqn:tp1}
\end{equation}

Differentiating (\ref{eqn:tp1}) again with respect to $h$, we arrive at the
equation 
\begin{equation}
{{\partial^2 t} \over {\partial h^2}} = 
{{\partial^2 F} \over {\partial h^2}} +
{{\partial^2 F} \over {\partial h\,\partial x}}\,
{{\partial x} \over {\partial h}}\;.
\label{eqn:tpp}
\end{equation}
Interchanging the source and receiver locations doesn't change the
reflection point position (the principle of reciprocity). Therefore,
$x$ is an even function of the offset $h$, and we can simplify equation
(\ref{eqn:tpp}) at zero offset, as follows:
\begin{equation}
\left.{{\partial^2 t} \over {\partial h^2}}\right|_{h=0} = 
\left.{{\partial^2 F} \over {\partial h^2}}\right|_{h=0}\;.
\label{eqn:tpp1}
\end{equation}
Substituting the expression for the function $F$ (\ref{eqn:f2t}) into
(\ref{eqn:tpp1}) leads to the equation
\begin{equation}
\left.{{\partial^2 t} \over {\partial h^2}}\right|_{h=0} = 
2\,{{\partial^2 T} \over {\partial y^2}}\;,
\label{eqn:tpp2}
\end{equation}
which is the mathematical formulation of the NIP theorem. It proves
that the second-order derivative of the reflection traveltime with
respect to the offset is equal, at zero offset, to the second
derivative of the direct wave traveltime for the wave propagating from
the incidence point of the zero-offset ray. One immediate conclusion
from the NIP theorem is that the short-spread normal moveout velocity,
connected with the derivative in the left-hand-side of equation
(\ref{eqn:tpp2}) can depend on the reflector dip but doesn't depend on
the curvature of the reflector. Our derivation up to this point has
followed the derivation suggested by \cite{chergr}.

Differentiating equation (\ref{eqn:tpp}) twice with respect to $h$ evaluates,
with the help of the chain rule, the fourth-order derivative, as
follows:
\begin{equation}
{{\partial^4 t} \over {\partial h^4}}  = 
{{\partial^4 F} \over {\partial h^4}} +
3\,{{\partial^4 F} \over {\partial h^3\,\partial x}}\,
{{\partial x} \over {\partial h}} +
3\,{{\partial^4 F} \over {\partial h^2\,\partial x^2}}\,
\left({{\partial x} \over {\partial h}}\right)^2 +
{{\partial^4 F} \over {\partial h\,\partial x^3}}\,
\left({{\partial x} \over {\partial h}}\right)^3 + 
\nonumber 
\end{equation}
\begin{equation}
 +  3\,{{\partial^3 F} \over {\partial h^2\,\partial x}}\,
{{\partial^2 x} \over {\partial h^2}} +
3\,{{\partial^3 F} \over {\partial h\,\partial x^2}}\,
{{\partial^2 x} \over {\partial h^2}}\,
{{\partial x} \over {\partial h}} +
{{\partial^2 F} \over {\partial h\,\partial x}}\,
{{\partial^3 x} \over {\partial h^3}}\;.
\label{eqn:tp4}
\end{equation}
Again, we can apply the principle of reciprocity to eliminate the
odd-order derivatives of $x$ in equation (\ref{eqn:tp4}) at the zero
offset. The resultant expression has the form
\begin{equation}
\left.{{\partial^4 t} \over {\partial h^4}}\right|_{h=0} = 
\left.\left({{\partial^4 F} \over {\partial h^4}} +
3\,{{\partial^3 F} \over {\partial h^2\,\partial x}}\,
{{\partial^2 x} \over {\partial h^2}}\right)\right|_{h=0}.
\label{eqn:tp41}
\end{equation}
In order to determine the unknown second derivative of the reflection
point location ${{\partial^2 x} \over {\partial h^2}}$, we
differentiate Fermat's equation (\ref{eqn:ferma}) twice, obtaining
\begin{equation}
{{\partial^3 F} \over {\partial^2 h\,\partial x}} + 
2\,{{\partial^3 F} \over {\partial h\,\partial^2 x}}\,
{{\partial x} \over {\partial h}} +
{{\partial^3 F} \over {\partial^3 x}}\,
\left({{\partial x} \over {\partial h}}\right)^2 +
{{\partial^2 F} \over {\partial^2 x}}\,
{{\partial^2 x} \over {\partial h^2}} = 0\;.
\label{eqn:ferma2}
\end{equation}
Simplifying this equation at zero offset, we can solve it for the
second derivative of $x$. The solution has the form
\begin{equation}
\left.{{\partial^2 x} \over {\partial h^2}}\right|_{h=0} = 
- \left[\left({{\partial^2 F} \over {\partial^2 x}}\right)^{-1}\,
{{\partial^3 F} \over {\partial^2 h\,\partial x}}\right]_{h=0}\;.
\label{eqn:dxdh}
\end{equation}
Here we neglect the case of ${{\partial^2 F} \over {\partial^2 x}} =
0$, which corresponds to a focusing of the reflected rays at the
surface.  Finally, substituting expression (\ref{eqn:dxdh}) into (\ref{eqn:tp41})
and recalling the definition of the $F$ function
from (\ref{eqn:f2t}), we obtain
the equation
\begin{equation}
\left.{{\partial^4 t} \over {\partial h^4}}\right|_{h=0}  = 
2\,{{\partial^4 T} \over {\partial y^4}}-
6\,\left({{\partial^2 T} \over {\partial x^2}}\right)^{-1}\,
\left({{\partial^3 T} \over {\partial y^2\,\partial x}}\right)^2\;,
\label{eqn:tp42} 
\end{equation}
which is the same as equation (\ref{eqn:NIP4}) in the main
text. Higher-order derivatives can be expressed in an analogous way
with a set of recursive algebraic functions \cite[]{fomel}.

In the derivation of equations (\ref{eqn:tpp2}) and (\ref{eqn:tp42}),
we have used Fermat's principle, the principle of reciprocity, and the
rules of calculus. Both these equations remain valid in anisotropic
media as well as in heterogeneous media, providing that the traveltime
function is smooth and that focusing of the reflected rays doesn't
occur at the surface of observation.




