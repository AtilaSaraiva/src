\begin{abstract}
%% 
 % angle decomposition by wave-equation migration
 %%
  Extended common-image-point-gathers (CIP) contain all the necessary
  information for decomposition of reflectivity as a function of the
  reflection and azimuth angles at selected locations in the
  subsurface. This decomposition operates after the imaging condition
  applied to wavefields reconstructed by any type of wide-azimuth
  migration method, e.g. using downward continuation or time reversal.
%% 
 % use CIPs to cut cost 
 %%
  The reflection and azimuth angles are derived from the extended
  images using analytic relations between the space-lag and time-lag
  extensions. The transformation amounts to a linear Radon transform
  applied to the CIPs obtained after the application of the extended
  imaging condition. If information about the reflector dip is
  available at the CIP locations, then only two components of the
  space-lag vectors are required, thus reducing computational cost and
  increasing the affordability of the method.
%% 
 % applications 
 %%
  Applications of this method include the study of subsurface
  illumination in areas of complex geology where ray-based methods are
  not usable, and the study of amplitude variation with reflection and
  azimuth angles if the subsurface subsurface illumination is
  sufficiently dense. Migration velocity analysis could also be
  implemented in the angle domain, although an equivalent
  implementation in the extended domain is cheaper and more effective.
\end{abstract}
