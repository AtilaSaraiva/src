%
\def\cof{\cos\phi}
\def\sif{\sin\phi}
%

\section{Angle decomposition}

%% 
 % goal of this section
 %%
In this section we discuss the steps required to transform lag-domain
CIPs into angle-domain CIPs using the moveout function derived in the
preceding section. We also present the algorithm used for angle
decomposition and illustrate it using a simple 3D model of a
horizontal reflector in a medium with constant velocity which allows
us to validate analytically the procedure.

The outer loop of the algorithm is over the CIPs evaluated during
migration. The angle decompositions of individual CIPs are independent
of one-another, therefore the algorithm is easily parallelizable over
the outer loop. At every CIP, we need to access the information about
the reflector normal ($\nn$) and about the local velocity ($v$). The
reflector dip information can be extracted from the conventional
image, and the velocity is the same as the one used for migration.

Prior to the angle decomposition, we also need to define a direction
relative to which we measure the reflection azimuth. This direction is
arbitrary and depends on the application of the angle
decomposition. Typically, the azimuth is defined relative to a
reference direction (e.g. North). Here, we define this azimuth
direction using an arbitrary vector $\vv$. Using the reflector normal
($\nn$) we can build the projection of the azimuth vector ($\aa$) in
the reflector plane as
%
\beq \label{eqn:AVEC}
\aa = \lp \nn\times\vv \rp \times \nn \;.
\eeq
%
This construction assures that vector $\aa$ is contained in the
reflector plane (i.e. it is orthogonal on $\nn$) and that it is
co-planar with vectors $\nn$ and $\vv$, \rfg{eicawfl}. Of course, this
construction is just one of the many possible definitions of the
azimuth reference. In the following, we measure the azimuth angle
$\phi$ relative to vector $\aa$ and the reflection angle $\theta$
relative to the normal to the reflector given by vector $\nn$.

Then, for every azimuth angle $\phi$, using the reflector normal
($\nn$) and the azimuth reference ($\aa$), we can construct the trial
vector $\qq$ which lies at the intersection of the reflector and the
reflection planes. We scan over all possible vectors $\qq$, although
only one azimuth corresponds to the reflection from a given shot. This
scan ensures that we capture the reflection information from all shots
in the survey. Given the reflector normal (the axis of rotation) and
the trial azimuth angle $\phi$, we can construct the different vectors
$\qq$ by the application of the rotation matrix
%
\beq \label{eqn:ROTMAT}
Q\lp \nn,\phi \rp=
\lb \begin{array}{ccc}
n_x^2+\lp n_y^2+n_z^2 \rp\cof & n_x n_y\lp 1-\cof\rp -n_z\sif & n_x n_z\lp 1-\cof\rp +n_y\sif \\
n_y n_x\lp 1-\cof\rp +n_z\sif & n_y^2+\lp n_z^2+n_x^2 \rp\cof & n_y n_z\lp 1-\cof\rp -n_x\sif \\
n_z n_x\lp 1-\cof\rp -n_y\sif & n_z n_y\lp 1-\cof\rp +n_x\sif & n_z^2+\lp n_x^2+n_y^2 \rp\cof \\
\end{array} \rb
\eeq
to the azimuth reference vector $\aa$, i.e.
\beq \label{eqn:QVEC}
\qq = Q\lp \nn,\phi \rp \aa \;.
\eeq
%
In this formulation, the normal vector $\nn$ of components
$\{n_x,n_y,n_z\}$ can take arbitrary orientations and does not need to
be normalized. Then, for every reflection angle $\theta$, we map the
lag-domain CIP to the angle-domain by summation over the surface
defined by \req{CONstk}. This operation represents a planar Radon
transform (a slant-stack) over an analytically-defined surface in the
$\{\hh,\tt\}$ space.
%
The output is the representation of the CIP in the angle-domain.  In
order to preserve the signal bandwidth, the slant-stack needs to use a
``rho filter'' which compensates the high frequency decay caused by
the summation \cite[]{Claerbout.fgdp.1976}.  The explicit algorithm
for angle decomposition is given in Appendix A.

% ------------------------------------------------------------
\inputdir{flatEICangle}
% ------------------------------------------------------------

Consider a simple 3D model consisting of a horizontal reflector in a
constant velocity medium. We simulate one shot in the center of the
model at coordinates $x=4$~km and $y=4$~km, with receivers distributed
uniformly on the surface on a grid spaced at every $20$~m in the $x$
and $y$ directions. We use time-domain finite-differences for
modeling.
%
\rFg{img-3d} represents the image obtained by wave-equation migration
of the simulated shot using downward continuation. The illumination is
limited to a narrow region around the shot due to the limited array
aperture.

% ------------------------------------------------------------
%\sideplot{ro-3d}{width=\textwidth}{The model used for the simple
%  horizontal reflector example.}
\sideplot{img-3d}{width=\textwidth}{The image obtained for a
  horizontal reflector in constant velocity using one shot located in
  the center of the model.}
% ------------------------------------------------------------

\rFgs{hic-A}-\rfn{hic-D} depict CIPs obtained by migration of the
simulated shot at the reflector depth and at coordinates $\{x,y\}$
equal to $\{3.2,3.2\}$~km, $\{3.2,4.8\}$~km, $\{4.8,4.8\}$~km and
$\{4.8,3.2\}$~km, respectively. For these CIPs, the reflection angle
is invariant $\theta=48.5^\circ$, but the azimuth angles relative to
the $x$ axis are $-135^\circ$, $+135^\circ$, $+45^\circ$ and
$-45^\circ$, respectively. \rFgs{aca-A}-\rfn{aca-D} show the angle
decomposition in polar coordinates.  Here, we use the trigonometric
convention to represent the azimuth angle $\phi$ and we represent the
reflection angle in every azimuth in the radial direction (with normal
incidence at the center of the plot). Each radial line corresponds to
$30^\circ$ and each circular contour corresponds to $15^\circ$.

% ------------------------------------------------------------
\multiplot{4}{hic-A,hic-B,hic-C,hic-D,aca-A,aca-B,aca-C,aca-D}
{width=0.22\textwidth}{ Illustration of CIP angle decomposition for
  illumination at fixed reflection angle. Panels (a)-(d) show
  lag-domain CIPs, and panels (e)-(f) show angle-domain CIPs in polar
  coordinates. The angles $\phi$ and $\theta$ are indexed along the
  contours using the trigonometric convention and along the radial
  lines increasing from the center.}
% ------------------------------------------------------------

Similarly, \rFgs{hic-F}-\rfn{hic-I} depict CIPs obtained by migration
of the simulated shot the reflector depth and at coordinates $\{x,y\}$
equal to $\{2.8,2.8\}$~km, $\{3.2,3.2\}$~km, $\{3.6,3.6\}$~km and
$\{4.0,4.0\}$~km, respectively.  For these CIPs, the azimuth angle is
invariant $\phi=-135^\circ$, but the reflection angles relative to the
reflector normal are $59.5^\circ$, $48.5^\circ$, $29.5^\circ$, and
$0^\circ$ respectively.

% ------------------------------------------------------------
\multiplot{4}{hic-F,hic-G,hic-H,hic-I,aca-F,aca-G,aca-H,aca-I}
{width=0.22\textwidth}{ Illustration of CIP angle decomposition for
  illumination at fixed azimuth angle.  Panels (a)-(d) show lag-domain
  CIPs, and panels (e)-(h) show angle-domain CIPs in polar
  coordinates. The angles $\phi$ and $\theta$ are indexed along the
  contours using the trigonometric convention and along the radial
  lines increasing from the center.}
% ------------------------------------------------------------

In all examples, the decomposition angles correspond to the
theoretical values, thus confirming the validity of our decomposition.
