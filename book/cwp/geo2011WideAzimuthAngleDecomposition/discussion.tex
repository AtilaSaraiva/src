\section{Discussion}

%% 
 % the planar assumption
 %%

We do not suggest in this paper that the wavefields used for imaging
are planar prior to the interaction with the reflector. In complex
geology, such an assumption would be unrealistic. However, a wavefield
of arbitrary shape can be thought of as a superposition of plane waves
propagating in various directions, either because the wavefronts
characterizing the wavefields have curvature, or because the
wavefields have triplicated during propagation. Each incident plane
has a corresponding reflected plane related through Snell's law. Some
angle decomposition techniques make use explicitly of a planar
decomposition of the wavefields, followed of selection through
thresholding of the most energetic plane \cite[]{xu:3257}.  In
contrast, we rely on the fact that all planar components of the
wavefields have been transformed as planar events in the extended
images and rely on slant-stacks or equivalent methods to separate them
as a function of azimuth and reflection angles.

%% 
 % do not need to compute all lambdas if we know nn
 % do not need to know nn if we can afford to compute all lambdas
 %%

As indicated in the preceding sections, we do not need to compute all
space-lags at the considered CIP positions. We could compute just two
of them, e.g. $\lx$ and $\ly$ as shown in the examples of this paper,
and then reconstruct the third lag using the information given by the
reflector normal at the CIP position, \req{CONort}. If the reflector
is nearly vertical, it may be more relevant to compute the vertical
and one horizontal space-lags.
%
Alternatively, we could avoid computing the reflector normal vector
from the conventional image, but instead compute all three components
of the space-lag vector $\hh$. In this case, as indicated by
\cite{SavaVasconcelos.gpr.eic}, we could estimate the reflector dip
from the lag information prior to the angle decomposition.

%% 
 % the in-plane space-shift condition is related to the equal time-lag construction
 %%
We have also noted earlier in the paper that the relevant space-lags
are constructed in the reflector plane. This fact is a direct
consequence of the fact that we have considered equal but with
opposite sign time-shift of the source and receiver
wavefields. Without this convention, the angle decomposition problem
becomes more complex. In our experience to date, we did not find the
need to relax this requirement. 

%% 
 % resolution
 % illumination
 %%
The angle-domain CIPs accurately indicate the sampling of the
reflector as a function of azimuth and reflection angles. If the shot
distribution is sparse, or if the sub-surface geology creates shadow
zones, the illumination is also sparse. This is both beneficial,
assuming that the angle-domain CIPs are used to evaluate illumination,
but it can also be a drawback if the angle-domain CIPs are used
for AVA or MVA. However, a sparse sampling of a reflector is not a
feature of the angle decomposition, but a feature of the acquisition
geometry. Neither our, nor any other angle decomposition, can
compensate for the lack of adequate data illuminating the subsurface
on a dense angular grid.

%% 
 % AVA application ???
 % MVA application ???
 %%
Finally, we note that the most likely applications for angle
decomposition in complex geology is the study of the reflector
illumination itself. Assuming that the sampling is sufficiently dense
and that the imaging velocity is accurately known, then we can use the
angle decomposition discussed in this paper to evaluate amplitude
variation with azimuth and reflection angles. However, we emphasize
that this is a relevant exercise only if the reflector illumination is
sufficiently dense. Otherwise, AVA effects overlap with illumination
effects, rendering the analysis unreliable.
%
Migration velocity analysis in the angle domain may also suffer from
the lack of adequate illumination. This partial illumination may
deteriorate the moveout which would otherwise be observed in the
extended image domain. Furthermore, we do not advocate an
implementation of MVA in the angle-domain, but rather in the extended
image domain which contains all the relevant information and avoids
the additional step of angle decomposition. An extensive discussion
of this problem is outside the scope of our paper.
