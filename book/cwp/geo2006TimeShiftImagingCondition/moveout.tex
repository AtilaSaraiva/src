\section{Moveout analysis}
% ------------------------------------------------------------
\inputdir{flat}
% ------------------------------------------------------------
\plot{hyper}{width=6.0in}
{An image is formed when the Kirchoff stacking
curve (dashed line) touches the true reflection response. 
Left: the case of under-migration; right: over-migration.}
% ------------------------------------------------------------
\plot{off}{width=5.0in}
{Common-image gathers for
space-shift imaging (left column) and
time-shift imaging (right column).}
% ------------------------------------------------------------
\plot{ssk}{width=5.0in}
{Common-image gathers after slant-stack
for space-shift imaging (left column) and
for time-shift imaging (right column).
The vertical line indicates the migration velocity.}
% ------------------------------------------------------------

We can use the Kirchhoff formulation to analyze the moveout 
behavior of the time-shift imaging condition in the simplest 
case of a flat reflector in a constant-velocity medium 
(Figures~\ref{fig:hyper}-\ref{fig:ssk}). 

The synthetic data are imaged using shot-record 
wavefield extrapolation migration.
Figure~\ref{fig:off} shows offset
common-image gathers for three different migration
slownesses $s$, one of which is equal to the 
modeling slowness $s_0$.
The left column corresponds to the space-shift imaging condition
and the right column corresponds to the time-shift imaging condition.

For the space-shift CIGs imaged with correct slowness,
left column in Figure~\ref{fig:off},
the energy is focused at zero offset,
but it spreads in a region of offsets when the slowness is wrong.
Slant-stacking produces the images in left column of Figure~\ref{fig:ssk}.

For the time-shift CIGs imaged with correct slowness,
right column in Figure~\ref{fig:off},
the energy is distributed along a line with a slope equal 
to the local velocity at the reflector position,
but it spreads around this region when the slowness is wrong.
Slant-stacking produces the images in the right column of Figure~\ref{fig:ssk}.

Let $s_0$ and $z_0$ represent
the true slowness and reflector depth, and $s$ and $z$ stand for the
corresponding quantities used in migration.  An image is formed when
the Kirchoff stacking curve $t(\hat{h}) = 2\,s\,\sqrt{z^2+\hat{h}^2} +
2\,\tau$ touches the true reflection response $t_0(\hat{h}) =
2\,s_0\,\sqrt{z_0^2+\hat{h}^2}$ (Figure~\ref{fig:hyper}). 
Solving for $\hat{h}$ from the envelope condition 
$t'(\hat{h})=t_0'(\hat{h})$ yields two solutions:
\beq \label{eqn:h1}
\hat{h} = 0
\eeq
and
\beq \label{eqn:h2}
\hat{h} = \sqrt{\frac{s_0^2 z^2 - s^2 z_0^2}{s^2-s_0^2}} \;.
\eeq
Substituting solutions \ref{eqn:h1} and \ref{eqn:h2}
in the condition $t(\hat{h})=t_0(\hat{h})$ produces
two images in the $\{z,\tau\}$ space.
The first image is a straight line
\beq \label{eqn:line}
z(\tau) = \frac{z_0\,s_0 - \tau}{s}\;,
\eeq
and the second image is a segment of the second-order curve
\beq \label{eqn:tcurve}
z(\tau) = \sqrt{z_0^2 + \frac{\tau^2}{s^2-s_0^2}}\;.
\eeq
Applying a slant-stack transformation with $z = z_1 - \nu\,\tau$ turns
line~(\ref{eqn:line}) into a point
$\{z_0\,s_0/s,1/s\}$ in the $\{z_1,\nu\}$
space, while curve~(\ref{eqn:tcurve}) turns into the curve
\beq \label{eqn:tcurve2}
  z_1(\nu) = z_0\,\sqrt{1 + \nu^2\,\left(s_0^2-s^2\right)}\;.
\eeq
The curvature of the $z_1(\nu)$ curve at $\nu=0$ is a clear indicator
of the migration velocity errors. 

By contrast, the moveout shape $z(h)$ appearing in wave-equation
migration with the lateral-shift imaging condition is \cite[]{Bartana}
\beq \label{eqn:hcurve}
  z(h) = s_0\,\sqrt{\frac{z_0^2}{s^2} + \frac{h^2}{s^2-s_0^2}}\;.
\eeq
After the slant transformation $z = z_1 + h\,\tan{\theta}$, the
moveout curve~(\ref{eqn:hcurve}) turns into the curve
\beq \label{eqn:hcurve2}
  z_1(\theta) = \frac{z_0}{s}\,\sqrt{s_0^2 + \tan^2{\theta}\,\left(s_0^2-s^2\right)}\;,
\eeq
which is applicable for velocity analysis. A formal connection between
$\nu$-parameterization in equation~(\ref{eqn:tcurve2}) and
$\theta$-parameterization in equation~(\ref{eqn:hcurve2}) is given by
\beq \label{eqn:nu2theta}
  \tan^2{\theta} = s^2\,\nu^2 - 1\;,
\eeq
or
\beq \label{eqn:angTflat}
\cos \t = \frac{1}{\nu s} = \frac{\tt_z}{s} \;,
\eeq
where $\tt_z = \frac{\partial \tt}{\partial z}$.
\rEq{angTflat} is a special case of \req{angT} 
for flat reflectors.
Curves of shape~\ren{tcurve2} and \ren{hcurve2} are
plotted on top of the experimental moveouts in Figure~\ref{fig:ssk}.

