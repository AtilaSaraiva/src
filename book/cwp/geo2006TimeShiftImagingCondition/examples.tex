\section{Examples}
% ------------------------------------------------------------
\inputdir{zicig}
% ------------------------------------------------------------
We demonstrate the imaging condition introduced in this paper
with the Sigsbee 2A synthetic model \cite[]{SEG-2002-21222125}.
Figure~\ref{fig:IMGSLO0t} shows the correct migration velocity 
and the image created by shot-record migration 
with wavefield extrapolation using the time-shift 
imaging condition introduced in this paper. 
The image in the bottom panel of 
Figure~\ref{fig:IMGSLO0t} is extracted at $\tt=0$.
% ------------------------------------------------------------
\plot{IMGSLO0t}{width=6.0in}{Sigsbee 2A model: 
correct velocity (top) and migrated image obtained
by shot-record wavefield extrapolation migration with 
time-shift imaging condition (bottom).}
% ------------------------------------------------------------

The top row of Figure~\ref{fig:alloff} shows common-image gathers at 
locations $x=\{7,9,11,13,15,17\}$~km
obtained by time-shift imaging condition. 
As in the preceding synthetic 
example, we can observe events with linear trends at slopes 
corresponding to local migration velocity.
Since the migration velocity is correct, the strongest 
events in common-image gathers correspond to $\tt=0$.
For comparison, 
the bottom row of Figure~\ref{fig:alloff} shows
common-image gathers at the same locations
obtained by space-shift imaging condition.
In the later case, the strongest events occur at $\hh=0$.
The zero-offset images ($\tt=0$ and $\hh=0$) are identical.
% ------------------------------------------------------------
\plot{alloff}{width=6.0in}
{Imaging gathers at positions $x=\{7,9,11,13,15,17\}$~km.
Time-shift imaging condition (top row), and
space-shift imaging condition (bottom row).}
% ------------------------------------------------------------

Figure~\ref{fig:SRt0-7} shows the angle-decomposition
for the common-image gather at location $x=7$~km.
From left to right, the panels depict
the migrated image,
a common-image gather resulting from
migration by wavefield extrapolation with time-shift imaging,
the common-image gather after slant-stacking
in the $z-\tt$ plane, and an angle-gather
derived from the slant-stacked panel using equation~\req{angT}.

For comparison, Figure~\ref{fig:SRx0-7} depicts
a similar process for a common-image gather at the same location
obtained by space-shift imaging.
Despite the fact that the offset gathers are completely 
different, the angle-gathers are comparable showing similar 
trends of angle-dependent reflectivity.

% ------------------------------------------------------------
\plot{SRt0-7}{width=6.0in}
{
Time-shift imaging condition gather at $x=7$~km.
From left to right, the panels depict
the image,
the time-shift gather,
the slant-stacked time-shift gather and
the angle-gather.}
% ------------------------------------------------------------
\plot{SRx0-7}{width=6.0in}
{
Space-shift imaging condition gather at $x=7$~km:
From left to right, the panels depict
the image,
the space-shift gather,
the slant-stacked space-shift gather and
the angle-gather.}
% ------------------------------------------------------------

The top row of Figure~\ref{fig:allang} shows angle-domain 
common-image gathers for time-shift imaging
at locations $x=\{7,9,11,13,15,17\}$~km. 
Since the migration velocity is correct, all events are mostly
flat indicating correct imaging.
For comparison, 
the bottom row of Figure~\ref{fig:allang} shows
angle-domain common-image gathers for space-shift 
imaging condition at the same locations in the image.

% ------------------------------------------------------------
\plot{allang}{width=6.0in}
{
Angle-gathers at positions $x=\{7,9,11,13,15,17\}$~km.
Time-shift imaging condition (top row), and
space-shift imaging condition (bottom row).
Compare with Figure~\ref{fig:alloff}.}
% ------------------------------------------------------------

Finally, we illustrate the behavior of 
time-shift imaging with incorrect velocity.
The top panel in Figure~\ref{fig:IMGSLO2t} shows an incorrect 
velocity model used to image the Sigsbee 2A data, and
the bottom panel shows the resulting image. The incorrect
velocity is a smooth version of the correct interval velocity,
scaled by $10\%$ from a depth $z=5$~km downward.
The uncollapsed diffractors at depth $z=7$~km clearly indicate
velocity inaccuracy.
% ------------------------------------------------------------
\plot{IMGSLO2t}{width=6.0in}{Sigsbee 2A model: 
incorrect velocity (top) and migrated image obtained
by shot-record wavefield extrapolation migration with 
time-shift imaging condition.
Compare with Figure~\ref{fig:IMGSLO0t}. }
% ------------------------------------------------------------

Figures~\ref{fig:SRt2-7} and \ref{fig:SRx2-7} show
imaging gathers and the derived angle-gathers for time-shift and 
space-shift imaging at the same location $x=7$~km.
Due to incorrect velocity, focusing does not occur
at $\tt=0$ or $\hh=0$ as in the preceding case.
Likewise, the reflections in angle-gathers are non-flat,
indicating velocity inaccuracies.
Compare Figures~\ref{fig:SRt0-7} and \ref{fig:SRt2-7},
and Figures~\ref{fig:SRx0-7} and \ref{fig:SRx2-7}.
Those moveouts can be exploited for migration velocity 
analysis 
\cite[]{
SEG-1999-17231726,
Sava.gp.wemva1,Sava.gp.wemva2,
clapp.geology}.

% ------------------------------------------------------------
\plot{SRt2-7}{width=6.0in}
{
Time-shift imaging condition gather at $x=7$~km.
From left to right, the panels depict
the image,
the offset-gather,
the slant-stacked gather and
the angle-gather.
Compare with Figure~\ref{fig:SRt0-7}. }
% ------------------------------------------------------------
\plot{SRx2-7}{width=6.0in}
{
Space-shift imaging condition gather at $x=7$~km.
From left to right, the panels depict
the image,
the offset-gather,
the slant-stacked gather and
the angle-gather.
Compare with Figure~\ref{fig:SRx0-7}. }
% ------------------------------------------------------------
