\section{Imaging condition in wave-equation imaging}
A traditional imaging condition for shot-record
migration, often referred-to as $U \K{D}$ imaging condition 
\cite[]{Claerbout.blackwell.85},
consists of time cross-correlation at every image location 
between the source and receiver wavefields,
followed by image extraction at zero time:
\beqa
\UU  \lp \mm, t   \rp &=& 
\UR  \lp \mm, t   \rp \ast
\US  \lp \mm, t   \rp \;,
\\ 
\RR  \lp \mm      \rp &=&
\UU  \lp \mm,t=0  \rp \;,
\eeqa
where the symbol $\ast$ denotes cross-correlation in time.
Here, 
$\mm=\mv$ is a vector describing the locations of image points,
$\US(\mm,t)$ and 
$\UR(\mm,t)$ are source and receiver wavefields respectively, and 
$\RR(\mm)$ denotes a migrated image.
A final image is obtained by summation over shots.

\subsection{Space-shift imaging condition}
A generalized prestack imaging condition \cite[]{SavaFomel.pag}
estimates image reflectivity using cross-correlation in space and time,
followed by image extraction at zero time:
\beqa \label{eqn:imgX}
\UU  \lp \mm,\hh, t  \rp &=&
\UR  \lp \mm+\hh, t  \rp \ast
\US  \lp \mm-\hh, t  \rp \;,
\\   \label{eqn:imgXb}
\RR  \lp \mm,\hh     \rp &=&
\UU  \lp \mm,\hh,t=0 \rp \;.
\eeqa
Here, $\hh=\hv$ is a vector describing the space-shift
between the source and receiver wavefields prior to imaging.
Special cases of this imaging condition
are horizontal space-shift \cite[]{GEO67-03-08830889} and
vertical space-shift \cite[]{GEO69-05-12831298}.

For computational reasons,
this imaging condition is usually implemented in the
Fourier domain using the expression
\beq \label{eqn:imgXw}
   \RR \lp \mm,\hh    \rp = \sum_\w
   \UR \lp \mm+\hh,\w \rp
\K{\US}\lp \mm-\hh,\w \rp \;.
\eeq
The $^*$ sign represents a complex conjugate
applied on the receiver wavefield $\US$ in the
Fourier domain.

\subsection{Time-shift imaging condition}
Another possible imaging condition, advocated in this paper,
involves shifting of the source and receiver
wavefields in time, as opposed to space,
followed by image extraction at zero time:
\beqa \label{eqn:imgT}
\UU  \lp \mm,t,\tt \rp &=&
\UR  \lp \mm,t+\tt \rp \ast
\US  \lp \mm,t-\tt \rp \;,
\\   \label{eqn:imgTb}
\RR  \lp \mm,\tt     \rp &=&
\UU  \lp \mm,\tt,t=0 \rp \;.
\eeqa
Here, $\tt$ is a scalar describing the time-shift 
between the source and receiver wavefields prior to imaging.
This imaging condition can be implemented in the
Fourier domain using the expression
\beq \label{eqn:imgTw}
   \RR \lp \mm,\tt \rp = \sum_\w
   \UR \lp \mm,\w  \rp
\K{\US}\lp \mm,\w  \rp e^{2i\w\tt} \;,
\eeq
which simply involves a phase-shift applied 
to the wavefields prior to summation 
over frequency $\w$ for imaging at zero time.

\subsection{Space-shift and time-shift imaging condition}

To be even more general, we can formulate an imaging condition
involving both space-shift and time-shift,
followed by image extraction at zero time:
\beqa \label{eqn:imgA}
\UU  \lp \mm,\hh, t     \rp &=&
\UR  \lp \mm+\hh, t+\tt \rp \ast
\US  \lp \mm-\hh, t-\tt \rp \;,
\\   \label{eqn:imgAb}
\RR  \lp \mm,\hh,\tt      \rp &=&
\UU  \lp \mm,\hh,\tt,t=0  \rp \;.
\eeqa
However, the cost involved in this transformation is large,
so this general form does not have immediate practical value.
Imaging conditions described by equations~
\ren{imgX}-\ren{imgXb} and 
\ren{imgT}-\ren{imgTb} are special cases of 
equations~\ren{imgA}-\ren{imgAb} 
for $\hh=0$ and $\tt=0$, respectively.

