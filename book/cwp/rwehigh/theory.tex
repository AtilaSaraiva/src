\section{Extrapolation kernels}
Extrapolation using \req{rwemain} implies that the coefficients
defining the problem, $a$ and $b$, are not changing spatially. In this
case, we can perform extrapolation using a simple phase-shift
operation
%
\beq
\UU_{\tt+\Delta\tt} = \UU_{\tt} e^{i \kt \Delta\tt} \;,
\eeq
%
where $\UU_{\qt+\Delta\qt}$ and $\UU_{\qt}$ represent the acoustic
wavefield at two successive extrapolation steps, and $\kt$ is the
extrapolation wavenumber defined by \req{rwemain}.  \par For media
with lateral variability of the coefficients $a$ and $b$, due to
either velocity variation or focusing/defocusing of the coordinate
system, we cannot use in extrapolation the wavenumber computed
directly using \req{rwemain}. Like for the case of extrapolation in
Cartesian coordinates, we need to approximate the wavenumber $\kt$
using expansions relative to $a$ and $b$. Such approximations can be
implemented in the space-domain, in the Fourier domain or in mixed
space-Fourier domains.

\subsection{Space-domain extrapolation}
The space-domain finite-differences solution to \req{rwemain} is
derived based on a square-root expansion as suggested by Francis Muir
\cite[]{Claerbout.iei}:
%
\beq \label{eqn:rwexfd}
\kt   \approx  \ww a + \ww \frac{    \nu \yys }
                                {\mu-\ro \yys } \;,
\eeq
%
where the coefficients $\mu$, $\nu$ and $\ro$ take the form derived in Appendix A:
%
\beqa
\nu &=& - \co a \bas \;,
\\
\mu &=&   1          \;,
\\
\ro &=&   \ct   \bas \;. 
\eeqa
%
In the special case of Cartesian coordinates, $a=s$ and $b=1$,
\req{rwexfd} takes the familiar form
%
\beq
\kt   \approx  \ww s - \ww \frac{    \frac{\co}{s}   \yys }
                                {1  -\frac{\ct}{s^2} \yys } \;,
\eeq
%
where the coefficients $\co$ and $\ct$ take different values for
different orders of Muir's expansion: $(\co,\ct)=(0.50,0.00)$ for the
$15^\circ$ equation, and $(\co,\ct)=(0.50,0.25)$ for the $45^\circ$
equation, etc. For extrapolation in Riemannian coordinates, the
meaning of $15^\circ$, $45^\circ$ etc is not defined. We use this
terminology here to indicate orders of accuracy comparable to the ones
defined in Cartesian coordinates.

\subsection{Mixed-domain extrapolation}
Mixed-domain solutions to the one-way wave equation consist of
decompositions of the extrapolation wavenumber defined in
\req{rwemain} in terms computed in the Fourier domain for a reference
of the extrapolation medium, followed by a finite-differences
correction applied in the space-domain. For \req{rwemain}, a generic
mixed-domain solution has the form:
%
\beq \label{eqn:rwemixed}
\kt \approx \kto + \ww \lp a-\ao \rp + 
                   \ww \frac{    \nu \yys }
                            {\mu-\ro \yys } \;,
\eeq
%
where $\ao$ and $\bo$ are reference values for the medium
characterized by the parameters $a$ and $b$, and the coefficients
$\mu$, $\nu$ and $\ro$ take different forms according to the type of
approximation. As for usual Cartesian coordinates, $\kto$ is applied
in the Fourier domain, and the other two terms are applied in the
space domain. If we limit the space-domain correction to the thin lens
term, $\ww \lp a-\ao \rp$, we obtain the equivalent of the split-step
Fourier (SSF) method \cite[]{GEO55-04-04100421} in Riemannian
coordinates.
%
\par
%
Appendix A details the derivations for two types of expansions known
by the names of pseudo-screen \cite[]{GEO64-05-15241534}, and Fourier
finite-differences \cite[]{GEO59-12-18821893,GEO67-03-08720882}. Other
extrapolation approximations are possible, but are not described here,
for simplicity.

\begin{itemize}
\item {\bf Pseudo-screen method:}
\par
The coefficients for the pseudo-screen approximation to \req{rwemixed} are
\beqa \label{eqn:rwepsc}
\nu &=& \ao \lb \co \lp \frac{a}{\ao}-1 \rp - \lp \frac{b}{\bo}-1 \rp\rb \baos \;,
\\
\mu &=&   1 \;,
\\
\ro &=&3\ct\baos \;,
\eeqa
%
where $\ao$ and $\bo$ are reference values for the medium
characterized by parameters $a$ and $b$. In the special case of
Cartesian coordinates, $a=s$ and $b=1$, \req{rwemixed} with
coefficients \req{rwepsc} takes the familiar form
%
\beq
\kt \approx \kto + \ww
\lb 1+ \frac{    \frac{ \co}{s_0^2} \yys }
            {1  -\frac{3\ct}{s_0^2} \yys } 
\rb \lp s-s_0 \rp \;,
\eeq
%
where the coefficients $\co$ and $\ct$ take different values for
different orders of the finite-differences term:
$(\co,\ct)=(0.50,0.00)$, $(\co,\ct)=(0.50,0.25)$, etc. When
$(\co,\ct)=(0.00,0.00)$ we obtain the usual split-step Fourier
equation \cite[]{GEO55-04-04100421}.

\item {\bf Fourier finite-differences method:}
\par
%
The coefficients for the Fourier finite-differences solution to \req{rwemixed} are
%
\beqa \label{eqn:rweffd}
\nu &=& \hf\delta_1^2 \;,
\\
\mu &=&    \delta_1   \;,
\\
\ro &=& \qu\delta_2   \;,
\eeqa
where, by definition,
\beqa
\delta_1 &=& a\bas - \ao \baos \;,
\\
\delta_2 &=& a\baf - \ao \baof \;.
\eeqa
%
$\ao$ and $\bo$ are reference values for the medium characterized by
the parameters $a$ and $b$. In the special case of Cartesian
coordinates, $a=s$ and $b=1$, \req{rwemixed} with coefficients
\req{rweffd} takes the familiar form:
%
\beq
\kt \approx \kto + \ww
\lb 1+ \frac{          \frac{\co}{s s_0}   \yys }
            {1-\ct \lp \frac{1}{s^2}   + 
                       \frac{1}{s s_0} + 
                       \frac{1}{s_0^2} \rp \yys } 
\rb \lp s-s_0 \rp \;,
\eeq
%
where the coefficients $\co$ and $\ct$ take different values for
different orders of the finite-differences term:
$(\co,\ct)=(0.50,0.00)$ for $15^\circ$, $(\co,\ct)=(0.50,0.25)$ for
$45^\circ$, etc. When $\co=\ct=0.0$ we obtain the usual split-step
Fourier equation \cite[]{GEO55-04-04100421}.
\end{itemize}