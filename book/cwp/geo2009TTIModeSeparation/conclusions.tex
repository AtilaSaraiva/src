\section{Conclusions}
Different wave-modes in elastic media can be separated by projecting
the vector wavefields onto the polarization vectors of each mode.  For
heterogeneous models, it is necessary to separate wave-modes in the
space domain by non-stationary filtering.  I present a method for
obtaining spatially-varying wave-mode separators for TI models, which
can be used to separate elastic wave-modes in complex media. The
method computes the components of the polarization vectors in the
wavenumber domain and then transforms them to the space domain to
obtain spatially-varying filters. In order for the operators to work
in TI models with non-zero tilt angles, I incorporate one more
parameter---the local tilt angle $\nu$---in addition to the parameters
needed for the VTI operators. This kind of spatial filters can be used
to separate complicated wavefields in TI models with high
heterogeneity and strong anisotropy. I test the separation with
synthetic models that have realistic geologic complexity. The results
support the effectiveness of wave-mode separation with non-stationary
filtering.

I also extend the wave-mode separation to 3D TI models. The P-mode
separators can be constructed by solving the Christoffel equation for
the P-wave eigenvectors with local medium parameters.  The SV and SH
separators are constructed using the mutual orthogonality among P, SV,
and SH modes.  For the three modes, there are a total number of nine
separators, with three components for each mode. The separators vary
according to the medium parameters $V_{P0}$, $V_{S0}$, anisotropy
parameters $\epsilon$ and $\delta$ and tilt $\nu$ and azimuth $\alpha$
of the symmetry axis. The P-wave separators are constructed under no
kinematic assumptions, and amplitudes of P-mode correctly characterize
the plane-wave solution. Shear wave separators are constructed under
kinematic assumptions, and therefore the amplitudes of shear modes are
inaccurate in the singular directions. Nevertheless, the proposed
technique successfully separates fast and slow shear wavefields. The
process of constructing 3D separators and separating wave-modes in 3D
eliminates the step of decomposing the wavefields into symmetry
planes, which only works for models with an invariant symmetry
axis. Spatially-varying 3D separators have potential benefits for
complex models and can be used to separate wave-modes in elastic
reverse time migration (RTM) for TTI models. The spatially-varying 3D
separators imply large computational and storage cost, and therefore, a
more efficient separation method, such as the proposed table look-up
alternative, is necessary for a successful implementation.
