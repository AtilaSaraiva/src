

\section{Discussion}
\subsection{Computational issues}
%How to do separation for 3D heterogeneous TI media
The separation of wave-modes for heterogeneous TI models requires
non-stationary spatial filtering with large operators (operators of
$50$ samples in each dimension are used in this chapter), which is
computationally expensive. The cost is directly proportional to the
size of the model and to the size of each operator. Furthermore, in a
simple implementation, the storage for the separation operators of the
entire model is proportional to the size of the model and to the size
of each operator.  Suppose that a 3D elastic TTI model is
characterized by the model parameters $V_{P0}$, $V_{S0}$, Thomsen
parameters $\epsilon$ and $\delta$, and symmetry axis tilt angle $\nu$
and azimuth angle $\alpha$.  For a 3D model of $300 \times
300 \times 300$~grid points, if
one assumes that all operators have a size of $50 \times 50 \times
50$~samples, the storage for the operators is $300^3$~grid
points$~\times 50^3$~samples/independent operator$~\times 3$
independent operators/grid point$~\times 4$~Bytes/sample
$=40.5$~TB. This is not feasible in ordinary processing. However,
since there are relatively few medium parameters, i.e.,the
$V_{P0}/V_{S0}$ ratio, $\epsilon$, $\delta$, and angles $\nu$ and
$\alpha$, which determine the properties of the operators, one can
construct a look-up table of operators as a function of these
parameters, and search the appropriate operators at every location in
the model when doing wave-mode separation. For example, suppose one
knows that $V_{P0}/V_{S0}\in\lb1.5, 2.0\rb$, $\epsilon\in\lb0,
0.3\rb$, $\delta\in\lb0, 0.1\rb$, and the symmetry axis tilt angle
$\nu\in\lb-90^{\circ},90^{\circ}\rb$ and azimuth angle
$\alpha\in\lb-180^{\circ},180^{\circ}\rb$, one can sample the
$V_{P0}/V_{S0}$ ratio at every $0.1$, $\epsilon$ and $\delta$ at every
$0.03$, and the angles at every $15^{\circ}$. In this case, one only
needs a storage of $6 \times 10 \times 3 \times 12 \times 24$
combinations of medium parameters$\times 50^3$~sample/independent
operator$\times 3$~independent operators/combination of medium
parameters$\times$ $4$~Bytes/sample $=77$~GB; this is more manageable,
although it is still a large volume to store.

\subsection{S wave-mode amplitudes}
Although the procedure used in this chapter to separate S-waves into SV-
and SH- modes is simple, the amplitudes of S-modes are not accurate
because the S-wave separators are not normalized for any given
wavenumbers. The amplitudes of S-modes obtained in this way are zero
in the symmetry axis direction and they gradually increase to one in
the symmetry plane.

%However, since the symmetry axis direction usually corresponds to
%normal incidence of the elastic waves, it is important to obtain more
%accurate S-wave amplitudes in this direction.  
The main problem that prevents one from constructing the 3D global
shear wave separators is that the SV and SH polarization vectors are
singular in the symmetry axis direction, i.e., they are not defined by
the plane-wave solution of the TI elastic wave equation. Various
studies~\cite[]{Kieslev,Tsvankin} show that S-waves excited by point
forces can have non-linear polarizations in several special
directions. For example, in the direction of the source, the S-wave
can deviate from the linear polarization. This phenomenon exists even
in isotropic media.  Anisotropic velocity and amplitude variations can
also cause the S-waves to be polarized non-linearly.  For instance,
S-wave triplication, S-wave singularities, and S-wave velocity maximum
can all result in S-wave polarization anomalies. In these special
directions, SV- and SH-mode polarizations are incorrectly defined by
my convention.  One possibility for obtaining more accurate S-wave
amplitudes is to approximate the anomalous polarization with the major
axes of the quasi-ellipses of the S-wave polarization, which can be
obtained by incorporating the first-order term in the ray tracing
method. This extension remains outside the scope of this chapter.

Although the simplified approach used in this chapter ignores the complicated
polarization behavior in some wave propagation directions, it does
successfully separate fast and slow shear modes kinematically. This
allows one to use the separated scalar shear-modes for the subsequent
imaging condition and obtain images with clear physical meaning.
