\def\pcscwp{
Center for Wave Phenomena \\ 
Colorado School of Mines \\ 
psava@mines.edu
}

\def\pcscover{
\author[]{Paul Sava}
\institute{\pcscwp}
\date{}
\logo{WSI}
\large
}

\def\WSI{\textbf{WSI}~}

% ------------------------------------------------------------
% colors
\def\red#1{\textcolor{red}{#1}}
\def\green#1{\textcolor{green}{#1}}
\def\blue#1{\textcolor{blue}{#1}}
\def\yellow#1{\textcolor{yellow}{#1}}
\def\magenta#1{\textcolor{magenta}{#1}}

\def\black#1{\textcolor{black}{#1}}
\def\white#1{\textcolor{white}{#1}}
\def\gray#1{\textcolor{gray}{#1}}

\definecolor{DarkGreen}{rgb}{0,0.5,0}
\definecolor{DarkBlue}{rgb}{0,0,0.5}
\definecolor{DarkRed}{rgb}{0.5,0,0}
\definecolor{DarkYellow}{rgb}{0.5,0,0}
\definecolor{LightRed}{rgb}{1.000,0.752,0.796}
\definecolor{LightGreen}{rgb}{0.792,1.000,0.439}
\definecolor{LightBlue}{rgb}{0.690,0.886,1.000}
\definecolor{LightYellow}{rgb}{1.000,0.925,0.545}
\definecolor{DarkGray}{rgb}{0.45,0.45,0.45}
\definecolor{LightGray}{rgb}{0.90,0.90,0.90}

\def\darkgreen#1{\textcolor{DarkGreen}{#1}}
\def\darkblue#1 {\textcolor{DarkBlue}{#1}}
\def\darkred#1  {\textcolor{DarkRed}{#1}}
\def\lightred#1 {\textcolor{LightRed}{#1}}

\def\lightgray#1{\textcolor{LightGray}{#1}}
\def\darkgrey#1 {\textcolor{DarkGray}{#1}}

% ------------------------------------------------------------
% madagascar
\def\mg{\darkgreen{\sc madagascar~}}
\def\latex{\darkgreen{\sc \LaTeX}}
\def\mex#1{ \normalsize \red{ \texttt{#1} } \large }
\def\mvbt#1{\small{\blue{\begin{semiverbatim}#1\end{semiverbatim}}}}

% ------------------------------------------------------------
% equations
\def\bea{\begin{eqnarray}}
\def\eea{  \end{eqnarray}}

\def\beq{\begin{equation}}
\def\eeq{  \end{equation}}

%\def\req#1{(\ref{#1})}

\def\lp{\left (}
\def\rp{\right)}

\def\lb{\left [}
\def\rb{\right]}

\def\pbox#1{ \fbox {$ \displaystyle #1 $}}

\def\non{\nonumber \\ \nonumber}

% ------------------------------------------------------------
% REFERENCE (equations and figures)
\def\rEq#1{Equation~\ref{eqn:#1}}
\def\req#1{equation~\ref{eqn:#1}}
\def\rEqs#1{Equations~\ref{eqn:#1}}
\def\reqs#1{equations~\ref{eqn:#1}}
\def\ren#1{\ref{eqn:#1}}

\def\rFg#1{Figure~\ref{fig:#1}}
\def\rfg#1{Figure~\ref{fig:#1}}
\def\rFgs#1{Figures~\ref{fig:#1}}
\def\rfgs#1{Figures~\ref{fig:#1}}
\def\rfn#1{\ref{fig:#1}}

% ------------------------------------------------------------
% field operators

% trace
\def\tr{\texttt{tr}\;}

% divergence
\def\DIV#1{\nabla \cdot {#1}}

% curl
\def\CURL#1{\nabla \times {#1}}

% gradient
\def\GRAD#1{\nabla {#1}}

% Laplacian
\def\LAPL#1{\nabla^2 {#1}}

\def\dellin{
\lb
\begin{matrix}
\done{}{x} \; \done{}{y} \; \done{}{z}
\end{matrix}
\rb
}

\def\delcol{
\lb
\begin{matrix}
\done{}{x} \non
\done{}{y} \non
\done{}{z}
\end{matrix}
\rb
}

\def\aveclin{
\lb
\begin{matrix}
a_x \; a_y \; a_z
\end{matrix}
\rb
}


% ------------------------------------------------------------

% elastic tensor
\def\CC{{\bf C}}

% identity tensor
\def\I{\;{\bf I}}

% particle displacement vector
\def\uu{{\bf u}}

% particle velocity vector
\def\vv{{\bf v}}

% particle acceleration vector
\def\aa{{\bf a}}

% force vector
\def\ff{{\bf f}}

% wavenumber vector
\def\kk{{\bf k}}

% ray parameter vector
\def\pp{{\bf p}}

% distance vector
\def\xx{{\bf x}}
\def\kkx{{\kk_\xx}}
\def\ppx{{\pp_\xx}}

\def\yy{{\bf y}}

% normal vector
\def\nn{{\bf n}}
\def\ns{\nn_s}
\def\nr{\nn_r}

% source vector
\def\ss{{\bf s}}
\def\kks{{\kk_\ss}}
\def\pps{{\pp_\ss}}

% receiver vector
\def\rr{{\bf r}}
\def\kkr{{\kk_\rr}}
\def\ppr{{\pp_\rr}}

% midpoint vector
\def\mm{{\bf m}}
\def\kkm{{\kk_\mm}}
\def\ppm{{\pp_\mm}}

% offset vector
\def\ho{{\bf h}}
\def\kkh{{\kk_\ho}}
\def\pph{{\pp_\ho}}

% space-lag vector
\def\hh{ {\boldsymbol{\lambda}} }
\def\kkl{{\kk_\hh}}
\def\ppl{{\pp_\hh}}

% CIP vector
\def\cc{ {\bf c}}

% time-lag scalar
\def\tt{\tau}
\def\tts{\tt_s}
\def\ttr{\tt_r}

% frequency
\def\ww{\omega}

%
\def\dd{{\bf d}}

\def\bb{{\bf b}}
\def\qq{{\bf q}}

\def\ii{{\bf i}} % unit vector
\def\jj{{\bf j}} % unit vector

\def\lo{{\bf l}}

% ------------------------------------------------------------

\def\Fop#1{\mathcal{F}     \lb #1 \rb}
\def\Fin#1{\mathcal{F}^{-1}\lb #1 \rb}

% ------------------------------------------------------------
% partial derivatives

\def\dtwo#1#2{\frac{\partial^2 #1}{\partial #2^2}}
\def\done#1#2{\frac{\partial   #1}{\partial #2  }}
\def\dthr#1#2{\frac{\partial^3 #1}{\partial #2^3}}
\def\mtwo#1#2#3{ \frac{\partial^2#1}{\partial #2 \partial#3} }

\def\larrow#1{\stackrel{#1}{\longleftarrow}}
\def\rarrow#1{\stackrel{#1}{\longrightarrow}}

% ------------------------------------------------------------
% elasticity 

\def\stress{\underline{\textbf{t}}}
\def\strain{\underline{\textbf{e}}}
\def\stiffness{\underline{\underline{\textbf{c}}}}
\def\compliance{\underline{\underline{\textbf{s}}}}

\def\GEOMlaw{
\strain = \frac{1}{2} 
\lb \GRAD{\uu} + \lp \GRAD{\uu} \rp^T \rb
}

\def\HOOKElaw{
\stress = \lambda \; tr \lp \strain \rp {\bf I} + 2 \mu \strain 
}

\def\CONSTITUTIVElaw{
\stress = \stiffness \;\strain 
}


\def\NEWTONlaw{
\rho \ddot{\uu} = \DIV{\stress}
}

\def\NAVIEReq{
\rho \ddot\uu =
\lp \lambda + 2\mu \rp \GRAD{\lp \DIV{\uu} \rp}
             - \mu     \CURL{   \CURL{\uu}}
}

% ------------------------------------------------------------

% potentials
\def\VP{\boldsymbol{\psi}}
\def\SP{\theta}

% stress tensor
\def\ssten{{\bf \sigma}}

\def\ssmat{
\lp \matrix {
 \sigma_{11} &  \sigma_{12}   &  \sigma_{13} \cr
 \sigma_{12} &  \sigma_{22}   &  \sigma_{23} \cr
 \sigma_{13} &  \sigma_{23}   &  \sigma_{33} \cr
} \rp
}

% strain tensor
\def\eeten{{\bf \epsilon}}

\def\eemat{
\lp \matrix {
 \epsilon_{11} &  \epsilon_{12}   &  \epsilon_{13} \cr
 \epsilon_{12} &  \epsilon_{22}   &  \epsilon_{23} \cr
 \epsilon_{13} &  \epsilon_{23}   &  \epsilon_{33} \cr
} \rp
}


% plane wave kernel
\def\pwker{A e^{i k \lp \nn \cdot \xx - v t \rp}}


% ------------------------------------------------------------
% details for expert audience (math, cartoons)
\def\expert{
\colorbox{red}{\textbf{\LARGE \white{!}}}
}

% ------------------------------------------------------------
% image, data, wavefields

\def\RR{R}

\def\UU{W}
\def\US{{\UU_s}}
\def\UR{{\UU_r}}

\def\DD{D}
\def\DS{{\DD_s}}
\def\DR{{\DD_r}}

\def\UUw{\UU}
\def\USw{{\UU_s}}
\def\URw{{\UU_r}}

\def\DDw{\DD}
\def\DSw{{\DD_s}}
\def\DRw{{\DD_r}}

% perturbations

\def\ds{\Delta s}
\def\dl{\Delta l}
\def\di{\Delta \RR}
\def\du{\Delta \UU}

\def\dRR{\Delta \RR}
\def\dUU{\Delta \UU}
\def\dUS{\Delta \US}
\def\dUR{\Delta \UR}

\def\dtt{\Delta \tt}
\def\dhh{\Delta \hh}

% ------------------------------------------------------------
% Green's functions

\def\GG{G}

\def\GS{{\GG_s}}
\def\GR{{\GG_r}}

% ------------------------------------------------------------
% elastic data, wavefields

\def\eRR{\textbf{\RR}}

\def\eDS{{\textbf{\DD}_s}}
\def\eDR{{\textbf{\DD}_r}}
\def\eDD{{\textbf{\DD}}}

\def\eUS{{\textbf{\UU}_s}}
\def\eUR{{\textbf{\UU}_r}}
\def\eUU{{\textbf{\UU}}}

% ------------------------------------------------------------
% sliding bar
\def\tline#1{
\put(95,-3){\small \blue{time}}
\put(-4,-1){\small \blue{0}}
\thicklines
\put( 0,0){\color{blue} \vector(1,0){100}}
\put(#1,0){\color{red}  \circle*{2}}
}

% ------------------------------------------------------------
% arrow on figure
\def\myarrow#1#2#3{
\thicklines
\put(#1,#2){\color{green} \vector(-1,-1){5}}
\put(#1,#2){\color{green} \textbf{#3}}
}

\def\bkarrow#1#2#3{
\thicklines
\put(#1,#2){\color{black} \vector(-1,-1){5}}
\put(#1,#2){\color{black} \textbf{#3}}
}

\def\anarrow#1#2#3#4{
\thicklines
\put(#1,#2){\color{#4} \vector(-1,-1){5}}
\put(#1,#2){\color{#4} \textbf{#3}}
}

\def\myvec#1#2#3#4{
\thicklines
\put(#1,#2){\color{green} \rotatebox{#4}{\vector(4,0){20}}}
\put(#1,#2){\color{green} \textbf{#3}}
}

% ------------------------------------------------------------
% circle on figure
\def\mycircle#1#2#3{
\thicklines
\put(#1,#2){\color{green} \circle{#3}}
}

% ------------------------------------------------------------
% note on figure
\def\mynote#1#2#3{
\put(#1,#2){\color{green} \textbf{#3}}
}

\def\biglabel#1#2#3{
\put(#1,#2){\Large \textbf{#3}}
}

\def\wlabel#1#2#3{ \white{ \biglabel{#1}{#2}{#3} }}
\def\klabel#1#2#3{ \black{ \biglabel{#1}{#2}{#3} }}
\def\rlabel#1#2#3{ \red{   \biglabel{#1}{#2}{#3} }}
\def\glabel#1#2#3{ \green{ \biglabel{#1}{#2}{#3} }}
\def\blabel#1#2#3{ \blue { \biglabel{#1}{#2}{#3} }}
\def\ylabel#1#2#3{ \yellow{\biglabel{#1}{#2}{#3} }}

% ------------------------------------------------------------
% centering
\def\cen#1{ \begin{center} \textbf{#1} \end{center}}
\def\cit#1{ \begin{center} \textit{#1} \end{center}}
\def\ctt#1{ \begin{center} \texttt{#1} \end{center}}

% emphasis (bold+alert)
\def\bld#1{ \textbf{\alert{#1}}}

% huge fonts
\def\big#1{\begin{center} {\LARGE \textbf{#1}} \end{center}}
\def\hug#1{\begin{center} {\Huge  \textbf{#1}} \end{center}}

% ------------------------------------------------------------
% separator
\def\sep{ \vfill \hrule \vfill}
\def\itab{ \hspace{0.5in}}
\def\nsp{\\ \vspace{0.1in}}

% ------------------------------------------------------------
% integrals

\def\tint#1{\!\!\!\int\!\! #1 dt}
\def\xint#1{\!\!\!\int\!\! #1 d\xx}
\def\wint#1{\!\!\!\int\!\! #1 d\ww}
\def\aint#1{\!\!\!\alert{\int}\!\! #1 d\alert{\xx}}

\def\esum#1{\sum\limits_{#1}}
\def\eint#1{\int\limits_{#1}}

% ------------------------------------------------------------
\def\CONJ#1{\overline{#1}}
\def\MOD#1{\left| {#1} \right|}

% ------------------------------------------------------------
% imaging components

\def\IC{\colorbox{yellow}{\textbf{I.C.}}\;}
\def\WR{\colorbox{yellow}{\textbf{W.R.}}\;}
\def\WE{\colorbox{yellow}{\textbf{W.E.}}\;}
\def\SO{\colorbox{yellow}{\textbf{SOURCE}}\;}
\def\WS{\colorbox{yellow}{\textbf{W.S.}}\;}

% ------------------------------------------------------------
% summary/take home message
\def\thm{take home message}

% ------------------------------------------------------------
\def\dx{\Delta x}
\def\dy{\Delta y}
\def\dz{\Delta z}
\def\dt{\Delta t}

\def\dhx{\Delta h_x}
\def\dhy{\Delta h_y}

\def\kz{{k_z}}
\def\kx{{k_x}}
\def\ky{{k_y}}

\def\kmx{k_{m_x}}
\def\kmy{k_{m_y}}
\def\khx{k_{h_x}}
\def\khy{k_{h_y}}

\def\why{ \alert{\widehat{{\khy}}}}
\def\whx{ \alert{\widehat{{\khx}}}}

\def\lx{{\lambda_x}}
\def\ly{{\lambda_y}}
\def\lz{{\lambda_z}}

\def\klx{k_{\lambda_x}}
\def\kly{k_{\lambda_y}}
\def\klz{k_{\lambda_z}}

\def\mx{{m_x}}
\def\my{{m_y}}
\def\mz{{m_z}}
\def\hx{{h_x}}
\def\hy{{h_y}}
\def\hz{{h_z}}

\def\sx{{s_x}}
\def\sy{{s_y}}
\def\rx{{r_x}}
\def\ry{{r_y}}

% ray parameter (absolute value)
\def\modp#1{\left| \pp_{#1} \right|}

% wavenumber
\def\modk#1{\left| \kk_{#1} \right|}

% ------------------------------------------------------------
\def\kzwk{ {\kz^{\kk}}}
\def\kzwx{ {\kz^{\xx}}}
 
\def\PSk#1{e^{\red{#1 i \kzwk \dz}}}
\def\PSx#1{e^{\red{#1 i \kzwx \dz}}}
\def\PS#1{ e^{\red{#1 i k_z   \dz}}}

\def\TT{t}
\def\TS{t_s}
\def\TR{t_r}

\def\oft{\lp t \rp}
\def\ofw{\lp \ww \rp}

\def\ofx{\lp \xx \rp}
\def\ofh{\lp \hh \rp}
\def\ofk{\lp \kk \rp}
\def\ofs{\lp \ss \rp}
\def\ofr{\lp \rr \rp}
\def\ofz{\lp   z \rp}

\def\ofxt{\lp \xx, t  \rp}
\def\ofst{\lp \ss, t  \rp}
\def\ofrt{\lp \rr, t  \rp}

\def\ofxw{\lp \xx, \ww  \rp}
\def\ofsw{\lp \ss, \ww  \rp}
\def\ofrw{\lp \rr, \ww  \rp}

\def\ofxm{\lp \xx,\hh \rp}

\def\ofxmp{\lp \xx+\hh \rp}
\def\ofxmm{\lp \xx-\hh \rp}

\def\ofmm{\lp \mm      \rp}
\def\ofmz{\lp \mm, z   \rp}
\def\ofmw{\lp \mm, \ww \rp}
\def\ofkm{\lp \kkm     \rp}

% ------------------------------------------------------------
% source/receiver data and wavefields

\def\dst{$\DS\ofst$}
\def\drt{$\DR\ofrt$}
\def\ust{$\US\ofxt$}
\def\urt{$\UR\ofxt$}

\def\dsw{$\DS\ofsw$}
\def\drw{$\DR\ofrw$}
\def\usw{$\US\ofxw$}
\def\urw{$\UR\ofxw$}

% ------------------------------------------------------------
\def\Nx{N_x}
\def\Ny{N_y}
\def\Nz{N_z}
\def\Nt{N_t}
\def\Nw{N_{\ww}}
\def\Nm{N_{\mm}}

\def\Nlx{N_{\lambda_x}}
\def\Nly{N_{\lambda_y}}
\def\Nlz{N_{\lambda_z}}
\def\Nlt{N_{\tau}}

\def\wmin{\ww_{min}}
\def\wmax{\ww_{max}}
\def\zmin{z_{min}}
\def\zmax{z_{max}}
\def\tmin{t_{min}}
\def\tmax{t_{max}}
\def\lmin{\hh_{min}}
\def\lmax{\hh_{max}}
\def\xmin{\xx_{min}}
\def\xmax{\xx_{max}}

% ------------------------------------------------------------
% course qualifiers

\def\fun{\hfill \alert{concepts}}
\def\pra{\hfill \alert{applications}}
\def\fro{\hfill \alert{frontiers}}


% ------------------------------------------------------------
% wavefield extrapolation
\def\ws{ {\ww s} }

\def\kows{\lp \frac{\kx}{\ws} \rp}

\def\kmws{\lp \frac{\modk{\mm}}{\ws} \rp}
\def\kzws{\lp \frac{\kz}       {\ws} \rp}

\def\S{\lb\frac{\modk{\mm}}{\ws  }\rb}
\def\C{\lb\frac{\modk{\mm}}{\ws_0}\rb}
\def\K{\lb\frac{\modk{\mm}}{\ww  }\rb}

\def\Cs{\lb\frac{\modk{\mm}^2}{\lp \ws_0 \rp^2}\rb}

\def\SSR#1{  \sqrt{ \lp \ww {#1} \rp^2 - \modk{\mm}^2} }

\def\SQRsum#1{\sum\limits_{n=1}^{\infty} \lp -1 \rp^n
		\displaystyle{\frac{1}{2} \choose n} #1}

\def\TSE#1#2#3#4{\sum\limits_{#4=#3}^{\infty} \lp -1 \rp^#4
		\displaystyle{#2 \choose #4} {#1}^#4}

\def\onefrac#1#2{\frac{#2^2}{a_#1+b_#1 #2^2}}
\def\SQRfrac#1{
	\sum\limits_{n=1}^{\infty}
	\onefrac{n}{#1} }

\def\dkzds { \left. \frac{d {\kz}}  {d s} \right|_{s_b} }
\def\SSX#1#2{\sqrt{ 1 - \lb \frac{\MOD{#2}}{#1} \rb^2} }
\def\SST#1#2{1 + \sum_{j=1}^N c_j \lb \frac{\MOD{#2}}{#1} \rb^{2j} }

% ------------------------------------------------------------
% acknowledgment
\def\ackcwp{\cen{the sponsors of the\\Center for Wave Phenomena\\at\\Colorado School of Mines}}

% ------------------------------------------------------------
% citation in slides
\def\talkcite#1{{\small \sc #1}}

% ------------------------------------------------------------
\def\ise{GPGN302: Introduction to EM and Seismic Exploration}
\def\inv{GPGN409: Inversion}

% ------------------------------------------------------------
\def\model{m}
\def\data {d}

\def\Lop{ {\mathbf{L}}}
\def\Sop{ {\mathbf{S}}}
\def\Eop{ {\mathbf{E}}}
\def\Iop{ {\mathbf{I}}}
\def\Aop{ {\mathbf{A}}}
\def\Pop{ {\mathbf{P}}}
\def\Fop{ {\mathbf{F}}}
\def\Kop{ {\mathbf{K}}}


% ------------------------------------------------------------
\def\mybox#1{
  \begin{center}
    \fcolorbox{black}{yellow}
    {\begin{minipage}{0.8\columnwidth} {#1} \end{minipage}}
  \end{center}
}

\def\hibox#1{
  \begin{center}
    \fcolorbox{black}{LightGreen}
    {\begin{minipage}{0.8\columnwidth} {#1} \end{minipage}}
  \end{center}
}

% ------------------------------------------------------------
% Nota Bene
\def\nbnote#1{
  \vfill
  \begin{center}
    \colorbox{LightGray}
    {\begin{minipage}{\columnwidth} {\textbf{\black{\large N.B.}} #1} \end{minipage}}
  \end{center}
}

\def\notabene#1{
  \begin{leftbar}
    {\sc Nota Bene:~} #1
  \end{leftbar}
}

\def\sidebar#1{
  \begin{leftbar}
    {#1}
  \end{leftbar}
}


\def\highlight#1{
  \begin{center}
    \colorbox{LightRed}
    {\begin{minipage}{0.95\columnwidth} {#1} \end{minipage}}
  \end{center}
}

% ------------------------------------------------------------
\def\pcsshaded#1{
  \definecolor{shadecolor}{rgb}{0.8,0.8,0.8}
  \begin{shaded} {#1} \end{shaded}
  \definecolor{shadecolor}{rgb}{1.0,1.0,1.0}
}

\def\blueshade#1{
  \definecolor{shadecolor}{rgb}{0.690,0.886,1.000}
    \begin{shaded}
      {#1}
    \end{shaded}
  \definecolor{shadecolor}{rgb}{1.0,1.0,1.0}
}

\def\grayshade#1{
  \definecolor{shadecolor}{rgb}{0.8,0.8,0.8}
  \begin{shaded}
    {#1}
  \end{shaded}
  \definecolor{shadecolor}{rgb}{1.0,1.0,1.0}
}

\def\yellowshade#1{
  \definecolor{shadecolor}{rgb}{1.0,1.0,0.0}
  \begin{shaded}
    {#1}
  \end{shaded}
  \definecolor{shadecolor}{rgb}{1.0,1.0,1.0}
}


% ------------------------------------------------------------
\def\postit#1{
  \begin{center}
    \colorbox{yellow}
    {\begin{minipage}{0.80\columnwidth} {#1} \end{minipage}} 
  \end{center}
}

% ------------------------------------------------------------
\def\graybox#1{
  \begin{center}
    \colorbox{LightGray}
    {\begin{minipage}{1.00\columnwidth} {#1} \end{minipage}}
  \end{center}
}

\def\whitebox#1{
  \begin{center}
    \colorbox{white}
    {\begin{minipage}{1.00\columnwidth} {#1} \end{minipage}}
  \end{center}
}

\def\yellowbox#1{
  \begin{center}
    \colorbox{LightYellow}
    {\begin{minipage}{1.00\columnwidth} {#1} \end{minipage}}
  \end{center}
}

\def\greenbox#1{
  \begin{center}
    \colorbox{LightGreen}
    {\begin{minipage}{1.00\columnwidth} {#1} \end{minipage}}
  \end{center}
}

\def\bluebox#1{
  \begin{center}
    \colorbox{LightBlue}
    {\begin{minipage}{1.00\columnwidth} {#1} \end{minipage}}
  \end{center}
}

\def\redbox#1{
  \begin{center}
    \colorbox{LightRed}
    {\begin{minipage}{1.00\columnwidth} {#1} \end{minipage}}
  \end{center}
}

\def\hyellow#1{ \colorbox{yellow} #1 }
\def\hgreen #1{ \colorbox{green}  #1 }

% ------------------------------------------------------------
% boxes for vectors and matrices

\def\pcsbox#1#2#3#4{
  % #1 = hmax
  % #2 = height
  % #3 = width
  % #4 = text
  \begin{picture}(#3,#1)
    \linethickness{0.5mm}
    % 
    \multiput(0,#1)(#3, 0){2}{\line(0,-1){#2}}
    \multiput(0,#1)(0,-#2){2}{\line(+1,0){#3}}
    % 
    \put(1,-10){#4}
  \end{picture}
}

% annotate block equations
\def\pcssym#1#2{
  \begin{picture}(3,#1)
    \put(1,-10){#2}
  \end{picture}
}

% block equation sign
\def\pcsops#1#2#3#4{
  \begin{picture}(#3,#1)
    \put(0,#2){#4}
  \end{picture}
}

% ------------------------------------------------------------
% two color boxes
\def\sidebyside#1#2{
  \begin{center}
    \colorbox{LightBlue}{
      \begin{minipage}{1.0\columnwidth} {#1} \end{minipage}
    }
    \colorbox{LightYellow}{
      \begin{minipage}{1.0\columnwidth} {#2} \end{minipage}
    }
  \end{center}
}

% upward pointing arrow
\def\uparrow#1#2#3{
  \thicklines
  \put(#1,#2){\color{green} \vector(0,+1){5}}
  \put(#1,#2){\color{green} \textbf{#3}}
}

% acknowledge figure source
\def\ackfig#1#2#3{\blabel{#1}{#2}{\normalsize \sc #3}}

../macro.tex

% ------------------------------------------------------------
\title{Interferometric imaging condition for wave-equation migration}
\author{
Paul Sava (Colorado School of Mines) 
\\ and \\
Oleg Poliannikov (Massachusetts Institute of Technology)
}
\email{psava@mines.edu,poliann@erl.mit.edu}
\righthead{Interferometric imaging condition}
\lefthead{Sava and Poliannikov}
\ms{GEO-2006-0418}

\maketitle

% ------------------------------------------------------------
\begin{abstract}
The fidelity of depth seismic imaging depends on the accuracy of the
velocity models used for wavefield reconstruction. Models can be
decomposed in two components corresponding to large scale and small
scale variations. In practice, the large scale velocity model
component can be estimated with high accuracy using repeated
migration/tomography cycles, but the small scale component
cannot. When the Earth has significant small-scale velocity
components, wavefield reconstruction does not completely describe the
recorded data and migrated images are perturbed by artifacts.

There are two possible ways to address this problem: improve wavefield
reconstruction by estimating more accurate velocity models and image
using conventional techniques (e.g. wavefield cross-correlation), or
reconstruct wavefields with conventional methods using the known
background velocity model, but improve the imaging condition to
alleviate the artifacts caused by the imprecise reconstruction, which
is what we suggest in this paper.

We describe the unknown component of the velocity model as a random
function with local spatial correlations. Imaging data perturbed by
such random variations is characterized by statistical instability,
i.e. various wavefield components image at wrong locations that depend
on the actual realization of the random model. Statistical stability
can be achieved by pre-processing the reconstructed wavefields prior
to the imaging condition. We employ Wigner distribution functions to
attenuate the random noise present in the reconstructed wavefields,
parametrized as a function of image coordinates. Wavefield filtering
using Wigner distribution functions and conventional imaging can be
lumped-together into a new form of imaging condition which we call an
``interferometric imaging condition'' due to its similarity to
concepts from recent work on interferometry. The interferometric
imaging condition can be formulated both for zero-offset and for
multi-offset data, leading to robust and efficient imaging procedures
that are effective in attenuating imaging artifacts due to unknown
velocity models.
\end{abstract}

% ------------------------------------------------------------
\section{Introduction}
Seismic imaging in complex media requires accurate knowledge of the
medium velocity. Assuming single scattering (Born approximation),
imaging requires propagation of the recorded wavefields from the
acquisition surface, followed by the application of an imaging
condition highlighting locations where backscattering occurs,
i.e. where reflectors are present. Typically, this is achieved with
simple image processing techniques, e.g. cross-correlation of
wavefields reconstructed from sources and receivers.

The main requirement for good-quality imaging is accurate knowledge of
the velocity model. Errors in the model used for imaging lead to
inaccurate reconstruction of the seismic wavefields and to distortions
of the migrated images. In any realistic seismic field experiment the
velocity model is never known exactly. Migration velocity analysis
estimates large scale approximations of the model, but some fine scale
variations always remain elusive. For example, when geology includes
complicated stratigraphic structures or complex salt/carbonate bodies,
the rapid velocity variations on the scale of the seismic wavelength
and smaller cannot be estimated correctly by kinematic
methods. Therefore, even if the broad kinematics of the seismic
wavefields are reconstructed correctly, the extrapolated wavefields
also contain phase and amplitude distortions that lead to image
artifacts obstructing the image of the geologic structure under
consideration. While it is certainly true that even the recovery of a
long-wave background may prove to be a challenge in some
circumstances, we do not attempt to address that issue in this
paper. Instead, we concentrate solely on the problem of dealing with
the effect of a small scale random variations not estimated by
conventional methods.

There are two ways in which we can approach this problem. The first
option is to improve our velocity analysis methods to estimate the
small-scale variations in the model. Such techniques take advantage of
all information contained in seismic wavefields and are not limited to
kinematic information of selected events picked from the
data. Examples of techniques in this category are waveform inversion
\cite[]{tarantola.elsevier,GPR38-03-02870310,GPR38-03-03110312,GEO69-01-02310248},
wave-equation tomography \cite[]{GEO57-01-00150026} or wave-equation
migration velocity analysis
\cite[]{SavaBiondi.gp.wemva1,SavaBiondi.gp.wemva2,Shen:2005}. A more
accurate velocity model allows for more accurate wavefield
reconstruction. Then, wavefields can be used for imaging using
conventional procedures, e.g. cross-correlation.  The second option is
to concentrate on the imaging condition, rather than concentrate on
``perfect'' wavefield reconstruction. Assuming that the large-scale
component of the velocity models is known (e.g. by iterative
migration/tomography cycles), we can design imaging conditions that
are not sensitive to small inaccuracies of the reconstructed
wavefields. Imaging artifacts can be reduced at the imaging condition
step, despite the fact that the wavefields incorporate small kinematic
errors due to velocity fluctuations.

Of course, the two options are complementary to each other, and both
can contribute to imaging accuracy. In this paper, we concentrate on
the second approach. For purposes of theoretical analysis, it is
convenient to model the small-scale velocity fluctuations as random
but spatially correlated variations superimposed on a known
velocity. We assume that we know the background model, but that we do
not know the random fluctuations. The goal is to design an imaging
condition that alleviates artifacts caused by those random
fluctuations. Conventional imaging consists of cross-correlations of
extrapolated source and receiver wavefields at image locations. Since
wavefield extrapolation is performed using an approximation of the
true model, the wavefields contain random time delays, or equivalently
random phases, which lead to imaging artifacts.

One way of mitigating the effects of the random model on the quality
of the resulting image is to use techniques based on acoustic time
reversal \cite[]{Fink:1999}. Under certain assumptions, a signal sent
through a random medium, recorded by a receiver array, time reversed
and sent back through the same medium, refocuses at the source
location in a statistically stable fashion. Statistical stability
means that the refocusing properties (i.e. image quality) are
independent of the actual realization of the random medium
\cite[]{Papanicolaou:2004,Fouque:2005}.

We investigate an alternative way of increasing imaging statistical
stability. Instead of imaging the reconstructed wavefields directly,
we first apply a transformation based on Wigner distribution functions
\cite[]{Wigner.wdf} to the reconstructed wavefields. We consider a
special case of the Wigner distribution function (WDF) which has the
property that it attenuates random fluctuations from the wavefields
after extrapolation with conventional techniques. The idea for this
method is borrowed from image processing where WDFs are used for
filtering of random noise. Here, we apply WDFs to the reconstructed
wavefields, prior to the imaging condition. This is in contrast to
data filtering prior to wavefield reconstruction or to image filtering
after the application of an imaging condition.

Our procedure closely resembles conventional imaging procedures where
wavefields are extrapolated in the image volume and then
cross-correlated in time at every image location. Our method uses WDFs
defined in three-dimensional windows around image locations which
makes it both robust and efficient. From an implementation and
computational cost point of view, our technique is similar to
conventional imaging, but its statistical properties are
improved. Although conceptually separate, we can lump-together the WDF
transformation and conventional imaging into a new form of imaging
condition which resembles interferometric techniques
\cite[]{Papanicolaou:2004,Fouque:2005}. Therefore, we use the name
interferometric imaging condition for our technique to contrast it
with the conventional imaging condition.

A related method discussed in the literature is known under the name
of coherent interferometric imaging
\cite[]{Borcea.IP.2006,Borcea.GEO.2006,Borcea.Multiscale.2006}.  This
method uses similar local cross-correlations and averaging, but unlike
our method, it parametrizes reconstructed wavefields as a function of
receiver coordinates. Thus, the coherent interferometric imaging
functional requires separate wavefield reconstruction from every
receiver position, which makes this technique prohibitively expensive
and probably unusable in practice on large-scale seismic imaging
projects. In contrast, the imaging technique advocated in this paper
achieves similar statistical stability properties as coherent
interferometric imaging, but at an affordable computational cost since
we apply wavefield reconstruction only once for all receiver locations
corresponding to a given seismic experiment, typically a ``shot''.

% ------------------------------------------------------------
\section{Imaging conditions}

% ------------------------------------------------------------
\subsection{Conventional imaging condition}
Let $\DD{\xm,\tm}$ be the data recorded at time $\tm$ at receivers
located at coordinates $\xm$ for a seismic experiment with buried
sources (\rFg{parray}) also known as an exploding reflector seismic
experiment \cite[]{GPR24-02-03800399}. A conventional imaging
procedure for this type of data consists of two steps
\cite[]{Claerbout.iei}: wavefield reconstruction at image coordinates
$\ym$ from data recorded at receiver coordinates $\xm$, followed by an
imaging condition taking the reconstructed wavefield at time $\tm=0$
as the seismic image.

% ------------------------------------------------------------
\inputdir{XFig}
% ------------------------------------------------------------
\multiplot{2}{parray,aarray}{width=0.45\textwidth} {Zero-offset
seismic experiment sketch (a) and multi-offset seismic experiment
sketch (b). Coordinates $\xm=\{x,y\}$ characterize receiver positions
on the surface and coordinates $\ym=\{x,y,z\}$ characterize reflector
positions in the subsurface.}
% ------------------------------------------------------------

Mathematically, we can represent the wavefield $\V$ reconstructed at
 coordinates $\ym$ from data $\D$ recorded at coordinates $\xm$ as a
 temporal convolution of the recorded trace with the Green's function
 $\G$ connecting the two points:
%
\beq \label{eqn:wrec}
\VV{\xm,\ym,\tm} = \DD{\xm,\tm} *_t \GG{\xm,\ym,\tm} \;.
\eeq
%
The total wavefield $\U$ reconstructed at $\ym$ from all receivers is
the superposition of the wavefields $\V$ reconstructed from individual
traces
%
\beq \label{eqn:urec}
\UU{\ym,\tm} = \intxm \VV{\xm,\ym,\tm} \;,
\eeq
%
where the integral over $\xm$ spans the entire receiver space. As
stated, the imaging condition extracts the image $\IA{}{\ym}$ from the
wavefield $\UU{\ym,\tm}$ at time $\tm=0$, i.e.
%
\beq \label{eqn:cic}
\IA{}{\ym} = \UU{\ym,\tm=0} \;.
\eeq

The Green's function used in the procedure described by \req{wrec} can
be implemented in different ways. For our purposes, the actual method
used for computing Green's functions is not relevant. Any procedure
can be used, although different procedures will be appropriate in
different situations, with different cost of implementation. We assume
that a satisfactory procedure exists and is appropriate for the
respective velocity models used for the simulations. In our examples,
we compute Green's functions with time-domain finite-difference
solutions to the acoustic wave-equation, similar to reverse-time
migration \cite[]{GEO48-11-15141524}.

% ------------------------------------------------------------
\inputdir{sigs2}
% ------------------------------------------------------------

% ------------------------------------------------------------
\multiplot{2}{pvo,xm}{width=0.45\textwidth}{Velocity model (a)
and imaging target located around $z=8$~km (b). The shaded area is
imaged in \rFgs{sigs2-winpcic0}-\rfn{sigs2-winpiic1}.  }

\multiplot{6}{pwo-0,pwv-0,pwo-7,pwv-7,pdo,pdv}{width=0.45\textwidth}
{Seismic snapshots of acoustic wavefields simulated in the background
velocity model (a)-(c), and in the random velocity model (b)-(d).
Data recorded at the surface from simulation in the background
velocity model (e) and from the simulation in the random velocity
model (f).}
% ------------------------------------------------------------

Consider the velocity model depicted in \rFg{sigs2-pvo} and the
imaging target depicted in \rFg{sigs2-xm}. We assume that the model
with random fluctuations (\rFg{sigs2-pwv-0}) represents the real
subsurface velocity and use this model to simulate data. We consider
the background model (\rFg{sigs2-pwo-0}) to represent the migration
velocity and use this model to migrate the data simulated in the
random model. We consider one source located in the subsurface at
coordinates $z=8$~km and $x=13.5$~km, and receivers located close to
the top of the model at discrete horizontal positions and depth
$z=0.0762$~km. \rFg{sigs2-pwo-7}-\rfn{sigs2-pwv-7} show snapshots of
the simulated wavefields at a later time. The panels on the left
correspond to modeling in the background model, while the panels on
the right correspond to modeling in the random model.

\rFg{sigs2-pdv} shows the data recorded on the surface. The direct
wavefield arrival from the seismic source is easily identified in the
data, although the wavefronts are distorted by the random
perturbations in the medium. For comparison, \rFg{sigs2-pdo} shows
data simulated in the background model, which do not show random
fluctuations.

Conventional imaging using the procedure described above implicitly
states that data generated in the random model, \rFg{sigs2-pwv-0},
are processed as if they were generated in the background model,
\rFg{sigs2-pwo-0}. Thus, the random phase variations in the data are
not properly compensated during the imaging procedure causing
artifacts in the image. \rFg{sigs2-winpcic1} shows the image obtained
by migrating data from \rFg{sigs2-pdv} using the model from
\rFg{sigs2-pwo-0}. For comparison, \rFg{sigs2-winpcic0} shows the image 
obtained by migrating the data from \rFg{sigs2-pdo} using the same
model from \rFg{sigs2-pwo-0}.

Ignoring aperture effects, the artifacts observed in the images are
caused only by the fact that the velocity models used for modeling and
migration are not the same. Small artifacts caused by truncation of
the data on the acquisition surface can also be observed, but those
artifacts are well-known (i.e. truncation butterflies) and are not the
subject of our analysis. In this example, the wavefield reconstruction
procedure is the same for both modeling and migration
(i.e. time-domain finite-difference solution to the acoustic
wave-equation), thus it is not causing artifacts in the image. We can
conclude that the migration artifacts are simply due to the phase
errors between the Green's functions used for modeling (with the
random velocity) and the Green's functions used for migration (with
the background velocity). The main challenge for imaging in media with
random variations is to design procedures that attenuate the random
phase delays introduced in the recorded data by the unknown variations
of the medium without damaging the real reflections present in the
data.

% ------------------------------------------------------------
% ------------------------------------------------------------
% ------------------------------------------------------------

The random phase fluctuations observed in recorded data
(\rFg{sigs2-pdv}) are preserved during wavefield reconstruction using
the background velocity model.  We can observe the randomness in the
extrapolated wavefields in two ways, by reconstructing wavefields
using individual data traces separately, or by reconstructing
wavefields using all data traces at once.

The first option is to reconstruct the seismic wavefield at all image
locations $\ym$ from {\it individual} receiver positions on the
surface. Of course, this is not conventionally done in reverse-time
imaging, but we describe this concept just for illustration purposes.
\rFg{sigs2-uxx1} shows the wavefield reconstructed separately from
individual data traces depicted in \rFg{sigs2-pdv} using the
background model depicted in \rFg{sigs2-pwv-0}. In \rFg{sigs2-uxx1},
the horizontal axis corresponds to receiver positions on the surface,
i.e. coordinates $\xm$, and the vertical axis represents time.
According to the notations used in this paper, \rFg{sigs2-uxx1} shows
the wavefield $\VV{\xm,\ym,\tm}$ reconstructed to a particular image
coordinate $\ym$ from separate traces located on the surface at
coordinates $\xm$. A similar plot can be constructed for all other
image locations. Ideally, the reconstructed wavefield should line-up
at time $\tm=0$, but this is not what we observe in this figure,
indicating that the input data contain random phase delays that are
not compensated during wavefield reconstruction using the background
velocity.

The second option is to reconstruct the seismic wavefield at all image
locations $\ym$ from {\it all} receiver positions on the surface at
once. This is a conventional procedure for reverse-time imaging.
\rFg{sigs2-uyy1} shows the wavefield reconstructed from all
data traces depicted in \rFg{sigs2-pdv} using the background model
depicted in \rFg{sigs2-pwv-0}. In \rFg{sigs2-uyy1}, the vertical and
horizontal axes correspond to depth and horizontal positions around
the source, i.e. coordinates $\ym$, and the third cube axis represents
time. According to the notations used in this paper, \rFg{sigs2-uyy1}
corresponds to wavefield $\UU{\ym,\tm}$ reconstructed from data at all
receiver coordinates $\xm$ to image coordinates $\ym$.  The
reconstructed wavefield does not focus completely at the image
coordinate and time $\tm=0$ indicating that the input data contains
random phase delays that are not compensated during wavefield
reconstruction using the background velocity.

% ------------------------------------------------------------
% ------------------------------------------------------------
% ------------------------------------------------------------

% ------------------------------------------------------------
\subsection{Wigner distribution functions}
One possible way to address the problem of random fluctuations in
reconstructed wavefields is to use Wigner distribution functions
\cite[]{Wigner.wdf} to pre-process the wavefields prior to the 
application of the imaging condition. Appendix C provides a brief
introduction for readers unfamiliar with Wigner distribution
functions. More details about this topic are presented by
\cite{Cohen.timefrequencyanalysis}.

{\it Wigner distribution functions} (WDF) are bi-linear
representations of multi-dimensional signals defined in phase space,
i.e. they depend simultaneously on position-wavenumber ($\ym-\km$) and
time-frequency ($\tm-\wm$). \cite{Wigner.wdf} developed these concepts
in the context of quantum physics as probability functions for the
simultaneous description of coordinates and momenta of a given wave
function. WDFs were introduced to signal processing by
\cite{Ville.wdf} and have since found many applications in signal and
image processing, speech recognition, optics, etc.

A variation of WDFs, called {\it pseudo Wigner distribution functions}
are constructed using small windows localized in space and/or time
(Appendix C). Pseudo WDFs are simple transformations with efficient
application to multi-dimensional signals. In this paper, we apply the
pseudo WDF transformation to multi-dimensional seismic wavefields
obtained by reconstruction from recorded seismic data. We use pseudo
WDFs for decomposition and filtering of extrapolated space-time
signals as a function of their local wavenumber-frequency. In
particular, pseudo WDFs can filter reconstructed wavefields to retain
their coherent components by removing high-frequency noise associated
with random fluctuations in the wavefields due to random fluctuations
in the model.

The idea for our method is simple: instead of imaging the
reconstructed wavefields directly, we first filter them using pseudo
WDFs to attenuate the random phase noise, and then proceed to imaging
using a conventional or an extended imaging conditions.  Wavefield
filtering occurs during the application of the zero-frequency
end-member of the pseudo WDF transformation, which reduces the random
character of the field. For the rest of the paper, we use the
abbreviation WDF to denote this special case of pseudo Wigner
distribution functions, and not its general form.

As we described earlier, we can distinguish two options. The first
option is to use wavefield parametrization as a function of data
coordinates $\xm$. In this case, we can write the pseudo WDF of the
reconstructed wavefield $\VV{\xm,\ym,\tm}$ as
%
\beq \label{eqn:wxx}
\VA{x}{\xm,\ym,\tm} = \intth \intxh \lb
\VV{\xm-\frac{\xh}{2},\ym,\tm-\frac{\th}{2}}
\VV{\xm+\frac{\xh}{2},\ym,\tm+\frac{\th}{2}} 
\rb \;, 
\eeq 
%
where $\xh$ and $\th$ are variables spanning space and time intervals
of total extent $X$ and $T$, respectively. For 3D surface acquisition
geometry, the 2D variable $\xh$ is defined on the acquisition surface.
The second option is to use wavefield parametrization as a function of
image coordinates $\ym$. In this case, we can write the pseudo WDF of
the reconstructed wavefield $\UU{\ym,\tm}$ as
%
\beq \label{eqn:wyy}
\WA{y}{\ym,\tm} = \intth \intyh 
\UU{\ym-\frac{\yh}{2},\tm-\frac{\th}{2}}
\UU{\ym+\frac{\yh}{2},\tm+\frac{\th}{2}} \;,
\eeq
%
where $\yh$ and $\th$ are variables spanning space and time intervals
of total extent $Y$ and $T$, respectively. For 3D surface acquisition
geometry, the 3D variable $\yh$ is defined around image positions.

For the examples used in this section, we employ $41$ grid points for
the interval $X$ centered around a particular receiver position, $5
\times 5$ grid points for the interval $Y$ centered around a
particular image point, and $21$ grid points for the interval $T$
centered around a particular time. These parameters are not
necessarily optimal for the transformation, since they characterize
the local WDF windows and depend on the specific implementation of the
pseudo WDF transformation. The main criterion used for selecting the
size of the space-time window for the pseudo WDF transformation is
that of avoiding cross-talk between nearby events,
e.g. reflections. Finding the optimal size of this window is an
important consideration for our method, although its complete
treatment falls outside the scope of the current paper and we leave it
for future research. Preliminary results on optimal window selection
are discussed by \cite{Borcea.IP.2006}.

% ------------------------------------------------------------
\inputdir{sigs2}
% ------------------------------------------------------------

\multiplot{2}{uxx1,wxx1}{width=0.45\textwidth} {Reconstructed
seismic wavefield as a function of data coordinates (a) and its pseudo
Wigner distribution function (b) computed as a function of data
coordinates $\xm$ and time $\tm$. The wavefield is reconstructed using
the background model from data simulated in the random model.}

% ------------------------------------------------------------

\multiplot{2}{uyy1,wyy1}{width=0.45\textwidth} {Reconstructed
seismic wavefield as a function of image coordinates (a) and its
pseudo Wigner distribution function (b) computed as a function of
image coordinates $\ym$ and time $\tm$.The wavefield is reconstructed
using the background model from data modeled in the random model.}

% ------------------------------------------------------------

\rFg{sigs2-wxx1} depicts the results of applying the pseudo WDF 
transformation to the reconstructed wavefield in \rFg{sigs2-uxx1}.
%
For the case of modeling in the random model and reconstruction in the
background model, the pseudo WDF attenuates the random character of
the wavefield significantly, \rFg{sigs2-wxx1}. The random character of
the reconstructed wavefield is reduced and the main events cluster
more closely around time $\tm=0$. Similarly, \rFg{sigs2-wyy1} depicts
the results of applying the pseudo WDF transformation to the
reconstructed wavefields in \rFg{sigs2-uyy1}.
%
For the case of modeling in the random model and reconstruction in the
background model, the pseudo WDF also attenuates the random character
of the wavefield significantly, \rFg{sigs2-wyy1}. The random character
of the reconstructed wavefields is also reduced and the main events
focus at the correct image location at time $\tm=0$.

% ------------------------------------------------------------
\subsection{Zero-offset interferometric imaging condition}

After filtering the reconstructed wavefields with pseudo WDFs, we can
perform imaging with normal procedures. For the case of wavefields
parametrized as a function of data coordinates, we obtain the total
wavefield at image coordinates by summing over receiver coordinates
$\xm$
%
\beq \label{eqn:wxx2}
\WA{x}{\ym,\tm} = \intxm \VA{x}{\xm,\ym,\tm}\;,
\eeq
%
followed by a conventional imaging condition extracting time $\tm=0$
from the pseudo WDF of the reconstructed wavefields:
%
\beq \label{eqn:ixx}
\IA{x}{\ym} = \WA{x}{\ym,\tm=0} \;.
\eeq
%
The image obtained with this imaging procedure is shown in
\rFg{sigs2-winpcii1}. As expected, the artifacts caused by the unknown
random fluctuations in the model are reduced, leaving a
cleaner image of the source.

Similarly, for the case of wavefields parametrized as a function of
image coordinates, we obtain the image by application of the
conventional imaging condition extracting time $\tm=0$ from the pseudo
WDF of the reconstructed wavefield:
%
\beq \label{eqn:iyy}
\IA{y}{\ym} = \WA{y}{\ym,\tm=0} \;.
\eeq
%
The image obtained with this imaging procedure is shown in
\rFg{sigs2-winpiic1}. As in the preceding case, the artifacts caused
by the unknown random fluctuations in the model are reduced, producing
a cleaner image of the source, comparable with the one in
\rFg{sigs2-winpcii1}.

% ------------------------------------------------------------
\inputdir{sigs2}
% ------------------------------------------------------------

\multiplot{4}{winpcic0,winpcic1,winpcii1,winpiic1}{width=0.2\textwidth}
{ Images produced by the conventional imaging condition using the data
simulated in the background model (a) and using the data simulated in
the random model (b). Images produced from data simulated in the
random model using the interferometric imaging condition with
parametrization as a function of data coordinates (c) and as a
function of image coordinates (d).}
% ------------------------------------------------------------

% ------------------------------------------------------------
\subsection{Multi-offset interferometric imaging condition}

The imaging procedure in \reqs{wyy}-\ren{iyy} can be generalized for
imaging prestack (multi-offset) data (\rFg{aarray}). The conventional
imaging procedure for this type of data consists of two steps
\cite[]{Claerbout.iei}: wavefield simulation from the source location
to the image coordinates $\ym$ and wavefield reconstruction at image
coordinates $\ym$ from data recorded at receiver coordinates $\xm$,
followed by an imaging condition evaluating the match between the
simulated and reconstructed wavefields.

Let $\UA{S}{\ym,\tm}$ be the source wavefield constructed from the
location of the seismic source and $\UA{R}{\ym,\tm}$ the receiver
wavefield reconstructed from the receiver locations. A conventional
imaging procedure produces a seismic image as the zero-lag of the time
cross-correlation between the source and receiver
wavefields. Mathematically, we can represent this operation as
%
\beq
\IA{}{\ym} = \inttm \UA{S}{\ym,\tm} \UA{R}{\ym,\tm} \;,
\eeq
%
where $\IA{}{\ym}$ represents the seismic image for a particular
seismic experiment at coordinates $\ym$. When multiple seismic
experiments are processed, a complete image is obtained by summation
of the images constructed for individual experiments. The actual
reconstruction methods used to produce the wavefields
$\UA{S}{\ym,\tm}$ and $\UA{R}{\ym,\tm}$ are irrelevant for the present
discussion. As in the zero-offset/exploding reflector case, we use
time-domain finite-difference solutions to the acoustic wave-equation,
but any other reconstruction technique can be applied without changing
the imaging approach.

When imaging in random media, the data recorded at the surface
incorporates phase delays caused by the velocity variations
encountered while waves propagate in the subsurface. In a typical
seismic experiment, random phase delays accumulate both on the way
from the source to the reflectors, as well as on the way from the
reflectors to the receivers. Therefore, the receiver wavefield
reconstructed using the background velocity model is characterized by
random fluctuations, similar to the ones seen for wavefields
reconstructed in the zero-offset situation. In contrast, the source
wavefield is simulated in the background medium from a known source
position and, therefore, it is not affected by random
fluctuations. However, the zero-lag of the cross-correlations between
the source wavefields (without random fluctuations) and the receiver
wavefield (with random fluctuations), still generates image artifacts
similar to the ones encountered in the zero-offset case.

Statistically stable imaging using pseudo WDFs can be obtained in this
case, too. What we need to do is attenuate the phase errors in the
reconstructed receiver wavefield and then apply a conventional imaging
condition. Therefore, a multi-offset interferometric imaging condition
can be formulated as
%
\beq
\IA{}{\ym} = \inttm \UA{S}{\ym,\tm} \WA{R}{\ym,\tm} \;,
\eeq
%
where $\WA{R}{\ym,\tm}$ represents the pseudo WDF of the receiver
wavefield $\UA{R}{\ym,\tm}$ which can be constructed, in principle,
either with parametrization relative to data coordinates, according to
\reqs{wxx}-\ren{wxx2}, or relative to image coordinates, according to
\req{wyy}. Of course, our choice is to use image-space parametrization
for computational efficiency reasons.

% ------------------------------------------------------------
\subsection{Discussion}

The strategies described in the preceding section have notable
similarities and differences.
% 
The imaging procedures \ren{wxx}-\ren{wxx2}-\ren{ixx} and
\ren{wyy}-\ren{iyy} are similar in that they employ wavefields
reconstructed from the surface data in similar ways. Neither method
uses the surface recorded data directly, but they use wavefields
reconstructed from those data as boundary conditions to numerical
solutions of the acoustic wave-equation. The actual wavefield
reconstruction procedure is identical in both cases.

The techniques are different because imaging with
\reqs{wxx}-\ren{wxx2}-\ren{ixx} employs independent wavefield
reconstruction from receiver locations $\xm$ to image locations
$\ym$. In practice, this requires separately solving the acoustic
wave-equation, e.g. by time-domain finite-differences, from all
receiver locations on the surface. Such computational effort is often
prohibitive in practice. In contrast, imaging with
\reqs{wyy}-\ren{iyy} is similar to conventional imaging
because it requires only one wavefield reconstruction using all
recorded data at once, i.e. only one solution to the acoustic
wave-equation, similar to conventional shot-record migration.

The techniques \ren{wxx}-\ren{wxx2}-\ren{ixx} and \ren{wyy}-\ren{iyy}
are similar in that they both employ noise suppression using pseudo
Wigner distribution functions. However, the methods are parametrized
differently, the former relative to data coordinates with 2D local
space averaging and the later relative to image coordinates with 3D
local space averaging.

The imaging functionals presented in this paper are described as
functions of space coordinates, $\xm$ or $\ym$, and time, $\tm$. As
suggested in Appendix C, pseudo WDFs can be implemented either in time
or frequency, so potentially the imaging conditions discussed in this
paper can also be implemented in the frequency-domain. However, we
restrict our attention in this paper to the time-domain implementation
and leave the frequency-domain implementation subject to future study.

\rEqs{wxx}-\ren{wxx2}-\ren{ixx} can be collected into the zero-offset 
imaging functional
%
\beq \label{eqn:cint} 
\IA{CINT}{\ym} = \delta\lp t\rp 
\intxm \intth \intxh 
\lb \VV{\xm-\frac{\xh}{2},\ym,\tm-\frac{\th}{2}}
    \VV{\xm+\frac{\xh}{2},\ym,\tm+\frac{\th}{2}} 
\rb \;, 
\eeq 
%
where the temporal $\delta$ function implements the zero time imaging
condition. A similar form can be written for the multi-offset
case. \rEq{cint} corresponds to the time-domain version of the
coherent interferometric functional proposed by
\cite{Borcea.IP.2006,Borcea.GEO.2006,Borcea.Multiscale.2006}. Consistent
with the preceding discussion, the cost required to implement this
imaging functional is often prohibitive for practical application to
seismic imaging problems.

% ------------------------------------------------------------
\section{Statistical stability}
The interferometric imaging condition described in the preceding
section is used to reduce imaging artifacts by attenuating the
incoherent energy corresponding to velocity errors, as illustrated in
\rFgs{sigs2-winpcic1} and \rfn{sigs2-winpiic1}. The random model used
for this example corresponds to the weak fluctuation regime, as
explained in Appendix A (characteristic wavelength of similar scale
with the random fluctuations in the medium and fluctuations with small
magnitude).

By statistical instability we mean that images obtained for different
realizations of random models with the identical statistics are
different. \rFgs{sigs2-pdv-r0}-\rfn{sigs2-pdv-r2} illustrate data
modeled for different realizations of the random model in
\rFg{sigs2-pwv-0}. The general kinematics of the data are the same,
but subtle differences exist between the various datasets due
to the random model variations.  Migration using a
conventional imaging condition leads to the images in
\rFgs{sigs2-winpcic-r0}-\rfn{sigs2-winpcic-r2} which also show
variations from one realization to another.  In
contrast, \rFgs{sigs2-winpiic-r0}-\rfn{sigs2-winpiic-r2} show images
obtained by the interferometric imaging condition in
\reqs{wyy}-\ren{iyy}, which are more similar to one-another since many
of the artifacts have been attenuated.

% ------------------------------------------------------------
\inputdir{sigs2}
% ------------------------------------------------------------
\multiplot{9}{pdv-r0,pdv-r1,pdv-r2,winpcic-r0,winpcic-r1,winpcic-r2,winpiic-r0,winpiic-r1,winpiic-r2}{width=0.30\textwidth}
{Illustration of statistical stability for the interferometric imaging
condition in presence of random model variations. Data modeled using
velocity with random variations of magnitude $\sigma=30\%$, for
different realizations of the noise model $n$. Images obtained by
conventional imaging (d)-(f) and images obtained by interferometric
imaging (g)-(i).}
% ------------------------------------------------------------

In typical seismic imaging problems, we cannot ensure that random
velocity fluctuations are small (e.g. $\sigma\le 5\%$). It is
desirable that imaging remains statistically stable even in cases when
velocity varies with larger magnitude. We investigate the statistical
properties of the imaging functional in \reqs{wyy}-\ren{iyy} using
numerical experiments similar to the one used earlier. We describe the
random noise present in the velocity models using the following
parameters explained in Appendix A: seismic spatial wavelength
$\lambda=76.2$~m, wavelet central frequency $\omega=20$~Hz, random
fluctuations parameters: $r_a=0.0762$, $r_c=0.0762$, $\alpha=2$, and
random noise magnitude $\sigma$ between $15\%$ and $45\%$. This
numerical experiment simulates a situation that mixes the theoretical
regimes explained in Appendix B: random model fluctuations of
comparable scale with the seismic wavelength lead to destruction of
the wavefronts, as suggested by the ``weak fluctuations'' regime;
large magnitude of the random noise leads to diffusion of the
wavefronts, as suggested by the ``diffusion approximation''
regime. This combination of parameters could be regarded as a
worst-case-scenario from a theoretical standpoint.

\rFgs{sigs2-pdv-m0}-\rfn{sigs2-pdv-m2} show data simulated in models
similar to the one depicted in \rFg{sigs2-pwv-0}, but where the random
noise component is described by $\sigma=15,30,45\%$, respectively. As
expected, the wavefronts recorded at the surface are increasingly
distorted to the point where some of the later arrival are not even
visible in the data.

Migration using a conventional imaging condition leads to the images in
\rFgs{sigs2-winpcic-m0}-\rfn{sigs2-winpcic-m2}.  As expected, the
images show stronger artifacts due to the larger defocusing caused by
the unknown random fluctuations in the model.  However, migration
using the interferometric imaging condition leads to the images in
\rFgs{sigs2-winpiic-m0}-\rfn{sigs2-winpiic-m2}.  Artifacts are
significantly reduced and the images are much better focused.

% ------------------------------------------------------------
\multiplot{9}{pdv-m0,pdv-m1,pdv-m2,winpcic-m0,winpcic-m1,winpcic-m2,winpiic-m0,winpiic-m1,winpiic-m2}{width=0.30\textwidth}
{Illustration of interferometric imaging condition robustness in
presence of random model variations. Data modeled using velocity with
random variations of magnitudes $\sigma=15\%$ (a), $\sigma=30\%$ (b),
and $\sigma=45\%$ (c). Images obtained by conventional imaging (d)-(f)
and images obtained by interferometric imaging (g)-(i).}
% ------------------------------------------------------------


% ------------------------------------------------------------
\section{Multi-offset imaging examples}
There are many potential applications for this interferometric imaging
functional. One application we illustrate in this paper is imaging of
complex stratigraphy through a medium characterized by unknown random
variations. In this situation, accurate imaging using conventional
methods requires velocity models that incorporate the small scale
(random, as we view them) velocity variations. However, practical
migration velocity analysis does not produce models of this level of
accuracy, but approximates them with smooth, large-scale fluctuations
one order of magnitude larger than that of the typical seismic
wavelength.  Here, we study the impact of the unknown (random)
component of the velocity model on the images and whether
interferometric imaging increases the statistical stability of the
image.

For all our examples, we extrapolate wavefields using time-domain
finite-differences both for modeling and for migration. Thus, we
simulate a reverse-time imaging procedure, although the theoretical
results derived in this paper apply equally well to other wavefield
reconstruction techniques, e.g. downward continuation, Kirchhoff
integral methods, etc. The parameters used in our examples, explained
in Appendix A, are: seismic spatial wavelength $\lambda=76.2$~m,
wavelet central frequency $\omega=20$~Hz, random fluctuations
parameters: $r_a=76.2$~m, $r_c=76.2$~m, $\alpha=2$, and random noise
magnitude $\sigma=20\%$.

Consider the model depicted in \rFgs{sact1-awo-5}-\rfn{sact1-awv-7}.
As in the preceding example, the left panels depict the known smooth
velocity $v_0$, and the right panels depict the model with random
variations. The imaging target is represented by the oblique lines,
\rFg{sact1-xm}, located around $z=8$~km, which simulates a
cross-section of a stratigraphic model.

We model data with a random velocity model and image using the smooth
model. \rFgs{sact1-awo-5}-\rfn{sact1-awv-7} show wavefield snapshots
in the two models for different propagation times, one before the
source wavefields interact with the target reflectors and one after
this interaction. The propagating waves are affected the the random
fluctuations in the model both before and after their interaction with
the reflectors. \rFgs{sact1-ado} and \rfn{sact1-adv} show the
corresponding recorded data on the acquisition surface located at
$z=\lambda$, where $\lambda$ represents the wavelength of the source
pulse.

Migration with a conventional imaging condition of the data simulated
in the background model using the same velocity produces the image in
\rFg{sact1-winacic0}. The targets are well imaged, although the image
also shows artifacts due to truncation of the data on the acquisition
surface. In contrast, migration with the conventional imaging
condition of the data simulated in the random model using the
background velocity produces the image in \rFg{sact1-winacic1}. This
image is distorted by the random variations in the model that are not
accounted for in the background migration velocity. The targets are
harder to discern since they overlap with many truncation and
defocusing artifacts caused by the inaccurate migration velocity.

Finally, \rFg{sact1-winaiic1} shows the migrated image using the
interferometric imaging condition applied to the wavefields
reconstructed in the background model from the data simulated in the
random model. Many of the artifacts caused by the inaccurate velocity
model are suppressed and the imaging targets are more clearly visible
and easier to interpret. Furthermore, the general patterns of
amplitude variation along the imaged reflectors are similar between
\rFgs{sact1-winacic1} and \rfn{sact1-winaiic1}.

We note that the reflectors are not as well imaged as the ones
obtained when the velocity is perfectly known. This is because the
interferometric imaging condition described in this paper does not
correct kinematic errors due to inaccurate velocity. It only acts on
the extrapolated wavefields to reduce wavefield incoherency and add
statistical stability to the imaging process. Further extensions to
the interferometric imaging condition can improve focusing and enhance
the images by correcting wavefields prior to imaging. However, this
topic falls outside the scope of this paper and we do not elaborate on
it further.

% ------------------------------------------------------------
\inputdir{sact1} 
% ------------------------------------------------------------

\multiplot{2}{avo,xm}{width=0.45\textwidth}{Velocity model (a) and
imaging target located around $z=8$~km (b). The model consists of a
smooth version of the Sigsbee2A velocity, in order to avoid
backscattering during reverse-time migration. The shaded area is
imaged in \rFgs{sact1-winacic0}-\rfn{sact1-winaiic1}.}

\multiplot{6}{awo-5,awv-5,awo-7,awv-7,ado,adv}{width=0.45\textwidth}
{Seismic snapshots of acoustic wavefields simulated in the
background velocity model (a)-(c), and in the random velocity model
(b)-(d).  Data recorded at the surface from the simulation in the
background velocity model (e) and from the simulation in the random
velocity model (f).}

\multiplot{3}{winacic0,winacic1,winaiic1}{width=0.80\textwidth}{Image
produced using the conventional imaging condition from data simulated
in the background model (a) and from data simulated in the random
model (b). Image produced using the interferometric imaging condition
from data simulated in the random model (c). }

% ------------------------------------------------------------
\section{Conclusions}
We extend the conventional seismic imaging condition based on
wavefield cross-correlations to achieve statistical stability for
models with rapid, small-scale velocity variation. We assume that the
random velocity variations on a scale comparable with the seismic
wavelength are modeled by correlated Gaussian distributions.  Our
proposed interferometric imaging condition achieves statistical
stability by applying conventional imaging to the Wigner distribution
functions of the reconstructed seismic wavefields. The interferometric
imaging condition is a natural extension of the cross-correlation
imaging condition and adds minimally to the cost of migration. The
main characteristic of the method is that it operates on extrapolated
wavefields at image positions (thus the name interferometric imaging
condition), in contrast with costlier alternative approaches using
interferometry parametrized as a function of receiver coordinates.

% ------------------------------------------------------------
\section{Acknowledgments}
We acknowledge Associate Editor John Etgen for numerous comments and
suggestions which have significantly improved the manuscript. This
work is partially supported by the sponsors of the Center for Wave
Phenomena at Colorado School of Mines.

% ------------------------------------------------------------
\bibliographystyle{seg}
\bibliography{SEG2005,SEP,PCS,MISC}

% ------------------------------------------------------------
\appendix
\section{Appendix A}
\def\ofxyz{ \lp x,y,z \rp}

\subsection{Noise model}

Consider a medium whose behavior is completely defined by the acoustic
velocity, i.e. assume that the density $\rho\ofxyz = \rho_0$ is
constant and the velocity $v\ofxyz$ fluctuates around a homogenized
value $v_0 \lp x,y,z\rp$ according to the relation
%
\beq
\frac{1}                 {v^2   \ofxyz} = 
\frac{1 + \sigma m\ofxyz}{v_0^2 \ofxyz}\;,
\eeq
%
where the parameter $m$ characterizes the type of random fluctuations
present in the velocity model, and $\sigma$ denotes their strength.

Consider the covariance orientation vectors
%
\beqa
\aa &=& \lp a_x,a_y,a_z\rp^{\top} \in \mathbb{R}^3 \\
\bb &=& \lp b_x,b_y,b_z\rp^{\top} \in \mathbb{R}^3 \\
\cc &=& \lp c_x,c_y,c_z\rp^{\top} \in \mathbb{R}^3
\eeqa
%
defining a coordinate system of arbitrary orientation in space. Let
$r_a, r_b, r_c > 0$ be the covariance range parameters in the
directions of $\aa$,$\bb$ and $\cc$, respectively.

We define a covariance function
%
\beq
\cov \ofxyz = \exp \lb -l^{\alpha}\ofxyz \rb \;,
\eeq
%
where $\alpha \in [0,2]$ is a distribution shape parameter and
%
\beq
l \ofxyz = 
\sqrt{\lp \frac{\aa \cdot \rr}{r_a} \rp^2 +
      \lp \frac{\bb \cdot \rr}{r_b} \rp^2 +
      \lp \frac{\cc \cdot \rr}{r_c} \rp^2}
\eeq
%
is the distance from a point at coordinates $\rr=\lp x,y,z \rp$ to the
origin in the coordinate system defined by $\{r_a \aa, r_b \bb, r_c
\cc\}$.

Given the IID Gaussian noise field $n\ofxyz$, we obtain the random
noise $m\ofxyz$ according to the relation
%
\beq
m \ofxyz =
\mathscr{F}^{-1}
\lb \sqrt{\widehat{\cov} \lp k_x,k_y,k_z \rp} \;
          \widehat{   n} \lp k_x,k_y,k_z \rp 
\rb\;,
\eeq
%
where $k_x,k_y,k_z$ are wavenumbers associated with the spatial
coordinates $x,y,z$, respectively. Here,
%
\beqa
\widehat{\cov} &=& \mathscr{F} \lb \cov \rb
\\
\widehat{n}    &=& \mathscr{F} \lb n \rb
\eeqa
%
are Fourier transforms of the covariance function $\cov$ and the noise
$n$, $\mathscr{F}[\cdot]$ denotes Fourier transform, and
$\mathscr{F}^{-1}[\cdot]$ denotes inverse Fourier transform. The
parameter $\alpha$ controls the visual pattern of the field, and
$\aa,\bb,\cc, r_a,r_b,r_c$ control the size and orientation of a
typical random inhomogeneity.

% ------------------------------------------------------------
\appendix
\section{Appendix B}
\subsection{Wave propagation and scale regimes}

Acoustic waves characterized by pressure $p \lp x,y,z,t \rp$
propagate according to the second order acoustic wave-equation 
for constant density
%
\beqa
\dtwo{p}{t} = v^2 \nabla^2 p + F_{\lambda}(t) \;,
\eeqa
%
where $F_{\lambda}\lp t \rp$ is a wavelet of characteristic wavelength
$\lambda$.

Given the parameters $l$ (size of inhomogeneities), $\lambda$
(wavelength size), $L$ (propagation distance) and $\sigma$ (noise
strength), we can define several propagation regimes.

The {\it weak fluctuations regime} characterized by waves with
wavelength of size comparable to that of typical inhomogeneities
propagating over a medium with small fluctuations to a distance of
many wavelengths. This regime is characterized by negligible back
scattering, and the randomness impacts the propagating waves through
forward multipathing. The relevant length parameters are related by
%
\beq
l \sim \lambda \ll L \;,
\eeq
%
and the noise strength is assumed small
%
\beq
\sigma \ll 1 \;.
\eeq

The {\it diffusion approximation regime} characterized by waves with
wavelength much larger than that of typical inhomogeneities propagate
over a medium with strong fluctuations to a distance of many
wavelengths. This regime is characterized by traveling waves that are
statistically stable but diffuse with time. Back propagation of such
waves in a medium without random fluctuations results in loss of
resolution. The relevant length parameters are related by
%
\beq
l \ll \lambda \ll L \;,
\eeq
%
and the noise strength is not assumed small
%
\beq
\sigma \sim 1 \;.
\eeq

% ------------------------------------------------------------
\appendix
\section{Appendix C}
\subsection{Wigner distribution functions}

Consider the complex signal $u\lp \tm \rp$ which depends on time
$\tm$. By definition, its {\it Wigner distribution function} (WDF)
is \cite[]{Wigner.wdf}:
%
\beq \label{eqn:wigt}
W   \lp \tm,\wm \rp = \frac{1}{2\pi} \int
u^* \lp \tm-\frac{\th}{2} \rp
u   \lp \tm+\frac{\th}{2} \rp 
e^{-i\; \wm \; \th} 
d\th \;,
\eeq
%
where $\wm$ denotes temporal frequency, $\th$ denotes the relative
time shift of the considered signal relative to a reference time $\tm$
and the sign $^*$ denotes complex conjugation of complex signal $s$.
The same WDF can be obtained in terms of the spectrum $U\lp \wm \rp$
of the signal $u\lp \tm \rp$:
%
\beq \label{eqn:wigw}
W   \lp \tm,\wm \rp = \frac{1}{2\pi} \int
U^* \lp \wm+\frac{\wh}{2} \rp
U   \lp \wm-\frac{\wh}{2} \rp 
e^{-i\; \tm \; \wh} 
d\wh \;,
\eeq
%
where $\wh$ denotes the relative frequency shift of the considered
spectrum relative to a reference frequency $\wm$.  The time integral
in \req{wigt} or the frequency integral in \req{wigw} spans the entire
domain of time and frequency, respectively. When the interval is
limited to a region around the reference value, the transformation is
known as {\it pseudo Wigner distribution function}.

A special subset of the transformation \req{wigt} corresponds to zero
temporal frequency. For input signal $u\lp \tm \rp$, we obtain the
output Wigner distribution function $W\lp \tm \rp$ as
%
\beq \label{eqn:wigto}
W   \lp \tm \rp = \frac{1}{2\pi} \int
u^* \lp \tm-\frac{\th}{2} \rp
u   \lp \tm+\frac{\th}{2} \rp 
d\th \;.
\eeq

The WDF transformation can be generalized to multi-dimensional signals
of space and time. For example, for 2D real signals function of space,
$u\lp x,y\rp$, the zero-wavenumber pseudo WDF can be formulated as
%
\beq \label{eqn:wigxy}
W \lp x,y \rp = \frac{1}{4\pi^2} 
\int\limits_{|x_h| \le X} 
\int\limits_{|y_h| \le Y}
u   \lp x-\frac{x_h}{2}, y-\frac{y_h}{2} \rp
u   \lp x+\frac{x_h}{2}, y+\frac{y_h}{2} \rp 
d x_h d y_h \;,
\eeq
where $x_h$ and $y_h$ denote relative shift of the signal $s$ relative
to positions $x$ and $y$, respectively. In this particular form, the
pseudo WDF transformation has the property that it filters the input
of random fluctuations preserving in the output image the spatially
coherent components in a noise-free background.

For illustration, consider the model depicted in \rFg{circle-ss}. This
model consists of a smoothly-varying background with $25\%$ random
fluctuations. The acoustic seismic wavefield corresponding to a source
located in the middle of the model is depicted in
\rFg{circle-wfl-6}. This wavefield snapshot can be considered as the
random ``image''. The application of the 2D pseudo WDF transformation
to images shown in \rFg{circle-wfl-6} produces the image shown in
\rFg{circle-wdf-6}. We can make three observations on this image:
first, the random noise is strongly attenuated; second, the output
wavelet is different from the input wavelet, as a result of the
bi-linear nature of the pseudo WDF transformations; third, the
transformation is isotropic, i.e. it operates identically in all
directions.  The pseudo WDF applied to this image uses $11 \times 11$
grid points in the vertical and horizontal directions. As indicated in
the body of the paper, we do not discuss here the optimal selection of
the WDF window. Further details of Wigner distribution functions and
related transformations are discussed by
\cite{Cohen.timefrequencyanalysis}.

% ------------------------------------------------------------
\inputdir{circle}
% ------------------------------------------------------------

\multiplot{3}{ss,wfl-6,wdf-6}{width=0.30\textwidth}
{ Random velocity model (a), wavefield snapshot simulated in this
model by acoustic finite-differences (b), and its 2D pseudo Wigner
distribution function (c).}
