\begin{abstract}
Correctly propagating waves from overhanging reflectors is crucial for imaging in complex geology. This type of reflections are difficult or impossible to use in imaging using one-way downward continuation, because they violate an intrinsic assumption of this imaging method, i.e. vertical upward propagation of reflection data.
\par
Riemannian wavefield extrapolation is one of the techniques developed to address the limitations of one-way wavefield extrapolation in Cartesian coordinates. This method generalizes one-way wavefield extrapolation to general Riemannian coordinate system. Such coordinate systems can be constructed in different ways, one possibility being construction using ray tracing in a smooth velocity model from a starting plane in the imaged volume. This approach incorporates partially the propagation path into the coordinate system and leaves the balance for the one-way wavefield extrapolation operator. Thus, wavefield extrapolation follows overturning wave paths and extrapolated waves using low-order operators, which makes the extrapolation operation fast and robust.
\end{abstract}
