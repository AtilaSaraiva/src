\begin{abstract}
Multicomponent data are not usually processed with specifically
designed procedures, but with procedures analogous to the ones used
for single-component data. \geouline{In isotropic media}, the vertical
and horizontal components of the data are \geouline{commonly} taken as
proxies \geouline{for the} P- and S-wave modes which are imaged
independently with \geosout{the} acoustic wave
equation\geouline{s}. This procedure works only if the vertical and
horizontal component accurately represent P- and S-wave modes, which
is not true in general. Therefore, multicomponent images constructed
with this procedure exhibit artifacts caused by the incorrect wave
mode separation at the surface.

An alternative procedure for elastic imaging uses the \geouline{full}
vector fields for wavefield reconstruction and imaging. The wavefields
are reconstructed using the multicomponent data as a boundary
condition for a numerical solution to the elastic wave equation. The
key component for wavefield migration is the imaging condition that
evaluates the match between wavefields reconstructed from sources and
receivers. For vector wavefields, a simple component-by-component
cross-correlation between two wavefields leads to artifacts caused by
crosstalk between the unseparated wave modes. An alternative method is
to separate elastic wavefields after reconstruction in the subsurface
and implement the imaging condition as cross-correlation of pure wave
modes instead of the Cartesian components of the displacement
wavefield. This approach leads to images that are easier to interpret,
since they describe reflectivity of specified wave modes at interfaces
of physical properties.

As for imaging with acoustic wavefields, the elastic imaging condition
can be formulated conventionally (cross-correlation with zero lag in
space and time), as well as extended to non-zero space and time
lags. The elastic images produced by an extended imaging condition can
be used for angle decomposition of primary (PP or SS) and converted
(PS or SP) reflectivity. Angle gathers constructed with this procedure
have applications for migration velocity analysis and amplitude versus
angle analysis.
\end{abstract}
