\section{Angle decomposition}

% ------------------------------------------------------------
\plot{cwgat}{width=\textwidth}{(a) Model showing one shot over
multiple reflectors dipping at $0^\circ$, $15^\circ$, $30^\circ$,
$45^\circ$ and $60^\circ$. The vertical dashed line shows a CIG
location. Incidence ray is vertically down and P to S conversions are
marked by arrowed \geosout{liens} \geouline{lines} pointing away from
reflectors. (b) Converted wave angle gather obtained from algorithm
described by {\cite{sava:2460}}. Notice that converted wave angles are
always smaller than incidence angles (in this case, the dips of the
reflectors) except for normal incidence. }
% ------------------------------------------------------------

As indicated earlier, the main uses of images constructed using
extended imaging conditions are migration velocity analysis (MVA) and
amplitude versus angle analysis (AVA). Such \geosout{analysis}
\geouline{analyses}, however, require that the images be decomposed in
components corresponding to various angles of incidence \geosout{,
procedure which is often referred to as angle decomposition}. Angle
decomposition takes different forms corresponding to the type of
wavefields involved in imaging. Thus, we can distinguish angle
decomposition for scalar (acoustic) wavefields and angle decomposition
for vector (elastic) wavefields.

% ------------------------------------------------------------
\subsection{Scalar wavefields}

For the case of imaging with the acoustic wave equation, the
reflection angle corresponding to incidence and reflection of P-wave
mode can be constructed after imaging, using mapping based on the
relation \cite[]{sava:947}
%
\beq\label{eqn:PPang}
\tan\theta_a = \frac
{|\kk_\hh|  }
{|\kk_\xx|  } \;,
\eeq
%
where $\theta_a$ is the incidence angle, and $\kk_\xx = \kk_\rr-\kk_\ss
$ and $\kk_\hh = \kk_\rr+\kk_\ss$ are defined using the source and
receiver wavenumbers, $\kk_\ss$ and $\kk_\rr$. The information
required for decomposition of the reconstructed wavefields \geouline{as
  a} function of wavenumbers $\kk_\xx$ and $\kk_\hh$ is readily
available in the images $\IM{}\ofxlt$ constructed by extended imaging
conditions \reqs{XICa} or \ren{XICe}.  \geosout{Needless to say, after}
\geouline{After} angle decomposition, the image $\IM{}\ofxa$ represents a
mapping of the image $\IM{}\ofxlt$ from offsets to angles. In other
words, all information for characterizing angle-dependent reflectivity
is already available in the image obtained by the extended imaging
conditions.

% ------------------------------------------------------------
\subsection{Vector wavefields}

A similar approach can be used for decomposition of the reflectivity
as a function of incidence and reflection angles for elastic
wavefields imaged with extended imaging conditions \reqs{XICa} or
\ren{XICe}. The angle $\theta_e$ characterizing the average angle
between incidence and reflected rays can be computed using the
expression \cite[]{sava:947}
%%
\beq\label{eqn:PSang}
\tan^2\theta_e = \frac
{ \lp 1+\gamma \rp^{2} |\kk_\hh|^2 - \lp 1-\gamma \rp^{2} |\kk_\xx|^2 }
{ \lp 1+\gamma \rp^{2} |\kk_\xx|^2 - \lp 1-\gamma \rp^{2} |\kk_\hh|^2 } \; ,
\eeq
%%
where $\gamma$ is the velocity ratio of the incident and reflected
waves, e.g. $V_P/V_S$ ratio for incident P mode and reflected S mode.
\rFg{cwang} shows the schematic and the notations used in \req{PSang},
\geouline{where $\mathbf{|p_x|=|k_x|}/\omega$,
$\mathbf{|p_\lambda|=|k_\lambda|}/\omega$, and $\omega$ is the angular
frequency at the imaging location $\xx$.  } The angle decomposition
equation \ren{PSang} is designed for PS reflections and reduces to
\req{PPang} for PP reflections when $\gamma=1$.

Angle decomposition using \req{PSang} requires computation of
\geouline{an} extended imaging condition with 3D space lags
($\lambda_x,\lambda_y,\lambda_z$), which is computationally costly.
Faster computation can be done if we avoid computing the vertical lag
$\lambda_z$, in which case the angle decomposition can be done using
the expression \cite[]{sava:947}:
%
\beq\label{eqn:PSang_hor}
\tan\theta_e = \frac
{ \lp 1+\gamma\rp \lp\ahx+\bmx\rp }
{ 2\gamma\; \kmz+\sqrt{4\gamma^2\kmz^2+\lp\gamma^2-1\rp \lp\ahx+\bmx\rp \lp\amx+\bhx\rp} } \;,
\eeq
%
where 
$\ahx=\lp 1+\gamma\rp \khx$,
$\amx=\lp 1+\gamma\rp \kmx$,
$\bhx=\lp 1-\gamma\rp \khx$, and 
$\bmx=\lp 1-\gamma\rp \kmx$.
%
\rFg{cwgat} shows a model of five reflectors and the extracted angle
gathers for these reflectors at the location of the source.  For PP
reflections, they would occur in the angle gather at angles equal with
the reflector slopes. However, for PS reflections, as illustrated in
\rfg{cwgat}, the reflection angles are smaller than the reflector
slopes, as expected.
