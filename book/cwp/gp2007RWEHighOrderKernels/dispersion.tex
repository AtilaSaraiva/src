\section{Riemannian wavefield extrapolation}
Riemannian wavefield extrapolation \cite[]{SavaFomel.geo.rwe}
generalizes solutions to the Helmholtz equation of the acoustic
wave-equation
%
\beq \label{eqn:helmholtz}
\DEL \UU=-\ww^2 s^2 \UU \;,
\eeq 
%
to coordinate systems that are different from simple Cartesian, where
extrapolation is performed strictly in the downward direction. In
\req{helmholtz}, $s$ is slowness, $\ww$ is temporal frequency, and
$\UU$ is a monochromatic acoustic wave. The Laplacian operator
$\Delta$ takes different forms according to the coordinate system used
for discretization.
%
\par Assume that we describe the physical space in Cartesian
coordinates $x$, $y$ and $z$, and that we describe a Riemannian
coordinate system using coordinates $\qx$, $\qy$ and $\qz$. The two
coordinate systems are related through a mapping
%
\beqa \label{eqn:mapx}
x&=&x\lp \qx,\qy,\qz \rp
\\    \label{eqn:mapy}
y&=&y\lp \qx,\qy,\qz \rp
\\    \label{eqn:mapz}
z&=&z\lp \qx,\qy,\qz \rp
\eeqa
%
which allows us to compute derivatives of the Cartesian coordinates
relative to the Riemannian coordinates.
%
\par A special case of the mapping \ren{mapx}-\ren{mapz} is defined
when the Riemannian coordinate system is constructed by ray
tracing. The coordinate system is defined by traveltime $\qt$ and
shooting angles, for example. Such coordinate systems have the
property that they are semi-orthogonal, i.e. one axis is orthogonal on
the other two, although the later axes are not necessarily orthogonal
on one-another.
%
\par Following the derivation in \cite{SavaFomel.geo.rwe}, the
acoustic wave-equation in Riemannian coordinates can be written as:
%
\beq \label{eqn:rweacoustic}
\czz \dtwo{\UU}{\qz} +
\cxx \dtwo{\UU}{\qx} + 
\cyy \dtwo{\UU}{\qy} +
\cz  \done{\UU}{\qz} +
\cx  \done{\UU}{\qx} + 
\cy  \done{\UU}{\qy} +
\cxy \mtwo{\UU}{\qx}{\qy} = - \lp \ww s \rp^2 \UU \;,
\eeq
%
where coefficients $c_{ij}$ are functions of the coordinate system and
can be computed numerically for any given coordinate system mapping
\ren{mapx}-\ren{mapz}.
%
\par The acoustic wave-equation in Riemannian coordinates
\ren{rweacoustic} contains both first and second order terms, in
contrast with the normal Cartesian acoustic wave-equation which
contains only second order terms. We can construct an approximate
Riemannian wavefield extrapolation method by dropping the first-order
terms in \req{rweacoustic}. This approximation is justified by the
fact that, according to the theory of characteristics for second-order
hyperbolic equations \cite[]{courant}, the first-order terms affect
only the amplitude of the propagating waves. To preserve the
kinematics, it is sufficient to retain only the second order terms of
\req{rweacoustic}:
%
\beq \label{eqn:rwekinematic}
\czz \dtwo{\UU}{\qz} +
\cxx \dtwo{\UU}{\qx} + 
\cyy \dtwo{\UU}{\qy} +
\cxy \mtwo{\UU}{\qx}{\qy} = - \lp \ww s \rp^2 \UU \;.
\eeq
%
\par From \req{rwekinematic} we can derive the following dispersion
relation of the acoustic wave-equation in Riemannian coordinates
%
\beq \label{eqn:rwedispersion}
- \czz \kqz^2
- \cxx \kqx^2
- \cyy \kqy^2
- \cxy \kqx\kqy = - \lp \ww s \rp^2 \;,
\eeq
%
where $\kqz$, $\kqx$ and $\kqy$ are wavenumbers associated with the
Riemannian coordinates $\qz$, $\qx$ and $\qy$. Coefficients $\cxx$,
$\cyy$ and $\czz$ are known quantities defined using the coordinate
system mapping \ren{mapx}-\ren{mapz}. For one-way wavefield
extrapolation, we need to solve the quadratic \req{rwedispersion} for
the wavenumber of the extrapolation direction $\kqz$, and select the
solution with the appropriate sign for the desired extrapolation
direction:
%
\beq \label{eqn:rweoneway3d}
\kqz = 
\sqrt{
\frac{\lp\ww s\rp^2}{\czz}
- \frac{\cxx}{\czz}\kqx^2
- \frac{\cyy}{\czz}\kqy^2
- \frac{\cxy}{\czz}\kqx\kqy 
}\;.
\eeq
%
The 2D equivalent of \req{rweoneway3d} takes the form:
%
\beq \label{eqn:rweoneway2d}
\kqz = 
\sqrt{
\frac{\lp\ww s\rp^2}{\czz}
- \frac{\cxx}{\czz}\kqx^2
}\;.
\eeq
%
In ray coordinates, defined by $\qz \equiv \qt$ (propagation
time) and $\qx \equiv \qg$ (shooting angle), we can re-write
\req{rweoneway2d} as
%
\beq \label{eqn:rweoneway2d.t}
\kt = 
\sqrt{
  \lp\ww s \aa\rp^2
- \lp \frac{\aa}{\jj} \kk \rp^2
}\;,
\eeq
%
where $\aa$ represents velocity and $\jj$ represents geometrical
spreading. The quantities $\aa$ and $\jj$ characterize the
extrapolation coordinate system: $\aa$ describes the velocity used for
construction of ray coordinate system; $\jj$ describes the spreading
or focusing of the coordinate system. In general, the velocity used
for construction of the coordinate system is different from the
velocity used for extrapolation, as suggested by
\cite{SavaFomel.geo.rwe} and illustrated later in this paper.
%
\par
We can further simplify the computations by introducing the notation
%
\beqa \label{eqn:rwea}
a &=& s \aa \;,
\\ \label{eqn:rweb}
b &=& \frac{\aa}{\jj} \;,
\eeqa
%
thus \req{rweoneway2d.t} taking the form
%
\beq  \label{eqn:rwemain}
\kt = \sqrt{ \lp \ww a \rp^2 - \lp b\kk \rp^2} \;.
\eeq
%
For Cartesian coordinate systems, $\aa=1$ and $\jj=1$, \req{rwemain}
reduces to the known dispersion relation
%
\beq
k_z = \sqrt{\ww^2 s^2 - k_x^2} \;,
\eeq
%
where $k_z$ and $k_x$ are depth and position extrapolation wavenumbers.
