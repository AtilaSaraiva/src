\section{Introduction}
Riemannian wavefield extrapolation \cite[]{SavaFomel.geo.rwe}
generalizes solutions to the Helmholtz equation in general Riemannian
coordinate systems. Conventionally, the Helmholtz equation is solved
in Cartesian coordinates which represent special cases of Riemannian
coordinates. The main requirements imposed on the Riemannian
coordinate systems are that they maintain orthogonality between the
extrapolation coordinate and the other coordinates (2 in 3D, 1 in
2D). This requirement can be relaxed when using an even more general
form of RWE in non-orthogonal coordinates
\cite[]{Shragge.geo.nonlinear}. In addition, it is desirable that the
coordinate system does not triplicate, although numerical methods can
stabilize extrapolation even in such situations
\cite[]{SavaFomel.geo.rwe}. Thus, wavefield extrapolation in
Riemannian coordinates has the flexibility to be used in many
applications where those basic conditions are fulfilled. Cartesian
coordinate systems, including tilted coordinates, are special cases of
Riemannian coordinate systems.
%
\par Two straightforward applications of wave propagation in
Riemannian coordinates are extrapolation in a coordinate system
created by ray tracing in a smooth background velocity
\cite[]{SavaFomel.geo.rwe}, and extrapolation with a coordinate system
created by conformal mapping of a given geometry to a regular space,
for example migration from topography \cite[]{ShraggeSava.segab.2005}.
%
\par Coordinate systems created by ray tracing in a background medium
often well represent wavefield propagation. In this context, we
effectively split wave propagation effects into two parts: one part
accounting for the general trend of wave propagation, which is
incorporated in the coordinate system, and the other part accounting
for the details of wavefield scattering due to rapid velocity
variations. If the background medium is close to the real one, the
wave-propagation can be properly described with low-order
operators. However, if the background medium is far from the true one,
the wavefield departs from the general direction of the coordinate
system and the low-order extrapolators are not enough for accurate
description of wave propagation.
%
\par For coordinate system describing a geometrical property of the
medium (e.g. migration from topography), there is no guarantee that
waves propagate in the direction of extrapolation. This situation is
similar to that of Cartesian coordinates when waves propagate away
from the vertical direction, except that conformal mapping gives us
the flexibility to define any coordinates, as required by
acquisition. In this case, too, low-order extrapolators are not enough
for accurate description of wave propagation.  
%
\par Therefore, there is need for higher-order Riemannian wavefield
extrapolators in order to handle correctly waves propagating obliquely
relative to the coordinate system. Usually, the high-order
extrapolators are implemented as mixed operators, part in the Fourier
domain using a reference medium, part in the space domain as a
correction from the reference medium. Many methods have been developed
for high-order extrapolation in Cartesian coordinates. In this paper,
we explore some of those extrapolators in Riemannian coordinates, in
particular high-order finite-differences solutions
\cite[]{Claerbout.iei}, and methods from the pseudo-screen family
\cite[]{GEO64-05-15241534} and Fourier finite-differences family
\cite[]{GEO59-12-18821893,GEO67-03-08720882}. In theory, any other
high-order extrapolator developed in Cartesian coordinates can have a
correspondent in Riemannian coordinates.  
%
\par In this paper, we implement the finite-differences portion of the
high-order extrapolators with implicit methods. Such solutions are
accurate and robust, but they face difficulties for 3D implementations
because the finite-differences part cannot be solved by fast
tridiagonal solvers anymore and require more complex and costlier
approaches \cite[]{GEO63-05-15321541,SEG-1998-1124}. The problem of 3D
wavefield extrapolation is addressed in Cartesian coordinates either
by splitting the one-way wave-equation along orthogonal directions
\cite[]{GEO62-02-05540567}, or by explicit numerical solutions
\cite[]{GEO56-11-17701777}. Similar approaches can be employed for 3D
Riemannian extrapolation. The explicit solution seems more
appropriate, since splitting is difficult due to the mixed terms of
the Riemannian equations. In this paper, we concentrate our attention
to higher-order kernels implemented with implicit methods.
