The conceptual framework of wave-equation MVA is similar to that of
conventional (ray-based) MVA in that the source of information for
velocity updating is extracted from features of migrated images. This
is in contrast with wave-equation tomography (or inversion), where the
source of information is represented by the mismatch between recorded
and simulated data. The main difference between wave-equation MVA and
ray-based MVA is that the carrier of information from the migrated
images to the velocity model is represented by the entire extrapolated
wavefield and not by a rayfield constructed from selected points in
the image based on an approximate velocity model.

The key element for the wave-equation MVA technique is a definition of
an {\it image perturbation} corresponding to the difference between
the image obtained with a known background velocity model and an
improved image. Such image perturbations can be constructed using
straight differences between images
\cite[]{BiondiSava.segab.1999,Albertin4D}, or by examining moveout
parameters in migrated images
\cite[]{SavaBiondi.gp.wemva1,SavaBiondi.gp.wemva2,ShenSymes.segab.2005,Albertin.eageab.2006,MaharramovAlbertin.segab.2007}. Then, using wave-equation MVA
operators, the image perturbations can be translated into {\it
slowness perturbations} which update the model. The direct analogy
between wave-based MVA and ray-based MVA is the following: wave-based
methods use image perturbations and back-propagation using waves,
while ray-based methods use traveltime perturbations and
back-propagation using rays. Thus, wave-equation MVA benefits from all
the characteristics of wave-based imaging techniques, e.g. stability
in areas of large velocity variation, while remaining conceptually
similar to conventional traveltime-based MVA.

%% 
 % write about and compare the various imaging frameworks 
 %%

We can formulate wave-equation migration and velocity analysis for
different configurations in which we process the recorded data. There
are two main classes of wave-equation migration, {\it survey-sinking
migration} and {\it shot-record migration} \cite[]{Claerbout.iei},
which differ in the way in which recorded data are processed. Both
wave-equation imaging techniques use similar algorithms for downward
continuation and, in theory, produce identical images for identical
implementation of extrapolation operators and if all data are used for
imaging \cite[]{Berkhout.1982,GEO68-04-13401347}. The main difference
is that shot-record migration is used to process separate seismic
experiments (shots) sequentially, while survey-sinking migration is
used to process all seismic experiments (shots)
simultaneously. The shot-record operators are more
computationally expensive but less memory intensive than the
survey-sinking operators. A special case of survey-sinking migration
assumes the sources and receivers are coincident on the acquisition
surface, a technique usually described as the exploding reflector
model \cite[]{GPR24-02-03800399} applicable to zero-offset
data. All operators described here can be used in models
characterized by complex wave propagation (multipathing).


%% 
 % introduce the operator nomenclature
 %%

In all situations, wave-equation migration can be formulated as
consisting of two main steps. The first step is {\it wavefield
reconstruction} (abbreviated $\EXTR$ for the rest of this paper) at
all locations in space and using all frequencies from the recorded
data as boundary conditions. This step requires numeric solutions to a
form of wave equation, typically the one-way acoustic wave
equation. The second step is the {\it imaging condition} (abbreviated
$\IMCO$ for the rest of this paper), which is used to extract from the
reconstructed wavefield(s) the locations where reflectors occur. This
step requires numeric implementation of image processing techniques,
e.g. cross-correlation, which evaluate properties of the wavefield
indicating the presence of reflectors. Needless to say, the two steps
are not implemented sequentially in practice, since the size of the
wavefield is usually large and cannot be handled efficiently on
conventional computers. Instead, wavefield reconstruction and imaging
condition are implemented on-the-fly, avoiding expensive data storage
and retrieval. Wave-equation MVA requires implementation of an
additional procedure which links image and slowness
perturbations. This link is given by a {\it wavefield scattering}
operation (abbreviated $\SCAT$ for the rest of this paper) which is
derived by linearization from conventional wavefield extrapolation
operators.

%% 
 % describe the next sections
 %%

In the following sections, we describe the migration and velocity
analysis operators for the various imaging configurations. We begin
with zero-offset imaging under the exploding reflector model, because
this is the simplest wave-equation imaging framework and can aid our
understanding of both survey-sinking and shot-record migration and
velocity analysis frameworks. We then continue with a description of
the wave-equation migration velocity analysis operator for
multi-offset data using the survey-sinking and shot-record migration
configuration. For each configuration, we describe the implementation
of the {\it forward operator} (used to translate model perturbations
into image perturbations) and of the {\it adjoint operator} (used to
transform image perturbations into model perturbations). Both forward
and adjoint operators are necessary for the implementation of
efficient numeric conjugate gradient optimization
\cite[]{Claerbout.iei}.
