\section{Conclusions}

% about the implementation of WEMVA operators
The wave-equation MVA operator discussed in this paper, can be
implemented in various imaging frameworks, e.g. zero-offset (exploding
reflector), survey-sinking or shot-record. In all cases, the forward
and adjoint operators follow similar patterns involving combinations
of scattering, imaging and extrapolation. The forward and adjoint
operators share common elements and can be implemented in the mixed
space-wavenumber domain, similarly to the implementation of the
wavefield extrapolation operators.

% about why we want all sorts of WEMVA operators
% talk about wide-azimuth WEMVA...

% about the challenges of using WEMVA
The real challenges in using wave-based MVA are two-fold. First, the
image perturbations need to be generated by techniques that do not
compare image features that are too far from one-another, which is a
property partially addressed by techniques based on differential
semblance. Second, the cost of the wave-equation MVA operator is
large, therefore a feasible implementation requires clever numeric
implementation, e.g. by frequency decimation similarly to the approach
taken in waveform inversion.

% about the characteristics of the WEMVA operators
The examples shown in this paper illustrate the main characteristics
of the various wave-equation MVA operators, i.e. stability during
back-projection in background models with sharp velocity variation
(e.g. salt), natural ability to characterize multi-pathing and wide
area of sensitivity which is commensurate with the frequency band of
the recorded data.

