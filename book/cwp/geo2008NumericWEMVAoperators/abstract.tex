\begin{abstract}

%% 
 % what is WEMVA?
 %%
Wave-equation migration velocity analysis (MVA) is a technique similar
to wave-equation tomography because it is designed to update velocity
models using information derived from full seismic wavefields. On the
other hand, wave-equation MVA is similar to conventional,
traveltime-based MVA because it derives the information used for model
updates from properties of migrated images, e.g. focusing and moveout.
%% 
 % why use WEMVA?
 %%
The main motivation for using wave-equation MVA is derived from its
consistency with the corresponding wave-equation migration, which
makes this technique robust and capable of handling multipathing
characterizing media with large and sharp velocity contrasts.
%% 
 % how is the WEMVA operator constructed?
 %%
The wave-equation MVA operators are constructed using linearizations
of conventional wavefield extrapolation operators, assuming small
perturbations relative to the background velocity model. Similarly to
typical wavefield extrapolation operators, the wave-equation MVA
operators can be implemented in the mixed space-wavenumber domain
using approximations of different orders of accuracy.

%% 
 % why formulate WEMVA in different frameworks?
 %%
As for wave-equation migration, wave-equation MVA can be formulated in
different imaging frameworks, depending on the type of data used and
image optimization criteria. Examples of imaging frameworks correspond
to zero-offset migration (designed for imaging based on focusing
properties of the image), survey-sinking migration (designed for
imaging based on moveout analysis using narrow-azimuth data) and
shot-record migration (also designed for imaging based on moveout
analysis, but using wide-azimuth data).
%% 
 % what is the main message of this paper?
 %%

The wave-equation MVA operators formulated for the various imaging
frameworks are similar because they share common elements derived from
linearizations of the single square-root equation. Such operators
represent the core of iterative velocity estimation based on
diffraction focusing or semblance analysis, and their applicability in
practice requires efficient and accurate implementation. This tutorial
concentrates strictly on the numeric implementation of those operators
and not on their use for iterative migration velocity analysis.

\end{abstract}
