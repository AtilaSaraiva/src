\section{Examples}

% ------------------------------------------------------------
\inputdir{XFig}
% ------------------------------------------------------------
\plot{zokernel}{width=\textwidth} {Schematic representation of the
forward and adjoint operators for ray-based MVA and wave-based
MVA. The forward operator $F$ applied to a slowness anomaly $\ds$
generates a traveltime perturbation (a) or an image perturbation
(b). The ray-based adjoint MVA operator $A$ applied to the traveltime
perturbation generates a slowness perturbation uniformly distributed
along a ray normal to the reflector (c). The wave-based adjoint MVA
operator $A$ applied to the image perturbation generates a slowness
perturbation with a wider space distribution but with a relative focus
at the location of the original slowness anomaly (d).}

We illustrate the wave-equation migration velocity analysis operators
using impulse responses corresponding to different imaging
configurations. We concentrate on imaging in the zero-offset and
shot-record frameworks, since they also implicitly characterize the
essential elements of the survey-sinking framework. In all cases, we
use wavefield reconstruction based on one-way wavefield extrapolation
with the multi-reference split-step Fourier method
\cite[]{GEO55-04-04100421,GEO61-05-14121416}.

% ------------------------------------------------------------
\inputdir{flatWEMVA}
% ------------------------------------------------------------

\multiplot{2}{vel,img}{width=\textwidth}{ Simple synthetic model with
(a) linear $v\lp z \rp$ velocity and (b) a horizontal reflector. }

A fundamental question concerning the wavefield scattering operator
($\SCAT$) is what is its sensitivity for a given perturbation of the
image or of the slowness model. This sensitivity is usually
characterized using the so-called ``sensitivity kernels'' which are
often discussed in the literature in the context of tomography
problems. For wave-equation MVA, this topic was discussed in the
context of zero-offset imaging by
\cite{SavaBiondi.gp.wemva1,SavaBiondi.gp.wemva2}.  The important topic
of sensitivity and model resolution falls outside the scope of this
paper, so we do not discuss it here in any detail. We merely concern
ourselves with describing the behavior of the wave-equation MVA
operators described earlier.

% ------------------------------------------------------------
\multiplot{2}{ds4,di4}{width=\textwidth}{ (a) Slowness perturbations
used to demonstrate the WEMVA operators in
\rFgs{flatWEMVA-ZFds4}-\rfn{flatWEMVA-SAFds4}, and (b) image
perturbation used to demonstrate the WEMVA operators in
\rFgs{flatWEMVA-ZAdi4}-\rfn{flatWEMVA-SFAdi4}.}

\multiplot{2}{Zdat,Sdat}{width=\textwidth}{ (a) Simulated zero-offset
data and (b) simulated shot-record data for the model depicted in
\rfgs{flatWEMVA-vel}-\rfn{flatWEMVA-img} with a source located at
coordinates $x=6$~km and $z=0$~km.}
% ------------------------------------------------------------

We can analyze the sensitivity of the wavefield scattering operator in
two ways. The first option is to assume a localized slowness
perturbation, compute image perturbations using the forward scattering
operator and then return to the slowness perturbation using the
adjoint scattering operator. The second option is to assume a
localized image perturbation, compute the slowness perturbation using
the adjoint scattering operator and then return to the image
perturbation using the forward scattering operator. 

% ------------------------------------------------------------
\multiplot{2}{ZFds4,ZAFds4}{width=\textwidth}{(a) Zero-offset image
perturbation obtained by the application of the forward scattering
operator to the slowness perturbation from \rFg{flatWEMVA-ds4} and (b)
zero-offset slowness perturbation obtained by the application of the
adjoint scattering operator to the image perturbation from panel (a).
}

\multiplot{2}{SFds4,SAFds4}{width=\textwidth}{(a) Shot-record image
perturbation obtained by the application of the forward scattering
operator to the slowness perturbation from \rFg{flatWEMVA-ds4} and (b)
shot-record slowness perturbation obtained by the application of the
adjoint scattering operator to the image perturbation from panel (a).}
% ------------------------------------------------------------

As discussed in the preceding sections, the main difference between
ray-based and wave-based MVA techniques is that the connection between
measurements on the image and updates to the model is done with rays
and waves, respectively. The impact of this fundamental difference is
best seen if we analyze impulse responses of the wave-equation MVA and
compare them with those of conventional traveltime
tomography. \rFg{zokernel} shows a one-to-one comparison between the
forward and adjoint operators for ray-based MVA (traveltime
tomography) on the left and wave-based MVA on the right in the
context of zero-offset imaging.
%
Assuming a small slowness perturbation $\ds$, we can construct using
the forward MVA operators a traveltime perturbation and an image
perturbation corresponding to ray-based MVA (a) and wave-based MVA
(b), respectively. For this zero-offset configuration, the ray-based
MVA produces a traveltime anomaly strictly located on the reflector
under the slowness anomaly, while the wave-based MVA produces an image
anomaly distributed in space in the vicinity of the reflector.
%
Then, we can construct respective slowness updates if we apply the
ray-based and wave-based adjoint MVA operators to the traveltime
perturbation and image perturbation, respectively. For the ray-based
MVA, the slowness update spreads uniformly along a ray orthogonal to
the reflector (c), while for wave-based MVA, the slowness update is
distributed in space from the image perturbation to the surface, but
with a concentration at the location of the true anomaly (d). Similar
behavior characterizes wave-equation MVA under shot-record or
survey-sinking frameworks.

%% 
 % flat horizontal reflector
 %%

% ------------------------------------------------------------
\multiplot{2}{ZAdi4,ZFAdi4}{width=\textwidth} { (a) Zero-offset
slowness perturbation obtained by the application of the adjoint
scattering operator to the image perturbation from \rFg{flatWEMVA-di4}
and (b) zero-offset image perturbation obtained by the application of
the forward scattering operator to the slowness perturbation from
panel (a).  }

\multiplot{2}{SAdi4,SFAdi4}{width=\textwidth} { (a) Shot-record
slowness perturbation obtained by the application of the adjoint
scattering operator to the image perturbation from \rFg{flatWEMVA-ds4}
and (b) shot-record image perturbation obtained by the application of
the forward scattering operator to the slowness perturbation from
panel (a).  }
% ------------------------------------------------------------

The first set of examples corresponds to a simple model consisting of
a linear $v\lp z \rp$ velocity model and a horizontal reflector,
\rFgs{flatWEMVA-vel}-\rfn{flatWEMVA-img}. The velocity is linearly
increasing from $1.5$~km/s to $2.75$~km/s. We simulate zero-offset
data, \rfg{flatWEMVA-Zdat}, and one shot corresponding to horizontal
position $x=6$~km, \rfg{flatWEMVA-Sdat}.

Assuming a localized slowness perturbation, \rfg{flatWEMVA-ds4}, we
can compute image perturbations using the forward scattering
operators, as defined in the preceding sections. \rFg{flatWEMVA-ZFds4}
shows the image perturbation for the zero-offset case and
\rFg{flatWEMVA-SFds4} shows the similar image perturbation for the
shot-record case. As illustrated in \rFg{zokernel}, the image
perturbations are distributed in the vicinity of the reflector. Two
interfering events are seen for the shot-record case, corresponding to
the source and receiver wavefields, respectively.

Similarly, we can compute slowness perturbations using the adjoint
scattering operators. \rFg{flatWEMVA-ZAFds4} shows the slowness
perturbation for the zero-offset case computed from the image
perturbation in \rFg{flatWEMVA-ZFds4} and \rFg{flatWEMVA-SAFds4} shows
the similar slowness perturbation for the shot-record case computed
from the image perturbation in \rFg{flatWEMVA-SFds4}.  As illustrated
in \rFg{zokernel}, the slowness perturbations are distributed in an
area connecting the reflector to the surface, but with a relative
focus at the location of the original anomaly. For the shot-record
case, the back-projection splits toward the source and receivers,
corresponding to the upward continuation of the source and receiver
wavefields.

We can also analyze the wave-equation MVA operator sensitivity in
another way. Assuming a localized image perturbation,
\rFg{flatWEMVA-di4}, we can compute slowness perturbations using the
adjoint scattering operators, as defined in the preceding
sections. \rFg{flatWEMVA-ZAdi4} shows the slowness perturbation for
the zero-offset case and \rFg{flatWEMVA-SAdi4} shows the similar
slowness perturbation for the shot-record case. Here, too, we see
slowness perturbations distributed in an area connecting the reflector
to the surface, but in this case, there is no relative focus of the
anomaly because the image perturbation is strictly localized on the
reflector. For the shot-record case, the back-projection splits toward
the source and receivers, corresponding to the upward continuation of
the source and receiver wavefields. This case corresponds to the case
of practical MVA where measurements of defocusing features are made on
the image itself.

As we have done in the preceding experiment, we can also compute image
perturbations using the forward scattering operators based on the
back-projections created using the adjoint scattering operators.
\rFg{flatWEMVA-ZFAdi4} shows the image perturbation for the
zero-offset case computed from the slowness perturbation in
\rFg{flatWEMVA-ZAdi4} and \rFg{flatWEMVA-SFAdi4} shows the similar
image perturbation for the shot-record case computed from the slowness
perturbation in \rFg{flatWEMVA-SAdi4}. We can observe that the
resulting image perturbations spread beyond the original location,
indicating wider sensitivity of the wave-based MVA kernels to image
perturbations than that of the corresponding ray-based MVA kernels.

%% 
 % Sigsbee 2A
 %%

Similar sensitivity can be observed for the more complex Sigsbee 2A
model \cite[]{SEG-2002-21222125},
\rFgs{saltWEMVA-vel}-\rfn{saltWEMVA-img}. Similarly to the preceding
example, we simulate zero-offset data, \rfg{saltWEMVA-Zdat}, and one
shot corresponding to horizontal position $x=14.6$~km,
\rfg{saltWEMVA-Sdat}.

\rFgs{saltWEMVA-ZFds2} and \rfn{saltWEMVA-SFds2} correspond to the
image perturbations for the slowness anomaly shown in
\rFg{saltWEMVA-ds2}. We can observe image perturbations that spread in
the vicinity of the reflector, similarly to the simpler example
described earlier. The multi-pathing from the source to the reflector
generates the multiple events characterizing the image
perturbations.
%
\rFgs{saltWEMVA-ZAFds2} and \rfn{saltWEMVA-SAFds2} correspond to the
slowness perturbations constructed by applying the zero-offset and
shot-record adjoint scattering operators to the image perturbations
from \rFgs{saltWEMVA-ZFds2} and \rfn{saltWEMVA-SFds2}. We see similar
back-projection patterns to the ones observed in the preceding
example, except that the propagation pattens are more complicated due
to the presence of the salt body in the background model.

\rFgs{saltWEMVA-ZAdi2} and \rfn{saltWEMVA-SAdi2} correspond to the
slowness perturbations for the image anomaly shown in
\rFg{saltWEMVA-di2}. We can observe slowness perturbations that spread
in the vicinity of the reflector, similarly to the simpler example
described earlier.  
%
Finally, \rFgs{saltWEMVA-ZFAdi2} and \rfn{saltWEMVA-SFAdi2} correspond
to the image perturbations for the slowness perturbations constructed
by the adjoint MVA operators shown in \rFgs{saltWEMVA-ZAdi2} and
\rfn{saltWEMVA-SAdi2} for the zero-offset and shot-record cases,
respectively.

% ------------------------------------------------------------
\inputdir{saltWEMVA}
% ------------------------------------------------------------

\multiplot{2}{vel,img}{width=\textwidth}{ (a) Sigsbee 2A synthetic
model and (b) a sub-salt horizontal reflector.  }

% ------------------------------------------------------------
\multiplot{2}{ds2,di2}{width=\textwidth}{ (a) Slowness perturbations
used to demonstrate the WEMVA operators in
\rFgs{saltWEMVA-ZFds2}-\rfn{saltWEMVA-SAFds2}, and (b) image
perturbation used to demonstrate the WEMVA operators in
\rFgs{saltWEMVA-ZAdi2}-\rfn{saltWEMVA-SFAdi2}.}

\multiplot{2}{Zdat,Sdat}{width=\textwidth}{(a) Simulated zero-offset
data and (b) simulated shot-record data for the model depicted in
\rfgs{saltWEMVA-vel}-\rfn{saltWEMVA-img} with a source located at
coordinates $x=14.6$~km and $z=1.52$~km.}
% ------------------------------------------------------------

% ------------------------------------------------------------
\multiplot{2}{ZFds2,ZAFds2}{width=\textwidth}{(a) Zero-offset
image perturbation obtained by the application of the forward
scattering operator to the slowness perturbation from
\rFg{saltWEMVA-ds2} and (b) zero-offset slowness perturbation obtained
by the application of the adjoint scattering operator to the image
perturbation from panel (a).  }

\multiplot{2}{SFds2,SAFds2}{width=\textwidth}{(a) Shot-record
image perturbation obtained by the application of the forward
scattering operator to the slowness perturbation from
\rFg{saltWEMVA-ds2} and (b) shot-record slowness perturbation obtained
by the application of the adjoint scattering operator to the image
perturbation from panel (a).}
% ------------------------------------------------------------

% ------------------------------------------------------------
\multiplot{2}{ZAdi2,ZFAdi2}{width=\textwidth}{(a) Zero-offset slowness
perturbation obtained by the application of the adjoint scattering
operator to the image perturbation from \rFg{saltWEMVA-di2} and (b)
zero-offset image perturbation obtained by the application of the
forward scattering operator to the slowness perturbation from panel
(a).}

\multiplot{2}{SAdi2,SFAdi2}{width=\textwidth}{(a) Shot-record slowness
perturbation obtained by the application of the adjoint scattering
operator to the image perturbation from \rFg{saltWEMVA-ds2} and (b)
shot-record image perturbation obtained by the application of the
forward scattering operator to the slowness perturbation from panel
(a). }
% ------------------------------------------------------------

