% ------------------------------------------------------------
\subsection{Survey-sinking migration and velocity analysis}

Wavefield reconstruction for multi-offset migration based on the
one-way wave-equation under the survey-sinking framework
\cite[]{Claerbout.iei} is implemented similarly to the zero-offset
case by recursive phase-shift of prestack wavefields
%
\beq \label{eqn:PHS-SS}
\UU_{z+\dz}\ofmh = \PS{-}\UU_z\ofmh \;,
\eeq
%
followed by extraction of the image at time $t=0$. Here, $\mm$ and
$\ho$ stand for midpoint and half-offset coordinates, respectively,
defined according to the relations
%
\bea
\mm &=& \frac{\rr+\ss}{2}
\\
\ho &=& \frac{\rr-\ss}{2} \;,
\eea
%
where $\ss$ and $\rr$ are coordinates of sources and receivers on the
acquisition surface. In \req{PHS-SS}, $\UU_z\ofmh$ represents the
acoustic wavefield for a given frequency $\ww$ at all midpoint
positions $\mm$ and half-offsets $\ho$ at depth $z$, and
$\UU_{z+\dz}\ofmh$ represents the same wavefield extrapolated to depth
$z+\dz$. The phase shift operation uses the depth wavenumber $\kz$
which is defined by the double square-root (DSR) equation
%
\bea
\kz &=& \SSR{\ws\ofmm}{\frac{\km-\kh}{2}} \nonumber \\  \label{eqn:DSR}
    &+& \SSR{\ws\ofmp}{\frac{\km+\kh}{2}} \;.
\eea
%
The image is obtained from this extrapolated wavefield by selection of
time $t=0$, which is usually implemented as summation over
frequencies:
%
\beq \label{eqn:IMC-SS}
\RR_z\ofmh = \sum_\ww \UU_z\ofmhw \;.
\eeq
%

Similarly to the derivation done in the zero-offset case, we can
assume the separation of the extrapolation slowness $s\ofm$ into a
background component $s_0\ofm$ and an unknown perturbation component
$\ds\ofm$. Then we can construct a wavefield perturbation $\dUU\ofmh$
at depth $z$ and frequency $\ww$ related linearly to the slowness
perturbation $\ds\ofm$. Linearizing the depth wavenumber given by the
DSR equation \ren{DSR} relative to the background slowness $s_0\ofm$,
we obtain
%
\beq \label{eqn:DSR-taylor}
\kz \approx \kz_0 + \dkzdsm \ds\ofmm + \dkzdsp \ds\ofmp \;,
\eeq
where the depth wavenumber in the background medium is
%
\bea
\kz_0 &=& \SSR{\ws_0\ofmm}{\frac{\km-\kh}{2}} \nonumber \\
      &+& \SSR{\ws_0\ofmp}{\frac{\km+\kh}{2}} \;.
\eea
%
Here, $s_0\ofm$ represents the spatially-variable {\it background}
slowness at depth level $z$. Using the wavenumber linearization given
by \req{DSR-taylor}, we can reconstruct the acoustic wavefields in the
background model using a phase-shift operation
%
\beq
\UU_{z+\dz}\ofmh = \PSo{-}\UU_z\ofmh \;.
\eeq
%
We can represent wavefield extrapolation using a generic solution to
the one-way wave-equation using the notation
$\UU_{z+\dz}\ofmh=\PSop{-}{SSM}{{s_0}_z\ofm,\UU_z\ofmh}$.  This
notation indicates that the wavefield $\UU_{z+\dz}\ofmh$ is
reconstructed from the wavefield $\UU_z\ofmh$ using the background
slowness $s_0\ofm$. This operation is repeated independently for all
frequencies $\ww$.  A typical implementation of survey-sinking
wave-equation migration uses the following algorithm:
% ------------------------------------------------------------
%\newpage
\ssmig
% ------------------------------------------------------------
This algorithm is similar to the one described in the preceding
section for zero-offset migration, except that the wavefield and image
are parametrized by midpoint and half-offset coordinates and that the
depth wavenumber used in the extrapolation operator is given by the
DSR equation using the background slowness $s_0\ofm$. Wavefield
extrapolation is usually implemented in a mixed domain
(space-wavenumber), as briefly summarized in Appendix A.

Similarly to the derivation of the wavefield perturbation in the
zero-offset migration case, we can write the linearized wavefield
perturbation for survey-sinking migration as
%
\bea
\dUU\ofmh 
\approx &-&i \dkzdsm \ds\ofmm \dz \; \UU\ofmh
\nonumber \\ 
        &-&i \dkzdsp \ds\ofmp \dz \; \UU\ofmh
\nonumber \\ 
\approx &-&i\dz \SQREXP{\ww\UU\ofmh \ds\ofmm}{2\ws_0\ofmm}{\km-\kh}
\nonumber \\ \label{eqn:SSFSOP}
        &-&i\dz \SQREXP{\ww\UU\ofmh \ds\ofmp}{2\ws_0\ofmp}{\km+\kh} \;.
\eea
%
\rEq{SSFSOP} defines the forward scattering operator
$\FSop{-}{SSM}{\UU\ofmh,s_0\ofm,\ds\ofmh}$, producing the scattered
wavefield $\dUU\ofmh$ from the slowness perturbation $\ds\ofm$, based
on the background slowness $s_0\ofm$ and background wavefield
$\UU\ofmh$. The image perturbation at depth $z$ is obtained from the
scattered wavefield using the time $t=0$ imaging condition, similar to
the procedure used for imaging in the background medium:
\beq
\dR\ofmh = \sum_\ww \dUU\ofmhw \;.
\eeq

Given an image perturbation $\dR\ofmh$, we can construct the scattered
wavefield by the adjoint of the imaging condition
\beq
\dUU\ofmhw = \dR\ofmh \;,
\eeq
for every frequency $\ww$. Then, similarly to the procedure used in
the zero-offset case, the slowness perturbation at depth $z$ caused by
a wavefield perturbation at depth $z$ under the influence of the
background wavefield at the same depth $z$ can be written as
%
\bea
\ds\ofmm 
\approx &+&i \dkzdsm \dz \; \CONJ{\UU\ofmh} \dUU\ofmh
\nonumber \\  \label{eqn:SSASOPs}
\approx &+&i\dz \SQREXP{\ww\CONJ{\UU\ofmh} \dUU\ofmh}{2\ws_0\ofmm}{\km-\kh} \;,
\eea
and
\bea
\ds\ofmp 
\approx &+&i \dkzdsp \dz \; \CONJ{\UU\ofmh} \dUU\ofmh
\nonumber \\  \label{eqn:SSASOPr}
\approx &+&i\dz \SQREXP{\ww\CONJ{\UU\ofmh} \dUU\ofmh}{2\ws_0\ofmp}{\km+\kh} \;.
\eea
%
\rEqs{SSASOPs}-\ren{SSASOPr} define the adjoint scattering operator
$\ASop{+}{SSM}{\UU\ofmh,s_0\ofm,\dUU\ofmh}$, producing the slowness
perturbation $\ds\ofm$ from the scattered wavefield $\dUU\ofmh$, based
on the background slowness $s_0\ofm$ and background wavefield
$\UU\ofmh$. A typical implementation of survey-sinking forward and
adjoint scattering follows the algorithms:
% ------------------------------------------------------------
%\newpage
\ssfor 
%\newpage
\ssadj
% ------------------------------------------------------------
These algorithms are similar to the ones described in the preceding
section for zero-offset migration, except that the wavefield and image
are parametrized by midpoint and half-offset coordinates.
Furthermore, the two square-roots corresponding to the DSR equation
update the slowness model separately, thus characterizing the source
and receiver propagation paths to the image positions.  
%
Both forward and adjoint scattering algorithms require the background
wavefield, $\UU\ofmh$, to be precomputed at all depth levels.
%
Scattering and wavefield extrapolation are implemented in the mixed
space-wavenumber domain, as briefly explained in Appendix A.
