% ------------------------------------------------------------
\subsection{Zero-offset migration and velocity analysis}

Wavefield reconstruction for zero-offset migration based on the
one-way wave-equation is performed by recursive phase-shift in depth
starting with data recorded on the surface as boundary conditions. In
this configuration, the imaging condition extracts the image as time
$t=0$ from the reconstructed wavefield at every location in
space. Thus, the surface data need to be extrapolated backward in time
which is achieved by selecting the appropriate sign of the phase-shift
operation (which depends on the sign convention for temporal Fourier
transforms):
%
\beq \label{eqn:PHS-ZO}
\UU_{z+\dz}\ofm = \PS{-}\UU_z\ofm \;.
\eeq
%
In \req{PHS-ZO}, $\UU_z\ofm$ represents the acoustic wavefield at
depth $z$ for a given frequency $\ww$ at all positions in space $\mm$,
and $\UU_{z+\dz}\ofm$ represents the same wavefield extrapolated to
depth $z+\dz$. The phase shift operation uses the depth wavenumber
$\kz$ which is defined by the single square-root (SSR) equation
%
\beq \label{eqn:SSR-ZO}
\kz = \SSR{2\ws\ofm}{\km}\;,
\eeq
%
where $s\ofm$ represents the spatially-variable slowness at depth
level $z$. \rEqs{PHS-ZO}-\ren{SSR-ZO} describe wavefield extrapolation
using a pseudo-differential operator, which justifies our use of
laterally-varying slowness $s\ofm$. As indicated earlier, the image is
obtained from this extrapolated wavefield by selection of time $t=0$,
which is typically implemented as summation of the extrapolated
wavefield over frequencies:
%
\beq \label{eqn:IMC-ZO}
\RR_z\ofm = \sum_\ww \UU_z\ofmw \;.
\eeq
%
Phase-shift extrapolation using wavenumbers computed using
\reqs{PHS-ZO} and \ren{SSR-ZO} is not feasible in media with lateral
variation. Instead, implementation of such operators is done using
approximations implemented in a mixed space-wavenumber domain
\cite[]{GEO55-04-04100421,GEO59-12-18821893,GEO64-05-15241534}. A
brief summary the mixed-domain implementation of the split-step
Fourier (SSF) operator is presented in Appendix A.

For velocity analysis, we assume that we can separate the {\it total
slowness} $s\ofm$ into a known background component $s_0\ofm$ and an
unknown component $\ds\ofm$. With this convention, we can linearize
the depth wavenumber $\kz$ relative to the background slowness $s_0$
using a truncated Taylor series expansion
%
\beq \label{eqn:SSR-taylor}
\kz \approx \kz_0 + \dkzds \ds\ofm \;,
\eeq
%
where the depth wavenumber in the background medium characterized by
slowness $s_0\ofm$ is
%
\beq
\kz_0 = \SSR{2\ws_0\ofm}{\km} \;.
\eeq
%
Here, $s_0\ofm$ represents the spatially-variable {\it background
slowness} at depth level $z$. Using the wavenumber linearization from
\req{SSR-taylor}, we can reconstruct the acoustic wavefields in the
background model using a phase-shift operation
%
\beq
\UU_{z+\dz}\ofm = \PSo{-}\UU_z\ofm \;.
\eeq
%
We can represent wavefield extrapolation using a generic solution to
the one-way wave-equation using the notation
$\UU_{z+\dz}\ofm=\PSop{-}{ZOM}{2{s_0}_z\ofm,\UU_z\ofm}$.  This
notation indicates that the wavefield $\UU_{z+\dz}\ofm$ is
reconstructed from the wavefield $\UU_z\ofm$ using the background
slowness $s_0\ofm$. This operation is repeated independently for all
frequencies $\ww$. A typical implementation of zero-offset
wave-equation migration uses the following algorithm:
% ------------------------------------------------------------
%\newpage
\zomig
% ------------------------------------------------------------
A seismic image is produced using migration by wavefield extrapolation
as follows: for each frequency, read data at all spatial locations
$\mm$; then, for each depth, sum the wavefield into the image at that
depth level (i.e. apply the imaging condition) and then reconstruct
the wavefield to the next depth level (i.e. perform wavefield
extrapolation). The ``-'' sign in this algorithm indicates that
extrapolation is anti-causal (backward in time), and the factor ``2''
indicates that we are imaging data recorded in two-way traveltime with
an algorithm designed under the exploding reflector model. Wavefield
extrapolation is usually implemented in a mixed domain
(space-wavenumber), as briefly summarized in Appendix A.

The wavefield perturbation $\dUU\ofm$ caused at depth $z+\dz$ by a
slowness perturbation $\ds\ofm$ at depth $z$ is obtained by
subtraction of the wavefields extrapolated from $z$ to $z+\dz$ in the
true and background models:
%
\bea 
\dUU_{z+\dz}\ofm 
&=& \PS{-}\UU_z\ofm - \PSo{-}\UU_z\ofm
\nonumber \\ \label{eqn:ZODW-nonlinear}
&=& \PSo{-} \lb e^{- i \dkzds \ds\ofm \dz} - 1\rb \UU_z\ofm \;.
\eea
Here, $\dUU\ofm$ and $\UU\ofm$ correspond to a given depth level $z$
and frequency $\ww$. A similar relation can be applied at all depths
and all frequencies.

\rEq{ZODW-nonlinear} establishes a non-linear relation between 
the wavefield perturbation $\dUU\ofm$ and the slowness perturbation
$\ds\ofm$. Given the complexity and cost of numeric optimization based
on non-linear relations between model and wavefield parameters, it is
desirable to simplify this relation to a linear relation between model
and data parameters. Assuming small slowness perturbations, i.e. small
phase perturbations, the exponential function $e^{\pm i \dkzds \ds\ofm
\dz}$ can be linearized using the approximation $e^{i\phi}\approx
1+i\phi$ which is valid for small values of the phase
$\phi$. Therefore the wavefield perturbation $\dUU\ofm$ at depth $z$
can be written as
%
\bea 
\dUU\ofm &\approx& - i \dkzds \dz \; \UU\ofm \ds\ofm 
\nonumber \\ \label{eqn:ZOFSOP}
         &\approx& - i\dz \SQREXP{2\ww \UU\ofm \ds\ofm}{2\ws_0\ofm}{\km} \;.
\eea
\rEq{ZOFSOP} defines the zero-offset forward scattering operator
$\FSop{-}{ZOM}{\UU\ofm,2s_0\ofm,\ds\ofm}$, producing the scattered
wavefield $\dUU\ofm$ from the slowness perturbation $\ds\ofm$, based
on the background slowness $s_0\ofm$ and background wavefield
$\UU\ofm$ at a given frequency $\ww$. The image perturbation at depth
$z$ is obtained from the scattered wavefield using the time $t=0$
imaging condition, similar to the procedure used for imaging in the
background medium:
\beq
\dR\ofm = \sum_\ww \dUU\ofmw \;.
\eeq

Given an image perturbation $\dR\ofm$, we can construct the scattered
wavefield by the adjoint of the imaging condition
\beq
\dUU\ofmw = \dR\ofm \;,
\eeq
for every frequency $\ww$. Then, the slowness perturbation at depth
$z$ and frequency $\ww$ caused by a wavefield perturbation at depth
$z$ under the influence of the background wavefield at the same depth
$z$ can be written as
%
\bea
\ds\ofm &\approx& + i \dkzds \dz \; \CONJ{\UU\ofm} \dUU\ofm 
\nonumber \\ \label{eqn:ZOASOP}
        &\approx& + i\dz \SQREXP{2\ww \CONJ{\UU\ofm} \dUU\ofm}{2\ws\ofm}{\km} \;.
\eea
%
\rEq{ZOASOP} defines the adjoint scattering operator
$\ASop{+}{ZOM}{\UU\ofm,2s_0\ofm,\dUU\ofm}$, producing the slowness
perturbation $\ds\ofm$ from the scattered wavefield $\dUU\ofm$, based
on the background slowness $s_0\ofm$ and background wavefield
$\UU\ofm$. A typical implementation of zero-offset forward and adjoint
scattering uses the following algorithms:
% ------------------------------------------------------------
%\newpage
\zofor
%\newpage
\zoadj
% ------------------------------------------------------------
The forward zero-offset wave-equation MVA operator follows a similar
pattern to the implementation of the downward continuation operator:
for each frequency and for each depth, read the slowness perturbation
$\ds$ at all spatial locations $\mm$, then apply the scattering
operator ($\SCAT$) given \req{ZOASOP} to the slowness perturbation and
accumulate the scattered wavefield for downward continuation; then,
apply the imaging condition ($\IMCO$) producing the image perturbation
$\dR$ at depth $z$ and then reconstruct the scattered wavefield
backward in time using the wavefield extrapolation operator ($\EXTR$)
to the next depth level.
%
The adjoint zero-offset wave-equation MVA operator also follows a
similar pattern to the implementation of the downward continuation
operator: for each frequency and for each depth, reconstruct the
scattered wavefield forward in time using the wavefield extrapolation
operator ($\EXTR$) to the next depth level, then apply the adjoint of
the imaging condition ($\IMCO$) by adding the image to the scattered
wavefield and then apply the adjoint wavefield scattering operator
($\SCAT$) to obtain the slowness perturbation $\ds$.
% 
Both forward and adjoint scattering algorithms require the background
wavefield, $\UU$, to be precomputed at all depth levels, although more
efficient implementations using optimal checkpointing are possible
\cite[]{Symes.checkpointing}.
%
Scattering and wavefield extrapolation are implemented in the mixed
space-wavenumber domain, as briefly explained in Appendix A.
