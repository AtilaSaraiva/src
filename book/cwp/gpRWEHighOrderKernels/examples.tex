\def\COSlabel{Velocity map and Riemannian coordinate system}
\def\RWElabel{
Migration impulse responses in Riemannian coordinates.
(a) Extrapolation with the $15^\circ$ finite-differences equation.
(c) Extrapolation with the $60^\circ$ finite-differences equation.
(b) Extrapolation with the pseudo-screen (PSC) equation.
(d) Extrapolation with the Fourier finite-differences (FFD) equation.
}
\def\CARlabel{
Migration impulse responses in Riemannian coordinates after mapping to Cartesian coordinates.
(a) Extrapolation with the $15^\circ$ finite-differences equation.
(c) Extrapolation with the $60^\circ$ finite-differences equation.
(b) Extrapolation with the pseudo-screen (PSC) equation.
(d) Extrapolation with the Fourier finite-differences (FFD) equation.
}

\def\ABlabel{
Coordinate system coefficients defined in \reqs{rwea} and \ren{rweb}.
(a) Parameter $a=s \aa$  in ray coordinates.
(b) Parameter $b=\aa/\jj$ in ray coordinates.
}

\def\COMPlabel{
Comparison of extrapolation in Cartesian and Riemannian coordinates.
(a) Split-step Fourier extrapolation in Cartesian coordinates.
(b) Split-step Fourier extrapolation in Riemannian coordinates.
}

% ------------------------------------------------------------

\section{Examples}
We illustrate the higher-order RWE extrapolators with impulse
responses for two synthetic models.
\par
The first example is based on the Marmousi model
\cite[]{TLE13-09-09270936}. We construct the coordinate system by ray
tracing from a point source at the surface in a smooth version of the
real velocity model. \rFg{M-cos} shows the velocity model with the
coordinate system overlaid, and \rFgs{M-abmRCa}-\rfn{M-abmRCb} show
the coordinate system coefficients $a$ and $b$ defined in \reqs{rwea}
and \ren{rweb}.
%%
\inputdir{marm}
\plot*{M-cos}{width=0.80\textwidth}
{\COSlabel~for the Marmousi example.}
%%
\multiplot{2}{M-abmRCa,M-abmRCb}
{angle=0,width=0.45\textwidth}
{\ABlabel}
%%
\par
The goal of this test model is to illustrate the higher-order
extrapolation kernels in a fairly complex model using a simple
coordinate system. In this way, the coordinate system and the real
direction of wave propagation depart from one-another, thus accurate
extrapolation requires higher order kernels. The coordinate system is
constructed from a point at the location of the wave source. This
setting is similar to the case of extrapolation from a point source in
Cartesian coordinates, where high-angle
\footnote{
If the extrapolation axis is time, the meaning of higher angle
accuracy is not well defined. We can use this terminology to associate
the mathematical meaning of the approximation for the square-root by
analogy with the Cartesian equivalents.  } propagation requires
high-order kernels.
\par
\rFgs{M-migRC-F15}-\rfn{M-migRC-FFD} show impulse responses for a 
point source computed with various extrapolators in ray coordinates
($\qt$ and $\qg$). Panels (a) and (c) show extrapolation with the
$15^\circ$ and $60^\circ$, respectively. Panels (b) and (d) show
extrapolation with the pseudo-screen (PSC) equation, and the Fourier
finite-differences (FFD) equation, respectively. All plots are
displayed in ray coordinates. We can observe that the angular accuracy
of the extrapolator improves for the more accurate extrapolators. The
finite-differences solutions (panels a and c) show the typical
behavior of such solutions for the $15^\circ$ and $60^\circ$ equations
(e.g.  the cardioid for $60^\circ$), but in the more general setting
of Riemannian extrapolation. The mixed-domain extrapolators (panels b
and d) are more accurate the finite-differences extrapolators. The
main differences occur at the highest propagation angles. As for the
case of Cartesian extrapolation, the most accurate kernel of those
compared is the equivalent of Fourier finite-differences.
\par
\rFgs{M-migCC-F15}-\rfn{M-migCC-FFD} show the corresponding plots 
in \rFgs{M-migRC-F15}-\rfn{M-migRC-FFD} mapped in the physical
coordinates. The overlay is an outline of the extrapolation coordinate
system. After re-mapping to the physical space, the comparison of
high-angle accuracy for the various extrapolators is more apparent,
since it now has physical meaning.
\par
\rFgs{M-imgCC}-\rfn{M-migCC-SSF} show a side-by-side comparison of 
equivalent extrapolators in Riemannian and Cartesian coordinates. The
impulse response in \rFg{M-imgCC} shows the limits of Cartesian
extrapolation in propagating waves correctly up to $90^\circ$. The
Riemannian extrapolator in \rFg{M-migCC-SSF} handles much better waves
propagating at high angles, including energy that is propagating
upward relative to the physical coordinates.
%%
\multiplot{4}{M-migRC-F15,M-migRC-PSC,M-migRC-F60,M-migRC-FFD}
{angle=0,width=0.45\textwidth}{\RWElabel}
%%
\multiplot{4}{M-migCC-F15,M-migCC-SSF,M-migCC-F60,M-migCC-FFD}
{angle=0,width=0.45\textwidth}{\CARlabel}
%%
\multiplot{2}{M-imgCC,M-migCC-SSF}
{angle=0,width=0.80\textwidth}{\COMPlabel}
% ------------------------------------------------------------

% ------------------------------------------------------------
The second example is based on a model with a large lateral gradient
which makes an incident plane wave overturn. A small Gaussian anomaly,
not used in the construction of the coordinate system, forces the
propagating wave to triplicate and move at high angles relative to the
extrapolation direction. \rFg{D-cos} shows the velocity model with the
coordinate system overlaid. \rFgs{D-abmRCa}-\rfn{D-abmRCb} show the
coordinate system coefficients, $a$ and $b$ defined in \reqs{rwea} and
\ren{rweb}.
%%
\inputdir{diving}
\plot{D-cos}{width=0.80\textwidth}
{\COSlabel~for the large-gradient model experiment.}
%%
\multiplot{2}{D-abmRCa,D-abmRCb}
{angle=0,width=0.45\textwidth}{\ABlabel}
%%
\par
The goal of this model is to illustrate Riemannian wavefield
extrapolation in a situation which cannot be handled correctly by
Cartesian extrapolation, no matter how accurate an extrapolator we
use. In this example, an incident plane wave is overturning, thus
becoming evanescent for the solution constructed in Cartesian
coordinates. Furthermore, the Gaussian anomaly shown in \rFg{D-abmRCa}
causes wavefield triplication, thus requiring high-order kernels for
the Riemannian extrapolator.
\par
\rFgs{D-migRC-F15}-\rfn{D-migRC-FFD} show impulse responses for an 
incident plane wave computed with various extrapolators in ray
coordinates ($\qt$ and $\qg$). Panels (a) and (c) show extrapolation
with the $15^\circ$ and $60^\circ$ finite-differences equations,
respectively. Panel (b) and (d) show extrapolation with the
pseudo-screen (PSC) equation and the Fourier finite-differences (FFD)
equation, respectively. All plots are displayed in ray coordinates. As
for the preceding example, we observe higher angular accuracy as we
increase the order of the extrapolator. The equivalent FFD
extrapolator shows the highest accuracy of all tested extrapolators.
\par
As in the preceding example, \rFgs{D-migCC-F15}-\rfn{D-migCC-FFD} show
the corresponding plots in \rFgs{D-migRC-F15}-\rfn{D-migRC-FFD} mapped
in the physical coordinates. The overlay is an outline of the
extrapolation coordinate system.
\par
Finally, \rfgs{D-imgCC-bp} and \rfn{D-migCC-SSF} show a side-by-side
comparison of equivalent extrapolators in Riemannian and Cartesian
coordinates. The impulse response in \rFg{D-imgCC-bp} clearly shows
the failure of the Cartesian extrapolator in propagating waves
correctly even up to $90^\circ$. The Riemannian extrapolator in
\rFg{D-migCC-SSF} handles much better overturning waves, including
energy that is propagating upward relative to the vertical direction.
%%
\multiplot{4}{D-migRC-F15,D-migRC-PSC,D-migRC-F60,D-migRC-FFD}
{angle=0,width=0.45\textwidth}{\RWElabel}
%%
\multiplot{4}{D-migCC-F15,D-migCC-PSC,D-migCC-F60,D-migCC-FFD}
{angle=0,width=0.45\textwidth}{\CARlabel}
%%
\multiplot{2}{D-imgCC-bp,D-migCC-SSF}
{angle=0,width=0.80\textwidth}{\COMPlabel}
