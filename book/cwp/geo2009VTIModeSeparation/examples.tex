


%%%%%%%%%%%%%%%%%%%%%%%%%%%%%%%%%%%%%%%%%%%%%%%%%%%%%%%%%%%%%%

%%%%%%%%%%%%%%%%%%%%%%%%%%%%%%%%%%%%%%%%%%%%%%%%%%%%%%%%%%%%%%

\section{Examples}

I illustrate the anisotropic wave mode separation with a simple
synthetic example and a more challenging elastic Sigsbee 2A
model~\cite[]{SEG-2002-21222125}.
%\subsection{Isotropic media}
\subsection{Simple model}
I consider a 2D isotropic model characterized by the $V_{P}$, $V_{S}$
and density shown in \rfgs{separate3-aoppos}--\subrfn{separate3-ro}.
The model contains negative P and S velocity anomalies that triplicate
the wavefields.  The source is located at the center of the
model.  \rFg{separate3-uI-f06} shows the vertical and horizontal
components of one snapshot of the simulated elastic wavefield
{(generated using the $8^{th}$ order finite difference solution of the
elastic wave equation)}, \rfg{separate3-qI-f06} shows the separation
to P and S modes using $\DIV{}$ and $\CURL{}$ operators,
and \rfg{separate3-pI-f06} shows the mode separation obtained using
the pseudo operators which are dependent on the medium parameters.  A
comparison of
\rfgs{separate3-qI-f06} and \subrfn{separate3-pI-f06} indicates 
that the $\DIV{}$ and $\CURL{}$ operators and the pseudo operators
work identically well for this isotropic medium.

%\subsection{Anisotropic media}
I then consider a 2D anisotropic model similar to the previous model
shown in \rfgs{separate3-aoppos}--\subrfn{separate3-ro} (with $V_{P}$,
$V_{S}$ representing the vertical P and S wave velocities), and
additionally characterized by the parameters $\epsilon$ and $\delta$
shown in \rfgs{separate3-epsilon} and \subrfn{separate3-delta},
respectively.  The parameters $\epsilon$ and $\delta$ vary gradually
from top to bottom and left to right, respectively. The upper left
part of the medium is isotropic and the lower right part is highly
anisotropic. Since the difference of $\epsilon$ and $\delta$ is great
at the bottom part of the model, the {\it q}S waves in this region are
severely triplicated due to this strong anisotropy.

\rFg{separate3-aop00,aop10,aop20,aop01,aop12,aop21,aop02,aop12,aop22} 
illustrates the pseudo derivative operators obtained at different
locations in the model defined by the intersections of $x$ coordinates
0.3, 0.6, 0.9~km and $z$ coordinates 0.3, 0.6, 0.9~km. Since the
operators correspond to different combination of the parameters
$\epsilon$ and $\delta$, they have different forms. The isotropic
operator {at coordinates $x=0.3$~km and $z=0.3$~km, shown
in} \rfg{separate3-aop00}, is purely vertical and horizontal, while
the anisotropic operators (\rfg{separate3-aop10}
to \subrfn{separate3-aop22}) have ``tails'' radiating from the
center. {The operators become larger at locations where the medium is
more anisotropic, for example, at coordinates $x=0.9$~km and
$z=0.9$~km.}

\rFg{separate3-uA-f06} shows the vertical and horizontal components of 
one snapshot of the simulated elastic anisotropic
 wavefield, \rfg{separate3-qA-f06} shows the separation to {\it q}P and
 {\it q}S modes using conventional {isotropic} $\DIV{}$ and $\CURL{}$ operators,
 and \rfg{separate3-pA-f06} shows the mode separation obtained using the
pseudo operators constructed using the local medium parameters. 
A comparison of \rfg{separate3-qA-f06} and \rfn{separate3-pA-f06} indicates
that the spatially-varying derivative operators successfully separate
the elastic wavefields into {\it q}P and {\it q}S modes, while the
$\DIV{}$ and $\CURL{}$ operators only work in the isotropic region of
the model.


% ------------------------------------------------------------
\inputdir{separate3}
% ------------------------------------------------------------
\multiplot{5}{aoppos,vs,ro,epsilon,delta}{width=0.3\textwidth}
{A $1.2$~km$\times$ $1.2$~km model with parameters (a) $V_{p0}=3$~km/s
except for a low velocity Gaussian anomaly around $x=0.65$~km and
$z=0.65$~km, (b) $V_{S0}=1.5$~km/s except for a low velocity Gaussian
anomaly around $x=0.65$~km and $z=0.65$~km, (c) $\rho=1.0$~g/cm$^3$ in the
top layer and $2.0$~g/cm$^3$ in the bottom layer, (d) $\epsilon$
smoothly varying from $0$ to $0.25$ from top to bottom, (e) $\delta$
smoothly varying from $0$ to $-0.29$ from left to right. A vertical
point force source is located at $x=0.6$~km and $z=0.6$~km {shown
by the dot in panels (b), (c), (d), and (e). The dots in panel (a)
correspond to the locations of the anisotropic operators shown
in \rFg{separate3-aop00,aop10,aop20,aop01,aop12,aop21,aop02,aop12,aop22}
}.  }




% ------------------------------------------------------------
\multiplot{3}{uI-f06,qI-f06,pI-f06}{width=0.65\textwidth}
{(a) One snapshot of the isotropic wavefield modeled with a vertical
point force source at $x$=0.6~km and $z$=0.6~km for the model shown in
{\rfg{separate3-aoppos,vs,ro,epsilon,delta}, (b) isotropic P and S wave modes
separated using $\DIV{}$ and $\CURL{}$, and (c) isotropic P and S wave modes
separated using pseudo derivative operators. Both (b) and (c) show good separation
results.}}

\multiplot{3}{aop00,aop10,aop20,aop01,aop12,aop21,aop02,aop12,aop22}
{width=0.28\textwidth} {The $8^{th}$ order anisotropic pseudo
derivative operators in the $z$ and $x$ directions at the
intersections of $x$=0.3, 0.6, 0.9~km and $z$=0.3, 0.6, 0.9~km for 
the model shown in {\rfg{separate3-aoppos,vs,ro,epsilon,delta}}.
}



\multiplot{3}{uA-f06,qA-f06,pA-f06}{width=0.65\textwidth}
{(a) One snapshot of the anisotropic wavefield modeled with a vertical
point force source at $x$=0.6~km and $z$=0.6~km for the model shown in
{\rfg{separate3-aoppos,vs,ro,epsilon,delta}}, (b) anisotropic {\it q}P
and {\it q}S modes separated using $\DIV{}$ and $\CURL{}$, and (c)
anisotropic {\it q}P and {\it q}S modes separated using pseudo
derivative operators.  The separation of wavefields into {\it q}P and
{\it q}S modes in (b) is not complete, which is obvious at places such
as at coordinates $x=0.4$~km $z=0.9$~km. In contrast, the separation
in (c) is {much better, because the correct
anisotropic derivative operators are used.}.  }




\subsection{Sigsbee model}
My second model (\rfg{sigsbee-vp,vs,ro,epsilon,delta}) uses an
elastic anisotropic version of the Sigsbee 2A
model~\cite[]{SEG-2002-21222125}. In the modified model, $V_{P0}$ 
is taken from the original model, the $V_{P0}/V_{S0}$ ratio ranges
from $1.5$ to $2$, the parameter $\epsilon$ ranges from $0$ to $0.48$
(\rfg{sigsbee-epsilon}) and the parameter $\delta$ ranges $0$ from to
$0.10$ (\rfg{sigsbee-delta}). The model is isotropic in the salt and
the top part of the model. A vertical point force source is located at
coordinates $x=14.5$~km and $z=5.3$~km to simulate the elastic
anisotropic wavefield.

%\rFg{sigsbee-uI-f21} shows one snapshot of the modeled elastic isotropic vertical 
%and horizontal wavefields, using the model shown
%in \rfg{sigsbee-vp,vs,ro,epsilon,delta} and assuming the whole model
%is isotropic. \rFg{sigsbee-qI-f21} shows the separation of the elastic
%wavefields into P and S wave modes using $\DIV{}$ and $\CURL{}$
%operators. \rFg{sigsbee-pI-f21} shows the separation using
%derived pseudo derivative operators. Both types of operators work well
%for isotropic wavefields as expected.

\rFg{sigsbee-uA-f24-wom} shows one snapshot of the modeled elastic anisotropic 
wavefields using the model shown in \rfg{sigsbee-vp,vs,ro,epsilon,delta}. 
\rFg{sigsbee-qA-f24-wom} illustrates the separation of the anisotropic elastic
 wavefields using the $\DIV{}$ and $\CURL{}$ operators, and 
\rFg{sigsbee-pA-f24-wom} illustrates the separation using my pseudo 
derivative operators.  \rFg{sigsbee-qA-f24-wom} shows the residual of
unseparated P and S wave modes, such as at coordinates $x=13$~km and
$z=7$~km in the {\it q}P panel and at $x=11$~km and $z=7$~km in the
{\it q}S panel.  The residual of S waves in the {\it q}P panel
of \rfg{sigsbee-qA-f24-wom} is very significant because of strong
reflections from the salt bottom.  This extensive residual can be
harmful to under-salt elastic or even acoustic migration, if not
removed completely.  In contrast, \rfg{sigsbee-pA-f24-wom} shows the
{\it q}P and {\it q}S modes {better} separated,
demonstrating the effectiveness of the anisotropic pseudo derivative
operators constructed using the local medium parameters.  These
wavefields composed of well separated {\it q}P and {\it q}S modes are
essential to producing clean seismic images.

{In order to test the separation with a homogeneous assumption of 
anisotropy in the model, I show in} \rfg{sigsbee-mA-f24-wom} the
separation with $\epsilon=0.3$ and $\delta=0.1$ in the $k$
domain. This separation assumes a model with homogeneous
anisotropy. The separation shows that there is still residual in the
separated panels. Although the residual is much weaker compared to
separating using an isotropic model, it is still visible at locations
such as at coordinates $x=13$~km and $z=7$~km, and $x=13$~km and
$z=4$~km in the {\it q}P panel and at $x=16$~km and $z=2.5$~km in the
{\it q}S panel.

%%%%%%%%%%%%%%%%%%%%%%%%%%%%%%%%%%%%%%%%%%%%%%%%%%%%%%%%%%%%%%
\inputdir{sigsbee}
% ------------------------------------------------------------
\multiplot{5}{vp,vs,ro,epsilon,delta}{width=0.3\textwidth}
%\multiplot{5}{vp1,vs1,ro1,epsilon1,delta1}{width=0.3\textwidth}
{A Sigsbee 2A model in which (a) is the P wave velocity (taken from
the original Sigsbee 2A model~{\cite[]{SEG-2002-21222125} }), (b) is
the S wave velocity, where $V_{P0}/V_{S0}$ ratio ranges from $1.5$ to $2.0$,
(c) is the density ranging from $1.0$~g/cm$^3$ to $2.2$~g/cm$^3$, (d) is the
parameter $\epsilon$ ranging from $0.20$ to $0.48$, and (e) is the parameter
$\delta$ {ranging} from $0$ to $0.10$ in the rest of the model.  }

%\multiplot{3}{aop00,aop10,aop20,aop01,aop12,aop21,aop02,aop12,aop22}
%{width=0.3\textwidth} {The $8^{th}$ order anisotropic pseudo
%derivative operators in the $z$ and $x$ directions at the
%intersections of $x$=0.3, 0.6, 0.9~km and $z$=0.3, 0.6, 0.9~km.}
% ------------------------------------------------------------
%\multiplot{3}{uI-f21,qI-f21,pI-f21}{width=0.9\textwidth}
%{(a) Isotropic wavefield modeled with a vertical source at $x$=0.6~km
%and $z$=0.6~km, isotropic wave modes separated by (b) $\DIV{}$ and
%$\CURL{}$ and (c) pseudo derivative operators.}

\plot{uA-f24-wom}
{width=\textwidth} {Anisotropic wavefield modeled with a vertical
point force source at $x=14.3$~km and $z=5.3$~km for the model shown in
{\rfg{sigsbee-vp,vs,ro,epsilon,delta}}.}


\plot{qA-f24-wom}
{width=\textwidth} {Anisotropic {\it q}P and {\it q}S modes
separated using $\DIV{}$ and $\CURL{}$ for the vertical and horizontal
components of the elastic wavefields shown in
{\rfg{sigsbee-uA-f24-wom}}.  Residuals are obvious at places such as
at coordinates $x=13$~km and $z=7$~km in the {\it q}P panel and at
$x=11$~km and $z=7$~km in the {\it q}S panel.  }

\plot{pA-f24-wom}
{width=\textwidth} {Anisotropic {\it q}P and {\it q}S modes
separated using pseudo derivative operators for the vertical and
horizontal components of the elastic wavefields shown in
{\rfg{sigsbee-uA-f24-wom}}.  They show {better}
separation of {\it q}P and {\it q}S modes. }

%%%need to compute this
\plot{mA-f24-wom}
{width=\textwidth} {Anisotropic {\it q}P and {\it q}S modes
separated in the $k$ domain for the vertical and horizontal components
of the elastic wavefields shown in {\rfg{sigsbee-uA-f24-wom}}. The
separation assumes $\epsilon=0.3$ and $\delta=0.1$ throughout the
model. The separation is incomplete. Residuals are still visible at
places such at coordinates $x=13$~km and $z=7$~km, and $x=13$~km
and $z=4$~km in the {\it q}P panel and at $x=16$~km and $z=2.5$~km in
the {\it q}S panel. }

