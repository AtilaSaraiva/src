% ------------------------------------------------------------
\begin{abstract}
%motivation
{Elastic wave propagation in anisotropic media is well represented by
elastic wave equations. Modeling based on elastic wave equations
characterizes both kinematics and dynamics correctly. However, because
P and S modes are both propagated using elastic wave equations, there
is a need to separate P and S modes to obtain clean elastic images.}
% scope and princople objectives of the research
The separation of wave modes {to P and S} from isotropic elastic
wavefields is typically done using Helmholtz decomposition. However,
Helmholtz decomposition using conventional divergence and curl
operators in anisotropic media does not give satisfactory results and
leaves the different wave modes only partially separated. The
separation of anisotropic wavefields requires the use of more
sophisticated operators which depend on local material parameters.
% methods used
Anisotropic wavefield separation operators are constructed using the
polarization vectors evaluated by solving
the Christoffel equation at each point of the medium. These
polarization vectors can be represented in the space domain as
localized filtering operators, which resemble conventional derivative
operators.
% summerize the results
The spatially-variable ``pseudo'' derivative operators perform well in
heterogeneous VTI media even at places of rapid {velocity/density}
variation.
% principle conclusion
Synthetic results indicate that the operators can be used to separate
wavefields for VTI media with an arbitrary degree of anisotropy. 
\end{abstract}
