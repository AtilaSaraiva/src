% ------------------------------------------------------------
\section{Introduction}
%Review
Wave equation migration for elastic data usually consists of two
steps. The first step is wavefield reconstruction in the subsurface
from data recorded at the surface. The second step is {the
application of} an imaging condition which extracts reflectivity
information from the reconstructed wavefields.

The elastic wave equation migration for multicomponent data can be
implemented in two ways. The first approach is to separate recorded
elastic data into compressional and transverse (P and S) modes and
use the separated data for acoustic wave equation migration
separately. This acoustic imaging approach to elastic waves is more
frequently used, but it is fundamentally based on the assumption that
P and S data can be successfully separated on the surface, which is
not always true
\cite[]{etgen:972,GEO62-02-05980613}.
% elastic wavefield reconstruction (extrapolation), which can be done
% using various techniques, including RTM (reference)
The second approach is to not separate P and S modes on the surface,
but to extrapolate the entire elastic wavefield at once, and then
separate wave modes prior to applying an imaging condition. The
reconstruction of elastic wavefields can be implemented using various
techniques, including reconstruction by time reversal (RTM)
\cite[]{chang:67,chang:597} or by Kirchhoff integral 
techniques~\cite[]{hokstad:861}.
% imaging condition applied to the reconstructed vector wavefields
% (cite myself)

The imaging condition applied to the reconstructed vector wavefields
directly determines the quality of the images. Conventional
crosscorrelation imaging condition does not separate the wave modes
and crosscorrelates the Cartesian components of the
elastic.  In general, the various wave
modes (P and S) are mixed on all wavefield components and cause
crosstalk and image artifacts.
% imaging condition requires that cross-correlate performed on
% separated modes, thus the need for wavefield separation
\cite{yan:WB19} suggest using imaging conditions based on elastic 
potentials, which require crosscorrelation of separated modes.
Potential-based imaging condition creates images that have clear
physical meaning, in contrast with images obtained with Cartesian
wavefield components, thus justifying the need for wave mode
separation.

As the need for anisotropic imaging increases, more processing and
migration are performed based on anisotropic acoustic one-way wave
equations~\cite[]{alkhalifah:623,GEO65-04-12391250,shan:2367,
shan:104,fletcher:WCA179,fowler:S11}.
However, much less research has been done on anisotropic elastic
migration based on two-way wave equations. Elastic Kirchhoff
migration~\cite[]{hokstad:861} obtains pure-mode and converted mode
images by downward continuation of elastic vector wavefields with a
visco-elastic wave equation. The wavefield separation is effectively
done with elastic Kirchhoff integration, which handles both P and S
waves.  However, Kirchhoff migration does not perform well in areas of
complex geology where ray theory breaks
down~\cite[]{GEO66-05-16221640}, thus requiring migration with more
accurate methods, such as reverse time migration.


% ------------------------------------------------------------

%Claim
One of the complexities that impedes elastic wave equation anisotropic
migration is the difficulty to separate anisotropic wavefields into
different wave modes after reconstructing the elastic
wavefields. However, the proper separation of anisotropic wave modes
is as important for anisotropic elastic migration as is the separation
of isotropic wave modes for isotropic elastic migration. The main
difference between anisotropic and isotropic wavefield separation is
that Helmholtz decomposition is only suitable for the separation of
isotropic wavefields and is inadequate for anisotropic wavefields.

%Agenda
In this chapter, I show how to construct wavefield separators for VTI
(vertical transverse isotropy) media applicable to models with
spatially varying parameters.  I apply these operators to anisotropic
elastic wavefields and show that they successfully separate
anisotropic wave modes, even for extremely anisotropic media.

The main application of this technique is in the development of
elastic reverse time migration. In this case, complete wavefields
containing both P and S wave modes are reconstructed from recorded
data. The reconstructed wavefields are separated in pure wave modes
prior to the application of a conventional crosscorrelation imaging
condition. I limit the scope of this chapter only to the wave-mode
separation procedure in highly heterogeneous media, although the
ultimate goal of this procedure is to aid elastic RTM.
