\section{Time-shift imaging in Kirchhoff migration}
The imaging condition described in the preceding section
has an equivalent formulation in Kirchhoff imaging.
Traditional construction of common-image gathers using
Kirchhoff migration is represented by the expression
\beq \label{eqn:KirOld}
\RR \lp \mm,\ho \rp =
\sum_{\mo} 
\Uo \left[\mo,\ho, 
        t_s \lp \mm,\mo-\ho \rp +
        t_r \lp \mm,\mo+\ho \rp
    \right]
\;,
\eeq
where $\Uo\lp\mo,\ho,t\rp$ is the recorded wavefield at 
the surface as a function of surface 
midpoint $\mo$ and offset $\ho$ (Figure~\ref{fig:kir}).
$t_s$ and $t_r$ stand for traveltimes from sources and 
receivers at coordinates $\mo-\ho$ and $\mo+\ho$ to points 
in the subsurface at coordinates $\mm$.
For simplicity, the amplitude and phase correction term 
$A \lp \mm,\mo,\ho \rp \frac{\partial}{\partial t}$
is omitted in \req{KirOld}.
% ------------------------------------------------------------
\inputdir{XFig}
\plot{kir}{width=3.0in}{
Notations for Kirchhoff imaging.
S is a source and R is a receiver.}
% ------------------------------------------------------------
The time-shift imaging condition can be implemented 
in Kirchhoff imaging using a modification
of \req{KirOld} that is equivalent to 
\reqs{imgT} and \ren{imgTw}:
\beq \label{eqn:KirNew}
\RR \lp \mm,\tt \rp=\sum_{\mo} \sum_{\ho}
\Uo \left[                \mo,\ho, 
              t_s \lp \mm,\mo-\ho \rp +
              t_r \lp \mm,\mo+\ho \rp + 2 \tt
    \right] \;.
\eeq

Images obtained by Kirchhoff migration as discussed in 
\req{KirNew} differ from image constructed with \req{KirOld}.
Relation~\ren{KirNew} involves a double summation over
surface midpoint $\mo$ and offset $\ho$ to produce an image 
at location $\mm$. Therefore, the entire input data contributes
potentially to every image location.
This is advantageous because migrating using
relation~\ren{KirOld} different offsets $\ho$ independently
may lead to imaging artifacts as discussed by 
\cite{GEO69-02-05620575}.
After Kirchhoff migration using relation~\ren{KirNew},
images can be converted to the angle domain using \req{angT}.
