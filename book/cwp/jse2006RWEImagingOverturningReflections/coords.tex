\section{Coordinate systems}
Riemannian wavefield extrapolation operates in coordinate systems that may or may not be defined according to the model used for imaging. As indicated earlier, there are several options for constructing such coordinate systems, but the solution selected for the case described in this paper uses ray tracing. In this case, the coordinate system is semi-orthogonal, i.e. the extrapolation direction is orthogonal to the other two directions defining a 3D coordinate system. Cartesian coordinates are special cases of Riemannian coordinates constructed by tracing rays orthogonal to a flat surface in constant velocity.
\par
The accuracy of one-way wavefield extrapolation operators maximizes in the direction of extrapolation (vertical for downward continuation; along rays for Riemannian coordinate systems constructed by ray tracing). In addition, for steeply dipping reflectors, the angle of reflection is likely to be relatively small since this reflector is illuminated from a large distance using a small limite acquisition aperture. 
\par
It is thus desirable to construct a coordinate system that minimizes the angles between the extrapolation direction, and the directions of wave propagation and normal to the imaged reflectors. One way to achieve this goal is to construct the coordinate system by tracing rays orthogonal to an imaginary line (plane in 3D) located behind the overhanging reflector. 
\par
These ideas are illustrated with a model based on the Sigsbee 2A synthetic \cite[]{SEG-2002-21222125} extended vertically and horizontally to allow diving waves from the overhanging salt flank to arrive at the surface. \rFg{cos1} shows an example of Riemannian coordinates constructed from ray tracing from behind the salt flank reflector. For comparison, \rFg{cos2} shows a Cartesian coordinate system tilted relative to the vertical direction to minimize the angle between the extrapolation direction and the normal to the reflector.

\inputdir{sigsbee}

\multiplot{2}{cos1,cos2}{angle=90,width=0.45\textwidth}
{Riemannian coordinate system (a) and 
tilted Cartesian coordinate system (b).}

As indicated in the preceding section, we can describe Riemannian coordinate systems with several coefficients incorporating all the information about the coordinate system shape and the extrapolation slowness. For this 2D example, there are two coefficients, $a$ and $b$ depicted in \rFgs{abmRC1a}-\rfn{abmRC1b} for the Riemannian coordinate system and in \rFgs{abmRC2a}-\rfn{abmRC2b} for the tilted Cartesian coordinate system. The plots depict $a$ and $b$ function of the Riemannian coordinates, $\tau$ and $\gamma$. $\tau$ has time units and it represents the extrapolation direction, and $\gamma$ is non-dimensional and represents an index of the rays shot from the linear origin behind the imaged reflector.
\par
Coefficient $a$ describes the ratio of the extrapolation velocity to the velocity used for ray tracing, and coefficient $b$ describes the focusing/defocusing of the coordinate system. In both cases, coefficient $a \ne 1$ since the velocity used for extrapolation is different from the velocity used for the coordinate system. For the Cartesian coordinate system, coefficient $b$ is a constant, as depicted in \rFg{abmRC2b}.

\rFgs{edgeRC1} and \rfn{edgeRC2} depict the acquisition surface and the salt body outline mapped in Riemannian and tilted Cartesian coordinates, respectively. We can observe that in Riemannian coordinates, the overhanging salt flank is nearly orthogonal to the extrapolation direction, unlike its layout in tilted Cartesian coordinates. Therefore, imaging accuracy for such reflectors can be achieved in Riemannian coordinates with lower order extrapolation kernels than in the case of tilted Cartesian coordinates, as demonstrated in the following section.

\multiplot{3}{abmRC1a,abmRC1b,edgeRC1}{width=0.5\textwidth}
{Riemannian coordinate system coefficients, (a) and (b), and outline of acquisition geometry and salt body (c).}

\multiplot{3}{abmRC2a,abmRC2b,edgeRC2}{width=0.5\textwidth}
{Tilted Cartesian coordinate system coefficients, (a) and (b), and outline of acquisition geometry and salt body (c).
}

