\section{Introduction}
All anisotropy arises from ordered heterogeneity much smaller than the wavelength \cite[]{winterstein}.
 With the increased resolution of seismic data and because of wider seismic acquisition aperture
(both with respect to offset and azimuth), there is a growing awareness that an 
isotropic description of the Earth may no longer be adequate. Anisotropy appears to be a near-ubiquitous
 property of earth materials, and its effects on seismic data must be quantified.

 Wave equation is the central ingredient in characterizing wave propagation for seismic imaging and elastic
parameters inversion. In isotropic media, it is common to use scalar acoustic wave 
equations to describe the propagation of seismic data as representing only P-wave energy \cite[]{yilmaz}.
 Compared to the elastic wave equation, the acoustic wave equation is simpler and
 more efficient to use, and does not yield S-waves for modeling of P-waves.
 Anisotropic media are inherently described by elastic wave equations with P- and S-wave modes
 intrinsically coupled.
It is well known that a S-wave passing through an anisotropic medium splits into two mutually
 orthogonal waves \cite[]{crampin:1984}. Generally the P-wave and the two S-waves are not polarized parallel 
 and perpendicular to the wave vector, thus are called quasi-P (qP) and qausi-S (qS) waves.
 However, most anisotropic migration implementations
 do not use the full elastic anisotropic wave equation because of the high computational cost involved, and
 the difficulties in separating wavefields into different wave modes. Although an acoustic wave does not
 exist in anisotropic media, \cite{alkhalifah:2000} introduced a pseudo-acoustic
 approximate dispersion relation for vertically transverse isotropic (VTI) media by setting the shear 
velocity along the axis of symmetry to zero, which leads to a fourth-order partial differential equation (PDE)
 in space-time domain.
Following the same procedure, he also presented a pseudo-acoustic wave equation of sixth-order in
vertical orthorhombic (ORT) anisotropic media \cite[]{alkhalifah:2003}. Many authors have
 implemented pseudo-acoustic VTI modeling and migration based on
 various coupled second-order PDE systems derived
 from Alkhalifah's dispersion relation \cite[]{alkhalifah:2000,klie,zhou:2006eage,hestholm}.
Alternatively, coupled first-order and second-order systems are derived starting from first principles
 (the equations of motion and Hooke's law) under the pseudo-acoustic approximation for
 VTI media \cite[]{duveneck:2011} and recently for orthorhombic media 
as well \cite[]{fowler.king:2011,zhang.zhang:2011}.
 The pseudo-acoustic tilted transversely isotropic (TTI) or tilted orthorhombic wave equations
 can be obtained from their pseudo-acoustic VTI (or pseudo-acoustic vertical orthorhombic)
counterparts by simply performing a coordinate rotation according to the directions of the
 symmetry axes \cite[]{zhou:2006seg,fletcher:2009}. 
Pseudo-acoustic wave equations have been
 widely used for RTM in transversely isotropic (TI) media because
they describe the kinematic signatures of qP-waves
 with sufficient accuracy and are simpler than their elastic 
counterparts, which leads to computational savings in practice.

On the other hand, the pseudo-acoustic approximation may result in some problems in characterizing wave propagation
 in anisotropic media. First, to enhance computational stability,
pseudo-acoustic approximations reduce the freedom
 to choose the material parameters compared with their elastic counterparts \cite[]{grechka:2004}.
Practitioners often observe instability in practice when the pseudo-acoustic
 equations are used in complex TI media \cite[]{fletcher:2009,zhang:2011,bube:2012}.
 Stable RTM implementations in TTI media 
can be achieved by using pseudo-acoustic wave equations derived directly from first
 principles \cite[]{duveneck:2011} 
using self-adjoint or covariant derivative operators \cite[]{macesanu,zhang:2011}. 
Second, the widely-used pseudo-acoustic approximation
 still results in significant shear wave presence in both modeling data and RTM images \cite[]{zhang:2003,
 grechka:2004, jin}. It is not easy to get rid of qSV-waves from
 the wavefields simulated by the pseudo-acoustic
  wave equations when a full waveform modeling for 
qP-wave is required. Placing both sources and receivers into an artificial isotropic or elliptic
 anisotropic acoustic layer could eliminate many of the undesired
qSV-wave energy \cite[]{alkhalifah:2000}, but the propagated qP-wave 
may get converted to qSV-wave and the qSV-wave might get
 converted back to qP-wave in other portions of the model.
 A projection filtering based on an approximate representation of characteristic-waveform
of qP-waves was suggested to suppress undesired 
qSV-wave energy at each output time step \cite[]{zhang:2009}.
But qS-wave artifacts still remain and
 qP-wave amplitudes may be not correctly restored due to the
 approximation introduced in the used wave equation. To avoid the qSV-wave energy completely,
different approaches have recently been proposed to model the pure qP-wave
 propagation in VTI and TTI media. The optimized separable approximation \cite[]{liu:2009, zhang.zhang:2009,
 du:2010}, the pseudo-analytical method \cite[]{etgen:2009}, the low-rank
 approximation \cite[]{fomel:2013}, the Fourier finite-difference method \cite[]{song.fomel:2011} and the
 rapid expansion method \cite[]{pestana.stoffa:2010} belong to this category.

 In fact, anisotropic phenomena are especially noticeable in shear and mode-converted wavefields.
 Therefore, modeling of anisotropic shear waves
may be important both on theoretical and practical aspects. \cite{liu:2009}
 factorized the pseudo-acoustic VTI dispersion relation
 and obtained two pseudo-partial differential (PPD) equations, of which the qP-wave equation
is well-posed for any value of the anisotropic parameters, but the qSV-wave
 equation becomes well-posed only when the condition $\epsilon > \delta$ is satisfied. These PPD equations are very hard
 to solve in heterogeneous media unless further approximations are introduced \cite[]{liu:2009,
 chu:2011} or recently developed FFT-based approaches are used \cite[]{pestana:2011,song.fomel:2011,fomel:2013}.
Note that some of the above efforts to model pure-mode wavefields suffer from accuracy
 loss more or less due to the approximations to the phase velocities or dispersion relations. Furthermore,
these pure-mode propagators only consider the phase term in wave propagation, so they are appropriate
 for seismic migration but not necessarily for accurate seismic modeling, which may require taking account 
of amplitude effects and other elastic phenomena such as mode conversion.

In kinematics, there are various forms equivalent to the original first- or second-order elastic wave equations.
Mathematically, analysis of the dispersion relation as matrix eigenvalue system allows one to generate equivalent
coupled linear second-order systems by similarity transformations \cite[]{fowler:2010}.
Accordingly, \cite{kang.cheng:2011a} proposed new coupled second-order systems for both
qP- and qS-waves
 in TI media by applying specified similarity transformations to the Christoffel equation.
 Their coupled system for qP-waves represents dominantly
the energy propagation of scalar qP-waves while that for qSV-waves 
propagates dominantly the scalar qSV-wave
 energy. However, each of the two systems still contains
 relatively weak residual energy of the other mode. 
\cite{cheng.kang:2012} and \cite{kang.cheng:2012} 
 called such coupled systems ``pseudo-pure-mode wave equations'' and further proposed an approach
to get separated qP- or qS-wave data from the pseudo-pure-mode wavefields in general anisotropic media.
In the two articles of this series, we demonstrate how to simulate propagation of separated wave-modes based on 
a new simplified description of wave propagation in general anisotropic media.
We shall focus on qP- and qS-waves in each article separately.

The first paper is structured as follows: First, we revisit the plane-wave analysis of the full elastic anisotropic
 wave equation. Then we introduce a similarity transformation to the Christoffel equation required to
 derive the pseudo-pure-mode qP-wave equation, and give the simplified forms under pseudo-acoustic or/and isotropic
 approximations to illustrate the physical interpretation.
 After that, we discuss how to obtain separated qP-wave
 data from the extrapolated wavefields coupled with residual qS-waves.
Finally, we show numerical examples
to demonstrate the features and advantages of our approach in wavefield modeling and RTM in 
TI and orthorhombic media.
