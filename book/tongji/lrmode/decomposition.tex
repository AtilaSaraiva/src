\section{Theory of elastic wave vector decomposition}

Wavefield vector decomposition aims to achieve wave mode separation and vector decomposition simultaneously.
Based on the Helmholtz theory and the theory of wave mode separation in anisotropic
media via the Christoffell equation, \cite{zhang.mcmechan:2010} 
develop a new solution to the problem of decomposition of an elastic wavefield into P- and S-waves for
isotropic and VTI media. I summarize here only the results used in this study.

For isotropic media, the Helmholtz equations for P wave (equation 2 and 3) are transformed into the wavenumber
 domain as,
\begin{equation}
\mathbf{k}\times\tilde{\mathbf{U}}^P = \mathbf{0},
\end{equation}
and
\begin{equation}
\mathbf{k}\cdot\tilde{\mathbf{U}} = \mathbf{k}\cdot\tilde{\mathbf{U}}^P.
\end{equation}
From these two equations, the solution for the three components of P-wavefield in Fourier domain is
 \cite[]{zhang.mcmechan:2010}
\begin{equation}
\begin{array}{l}
\displaystyle
\tilde{U}^P_{x} = k_{x}^2\tilde{U}_{x} + k_{x}k_{y}\tilde{U}_{y} + k_{x}k_{z}\tilde{U}_{z},
\nonumber \\
\displaystyle
\tilde{U}^P_{y} = k_{x}k_{y}\tilde{U}_{x} + k_{y}^2\tilde{U}_{y} + k_{y}k_{z}\tilde{U}_{z},
\nonumber \\
\displaystyle
\tilde{U}^P_{z} = k_{x}k_{z}\tilde{U}_{x} + k_{y}k_{z}\tilde{U}_{y} + k_{z}^2\tilde{U}_{z},
\end{array}
\end{equation}
where $k_{x}$, $k_{y}$ and $k_{z}$ have been normalized by $|\mathbf{k}|$ and thus are dimensionless.
In short form, we can rewrite equation 23 as
\begin{equation}
\tilde{\mathbf{U}}^P = \mathbf{k}(\mathbf{k}\cdot\tilde{\mathbf{U}}).
\end{equation}

In isotropic media, the wavenumber $\mathbf{k}$ is not only the wave-propagation direction but also the
P-wave-polarization direction. The P-wave and S-wave polarizations are perpendicular to each other, so
polarization can be used as the basis of P- and S-wavefield decomposition.
In anisotropic media, wave propagation direction is neither parallel nor perpendicular to the P-, SV-,
 or SH-wave polarization, so $\mathbf{k}$ cannot be used directly to decompose the elastic wavefields.
Therefore, \cite{zhang.mcmechan:2010} extended equation 24 to decompose qP wavefield with substituting
 $\mathbf{a}_{p}$ (the normalized polarization vector of qP wave) for $\mathbf{k}$, namely,
\begin{equation}
\tilde{\mathbf{U}}^P = \mathbf{a}_{p}(\mathbf{a}_{p}\cdot\tilde{\mathbf{U}}).
\end{equation}
Three-dimensional inverse Fourier transforms of $\tilde{U}_{x}^{P}$, $\tilde{U}_{y}^{P}$ and $\tilde{U}_{z}^{P}$
give the 3D, 3-C decomposed P-wavefield in space domain. 
Similarly,
\begin{equation}
\tilde{\mathbf{U}}^{SV} = \mathbf{a}_{sv}(\mathbf{a}_{sv}\cdot\tilde{\mathbf{U}}).
\end{equation}
and
\begin{equation}
\tilde{\mathbf{U}}^{SH} = \mathbf{a}_{sh}(\mathbf{a}_{sh}\cdot\tilde{\mathbf{U}}).
\end{equation}
decompose the qSV and SH waves using their owns polarization vectors, respectively.
Equation 25 was first presented by \cite{dellinger.thesis} to avoid the sign-choice problem in determine
the polarization directions around the singularities in a slightly modified wave mode separation algorithms.

Note that, for TI media, the above vector decomposition honors the linear superposition relation:
\begin{equation}
\tilde{\mathbf{U}} = \tilde{\mathbf{U}}^{P} + \tilde{\mathbf{U}}^{SH} + \tilde{\mathbf{U}}^{SV},
\end{equation}
and $\tilde{\mathbf{U}}^{P}$, $\tilde{\mathbf{U}}^{SH}$ and $\tilde{\mathbf{U}}^{SV}$ are perpendicular to one another.
The separated P and S wavefields have the same amplitude, phase, and physical units as the input wavefields.
in equation 25 give the 3D, 3-C decomposed P-wavefield in space.

For numerical implementation, we rewrite equation 25 in long form (line in equation 23) as 
\begin{equation}
\begin{array}{l}
\displaystyle
\tilde{U}^P_{x} = a_{px}^2\tilde{U}_{x} + a_{px}a_{py}\tilde{U}_{y} + a_{px}a_{pz}\tilde{U}_{z},
\nonumber \\
\displaystyle
\tilde{U}^P_{y} = a_{px}a_{py}\tilde{U}_{x} + a_{py}^2\tilde{U}_{y} + a_{py}a_{pz}\tilde{U}_{z},
\nonumber \\
\displaystyle
\tilde{U}^P_{z} = a_{px}a_{pz}\tilde{U}_{x} + a_{py}a_{pz}\tilde{U}_{y} + a_{pz}^2\tilde{U}_{z}.
\end{array}
\end{equation}
Like wave mode separation, this wavefield vector decomposition is also based on polarization. 
Comparing equation 28 with 10, we find the main steps involved in decomposition of each component of certain wave mode 
are similar to those for separation of a single mode, except the decomposition operators need more multiplication operations. 
Therefore, we should spend at least three times of the computations for wave mode separation to finish wavefield vector decomposition.

For heterogeneous media, \cite{zhang.mcmechan:2010} suggest to decompose the wavefield separately for 
each geological unit that has a different polarization. Obviously, it is difficult to disassemble the
 medium to apply their strategy when the medium is not comprised of distinct units but has continuously
 changing parameters.
