\section {Examples}

\new{This section contains four examples. They are for 2D and 3D two-layer models, the SEG/Hess VTI model
and the BP 2007 TTI model, respectively. We use 10th-order finite-differnece algorithm for elastic wavefield 
extrapolation. To accurately compare the used CPU times, algorithm parallelization is not considered for 
wave mode separation and vector decomposition.}
}

\subsection{A 2D two-layer TI model}
\old{For all examples in this section, relative accuracy of $10^{-6}$ is required for the low-rank decomposition.}
We first test our approach on a two-layer TI model \new{with the size of $N_{x}=401\times401$.}\old{, in which t}\new{T}he 
first layer is a \old{strong} VTI medium
with $v_{p0}=2500 m/s$, $v_{s0}=1200 m/s$, $\epsilon=0.25$, and $\delta=-0.25$,
and the second layer is a TTI medium with $v_{p0}=3600 m/s$, $v_{s0}=1800 m/s$, $\epsilon=0.2$, 
$\delta=0.1$, and \new{the tilt angle} $\theta=30^{\circ}$. A point-source is placed at the center of this model.
To aim for the relative accuracy, rank $N=M=2$ is re\new{q}uired for both mode separation and vector decomposition.
Thanks to small approximation errors \old{of}\new{in} low-rank decompositions (see Figures~\ref{fig:Polxp1,Polzp1,Polxp2,Polzp2,Errpolxp1,Errpolzp1,Errpolxp2,Errpolzp2} and Figure~\ref{fig:Decompxp1,Decompzp1,Decompxzp1,Errdecxp1,Errdeczp1,Errdecxzp1,Decompxp2,Decompzp2,Decompxzp2,Errdecxp2,Errdeczp2,Errdecxzp2}), we obtain good mode separation 
 and vector decomposition for the synthesized elastic wavefields (see Figures 2 and 4).
\new{It took CPU time of 7.5 seconds to construct the separated forms (as equation 15 expressed) of the mode separation matrixes 
	for qP- and qSV-waves.}
For one time step, it took CPU time of 0.12, 0.21, and 0.33 seconds to extrapolate, separate and decompose the elastic
wavefields, repectively.

\subsection{A 3D two-layer TI model}
We also test the mode separation approach on a 3D two-layer TI model, with 
$v_{p0}=2500 m/s$, $v_{s0}=1200 m/s$, $\epsilon=0.25$, $\delta=-0.25$ and $\gamma=0$ in the first layer, and
$v_{p0}=3600 m/s$, $v_{s0}=1800 m/s$, $\epsilon=0.2$, $\delta=0.1$ and $\gamma=0.05$ in the second layer.
\new{The size of the model is $N_{x}=201\times201\times201$.}
A displacement source located at the center of the model and oriented at tilt
$45^{\circ}$ and azimuth $45^{\circ}$. Figure 5 displays the elastic wavefields and the separated qP-, qSV- and
SH-wave fields using the low-rank approximate algorithm.
\new{
Because the substantial increase of the model size, it is still time consuming to separate the 3D wave modes even if 
the proposed fast algorithm is used.
It took 4008.0, 4130.0 and 91.8 seconds to construct the separated forms of the mode separation matrixes for qP-, qSV- and SH-waves, respectively.
For one time step, it took 61.4 seconds to extrapolate the elastic wavefield, and 15.2, 15.8 and 6.8 seconds to
separate qP-, qSV- and SH-wave fields with the rank $N=M=2$.}

\new{
For comparison, we only change the second layer to a TTI medium with a tilt angle $\theta=30^{\circ}$ and azimuth $\phi=30^{\circ}$
(other paramters continue to use).
Figure 6 displays the corresponding elastic wavefields and their mode separation results.
It took 4087.8, 4280.8 and 206.2 seconds to construct the separated forms of the mode separation matrixes for qP-, qSV- and SH-waves, respectively.
For one time step, it took 101.0 seconds to extrapolate the elastic wavefield, and 15.2 and 15.8 seconds to separate qP- and qSV-wave modes with
the rank $N=M=2$. It took 14.1 seconds to separate SH-wave with the rank $N=M=2$.
As we observed, the most time-consuming task here is to construct the separated forms of the mode separation matrixes.
More CPU time is required to separate SH-wave in 3D TTI media as well.
}

\inputdir{twolayer2dti}

\multiplot{8}{Polxp1,Polzp1,Polxp2,Polzp2,Errpolxp1,Errpolzp1,Errpolxp2,Errpolzp2}{width=0.28\textwidth}
{
Low-rank approximate mode separators of qP-wave in a 2D two-layer TI model:
(a) $a_{px}(\mathbf{x},\mathbf{k})$ and (b) $a_{pz}(\mathbf{x},\mathbf{k})$ constructed by using low-rank decomposition in the VTI layer;
(c) $a_{px}(\mathbf{x},\mathbf{k})$ and (d) $a_{pz}(\mathbf{x},\mathbf{k})$ constructed by using low-rank decomposition in the TTI layer;
(e), (f), (g) and (h) represent the low-rank approximation errors of these operators.
According to the qP-qSV mode polarization orthogonality, we have the following relations:
$a_{svx}(\mathbf{x},\mathbf{k})=-a_{pz}(\mathbf{x},\mathbf{k})$ and
$a_{svz}(\mathbf{x},\mathbf{k})=a_{px}(\mathbf{x},\mathbf{k})$. Therefore, the above pictures also
demonstrate the low-rank approximate separators and their errors for qSV-wave.
}
\multiplot{4}{ElasticxInterf,ElasticzInterf,ElasticSepPInterf,ElasticSepSVInterf}{width=0.28\textwidth}
{
Elastic wave mode separation in the two-layer TI model:
(a) x- and (b) z-components of the synthetic elastic displacement wavefields synthesized at 0.3s;
(c) and (d) are the separated scalar qP- and qSV-wave fields using low-rank approximation.
}

\multiplot{12}{Decompxp1,Decompzp1,Decompxzp1,Errdecxp1,Errdeczp1,Errdecxzp1,Decompxp2,Decompzp2,Decompxzp2,Errdecxp2,Errdeczp2,Errdecxzp2}{width=0.28\textwidth}
{
Low-rank approximate vector decomposition operators of qP-wave in the 2D two-layer TI model:
(a) $a_{px}(\mathbf{x},\mathbf{k})a_{px}(\mathbf{x},\mathbf{k})$,
	(b) $a_{pz}(\mathbf{x},\mathbf{k})a_{pz}(\mathbf{x},\mathbf{k})$,
	and (c) $a_{px}(\mathbf{x},\mathbf{k})a_{pz}(\mathbf{x},\mathbf{k})$,
	and (e), (f) and (g) represent their low-rank approximation errors in the VTI layer.
(h), (i), (j), (k), (l) and (m) are these operators and their low-rank approximation errors in the TTI layer.
\old{Similarly, these pictures also demonstrate the low-rank approximate vector decomposition operators
and their errors for qSV-wave.}
}

\multiplot{4}{ElasticPxInterf,ElasticPzInterf,ElasticSVxInterf,ElasticSVzInterf}{width=0.30\textwidth}
{
Elastic wave vector decomposition in the two-layer VTI/VTI model:
(a) x- and (b) z-components of vector qP-wave fields;
(c) x- and (d) z-components of vector qSV-wave fields.
}

\inputdir{twolayer3dvti}
\multiplot{6}{ElasticxInterf,ElasticyInterf,ElasticzInterf,ElasticPInterf,ElasticSVInterf,ElasticSHInterf}{width=0.32\textwidth}
{
Elastic wave mode separation in the 3D two-layer VTI model:
(a) x-, (b) y- and (c) z-components of the synthetic elastic displacement wavefields synthesized at 0.17s;
(d) qP-, (e) qSV- and (e) SH-wave fields separated from the elastic wavefields.
}

\inputdir{twolayer3dtti}
\multiplot{6}{ElasticxInterf,ElasticyInterf,ElasticzInterf,ElasticPInterf,ElasticSVInterf,ElasticSHInterf}{width=0.32\textwidth}
{
Elastic wave mode separation in the 3D two-layer VTI/TTI model:
(a) x-, (b) y- and (c) z-components of the synthetic elastic displacement wavefields synthesized at 0.17s;
(d) qP-, (e) qSV- and (e) SH-wave fields separated from the elastic wavefields.
}

\subsection{SEG Hess VTI model}
\inputdir{hessvti}

Then we demonstrate the approach in the 2D Hess VTI model (Figure~\ref{fig:vp-hess,epsilon-hess,delta-hess}).
Vertical S-wave velocity is set \old{a}\new{to equal} half the vertical P-wave velocity everywhere. A point-source is placed
 at \old{the} location of (13.264, 4.023) km.
Figure 7 shows results of mode separation and vector decomposition, with the rank of about $6$ in both cases.
It took 133.0 seconds to decompose the operator matrixes for mode separation with rank $N=M=6$, and 154.1 seconds to decompose the operator matrixes for vector decomposition with rank $N, M\in[6,7]$.
For one time step, it took about 2.2, 4.0, and 7.7 seconds to extrapolate, separate and decompose the elastic wavefields, repectively.
However, if nonstaionary spatial filtering is used to separate qP- and qSV-waves at every grid-point, it took about 2340.7 seconds
to calculate the filters in advance, and about 444.0 seconds with the truncated operator of size $51\times51$ 
at each time step during wavefield extrapolation.
 This indicates that the mixed domain algorithms using low-rank approximation significantly improve the efficiency for wave mode separation.
Of course, larger amount of CPU time has been saved for vector decomposition by using the corresponding low-rank approximate algorithm.

\multiplot{3}{vp-hess,epsilon-hess,delta-hess}{width=0.4\textwidth}
{
Hess VTI model with parameters of (a) vertical P-wave velocity, Thomsen coefficients (b) $\epsilon$ and (c) $\delta$.
}

\multiplot{8}{Elasticx,Elasticz,ElasticSepP,ElasticSepSV,ElasticPx,ElasticPz,ElasticSVx,ElasticSVz}{width=0.28\textwidth}
{
Elastic wave mode separation and vector decomposition in the Hess VTI model:
(a) x- and (b) z-components of the synthetic elastic displacement wavefields at 1.1s;
(c) and (d) are the separated scalar qP- and qSV-wave fields;
(e) x- and (f) z-components of vector qP-wave fields;
(g) x- and (h) z-components of vector qSV-wave fields.
} 

\subsection{BP 2007 TTI model}
\inputdir{bptti2007}
This example displays the wave mode separation and vector decomposition results in the BP 2007 TTI  model (Figure 9).
A point-source is placed near the second salt body at the location of (35.625, 5.0) km.
Before wavefield extrapolation, separated representations of the operator matrixes are constructed using the 
low-rank decomposition approach within the computational zone.
It took 217.0 seconds to decompose the operator matrixes for mode separation with rank $N, M\in[15,17]$, and 
345.5 seconds to decompose the operator matrixes for vector decomposition with rank $N, M\in[16,18]$.
For one time step, it took 4.3, 16.0 and 34.8 seconds to extrapolated, separate and decompose the elastic wavefields, repectively.
From the separated and decomposed wavefields (Figure 10), we can clearly observe the converted waves from the dipping salt flanks.
Due to the low velocities of qSV-wave in some directions at some locations, there are numerical dispersion in the qSV-wave fields. 
In spite of the dispersion, we obtain well separated qP- and qSV-wave fields, as well as their decomposed x- and z-components.

\new{Then we investigate the effect of the relative accuracy requirement on wave mode separation. 
Figure 11 demonstrates the separated P- and qSV-wave fields, and their variations when we relax the approximation level
from $10^{-6}$ to $10^{-3}$.
It took 174.0 seconds to decompose the operator matrixes for mode separation with rank $N, M\in[7,8]$, and 
345.5 seconds to decompose the operator matrixes for vector decomposition with rank $N, M\in[8,9]$.
And it took 8.0 and 14.9 seconds to separate and decompose the elastic wavefields, repectively.
The results are acceptable although more errors are introduced in the separated wavefields
when we turn down the relative accuracy requirement.
}

\new{
To further analysize the rough relationship of rank $(N,M)$ with the model complexity, we smooth the BP TTI model 
by applying a 2D triangle smoothing operator with the radius of 1875m on both x- and z-axes (Figure 12).
To maintain the range of the tilt angles, we first double the values of the original model and then apply the smoothing operation
for this parameter.
Figure 13 demonstrates the synthetic elastic wavefields and the mode separation and vector decomposition results.
In this case, it took 207.0 seconds to decompose the operator matrixes for mode separation with rank $N, M\in[13,14]$, and
310.2 seconds to decompose the operator matrixes for vector decomposition with rank $N, M\in[14,16]$.
It took 15.0 and 29.4 seconds to separate and decompose the elastic wavefields, repectively.
We observe that the ranks further decrease to about 12 if we double the smoothing radius to 3750m.
For homogeneous TI medium, the ranks automatically decrease to 1.
We obtain accurate mode separation and decomposition of the isotropic and elastic wavefields
at negligible computational cost with rank $N=M=1$,
if $\epsilon$, $\delta$ and $\theta$ are all set as $0.0$ in the models.
}

\multiplot{4}{vp0,epsi,del,the}{width=0.30\textwidth}
{
BP 2007 TTI model with parameters of (a) vertical P-wave velocity, Thomsen coefficients
(b) $\epsilon$ and (c) $\delta$, and (d) tilt angle $\theta$.
}

\multiplot{8}{Elasticx,Elasticz,ElasticSepP,ElasticSepSV,ElasticPx,ElasticPz,ElasticSVx,ElasticSVz}{width=0.28\textwidth}
{
Mode separation and vector decomposition using low-rank approximate algorithms in the BP 2007 TTI model:
(a) x- and (b) z-components of the synthetic elastic displacement wavefields at 1.4s;
(c) and (d) are the separated scalar qP- and qSV-wave fields;
(e) x- and (f) z-components of vector qP-wave fields;
(g) x- and (h) z-components of vector qSV-wave fields.
}

\inputdir{bptti2007.comparison}

\multiplot{8}{ElasticSepP3,ElasticSepSV3,ElasticSepP6vs3Dif,ElasticSepSV6vs3Dif}{width=0.3\textwidth}
{
Elastic wave mode separation using low-rank approximation with relaxed accuracy requirements:
Separated (a) qP- and (b) qSV-wave fields at the error level of $10^{-3}$ in low-rank decomposition;
Differences of (c) qP- and (d) qSV-wave fields to those separated with the error level of $10^{-6}$.
}

\inputdir{bptti2007.smth}

\multiplot{4}{vp0,epsi,del,the}{width=0.30\textwidth}
{
Smoothed BP 2007 TTI model with parameters of (a) vertical P-wave velocity, Thomsen coefficients
(b) $\epsilon$ and (c) $\delta$, and (d) tilt angle $\theta$. 2D triangle smoothing with the smoothing radius
of 1875m on both axis is applied to the paramaters shown in Figure 9.
}

\multiplot{8}{Elasticx,Elasticz,ElasticSepP,ElasticSepSV,ElasticPx,ElasticPz,ElasticSVx,ElasticSVz}{width=0.28\textwidth}
{
Mode separation and vector decomposition using low-rank approximate algorithms in the BP 2007 TTI model:
(a) x- and (b) z-components of the synthetic elastic displacement wavefields at 1.4s;
(c) and (d) are the separated scalar qP- and qSV-wave fields;
(e) x- and (f) z-components of vector qP-wave fields;
(g) x- and (h) z-components of vector qSV-wave fields.
}
