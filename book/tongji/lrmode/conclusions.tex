\section{Conclusions and discussion}
We have developed two fast algorithms for wave mode separation and vector decomposition for heterogeneous TI media.
They are based on \new{the} low-rank approximation of the space-wavenumber-domain operators, and reduces the 
cost to that of a small number of FFT operations per time step, \old{which} correspond\new{ing} to the approximation rank times
the number of components. Synthetic examples show that our approach have high accuracy and efficiency.
\new{In general TI media, the rank increases when the models become complex but is always far smallar than the model size.
For the 3D elastic wave mode separation and decomposition in heterogeneous TI media, however,
constructing the separated representations of the 
mixed-domain operator matrixes is still time consuming due to the substantial increase of the model size.
Parallelizing the algorithm for this procedure may provide a practical solution.
}

The key concepts of mode separation and vector decomposition are based on polarizations. Unlike the well-behaved
P-wave mode, the S-wave modes do not consistently polarize as a function of the propagation direction,
and thus can not be designated as SV and SH waves, except in isotropic and TI media. For a 3D TI medium,
the effects of kiss singularity are mitigated by using the mutual orthogonality among the qP, qSV and SH modes.
This procedure only ensures that the two S-modes are accurately separated in kinematics.
It is a challenge to find the right solution of the singularity problem and obtain completely separated 
two S-modes with correct amplitudes.
